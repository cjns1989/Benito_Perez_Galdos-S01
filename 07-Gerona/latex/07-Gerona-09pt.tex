\PassOptionsToPackage{unicode=true}{hyperref} % options for packages loaded elsewhere
\PassOptionsToPackage{hyphens}{url}
%
\documentclass[oneside,9pt,spanish,]{extbook} % cjns1989 - 27112019 - added the oneside option: so that the text jumps left & right when reading on a tablet/ereader
\usepackage{lmodern}
\usepackage{amssymb,amsmath}
\usepackage{ifxetex,ifluatex}
\usepackage{fixltx2e} % provides \textsubscript
\ifnum 0\ifxetex 1\fi\ifluatex 1\fi=0 % if pdftex
  \usepackage[T1]{fontenc}
  \usepackage[utf8]{inputenc}
  \usepackage{textcomp} % provides euro and other symbols
\else % if luatex or xelatex
  \usepackage{unicode-math}
  \defaultfontfeatures{Ligatures=TeX,Scale=MatchLowercase}
%   \setmainfont[]{EBGaramond-Regular}
    \setmainfont[Numbers={OldStyle,Proportional}]{EBGaramond-Regular}      % cjns1989 - 20191129 - old style numbers 
\fi
% use upquote if available, for straight quotes in verbatim environments
\IfFileExists{upquote.sty}{\usepackage{upquote}}{}
% use microtype if available
\IfFileExists{microtype.sty}{%
\usepackage[]{microtype}
\UseMicrotypeSet[protrusion]{basicmath} % disable protrusion for tt fonts
}{}
\usepackage{hyperref}
\hypersetup{
            pdftitle={GERONA},
            pdfauthor={Benito Pérez Galdós},
            pdfborder={0 0 0},
            breaklinks=true}
\urlstyle{same}  % don't use monospace font for urls
\usepackage[papersize={4.80 in, 6.40  in},left=.5 in,right=.5 in]{geometry}
\setlength{\emergencystretch}{3em}  % prevent overfull lines
\providecommand{\tightlist}{%
  \setlength{\itemsep}{0pt}\setlength{\parskip}{0pt}}
\setcounter{secnumdepth}{0}

% set default figure placement to htbp
\makeatletter
\def\fps@figure{htbp}
\makeatother

\usepackage{ragged2e}
\usepackage{epigraph}
\renewcommand{\textflush}{flushepinormal}

\usepackage{indentfirst}

\usepackage{fancyhdr}
\pagestyle{fancy}
\fancyhf{}
\fancyhead[R]{\thepage}
\renewcommand{\headrulewidth}{0pt}
\usepackage{quoting}
\usepackage{ragged2e}

\newlength\mylen
\settowidth\mylen{...................}

\usepackage{stackengine}
\usepackage{graphicx}
\def\asterism{\par\vspace{1em}{\centering\scalebox{.9}{%
  \stackon[-0.6pt]{\bfseries*~*}{\bfseries*}}\par}\vspace{.8em}\par}

 \usepackage{titlesec}
 \titleformat{\chapter}[display]
  {\normalfont\bfseries\filcenter}{}{0pt}{\Large}
 \titleformat{\section}[display]
  {\normalfont\bfseries\filcenter}{}{0pt}{\Large}
 \titleformat{\subsection}[display]
  {\normalfont\bfseries\filcenter}{}{0pt}{\Large}

\setcounter{secnumdepth}{1}
\ifnum 0\ifxetex 1\fi\ifluatex 1\fi=0 % if pdftex
  \usepackage[shorthands=off,main=spanish]{babel}
\else
  % load polyglossia as late as possible as it *could* call bidi if RTL lang (e.g. Hebrew or Arabic)
%   \usepackage{polyglossia}
%   \setmainlanguage[]{spanish}
%   \usepackage[french]{babel} % cjns1989 - 1.43 version of polyglossia on this system does not allow disabling the autospacing feature
\fi

\title{GERONA}
\author{Benito Pérez Galdós}
\date{}

\begin{document}
\maketitle

En el invierno de 1809 a 1810 las cosas de España no podían andar peor.
Lo de menos era que nos derrotaran en Ocaña a los cuatro meses de la
casi indecisa victoria de Talavera: aún había algo más desastroso y
lamentable, y era la tormenta de malas pasiones que bramaba en torno a
la Junta central. Sucedía en Sevilla una cosa que no sorprenderá a mis
lectores, si, como creo, son españoles, y es que allí todos querían
mandar. Esto es achaque antiguo, y no sé qué tiene para la gente de este
siglo el tal mando, que trastorna las cabezas más sólidas, da prestigio
a los tontos, arrogancia a los débiles, al modesto audacia y al honrado
desvergüenza. Pero sea lo que quiera, ello es que entonces andaban a la
greña, sin atender al formidable enemigo que por todas partes nos
cercaba.

Y aquel era enemigo, lo demás es flor de cantueso. Me río yo de
insurrecciones absolutistas y republicanas, en tiempos en que el poder
central cuenta con grandes elementos para sofocarlas. Aquello no se
parecía a ninguna de estas niñerías de ahora, pues con las tropas que
Napoleón envió a España a fines del año 9 constaba de trescientos mil
hombres el ejército invasor. Los nuestros, dispersos y desanimados, no
tenían un general experto que los mandase; faltaban recursos de todas
clases, especialmente de dinero, y en esta situación el poder central
era un hervidero de intriguillas. Las ambiciones injustificadas, las
miserias, la vanidad ridícula, la pequeñez inflándose para parecer
grande como la rana que quiso imitar al buey, la intolerancia, el
fanatismo, la doblez, el orgullo rodeaban a aquella pobre Junta, que ya
en sus postrimerías no sabía a qué santo encomendarse. Bullían en torno
a ella políticos de pacotilla de la primera hornada que en España
tuvimos, generales pigmeos que no supieron ganar batalla alguna; y
aunque había también varones de mérito así en la milicia como en lo
civil, estos o no tenían arrojo para sobreponerse a los tontos, o
carecían de aquellas prendas de carácter sin las cuales, en lo de
gobernar, de poco valen la virtud y el talento.

Tuvo la Junta allá por Marzo el malísimo acuerdo de establecer el
Consejo de Castilla, fundiendo en él todos los demás Consejos
suprimidos, y cuando esta antigualla se vio de nuevo con vida; cuando
esta máquina roñosa, inútil y gastada se encontró puesta otra vez en
movimiento, allí era de ver cómo pretendía gobernar el mundo. La
fatuidad de aquellos consejeros que tanto adularon a José no tenía
igual. Desde que se les puso en juego, empezaron a intrigar contra quien
les había sacado del olvido, y decían que la Junta era ilegítima.
Valiéndose de D. Francisco Palafox, hermano del defensor de Zaragoza; de
Montijo, a quien hemos visto en alguna parte, del marqués de la Romana y
de otros pájaros, llenaron de enredos a la Junta y a la comisión
ejecutiva. Por último, en la Regencia, última metamorfosis de aquel
poder tan nacional como desgraciado, también sembraron cizaña los del
Consejo. Esta pandilleja no era otra cosa que el partido absolutista,
que ya empezaba a sacar la oreja; y para que desde el principio se
tuviera completa noticia de su existencia, también repartió dinero entre
la tropa, fiando sus esperanzas a una sedición militar que por entonces
quedó frustrada. Nada de esto era ya nuevo en España, porque el motín
del 19 de Marzo en Aranjuez, de que, si mal no recuerdo, hice mención,
obra fue de la misma gente; mas no se valieron sólo de la tropa sino
también de varios cuerpos facultativos y distinguidos, como los lacayos,
pinches y mozos de cuadra de la regia casa. En Sevilla azuzaron a lo que
un gran historiador llama con enérgico estilo \emph{la bozal
muchedumbre}, y hubo frecuentes serenatas de berridos y patadas por las
calles; mas no pasó de aquí.

Un arma moral esgrimían entonces unos contra otros los políticos
menudos, y era el acusarse mutuamente de malversadores de los caudales
públicos, cuyo grosero recurso hacía el mejor efecto en el pueblo.
Cuando se disolvió la Junta en Cádiz, hubo un registro de equipajes, que
es de lo más vil y bochornoso que contiene nuestra moderna historia;
pero no se encontró nada en las maletas de los patriotas, porque estos,
malos o buenos, tontos o discretos, no tenían el alma en los bolsillos,
ni la tuvieron aun sus inmediatos sucesores, años adelante.

Perdonen ustedes, si me ocupo de estos sainetes de la epopeya. Lo
extraño es que las miserias de los partidos (pues también entonces había
partidos, aunque alguien lo dude) no impedían la continuación de la
guerra, ni debilitaban el formidable empuje de la nación, con
independencia de las victorias o derrotas del ejército. Verdad es que
las discordias de arriba no habían cundido a la masa común del país, que
conservaba cierta inocencia salvaje con grandes vicios y no pocas
prendas eminentes, por cuya razón la homogeneidad de sentimientos sobre
que se cimentara la nacionalidad, era aún poderosa, y España,
hambrienta, desnuda y comida de pulgas, podía continuar la lucha.

~

Cansaría a mis amados lectores si les contara detalladamente mi vida
durante aquel funesto año 9, que comenzado con las proezas de Zaragoza,
terminaba con el desastre de Ocaña y la dispersión del ejército español.
Por fortuna no me encontré en aquella jornada, pues incorporado al
principio del año al ejército del Centro, me destinaron en Agosto a la
división del duque del Parque, y asistí a la acción de Tamames. Poco
puedo decir de la de Talavera, que no sea por referencia, pues el 27 y
el 28 de Julio me encontraba en Puente del Arzobispo, y aunque algo
podría contar de la campaña del duque del Parque, lo omito por no cansar
a mis amigos.

A fin del año servía en la división de D. Francisco Copons, que con las
de D. Tomás Zeraín, de Lacy y Zayas guardaba el paso de Sierra-Morena,
porque ha de saberse que los franceses, envalentonados hasta lo sumo y
reforzados con nueva tropa, se disponían a invadir la Andalucía, a los
diez y ocho meses de la batalla de Bailén, ¡a los diez y ocho meses! Las
fuerzas de que disponíamos apenas merecían el nombre de ejército, y el
del duque de Alburquerque, único que aún se conservaba en buen estado,
no podía tampoco resistir el empuje de los franceses victoriosos, y se
retiraba hacia el Mediodía para proteger la resistencia del poder
central.

¡Qué situación, amigos míos! Esto pasaba, como he dicho, al poco tiempo
de aquella brillante y rápida campaña de Junio y Julio de 1808; y los
mismos lugares que antes nos vieron victoriosos y llenos de orgullo
presenciaban ahora el triste desfile de los dispersos de Ocaña, que a
cada instante volvían el rostro con inquietud, creyendo sentir las
pisadas de los caballos de Víctor, Sebastiani y Mortier.

---¡Quién hubiera creído---dije a Andresillo Marijuán, cuando
almorzábamos en una venta de Collado de los Jardines---que habíamos de
desandar tan pronto este camino! Ahora me parece que no paramos hasta
Cádiz.

---Con paciencia se gana el cielo---me contestó.---Yo tengo toda la que
pueden dar siete meses de bloqueo como el de Gerona. Todavía estoy
admirado de encontrarme vivo, Gabriel. Pero dime, ¿dónde has ganado esa
charretera? ¿Creerás que yo no soy nada? Digo mal porque dentro de la
plaza me hicieron al modo de sargento y a estas horas nadie me ha
reconocido mi grado. Haré una reclamación a la Junta.

---Yo gané mis grados en Zaragoza---respondí con orgullo---y también te
aseguro que al cabo de un año conservo cierta duda de si seré yo mismo
el que en aquellos fieros combates se halló, o si después de muerto me
habré trocado en otro sujeto.

---Bien dicen que en Zaragoza y en el ejército del Centro se dieron los
grados como quien echa almorzadas de trigo a las gallinas. Amigo
Gabriel, en España no se premia más que a los tontos y a los que meten
bulla sin hacer nada. Dime, teniente de almíbar, ¿en Zaragoza comistes
ratones flacos y pedazos de estera fritos con grasa de asno viejo?

Reíme de la pregunta, y los circunstantes dieron broma a Marijuán,
porque este desde que se nos unió cerca de Almadén del Azogue en los
últimos días del año, nos había venido aturdiendo con el perenne contar
de sus privaciones y hambres en Gerona.

---En mi mochila---continuó el aragonés---tengo un diario del sitio que
escribió en la plaza el Sr.~D. Pablo Nomdedeu, y os lo daré a leer, para
despertar el apetito cuando estéis desganados. Por ahora en marcha, que
me parece dan orden de tomar soleta hacia abajo.

En efecto, después de una hora de descanso emprendimos el camino hacia
el Mediodía, y Marijuán repetía la canción con que nos aporreaba los
oídos desde que le encontramos:

\small
\newlength\mlena
\settowidth\mlena{ \quad Com vols que m'rendesca}
\begin{center}
\parbox{\mlena}{Digasme tú, Girona                                \\
                Si te n'arrendirás...                             \\
                \quad Lirom lireta.                               \\
                Com vols que m'rendesca                           \\
                Si España non vol pas.                            \\
                \quad Lirom fa lá garideta,                       \\
                \quad Lirom fa lireta lá.}                        \\
\end{center}
\normalsize

En Bailén hicimos noche. ¡Qué triste impresión produjo en mí la vista de
aquellos campos, al considerar que los atravesábamos después de dejar
casi toda Castilla en poder de los franceses, a quienes poco antes
habíamos sojuzgado con tanta fortuna en el mismo sitio! ¡Cómo se
representó en mi imaginación lo que allí había visto y oído, la
perspectiva y el estruendo glorioso de la acción, iluminada por el
ardoroso sol de Julio! Todo estaba frío, helado, quieto, triste,
silencioso, oscuro, y parecía que sobre los llanos y las mansas colinas
de Bailén, una pesada e informe sombra se paseaba a flor del suelo.
Visitamos luego Marijuán y yo el palacio de Rumblar, creyendo encontrar
allí todavía a la condesa y a su familia, y aunque era ya de noche, nos
propusimos penetrar seguros de ser bien recibidos. Cuando dimos los
primeros aldabazos en la puerta, contestonos el lejano ladrido de un
perro, sin que rumor alguno indicase la presencia de criatura humana en
el palacio, lo cual nos hizo comprender que estaba abandonado.
Insistimos, sin embargo, en dar golpes, y al cabo oímos una voz que
desde el patio con enojado tono nos respondía, mejor dicho, nos
increpaba exclamando:

---Allá voy. ¡Condenados muchachos, qué querrán a estas horas!

Abrionos echando sapos y culebras por su fea boca el tío Tinaja, antiguo
servidor de la casa (pues no era otro el que a la sazón la guardaba), y
luego que nos hubo reconocido, desarrugó el ceño, hízonos entrar
ofreciéndonos un asiento junto a la lumbre, y allí nos contó cómo toda
la familia con buena parte de la servidumbre había marchado a Cádiz
huyendo de la invasión francesa.

---Mi señora la condesa doña María estaba en que se había de
quedar---nos dijo;---pero sus primas de Madrid, que llegaron por Todos
los Santos, le volvieron la cabeza del revés. D. Paco también tenía
mucho miedo, y entre él, las primas y las tres señoritas, todos llorando
y moqueando en ruedo, ablandaron el alma de bronce de la condesa,
obligándola a marchar.

---¿No ha venido también el Sr.~D. Felipe?---pregunté comprendiendo a
qué personas se refería el tío Tinaja.

---El Sr.~D. Felipe no ha venido, porque según dijeron, está con el
francés. Su hermana, la señora marquesa, es muy española, y habían de
ver ustedes cómo disputa con su sobrina, que se ríe del Lord y dice que
ningún general español vale dos cuartos.

---¿Ha venido también D. Diego?

---No señor. Pues pocas lágrimas han derramado las niñas, y pocos mares
han corrido de los ojos de la señora por las calaveradas de don Diego.
No hay quien le saque de Madrid, donde se junta con flamasones, anteos,
perdularios, gabachos y gente mala que le trae al retortero. Parece que
ya no se casa con la señorita Inés, por cuya razón mi ama está que
trina, y el otro día ella y sus primas hablaron más de lo regular. D.
Paco se puso por medio y echó una arenga en latín. Las señoritas
empezaron a llorar, y aquel día en la mesa nadie habló una palabra. No
se oía más ruido que el de los dientes mascando, el de los tenedores
picando en los platos y el de las moscas que iban a golosinear.

---¿Y cuándo salieron para Cádiz?

---Hace cuatro días. Las tres señoritas iban muy contentas, y doña María
muy triste y ensimismada. La mala conducta del señor don Diego la tiene
en ascuas y la buena señora se va acabando.

Nada más me dijo aquel hombre que merezca mención, y a varias preguntas
mías harto prolijas e impertinentes, no contestó cosa alguna de
provecho. Después que nos ofreció parte de su cena, díjonos que podíamos
albergarnos en la casa por aquella noche, y como la tropa se alojaba en
el pueblo, nos quedamos allí. Solo, y mientras Marijuán dormía, recorrí
varias habitaciones altas de la casa, iluminadas no más que por la luna,
y una dulce e inexplicable claridad llenaba mi alma durante aquella muda
y solitaria exploración. No hubo mueble que no me dijese alguna cosa, y
mi imaginación iba poblando de seres conocidos las desiertas salas. La
alfombra conservaba a mis ojos una huella indefinible, más bien pensada
que vista; vi un cojín que aún no había perdido el hundimiento producido
por el brazo que acababa de oprimirlo, y en los espejos creí ver no la
huella, ni la sombra, porque estas voces no son propias, sino una nada,
mejor dicho un vacío, dejado allí por la imagen que había desaparecido.

En una habitación que daba a la huerta vi tres camas pequeñas. Dos de
ellas parecía como que tenían un lugar fijo en los dos testeros de
derecha e izquierda. La tercera que estorbaba el paso, revelaba haber
sido puesta para un huésped de pocos días. Las tres estaban cubiertas de
blanquísimas colchas, bajo las cuales los fríos colchones se inflamaban
sin peso alguno. La pila de agua bendita estaba llena aún y mojé las
puntas de los dedos, haciéndome en la frente la señal de la cruz. Un
fuerte escalofrío corrió por mi cuerpo al contacto helado, como si los
dedos que habían tomado las últimas gotas se rozaran con los míos en la
superficie del agua. Recogí del suelo una pequeña cinta y unos pedacitos
de papel retorcidos, engrasados y perfumados, que indicaban haber
servido para moldear los rizos de una cabellera. El silencio de aquel
lugar no me parecía el silencio propio de los lugares donde no hay
nadie, sino aquel que se produce en los intervalos elocuentes de un
diálogo, cuando hecha la pregunta el interlocutor medita para responder.

Salí de aquella estancia, y después de recorrer otras con igual interés,
sintiéndome al fin cansado, me recosté en un sofá, donde cerca ya del
alba me dormí profundamente. La luz del día entraba a torrentes por las
ventanas y balcones cuando me despertó Andrés cantando su estribillo
catalán:

\small
\newlength\mlenb
\settowidth\mlenb{Dígasme tú Girona si te n'arrendiràs.}
\begin{center}
\parbox{\mlenb}{Dígasme tú Girona si te n'arrendiràs.}              \\
\end{center}
\normalsize

En aquellos días, los últimos del mes de Enero de 1810, ocurrieron las
más lamentables desgracias del ejército español. Creeríase que el genio
de la guerra, fundamental en nosotros como el eje del alma, nos había
faltado, y la lucha fue desordenada y a la aventura. El general Desolles
atacó en Puerto del Rey a la división Girón que se desbandó junto a las
Navas de Tolosa, y al mismo tiempo Gazán acometía el paso de Nuradal,
mientras Mortier forzaba el de Despeñaperros. El mariscal Víctor penetró
por Torrecampo para caer sobre Montoro, y Sebastiani por Montizón, de
modo que la invasión de Andalucía se verificó por cuatro puntos
distintos con estrategia admirable que acabó de desconcertarnos. Verdad
es, y sírvanos esto de disculpa, que teníamos por general en jefe a D.
Juan Carlos de Areizaga, hombre nulo en el arte de la guerra, y en cuya
cabeza no cabían tres docenas de hombres. La pericia de algunos jefes
subalternos servía de muy poco, y desmoralizada la tropa, convencida de
su incapacidad para la resistencia, no veía delante de sí ni gloria, ni
honor, sino el cómodo refugio de Córdoba, Sevilla o la isla gaditana.
Resistencia formal sólo la hallaron los franceses por Montizón entre
Venta Nueva y Venta Quemada, donde mandaba D. Gaspar Vigodet, el cual
después de batirse con mucho arrojo ordenó la retirada en regla. En
suma, señores míos, doloroso es decirlo y doloroso es recordarlo; pero
es lo cierto que los franceses avanzaron hacia Córdoba cuando nosotros
llorábamos nuestra impotencia camino de Sevilla.

¿Y qué podré deciros del espectáculo que nos ofreció esta ciudad
amotinada, sometida a las intrigas de una facción tan pequeña como
audaz? De buena gana no diría nada, tragándome todo lo que sé y
ocultando todo lo que vi, para que semejantes fealdades no
entristecieran estos cuadros; pero ya la fama ha dicho cuanto había que
decir, y no porque yo lo calle dejará de saberse, que si en mí
consistiera, a este y a otros hoyos de nuestra historia les echaría
tierra, mucha tierra.

Es el caso que fugitiva la Central, los conspiradores erigieron allí una
juntilla suprema, y azuzado el populacho, no se oían más que vivas y
mueras, olvidándose del francés que tocaba a las puertas, cual si en el
suelo patrio no hubiese más enemigos que aquellos desgraciados
centrales. ¡Lo que es la pasión política, señores! No conozco peor ni
más vil sentimiento que este, que impulsa a odiar al compatricio con
mayor vehemencia que al extranjero invasor. Yo me espantaba presenciando
los atropellos verificados contra algunos y la salvaje invasión de las
casas de otros. ¡Y gracias que escaparon con vida de manos de aquella
plebe holgazana y chillona! En una palabra, aquello era de lo más
denigrante que he visto en mi vida, y si la Junta central valía poco,
los individuos que en Sevilla y después en Cádiz agujerearon, como
inquietos y vividores reptiles, sus fundamentos, no ocupan, a pesar de
su mucho bullir y de las distintas posturas que tomaron, un lugar
visible en la historia. Su pequeñez los hace desaparecer en las
perspectivas de lo pasado, y sus nombres sin eco no despiertan
admiración ni encono. Pertenecen a ese vulgo que, con ser tan vulgo, ha
influido en los destinos del país desde la primera revolución acá;
gentezuela sin ideal, que se perdería en las muchedumbres como las gotas
de lluvia en el Océano, si la vituperable neutralidad política de los
españoles honrados, decentes, entendidos y patriotas, que son los más,
no les permitiera actuar en la vida pública, tratando al país como un
objeto de exclusiva pertenencia que se les ha dado para divertirse.

Pero quiero poner punto en esta materia, que seduce poco mi
entendimiento. Continuando nuestra retirada llegamos al Puerto de Santa
María, donde estuvimos dos días con sus noches, y allí fue donde adquirí
sobre el formidable cerco de Gerona estupendas noticias. Debo una
explicación a mis lectores, y voy a darla.

Mi objeto al comenzar esta última sesión, en que apaciblemente nos
encontramos, amados señores míos, fue referir lo mucho y bueno que vi en
Cádiz cuando nos refugiamos allí, después que los franceses penetraron
en Andalucía; pero un deber patriótico me obliga a aplazar por breve
tiempo este mi natural deseo, haciendo lugar a algunos hechos del sitio
de Gerona, que contaré también, si bien los contaré de oídas. Un amigo
de aquellos tiempos, y que después lo fue también mío en épocas más
bonancibles, me entretuvo durante dos largas noches con la descripción
de maravillosas hazañas que no debo ni puedo pasar en silencio. Aquí las
pongo, pues, suspendiendo el curso de mi historia, que reanudaré en
breve, si Dios me da vida a mí y a ustedes paciencia. Sólo me permito
advertir, que he modificado un tanto la relación de Andresillo Marijuán,
respetando por supuesto todo lo esencial, pues su rudo lenguaje me
causaba cierto estorbo al tratar de asociar su historia a las mías. Hago
esta advertencia para que no se maravillen algunos de encontrar en las
páginas que siguen observaciones y frases y palabras impropias de un
muchacho sencillo y rústico. Tampoco yo me hubiera expresado así en
aquellos tiempos; pero téngase presente que en la época en que hablo,
cuento algo más de ochenta años, vida suficiente a mi juicio para
aprender alguna cosa, adquiriendo asimismo un poco de lustre en el modo
de decir.

\clearpage
\Large
\begin{center}
RELACIÓN DE ANDRESILLO MARIJUÁN 
\end{center}
\normalsize

\hypertarget{i}{%
\chapter{I}\label{i}}

Entré en Gerona a principios de Febrero, y me alojé en casa de un
cerrajero de la calle de Cort-Real. A fines de Abril salí con la
expedición que fue en busca de víveres a Santa Coloma de Farnés, y a los
pocos días de mi regreso, murió a consecuencia de las heridas recibidas
en el segundo sitio aquel buen hombre que me había dado asilo. Creo que
fue el 6 de Mayo, es decir, el mismo día en que aparecieron los
franceses, cuando al volver de la guardia en el fuerte de la Reina Ana,
encontré muerto al Sr.~Mongat, rodeado de sus cuatro hijos que lloraban
amargamente.

Hablaré de los cuatro huérfanos, que ya lo eran completamente por haber
perdido a su madre algunos meses antes. Siseta, o como si dijéramos,
Narcisita, la mayor en edad, tenía poco más de los veinte, y los tres
varoncillos no sumaban entre todos igual número de años, pues
Badoret\footnote{Diminutivo de Salvador.} apenas llegaba a los diez,
Manalet\footnote{Id. de Manuel.} no tenía más de seis, y Gasparó
empezaba a vivir, hallándose en el crepúsculo del discernimiento y de la
palabra.

Cuando penetré en la casa y vi cuadro tan lastimoso, no pude contener
las lágrimas y me puse a llorar con ellos. El Sr.~Cristòful Mongat era
una excelente persona, buen padre y patriota ardiente; pero aun más que
el recuerdo de las buenas prendas del difunto me contristaba la soledad
de las cuatro criaturas. Yo les amaba mucho, y como mi buen humor y
franca condición propendían a enlazar el alma de aquellos inocentes con
la mía, en algunos meses de trato, Badoret, Manalet y Gasparó, se
desvivían por mí. No hablo aquí de Siseta, porque para esta tenía yo un
sentimiento extraño, de piedad y admiración compuesto, como se verá más
adelante. Mi ocupación en la casa mientras vivió el Sr.~Mongat era en
primer término hablar con este de las cosas de la guerra, y en segundo
término divertir a los chicos con toda clase de juegos, enseñándoles el
ejercicio y representando con ellos detrás de un cofre las escenas del
ataque, defensa y conquista de una trinchera. Cuando yo iba de guardia,
bien a Monjuich, bien a los reductos del Condestable o del Cabildo, los
tres, incluso Gasparó, me seguían con sendas cañas al hombro remedando
con la boca el son de cajas y trompetas o relinchando al modo de
caballos.

Asociado cordialmente a su desgracia, les consolé como pude, y al día
siguiente, después que echamos tierra al buen cerrajero, y luego que se
retiraron los vecinos fastidiosos que habían ido a hacer pucheros
condoliéndose ruidosamente de los huérfanos, pero sin darles auxilio
alguno, tomé por la mano a Siseta, y llevándola a la cocina, le dije:

---Siseta, ya tú sabes\ldots{}

Pero antes quiero decir que Siseta era una muchacha gordita y fresca,
que sin tener una hermosura deslumbradora, cautivaba mi alma de un modo
extraño, haciéndome olvidar a todas las demás mujeres y principalmente a
la que había sido mi novia en la Almunia de Doña Godina. Rosada y
redondita, Siseta parecía una manzana. No era esbelta, pero tampoco
rechoncha. Tenía mucha gracia en su andar, y poseyendo bastante soltura
e ingenio en la conversación, sabía sin embargo acomodarse a las
situaciones, distinguiéndose por una gran disposición para no estar
nunca fuera de su lugar, de cuyas prendas puede colegirse que Siseta
tenía talento.

Pues bien, como antes indiqué, tomándole una mano, le dije:

---Siseta\ldots{}

No sé qué me pasó en la lengua, pues callé un buen rato, hasta que al
fin pude continuar así:

---Siseta, ya tú sabes que va para cuatro meses que estoy alojado en tu
casa\ldots{}

La muchacha hizo un signo afirmativo, demostrando estar convencida de mi
permanencia en la casa durante cuatro meses.

---Quiero decir---proseguí---que durante tanto tiempo he estado comiendo
de tu pan, aunque también os he dado el mío. Ahora con la muerte del
Sr.~Cristòful, os habéis quedado huérfanos. ¿Tienen ustedes tierras,
alguna casa, alguna renta?\ldots{}

---No tenemos nada---me contestó Siseta dirigiendo tristes miradas a los
cacharros de la cocina.---No tenemos nada más que lo que hay en casa.

---Las herramientas valen alguna cosa---dije---mas en fin no hay que
apurarse, que Dios aprieta, pero no ahoga. Aquí está el brazo de Andrés
Marijuán. ¿Dejó tu padre algún dinero?

---Nada---respondió---no ha dejado nada. Durante su enfermedad trabajaba
muy poco.

---Bien, muy bien---dije yo.---Con eso podéis recibir el plus que nos
dan ahora y la ración que me toca todos los días. No hay que apurarse.
Tú serás madre de tus hermanos, y yo seré su padre, porque estoy
decidido a ahorcarme contigo. Ea, dejarse de lloriqueos; Siseta, yo te
quiero. Tal vez creerás tú que yo no tengo tierras. ¡Qué tonta! Si
vieras qué dos docenas de cepas tengo en la Almunia; si vieras qué
casa\ldots{} sólo le falta el techo; pero es fácil componerla, sin
fabricarla toda de nuevo. Con que lo dicho, dicho. En cuanto se acabe
este sitio, que será cosa de días a lo que pienso, venderás los
cachivaches de la herrería, me darán mi licencia, pues también se
concluirá la guerra; pondremos sobre un asno a la señora Siseta con
Gasparó y Manalet, y tomando yo de la mano a Badoret, camina que
caminarás, nos iremos a ese bajo Aragón, que es la mejor tierra del
mundo, donde nos estableceremos.

Una vez que desembuché este discurso, volví al taller, con objeto de
examinar las herramientas, y todo aquel mueblaje me pareció de poquísimo
valor. La huérfana después que me oyera, sin decir cosa alguna, púsose a
arreglar los trastos ordenando todo con hábil mano y a limpiar el polvo.
Los chicos me rodearon al punto, corriendo precipitadamente a traer sus
cañas, palos y demás aparatos de guerra, viéndome yo obligado en razón
de esta diligencia a recomendarles gran celo en el servicio de la patria
y del rey, pues bien pronto, si los franceses apretaban el cerco, Gerona
necesitaría de todos sus hijos, aun de los más pequeñitos. Por último,
después que durante media hora pusieron armas al hombro y en su lugar,
cebaron, cargaron, atacaron e hicieron varias descargas imaginarias,
pero que retumbaban en el angosto taller, les vi soltar las armas
decaído el marcial ardor, y volver a su hermana con elocuente expresión
los ojos.

---¿Qué?---pregunté yo, comprendiendo lo que significaba aquel mudo
interrogatorio.---Siseta, ¿no hay qué comer?

Siseta disimulando sus lágrimas, registraba los negros andamios de una
alacena, en cuyas cavernosas profundidades la infeliz se empeñaba en ver
alguna cosa.

---¿Cómo es eso?---dije.---Siseta, no me habías dicho nada. ¿Qué me
costaría ir al cuartel y pedir que me adelanten la ración de
mañana?\ldots{} ¿Y para qué quiero yo los siete cuartos que tengo
ahorrados? Nada, hija; es preciso no sólo traer lo necesario para hoy,
sino también provisiones abundantes, por si escasean los víveres dentro
de la plaza. Dicen que ahora nos van a dar dos reales diarios. Ya me
figuro lo que harás tú con esta riqueza. Pero no es ocasión de detenerme
en habladurías, que estos valientes soldados se mueren de hambre. Toma
los siete cuartos: voy al punto por la libreta.

No tardé en volver con el pan, y tuve el gusto de ver comer a mis hijos
(desde entonces empecé a darles este nombre). Siseta se mantuvo en los
límites de una sobriedad excesiva, y mientras duró el festín les hablé
de los grandes acopios de víveres que se estaban haciendo en Gerona,
conversación que parecía muy del agrado de los pequeñuelos. En esto el
Sr.~Nomdedeu, habitante del piso superior de la casa, pasó por delante
de la tienda en dirección al portal contiguo. Saludonos afablemente a
todos, y después de decir algunas palabras de desconsuelo con motivo de
la pérdida del excelente señor Mongat, subió a su casa, rogándome que le
acompañara. Yo tenía costumbre de ir todas las mañanas a referirle lo
que se decía en los cuerpos de guardia, y estas visitas tenían para mí
el doble atractivo de contar lo que sabía y de oír las agradables
pláticas del Sr.~Nomdedeu, hombre con quien no se hablaba una sola vez
sin sacar alguna enseñanza provechosa.

\hypertarget{ii}{%
\chapter{II}\label{ii}}

El Sr.~D. Pablo Nomdedeu era médico. No pasaba de los cuarenta y cinco
años; pero los estudios o penas domésticas, para mí desconocidas, habían
trabajado en tales términos su naturaleza que aparentaba mucho más del
medio siglo. Era acartonado, enjuto, amarillo, con gran corva en la
espina dorsal, y la cabeza salpicada de escasos pelos rubios y blancos,
como yerba que nace al azar en ingrata tierra. Todo anunciaba en él
debilidad y prematura vejez, excepto su mirar penetrante, imagen del
alma enérgica y del entendimiento activo. Vivía en apacible medianía,
sin lujo, pero también sin pobreza, muy querido de sus paisanos,
consagrado fuera de casa a los enfermos del hospital, y dentro de ella
al cuidado de su hija única, enferma también de doloroso e incurable
mal. Para que ustedes acaben de conocer a aquel apacible sujeto, me
falta decirles que Nomdedeu era un hombre de gran saber y de mucha
amenidad en su sabiduría. Todo lo observaba, y no se permitía ignorar
nada, de modo que jamás ha existido hombre que más preguntase. Yo no
creí que los sabios preguntasen tonterías de las que no ignora un
rústico; pero él me dijo varias veces que la ciencia de los libros no
valdría nada, si no se cursase el doctorado de la conversación con toda
clase de personas.

De su casa poco diré. Era tan humilde como decente. Muchos libros,
algunas estampas francesas de anatomía, emparejadas con otras de santos,
y bastantes cuadros que ostentaban detrás del vidrio innumerables yerbas
secas con sendos letreros manuscritos al pie. Pero lo que principalmente
impresionaba mi ánimo al subir a casa del Sr.~Nomdedeu era una criatura
tierna y sensible, una belleza consumida y marchita, una triste vida que
junto a la pequeña ventana abierta al Mediodía quería prolongarse
absorbiendo los rayos del sol. Me refiero a la desgraciada Josefina,
hija del insigne hombre que he mencionado, la cual, enferma y postrada,
se me representaba como las flores secas guardadas por el doctor detrás
de un vidrio. Josefina había sido hermosa; pero perdidos algunos de sus
encantos, otros se habían sublimado en aquel descendente crepúsculo que
iba difundiendo sobre ella las sombras de la muerte. Inmóvil en un
sillón, su aspecto era por lo común el de una absoluta indiferencia.
Cuando su padre entró conmigo el día a que me refiero, Josefina no
respondió a sus caricias con una sola palabra. Nomdedeu me dijo:

---Su existencia de plomo está pendiente de una hebra de seda.

Pronunció estas palabras en voz alta y delante de ella, porque Josefina
estaba completamente sorda.

---El profundo silencio que la rodea---continuó el padre,---es favorable
a su salud, porque siendo su mal un desarrollo excesivo de la
sensibilidad, todo lo que disminuya las impresiones exteriores,
aumentará el reposo, a que debe esa lánguida y decadente vida. No espero
salvarla; y todo mi afán consiste hoy en embellecer sus días, fingiendo
que nos hallamos rodeados de felicidades y no de peligros. Desearía
llevarla al campo, pero el deber y el patriotismo me obligan a no
abandonar el cuidado del hospital, cuando nos amenaza un tercer cerco,
que parece va a ser más riguroso que los dos primeros. Dios nos saque en
bien. ¿Con que se murió ese pobre Sr.~Mongat?

---Sí, señor---respondí,---y ahí tiene usted cuatro huérfanos desvalidos
que pedirían limosna por las calles de Gerona, si yo no estuviera
decidido a quitarme el pan de la boca para dárselo.

---Dios te premiará tu generosidad. Yo también haré lo que pueda por
esos infelices. Siseta parece una buena muchacha, y sube algunas veces a
acompañar a mi hija. Dile que venga más a menudo, y hoy mismo encargaré
a la señora Sumta\footnote{Lo mismo que Asunción.} que les dé a los
hijos de Cristòful Mongat todo lo que sobre en la casa. Pero cuéntame,
¿qué has oído en el cuerpo de guardia? Antes dime lo que ha ocurrido en
esa expedición a Santa Coloma de Farnés. ¿Fuiste allá?

---Sí, señor; mas no nos ocurrió nada de particular. Los franceses se
nos presentaron en la tarde del 24 de Abril; pero como éramos pocos, y
no llevábamos por objeto el batirnos con ellos, sino traer provisiones a
Gerona, luego que cargamos los carros y las mulas, nos vinimos para acá
con D. Enrique O'Donnell. Los cerdos dominan toda la Sagarra; pero los
somatenes les hacen perder mucha gente, y para abastecerse pasan la pena
negra. El general francés Pino mandó hace poco un batallón a San Martín
en busca de víveres. Al llegar, el coronel pidió al alcalde para el día
siguiente de madrugada cierto número de raciones de tocino (porque
abundan en aquel pueblo los animalitos de la vista baja); y como el
batallón estaba cansado, dioles boletas de alojamiento, distribuyendo a
los soldados en las casas de los vecinos. El alcalde aparentó deseo de
servir al señor coronel, y al anochecer el pregonero salió por las
calles gritando: \emph{«Eixa nit a las dotse, cada vehí matarà son
porch»}.

---Y cada vecino mató su francés.

---Así parece, señor, y así me lo contaron en el camino; pero no
respondo de que sea verdad, aunque la gente de San Martín es capaz de
eso. Luego que hicieron su matanza, escondieron armas, morriones y
cuanto pudiera descubrirlos; y cuando se presentó el general Pino
trataron de probarle que \emph{allí no había estado nadie.}

---Sabes, Andrés---me dijo Nomdedeu,---que eso parece cosa de cuento.

---Séalo o no---repuse,---con estos y otros cuentos se anima la gente.
Los \emph{cerdos} están ya sobre Gerona, y esta mañana les hemos visto
en los altos de Costa-Roja. Aquí dentro no somos más que cinco mil
seiscientos hombres, que no son bastantes para defender la mitad de los
fuertes. De estos el que no se ha caído ya, es porque no se le ha dado
licencia. Si Zaragoza, que tenía dentro de murallas cincuenta mil
hombres, ha caído al fin en poder del francés, ¿qué va a hacer Gerona
con cinco mil seiscientos?

---Ya serán algunos más---dijo Nomdedeu paseándose por la habitación con
la inquietud nerviosa y retozona que se apoderaba de él hablando de las
cosas de la guerra.---Todos los vecinos de Gerona toman las armas, y hoy
mismo se están formando en el claustro de San Félix las listas de las
ocho compañías que componen la \emph{Cruzada gerundense}. Yo he querido
afiliarme; pero como médico, cuyos servicios no pueden reemplazarse, me
han dejado fuera con sentimiento mío. También se está formando hoy el
batallón de señoras, de que es coronela doña Lucía Fitz-Gerard, ¿la
conoces? En verdad te digo, amigo Andrés, que en medio de la pena que
causa el considerar los desastres que nos amenazan, se alegra uno al ver
los belicosos preparativos que tanto enaltecen al vecindario de esta
ciudad.

Mientras esto decíamos, expresándonos uno y otro con bastante
exaltación, Josefina fijaba en nosotros sus ojos sorprendida y aterrada,
y atendía a nuestros gestos, dando a conocer que los comprendía tan bien
como la misma palabra. Advirtiolo su padre y volviéndose a ella, la
tranquilizó con ademanes y sonrisas cariñosas, diciéndome:

---La pobrecita ha comprendido al instante que estamos hablando de la
guerra. Esto le causa un terror extraordinario.

La enferma tenía delante de sí en una mesilla de pino un gran pliego de
papel con pluma y tintero. La escritura servía a padre e hija de medio
de comunicación.

Nomdedeu, tomando la pluma escribió:

---Hija mía, no tengas miedo. Hablábamos de las bandadas de palomas que
vio ayer Andresillo en Pedret. Dice que mató todas las que quiso y que
te traerá un par esta tarde. No, no temas, hija mía, no volverá a haber
más sitios en Gerona. Si se ha concluido la guerra. Pues qué, ¿no lo
sabías? Esas noticias ha traído el Sr.~Andresillo. Verdad que se me
había olvidado decírtelo. Estamos en paz. Veremos si mañana puedes salir
a dar un paseo por Mercadal. La semana que entra iremos a Castellà. Dice
nostramo Mansió que están los rosales tan cargados de rosas\ldots{}
¿Pues y los cerezos? Este año habrá tanta cereza, que no sabremos qué
hacer de ella. He mandado que pongan dos colmenas más, y parece que
dentro de un mes la vaca tendrá su cría. A la gallina pintada se le ha
puesto una buena echadura con seis o siete huevos de pata. Dentro de
diez días los sacará a todos, y dará gusto ver a esa familia.

Luego que esto escribió, volviose a mí el Sr.~D. Pablo, y procurando
disimular su aflicción, me dijo:

---De este modo la voy engañando, para arrancar su ánimo a la tristeza.
Si ella supiera que mi casa de campo con todas las plantas y los
animalitos que allí tenía no existe ya\ldots{} Los franceses no han
dejado piedra sobre piedra. ¡Pobre de mí! Rodeado de desastres,
amenazado como todos los gerundenses de los horrores de la guerra, del
hambre y de la miseria, tengo que fingir junto a esta niña infeliz un
bienestar y una paz que está muy lejos de nosotros, y he de ocultar la
amargura de mi corazón destrozado, mintiendo como un histrión. Pero así
ha de ser. Tengo la convicción de que si mi hija llegase a conocer la
situación en que nos encontramos y tuviese conocimiento del bombardeo y
de las escaseces que nos amagan, su muerte sería inmediata; y quiero
prolongarle la vida todo el tiempo que me sea posible, porque confío en
que si algún día Dios y San Narciso resuelven poner fin a las desgracias
de esta ciudad, podré salir de Gerona y llevarla a disfrutar la vida del
campo, única medicina que la aliviará.

Josefina al concluir de leer el papel, movió tristemente la cabeza en
señal de incredulidad, y luego dijo:

---Pues marchémonos mañana a Castellà.

---Este sí que es apuro---me dijo Nomdedeu, tomando la pluma para
contestar a su hija.---¿Qué le voy a decir?

Pero sin detenerse escribió:

---Hija mía, ten un poco de paciencia. El tiempo que parece bueno, está
muy malo, y mañana ha de llover. Yo lo conozco por lo que dicen mis
libros. Además tengo que hacer en el hospital durante algunos días.

Entonces la enferma, que sin duda se fatigaba hablando o no tenía gusto
en pronunciar palabras que no oía, tomó también la pluma, y con rapidez
nerviosa trazó lo siguiente:

---Andrés estaba hablando de batallas.

---No, no, corazón mío---repuso el padre, acentuando su negativa con
risas y ademanes festivos.

---¡No, no, señorita Josefina!---exclamé yo a gritos, pues es costumbre
instintiva alzar la voz delante de los sordos, aun sabiendo que estos no
pueden oír.

---Precisamente---escribió D. Pablo---ahora me estaba diciendo que le
van a dar licencia, porque ya no se necesitan soldados. Hija mía, esta
tarde vendrán aquí algunos amigos para que bailen la sardana y te
distraigan un rato. ¿Por qué no sigues tu lectura?

Y luego puso en manos de su hija un tomo, que era la primera parte del
\emph{Quijote}, el cual abrió ella por donde lo tenía marcado,
comenzando a leer tranquilamente.

\hypertarget{iii}{%
\chapter{III}\label{iii}}

Nomdedeu llevándome junto a la ventana, me dijo:

---La idea de la guerra y del bombardeo le causa mucho horror. Es
natural que así sea, puesto que de una fuerte y dolorosa impresión de
miedo proviene su desorden nervioso y la pasión de ánimo que la tiene en
tan lamentable estado. En el segundo sitio, amigo Andrés, puedo decir
que perdí a mi querida niña, único consuelo mío en la tierra. Ya sabes
que llego aquí el bárbaro Duhesme a mediados de Julio del año pasado,
cuando dijo aquellas arrogantes palabras: \emph{El 24 llego, el 25 la
ataco, la tomo el 26 y el 27 la arraso}. Hombre que tales bravatas
decía, igualándose a César, era forzosamente un necio. Llegó en efecto,
y atacó, pero no pudo tomar ni arrasar cosa alguna, como no fuese su
propia soberbia, que quedó por tierra ante esos muros. Tenía 9.000
hombres, y aquí dentro apenas pasaban de 2.000, con los paisanos que se
habían armado a toda prisa. Duhesme puso cerco a la plaza, y abiertas
trincheras entre Monjuich y los fuertes del Este y Mercadal, el 13
empezó a bombardearnos sin piedad. El 16 intentaron asaltar el Monjuich,
pero sí\ldots{} para ellos estaba. El regimiento de Ultonia lo
defendía\ldots{} Pero voy a mi objeto. Como te iba diciendo, mi pobre
niña perdió el sosiego, y su espanto la tenía en vela de día y de noche,
cuyo estado de excitación, junto con la resistencia a tomar alimento, la
puso a punto de morir. Figúrate mi pena y la de mi sobrino. Porque he de
advertirte que yo tenía un sobrino llamado Anselmo Quixols, hijo de mi
hermana doña Mercedes, residente en La-Bisbal. No sé si sabrás que mi
hermana y yo teníamos concertado casar a Anselmo con Josefina, enlace
que era muy agradable a entrambos muchachos, porque desde algunos meses
antes habían gastado algunas manos de papel en escribirse cartas, y
díchose mil amorosas palabras en honesto lenguaje. Entonces vivíamos en
la calle de la Neu, muy cerca de la plaza. El día 15 habíamos bajado al
portal, donde nos creíamos más seguros del bombardeo, y estábamos
comiendo en compañía de Anselmo, que por breve rato dejó el servicio
para venir a informarse de nuestra situación. ¡Ay, amigo Andrés! ¡Qué
día, qué momento! Una bomba penetró por el techo, atravesó el piso alto,
y horadando las tablas cayó en el bajo, donde al estallar con horrible
estruendo causó espantosos estragos. Anselmo quedó muerto en el acto
atravesado el pecho por un casco, mi fámulo fue mortalmente herido, y la
señora Sumta también aunque sin gravedad. Yo recibí un golpe, y sólo mi
hija quedó aparentemente ilesa; pero ¡qué trastorno en su organismo!,
¡qué desquiciamiento, qué horrible perturbación en su pobre alma! La
horrenda explosión, el súbito peligro, la muerte de su primo y futuro
esposo, a quien recogimos del suelo en el momento de expirar, el riesgo
que corríamos con el incendio de la casa hirieron con golpe tan rudo su
naturaleza endeble y resentida, que desde entonces mi hija, aquella
muchacha amable, graciosa y discreta dejó de existir, y en su lugar
dejome el cielo esta desvalida y lastimosa criatura, cuyos padecimientos
más me duelen a mí que a ella propia; esta vida que se me va aniquilando
entre el dolor y la melancolía, sin que nada puede reanimarla. En el
primer momento de la catástrofe, Josefina se quedó como si hubiera
perdido la razón. A pesar de nuestros esfuerzos por sujetarla, salió
corriendo a la calle, y sus lamentos dolorosos detenían al pasajero y
contristaban al invencible soldado. Seguímosla, y llamándola sin cesar
con las palabras más cariñosas, intentábamos llevarla a sitio seguro
donde se tranquilizase, pero Josefina no nos oía. En su cerebro agitado
por hirviente excitación reinaba el silencio absoluto. Yo creí que no
sobreviría a aquel trastorno; pero ¡ay, Andresillo!, vive gracias a mis
cuidados, a mi vigilante y previsor estudio por salvarla. Ha permanecido
en cama todo el invierno. Ya ves cómo está. ¿Vivirá? ¿Alargará sus
tristes días hasta el verano? ¿Podrá salir de Gerona dentro de algunos
meses, si resistimos el asedio y se van los franceses? ¿Qué suerte nos
destina Dios en los días que vienen? ¡Pobre niñita mía! Inocente y
débil, sufrirá los horrores del sitio tal vez mejor que nosotros los
fuertes. No sé qué daría porque esta situación terminara pronto,
permitiéndome salir una temporada de campo con mi pobre enferma Pero
figúrate lo que dirían de mí, si ahora escapase de Gerona. No lo quiero
pensar. Me llamarían cobarde y mal patriota. En verdad, muchacho, que no
sé cuál de estos dos calificativos me lastima más. ¡Cobarde o mal
patriota! No\ldots{} aquí, Sr.~de Nomdedeu, señor médico del hospital,
aquí, en Gerona, al pie del cañón, con la venda en una mano y el bisturí
en la otra para cortar piernas, sacar balas, vendar llagas y recetar a
calenturientos y apestados. Vengan granadas y bombas\ldots{} Puede que
se muera mi hija; puede que la débil luz de esta lamparita se apague, no
sólo por falta de aceite, sino por falta de oxígeno; morirá de terror,
de consunción física, de hambre; pero ¡qué vamos a hacer! Si Dios lo
dispone así\ldots{}

Diciendo esto, D. Pablo, vuelto hacia los cristales del balcón, se
limpiaba las lágrimas con un pañuelo encarnado tan grande como una
bandera.

\hypertarget{iv}{%
\chapter{IV}\label{iv}}

Por la noche, después de hacer la guardia en la Torre Gironella, volví a
mi alojamiento y me encontré con una novedad. Pichota había parido, sí,
señores, y la familia de que orgullosamente me consideraba jefe, estaba
aumentada con tres criaturas, a las cuales era preciso mantener. No sé
si he hablado a ustedes de Pichota, hermosa gata parda con manchas, a
quien los tres muchachos profesaban un amor sin límites. Perdóneseme el
descuido por no haberla mencionado antes, y ahora sólo falta decir que
al ver los tres retoños que nos había regalado, dije a Siseta:

---Es preciso que dos de estos caballeritos sean arrojados al Oñá,
porque no estamos para mantener a tanta gente. Luego que acaben de
mamar, será preciso una ración diaria para alimentarlos, y dicen que
vamos a andar escasos.

---Déjalos, hombre---me respondió.---Dios dará para todos, y si no que
se lo busquen ellos mismos. No faltará qué comer en Gerona. Los cerdos
no se meterán con ustedes, y hasta me parece que no se atreverán a
asomar las narices por acá.

---¿Quia, qué se han de atrever?---exclamé yo con festiva ironía.---Nos
tienen mucho miedo. Sube conmigo a la Torre Gironella, y verás los
mosquitos que andan allá por Levante y Mediodía. Franceses en San-Medir,
Montagut y Costa-Roja, franceses en San Miguel y en los Ángeles, y por
variar, franceses en Montelibi, Pau y el llano de Salt. Ya verás, prenda
mía. Aquí somos seis mil quinientos hombres que no bastan para empezar y
tenemos unas murallitas\ldots{} ¡qué obras, válgame Dios! Da miedo
verlas. Figúrate que cuando los lagartos corren por entre las piedras,
estas se mueven y dan unas contra otras. No se puede hablar recio junto
a ellas, porque con el estremecimiento del sonido, se caen de su sitio.
En fin, yo no sé lo que va a pasar cuando abran batería los franceses y
empiecen a bombardearnos.

La señora Sumta, ama de gobierno de don Pablo Nomdedeu, que solía bajar
a darnos conversación en sus ratos de ocio, metió su hocico en nuestro
diálogo, diciendo:

---Tiene razón Andrés. Las murallas de los fuertes parecen una
almendrada hecha con azúcar sin punto. Mi difunto esposo, que de Dios
goce, y que hizo la campaña del Rosellón contra la República de los
cerdos, me decía varias veces: «Si no fuera porque está allí San
Fernando de Figueras con sus murallas de diamante, y aquí los
gerundenses con sus corazones de acero, todas las plazas del Ampurdán
caerían en poder de cualquier atrevido que pasase la frontera.» En fin,
lo de menos será la piedra, con tal que haya hombres de pecho y un buen
español que sepa mandarlos. ¿Y qué me dice usted, Sr.~Andresillo, de ese
encanijado gobernador que nos han puesto?

---D. Mariano Álvarez de Castro. Este fue el que no quiso entregar a los
franceses el Monjuich de Barcelona. Dicen que es hombre de mucho temple.

---Pues no lo parece---repuso la señora Sumta.---Cuando nos mandaron acá
este sujeto en febrero y le vi, al punto lo diputé por poca cosa. ¡Qué
se puede esperar de quien no levanta tanto así del suelo! El otro día
pasó junto a mí, y\ldots{} créalo usted, no me llega al hombro. El tal
D. Mariano Álvarez de Castro me serviría de bastón. ¿Le ha visto usted
la cara? Es amarillo como un pergamino viejo, y parece que no tiene
sangre en las venas. ¡Qué hombres los del día! Quien conoció a aquel
general Ricardos, que no cabía por esa puerta, con un pecho y una
espalda\ldots{} Daba gusto ver su cara redondita y sus carrillos como
clavellinas\ldots{}

---Señora Sumta---dije riendo,---cuando los generales tengan un oficio
semejante al de las amas de cría, entonces se podrá renegar de los que
sean flacos y encanijados.

---No, Andresillo, no digo eso---repuso la matrona.---Lo que digo es que
sin presencia no se puede mandar. Considera tú: cuando una ve a doña
Lucía Fitz-Gerard, coronela del batallón de Santa Bárbara; cuando una ve
aquellas carnes, aquel andar imponente, dan ganas de correr tras ella a
matar franceses. Pero dime, Siseta: ¿no estás tú afiliada en el batallón
de Santa Bárbara?

---Yo, señora Sumta, no sirvo para eso---repuso mi futura
esposa.---Tengo miedo a los tiros.

---Es que nosotras no hacemos fuego, hija mía, al menos mientras estén
vivos los hombres. Llevar municiones, socorrer a los heridos, dar agua a
los artilleros, y si se ofrece, ir aquí o allí con una orden del
general; esta será nuestra ocupación. Ya les he dicho que cuenten
conmigo para todo, para todo, aunque sea para llevar la bandera del
batallón. De veras te digo, Andresillo, que es gran lástima no tener
mejores murallas y un general menos amarillo y con algunos dedos más de
talla.

Yo me reía de las cosas de la señora Sumta, mujer tan amable como
entrometida, y lejos de enojarme sus barrabasadas, nos causaban sumo
gusto a Siseta y a mí, mayormente al ver que en sus visitas, el ama de
gobierno de D. Pablo Nomdedeu no bajaba nunca sin traer algún condumio
para los huérfanos. A eso de las nueve se despidió para regresar a su
alojamiento, y entonces nos dijo:

---Ya la señorita ha de estar acostada. El señor acaba de entrar, y
ahora estará escribiendo su Diario de todos los días, uno al modo de
libro de coro, donde va apuntando lo que le pasa. ¡Ay!, el amo confía
que la niña se curará, y yo, sin ser médico, digo y aseguro que si
alarga hasta que caigan las hojas, será mucho alargar\ldots{} Ahora
estamos empeñados en hacerle creer que la semana que viene iremos a
Castellà. Sí, ¡buena temporada de campo nos espera! Bombas y más bombas.
La niña no se ha de enterar de nada, y el amo dice que aunque arda la
ciudad toda y caigan a pedazos todas las casas, Josefina no lo ha de
conocer. Pues digo, si los cerdos aprietan el cerco como se dice, y
escasean los víveres\ldots{} Pero el amo tampoco quiere que la niña
comprenda que escasean las vituallas. Si tenemos hambre, capaz es mi
señor D. Pablo de cortarse un brazo y aderezar un guisote con él,
haciendo creer a la enferma que tenemos aquel día pierna de carnero.
Bueno va, bueno va. Adiós, Siseta, adiós, Andrés.

Cuando nos quedamos solos dije a mi futura, mirando a los gatillos:

---Sálvense los tres infantes de España. Si hay hambre en Gerona la
carne de gato dicen que no es mala. ¡Ay, Siseta de mi corazón! ¡Cuándo
nos veremos fuera de estas murallas! ¡Cuándo se acabará esta maldita
guerra! ¡Cuándo estaremos tú y yo con los muchachos, Pichota y sus
niños, camino de la Almunia de Doña Godina! ¿Estará de Dios que no nos
sentaremos a la sombra de mis olivos mirando a las ramas para ver cómo
va cuajando la aceituna?

Hablando de este modo me engolfaba en tristes presagios; pero Siseta,
con sus observaciones impregnadas de sentimiento cristiano, daba cierta
serenidad celeste a mi espíritu.

\hypertarget{v}{%
\chapter{V}\label{v}}

El 13 de Junio, si no estoy trascordado, rompieron los franceses el
fuego contra la plaza, después de intimar la rendición por medio de un
parlamentario. Yo estaba en la Torre de San Narciso, junto al barranco
de Galligans, y oí la contestación de D. Mariano, el cual dijo que
recibiría a metrallazos a todo francés que en adelante volviese con
embajadas.

Estuvieron arrojando bombas hasta el día 25, y quisieron asaltar las
torres de San Luis y San Narciso, que destrozaron completamente,
obligándonos a abandonarlas el 19. También se apoderaron del barrio de
Pedret, que está sobre la carretera de Francia, y entonces dispuso el
gobernador una salida para impedir que levantasen allí baterías. Pero
exceptuando la salida y la defensa de aquellas dos torres no hubo hechos
de armas de gran importancia hasta principios de Julio, cuando los dos
ejércitos principiaron a disputarse rabiosamente la posesión de
Monjuich. Los franceses confiaban en que con este castillo tendrían
todo. ¿Creerán ustedes que sólo había dentro del recinto 900 hombres,
que mandaba D. Guillermo Nash? Los imperiales habían levantado varias
baterías, entre ellas una con veinte piezas de gran calibre, y sin cesar
arrojaban bombas a los del castillo, que rechazaron los asaltos con
obuses cargados con balas de fusil. Por cuatro veces se echaron los
cerdos encima, hasta que en la última dijeron «ya no más» y retiraron,
dejando sobre aquellas peñas la bicoca de dos mil hombres entre muertos
y heridos. No puedo apropiarme ni una parte mínima de la gloria de esta
defensa porque la estuve presenciando tranquilamente desde la torre
Gironella\ldots{}

En todo el mes de Julio siguieron los franceses haciendo obras para
aproximarse a la plaza, y viendo que no la podían tomar a viva fuerza,
ponían su empeño en impedir que nos entraran víveres, de cuyo plan
comenzaron a resentirse los ya alarmados estómagos.

En casa de Siseta, sin reinar la abundancia, no se pasaba mal, y con lo
que yo les llevaba, unido a los frecuentes regalos del señor D. Pablo
Nomdedeu, iban tirando los habitantes todos de la cerrajería. Verdad que
yo me quedaba los más de los días mirando al cielo para darles a ellos
lo mío; pero el militar con un bocado aquí y otro allí se mantiene,
sostenido también por el espíritu, que toma su sustancia no sé de dónde.
Yo tenía un placer inmenso, al retirarme a descansar unas cuantas horas
o simplemente unos cuantos minutos nada más, en ver cómo trabajaba
Siseta en su casa, arreglando por puro instinto y nativo genio
doméstico, aquello que no tenía arreglo posible. Los platos rotos eran
objeto de una escrupulosa y diaria revisión, y la vajilla más perfecta
no habría sido puesta con mejor orden ni con tan brillante aparato. En
las alacenas donde no había nada que comer, mil chirimbolos de loza y
lata, que fueron en sus buenos tiempos bandejas, escudillas, soperas y
jarros, aguardaban los manjares a que los destinó el artífice, y los
muebles desvencijados que apenas servían para arder en una hoguera de
invierno, adquirieron inusitado lustre con el tormento de los diarios
lavatorios y friegas a que la diligente muchacha los sujetaba.

---Mira, prenda mía---le decía yo,---se me figura que no vendrá ninguna
visita. ¿A qué te rompes las manos contra esa caoba carcomida y ese pino
apolillado que no sirve ya para nada? Tampoco viene al caso la
deslumbradora blancura de esas cortinas desgarradas, y de esos manteles,
sobre los cuales, por desgracia, no chorreará la grasa de ningún pavo
asado.

Yo me reía, y hasta aparentaba burlarme de ella; pero entretanto una
secreta satisfacción ensanchaba mi pecho al considerar las eminentes
cualidades de la que había elegido para compañera de mi existencia. Un
día, después de hablar de estas cosas, subí a visitar al Sr.~Nomdedeu y
encontrele sumamente inquieto al lado de su hija, que seguía leyendo el
\emph{Quijote}.

---Andrés---me dijo dulcificando su fisonomía para disimular con los
ojos lo que expresaban las palabras,---principian a faltar víveres de un
modo alarmante, y los franceses no dejan entrar en la plaza ni una libra
de habichuelas. Yo estoy decidido a comprar todo lo que haya, a
cualquier precio, para que mi hija no carezca de nada; pero si llegan a
faltar los alimentos en absoluto ¿qué haré?, he reunido bastantes aves;
pero dentro de un par de semanas se me concluirán. Las pobres están tan
flacas que da lástima verlas. Amigo, ya sabes que desde hoy empezamos a
comer carne de caballo. ¡Bonito porvenir! Álvarez dice que no se
rendirá, y ha puesto un bando amenazando con la muerte al que hable de
capitulación. Yo tampoco quiero que nos rindamos\ldots{} de ninguna
manera; pero ¿y mi hija? ¿Cómo es posible que su naturaleza resista los
apuros de un bloqueo riguroso? ¿Cómo puede vivir sin alimento sano y
nutritivo?

La enferma arrojó el libro sobre la mesa, y al ruido del golpe volviose
el padre, en cuya fisonomía vi mudarse con la mayor presteza la
expresión dolorosa en afectada alegría.

En aquel momento trajo la señora Sumta la comida de la señorita, y esta,
como viese un pan negro y duro, lo apartó de sí con ademán desagradable.

El padre hizo esfuerzos por reírse, y al punto escribió lo siguiente:

---¡Qué tonta eres! Este pan no es peor que el de los demás días, sino
mucho mejor. Es negro porque he mandado al panadero que lo amasase con
una medicina que le envié y que te hará muchísimo provecho.

Mientras ella leía, él trinchaba un medio pollo, mejor dicho un medio
esqueleto de pollo, sobre cuya descarnada osamenta se estiraba un
pellejo amarillo.

---No sé cómo la convenceré de que tiene delante un bocado
apetitoso---me dijo con dolor profundo, pero cuidando de conservar la
sonrisa en los labios.---¡Dios mío, no me desampares!

La señora Sumta que estaba detrás del sillón de la enferma, dijo a su
amo:

---Señor, yo no quería decirlo; pero ello es preciso: de las cinco
gallinas que quedaban se han muerto tres, y dos están enfermas.

---¿Es posible? ¡La Santa Virgen nos ayude!---exclamó el doctor,
chupando los huesos del pollo para animar a su hija a que imitara tan
meritoria abnegación.---¡Con que se han muerto! Ya lo esperaba. Dicen
que todas las aves del pueblo se están muriendo. ¿Ha ido usted a la
plaza de las Coles a ver si hay alguna gallina fresca y gorda?

---No hay más que alambres, y algunos lechuzos que dan asco.

---¡Dios me tenga de su mano! ¿Qué vamos a hacer?

Y diciendo esto chupaba y rechupaba un hueso, saboreándolo luego con
visajes de satisfacción, para ponderar de este modo a los ojos de la
enferma la excelencia de aquella vianda. Pero Josefina, después de
probar el seco animal, apartó el plato de sí con repugnancia. D. Pablo,
sin detenerse a escribir, porque en su azoramiento y ansiedad faltábale
la paciencia para recurrir a tan tardo medio, exclamó a gritos:

---¿Qué, no lo quieres? Pues está exquisito, delicioso. Algo flaco; pero
ahora se usan los pollos flacos. Así lo prescribe la higiene, y los
buenos cocineros jamás te ponen en el puchero un ave medianamente
entrada en carnes.

Pero Josefina no oía, como era de esperar, y cerrando los ojos con
desaliento, pareció más dispuesta a dormir que a comer. En tanto D.
Pablo levantábase, y paseando por el cuarto, cruzadas las manos y con
expresión de terror en los ojos, no se cuidaba de disimular su
desesperación.

---Andrés---me dijo,---es preciso que me ayudes a buscar algo que dar a
mi hija. Gallinas, patos, palomas; ¿se han concluido ya las aves de
corral en Gerona?

---Todo se ha concluido---afirmó la señora Sumta con oficiosidad.---Esta
mañana, cuando fui a la formación (pues yo pertenezco a la segunda
compañía del batallón de Santa Bárbara) todos los militares se quejaban
de la escasez de carnes, y la coronela doña Luisa dijo que pronto sería
preciso comer ratones.

---¡Vaya usted al demonio con sus batallones y sus coronelas! ¡Comer
animales inmundos! No, mi pobre enferma no carecerá de alimento sano. A
ver: busquen por ahí\ldots{} pagaré una gallina a peso de oro.

Luego volviéndose a mí, me dijo:

---Cuentan que se espera un convoy de víveres en Gerona, traído por el
general Blake. ¿Has oído tú algo de esto? A mí me lo dijo el mismo
intendente D. Carlos Beramendi, aunque también se manifestó que dudaba
pudiera llegar felizmente aquí. Parece que están en Olot con dos mil
acémilas, y todo se ha combinado para que salga de aquí D. Blas de
Fournás con alguna fuerza, con objeto de distraer a los franceses. ¡Oh!,
si esto ocurriera pronto y nos llegara harina fresca y alguna
carne\ldots{} Si no, dudo que nos escapemos de una horrorosa epidemia,
porque los malos alimentos traen consigo mil dolencias que se agravan y
se comunican con la insalubridad de un recinto estrecho y lleno de
inmundicias. ¡Dios mío! Yo no quiero nada para mí; me contentaré con
tomar en la calle un hueso crudo de los que se arrojan a los perros, y
roerlo; pero que no falte a mi inocente y desgraciada enfermita un
pedazo de pan de trigo y una hila de carne\ldots{} Andrés ¡si vieras qué
malos ratos paso en el hospital! El gobernador ha mandado que los
mejores víveres que quedan se destinen a los soldados y oficiales
heridos, lo cual me parece muy bien dispuesto, porque ellos lo merecen
todo. Esta mañana estaba repartiéndoles la comida. ¡Si vieras qué
perniles, qué alones, qué pechugas había allí! Tuve intenciones de
escurrir bonitamente una mano por entre los platos y pescar un muslo de
gallina, para metérmelo con disimulo en el bolsillo de la chupa y
traérselo a mi hija. Estuve luchando un largo rato entre el afán que me
dominaba y mi conciencia, y al fin, elevando el pensamiento y diciendo:
«Señor, perdóname lo que voy hacer,» me decidí a cometer el hurto.
Alargué los dedos temblorosos, toqué el plato, y al sentir el contacto
de la carne, la conciencia me dio un fuerte grito y aparté la mano; pero
se me representó el estado lastimoso de mi niña y volví a las andadas.
Ya tenía entre las garras el muslo, cuando un oficial herido me vio. Al
punto sentí que la sangre se me subía a la cara, y solté la presa
diciendo: «Señor oficial, no queda duda que esa carne es excelente y que
la pueden ustedes comer sin escrúpulo\ldots» Me vine a casa con la
conciencia tranquila pero con las manos vacías. Y hablando de otra cosa,
amigo Andrés, dicen que al fin se tendrá que rendir Monjuich.

---Así parece, Sr.~D. Pablo. El gobernador ha ofrecido premios y grados
a los seiscientos hombres de D. Guillermo Nash; pero con todo, parece
que no pueden resistir más tiempo. Los que hay dentro del castillo ya no
son hombres, pues ninguno ha quedado entero, y si se sostienen una
semana, es preciso creer que San Narciso hace hoy un milagro más
prodigioso que el de las moscas, ocurrido seiscientos años ha.

---Esta mañana me dijeron que los del castillo no están ya para fiestas;
pero que el gobernador Sr.~Álvarez les manda resistir y más resistir,
como si fueran de hierro los pobres hombres. Diez y nueve baterías han
levantado los franceses contra aquella fortaleza\ldots{} con que
figúrate el sin número de confites que habrán llovido sobre la gente de
D. Guillermo Nash.

---No necesito figurármelo, Sr.~D. Pablo---repuso,---que todo eso lo
tengo más que visto, pues la torre Gironella donde yo estoy, no tiene
ninguna varita de virtudes para impedir que las bombas caigan sobre
ella.

La enferma, levantándose de su asiento sin ser sentida, se acercó a
nosotros.

---Hija mía---le dijo Nomdedeu con sorpresa y cariño a pesar de la
certeza de no ser oído,---tu disposición a andar me prueba que estás
mucho mejor. Unos cuantos paseos por las afueras de la ciudad te
pondrían como nueva. ¡Ay, Andrés!---añadió dirigiéndose a mí,---daría
diez años de mi vida por poder dar diez paseos con mi hija por el camino
de Salt. Por espacio de muchos meses ha permanecido en una postración
lastimosa, y ahora su naturaleza, sintiéndose renacer, busca el
movimiento y quiere sacudir la mortal somnolencia.

Josefina recorría la habitación con paso ligero, y sus mejillas se
tiñeron de levísimo carmín.

---¡Oh, qué alegría!---exclamó D. Pablo.---En todo un año no has andado
tanto como en estos tres minutos. Mira, Andrés, cómo se le colorea el
semblante. La sangre circula, los miembros adquieren soltura y brío, la
apagada pupila brilla con nuevo ardor, y una respiración cadenciosa y
enérgica sale del oprimido pecho.

Diciendo esto mi amigo abrazó y besó a su hija con entusiasmo.

---Aquí tienes, insigne Marijuán---prosiguió con júbilo,---el resultado
de mi sistema. Todos decían: «El Sr.~D. Pablo Nomdedeu, que es tan buen
médico, no curará a su hija.» Y yo digo: «Sí, majaderos, el Sr.~D. Pablo
Nomdedeu, que es un mal médico, curará a su hija.» Mi hija está mejor,
mi hija está buena y con unos cuantos meses de temporada en
Castellà\ldots{}

La enferma, en efecto, manifestaba alguna animación. Al ver las
demostraciones de su padre hizo y repitió enérgicos signos que no
entendí. La falta de oído habíale quitado el hábito de expresarse por la
palabra, adquiriendo con esto insensiblemente la rápida movilidad facial
y manual de los sordo-mudos. Sólo en casos de apuro y cuando no era
comprendida, recurría instintivamente a poner en acción la lengua,
exprimiendo las ideas con cierta oscuridad y siempre con rapidez y
escasa armonía.

---Quiero vestirme---dijo agitando el guardapiés.

---¿Para qué, hija mía?

---¿No vamos esta tarde a Castellà? En el patio dos caballos\ldots{} los
he visto.

Nomdedeu hizo con la cabeza dolorosos signos negativos.

---Esos caballos---me dijo,---son el mío y el del vecino D. Marcos, que
van al matadero.

Josefina corrió a la ventana que daba al patio, volviendo luego a
nuestro lado.

---Quiero salir\ldots{} calle---exclamó con vehemencia.

---Hija mía---dijo D. Pablo, asociando los signos a las palabras,---ya
sabes que ha llovido. Están los pisos llenos de fango. No te sentará
bien. Toma mi brazo y demos unos cuantos paseos, de la sala a la cocina
y de la cocina a la sala.

Josefina mostró inmenso fastidio, y miró a la calle con desconsuelo.

---Aquí tienes un gran compromiso---me dijo el doctor, tirándose de un
mechón de cabellos.

Josefina mirando afuera al través de los vidrios, exclamó:

---¡Qué precioso\ldots{} el cielo!

---Es verdad---repuso el padre.---Pero\ldots{} más vale que te sientes
en tu silloncito. ¿Por qué no tomas alguna cosa? Mira\ldots{} uno de
estos bollitos.

La joven corrió a su asiento y dejose caer en él, apartando con
repugnancia las golosinas que le ofrecía su padre. Luego movió la cabeza
a un lado y otro cerrando los ojos, y pronunciando estas palabras que
caían sobre el corazón del padre como bombas en plaza sitiada.

---¡Guerra en Gerona!\ldots{} ¡Otra vez guerra en Gerona!

Nomdedeu, sin atreverse a contradecirla habíase sentado junto a ella, y
con la cabeza entre las manos lloraba como un chiquillo.

\hypertarget{vi}{%
\chapter{VI}\label{vi}}

A los dos días de acontecido esto, se rindió Monjuich. ¿Qué podían hacer
aquellos cuatrocientos hombres que habían sido novecientos y que
caminaban a no ser ninguno? El 12 de Agosto la guarnición del castillo
se componía de unos trescientos o cuatrocientos hombres, sin piernas los
unos, sin brazos los otros. Monjuich era un montón de muertos, y lo más
raro del caso es que Álvarez se empeñaba en que aún podía defenderse.
Quería que todos fuesen como él, es decir, un hombre para atacar y una
estatua para sufrir; mas no podía ser así, porque de la pasta de D.
Mariano Dios había hecho a D. Mariano, y después dijo: «basta, ya no
haremos más.»

Se rindió el castillo, después de clavar los pocos cañones que quedaron
útiles, y por la tarde de aquel día vimos desfilar a la que había sido
guarnición, marchando la mayor parte al hospital. Todos quisimos ver a
Luciano Aució, el tambor que después de haber perdido una pierna entera
y verdadera, siguió mucho tiempo señalando con redobles la salida de las
bombas; pero Luciano Aució había muerto sacudiendo el parche mientras
tuvo los brazos pegados al cuerpo. Daba lástima ver a aquella gente, y
yo le dije a Siseta que había ido con los tres chicos a la plaza de San
Pedro:

---Como estos medios hombres estaré yo dentro de poco, Siseta, porque ya
que acabaron con Monjuich, ahora la van a emprender con la torre
Gironella, cuyas murallas no se han caído ya\ldots{} por punto.

Los franceses no esperaron al día siguiente para combatir la ciudad, que
se les venía a la mano, una vez que tenían la gran fortaleza, y desde la
misma noche empezaron a levantar baterías por todos lados. Tanta prisa
se dieron que en pocos días alcanzamos a ver muchísimas bocas de fuego
por arriba, por abajo, por la montaña y por el llano, contra la muralla
de San Cristóbal y puerta de Francia. El gobernador, que harto conocía
la flaqueza de aquellas murallas de mazapán, dispuso que se ejecutaran
obras como las de Zaragoza, cortaduras por todos lados, parapetos,
zanjas y espaldones de tierra en los puntos más débiles.

Las mujeres y los ancianos trabajaron en esto, y yo me llevé a la plaza
de San Pedro a mis tres chiquillos, que metían mucho ruido sin hacer
nada. Por la noche regresaron a su casa, completamente perdidos de
suciedad y con los vestidos hechos jirones.

---Aquí te traigo estos tres caballeros---dije a Siseta,---para que los
repases.

Ella se enojó, viéndoles tan derrotados, y quiso pegarles; pero yo la
contuve diciendo:

---Si han ido al trabajo, fue porque así lo ordenó el gobernador D.
Mariano Álvarez de Castro. Son los tres muy buenos patriotas, y si no es
por ellos, creo que no se hubiera acabado hoy la cortadura que cierra el
paso de la calle de la Barca. ¿Ves? Esa arroba de fango que tiene
Gasparó en la cabeza, es porque quiso meter también sus manos en harina,
y subiendo al parapeto, rodó después hasta el fondo de la zanja, de
donde le sacaron con una azada.

Siseta al oír esto, empezó a solfearle en cierta parte, encareciéndole
con enérgicas palabras la conveniencia de que no tomase parte en las
obras de fortificación.

---¿Ves este verdugón que tiene Manalet en el carrillo y en la sien
derecha? ---proseguí librando a Gasparó de las justicias de su
hermana.---Pues fue porque se acercó demasiado al gobernador cuando este
iba con el intendente y toda la plana mayor a examinar las obras. Estas
criaturitas, no contentas con verle de cerca, se metían en el corrillo,
enredándose entre las piernas de D. Mariano en términos que no le
dejaban andar. Un ayudante las espantaba; pero volvían como las moscas
de San Narciso, hasta que al fin, cansados del juego, los oficiales
empezaron a repartir bofetones, y uno de ellos le cayó en la cara a tu
hermano Manalet.

---¡Ay, qué chicos estos!---exclamó Siseta.---Todos desean que se acabe
el sitio para poder vivir, y yo quiero que se acabe para que haya
escuela.

Entre tanto los tres patriotas volvían a todas partes sus ardientes
ojos, en cuya pupila resplandecía el rayo de una vigorosa y exigente
vida; miraban a su hermana y me miraban a mí, atendiendo principalmente
a los movimientos de mis manos, por ver si me las llevaba a los
bolsillos.

---Siseta---dije,---¿no hay nada que comer? Mira que estos tres
capitanes generales me quieren tragar con los ojos. Y verdaderamente,
cómo han de servir a la patria, si no se les pone algún peso en el
cuerpo.

---No hay nada---dijo la muchacha suspirando tristemente.---Se ha
concluido lo que tú trajiste la semana pasada, y hace dos días que la
señora Sumta no me da la más mínima hora, porque parece que arriba
faltan también las provisiones. ¿Nos traes algo esta noche?

Por única respuesta, fijé la vista en el suelo, y durante largo rato
guardamos todos profundo silencio, sin atrevernos a mirarnos. Yo no
llevaba nada.

---Siseta---dije al fin.---La verdad, hoy no he traído cosa alguna.
Sabes que no nos dan más que media ración, y yo había tomado adelantadas
dos o tres diciendo que eran para un enfermo. Esta mañana me dio un
compañero un pedazo de pan y\ldots{} ¿para qué negártelo?\ldots{} tenía
tanta hambre que me lo comí.

Felizmente para todos, bajó la señora Sumta trayendo algunos mendrugos
de pan y otros restos de comida.

\hypertarget{vii}{%
\chapter{VII}\label{vii}}

Así pasaban muchos días, y a los males ocasionados por el sitio, se unió
el rigor de la calorosa estación para hacernos más penosa la vida.
Ocupados todos en la defensa, nadie se cuidaba de los inmundos albañales
que se formaban en las calles, ni de los escombros, entre cuyas piedras
yacían olvidados cadáveres de hombres y animales; ni por lo general, la
creciente escasez de víveres preocupaba los ánimos más que en el momento
presente. Todos los días se esperaba el anhelado socorro y el socorro no
venía. Llegaban, sí, algunos hombres, que de noche y con grandes
dificultades se escurrían dentro de la plaza; pero ningún convoy de
vituallas apareció en todo el mes de agosto. ¡Qué mes, Santo Dios!
Nuestra vida giraba sobre un eje cuyos dos polos eran batirse y no
comer. En las murallas era preciso estar constantemente haciendo fuego,
porque siendo escasa la guarnición, no había lugar a relevos, además de
que el gobernador, como enemigo del descanso, no nos dejaba descabezar
un mal sueño. Allí no dormían sino los muertos.

Este continuado trabajo hizo que durante aquel mes aciago estuviese
hasta ocho días sin ver a mis queridos niños y a Siseta, los cuales me
juzgaron muerto. Cuando al fin los vi, casi les fue difícil reconocerme
en el primer instante; tal era mi extenuación y decaimiento a causa de
las grandes vigilias, del hambre y el continuo bregar.

---Siseta---le dije abrazándola,---todavía estoy vivo aunque no lo
parezca. Cuando recuerdo el enorme número de compañeros míos que han
caído para no volverse a levantar, me parece que mi pobre cuerpo está
también entre los suyos, y que esto que va conmigo es una fantasma que
dará miedo a la gente. ¿Cómo va por aquí de alimentos?

---Con el dinero que me quedaba de lo que tú me diste hemos comprado
alguna carne de caballo. De arriba nos envían algo, porque la señorita
enferma no quiere comer de estos platos que ahora se usan. El
Sr.~Nomdedeu parará en loco, según yo veo, y ayer estuvo aquí todo el
día rellenando de paja dos pieles de gallina, con lo cual hace creer a
su hija que ha recibido aves frescas de la plaza. Después le da carne de
caballo, y echándole discursos escritos le hace comer unas tajaditas. La
señora Sumta salió ayer con su fusil y volvió diciendo que había matado
no sé cuántos franceses. Los tres chicos no me han dejado respirar en
estos ocho días. ¿Querrás creer que ayer se subieron al tejado de la
catedral, donde están los dos cañones que mandó poner el gobernador? Yo
no sé por dónde subieron, mas creo que fue por los techos del claustro.
Lo que no creerás es que Manalet vino ayer muy orgulloso porque le había
rozado una bala el brazo derecho, haciéndole una regular herida, por lo
cual traía un papel pegado con saliva encima de la rozadura. Badoret
cojea de un pie. Yo quiero detener al pequeño; pero siempre se escapa,
marchándose con sus hermanos, y ayer trajo un pedazo de bomba como media
taza, llena de granos de arroz que recogió en medio del arroyo\ldots{} Y
tú ¿qué has oído? ¿Es cierto que vienen socorros por la parte de Olot?
El señor Nomdedeu no piensa más que en esto, y por las noches cuando
siente algún ruido en las calles, se levanta y asomándose por el
ventanillo del patio, dice: «Vecinita, esa gente que pasa me parece que
ha hablado de socorro.»

---Lo que yo te puedo decir, Siseta, es que esta noche a la madrugada
sale alguna tropa de aquí por la ermita de los Ángeles, y se dice que va
a entretener a los franceses por un lado mientras el convoy entra por
otro.

---Dios quiera que salga bien.

Esto decíamos, cuanto se sintió fuerte ruido de voces en la calle. Abrí
al punto la puerta, y no tardé en encontrar algunos compañeros, que
alojados en las casas inmediatas salieron al oír el estruendo de
carreras y voces. La señora Sumta se presentó también a mi vista, fusil
al hombro, y con rostro tan placentero cual si viniese de una fiesta.

---Ya tenemos ahí los socorros---dijo la matrona, descansando en tierra
el fusil con marcial abandono.

Al punto apareció en la ventana alta el busto del Sr.~Nomdedeu, quien
sin poder contener su alegría gritaba:

---¡Ya ha llegado el socorro! ¡Albricias, pueblo gerundense! Señora
Sumta, suba usted a informarme de todo. ¿Pero ha entrado ya el convoy?
Traiga usted inmediatamente todo lo que encuentre a cualquier precio que
lo vendan.

Un soldado, amigo y compañero mío, nos dijo:

---Todavía no ha entrado el convoy en la plaza, ni sabemos cuándo ni por
dónde entrará.

---Lo cierto es que hacia el lado de Bruñola se siente un vivo fuego, y
es que por allí don Enrique O'Donnell se está batiendo con los
franceses.

---También se oye tiroteo por los Ángeles, donde dicen que está Llauder.
El convoy entrará por el Mercadal, si no me engaño.

---Señora Sumta---dijo D. Pablo desde la ventana,---suba usted a
acompañar a mi hija mientras yo voy a enterarme de lo que ocurre; pero
deje usted fuera esos arreos militares y póngase el delantal y la
escofieta. Entre tanto, encienda el fuego, ponga agua en los pucheros,
que si usted va por los víveres yo mondaré luego las seis patatas que
compré hoy y haré todo lo demás que sea preciso en la cocina.

Estas conferencias no se prolongaron mucho tiempo, porque tocaron
llamada y corrimos a la muralla, donde tuvimos la indecible satisfacción
de oír el vivo fuego de los franceses, atacados de improviso a
retaguardia por las tropas de O'Donnell y de Llauder. Para ayudar a los
que venían a socorrernos se dispararon todas las piezas, se hizo un vivo
fuego de fusilería desde todas las murallas, y por diversos puntos
salimos a hostigar a los sitiadores, facilitando así la entrada del
convoy. Por último, mientras hacia Bruñola se empeñaba un recio combate
en que los franceses llevaron la peor parte, por Salt penetraron
rápidamente dos mil acémilas, custodiadas por cuatro mil hombres a las
órdenes del general don Jaime García Conde.

¡Qué inmensa alegría! ¡Qué frenesí produjo en los habitantes de Gerona
la llegada del socorro! Todo el pueblo salió a la calle al rayar el día
para ver las mulas, y si hubieran sido seres inteligentes aquellos
cuadrúpedos, no se les habría recibido con más cariñosas demostraciones,
ni con tan generosa salva de aplausos y vítores. Al pasar por la calle
de Cort-Real, ya entrado el día, encontré a Siseta, a los tres chicos y
a D. Pablo Nomdedeu, y todos nos abrazamos, comunicándonos nuestro gozo
más con gestos que con palabras.

---Gerona se ha salvado---decíamos.

---Ahora que aprieten los \emph{cerdos} el cerco---exclamó D.
Pablo.---¡Dos mil acémilas! Tenemos víveres para un año.

---Bien decía yo---añadió Siseta,---que por alguna parte había de venir.

Aquel día y los siguientes reinó en la plaza gran satisfacción, y hasta
nos hostilizaron flojamente los franceses, porque detuviéronse algunos
días en ocupar las posiciones que habían abandonado a causa de la
jugarreta que se les hizo. En cuanto a los auxilios, pasada la impresión
del primer instante, todos caímos en la cuenta de que los mismos que nos
los habían traído nos los quitarían, porque reforzada la guarnición con
los cuatro mil hombres de Conde, estos nos ayudaban a consumir los
víveres. ¡Funesto dilema de todas las plazas sitiadas! Pocas bocas para
comer dan pocos brazos para pelear. Muchos brazos traen muchas bocas, de
modo que si somos pocos nos vence el arte enemigo; si muchos nos vence
el hambre. Sobre esta contradicción se funda verdaderamente todo el arte
militar de los sitios.

Así lo decía yo a D. Pablo pocos días después de la llegada de las dos
mil acémilas, anunciándole que bien pronto nos quedaríamos otra vez en
ayunas, a lo cual me contestó:

---Yo he hecho grandes provisiones. Pero si el sitio se prolonga mucho,
también se me concluirán. Ahora, según dicen, Álvarez tiene proyectado
hacer un gran esfuerzo para quitarnos de encima esa canalla. Ya sabes
que a fuerza de cañonazos han abierto brecha en Santa Lucía, en Alemanes
y en San Cristóbal. De un día a otro intentarán el asalto. ¿Se podrá
resistir, Andrés? Yo iré a la brecha como todos; pero ¿qué podremos
hacer nosotros, infelices paisanos, contra las embestidas de tan fiero
enemigo?

Desde aquellos días hasta el 15 de Septiembre en que D. Mariano dispuso
una salida atrevidísima, no se habló más que de los preparativos para el
gran esfuerzo, y los frailes, las mujeres y hasta los chicos hablaban de
las hazañas que pensaban realizar, peligros que soportar y dificultades
que acometer, con tan febril inquietud y novelería, como si aguardasen
una fiesta. Yo le dije a Siseta que era preciso se dispusiera a tomar
parte con las de su sexo en la gran función; pero ella, que siempre se
negó a calzar el coturno de las acciones heroicas, me contestó con risas
y bromas que no servía para el caso, pero que si por fuerza la llevaban
a la batalla, haría la prueba de matar algún francés con las tenazas de
la herrería.

La salida del 15 no dio otro resultado que envalentonar a los señores
\emph{cerdos}, los cuales, deseosos de poner fin al cerco, tomando la
ciudad, se nos echaron encima el día 19, asaltando la muralla por
distintos puntos con cuatro fuertes columnas de a dos mil hombres. En
Gerona fueron tan grandes aquella mañana el entusiasmo y la ansiedad,
que hasta se olvidó aquella gente de que nuevamente nos faltaba un
pedazo de pan que llevar a la boca.

Los soldados conservaban su actitud serena e imperturbable; pero en los
paisanos se advertía una alucinación, una al modo de embriaguez, que no
era natural antes del triunfo. Los frailes, echándose en grupos fuera de
sus conventos, iban a pedir que se les señalase el puesto de mayor
peligro: los señores graves de la ciudad, entre los cuales los había que
databan del segundo tercio del siglo anterior, también discurrían de
aquí para allí con sus escopetas de caza, y revelaban en sus animados
semblantes la presuntuosa creencia de que ellos lo iban a hacer todo.
Menos bulliciosos y más razonables que estos, los individuos de la
Cruzada gerundense hacían todo lo posible para imitar en su reposada
ecuanimidad a la tropa. Las damas del batallón de Santa Bárbara no se
daban punto de reposo, anhelando probar con sus incansables idas y
venidas que eran el alma de la defensa; los chicos gritaban mucho,
creyendo que de este modo se parecían a los hombres, y los viejos, muy
viejos, que fueran eliminados de la defensa por el gobernador, movían la
cabeza con incrédula y desdeñosa expresión, dando a entender que nada
podría hacerse sin ellos.

Las monjas abrían de par en par las puertas de sus conventos, rompiendo
a un tiempo rejas y votos, y disponían para recoger a los heridos sus
virginales celdas, jamás holladas por planta de varón, y algunas salían
en falanges a la calle, presentándose al gobernador para ofrecerle sus
servicios, una vez que el interés nacional había alterado pasajeramente
los rigores del santo instituto. Dentro de las iglesias ardían mil velas
delante de mil santos; pero no había oficios de ninguna clase, porque
los sacerdotes, lo mismo que los sacristanes, estaban en la muralla.
Toda la vida, en suma, desde lo religioso hasta lo doméstico, estaba
alterada, y la ciudad no era la ciudad de otros días. Ninguna cocina
humeaba, ningún molino molía, ningún taller funcionaba, y la
interrupción de lo ordinario era completa en toda la línea social, desde
lo más alto a lo más bajo.

Lo extraño era que no hubiera confusión en aquel desbordamiento
espontáneo del civismo gerundense; pues tan grande como este era la
subordinación. Verdad es que D. Mariano sabía establecerla rigurosísima,
y no permitía desmanes ni atropellos de ninguna clase, siendo
inexorablemente enérgico contra todo aquel que sacara el pie fuera del
puesto que se le había marcado.

Las campanas tocaban a somatén, ocupándose en el servicio los chicos del
pueblo, por ausencia de los campaneros, y el cañón francés empezó desde
muy temprano a ensordecer el aire. Los tambores recorrían las calles,
repicando su belicosa música, y los resplandores de los fuegos
parabólicos comenzaron a cruzar el cielo. Todo estaba perfectamente
organizado, y cada uno fue derecho a su sitio, no necesitando preguntar
a nadie cuál era. Sin que sus habitantes salieran de ella, la ciudad
quedó abandonada, quiero decir que ninguno se cuidaba de la casa que
ardía, del techo desplomado, de los hogares a cada instante destruidos
por el horrible bombardeo. Las madres llevaban consigo a los niños de
pecho, dejándoles al abrigo de una tapia, o de un montón de escombros,
mientras desempeñaban la comisión que el instituto de Santa Bárbara les
encomendara. Menos aquellas en que había algún enfermo, todas las casas
estaban desiertas, y muebles y colchones, trapos y calderos en revuelto
hacinamiento obstruían las plazas del Aceite y del Vino.

\hypertarget{viii}{%
\chapter{VIII}\label{viii}}

Yo estaba en Santa Lucía, donde había mucha tropa y paisanos. Allí me
encontré a D. Pablo Nomdedeu, que me dijo:

---Andrés, mis funciones de médico y mi deber de patriota me obligan a
apartarme hoy de mi hija. Mucho he sermoneado a la señora Sumta para que
se quedara en casa: pero ese marimacho me amenazó con denunciarme al
gobernador como patriota tibio si persistía en apartarla de la senda de
gloria por la cual la llevan los acontecimientos. Mírala; ahí está entre
aquellos artilleros, y será capaz de servir sola el cañón de a 12 si la
dejan. La buena Siseta se ha quedado acompañando a mi querida enfermita.
Ya le he dicho que le haré un buen regalo si consigue entretener a la
niña, de modo que esta no comprenda nada de lo que pasa. Es cosa
difícil; pero como no oye ni los cañonazos\ldots{} He clavado todas las
ventanas para que no se asome, y dejando cerrada a la luz solar la
habitación, he encendido el candil, haciéndole creer que hay una fuerte
tempestad de truenos y rayos. Como no caiga una bomba allí mismo o en
las inmediaciones, es probable que nada comprenda, engañada por el
profundo y saludable silencio en que yace su cerebro. ¡Dios mío, aparta
de mí las tribulaciones y libra mi hogar del fuego enemigo! ¡Si me has
de quitar el único consuelo que tengo en la tierra, dale una muerte
tranquila y no conturbes su último instante con la cruel agonía del
espanto! ¡Si ha de ir al cielo, que vaya sin conocer el infierno, y que
este ángel no vea demonios junto a sí en el momento de su muerte!

La señora Sumta, empujando a un lado y otro con sus membrudos brazos,
llegó a nosotros, hablando así a su amo:

---¿Qué hace ahí, señor mío, como un dominguillo? ¿Pero no tiene fusil,
ni escopeta, ni pistolas, ni sable? Ya\ldots{} no lleva más que la
herramienta para cortar brazos y piernas al que lo haya menester.

---Médico soy, y no soldado---repuso don Pablo:---mis arreos son las
vendas y el ungüento, mis armas el bisturí, y mi única gloria la de
dejar cojos a los que debían ser cadáveres. Pero si preciso fuere, venga
un fusil, que curaré españoles con una mano y mataré franceses con la
otra.

Teníamos por jefe en Santa Lucía a uno de los hombres más bravos de esta
guerra, un irlandés llamado D. Rodulfo Marshall, que había venido a
España sin que nadie lo trajese y sólo por gusto de defender nuestra
santa causa. Aventurero o no, Marshall por lo valiente debía haber sido
español. Era rozagante, corpulento, de semblante festivo y mirar
encendido, algo semejante al de D. Juan Coupigny que vimos en Bailén.
Hablaba mal nuestra lengua; pero aunque alguna de sus palabrotas nos
causaban risa, decíalas con la suficiente claridad para ser entendidas,
y nada importaba que destrozara el castellano con tal que destrozase
también a los franceses, como lo hizo en varias ocasiones.

Había que ver el empuje de aquellas columnas de \emph{cerdos}, señores.
No parecían sino lobos hambrientos, cuyo objeto no era vencernos, sino
comernos. Se arrojaban ciegos sobre la brecha, y allí de nosotros para
taparla. Dos veces entraron por ella dispuestos a echarnos de la
cortina; pero Dios quiso que nosotros les echásemos a ellos. ¿Por qué?
¿De qué modo? Esto es lo que no sabré contestar a ustedes si me lo
preguntan. Sólo sé que a nosotros no se nos importaba nada morir, y con
esto tal vez está dicho todo. D. Mariano se presentó allí, y no crean
ustedes que nos arengó hablándonos de la gloria y de la causa nacional,
del rey y de la religión. Nada de eso. Púsose en primera línea,
descargando sablazos contra los que intentaban subir, y al mismo tiempo
nos decía: «Las tropas que están detrás tienen orden de hacer fuego
contra las que están delante, si estas retroceden un solo paso.» Su
semblante ceñudo nos causaba más terror que todo el ejército enemigo.
Como algún jefe le dijera que no se acercase tanto al peligro,
respondió: «Ocúpese usted de cumplir su deber, y no se cuide tanto de
mí. Yo estaré donde convenga.»

Marchose después a otro punto, donde creía hacer falta, y sin él nos
aturdimos de nuevo. Aquel hombre traía consigo una luz milagrosa, que
nos permitía ver mejor el sitio y medir nuestros movimientos y los de
los franceses, para que estos no pudieran echársenos encima. Los
soldados enemigos morían como moscas al pie de la brecha; pero de los
nuestros caían también por docenas. Recuerdo que un compañero mío muy
amado fue herido en el pecho y cayó junto a mí en uno de los momentos de
mayor apuro, de más vivo fuego, de verdadera angustia y cuando un ligero
esfuerzo de más o de menos por una parte u otra habría decidido si la
muralla quedaba por Francia o por España. El desgraciado muchacho quiso
levantarse, pero inútilmente. Dos monjas se acercaron, despreciando el
fuego, y lo apartaron de allí.

Pero la pérdida más sensible fue la del jefe don Rodulfo Marshall. Tengo
la gloria de haberle recogido en mis brazos en el mismo boquete de la
brecha, y no se me olvidará lo que dijo poco después, tendido en la
calle en el momento de expirar: «Muero contento por causa tan justa y
por nación tan brava.»

Cuando esto pasó, ya los franceses indicaban haber desistido de entrar
en la ciudad por aquella parte. Y hacían bien, porque estábamos cada vez
más decididos a no dejarles entrar. Si a tiros no lográbamos
contenerlos, los acuchillábamos sin compasión; y como esto no bastara,
aún teníamos a la mano las mismas piedras de la muralla para arrojarlas
sobre sus cabezas. Esta era un arma que manejaban las mujeres con mucho
denuedo, y desde los contornos llovían guijarros de medio quintal sobre
los sitiadores. Cuando la función en la muralla de Santa Lucía
terminaba, no nos veíamos unos a otros, porque el polvo y el humo
formaban densa atmósfera en toda la ciudad y sus alrededores, y el ruido
que producían las doscientas piezas de los franceses vomitando fuego por
diversos puntos, a ningún ruido de máquinas de la tierra ni de
tempestades del cielo era comparable. La muralla estaba llena de muertos
que pisábamos inhumanamente al ir de un lado para otro, y entre ellos
algunas mujeres heroicas expiraban confundidas con los soldados y
patriotas. La señora Sumta estaba ronca de tanto gritar, y D. Pablo
Nomdedeu, que había arrojado muchas piedras, tenía los dedos magullados;
pero no por esto dejaba de cuidar a los heridos, ayudándole muchas
señoras, algunas monjas y dos o tres frailes, que no valían para cargar
un arma.

De pronto veo venir un chico que se me acerca haciendo cabriolas,
saludándome desde lejos a gritos y esgrimiendo un palo en cuya punta
flotaba el último jirón de su barretina. Era Manalet.

---¿Dónde has estado?---le pregunté.---Corre a tu casa, entérate de si
tu hermana ha tenido novedad, y dile que yo estoy sano y bueno.

---Yo no voy ahora a casa. Me vuelvo a San Cristóbal.

---¿Y qué tienes tú que hacer allí, en medio del fuego?

---La barretina tiene tres balazos---me dijo con el mayor orgullo,
mostrándome el gorro hecho trizas.---Cuando se quedó así la tenía puesta
en la cabeza. No creas que estaba en el palo, Andrés. Después la he
puesto aquí para que la gente la viera toda llena de agujeros.

---¿Y tus hermanos?

---Badoret ha estado en Alemanes, y ahora me dijo que él solo había
matado no sé cuántos miles de franceses, tirándoles piedras. Yo estaba
en San Cristóbal: un soldado me dijo que se le habían acabado las balas,
y que le llevara huesos de guinda, y le llevé más de veinte, Andrés.

---¿Y Gasparó?

---Gasparó anda siempre con mi hermano Badoret. También estuvo en
Alemanes, y aunque Siseta le quiso dejar encerrado en casa, él se escapó
por la puerta de atrás. Ahora hemos estado juntos, buscando algo que
comer en aquel montón de desperdicios que hay en la calle del Lobo; pero
no encontramos nada. ¿Tienes algo, Andrés?

---Algo, ¿qué es eso? ¿Pues acaso queda algo que comer en Gerona? Aquí
no se come más que humo de pólvora. ¿Has visto al gobernador?

---Ahora iba por ahí arriba. Parece como que va al Calvario. Nosotros
bajábamos con otros chicos, y cuando le vimos, pusímonos en fila,
gritando: «Viva Su Majestad el gobernador D. Mariano.» ¡Pues querrás
creer que no nos dijo tanto así! Ni siquiera nos miró.

---¡Hombre, qué falta de cortesía! ¡No saludar a gente tan respetable!

---Después Badoret se metió en las Capuchinas, porque estaba abierta la
puerta. Andrés, ¿sabes que allí hay un soldado muerto que tiene un
tronco de col en la mano? Si me das licencia se lo quitaré.

---No se toca a los muertos, Manalet. Veremos si ahora que hemos
destrozado a los franceses, nos dan alguna cosa.

Infinidad de mujeres ocupábanse allí en retirar a los heridos, y también
repartían a los sanos algunas raciones de pan negro y muy poco vino.
Nosotros veíamos a los franceses, retirándose por el llano adelante, y
no podíamos reprimir un sentimiento de ardiente orgullo al ver el
resultado tan colosal con tan pequeños medios. Parecía realmente un
milagro que tan pocos hombres contra tantos y tan aguerridos nos
defendiéramos detrás de murallas cuyas piedras se arrancaban con las
manos. Nosotros nos caíamos de hambre, ellos no carecían de nada;
nosotros apenas podíamos manejar la artillería, ellos disparaban contra
la plaza doscientas bocas de fuego. Pero ¡ay!, no tenían ellos un D.
Mariano Álvarez que les ordenara morir con mandato ineludible, y cuya
sola vista infundiera en el ánimo de la tropa un sentimiento singular
que no sé cómo exprese, pues en él había además del valor y la
abnegación, lo que puede llamarse miedo a la cobardía, recelo de
aparecer cobarde a los ojos de aquel extraordinario carácter. Nosotros
decíamos que el yunque y el martillo con que Dios forjó el corazón de D.
Mariano no había servido después para hacer pieza alguna.

Manalet se separó de mí, y al poco rato le vi aparecer con otros muchos
chicos, todos descalzos, sucios, harapientos y tiznados, entre los
cuales venía su hermano Badoret, trayendo a cuestas a Gasparó, cuyos
brazos y piernas colgaban sobre los hombros y por la cintura de aquel.
Todos venían muy contentos, y especialmente Badoret que repartía algunas
guindas a sus compañeros.

---Toma, Andrés---me dijo el chico dándome una guinda.---Ya tienes para
todo el día. Toma esta otra y repártela entre tus compañeros, que
tendrán un hambre\ldots{} ¿Sabes cómo las he ganado? Pues te contaré.
Iba yo con Gasparó a cuestas por la calle del Lobo, y vi abierta la
puerta del convento de Capuchinas, que siempre está cerrada. Gasparó me
pedía pan con chillidos y más chillidos, y yo le pegaba de coscorrones
para que callara, diciéndole que si no callaba, se lo contaría al señor
gobernador. Pero cuando vi abierta la puerta del convento, dije: «aquí
ha de haber algo,» y me colé dentro. Metime en el patio, entré después
en la iglesia, pasé al coro; luego a un corredor largo donde había
muchos cuartos chicos, y no vi a nadie. Registré todo, por si caía
cualquier cosa; pero no encontré sino algunos cabos de vela y dos o tres
madejas de seda, que estuve chupando a ver si daban algún jugo. Ya me
volvía a la calle, cuando sentí detrás de mí, \emph{pist},
\emph{pist}\ldots{} pues\ldots{} como llamándome. Miré y no vi nada.
¡Qué miedo, Andrés, qué miedo! Allá a lo último del corredor había una
lámina grande, muy grande, donde estaba pintado el diablo con un gran
rabo verde. Pensé que era el diablo quien me llamaba, y eché a correr.
Pero ¡ay de mí!, que no podía encontrar la salida, y todo era dar
vueltas y más vueltas en aquel maldito corredor; y a todas estas
\emph{pist}, \emph{pist}\ldots{} Después oí que dijeron:---Muchacho, ven
acá.---Y tanto miré por el techo y las paredes que alcancé a ver detrás
de una reja una mano blanca, y una cara arrugada y petiseca. Ya no tuve
miedo, y fui allá. La monjita me dijo:---Ven, no temas, tengo que
hablarte.---Yo me acerqué a la reja y le dije:---Señora, perdóneme usía;
yo creí que era usted el demonio.

---Sería una pobre monja enferma que no pudo salir con las demás.

---Eso mismo. La señora me dijo:---Muchacho, ¿cómo has entrado aquí?
Dios te manda para que me hagas un gran servicio. La comunidad se ha
marchado. Estoy enferma y baldada. Quisieron llevarme; pero se hizo
tarde y aquí me dejaron. Tengo mucho miedo. ¿Se ha quemado ya toda la
ciudad? ¿Han entrado los franceses? Ahora quedándome medio dormida soñé
que todas las hermanas habían sido degolladas en el matadero, y que los
franceses se las estaban comiendo. Muchacho, ¿te atreverás tú a ir ahora
mismo al fuerte de Alemanes y dar esta esquela a mi sobrino don Alonso
Carrillo, capitán del regimiento de Ultonia? Si lo haces, te daré este
plato de guindas que ves aquí, y este medio pan\ldots---Aunque no me lo
diera, lo habría hecho, encantimás\ldots{} Cogí la esquela, ella me dijo
por dónde había de salir, y corrí a los Alemanes. Gasparó chillaba más;
pero yo le dije:---Si no callas te metemos dentro de un cañón como si
fueras bala, disparamos, y vas a parar rodando a donde están los
franceses, que te pondrán a cocer en una cacerola para comerte.---Llegué
a Alemanes. ¡Qué fuego! Lo de aquí no es nada. Las balas de cañón
andaban por allí como cuando pasa una bandada de pájaros. ¿Crees que yo
les tenía miedo? ¡Quia! Gasparó seguía llorando y chillando; pero yo le
enseñaba las luces que despedían las bombas, le enseñaba las chispas de
los fogonazos, y le decía:---¡Mira qué bonito! Ahora vamos nosotros a
disparar también los cañones.---Un soldado me dio una manotada,
echándome para afuera, y caí sobre un montón de muertos; pero me levanté
y seguí \emph{palante}. Entró el gobernador, y cogiendo una gran bandera
negra que parece un paño de ánimas, la estuvo moviendo en el aire, y
luego les dijo que al que no fuera valiente le mandaría ahorcar. ¿Qué
tal? Yo me puse delante y grité:---Está muy bien hecho.---Unos soldados
me mandaron salir, y las mujeres que curaban a los heridos se pusieron a
insultarme, diciendo que por qué llevaba allí esta criatura\ldots{} ¡Qué
fuego! Caían como moscas; uno ahora, otro en seguida\ldots{} Los
franceses querían entrar, pero no los dejamos.

---¿Tú también?

---Sí; las mujeres y los paisanos echaban piedras por la muralla abajo
sobre los marranos que querían subir; yo solté a Gasparó, poniéndolo
encima de una caja donde estaba la pólvora y las balas de los cañones, y
también empecé a echar piedras. ¡Qué piedras! Una eché que pesaba lo
menos siete quintales, y cogió a un francés, partiéndolo por mitad.
Aquello tenía que ver. Los franceses eran muchos, y nada más sino que
querían subir. Vieras allí al gobernador, Andresillo. D. Mariano y yo
nos echamos pa delante\ldots{} y nos pusimos a donde estaba más apurada
la gente. Yo no sé lo que hice, pero yo hice algo, Andrés. El humo no me
dejaba ver, ni el ruido me dejaba oír. ¡Qué tiros! En las mismas orejas,
Andrés\ldots{} Está uno sordo. ¡Yo me puse a gritar llamándoles
marranos, ladrones y diciendo que Napoleón era un acá y un allá! Puede
que no me oyeran con el ruido; pero yo les puse de vuelta y media. Nada,
Andrés, para no cansarte, allí estuve mientras no se retiraron. El
gobernador me dijo que estaba satisfecho, no, a mí no me habló nada, se
lo dijo a los demás.

---¿Y la carta?

---Busqué al Sr.~Carrillo. Yo le conocía; lo encontré al fin cuando todo
se acabó. Dile el papel, y me dio un recado para la señora monja. Luego
acordándome de Gasparó, fui a recogerle donde le había dejado, pero no
lo encontré. Todo se me volvía gritar: «¡Gasparó, Gasparó!» pero el niño
no parecía. Por fin me lo veo debajo de una cureña, hecho un ovillo, con
los puños dentro de la boca, mirando afuera por entre los palos de la
rueda y con cada lagrimón\ldots{} Echémele a cuestas y corrí a las
Capuchinas. Pero aquí viene lo bueno, y fue que como yo venía pensando
en batallas, y con la cabeza llena de todo aquello que había visto, se
me olvidó el recado que me dio el señor Carrillo para la monjita. Ella
me reprendió, diciéndome que yo había roto la carta y que la quería
engañar, por lo cual no pensaba darme el plato de guindas ni el pan
ofrecidos. Se puso a gruñir y me llamó mal criado y bestia. Gasparó
echaba sangre del dedo de un pie y la monjita le lió un trapo; pero las
guindas\ldots{} nones. Por fin, amigo Andrés, todo se arregló porque
vino el mismo Sr.~Carrillo, con lo cual la señora me dio las guindas y
el pan y eché a correr fuera del convento.

---Lleva este chico a tu casa para que le cuide tu hermana---dije
reparando que el pobre Gasparó sangraba aún del pie.

---Después---me contestó.---He guardado algunas guindas para Siseta.

---Muchachos---gritó Manalet que se había alejado con sus compañeros y
volvía a la carrera:---por la calle de Ciudadanos va el gobernador con
mucha gente, muchas banderas; delante van las señoras cantando, y los
frailes bailando, y el obispo riendo, y las monjas llorando. Vamos allá.

Como se levanta y huye una bandada de pájaros, así corrieron y volaron
aquellos muchachos, dejando libre de su infantil algazara la muralla de
Santa Lucía. Yo no me moví de allí en todo el día, y las señoras nos
repartieron raciones de pan y carne, ambos manjares de detestable sabor
y olor; pero como no había otra cosa, fuerza era apechugar con ello, sin
mostrar asco, ni repugnancia, ni desgana, para no enojar a D. Mariano.

Al anochecer, y cuando marchaba de Santa Lucía al Condestable, encontré
a D. Pablo Nomdedeu en la calle de la Zapatería, donde había varios
heridos arrojados por el suelo.

---Andrés---me dijo,---todavía no he vuelto a mi casa. ¿Pasará algo?
Creo que en la calle de Cort-Real no ha caído ninguna bomba. ¡Cuánto
herido, Dios mío! La jornada ha sido gloriosa; pero nos ha costado cara.
Ahora mismo estuvo aquí el gobernador visitando a esta pobre gente, y
les dijo que la guarnición y los paisanos habían dejado atrás en el día
de hoy a los más grandes héroes de la antigüedad.

---¿Ha curado usted muchos heridos?

---Muchísimos, y aún quedan bastantes. Mis compañeros y yo nos
multiplicamos; pero no es posible hacer más. Yo quisiera tener cien
manos para atender a todos. También yo estoy herido. Una bala me tocó el
brazo izquierdo; pero no es cosa de cuidado. Me he liado un trapo y no
he tenido tiempo para más\ldots{} ¿Qué habrá sido de mi pobre hija?

---Pronto lo sabremos, Sr.~D. Pablo. La noche llega. Hecha la primera
cura de estos heridos, usted podrá ir un rato a su casa, y yo espero que
me den licencia por una hora.

\hypertarget{ix}{%
\chapter{IX}\label{ix}}

Cuando fui a la casa, ya cerca de las diez, aún no había regresado D.
Pablo. Dejé abajo el fusil, y subí sin tardanza, anhelando saber de
Siseta y de la señorita, y a las dos me las encontré en la sala en
actitud no muy tranquilizadora. Estaba Josefina recostada en su silla
con muestras de decaimiento y postración; pero con los ojos abiertos,
atentamente fijos en la puerta. De rodillas a su lado, Siseta le tomaba
las manos y con ademanes y palabras tiernas, a pesar de no ser oídas,
procuraba tranquilizarla.

---Gracias a Dios que viene alguien de la casa---me dijo Siseta.---¡Qué
día hemos pasado! ¿Y el Sr.~D. Pablo, y la señora Sumta, y mis tres
hermanos?

Respondile que a ninguno de los nuestros había pasado desgracia, y ella
prosiguió:

---La señorita quería salir a la calle, y he tenido que luchar con ella
para detenerla. Todo lo comprende, y aunque no oye los cañonazos, se
estremece toda y tiembla cuando resuena alguno, aunque sea muy lejano.
Tan pronto lloraba, como caía en mis brazos desmayada llamando sin cesar
a su padre. La pobrecita sabe muy bien que hay guerra en Gerona. Yo
también he tenido un miedo\ldots{} Figúrate: aquí solas\ldots{} A cada
instante me parecía que la casa se venía al suelo. Pero lo peor fue que
se nos metieron aquí unos hombres. No me quiero acordar, Andrés. A eso
de las dos, y cuando pareció que se acababan los tiros, entraron seis o
siete patriotas, unos con uniforme, otros sin él y todos con fusiles.
Cuando nos vieron, empezaron a reírse de nuestro susto, y luego dieron
en registrar la casa, diciendo que querían llevarse todo lo que había de
comida, porque la tropa estaba muerta de hambre. La señorita se quedó
como difunta cuando los vio, y ellos por broma nos apuntaban con los
fusiles para oírnos gritar llamando a todos los santos en nuestra ayuda.
Aunque eran unos bárbaros, no nos hicieron daño alguno más que el gran
susto y el llevarse cuanto encontraron en la cocina y en la despensa.
¡Ay, Andrés! No han dejado nada de lo que el Sr.~D. Pablo había
guardado, y esta noche no se encontrará aquí ni una miga de pan que
llevar a la boca. ¡Cómo se reían los malditos al meter en un gran saco
lo mucho y bueno que encontraron! Yo les rogué que dejasen alguna cosa;
pero volvieron a apuntarme con los fusiles, diciendo que la tropa tenía
ganas, y que la señora Sumta les había dicho que estas despensas estaban
bien provistas.

No había concluido mi amiga su relación, cuando entró el Sr.~D. Pablo;
mas para no presentarse a su hija con el brazo manchado de sangre, pasó
a una habitación interior, con objeto de arreglarse un poco y vendar su
herida, en cuyo sitio me reuní con él para contarle lo ocurrido.

---¡Dios y la Virgen Santísima nos amparen!---exclamó con
consternación.---¡Con que me han saqueado la casa! La culpa la tiene esa
maldita, y siempre habladora Sumta, que por todas partes ha de ir
pregonando si tenemos o no tenemos provisiones. ¿Y mi hija? La pobrecita
habrá comprendido que se encuentra en el cráter de un espantoso volcán,
y serán inútiles todas nuestras comedias para convencerla de lo
contrario. Es preciso buscar algo que comer, Andrés, sí, algo que comer.
Mi hija se morirá de terror; pero no quiero que se muera de hambre.

---Nada se encuentra en Gerona---respondí,---y menos a estas horas.

---¡Qué calamidad! Pero cómo es posible\ldots---dijo en la mayor
confusión, mientras yo le vendaba la herida, y se mudaba de
vestido.---¡Ay!, cómo me duele el brazo; pero es preciso disimular.
Andrés, no te marches. Esta noche necesito de tu ayuda\ldots{} Es
preciso que busquemos algún alimento.

Al presentarse delante de su hija, ésta mostró su alegría claramente,
abrazándole con cariño; pero al punto sus ojos revelaron vivísimo
espanto, echó atrás la cabeza y cruzando las manos exclamó, «sangre.»

---¿Qué hablas de sangre, hija mía?---dijo el padre desconcertado.---Que
estoy manchado de sangre\ldots{} Ya\ldots{} sí, en la chupa hay algunas
gotas\ldots{} pero déjame que te cuente. ¿Sabes que he ido de caza?

La muchacha no entendía.

---Que fui de caza---escribió en el pliego de papel D. Pablo.---Fue un
compromiso; no me pude evadir. El magistral y D. Pedro me cogieron, y
zas, al campo\ldots{} He matado tres conejos.

La enferma oprimiéndose la cabeza entre las manos, exclamó:

---¡Guerra en Gerona!

---¿Qué hablas ahí de guerra? Lo que hay es que hemos tenido un fuerte
temporal\ldots{} Me he mudado de ropa, porque me puse como una uva. ¿Has
comido hoy bien?

---No ha tomado nada---dijo Siseta.---Ya sabrá su merced por Andrés, que
unos bergantes saquearon la casa.

Esto pasaba, cuando sentimos gran estruendo en lo bajo de la casa, no
estampido de bombas y granadas, sino clamor chillón y estridente, de mil
desacordes ruidos compuesto, tales como patadas, bufidos, cacharrazos y
sones bélicos de varia índole; pero que al pronto revelaban proceder de
una muchedumbre infantil que se había metido por las puertas adentro.
Nomdedeu lleno de confusión, miraba a todos lados, inquiriendo con los
ojos qué podía ser aquello; pero pronto él y los demás salimos de dudas,
viendo entrar una turba de chiquillos, que desvergonzadamente y sin
respeto a nadie, se colaron en la sala, dando golpes, empujándose,
chillando, cacareando y berreando en los más desacordes tonos. Dos de
ellos llevaban sendos cacharros colgados al cinto, y sobre cuyo abollado
fondo redoblaban con palillos de sillas viejas; varios tocaban la
trompeta con la nariz, y todos al compás de la inaguantable música
bailaban con ágiles brincos y cabriolas. Parecía una chusma infernal que
salía de las escuelas de Plutón.

No necesito decir que al frente del ejército venían Manalet y Badoret,
este último llevando a cuestas a Gasparó, tal como le vi en la muralla.
Ninguno dejaba de llevar palo, caldero viejo o vara con pingajos
colgados de la punta, con cuyos objetos se simulaban fusiles, tambores y
banderas. Un fondo de silla de paja atado a una cuerda y arrastrado por
el suelo, servía de trofeo a uno, y otro adornaba su cabeza con un cesto
medio deshecho, no faltando las casacas de militares hechas jirones y
los morriones de antigua forma con descoloridas plumas adornados.

D. Pablo, ciego de cólera y fuera de sí, apostrofó a los muchachos tan
violentamente, que casi casi estuvieron a punto de aplacar un poco su
entusiasmo bélico.

---Granujas, largo de aquí al instante---les dijo.---¿Qué desvergüenza
es esta? ¡Meterse en mi casa de este modo!

Siseta, indignada de tal audacia, cogió por un brazo a Manalet, que
acertara a pasar junto a ella, y comenzó a vapulearle de un modo
lastimoso. Yo también tomé parte en la persecución del enjambre, y
empezó el reparto de pescozones a diestra y siniestra. Pero de pronto
observamos que la enferma contemplaba a los desvergonzados muchachos con
complaciente atención y sonreía con tanta espontaneidad y desahogo como
si su alma sintiera indecible gozo ante aquel espectáculo. Hícelo notar
al Sr.~D. Pablo, y al punto este se puso de parte de los alborotadores,
conteniendo a Siseta que iba sobre ellos con implacable furor.

---Dejarlos---dijo Nomdedeu.---Mi hija demuestra que está muy complacida
viendo a esta canalla. Mira cómo se ríe, Andrés; observa cómo les
aplaude. Bien, muchachos; corred y chillad alrededor del cuarto.

Y diciendo esto D. Pablo, poniéndose en medio de la sala, empezó a
llevar el compás. En mal hora se les ordenó seguir. ¡Santo Dios! ¡Qué
algazara, qué estrépito! Parecía que la sala se iba a hundir. Baste
decir que se extralimitaron de tal modo, y de tal modo se dejaron llevar
a los últimos delirios de la travesura, que al fin fue preciso poner
freno a tanto juego y vocerío, porque hasta llegó el caso de que los
transeúntes se detuvieran en la calle, sorprendidos y escandalizados por
tan desusado rumor.

---¿Dónde has estado todo el día?---exclamó Siseta echando mano a
Barodet y deteniéndole.---¡Y la criatura tiene sangre en el pie! Ven
acá, condenado; me las pagarás todas juntas. Espera a que bajemos a
casa, y verás. Y tú, Manalet de mil demonios, ¿qué has hecho de la
camisa?

---En la calle de la Ballestería estaban curando unos heridos y no
tenían trapos. Me quité la camisa y la di.

---¿Para qué habéis traído a casa tanto muchacho mal criado?

---Son nuestros amigos, hermana---repuso Badoret.---Hemos estado en el
Capitol y allí nos han dado un poco de vino. Hermana, aquí en el seno te
traigo cinco guindas.

---Marrano, ¿piensas que las voy a comer de tus manos asquerosas? Ven
acá, Gasparó. Este pobrecito no habrá comido nada. ¿Qué te han hecho en
el pie, que tienes sangre?

---Hermana, una bala de cañón pasó por donde estábamos, y si Gasparó no
se hace para un lado, le lleva medio cuerpo; no le cogió más que la uña
chica. ¡Si vieras qué valiente ha estado! Se metió debajo del cañón y
allí se estuvo mirando a los franceses que querían subir a la muralla. Y
les amenazaba con el puño cerrado. ¡Bonito genio tiene mi niño! Pues no
creas\ldots{} Ningún francés se metió con él.

---Te voy a desollar vivo---le dijo Siseta.---Espera, espera a que
bajemos. A ver si se marcha pronto de aquí toda esa canalla.

---No, que se aguarden un poco---indicó don Pablo.---Son unos
jovenzuelos muy salados. Mira qué contenta está Josefina. Lo que quiero,
Badoret, es que no metáis mucho ruido. Bailen ustedes, y marchen de
largo a largo por toda la casa; pero sin gritar para que no se
escandalice la vecindad. Y dime, Manalet, ¿traen ustedes algo de comer?

---Yo traigo cinco guindas---dijo prontamente Badoret, sacándolas del
seno.

---Dadme con disimulo y sin que lo vea mi hija todo lo que traigáis, que
yo os daré ochavos para que compréis pólvora.

---Pauet tiene cuatro guindas---dijo Manalet.

---Pues vengan acá.

---Y yo tengo también un pedazo de pan, que me sobró del de la monja.

---Pepet---dijo otro de mis chicos,---trae acá ese medio pepino que le
cogiste al soldado muerto.

---Yo doy este pedazo de bacalao---dijo otro entregando la ofrenda en
manos de D. Pablo.

---Y yo esta cabeza de gallina cruda---añadió un tercero.

En un momento se reunieron diversos manjares tales como troncos de col,
que llevaban impreso el sello de las limpias manos de sus generosos
dueños; garbanzos crudos que habían sido sacados por los agujeros de las
sacas por sutilísimos dedos; algunos pedazos de cecina, andrajos de
buñuelos, zanahorias, dos o tres almendras en confite, que ya habían
recibido muchas mordidas, y otras viandas, tan liberalmente entregadas
como alegremente recibidas. Procurando que no se enterase su hija, llamó
D. Pablo a la señora Sumta, que acababa de llegar en aquel instante, y
llevándola tras el sillón de la enferma, le dijo:

---A ver si con todo esto compone usted una cena para la enferma. Es
preciso hacerle creer que nadamos en la abundancia.

---¿Qué hemos de hacer con esto, señor, si no lo querrán ni las
gallinas? En casa no falta qué comer.

---¡Maldita sargentona; todo se lo han llevado, todo lo han saqueado
unos malditos militares que se entraron aquí! Si usted no fuera tan
entrometida, tan bocona, y tan amiga de meterse donde no la llaman y de
hablar lo que nadie la pregunta, no nos veríamos en esta\ldots{} Y no
digo más. Avíe usted una cena con esto; que mañana Dios dirá. ¿Se ha
olvidado usted de cocinar? ¡Lástima que no se le reventara el fusil
entre las manos, a ver si se curaba de sus locuras! A la cocina. ¡Uf!
Pronto, a la cocina. Está usted apestando a pólvora.

Los muchachos, que como todos los de su edad, eran de los que si se les
da el pie se toman la mano, luego que se vieron autorizados por el dueño
de la casa para hacer de las suyas, dieron rienda suelta a la bulliciosa
iniciativa, y no fue gresca la que armaron. Rodeando la mesa que la
enferma tenía ante su sillón, no se dieron por satisfechos con mirar los
distintos objetos que en ella había, sino que en todos pusieron las
manos, tocando, tentando y moviendo cuanto vieron. Josefina, lejos de
manifestar disgusto por tanta impertinencia, se reía de ver su
inquietud. Por señas indicó a su padre que debía dar de cenar a los
importunos visitantes, a lo que contestó con palabras y cierta festiva
ironía D. Pablo:

---Sí, ahora. Sumta les está preparando un opíparo banquete.

Padre e hija dialogaron un rato como Dios les dio a entender, y al fin
la enferma, con voz clara y entera, habló así:

---No, no me pueden convencer de que no hay guerra en Gerona. Usted no
ha ido de caza, sino a curar los heridos, y estos chicos que vienen
imitando a los soldados hacen ahora lo mismo que han visto.

---¡Qué habladora está!---dijo Nomdedeu.---Buen síntoma. En un año no le
he oído tantas palabras juntas. Está visto que las travesuras y lindezas
de estos muchachos han reanimado su espíritu. Andrés y tú, Siseta;
riámonos todos, mostrando hallarnos muy satisfechos.

Según la orden del amo, prorrumpimos en sonoras risas, siendo al punto
excesivamente secundados al punto por el coro infantil. D. Pablo sentose
luego junto a ella, y tomando la pluma se preparó a comunicarle algo
grave y largo y difícil de exprimir por señas, pues sólo en este caso se
valía Nomdedeu del lenguaje escrito. Púseme tras de su asiento, y pude
leer, mientras escribía, lo que sigue:

---Hija mía, tienes razón. Hay guerra en Gerona. Yo no te lo quería
decir por no asustarte; pero pues lo has adivinado, basta de engaños y
comedias. Ni yo he estado de caza, ni he pensado en ello. Voy a contarte
lo ocurrido para que no estimes ni en más ni en menos los sucesos de
este gran día. Cierto es que los franceses han vuelto a poner cerco a
Gerona. Hace tiempo que se presentó amenazándonos un ejército de
docientos mil hombres, mandados por el mismo emperador Napoleón en
persona.

Josefina al leer esto que era de lo más gordo, mironos a todos,
interrogándonos con los ojos acerca de la exactitud de tal noticia, y no
necesitamos que D. Pablo nos lo advirtiera para hacer demostraciones
afirmativas que hubieran convencido a la misma duda. El padre continuó
así:

---Has de saber que ahora tenemos aquí un gobernador que llaman D.
Mariano Álvarez de Castro, el cual en cuanto vio venir a los franceses
dispuso las cosas de manera que no quedara uno solo para contarlo.
Concertó de modo que un ejército español de quinientos mil hombres, que
estaba ahí por Aragón sin saber qué hacerse, viniese en nuestra ayuda
por el lado de Montelibi, precisamente cuando los franceses nos atacaban
esta mañana por el otro lado. Al amanecer rompieron el fuego; desde la
muralla de Alemanes se veía a Napoleón I montado en un caballo y con un
grandísimo morrión todo lleno de plumas en la cabeza. Embisten los
franceses\ldots{} ¡Ay!, hija mía: habías tú de ver aquello. Nuestros
soldados los barrían materialmente, y como a la hora de empezar el
combate apareció el ejército de quinientos mil hombres como llovido, los
pobres cerdos no supieron a qué santo encomendarse. En fin, hija mía,
les hemos dado una paliza tal, que a estas horas van todos camino de
Francia con su Emperador a la cabeza, con lo cual se acaba la guerra y
pronto tendremos aquí a nuestro rey Femando.

Josefina volvió a asesorarse de nosotros antes de dar crédito a tales
maravillas.

---Yo no te lo había querido decir---continuó Nomdedeu,---por no
asustarte; pero el júbilo de la ciudad es tan grande, que ni aun tú que
estás tan retraída podrías dejar de conocerlo. Lo mismo que estos
chicos, andan los mayores por el pueblo, entregados a las
manifestaciones de un delirante regocijo. Figúrate que en los pasados
días, los franceses que andaban por ahí, no permitían llegar comestibles
al pueblo y hoy todo es abundancia, y además de lo que puede venir,
tenemos todo lo que al enemigo se ha cogido, que es, si no me engaño,
tantos miles de bueyes, no sé cuántos millones de sacos de harina, y los
miles de los miles en gallinas, huevos, etc\ldots{} Ya podemos marchar a
Castellà cuando quieras\ldots{}

---Mañana mismo---dijo Josefina con afán.

---Sí, mañana mismo---escribió D. Pablo.---Estamos como queremos, y
jamás ha tenido Gerona temporada más alegre, más animada. La gente está
loca de contento, y todo se vuelve cantos y bailes y felicitaciones y
regocijos. Como los víveres han entrado esta tarde con abundancia
fenomenal, hija mía, yo te he traído de todo cuanto hay en la plaza; y
aunque tu estómago sigue débil, yo creo que debes tomar de todo, con tal
que sea en dosis muy pequeñas. Sobre esto consulté a D. Pedro, mi
compañero en el hospital, y me dijo que convenía alimentarte con una
gran diversidad de manjares, tomando de cada uno ración muy mínima y
cuidando según lo ordena Hipócrates, de que alternen en un mismo plato
la cecina y las guindas, los buñuelos con la leguminosa cicer pisum, que
llamamos garbanzo, y las almendras confitadas con esa planta salutífera
que se conoce en la ciencia por Beta vulgaris latifolia, y que
comúnmente llamamos acelga, manjar de gran virtud medicinal si se le
mezcla con dulce, con nueces y hasta con un poquito de bacalao. Con que
disponte a cenar, que mañana si el día está bueno, se podrá ir a
Castellà, aunque a decir verdad, hija mía, ahora caigo en que tal vez
sea difícil, porque todos los carros y caballerías del pueblo los ha
tomado la Junta con objeto de organizar la gran procesión y cabalgata
con que ha de celebrarse este triunfo sin igual. Pero será cosa de dos o
tres días. Es preciso que te animes para salir a ver las iluminaciones
de esta noche, aunque hablando en puridad no te conviene tomar el
sereno; y para que participes de la común alegría, aquí tenemos a Andrés
y a Siseta, que se prestarán a bailar delante de ti con los chicos un
poco de sardana y otro poco de tira-bou, comenzando esta noche, para que
también en esta casa se manifieste la inmensa satisfacción y patriótico
alborozo de que está poseída la ciudad. Como tú no oyes, suprimiremos el
fluviol y la tanora que sólo sirven para meter inútil ruido. Con que
puedes dar la señal para que comience la fiesta. Yo voy un instante a
preparar en el comedor la riquísima y abundante cena con que
obsequiaremos a estos jóvenes, así como a los preciosos y bien educados
niños.

Y luego volviéndose a Siseta y a mí, nos dijo:

---No hay más remedio. Es preciso bailar un poquito, aunque supongo,
Andrés, que ese cuerpo, venido hace poco de Santa Lucía, no estará para
sardanas. Pero, amigos, bailando hacéis una obra de caridad. ¡Quién lo
había de decir! ¡Hay tantas maneras de practicar el santo Evangelio!

\hypertarget{x}{%
\chapter{X}\label{x}}

El lector no lo creerá; el lector encontrará inverosímil que bailásemos
Siseta y yo en aquella lúgubre noche, precisamente en los instantes en
que incendiados varios edificios de la ciudad, esta ofrecía en su
estrecho recinto frecuentes escenas de desolación y angustia. Formando
con ocho chiquillos un gran ruedo, bailamos, sí, obedeciendo a la
apremiante sugestión de aquel padre cariñoso que nos pedía con lágrimas
en los ojos nuestra cooperación en la difícil comedia con que engañaba
al delicado espíritu de su hija; pero bailamos en silencio, sin música,
y nuestras figuras movibles y saltonas tenían no sé qué mortuorio
aspecto. Nuestras sombras proyectadas en la pared remedaban una danza de
espectros, y los únicos rumores que a aquel baile acompañaban eran,
además de nuestros pasos, el roce de los vestidos de Siseta, el
retemblar del piso, y un ligero canto entre dientes de Badoret que al
mismo tiempo hacía ademán de tocar el fluviol y la tanora.

Por mi parte sostenía interiormente una ruda lucha conmigo mismo para
contraer y esforzar mi espíritu en la horrible comedia que estaba
representando, e iguales angustias experimentaba Siseta, según después
me dijo.

Al fin la turbación moral, unida al cansancio, me hicieron exclamar: «ya
no puedo más,» arrojándome casi sin aliento en un sillón. Lo mismo hizo
Siseta.

Pero Josefina que nos contemplaba con indecible satisfacción y agrado,
pidionos que bailásemos más, y con elocuentes miradas dirigidas a su
padre, nos decía que éramos unos holgazanes sin cortesía. Vierais allí
al buen D. Pablo suplicándonos que bailáramos por la salvación eterna; y
¿qué habíamos de hacer? Bailamos como insensatos segunda y tercera
tanda. Al fin nos sirvió de pretexto para descansar el hecho de servirse
a la desgraciada joven la hipocrática cena de que antes he hecho
mención, la cual fue acompañada de elocuentes discursos mímicos y
literarios del doctor Nomdedeu, quien ponderaba a su idolatrada enferma
las excelencias del repugnante pisto, servido en nueve o diez platos con
raciones microscópicas. Todo aquello era una farsa lúgubre que oprimía
el corazón, y don Pablo que la presidía, el infeliz D. Pablo, escuálido,
ojeroso, amarillo, trémulo, parecía haber salido de la sepultura y
esperar el canto del gallo para volverse a ella. Siseta lloraba a
escondidas, y algunos de los chicos, rendidos al poderoso sueño y a la
gran fatiga, habían estirado los miembros y cerrado los ojos en diversos
puntos, y donde cada cual encontró mejor comodidad y fácil postura.

---Sr.~D. Pablo---dije al médico,---no nos mande usted bailar más,
porque nosotros mismos creeremos que estamos locos.

---Hijos míos---me contestó,---tengo el corazón partido de dolor.
Necesito estar en batalla constantemente para contener las lágrimas que
se me caen de los ojos. ¡Pobre Gerona! ¿Existirás mañana? ¿Estarán
mañana en pie tus nobles casas y con vida tus valientes hijos? ¡Yo tengo
espíritu para todo; para lamentar y llorar la muerte de mi ciudad natal,
y atender al cuidado de mi pobre hija! ¿Qué cuesta representar esta
farsa? Nada; la pobrecita se deja engañar fácilmente, y como su
enfermedad no es otra cosa que una fuerte pasión de ánimo, en el ánimo
se han de aplicar los cauterios, las cataplasmas, los tónicos y los
emolientes que le he recetado esta noche. Puede que le hayamos salvado
la vida. ¿Sabéis lo que significan en naturaleza tan delicada, tan
sutilmente sensible, una triste o agradable impresión? Pues significa
tanto como la vida o la muerte. Sí, hijos míos: si yo no cuidara de
ocultar a mi hija las angustias que atravesamos, se pondría su alma en
tales términos que el menor accidente la mataría, como un soplo de
viento apaga la luz. Es preciso resguardar esta pobre lámpara del aire
que la mata, y darla el que la vivifica. Así va tirando, tirando, y
quién sabe si la podré salvar. Sed, pues, caritativos, y procurad
divertirla. Ved cómo se ríe; reparad qué precioso color han tomado sus
mejillas. La creencia de que Gerona está llena de felicidades y la
esperanza de ser llevada pronto a Castellà, la fortifican y dan nueva
vida. Esta noche marchamos bien; pero mañana ¿qué haré, qué la diré
mañana? Si crece la escasez de víveres, como es probable, si se declaran
el hambre y la epidemia, y caen bombas en parajes cercanos o aquí mismo,
¿qué comedia representaremos? Dios me favorezca y me inspire, pues para
su infinita misericordia nada hay imposible.

---Estoy muerto de cansancio---dije yo, viendo que Josefina pedía más
baile---y además es tarde y tengo que marcharme a mi puesto.

Siseta ya no podía tenerse en pie, y la señora Sumta, que yacía en el
suelo con la inmovilidad de un talego, roncaba sonoramente, remedando en
la cavidad de sus fosas nasales el lejano zumbido del cañón. Badoret,
cansado ya de tocar en silencio el fluviol y la tanora, dormía como los
demás chicos. D. Pablo, bastante generoso para no exigirnos imposibles,
se apresuró a complacer a la enferma, poseída de cierto febril insomnio,
y se puso a danzar en medio de la sala haciendo corro con cuatro chicos
de los más despabilados. Cuando yo salí, quedaba el pobre señor haciendo
piruetas y cabriolas con ningún arte y mucha torpeza; pero su
incapacidad para el baile, provocando la hilaridad de su hija, más le
inducía a seguir bailando. Daba saltos, alzaba los brazos
descompasadamente, se descoyuntaba de pies y manos, tropezaba a cada
instante, inclinándose adelante o atrás, hacía mil paseos estrambóticos
y mil figuras grotescas que en otra ocasión me habrían hecho reír, y un
sudor angustioso afluía de su rostro macilento, desfigurado por las
muecas y visajes que le obligaban a hacer el fatigoso movimiento y los
agudos dolores de su herida. Nunca vi espectáculo que tanto me
entristeciera.

\hypertarget{xi}{%
\chapter{XI}\label{xi}}

Esto que he referido a ustedes se repitió algunos días. Después vinieron
circunstancias distintas y todo cambió. Los franceses escarmentados con
la vigorosa y nunca vista defensa del 19 de Setiembre, mediante la cual
estrelláronse contra todos los puntos de la muralla que quisieron
franquear, no se atrevían al asalto. Tenían miedo, dicho sea sin
petulancia; conocían la imposibilidad de abrir las puertas de Gerona por
la fuerza de las armas, y se detuvieron en su línea de bloqueo, con
intención de matarnos de hambre. El 26 de Setiembre llegó al campo
enemigo el mariscal Augereau, el cual dicen se había distinguido en las
guerras de la república y en el Rosellón; trajo consigo más tropas, las
cuales poniéndonos por todos lados cerco muy estrecho, nos encerraron en
términos que no podía entrar ni una mosca. Excusado es decir a ustedes
que los pocos víveres que había se fueron acabando hasta que no quedó
nada, sin que el gobernador diera a esto importancia aparente, pues cada
hora se sostenía más en su tema de que Gerona no se rendiría mientras él
viviese, y aunque media población sucumbiera a las penas del hambre y a
las calenturas que se iban desarrollando al compás de no comer.

Ya no era posible pensar en socorros, como no vinieran por los aires. Ya
no teníamos el triste recurso de buscar la muerte en las murallas,
porque ellos no se cuidaban de asaltarlas, y era forzoso cruzarse de
brazos y dejarse morir, mirando la efigie impasible de don Mariano
Álvarez, cuyos ojos vivos no paraban nunca observando aquí y allí
nuestras caras, por ver si alguna tenía trazas de desaliento o cobardía.
Estábamos moralmente aprisionados entre las garras de acero de su
carácter, y no nos era dado exhalar una queja ni un suspiro, ni hacer
movimiento que le disgustara, ni dar a entender que amábamos la
libertad, la vida, la salud. En suma, le teníamos más miedo que a todos
los ejércitos franceses juntos.

Morir en la brecha es no sólo glorioso, sino hasta cierto punto
placentero. La batalla emborracha como el vino, y deliciosos humos y
vapores se suben a la cabeza, borrando de nuestra mente la idea del
peligro, y en nuestro corazón el dulce cariño a la vida; pero morir de
hambre en las calles es horrible, desesperante, y en la tétrica agonía
ningún sentimiento consolador ni risueña idea alborozan el alma irritada
y furiosa contra el mísero cuerpo que se le escapa. En la batalla, la
vista del compañero anima; en el hambre el semejante estorba. Pasa lo
mismo que en el naufragio; se aborrece al prójimo, porque la salvación,
sea tabla, sea pedazo de pan, debe repartirse entre muchos.

Llegó el mes de Octubre y se acabó todo, señores: se acabó la harina, la
carne, las legumbres. No quedaba sino algún trigo averiado, que no se
podía moler. ¿Por qué no se podía moler? Porque nos comimos las
caballerías que movían los molinos. Se pusieron hombres; pero los
hombres extenuados de hambre, se caían al suelo. Era preciso comer el
trigo como lo comen las bestias, crudo y entero. Algunos lo machacaban
entre dos piedras, y hacían tortas, que cocían en el rescoldo de los
incendios. Aún quedaban algunos asnos; pero se acabó el forraje, y
entonces los animalitos se juntaban de dos en dos y se mantenían
comiéndose mutuamente sus crines. Fue preciso matarlos antes que
enflaquecieran más; al fin la carne de asno, que es la más desabrida de
las carnes, se acabó también. Muchos vecinos habían sembrado hortalizas
en los patios de las casas, en tiestos y aun en las calles; pero las
hortalizas no nacieron. Todo moría, humanidad y naturaleza, todo era
esterilidad dentro de Gerona, y empezó una guerra espantosa entre los
diversos órdenes de la vida, destruyéndose de mayor a menor. Era una
guerra a muerte en la animalidad hambrienta, y si al lado del hombre
hubiera existido un ser superior, nos hubiéramos visto cazados y
engullidos.

Yo padecía las más crueles penas, no sólo por mí, sino por la infeliz
Siseta y sus tres hermanos, que carecían absolutamente de todo. Los
chicos eran al principio los mejor librados, porque ellos salían a la
calle, y merodeando o husmeando aquí y allá, siempre sacaban alguna
cosa; pero Siseta, la pobre Siseta, no tenía más amparo que yo, y yo me
volvía loco para buscarle el sustento. Había, sí, algunos víveres en la
plaza, y se encontraban pececillos del Oñá, que más que peces parecían
insectos, y pájaros escuálidos, que eran cazados desde los tejados:
también había alguna carne de mulo y de perro; pero para adquirir estos
artículos se necesitaba dinero, mucho dinero, y nosotros no lo teníamos.
La ración de trigo seco había llegado a sernos tan repugnante como un
veneno.

D. Pablo Nomdedeu gastaba todos sus ahorros para poner a su hija una
mala comida, y fue de los que dieron por una gallina diez y seis o
veinte pesos, cuando algún payés, afrontando mil peligros y venciendo
obstáculos mil, lograba entrar en la plaza. En los días de la gran
escasez, la señora Sumta no bajaba nada a casa de Siseta, y los chicos
se secaban los ojos mirando a la escalera por ver si descendía por ella
algún maná. Llegó también el día en que Badoret, Manalet y Gasparó se
cansaron de sus correrías por las calles, porque de todas partes eran
expulsados los muchachos vagabundos, por la mala opinión que había
respecto a la limpieza de sus manos. Flacos y casi desnudos, mis tres
hermanos o mis tres hijos, pues como a tales traté siempre, inspiraban
profunda compasión, y formando lastimero grupo junto a Siseta,
permanecían largas horas en silencio, sin juegos ni risas, tan graves
como ancianos decrépitos; inertes y quebrantados, sin más apariencia de
vida que el resplandor de sus grandes ojos negros, llenos de ansioso
afán. Siseta les miraba lo menos posible, deseando así conservar la
calma que se había impuesto como un deber, y hasta se atrevía a mostrar
conatos de severidad, creyendo equivocadamente que en tal trance la
fuerza moral servía de alguna cosa.

Yo estuve tres días sin verlos, porque mis obligaciones me impedían ir a
la casa. Cuando fui, encontreles en la situación que he descrito.

Desde luego admiré la entereza de los pobres niños, bastante
inteligentes para no importunarnos pidiéndonos lo que sabían no podíamos
darles. Únicamente Gasparó, comiéndose sus puños y bebiéndose sus
lágrimas, faltaba a la circunspección sostenida por sus hermanos. Llegó
un momento en que Siseta, no pudiendo contener su dolor, empezó a llorar
amargamente registrando después los últimos rincones de la casa por ver
si parecía de milagro alguna vianda. Yo salí, volví a entrar, salí de
nuevo y regresé, después de dar mil vueltas, con la terrible evidencia
de que no podía encontrar nada.

Repentinamente me ocurrió una idea salvadora.

---Siseta---dije a mi amiga.---Hace días que no veo a Pichota; pero
supongo que andará por ahí con sus tres gatitos.

---¡Oh!---me respondió con dolor.---¿No sabes que el Sr.~D. Pablo ha
acabado con toda la familia? ¡Pobre Pichota! Él dice que es una carne
excelente; pero yo creo que me moriría de hambre antes de comerla.

---¿Ha muerto Pichota? No sabía nada: ¿y también los tres
angelitos?\ldots{}

---No te lo quería decir. En estos últimos días que has faltado de casa,
D. Pablo bajaba con frecuencia. Un día se me puso delante de rodillas
rogándome que le diera algo para su hija, pues ya no tenía víveres, ni
dinero para comprarlos. Cuando esto me decía, uno de los gatitos me
saltó al hombro, y D. Pablo, echándole mano con mucha presteza, se lo
guardó en el bolsillo. Al día siguiente bajó de nuevo y me ofreció los
muebles de su sala si le daba otro de los hijos de Pichota, y sin
aguardar mi contestación, entró en la cocina, después en el cuarto
oscuro, púsose en acecho y lo mismo que un gato caza al ratón, así cazó
él al gato. Cuando salió tuve que curarle los arañazos que traía en la
cara. El tercero pereció de la misma manera, y después de esto Pichota
ha desaparecido de la casa, tal vez por haber entendido que no está
segura.

Siseta y yo convenimos en que era preciso rezar, con la esperanza de que
a fuerza de ruegos, nos enviase Dios por sus misteriosos caminos, algo
de lo que tanto necesitábamos. Pero rezamos y Dios no nos mandó nada.

\hypertarget{xii}{%
\chapter{XII}\label{xii}}

Meditaba sobre la deserción del pobre animal cuando se nos presentó de
repente Nomdedeu. Su aspecto era por demás macilento y cadavérico,
habiendo perdido a fuerza de padeceres físicos y morales hasta aquella
bondadosa expresión y el dulce acento que le distinguían. Su vestido
estaba desordenado y roto, y traía la escopeta de caza y un largo
cuchillo de monte.

---Siseta---dijo bruscamente, y olvidándose de saludarme, a pesar de que
hacía algunos días que no nos veíamos.---Ya sé dónde está esa pícara de
Pichota.

---¿En dónde, Sr.~D. Pablo?

---En el desván que hay en el fondo del patio y que servía de pajar y
granero cuando yo tenía caballo.

---Tal vez no será ella---dijo mi amiga en su generoso anhelo de salvar
al pobre animal.

---Sí, es ella, te digo que es ella. A mí no se me despinta Pichota. La
muy tunanta saltó esta mañana por la ventana de la despensa y me robó un
pernil que allí tenía. ¡Qué atrevimiento! Comerse la carne de su propio
hijo. Es preciso acabar con ese animal. Siseta, ya te he dado gran parte
de mis muebles en cambio de los gazapos. No me queda otra cosa de valor
que mis libros de medicina. ¿Los quieres a trueque de Pichota?

---Sr.~D. Pablo, ni los muebles, ni los libros tomaré; coja usted a
Pichota, y ya que nos vemos reducidos a tal extremidad, dé una parte a
mis hermanos.

---Está bien---respondió Nomdedeu.---Andrés, ¿te atreves a cazar ese
terrible animal?

---No creo que sean precisos tantos pertrechos militares---respondí.

---Pues yo sí lo creo. Vamos allá.

Barodet y su hermano quisieron seguirnos, pero Siseta los contuvo,
diciéndoles que no fueran curiosos ni entrometidos; y solos el médico y
yo subimos al desván, entrando despacio y con precauciones por temor a
ser acometidos del rabioso carnicero, a quien el hambre y el instinto de
conservación debían haber dado una ferocidad extraordinaria. D. Pablo,
porque la presa no se escapara, cerró por dentro la puerta y quedamos
casi en completa oscuridad, pues la débil luz que por un estrecho
ventanillo entraba, no aclaró el lóbrego recinto sino cuando nuestros
ojos fueron perdiendo poco a poco el deslumbramiento de la luz exterior.
Multitud de objetos, como muebles destrozados y viejos obstruían buena
parte de la estancia y sobre nuestras cabezas flotaban densos cortinajes
de tela de araña, guarnecidos por el polvo de un siglo. Cuando empezamos
a ver los contornos y las oscuras tintas del recinto, buscamos con los
ojos al prófugo; pero nada vimos, ni se oyó ruido alguno que indicase su
presencia. Manifesté mis dudas a D. Pablo; pero él me dijo:

---Sí, aquí está. La vi entrar hace un momento.

Movimos algunas cajas vacías, arrojamos a un lado algunos pedazos de
silla y un pequeño tonel, y entonces sentimos el roce de un cuerpo que
se deslizaba en el fondo de la pieza atropellando los hacinados objetos.
Era Pichota. Vimos en el fondo oscuro sus dos pupilas de un verde
aurífero, vigilando con feroz inquietud los movimientos de sus
perseguidores.

---¿La ves?---dijo el doctor.---Toma mi escopeta y suéltale un tiro.

---No---repuse riendo.---Es muy fácil errar la puntería. De nada sirve
en este caso el fusil. Póngase usted a ese lado y deme el cuchillo.

Las dos pupilas permanecían inmóviles en su primera posición, y aquella
lumbre verdosa y dorada que no se parece a la irradiación de ninguna
otra mirada, ni de piedra alguna, produjo en mí fuerte impresión de
terror. Después distinguí el bulto del animal, y sus manchas parduscas y
negras sobre amarillo se multiplicaban a mis ojos, ensanchando su cuerpo
hasta darle las proporciones de un tigre. Yo tenía miedo, ¿a qué negarlo
con pueril soberbia?, y por un momento sentime arrepentido de haber
emprendido obra tan difícil. D. Pablo que tenía más miedo que yo, daba
diente con diente.

Celebramos consejo de guerra, del cual salió que debíamos tomar la
ofensiva; pero cuando cobrábamos algún valor sentimos un sordo ronquido,
un ruido entre arrullo y estertor que anunciaba las disposiciones
hostiles de Pichota. En su lenguaje, la gata nos decía: «Asesinos de mis
hijos, venid acá, que os espero.»

Pichota, que primero estaba en postura de esfinge, se agachó sentando la
angulosa cabeza sobre las patas delanteras, y entonces su mirada cambió,
despidiendo una luz azul que proyectaba de dos rayas verticales. Parecía
fruncir el torvo ceño. Luego irguió la cabeza, pasose las patas por la
cara, limpiando los largos bigotes; y dio algunas vueltas sobre sí
misma, para bajar a un sitio más cercano, donde se puso en actitud de
salto. La fuerza muscular que estos animales tienen en las
articulaciones de sus patas traseras es inmensa, y desde su puesto podía
saltar hasta nosotros. Yo observé que las miradas del animal se dirigían
más rectamente a D. Pablo que a mí.

---Andrés---me dijo,---si tú tienes miedo, yo me voy encima de ella. Es
una vergüenza que un animal tan pequeño acobarde de este modo a dos
hombres. Sí; señora Pichota, nos la comeremos a usted.

Parece que el animal oyó y entendió estas amenazadoras palabras, porque
aún no había acabado de pronunciarlas mi amigo, cuando con ligereza suma
lanzose sobre él, haciéndole presa en el cuello y en los hombros. La
lucha fue breve y la gata había puesto ya en ejecución el conjunto de su
potencia ofensiva, de modo que el resto del combate no podía menos de
sernos favorable. Acudí en defensa de mi amigo, y el animal cayó al
suelo, llevándose en las uñas algunas pequeñas partículas de la persona
del buen doctor, haciéndome a mí algunos desperfectos en la mano
derecha. Corrió luego en distintas direcciones, pero al lanzarse sobre
mí, tuve la buena suerte de recibirla con la punta del cuchillo de
monte, lo cual puso fin al desigual combate.

---Este animal es más temible de lo que creí---me dijo D. Pablo,
apoderándose del cuerpo palpitante.

---Ahora, Sr.~Nomdedeu---dije yo,---partiremos como hermanos la presa.

El doctor hizo una mueca que indicaba su profundo disgusto, y
limpiándose la sangre del cuello, me dijo con tono agresivo que por
primera vez entonces oí de sus labios:

---¿Qué es eso de partir? Siseta contrató conmigo a Pichota a cambio de
mis libros. ¿Tú sabes que mi hija no ha comido nada ayer?

---Todos somos hijos de Dios---repuse,---y también Siseta y los de abajo
han de comer, Sr.~D. Pablo.

Nomdedeu se rascó la cabeza, haciendo con boca y narices contracciones
bastante feas; y tomando el animal por el cuello me dijo:

---Andrés, no me incomodes. Siseta y los bergantes de sus hermanos
pueden alimentarse con cualquier piltrafa que busquen en la calle; pero
mi enferma necesita ciertos cuidados. Después de hoy viene mañana, y
tras mañana pasado. Si ahora te doy media Pichota, ¿qué le daré a mi
hija dentro de un par de días? Andrés, tengamos la fiesta en paz. Busca
por ahí algo que echar a tus chiquillos, que ellos con roer un hueso
quedarán satisfechos; pero haz el favor de no tocarme a Pichota.

De esta manera el corazón de aquel hombre bondadoso y sencillo se
llenaba de egoísmo obedeciendo a la ley de las grandes calamidades
públicas, en las cuales, como en los naufragios, el amigo no tiene
amigo, ni se sabe lo que significan las palabras prójimo y semejante.
Oyendo a D. Pablo, despertose en mí igual sentimiento egoísta de la
vida, y vi en él un aborrecido partícipe de la tabla de salvación.

---Sr.~Nomdedeu---exclamé con súbita cólera,---he dicho que Pichota se
partirá, y no hay más sino que se partirá.

El médico al oír este resuelto propósito, mirome con profunda aversión
por algunos segundos. Sus labios temblaban sin articular palabra alguna:
púsose pálido, y luego con un gesto repentino, me empujó hacia atrás
fuertemente. Yo sentí que mi sangre abrasada corría hacia el cerebro, un
repentino escalofrío que circuló por mi cuerpo me crispaba los nervios.
Cerrando los puños, alargué las manos casi hasta tocar con ellas la cara
de Nomdedeu, y grité:

---¿Con que no se parte Pichota? Pues mejor. Mejor, porque es toda para
mí. ¿Qué tengo yo que ver con la señorita Josefina, ni con sus males
ridículos? Dele usted telarañas.

Nomdedeu rechinó los dientes, y sin contestarme se fue derecho hacia el
animal que yacía en tierra desangrándose. Hice yo igual movimiento;
nuestras manos se chocaron, forcejeamos un breve instante, descargué
sobre él mis puños, y Nomdedeu rodó por el suelo largo trecho, dejándome
en completa posesión de la presa.

---¡Ladrón!---exclamó.---¿Así me robas lo que es mío? Aguarda y verás.

Recogiendo la víctima, me dispuse a salir. Pero Nomdedeu corrió, mejor
dicho, saltó como un gato hacia donde estaba la escopeta, y tomándola,
me apuntó al pecho diciendo con trémula y ronca voz:

---Andrés, canalla: suéltala o te asesino.

Miré en derredor mío buscando el cuchillo de monte; pero ya D. Pablo lo
tenía en el cinto. Corrí a la puerta del desván y no pude abrirla;
entrome de súbito un terror que no pude vencer, y salté maquinalmente,
sin saber lo que hacía, hacia los cajones vacíos, los muebles viejos y
el montón de cachivaches donde se nos había aparecido Pichota. Mis pies
se hundían entre tablas desvencijadas cuyos clavos me lastimaban, y mi
cabeza tropezó en las vigas del techo haciendo caer el polvo, la polilla
y las repugnantes inmundicias depositadas por dos siglos.

---Bárbaro---grité desde arriba,---ya me las pagarás todas juntas.

Pero Nomdedeu seguía tras mí, buscando la puntería y con pie firme
hollaba las rotas tablas; yo corrí de un extremo a otro seguido por él,
y dimos varias vueltas, subiendo, bajando, hundiéndonos y levantándonos
en los desfiladeros, laberintos y sinuosidades de aquella caverna.

Por fin, habiendo salido el tiro, Nomdedeu extendió su hocico como ávido
cazador, por ver si me había alcanzado. Felizmente la bala no me tocó.

---No me ha tocado---dije con furiosa alegría, disponiéndome a caer
sobre mi enemigo.

Pero él desenvainó al instante su cuchillo, y con acento más
frenéticamente alegre que el mío, gritó en medio del desván:

---¡Ven, ven!\ldots{} ¡Ladrón, que quieres matar de hambre a mi
hija!\ldots{} Suelta a Pichota, suéltala, miserable.

Y sin esperar a que yo le acometiera, corrió hacia mí. Entrome mayor
pánico que cuando me perseguía con la escopeta, y de nuevo nos lanzamos
a los precipicios en miniatura, tropezando y saltando, yo delante, él
detrás, yo gritando, él rugiendo, hasta que rendido de fatigas caí entre
destrozadas tablas que me impedían todo movimiento. Me encontré débil y
me reconocí cobarde, sintiéndome incapaz de luchar con aquella furia,
metamorfosis del hombre más manso, más generoso y humanitario que yo
había conocido.

---Sr.~D. Pablo---dije,---tome usted a Pichota. No puedo más. Se ha
vuelto usted tigre.

Sin contestarme nada, y mostrando la horrible agitación y crisis de su
alma en un sordo mugido, recogió el animal que yo había arrojado lejos
de mí, y abriendo la puerta, se marchó.

Yo, después de pasada la irascibilidad de aquel cuarto de hora, apenas
me podía tener, salí, bajé a casa de Siseta, y cuando esta me vio
magullado, arañado y cubierto de polvo, tuvo miedo. En pocas palabras
contele lo ocurrido, y los tres muchachos me oyeron con espanto.

---No hay nada por hoy---les dije con angustia.---Voy a la calle a ver
si encuentro una persona caritativa.

Siseta se abrazó a sus hermanos, derramando lágrimas de desesperación, y
yo corrí desolado fuera de la casa. En la calle marchaba como un ebrio,
sin dirección, ni aplomo, ni camino, y con la mente en ebullición,
cargada, atestada y henchida de criminales ideas.

\hypertarget{xiii}{%
\chapter{XIII}\label{xiii}}

A mi paso encontraba las familias desvalidas, formando horrorosos grupos
de desolación en medio de la vía pública, con los pies en el lodo y
guarecida la cabeza del sol y la lluvia bajo miserables toldos de sucias
esteras. Se arrancaban de las manos unos a otros la seca raíz de
legumbre, el fétido pez del Oñá, las habas carcomidas y los huesos de
animales no criados para la matanza. Diestros carniceros, improvisados
por la necesidad, perseguían por todos los rincones de Gerona a los
pobres perros, que bastante inteligentes para comprender su próxima
suerte, buscaban refugio en lo más recóndito, y aún se atrevían a
traspasar la muralla, corriendo a escape hacia el campo francés, donde
eran acogidas con aplauso y algazara tales pruebas de nuestra penuria.
Por todas partes, en sótanos y tejados, los gatos se defendían con sus
ásperas uñas del ataque de la humanidad, empeñada en vivir.

Los soldados recibían su ración de trigo seco; pero los habitantes de la
ciudad tenían que buscarse el sustento como Dios les daba a entender. La
caza y la pesca eran la ocupación más importante. En cuanto a los
trabajos militares, no había nada, porque nuestra situación consistía en
recibir bombas y granadas, sin poder apenas devolverles los saludos. En
varias partes pedí que me dieran algo para unos pobres huérfanos, pero
la gente me miraba con indignación, y alguno me echó en cara mi
robustez. Yo estaba en los puros huesos.

En la calle de Ciudadanos y en la plaza del Vino\footnote{Hoy de la
  Constitución.} vi muchos enfermos que habían sido sacados de los
sótanos para que se murieran menos pronto. Su mal era de los que
llamaban los médicos fiebre nerviosa castrense, complicada con otras
muchas dolencias, hijas de la insalubridad y del hambre; y en los de
tropa todas estas molestias caían sobre la fiebre traumática.

Sin quererlo yo, me apartaba a cada instante de mi objeto, que era
buscar alimento para mis niños, y aquí me llamaban para que ayudasen a
arrastrar un enfermo, allí me rogaban que ayudara a poner tierra encima
de los cadáveres. Mi deseo era arrojarme como los demás en medio del
arroyo esperando la muerte; pero el ejemplo de algunos que resistían con
sin igual tesón el cansancio, me obligaba a seguir en pie. En la calle
de la Zapatería Vieja sacamos fuera de los sótanos a varios clérigos,
ancianos y niños, mereciendo en premio de nuestro servicio algunos
pedazos de pan negro y de cecina. Los otros devoraban su parte; pero yo
guardé la mía, adquiriendo con su posesión la fuerza moral que había
perdido.

La calle o callejón de la Forsa, que conduce desde la Zapatería Vieja a
la catedral, era una horrible sentina, una acequia angosta y lóbrega,
donde algunos seres humanos yacían como en sepultura esperando quien los
socorriese o quien los matase. Entramos en ella, conducidos por D.
Carlos Beramendi, hombre de gran mérito que se multiplicaba para
disminuir en lo posible las desgracias de la ciudad, y recogimos los
cuerpos vivos y medio vivos, muertos y medio muertos, sacándolos a las
gradas de la catedral, donde les bañasen aires menos corrompidos. La
catedral ya no podía contener más enfermos y la plaza se fue
convirtiendo en hospital al descubierto. Allí vi aparecer en lo alto de
la gradería a D. Mariano Álvarez, que daba algunas disposiciones para el
socorro de los heridos. Su semblante era en toda Gerona el único que no
tenía huellas de abatimiento ni tristeza, y conservábase tal como en el
primer día del sitio. Gran número de gente le rodeaba, y entre ellos vi
con sorpresa a D. Pablo Nomdedeu con otros médicos, individuos de la
junta de salubridad y varias personas influyentes. La multitud vitoreó a
Álvarez, quien no dijo nada, absteniéndose de manifestar disgusto ni
alegría por la ovación, y descendió tranquilamente. La gradería ofrecía
el más lamentable aspecto y con la algazara de los vivas y aclamaciones
dirigidas al gobernador era difícil oír las quejas y lamentos. Desde
lejos se observaba claramente que muchos de los que componían la
comitiva del héroe estaban afligidos ante tan doloroso espectáculo. Sin
duda hablaban a D. Mariano de la escasez de víveres, porque se oyó una
voz de protesta que dijo: «Señor, cuando no haya otra cosa, comeremos
madera.»

En esto llegó junto a mí D. Pablo Nomdedeu, que se había separado un
poco de la comitiva.

¡Comer madera!---exclamó.---Eso se dice, pero no se hace. Andrés, me
alegro de verte por aquí. ¿Cómo estás, y Siseta y los chicos?

Aunque empezaba a extinguirse en mi alma el resentimiento, amenacé con
el puño a Nomdedeu.

---¡Ah, todavía me guardas rencor por lo de esta
mañana!---dijo.---Andresillo, en estos casos no es uno dueño de sí
mismo. Yo me espantaba entonces y me he espantado después de encontrarme
tan bárbaro y salvaje. Se trata de vivir, Andrés, y el pícaro instinto
de conservación hace que el hombre se convierta en fierecita. Que yo sea
capaz de matar a un semejante, es cosa que no se comprende; ¿no es
verdad? ¡Ay, amigo mío! La idea de que mi hija me pide de comer y no
puedo darle nada, ahoga en mí el patriotismo, el pensamiento, la
humanidad, trocándome en una bestia. Andrés, no somos más que miseria.
Indigno linaje humano, ¿qué eres? Un estómago y nada más. Se avergüenza
uno de ser hombre, cuando llegan estos casos en que todas las relaciones
sociales desaparecen y reina la Naturaleza pura. Pero estoy viendo que
el número de heridos es inmenso. Hoy hemos estado haciendo el recuento
de medicinas, y no hay ni para la décima parte en un solo día. ¿A dónde
vamos a parar? ¿Es posible que esto se prolongue? No, no puede ser. Mira
qué horroroso aspecto presenta la gradería cubierta de cuerpos humanos.

En efecto, los cien escalones que conducen a la catedral ofrecían en
pavoroso anfiteatro un cuadro completo de los males de la heroica
ciudad.

Álvarez con su comitiva seguía bajando, y la multitud apartábase para
abrirle paso.

---Señor---le dijo Nomdedeu, volviéndome la espalda.---Olvidé decir a
vuecencia que los medicamentos que tenemos no bastan ni para la décima
parte.

D. Mariano miró fríamente y sin marcada expresión al médico. ¡Qué bien
vi entonces al célebre gobernador, y cuán presentes se quedaron desde
entonces en mi mente sus facciones, su mirar y sus palabras! La cara
pálida y curtida, los ojos vivos, el pelo cano, la figura delgada y
enjuta, la contextura de acero, la fisonomía imperturbable y estatuaria,
la tranquilidad y la serenidad juntas en su semblante; todo lo examiné,
y todo lo retuve en la memoria.

---Si no hay bastantes medicinas---dijo,---empléense las que hay y
después se hará lo que convenga.

Esta muletilla de \emph{lo que convenga} era muy suya, y con ella solía
terminar sus discursos y amonestaciones, siendo en él muy natural decir:
«Si no se puede resistir el asalto, y los franceses entran en la ciudad,
moriremos todos y después \emph{se hará lo que convenga.»}

---Pero señor---añadió D. Pablo,---los enfermos no admiten espera. Si no
se les cura\ldots{} se podrá tirar un día, dos\ldots{}

Álvarez paseó serenamente la vista por el anfiteatro, y después
volviéndose a Nomdedeu, le dijo:

---Ninguno de ellos se queja. Pronto recibiremos auxilios. La plaza no
se rendirá, Señor Nomdedeu, por falta de medicinas. ¿No discurre usted
algún medio para aliviar la suerte de los enfermos y heridos?

---¡Oh; sí, señor!---dijo el médico alentado por algunos de la comitiva
que murmuraron frases más en consonancia con los pensamientos del médico
que con los del gobernador.---Me ocurre que Gerona ha hecho ya bastante
por la religión, la patria y el rey. Ha llegado ya al límite de la
constancia, señor, y exigir más de esta pobre gente es consumar su
completa ruina.

Álvarez agitó ligeramente el bastón de mando en la mano derecha, y sin
inmutarse dijo a Nomdedeu:

\emph{---Ya\ldots{} sólo usted es aquí cobarde. Bien: cuando ya no haya
víveres, nos comeremos a usted y a los de su ralea, y después resolveré
lo que más convenga.}

Cuando acabó de hablar, callaron todos de tal modo, que se oía el
zumbido de las moscas. Nomdedeu volvió atrás la cabeza buscándome con la
vista, para disimular su turbación; y harto confuso hubo de abandonar la
comitiva. Hasta mucho después de que esta pasara, no recobró el uso de
la palabra mi buen doctor, y estaba pálido y tembloroso, señal
inequívoca de su miedo.

---Andrés---me dijo en voz baja tomándome del brazo, y llevándome en
dirección de la plaza de San Félix---ese hombre va a acabar con
nosotros. Yo soy patriota, sí señor, muy patriota; pero todo tiene su
límite natural, y eso de que lleguemos a comernos unos a otros me parece
una temeridad salvaje.

---La entereza de D. Mariano---le respondí,---nos llevará a tragarnos
mutuamente; pero por lo que a mí toca, y mientras sepa que ese hombre
está vivo, antes me comeré a mordidas mi propia carne, que hablar de
capitulación delante de él.

---Grande y sublime es su constancia---me dijo,---yo la admiro y me
congratulo de que tengamos al frente de la plaza hombre cuya memoria ha
de vivir por los siglos de los siglos. ¡Oh, si yo fuera solo en el
mundo, Andrés! Si yo no tuviera más que mi indigna persona, si no
tuviera otro cuidado que la visita al hospital y el recorrido de los
enfermos que están en la calle, yo mismo le diría a D. Mariano: «Señor,
no nos rindamos mientras haya uno que pueda vivir almorzándose a los
demás;» pero mi hija no tiene la culpa de que una nación quiera
conquistar a otra\ldots{} Sin embargo, humillemos la frente ante la
voluntad de Dios, de la cual es ejecutor en estos días ese inflexible D.
Mariano Álvarez, más valiente que Leónidas, más patriota que Horacio
Cocles, más enérgico que Scévola, más digno que Catón. Es este un hombre
que en nada estima la vida propia ni la ajena, y como no sea el honor
todo lo demás le importa poco. En las jornadas de Setiembre, cuando
Vives, el capitán de Ultonia, se disponía para una pequeña excursión al
campo enemigo, preguntó a don Mariano que a dónde se acogería en caso de
tener que retirarse. El gobernador le contestó: «Al cementerio.» ¿Qué te
parece? ¡Al cementerio! Es decir, que aquí no hay más remedio que vencer
o morir, y como vencer a los franceses es imposible porque son ciento y
la madre, saca la consecuencia. ¡Esto entusiasma, Andresillo! Se le
llena a uno la boca diciendo: ¡Viva Gerona y Fernando VII!, le parece a
uno que ya está viendo las historias que se van a escribir ensalzándonos
hasta las nubes; pero yo quisiera poder decir ¡Viva España y viva
Josefina!, o que al menos entre las ruinas humeantes de esta ciudad y
entre el montón que han de formar nuestros cuerpos despedazados, se
alzara rebosando salud mi querida hija única que nunca ha hecho mal a
España ni a Francia, ni a Europa, ni a las potencias del Norte ni del
Sur.

El doctor detúvose a examinar varios enfermos, y corrí a casa de Siseta
para llevarles lo poco que había recogido.

\hypertarget{xiv}{%
\chapter{XIV}\label{xiv}}

Casi juntamente conmigo entró Barodet, que había salido a hacer una
excursión por la plaza de las Coles, y volvía tan alegre y saltón, que
le juzgué portador de víveres para ocho días.

---¿Qué hay, Badoret?---le preguntamos Siseta y yo.

Nos contestó abriendo los puños para mostrar algunas piezas de cobre, y
cerrábalos después, bailando con frenesí en medio de la sala.

---¿De dónde traes eso? ¿Lo has cogido en alguna parte?---le preguntó su
hermana con enojo, sospechando sin duda que el chico había hecho
incursiones lamentables en la propiedad ajena.

---Me los han dado por el ratón\ldots{} Andrés, un ratón tan grande como
un burro. En cuanto llegué con él a la plaza, un viejo soltó tres reales
por él.

---¿Para comérselo?---exclamó Siseta con horror.

---Sí---repuso Badoret dándole los cuartos.---Tú no lo quisiste, pues a
venderlo.

---Mira, Andrés---me dijo Siseta,---luego que tú te fuiste, estos
condenados bajaron al patio, y por la puertecilla que está junto al
pozo, se metieron en la casa del canónigo D. Juan Ferragut, que está
abandonada como sabes. A poco volvieron con una rata tan grande como de
aquí a mañana\ldots{} ¡Qué patas! ¡Qué rabo!

---La carne de este precioso e inteligentísimo animal---dije yo dando a
Siseta lo que llevaba,---no es mala, según dicen los muchos que en
Gerona la están consumiendo. Por ahora, muchachos, remediémonos con esto
que os traigo, y Dios dará más adelante otra cosa.

Comimos, si así puede llamarse una refacción tan exageradamente sobria,
que más parecía hecha para dar entretenimiento a los dientes, que
sustancia al cuerpo. Yo me dormí sobre el suelo poco después, y cuando
desperté, Siseta con gran aflicción me dijo:

---Gasparó está malo. Ha cesado de llorar, y está como desmayado con el
cuerpo ardiente y temblando de escalofríos. ¿Tardará en volver el
Sr.~Nomdedeu?

Examiné al chico, y su aspecto me hizo temblar, porque no dudé un
momento que estuviese atacado de la fiebre a que sucumbía diariamente
parte de la población; pero procuré tranquilizar a su hermana,
asegurando que los síntomas del mal que tenía delante, no eran parecidos
a los que a todas horas se observaban en los sitios más públicos de la
ciudad. Pero Siseta, en su buen sentido, no daba crédito a mis
consuelos, comprendiendo la gravedad de su hermanito. Con la mayor
naturalidad del mundo, y olvidando en su preocupación las circunstancias
de la ciudad, me mandó que le llevase algunas medicinas, y tuve que
emplear mil rodeos y circunlocuciones para decirle que no las había. La
infeliz muchacha estaba inconsolable.

Una hora después entró D. Pablo Nomdedeu, al cual llamamos para que
asistiese al enfermo, y se prestó a ello de buen grado.

---¡Pobre Gasparó!---dijo al verle.---Ya he dicho varias veces que con
los alimentos que diariamente se consumen aquí, estos chicos no han de
llegar a viejos.

---Pero mi hermano no se morirá, señor don Pablo---afirmó Siseta
llorando.---Usted que es tan buen médico, le curará.

---Hija mía---repuso fríamente el doctor,---tiende la vista por esas
calles, y observa de qué valen los buenos médicos. Lo que respiramos en
Gerona no es aire, es una sutil e invisible materia cargada de muertes.
¡Ay! Vivimos por especial don de Dios, los que vivimos. Tenemos un
gobernador de bronce que manda resistir a estos hombres que se caen
muertos por momentos. D. Mariano Álvarez no ve en el cuerpo humano sino
una cosa con que rellenar los cementerios, y que no pudiendo servir para
batirse no sirve para nada. Él no atiende más que al inmortal espíritu,
y fijando su atención en la vida perpetua que con los miserables ojos de
la carne no podemos ver, desprecia todo lo demás. Sí, la magnitud de ese
hombre me tiene asombrado por lo mismo que es superior a mí. El
gobernador resistirá el hambre, las privaciones, las enfermedades,
mientras tenga una gota de sangre que mantenga en pie la urna de su
grande espíritu, pues su alma es el alma menos atada al cuerpo que he
conocido; y si no pudiese resistir, será capaz de comerse a sí
mismo\ldots{} Pero veamos qué se hace con ese pobre Gasparó, hija mía;
yo creo que debes ir a enterrarle a la plaza del Vino, donde se ha hecho
una gran fosa, porque si dejamos aquí su pobre cuerpo, puede corromperse
la atmósfera de esta casa más de lo que está.

---¿De modo que usted le da por muerto?---preguntó Siseta con
desesperación.

---Siseta, nuestra misión en el estado a que han llegado las cosas, sin
alimentos ni medicinas que recomendar, se reduce a evitar los horribles
efectos de la descomposición atmosférica. Si pudiéramos tener a mano
buenas tazas de caldo, un poco de vino blanco y algunos emolientes y
heméticos, creo que sería fácil tornar la salud a la robusta naturaleza
de ese niño; pero es imposible: no hay nada. ¡Felices los que se mueren!
Si no consigo salvar a mi hija, me pondré en la muralla, cuando haya
otro asalto, para morir gloriosamente\ldots{} Pobre Gasparó: ¡con cuánto
placer te cuidaría si viera en ti esperanzas de vida! Siseta, sentiría
mucho que mi hija conociera la proximidad de un moribundo. En caso de
que Gasparó llore o chille, le mandarás callar. Adiós, adiós, hijos
míos; cuidado con mis instrucciones.

Y subió. Tenía todas la apariencia de un loco.

~

Siseta destrozó un mueble, calentó agua con él y diose a aplicar al
enfermo en diversas formas una terapéutica de su invención, compuesta de
agua tibia en bebida, en cataplasmas, en friegas, en rociadas, en
parches. Como advirtiera cierta quietud en el enfermo, creyola repentina
mejoría, por efecto de sus extraordinarios específicos, y dijo con tanta
inocencia como alegría:

---Andrés, me parece que está mejor. Se ha dormido. Mi madre decía que
el agua del Oñá era la mejor medicina del mundo, y con agua se curaba
ella todos sus males. ¿Ves cómo está más tranquilo? Cuando despierte
querrá ir a jugar con sus hermanos. ¿Pero dónde están esos malditos?
¡Badoret, Manalet!\ldots{}

Siseta los llamó gritando varias veces, y los muchachos no parecían.
Estaban en la casa del canónigo.

Yo subí a ver a D. Pablo y a su hija, y encontré a esta tan abatida y
desfigurada, que cuando cerraba los ojos quedándose sin movimiento con
la cabeza hundida entre los almohadones, parecía realmente muerta. Ya
era casi de noche y Nomdedeu, sentado junto al velador, escribía su
diario.

---Andrés---me dijo el doctor,---te agradezco que vengas a hacerme
compañía. ¿No me guardas rencor por lo de esta mañana? Eres un buen
muchacho, y sabes hacerte cargo de las circunstancias. En estos casos,
no hay amigo para amigo, ni hermano para hermano. Ahora mismo, si
metieras tu mano en el plato donde va a comer mi hija, creo que te
mataría.

---¿Y la señorita Josefina---le pregunté,---cree todavía que hay fiestas
en Gerona, y que mañana irá a Castellà?

---¡Ay!, no. La ilusión duró hasta el día siguiente nada más. Su estado
moral es espantoso. Ya no puede ocultársele nada, y es inútil
representar comedias como la de la otra noche. Lo sabe todo, y no ignora
los últimos pormenores, gracias a una indiscreción de esa endiablada
señora Sumta, a quien de buena gana arrastraría por los cabellos.
Figúrate, Andrés, que una de estas noches, cuando yo estaba curando
enfermos por esas calles, la tal señora Sumta, que a más de ser curiosa
como mujer, es entrometida y novelera como un chico de diez años,
deseando dar a su entendimiento el pasto de una belicosa lectura en
armonía con sus aficiones militares, sacó de la alacena de mi despacho
este diario que estoy escribiendo, y se puso a leerlo aquí mismo delante
de mi hija. Esta sintió al instante deseos de leer también, y la muy
necia de la señora Sumta se lo permitió, añadiendo de su propia cosecha
comentarios encomiásticos de los empeños y heroicidades del sitio.
Cuando volví, mi hija había llegado a las últimas páginas, y en su
calenturienta atención y curiosidad se le iba el alma a pedazos. La
lectura la embelesaba y la mataba al mismo tiempo, y el terror y la
admiración compartíanse el dominio de su alma. ¡Ay, cuánto trabajo me
costó arrancarle de las manos el malhadado diario! La pobrecita no
durmió en toda la noche, y puesto su cerebro en erección, allí era de
ver cómo imaginaba batallas en la calle, cómo sentía el ruido de las
bombas, cómo aseguraba estarse quemando con el resplandor de los
incendios, cómo miraba los ríos de sangre que enrojecían el Ter y el
Oñá, sin que me fuera posible tranquilizarla. La infeliz corría de una
parte a otra de la habitación como una loca; y llamaba a gritos a D.
Mariano Álvarez, ensalzando la bravura y grande ánimo de nuestro
gobernador. Otras veces, dominada por el miedo, me pedía que la
escondiese en lo más profundo de los pozos para no oír el zumbido de los
cañonazos ni ver el resplandor de las llamas. Tan pronto su delicado
organismo nervioso, que es su naturaleza toda, se crispaba dándole
actividad febril, como cuando dominados por el entusiasmo nos
centuplicamos; tan pronto abatiéndose llorosa, su cuerpo caía flojo y
blando como una madeja. Precisamente la falta del sentido acústico, que
parece debía ser un descanso para su espíritu, es un verdadero tormento,
porque oye rumores que sin tener existencia real retumban en su cerebro;
y los espectros del sonido aterran su imaginación más que los de la
vista. ¡Pobrecita hija mía! Creí verla morir en una de aquellas crisis.
Era su vida como un hilo muy delgado que por intervalos se pone tirante,
tirante, amenazando romperse. Yo tenía el alma en suspenso, y
comprendiendo que contra tal estado de nada valen la ciencia ni los
cuidados, me crucé de brazos y bajé la frente esperando el fallo de
Dios. De este modo ha pasado algunos días, Andrés, y últimamente todos
los síntomas de desorden nervioso han desaparecido, para no quedar más
que el del miedo, un miedo en el último grado de lo deprimente, que la
tiene aplanada, moribunda. ¿Ves esa cara, ves esa expresión soñolienta y
abatida, esa diafanidad propia de los primeros instantes de la muerte?
¿Por ventura eso tiene apariencia de vida? No parece sino que este
simulacro de existencia permanece ante mis ojos por disposición
milagrosa del cielo para consolarme durante la ausencia real de mi
verdadera hija.

Después de un largo y triste silencio, continuó así:

---Andrés, mañana saldrá el sol; mañana habrá lo que en nuestro lenguaje
llamamos día; mañana tendremos otro hoy, es decir, nuevos apuros.
Veremos qué miga de pan me reserva Dios para el día que ha de venir.
Como quiera que sea, mi hija tendrá mañana su plato en esta mesa. Así ha
de ser, cueste lo que cueste.

Y dicho esto, siguió redactando su diario.

Cuando volví al lado de Siseta, la encontré más tranquila, engañada por
el aparente alivio del pobre niño. Su principal inquietud consistía
entonces en la ausencia de Badoret y Manalet, que a pesar de lo avanzado
de la noche, no volvían a casa. Pero de acuerdo les supusimos ocupados
en explorar la habitación vecina, y no se habló más sobre el particular.
Retireme yo a mi guardia, pesaroso de dejarla sola, y durante toda la
noche estuve mortificado por cavilaciones y presentimientos que no me
dejaron dormir.

\hypertarget{xv}{%
\chapter{XV}\label{xv}}

Al día siguiente no ocurrió novedad particular. Gasparó seguía lo mismo.
Badoret y su hermano aparecieron tras larga ausencia, llenos de
rasguños, contusiones, magulladuras y mordidas; pero muy contentos con
los cuartos que recientemente les había proporcionado su industria. A
pesar de este refuerzo pecuniario, aquel día fue el abastecimiento de la
casa más penoso y difícil que otro alguno, y Siseta, desmejorándose por
grados, perdía robustez y salud de hora en hora. Como entonces
ocurrieron acontecimientos terribles en nuestra casa, no puedo pasarlos
en silencio. Después de un breve y violento sueño, despertome al rayar
el día el golpear de un pie, que no por ser de amigo carecía de dureza,
y cuando abrí los ojos, encaré con el tambor del regimiento, Felipe
Muro, que me dijo:

---Ha caído una bomba en la casa del canónigo Ferragut, calle de
Cort-Real, y el tejado ha ido a buscar refugio dentro de los cimientos.
Yo lo he visto, Andrés. Tu amigo el médico, D. Pablo Nomdedeu, salió a
la calle gritando y bufando en cuanto vio arder las barbas del vecino.
Felizmente la casa no ardió, y hasta hoy no tiene más avería que haber
sido aplastada como un buñuelo. ¿No vas allá?

De buena gana habría corrido al lugar de la catástrofe; pero la
ordenanza me ataba a la muralla de Alemanes durante algunas horas, y
esperé con la más cruel ansiedad. Cuando me encontré libre y pude
trasladarme a la calle de Cort-Real, vi con alegría que mi casa estaba
intacta, aunque amenazada de algún deterioro por la repentina falta de
apoyo de la contigua, cuya fachada yacía casi totalmente en el suelo,
viéndose desde la calle el interior de las habitaciones con parte de los
muebles en la misma situación en que los dejó el dueño al abandonar su
domicilio. Mentalmente di gracias a Dios por haber librado de la
desgracia la casa de los míos, y corrí al lado de Siseta, a quien
encontré en el taller y en el mismo sitio donde la había dejado la noche
anterior, junto al lecho de su hermano. La consternación de la pobre
muchacha era tal, que no acerté a tranquilizarla con inútiles consuelos.

---Siseta---le dije,---es preciso resignarse a lo que quiere Dios. ¿Y tu
hermano?

No me contestó ni había para qué, porque su hermano se moría. Ella misma
hallábase en tan lastimosa situación física y moral, que sólo por un
enérgico propósito de su fuerte espíritu, se mantenía vigilante y atenta
a la agonía del pobre Gasparó. Sin el dolor, Siseta habría caído al
suelo, abatida por el insomnio y la inanición; pero ella despreciaba su
propia existencia, y para atenderla era preciso que desapareciese la de
los demás.

---¿El Sr.~Nomdedeu no ha asistido a tu hermano?---le pregunté.

---No---repuso.---El Sr.~D. Pablo dice que aquí nada falta sino echarle
tierra encima.

---¿Y es posible que no te haya proporcionado algunas medicinas? Si él
quisiera, podría hacerlo.

---Dice que no hay medicinas.

---Dime: ¿Gasparó ha tomado algún alimento?

---Nada. Con los cuartos que trajeron ayer los chicos, se compró un
pedacito muy chico de cecina; y lo puse en las parrillas, y esta mañana
vino D. Pablo, se me arrodilló delante llorando a moco y baba, y como a
pesar de esto me resistiera a dárselo, amenazome con matarme y se lo
llevó.

---¿Tú tampoco has tomado nada?\ldots{} ¡Oh! Es preciso que yo le siente
la mano a ese ladronzuelo de D. Pablo. ¿Tenemos nosotros obligación de
mantenerle a su hija? ¿Y tus hermanos?

---No sé dónde están---repuso Siseta con profundo terror.---Desde anoche
no han vuelto a casa.

---Pero, Siseta---exclamé con angustia,---no irían a la casa del
canónigo. ¿Sabes que se ha venido al suelo?

---No sé si irían allá\ldots{} Esta mañana sentí un gran ruido. Creí que
era esta casa la que se venía al suelo; y abrazando a mi hermano cerré
los ojos y me encomendé a Dios. Pero luego que cesó el ruido, miré al
techo y lo vi en el mismo sitio. La gente gritaba en la calle, y era
difícil respirar a causa del polvo. No, Dios mío, no es posible que mis
hermanos estuvieran hasta hoy dentro de esa casa. Yo creo que habrán ido
al mercado a vender lo que hayan cogido.

Cada palabra pronunciada era un esfuerzo angustioso de la decaída
naturaleza de Siseta. Cubría su frente helado sudor, y sentada en el
suelo apoyaba sus brazos en la estera para sostenerse. Pálida como la
misma muerte, y con los ojos apagados y hundidos, daba pena de ver cómo
se agostaba aquella planta, sin poder echarle un poco de agua.

De repente bajó metiendo mucho ruido el Sr.~Nomdedeu, que al verme, me
dijo:

---¡Oh, Andresillo! ¡Cuánto me alegro de que estés aquí! Supongo que
traerás algo. Tú eres generoso y no te olvidas de los buenos amigos.

---Nada traigo, señor doctor; y si trajera, no sería para usted. Cada
cual se las componga como pueda.

---¡Qué bromas gastas! Supongo que traerás siquiera un poco de trigo. Y
tú, Siseta, ¿tienes algo para mí? ¿Tus hermanos no han traído nada? ¡Oh,
amigos de mi alma! ¿No hay nada para este pobre infeliz que ve morir a
su hija? Andrés, Siseta---añadió juntando las manos y poniéndose de
rodillas delante de nosotros,---haced la caridad, por amor de Dios, que
todo lo que tuviereis de menos en la tierra lo tendréis de más en el
cielo. Ya sabéis que \emph{aquí dan uno por ciento y allá dan ciento por
uno}. Andrés, Siseta, queridísimos amigos míos, vosotros que nadáis en
la abundancia, socorred a este mendigo. Nada me queda ya: he vendido
todos mis libros, y con las plantas de mi magnífico herbario, que he
reunido durante veinte años, he hecho un cocimiento para dárselo a ella.
Sólo me restan las plantas malignas o venenosas, y la incomparable
colección de \emph{polipodiums}, que os puedo vender\ldots{} ¿De veras
que no tenéis nada? No puede ser. Ustedes esconden lo que tienen;
ustedes me engañan, y esto no lo puedo consentir; no, no lo consentiré.

De esta manera, Nomdedeu pasaba de la aflicción más amarga a una cólera
hostil y atrabiliaria, que a Siseta y a mí nos infundió bastante recelo.

---Sr.~Nomdedeu---dije resuelto a alejar de nosotros huésped tan
importuno---no tenemos nada. Ya ve usted. El pobre Gasparó se muere, y
no podemos darle un buche de agua con vino. Déjenos usted en paz o
tendremos un disgusto.

---Eso se verá. Yo no me voy de aquí sin algo. Ustedes esconden lo que
van comprando con los cuartos que traen los chicos. Mi hija no puede
seguir así muchas horas, Andrés. Que se rinda Gerona, sí, señor, que se
rinda, y que se vaya al infierno con cien mil pares de demonios el
Sr.~D. Mariano Álvarez, que ha dicho esta mañana: «Cuando la ciudad
principie a desfallecer, se hará lo que convenga.» No sé a qué espera.
Aún no cree que la ciudad está bastante desfallecida. ¡Oh! Lo que
debiera hacer el gobernador es castigar a los pillos que acaparan las
vituallas, privando a sus semejantes de lo más preciso, y ustedes son
estos, sí, señor. Ustedes tienen esas arcas llenas de comestibles, y lo
menos hay ahí diez onzas de cecina y un par de docenas de garbanzos.
Esto es un robo, un robo manifiesto. Siseta, Andrés, amigos míos: ya he
vendido todas las estampas y cuadros de mi casa. ¿Queréis el perrito que
bordó en cañamazo mi difunta esposa cuando estaba en la escuela? ¿Lo
queréis? Pues os lo daré, aunque es una prenda que he estimado como un
tesoro, y de la cual hice propósito de no deshacerme nunca. Os doy el
perrito si me dais lo que está guardado en el arca.

Abrimos el arca, mostrándole su horrenda vaciedad; pero ni aun así se
dio por vencido. Estaba frenético, con apariencias de trastorno
semejante a la embriaguez o al delirio de los calenturientos, y al
hablar su lengua sin fuerza chasqueaba las palabras, entonándolas a
medias, como un badajo roto que no acierta a herir de lleno la campana.
Temblaba todo él, y el llanto y la risa, la pena, la ira, la resignación
o la amenaza se expresaban sucesivamente en las rápidas modificaciones
de su fisonomía agitada y movible como la de un cómico.

Cuando me levanté para obligarle a salir, amenazome con los puños, y en
un tono que no es definible, pues lo mismo podía ser dolorido llanto que
honda rabia, nos dijo:

---Miserables, ladrones de lo ajeno. Haré lo que dice el gobernador. Sí,
Andrés, Siseta. Mi hija no se morirá; mi pobre hija no se morirá, porque
cuando no haya otra cosa nos comeremos a ustedes y después se resolverá
lo que más convenga.

Cuando se retiró, Siseta me dijo:

---Andrés, yo no sé si viviré mucho más que Gasparó. Haz el favor de
buscar a mis hermanos. Si Dios ha determinado que en este día se acabe
todo, se acabará. Somos buenos cristianos y moriremos en Dios.

\hypertarget{xvi}{%
\chapter{XVI}\label{xvi}}

Dejando para más tarde la exploración al mercado, marché a la abandonada
vivienda de D. Juan Ferragut, canónigo de la catedral, que desde los
primeros días del sitio huyó de Gerona buscando lugar más seguro. Aunque
este veterano de las milicias docentes de Cristo no figura en mi
relación, debo indicar que era el primer anticuario de toda la alta
Cataluña; hombre eruditísimo e incansable en esto de reunir monedas,
escarbar ruinas, descifrar epígrafes y husmear todos los rastros de
pisadas romanas en nuestro suelo. Su colección numismática era célebre
en todo el país, y además poseía inapreciable tesoro en vasos, lámparas,
arneses y libros raros; pero el grande amor que tenía a estos objetos no
fue parte a detenerle en su huida, abandonando la historia romana y
carlovingia por poner en seguro la más que ninguna inestimable
antigualla de la propia vida. Luego una bomba arregló el museo a su
manera.

Entrábase en la desierta casa por una pequeña puerta que comunicaba
ambos patios, y que los vecinos solían tener abierta para venir a tomar
agua en el del nuestro. Cuando penetré en el patio, hallé que una gran
parte de este se había trocado en recinto cubierto, formado por la
acumulación de vigas y tabiques atascados en un ángulo antes de llegar
al piso. Aquel improvisado techo no necesitaba sino ligero impulso, una
voz fuerte, una trepidación insensible para caer al suelo. Adelantando
cuidadosamente llegué a la caja de la escalera, abierta a la luz y al
aire por el hundimiento de las salas de la fachada y de una parte del
techo por donde penetró la bomba. Cubrían el suelo muebles confundidos
con trozos de pared, vidrios y mil desiguales fragmentos de
preciosidades artísticas, materia caótica de la historia, que ningún
sabio podía ya reunir ni ordenar. La escalera había perdido uno de sus
tramos, y para subir era preciso trepar, saltando abruptas alturas.
Desde abajo veíase el interior de una alcoba que debía ser la del señor
canónigo, la cual pieza con un testero de menos, y conservando parte de
sus muebles, se asemejaba a los aposentos de juguete para los niños,
cuando se les quita la tapa o pared lateral, cuya ausencia permite ver
el lindo interior. Si algunos cuadros, cofres y roperos manteníanse
arriba en los mismos puestos que desde luengos años ocupaban, en cambio
la cama del canónigo yacía en lo hondo de la escalera en una postura que
podemos llamar boca abajo. Los gruesos pilares de aquel mueble, que no
era otra cosa que un mediano monte de roble, aparecían por diversos
puntos tronchados, esparciendo sus agudas astillas, y las colgaduras en
desorden dejaban ver entre sus pliegues los brazos de marfil de un Santo
Cristo, y las secas ramas de unas disciplinas. De entre los despojos de
la piedra, y en la oscuridad de los rincones y honduras que formaban, vi
surgir el brillo de dos discos luminosos, como dos puntos, como dos ojos
que me miraban. A pesar de que sentí súbito temor, bajeme a recoger
aquellas luces. Eran los espejuelos del buen Ferragut.

En la imposibilidad de subir, di voces al pie de la escalera, por ver si
desde aquellas solitarias cavidades me respondía alguno de los muchachos
a quienes buscaba. Grité con toda la fuerza de mis pulmones: ¡Badoret,
Manalet!, pero nadie me respondía. Recorrí todo lo bajo, explorando lo
más escondido y lo más peligroso de los escombros, y sólo encontré la
barretina de uno de los chicos; pero esto no era suficiente razón para
suponer que ellos existiesen bajo las ruinas. Por último, regresando al
hueco oí un agudo silbido, que resonaba en lo más alto del tejado.
Aguardé un rato, y en breve oyéronse de nuevo los mismos agudos sones, y
apareció una figura, que desde arriba con evidente peligro se inclinaba
para mirar hacia el fondo. Era Badoret.

El muchacho, poniéndose ambas manos en la boca, gritó: ¡Manalet, alerta!

Y luego forzando la voz, añadió:---¡Allá van! ¡Allá va Napoleón, con
toda la guardia imperial, y la tropa menuda!

Dicho esto desapareció, y yo me quedé absorto esperando ver a Napoleón
con toda la guardia imperial. En efecto; por la rota escalera descendía
a escape tendido un numeroso ejército cuyos precipitados pasos metían
bastante ruido. Saltaban de peldaño en peldaño por entre los pedazos de
vigas, y con ligereza suma franqueaban los claros de la escalera,
gruñendo, chillando, escarbando, describiendo piruetas, curvas,
círculos, y empujándose, confundiéndose y precipitándose unos sobre
otros.

Delante iba el mayor de todos que era grandísimo, como ser de
privilegiada magnitud y belleza entre los de su clase, y seguíanle otros
de menos talla y muchos pequeños, entre los cuales había jovenzuelos,
juguetones y muchos graciosos niños. No eran docenas, sino cientos,
miles, ¡qué sé yo!, un verdadero ejército, una nación entera, masa
imponente que en otras circunstancias me habría hecho retroceder con
espanto. Las oscilaciones de sus largos rabos negros eran tales, que
parecían culebras corriendo en medio de ellos, y sus brillantes ojos de
azabache expresaban el azoramiento y la ansiedad de retirada tan
vergonzosa. Venían hostigados, y la inmunda caterva pasó junto a mí y en
derredor mío con rapidez inapreciable escurriéndose por entre los
escombros hacia el patio. Seguíalos yo con la vista, y por una oscura
puertecilla que vi en la pared, sumergiéronse todos en un segundo, como
chorro que cae al abismo.

Yo no había visto aquella puerta abierta en un ángulo y que ocultaban
dos toneles puestos en el patio. Acerqueme a ella y desde la boca grité:

---Manalet, ¿estás ahí?

Al principio no sentí rumor alguno, sino un lejano y vago son de
hojarasca que me parecía producido por las pisadas de la guardia
imperial sobre montones de yerba seca. Pero al poco rato creí sentir
como voces y lamentos que al principio parecieron aprensión mía o eco de
mis propios gritos; pero oyendo que se repetían más acentuados cada vez,
resolví aventurarme en lo interior del aposento oscurísimo que ante mí
se abría.

Nada pude ver en los primeros momentos; mas a poco de estar allí
distinguí las formas robustas de las tinajas y toneles, cajones rotos,
arreos de caballerías y de carros, y mil objetos de indefinible
configuración, que iban saliendo poco a poco de la oscuridad a medida
que mis ojos se acostumbraban a ella.

El sitio era poco agradable, y no sé por qué las barrigas de aquellas
tinajas me ofrecían un aspecto temeroso, causa para mí de invencible
horror. Yo reconocí en aquellas formas extravagantes las de ciertos
monstruos que venían a amedrentarme en mis sueños de enfermo, y no les
faltaba más que cuatro patas resbaladizas, húmedas, cartilaginosas, para
arrojarse sobre mí. A los pocos pasos produje el mismo ruido de
hojarasca que antes había sentido, y observé que pisaba grandes capas de
yerba seca, depositada allí sin duda para bestias que no habían de
comerla.

De pronto, señores, sentí que las hojas sonaban pisadas por mil patitas,
y los cabellos se me erizaron de espanto. ¿Por qué, si allí no había
leones, ni tigres, ni culebras, ni ningún animal verdaderamente fuerte y
temible? Lo cierto es que tuve miedo, un miedo inmenso que heló la
sangre en mis venas, dejándome atónito y paralizado. Quise huir y
hundime en la yerba seca. Revolví los ojos en torno mío, y aumentó mi
terror al ver que se disponía para acometerme por distintos lados con la
rabia de mil bestias feroces todo el ejército imperial.

En un instante me sentí mordido y rasguñado en los tobillos, en las
piernas, en los muslos, en las manos, en los hombros, en el pecho.
¡Infame canalla! Sus ojuelos negros y relucientes como pequeñas cuentas,
me miraban gozándose en la perplejidad de la víctima, y sus hocicos
puntiagudos se lanzaban con voracidad sobre mí. Grité, pateé, manoteé;
pero la flojedad del suelo en que me sostenía imposibilitaba mi defensa,
y con esfuerzos extraordinarios pugnaba por echarme fuera de aquel mar
de hoja seca en el cual, si era difícil el correr, más difícil era el
nadar. La turba insolente, aguijoneada por el hambre, se atrevía a
atacarme. ¿Qué puede uno solo de aquellos miserables animales contra el
hombre? Nada; pero ¿qué puede el hombre contra millares de ellos, cuando
la necesidad les obliga a asociarse para combatir al rey de la creación?
Hallándome sin defensa, exclamé con angustia: ¡Badoret, Manalet, venid
en mi auxilio! ¡Socorro!

Por último, conseguí poner el pie en tierra firme, y sacudiendo
manotadas a diestra y siniestra, logré aminorar el vigor del ataque.
Corrí de un lado para otro, y me siguieron; subime a un gran tonel, y
veloces como el rayo subieron ellos también. Su estrategia era
admirable; adivinaban mis movimientos antes de que los realizase, y como
saltara de un punto a otro, me tomaban la delantera para recibirme en la
nueva posición. Animábanse en el combate por un himno de gruñidos que a
mí me daba escalofrío, y parecía que rechinaban en acordada música
militar sus dientes, demostrando gran rabia y despecho todos aquellos
que no podían hacerme presa.

¡Terrible animal! ¡Qué admirablemente le ha dotado la Providencia para
que se busque la vida a despecho del hombre, para que se defienda contra
las agresiones de fuerza superior, para que venza obstáculos naturales,
para que haga suyas las más laboriosas conquistas humanas; para que
mantenga su inmensa prole en lo profundo de la tierra y al aire libre,
en los despoblados lo mismo que en las ciudades! La Providencia le ha
hecho carnívoro para que encuentre alimento en todas partes; le ha hecho
un roedor para que devore a pedazos lo que no puede llevarse entero; le
ha dado ligereza para que huya; blandura para que no se sientan sus
alevosos pasos; finísimo oído para que conozca los peligros; vista
penetrante para que atisbe las máquinas preparadas en su daño, y agudo
instinto para que con hábiles maniobras burle vigilancias exquisitas y
persecuciones injustas. Además posee infinitos recursos y como bestia
cosmopolita, que igualmente se adapta a la civilización y al salvajismo,
posee vastos conocimientos de diversos ramos, de modo que es ingeniero,
y sabe abrirse paso por entre paredes y tabiques para explorar nuevos
mundos; es arquitecto habilísimo, y se labra grandiosas residencias en
los sitios más inaccesibles, en los huecos de las vigas y en los vanos
de los tapiales; es gran navegante, y sabe recorrer a nado largas
distancias de agua, cuando su espíritu aventurero le obliga a atravesar
lagunas y ríos; se aposenta en las cuadernas de los buques, dispuesto a
comerse el cargamento si le dejan, y a echarse al agua en la bahía para
tomar tierra si le persiguen; es insigne mecánico, y posee el arte de
trasportar objetos frágiles y delicados, secretos de que el hombre no es
ni puede ser dueño; es geógrafo tan consumado, que no hay tierra que no
explore, ni región donde no haya puesto su ligera planta, ni fruto que
no haya probado, ni artículo comercial en que no haya impreso el sello
de sus diez y seis dientes; es geólogo insigne y audaz minero, pues si
advierte que no disfruta de grandes simpatías a flor de tierra, se mete
allí donde jamás respiró pulmón nacido, y construye bóvedas admirables
por donde entra y sale orgullosamente, comunicando casas y edificios, y
huertas y fincas, con lo cual abre ricas vías al comercio y destruye
rutinarias vallas; y por último, es gran guerrero, porque además de que
posee mil habilidades para defenderse de sus enemigos naturales, cuando
se encuentra acosado por el hambre en días muy calamitosos, reúne y
organiza poderosos ejércitos, ataca al hombre, y al fin, si no halla
medio de salir del paso, estos ejércitos se arman unos contra otros,
embistiéndose con tanto coraje como táctica, hasta que al fin el
vencedor vive a costa del vencido.

Poseyendo un gran sentido civilizador, se acomoda al carácter de las
comarcas y regiones que escoge para desarrollar su genio activo, y come
siempre de lo que hay. Eso sí, no respeta ni sabe respetar nada: en el
tocador de la dama elegante se come los perfumes; y en casa del
boticario las medicinas. En la iglesia hace mil condimentos con las
reliquias de los santos, y en los teatros se apropia los coturnos de
Agamenón y la loriga de D. Pedro el Cruel. Artista a veces, si el
destino le lleva a los museos, se almuerza a Murillo y cena con algo de
Rafael, y cuando acierta a penetrar en casa de los anticuarios o de los
eruditos, se convierte en uno de estos por la influencia de la
localidad, es decir, que se traga los libros.

Todas estas eminentes cualidades las desplegó contra mí la inmensa
falange. Aquellos padres que por dar de comer a sus hijos; aquellos
amantes esposos que por librar de la muerte a sus mujeres, no vacilaban
en mirar frente a frente a un ser superior, tenían toda la perversidad
que dan las supremas exigencias de la vida. Pero era realmente una
vergüenza para mí el rendir mi superioridad de fuerza y de inteligencia
ante aquella chusma de los bodegones, que procedentes de distintos
puntos de la ciudad, por caminos sólo sabidos de ella sola, se había
reunido en tal sitio. Así es, que reponiéndome al cabo de algún tiempo
de mi primitivo susto, arrebaté un palo que al alcance de la mano vi, y
haciendo pie firme sobre el tonel, comencé a descargar golpes a todos
lados, increpando a mis enemigos con todos los vocablos insultantes,
groseros y desvergonzados de la lengua española. Si no obtuve desde
luego por este medio ventajas positivas, conseguí al menos amedrentar a
los pequeños, que eran los más insolentes, y sólo los grandes
continuaron empeñados en roerme. Pero los grandes me ofrecían un blanco
más seguro, y he aquí que después de un rato de combate peligroso,
incesante, en que multiplicaba los movimientos de mis brazos y piernas
con rapidez más propia de un bailarín que de un guerrero, comencé a
adquirir alguna ventaja. La ventaja en las batallas, una vez que se
manifiesta, va creciendo en proporción geométrica, determinada por los
temores y recelos del que flaquea, por el orgullo y reanimación del que
gana terreno, y esto me pasó a mí, que al fin, señores míos, a fuerza de
trabajo y de angustia pude adquirir el convencimiento de que no sería
devorado.

Cuando me vi libre de la guardia imperial (pues no renuncio a darle este
nombre) me hallaba tan cansado que di con mi cuerpo en tierra.

---Si me atacan otra vez---dije para mí,---acabarán conmigo.

\hypertarget{xvii}{%
\chapter{XVII}\label{xvii}}

Pero en la desbandada del numeroso ejército, no abandonaron el campo
todos los combatientes, no: allí enfrente de mí, arrastrando por el
suelo su panza formidable estaba uno, el más grande, el más fuerte ¿por
qué no decirlo?, el más hermoso de todos, fijando en mí el chispeante
rayo de sus negras pupilas, con la oreja atenta, el hocico husmeante,
las garras preparadas, el pelo erizado, y extendida la resbaladiza cola
escamosa y pardusca.

---¡Ah, eres tú, Napoleón!---exclamé en voz alta como si el terrible
animal entendiese mis palabras.---Ya te reconozco. Eres el mayor y el
más fuerte de todos, eres el que iba delante cuando bajabais por la
escalera. Infame, tu corpulencia y tus años te dan sobre los de tu ralea
la superioridad que demuestras; pero eres un egoísta que por tu propio
provecho reúnes a tus hermanos para que te ayuden en tus carnicerías.
Miserable, ellos están flacos y tú estás gordo. Lo que ellos husmean tú
te lo comes, y a falta de otro manjar, devorarás a los pequeñuelos que
te siguen, orgullosos de tener un general tan bravo. Miserable, ¿por qué
me miras? ¿Crees que te temo? ¿Crees que temo a una vil alimaña como tú?
El hombre, que a todos los animales domina, que de todos se vale, que se
alimenta con los más nobles ¿temblará ante un indigno roedor como tú?

Corrí hacia él, pero desapareció agachándose para esconderse entre unos
maderos. Despejé aquel sitio; pero él se escurrió ligeramente y le perdí
de vista. Esta exploración me llevó muy adelante en la larga bodega, y
en la crujía inmediata vi que se desparramaban a un lado y otro,
corriendo por encima de las tinajas y por las mil sinuosidades de la
pared, mis enemigos de un momento antes. Todos me miraban pasar y
corrían de un lado para otro. No me quedaba duda de que eran algunos
miles. A cada instante me parecía mayor su número.

En un rincón de la última crujía había un pequeño tonel en pie tapado
con una baldosa, con aspecto muy parecido al de una colmena. Cierto vago
rumor que de allí salía, me hizo fijar la atención, y entonces vi que
por la posición del tonel, la boca estaba de frente. Pero lo que me
causó sorpresa no fue esto, sino que por dicha boca apareció un dedo y
después dos. En el mismo momento una voz al mismo tiempo infantil y
cavernosa, como voz de niño que sale por el agujero de un tonel, llegó a
mis oídos diciendo:

---Andrés, ya te veo. Aquí estoy. Soy yo, Manalet. ¿Se ha ido esa
canalla? Me he encerrado aquí para que no me comieran, y he tapado mi
casa con una baldosa. ¿Tienes algo de comer?

---No; ya puedes salir. No tengas miedo---le respondí.

---Están ahí todavía. Siento sus patadas. Son cientos de miles. Ayer no
había tantos; pero Napoleón ha ido esta mañana y ha vuelto con no sé
cuántos miles más. Toma este eslabón y esta yesca, Andrés. Prende fuego
en un manojo de yerba, teniendo cuidado de que no se encienda todo y
verás cómo echan a correr.

Diome por el agujero el pedernal, eslabón y pajuela, y al punto hice
fuego. Cuando el resplandor de la llama iluminó las oscuras bóvedas y
muros, todos los caballeros corrieron despavoridos, y bien pronto no
quedó uno. Ignoro el lugar de su repentina retirada.

---Se han ido---dije.---Ya puedes salir.

Entonces vi que se levantaba la baldosa que tapaba el tonel y
aparecieron los cuatro picos negros de un bonete de clérigo. Debajo de
este tocado se sonreía con expresión de triunfo la cara de Manalet.

---Si tú no vienes---dijo,---¿qué hubiera sido de mí?

---¡Bonito sombrero!---exclamé riendo.

---Perdí la barretina, y como tenía frío en la cabeza\ldots{}

---¿Y Badoret?

---Está en el tejado. Oye lo que nos pasó. Ayer cazamos algunos; pero no
pudimos coger a Napoleón; que así le llamamos por ser el más grande y el
más malo de todos. Cuando anocheció, anduvimos dando vueltas por la casa
y nos encontramos una cama; ¡qué cama, Andresillo! Era la del canónigo.
Como valía más que la nuestra, nos acostamos en ella; pero no pudimos
dormir, porque al poco rato sentimos un rum de dientes y uñas\ldots{}
Eran esos pillos que se estaban cenando la biblioteca. Nos levantamos,
Andrés, y les apedreamos con los libros y con los muchos cacharros y
figuritas de barro que el canónigo tiene allí. ¿Pues creerás que no
pudimos coger ninguno vivo? Perseguidos por nosotros, se fueron en
bandada al tejado, luego bajaron al patio, volvieron, y nosotros siempre
tras ellos sin poderlos pescar. Pero me dijo Badoret: «Yo me voy al
tejado, y les hostigaré para que bajen. Ponte tú a la entrada de la
bodega, detrás de la puerta, y conforme vayan entrando, les vas
descargando palos, y alguno ha de caer.» Así lo hicimos. Yo bajé aquí, y
desde arriba Badoret me decía: «Alerta, Manalet. ¡Allá van!» ¿Querrás
creer que estando yo en esa puerta entraron todos en batallón con tanta
fuerza que me caí al suelo? Cuando me levanté encendí la luz y todos se
marcharon; pero luego volvieron y entre todos casi me comen. ¡Ay,
Andrés, qué miedo! Uno me roía por aquí, otro por allá, y yo empecé a
llorar, porque ya creía no volver a ver más a Siseta, a Gasparó, a ti ni
al Sr.~Nomdedeu. Pero, amigo, oye lo que hice para escapar: le recé a
San Narciso y a la Virgen unos ocho padrenuestros lo menos, y cátate
aquí que no había de decir más líbranos del mal amén, cuando, chico,
suenan unos truenos, unos cañonazos, unos estampidos tan terribles que
aquello parecía el fin del mundo. ¿Qué crees que era? Pues nada más sino
que un gigante empezó a dar patadas en la casa, encimita de aquí, y
desde esta misma bodega sentí caer las paredes. Allí habías de ver cómo
corrían estos bichos, llenos de miedo por los golpes que dio el gigante
mandado por la Virgen y San Narciso para salvarme. Me parece que le
estoy oyendo.

---Pues qué, ¿habló también?

---Sí, hombre. Pues no había de hablar. Después de dar muchas patadas
dijo con un vozarrón muy fuerte: «¡Canallas, dejad a Manalet!» Pues
verás. Después de esto quise salir, pero no encontré la puerta. Me volví
loco dando vueltas para arriba y para abajo, y otra vez recé a San
Narciso y a la Virgen para que me sacaran. Nada, no me querían sacar.
Luego volvió Napoleón, y con él muchos, muchísimos más, porque has de
saber que por el agujero que está debajo de aquella pipa se pasan de
esta casa al almacén de la calle de la Argentería, y también van al río,
y a las casas de la plaza de las Coles. Como ahora no encuentran qué
comer en ninguna parte, andan de aquí para allí y entran y salen. Pues,
hijito, la volvieron a emprender conmigo, y la segunda vez no me valió
rezar hasta diez y ocho o diez y nueve padrenuestros. Lo que hice fue
encender luz, y entonces me dejaron en paz; pero tenía tanto miedo que
me metí en el tonel donde me encontraste y lo tapé con la baldosa para
estar más seguro. Yo decía: «¿Pero tendré que estar aquí un par de años,
San Narcisito de mi alma?» Y me acordaba de Siseta y de Gasparó. ¡Ay,
Andrés, si no vienes tú, allí me quedo!

---Pues vámonos fuera---le dije tomándole por la mano,---y busquemos a
Badoret para salir de esta casa. Veo que los dos sois unos cobardes, que
os habéis dejado acoquinar por esos animalitos. ¿Habéis llevado algo al
mercado?

---¡Qué habíamos de llevar! Espérate y verás. Hemos de coger vivos un
par de docenas, y si tú nos ayudas\ldots{} Andresillo, Napoleón vale lo
menos nueve reales. Si le cogiéramos\ldots{}

Salimos fuera y Manalet se sorprendió de ver los destrozos causados en
la casa por la explosión del proyectil.

---Mira los desperfectos hechos por el gigante que vino a salvarte,
Manalet. Ahora tratemos de subir en busca de tu hermano.

---En el otro patio hay una escalera chica por donde se puede
subir---dijo.---¡Cómo está la casa! Bien decía yo que el gigante, por
querer meter mucho ruido, la destrozó toda.

Subimos, y en ninguna de las habitaciones del piso principal vimos al
buen Badoret. Le llamábamos, pero ninguna voz nos respondía. Por último,
le hallamos dormido sobre una cama colocada en uno de los últimos
aposentos del desván. Despertámosle y nos llevó a la biblioteca donde,
según dijo, tenía un repuesto de víveres que había encontrado en la
casa.

---Sí, señor D. Andrés---dijo sacando gravemente una llave del bolsillo
de sus andrajosos calzones.---Aquí tengo una buena cosa.

Y abrió la gaveta de una gran cómoda antigua chapeada de marfil y
madreperla. Lo primero que vi fue un gran número de antiguas monedas de
cobre y plata, todas romanas, a juzgar por lo que había oído contar de
las colecciones del canónigo Ferragut. Badoret apartó a un lado varios
objetos, y descubrió un niño Jesús de esa pasta de alfeñique que tan
bien han hecho siempre las monjas.

---Este es un regalito que hicieron las monjas al señor canónigo---dije
tomándolo.---Se lo llevaremos a Siseta. En casos de hambre, es lícito
comerse lo ajeno. Muchachos, cuidado con coger una sola de esas monedas.

Al Niño Jesús le faltaba una pierna devorada por Badoret, y no pude
evitar que Manalet se comiese la otra.

---¿Tienes algo más?---pregunté.

---Sí---dijo Badoret.---Si el Sr.~Andrés quiere unas lonjitas de
manuscrito de ochocientos años y una copa de tinta superior, se lo puedo
servir.

Por el suelo yacían arrojados en desorden y medio roídos por los
ratones, los preciosos manuscritos y los incunables, reunidos en tantos
años por el celo y la paciencia del ilustre clérigo; y con un plano a
pluma de la vía romana ampurdanesa, Badoret se había hecho un sombrero
de tres picos.

---Aquí tengo un pincho que voy a llevar esta tarde a la muralla para
ver qué dicen de él los franceses---dijo el mismo señalando una
partesana del renacimiento, cuyo rico damasquino causaría admiración al
menos inteligente.---Por ese agujero que está en el rincón, salieron
varios generales que venían de la otra casa, y para cortarles la
retirada lo tapé con la cabeza de aquella estatua de mármol que está
debajo del sillón.

En efecto, una cabeza de ángel tapaba un agujero que se abría por el
desconche de la mampostería en el zócalo de la pieza. Estaba ajustado y
atacado con papeles y trozos de vitela, entre cuyos pliegues se advertía
el hermoso colorido y el oro de las letras pintadas por los benedictinos
de la Edad Media.

---Habéis destrozado todas las maravillas que aquí tenía el
Sr.~Ferragut---dije con enfado.---En cambio de tanta pérdida, nada
habéis podido llevar hoy al mercado.

---Ya llevaremos, amigo Andrés---me contestó Badoret.---¿Cómo está mi
hermana? ¿Cómo está mi señor hermano D. Gasparó? No salgo de aquí sin
llevarles una buena pieza. La cabeza del niño Jesús será para el
chiquito, el cuerpo para Siseta, un brazo para la señorita Josefina, y
otro para el Sr.~Nomdedeu. Veremos si se coge a Napoleón. Anoche vino
aquí y quiso llevarse un pedazo de vela de cera. Si no estoy pronto a
coger el violín en que tocaba el señor canónigo y a estampárselo encima,
carga con ella.

En el suelo yacía hecho astillas el Estradivarius del buen Ferragut;
pero Manalet le recogió con intento, según dijo, de hacer un barco con
él.

---Andrés---dijo Badoret.---Napoleón es malo y traidor. No se deja
coger, y sabe más que todos nosotros. Cuando viene con su gente, él se
pone delante y les echa cada arenga\ldots{} Cuando encuentran algo, él
se lo come y da hocicadas a los demás. Aunque le tires encima palos,
cacharros, estatuas, cuadros, monedas, libros, violines, bonetes, mapas
y cuanto hay aquí no consigues matarle ni herirle. Te diré por qué. Tú
crees que Napoleón es una rata. Aviado estás. No es sino el demonio, el
demonio mismo. O si no, escucha. Anoche después que bajó Manalet, me
tendí en la cama del canónigo, que es más blanda que la mía, y desde que
cerré los ojos sentí que me roían un dedo. Sacudí la mano y aquello
pasó. Pero luego empezaron a roerme otro dedo. ¡Ay, chico, qué miedo!
Volviéndome del otro lado, me puse panza arriba. Entonces el condenado
animal se me subió encima del pecho. Chico, cada pata pesaba tanto como
la torre de San Félix; ya me iba aplastando, aplastando, y no podía
respirar. Ya tenía el pecho como el canto de un papel\ldots{} Aunque me
daba muchísimo miedo, tenía muchísima gana de verlo, y dije: «¿abro los
ojos o no los abro?» A veces decía: «los abro,» y a veces decía: «pues
no los abro.» Por fin, amigo, dije: «pues quiero verlo,» y lo vi. ¡Jesús
me valga! Lo tenía encima, echado sobre los cuartos traseros, y con las
patas delanteras tiesas. Me miraba y los ojos no eran sino como dos
lunas muy grandes. En la punta de cada pelo negro tenía una chispa de
fuego, y los bigotes eran tan grandes, tan grandísimos como de
aquí\ldots{} como de aquí, ¿hasta dónde diré?, hasta el campanario de
las monjas Descalzas. El picarón estaba muy satisfecho mirándome, y se
relamía con una lenguaza de fuego encamado tan grande como toda la calle
de Cort-Real, desde la plaza del Aceite hasta Ballesterías. Yo quería
saltar pero no podía. ¡Pobrecito de mí! Quise echarme a llorar llamando
a Siseta, pero tampoco pude. Así estuve hasta que me ocurrió decir:
«Huye, perro maldito, al infierno.» Amigo, el animal saltó bufando.
Corrí tras él de un aposento a otro y grité: «Por la señal de la Santa
Cruz.» Del dormitorio corrió a la biblioteca, de la biblioteca al
dormitorio, hasta que al fin\ldots{} ¿qué pensarás que hizo? ¡Bendita
sea mi boca! Pues reventó, quiero decir, saltó contra las paredes y el
techo, y paredes y techo todo se vino abajo. La escalera que está pegada
el dormitorio se cayó, haciendo un ruido, ¡qué ruido! Las paredes iban
retumbando así, bum, bum\ldots{} la cama, los muebles, todo se hizo
pedazos, todo se cayó al fondo, y luego, chico, el patio subió arriba:
yo vi el brocal del pozo volando por los aires, y el tejado se fue al
patio y media casa se hizo polvo. Yo me acurruqué detrás de ese armario,
y allí, con las manos en cruz, recé hasta que se me secó la lengua. Un
sudor se me iba y otro se me venía. En fin, Andresillo, hasta que no
llegó el día, no salí del rincón, ni se me quitó el miedo. Luego subí al
desván, estuve rondando por las bohardillas que no se habían hecho
pedazos, y allí me encontré otra vez con el señor Napoleón, seguido de
su guardia imperial. Les hostigué: se retiraron por la escalera abajo,
llamé a Manalet, no me respondió, me metí en el cuarto del ama del
canónigo, registrando todo y en el arca encontré el niño Jesús de
alfeñique y después, sin saber cómo ni cuándo quedeme dormido en la cama
donde me encontraste.

---Pues ahora a casa---le dije.---Que vuestra hermana está con cuidado
por ausencia tan larga.

---Despacio, amigo Andrés---me contestó el mayor.---Mira lo que tengo
aquí preparado. ¿Ves este gran artesón? Pues se le pone boca abajo,
levantado por un lado con una cañita, se ata a la punta alta de la
cañita un hilito, se ponen debajo unos pedazos de esos ratoncillos
muertos que hay en la escalera, los cuales quemaremos antes para que
huelan; plantamos en el patio toda esta artimaña, y nos escondemos en la
escalera, con el hilito en la mano para poder tirar sin que nos vean.
Hacemos humo en el sótano quemando la yerba. Salen todos, con el gran
Napoleón a la cabeza, y este los lleva al artesón, que es España;
empiezan a roer diciendo: «qué buena conquista hemos hecho;» entonces
tiramos del hilo, y España se les cae encima cogiéndoles vivos.

\hypertarget{xviii}{%
\chapter{XVIII}\label{xviii}}

Diciendo esto, cargaron con el artesón y bajáronlo al patio, y en un
instante el traidor aparato quedó muy bien instalado, con el cebo dentro
y el hilo en su sitio. España estaba dispuesta: no faltaba más que la
invasión francesa.

Badoret entró impertérrito en la bodega y volvió al poco rato diciendo:

---Están en guerra unos con otros. Vengan acá, que esto merece verse.

Entramos, y en efecto, vi la colosal batalla. Yo sabía que aquel
enérgico y emprendedor animal se vuelve en su desesperación contra su
propia casta cuando no encuentra en ninguna parte medios de
subsistencia; pero jamás había visto los choques de aquellos feroces
ejércitos, que se embestían con la saña salvaje de las primitivas
guerras entre los hombres. Se arrojaban unos sobre otros, enredándose en
horroroso vórtice, y se clavaban sin piedad las terribles armas de sus
agudos dientes. Esta lucha no era en modo alguno una revuelta explosión
de odios y hambres individuales, sino que tenía conjuntos poderosos, y
las masas parduscas indicaban empujes colectivos dirigidos por el
instinto militar que algunas castas zoológicas poseen en alto grado.

---Los que están bajo el tonel---dijo Badoret---son los del lado de allá
del Oñá que han venido nadando. Con ellos están todos los de la
parroquia de San Félix, y los de este lado son los de la plaza de las
Coles, los más gordos, los más bravos, y tienen por jefe a Napoleón.

---Pues esos que han venido nadando---dije yo,---no son otros que los
ingleses, y los de la parroquia de San Félix son la gente del Norte. Me
parece que va ganando Francia, es decir, la plaza de las Coles.

Sus gruñidos formaban un rumor espeluznante. Las desigualdades del
terreno permitían a los ejércitos desarrollar en gran escala poderosa
estrategia. Subían unos a apoderarse de un cajón vacío, y embestidos
hábilmente por la espalda, eran arrollados y expulsados de su posición.
Las masas pequeñas se reunían formando enorme cuña que al punto
desbarataba la extensa línea de los contrarios; estos, desorientados y
en desorden, reuníanse de nuevo concertando sus falanges, y sobre los
cadáveres exangües, las mil patitas marchaban con vertiginosa carrera.
Los más pequeños caían rodando impulsados por los grandes, y las panzas
blanquecinas vueltas hacia arriba, variaban el informe aspecto de los
valientes escuadrones. Las luchas individuales sucedían a los empujes
colectivos, y la heroica sangre teñía los feraces campos. ¿A quién
pertenece la victoria? Ahora lo veremos. Los de la plaza de las Coles
dominaron el tonel, y plantándose allá con provocativa presunción,
miraron jadeantes aún de cansancio, cómo huían hacía el fondo de la
bodega las huestes destrozadas de la parroquia de San Félix y del otro
lado del Oñá.

---Badoret, Manalet---exclamé yo.---Francia es vencedora. ¿Veis? Ya
domina la hermosa Italia; observad cómo corre hacia el Norte esa nube de
tudescos y sajones. Pero esto no ha concluido. Vedle allí. Ved cómo se
relame, cómo enrosca el largo rabo reluciente cual una cuerda de seda.
Con los ojuelos negros en que resplandece el genio de la guerra, observa
desde aquella altura las diversas comarcas que tiene a sus pies, y los
movimientos de sus desorganizados enemigos. Está midiendo el terreno, y
su previsión admirable adivina los sitios que escogerán los otros para
esperarle. Atended bien, Badoret y Manalet: reparad que después que ha
descansado un rato, gozándose allá arriba con sus rápidos triunfos, se
prepara a bajar de su trono. Inmensas falanges llenas de entusiasmo le
rodean, y allá en el Norte el espacio resuena con el chirrido de mil
dientes que chocan, y las colas azotan con impaciencia el suelo. Nuevas
batallas se preparan, Badoret, Manalet. Esto no quedará así, y si no me
engaño, el pérfido aspira a dominar todos los subterráneos, desde el
Galligans hasta el puente de piedra y ambas orillas del hermoso Oñá.
¿Oís? Las belicosas uñas se afilan en el suelo, y en las cuentecitas de
vidrio que tienen por ojos brilla el ardor de los combates. La hora
terrible se acerca, y el ogro, hambriento de carne y nunca saciado,
devorará a los hijos del Norte. ¡Ay! ¡Las pobres madres han concebido y
dado a luz nada más que para esto! Ya van; ya se acercan. Ved cómo todos
los de la otra crujía se reúnen, acudiendo de distintas partes. El ogro
desciende pausadamente de su trono, y una aureola de majestad le rodea.
A su vista los débiles se hacen fuertes y los tímidos se arrojan a los
primeros puestos. Ya se encuentran y está trabada de nuevo la feroz
pelea.

Avanzamos para ver mejor, y vimos cómo se devoraban llevando siempre la
mejor parte los de abajo, es decir, Francia. Si los otros eran más
fuertes, estos parecían más ligeros. Los del lado allá del Oñá, los de
San Félix y el Matadero, se sostenían enérgicamente, pero al fin no les
era posible resistir el empuje de sus contrarios, que parecían poseídos
de sublime enajenación, y sus hociquitos negros y bigotudos lo arrasaban
todo delante de sí. Si lo que les impulsaba a la lucha era pura y
simplemente el anhelo de satisfacer su apetito, una vez trabada aquella,
despierto y exaltado el genio militar, los escuálidos soldados no se
acordaban de llenar sus panzas con los despojos del vencido, y un ideal
de gloria les impelía a avanzar sobre los rotos escuadrones, sobre las
tinajas teñidas de sangre, sobre el tonel jamás conquistado, dominándolo
todo con su planta atrevida.

Creerán los oyentes que miento, que desfiguro los hechos, que pinto lo
que me conviene; juzgarán que mi cabeza trastornada por las penalidades
y debilitada por la inanición, forjó ella misma para su propio
entretenimiento estas batallas de roedores, estas ambiciones de la
última escala animal, para representar en pequeño las de la primera.
Pero yo juro y perjuro que nada he dicho que no sea cierto, así como
también lo es que Badoret, al ver cómo se destrozaban, encendió una
buena porción de yerba, apartándola del resto, para que no se declarase
incendio, y al instante el mucho y denso humo nos obligó a salir afuera
apresuradamente.

---Ahora no quedará uno dentro---dijo Badoret.---Andrés y tú, hermano,
coged un palo, y cuando salgan, de cada garrotazo caerá un regimiento.
Yo tiraré del hilo de la trampa. Si algún otro que el gran emperador se
acerca a comerse el cebo, espantadle con un golpe. En la trampa no ha de
caer sino Su Majestad.

Pronto la puerta de la oscura cueva empezó a vomitar gente y más gente,
es decir, guerreros de aquella formidable pelea que habíamos visto.
Corrieron por el patio en distintas direcciones, subieron la escalera,
tornaron a bajar, y no pocos de ellos se acercaron al artesón en quien
veían los chicos nada menos que la representación genuina de nuestra
querida y desgraciada madre España. Badoret de improviso impúsonos
silencio diciendo:

---Ahí viene; apártense todos, y abran paso a su grandeza.

En efecto, el más grande, el más hermoso, el más gordo de aquellos
caballeros, apareció en la puerta del subterráneo. Desde allí revolvió
con orgullo a todos lados los negros ojos, y moviéndose despaciosamente,
arrastraba con elegantes ondulaciones el largo rabo. Contrajo el hocico,
mostrando sus dientes de marfil, y rasguñó el suelo con majestuoso
gesto. Anduvo largo trecho entre la turbamulta de los suyos, que con
desdén miraba, y al llegar a la mitad del patio, vio aquel inusitado
aparato que teníamos dispuesto. Acercose, y estuvo mirándolo por
diversas partes, sorprendido sin duda de su extraña forma, y solicitado
de los olorosos reclamos del cebo hábilmente puesto dentro. Muy por lo
bajo, dije yo a Manalet:

---Este emperador tiene demasiado talento para meterse aquí.

---Quién sabe, Andresillo---me contestó el chico.---Como está tan
enfatuado con las batallas que acaba de ganar, y se le habrá puesto en
la cabeza que para él no hay ratoneras, ni trampas, ni lazos, puede que
se ciegue y se meta dentro.

Napoleón se acercó con paso resuelto. Aunque dotado de inmensa previsión
y de penetrante vista, el humo de gloria que llenaba su cerebro había
enturbiado sus poderosas facultades, y encontrándolo todo fácil, sin ver
más que a sí mismo y a su feliz estrella, precipitose decididamente
dentro de España. El hilo funcionó, y cayendo con estrépito la artesa,
Su Majestad quedó en la trampa.

---¡Ah, pícaro, tunante, ladrón!---exclamó Badoret saltando de
gozo.---Ahora las vas a pagar todas juntas.

---Irá vivo al mercado---añadió el otro,---y nos darán por su cuerpecito
nueve reales. Ni un cuarto menos, hermano Badoret.

\hypertarget{xix}{%
\chapter{XIX}\label{xix}}

Atado por el rabo el vencedor de Europa, los chicos querían llevarlo al
mercado; pero yo lo tomé para mí, diciéndoles:

---Si trabajáis un poco más no os faltarán otros respetables sujetos que
llevar al mercado. Dejad este para mí, que lo necesito, y coged a
Saint-Cyr, a Duhesme, a Verdier y a Augereau.

Haciendo, pues, nuevas y valiosas presas se marcharon.

Yo atravesaba la puertecilla, mejor dicho, el agujero que comunicaba al
patio de la casa de Ferragut con la mía, cuando mi cabeza tropezó con
otra cabeza. Nos topamos el señor Nomdedeu y yo, él queriendo entrar y
yo queriendo salir.

---Detente un rato más, Andrés---me dijo con agitación,---y ayúdame.
Pero qué hermoso animal tienes ahí. ¿Cuánto pides por él?

---No lo vendo---repuse con orgullo.

---Es que yo lo quiero---me dijo con firmeza, deteniéndome por un
brazo.---¿Sabes que se ha muerto Gasparó? Mi hija se muere también, es
decir, quiere morirse; pero yo no lo permito, no lo permitiré, no señor;
estoy decidido a no permitirlo.

---Nada de eso me importa, Sr.~Nomdedeu---repuse.---¿Cómo está Siseta?

---¿Siseta? Se morirá también. He aquí una muerte que importa poco.
Siseta no tiene padre que se quede sin hija. ¿Me das lo que llevas ahí?

---Usted bromea. Adiós, Sr.~Nomdedeu. Por aquella puerta se baja a donde
hay mucho de esto.

---¡Oh! ¡qué repugnante sitio!---exclamó el doctor.---¿Pero qué llevas
ahí? Un niño Jesús de alfeñique. Dámelo, Andrés, dámelo. ¡Azúcar! Dios
mío. ¡Azúcar! ¡Qué rayo de luz divina!

---No puedo darlo tampoco. Es para Siseta.

El doctor se puso lívido, más lívido de lo que estaba, y mirome con una
expresión rencorosa que me llenó de espanto. Le temblaban los labios, y
a cada instante llevábase las convulsas manos a su amarillo cráneo
desnudo. Me infundía lástima; me infundía además su vista poderoso
egoísmo, y le detestaba, sí, le detestaba, sobre todo desde que tuvo la
audacia de mirar con ávidos ojos el niño Jesús sin piernas que yo
llevaba.

---Andrés---me dijo,---yo quiero ese pedazo de azúcar. ¿Me lo darás?

Examine rápidamente a Nomdedeu. Ni él tenía armas, ni yo tampoco.

---Si no me lo das, Andrés---prosiguió,---yo estoy dispuesto a que se
pierda mi alma por quitártelo.

Diciendo esto, el doctor, sin darme tiempo a tomar actitud defensiva,
arrojose sobre mí, y me hizo caer al suelo. Clavome las manos en los
hombros, y digo que me clavó, porque parecía que manos de hierro,
horadando mi carne, se hundían en la tierra. Luché, sin embargo, en
aquella difícil posición, y conseguí incorporarme. La fuerza de Nomdedeu
era vigorosa pero de poca consistencia, y se consumía toda en el primer
movimiento. La mía, muscular e interna, carecía de rápidos impulsos;
pero duraba más. ¡Oh, qué situación, qué momento! Quisiera olvidarlo,
quisiera que se borrara por siempre de mi memoria; quisiera que aquel
día no hubiese existido en la esfera de lo real. Pero todo fue cierto y
lo mismo que lo voy contando. Yo pesé sobre D. Pablo, como él había
pesado sobre mí, y pugné por clavarlo en el suelo. Yo no era hombre, no,
era una bestia rabiosa, que carecía de discernimiento para conocer su
estúpida animalidad. Todo lo noble y hermoso que enaltece al hombre
había desaparecido, y el brutal instinto sustituía a las generosas
potencias eclipsadas. Sí, señores, yo era tan despreciable, tan bajo
como aquellos inmundos animales que poco antes había visto despedazando
a sus propios hermanos para comérselos. Tenía bajo mis manos, ¿qué
manos?, bajo mis garras a un anciano infeliz, y sin piedad le oprimía
contra el duro suelo. Un fiero secreto impulso que arrancaba del fondo
de mis entrañas, me hacía recrearme con mi propia brutalidad, y aquella
fue la primera, la única vez en que sintiéndome animal puro, me goce de
ello con salvaje exaltación. Pero no fui yo mismo, no, no, lo repetiré
mil veces; fue otro quien de tal manera y con tanta saña clavó sus manos
en el cuello enjuto del buen médico, y le sofocó hasta que los brazos de
éste se extendieron en cruz, exhaló un hondo quejido, y cerrando los
ojos, quedose sin movimiento, sin fuerzas y sin respiración.

Me levanté jadeante y trémulo, con el juicio trastornado, incapaz de
reunir dos ideas, y sin lástima miré al desgraciado que yacía inerte en
el suelo. El niño de alfeñique cayóseme de las manos, y Napoleón, que
durante la lucha se había visto libre, cargó con él, huyendo a todo
escape, con el hilo aún atado en la cola.

Esperé un momento. Nomdedeu no respiraba. La brutalidad principió a
disiparse en mí, y así como en las negras nubes se abre un resquicio,
dando paso a un rayo de sol, así en los negrores de mi espíritu se abrió
una hendidura, por donde la conciencia escondida escurrió un destello de
su divina luz. Sentí el corazón oprimido; mil voces extrañas sonaban en
mi oído, y un peso, ¡qué peso!, una enorme carga, un plomo abrumador
gravitó sobre mí. Quedeme paralizado, dudaba si era hombre, reflexioné
rápidamente sobre el sentimiento que me llevara a tan horrible extremo,
y al fin atemorizado por mi sombra, huí despavorido de aquel sitio.

Pasé al otro patio, y entrando en casa de Siseta, la vi exánime sobre el
suelo. A un lado estaba el cadáver del pobre niño, y más al fondo
advertí la presencia de una tercera persona. Era Josefina, que
hallándose sola por largo tiempo en su casa, había bajado arrastrándose.
Examiné a Siseta, que lloraba en silencio, y a su vista experimenté un
temor inmenso, una angustia de que no puedo dar idea, y la conciencia
que hace poco me enviara un solo rayo, me inundó todo de improviso con
espantosas claridades. Un gran impulso de llanto se determinaba en mi
interior; pero no podía llorar. Retorciéndome los brazos, golpeándome la
cabeza, mugiendo de desesperación, exclamé sin poder contener el grito
de mi alma irritada:

---Siseta, soy un criminal. He matado al Sr.~Nomdedeu, ¡le he matado!
Soy una bestia feroz. Él quería quitarme un pedazo de azúcar que
guardaba para ti.

Siseta no me contestó. Estaba estupefacta y muda, y la extenuación, lo
mismo que el profundo dolor, la tenían en situación parecida a la
estupidez. Josefina acercándose a mí y tirándome de la ropa, me
preguntó:

---Andrés, ¿has visto a mi padre?

---¿Al Sr.~Nomdedeu?---contesté temblando como si el ángel de la
justicia me interrogara.---No, no lo he visto\ldots{} Sí\ldots{} allí
está\ldots{} allí\ldots{} pasando al otro patio.

Y luego, anhelando arrojar lejos de mí las terribles imágenes que me
acosaban, volvime a Siseta y le dije:

---Siseta de mi corazón, ¿ha muerto Gasparó? ¡Pobre niño! ¿Y tú cómo
estás? ¿Te hace falta algo? ¡Ay! Huyamos, vámonos de esta casa, salgamos
de Gerona, vámonos a la Almunia a descansar a la sombra de nuestros
olivos. No quiero estar más aquí.

Un extraordinario y vivísimo ruido exterior no me dejó lugar a más
reflexiones ni a más palabras. Sonaban cajas, corría la gente, la
trompeta y el tambor llamaban a todos los hombres al combate. Siseta
alargó lentamente el brazo y con su índice me señaló la calle.

---Ya, ya lo entiendo---dije.---D. Mariano quiere que todos estos
espectros hagan una salida, o resistan el asalto de los franceses. Vamos
a morir. Anhelo la muerte, Siseta. Adiós. Aquí están los chicos. ¿Los
ves?

Eran Badoret y Manalet que entraron diciendo:

---Hermana Siseta, trece reales, traemos trece reales. ¿Has arreglado a
Napoleón? ¿Dónde está Napoleón?

Saliendo con mi fusil al hombro a donde el tambor me llamaba, corrí por
las calles. Estaba ciego y no veía nada ni a nadie. Mi cuerpo
desfallecido apenas podía sostenerse; pero lo cierto es que andaba,
andaba sin cesar. Hablando febrilmente conmigo me decía; ¿pero estoy
loco?\ldots{} ¿pero estoy vivo acaso? ¡Terrible situación de cuerpo y de
espíritu! Fui a la muralla de Alemanes, hice fuego, me batí con
desesperación contra los franceses que venían al asalto, gritaba con los
demás y me movía como los demás. Era la rueda de una máquina, y me
dejaba llevar engranado a mis compañeros. No era yo quien hacía todo
aquello: era una fuerza superior, colectiva, un todo formidable que no
paraba jamás. Lo mismo era para mí morir que vivir. Este es el heroísmo.
Es a veces un impulso deliberado y activo; a veces un ciego empuje, un
abandono a la general corriente, una fuerza pasiva, el mareo de las
cabezas, el mecánico arranque de la musculatura, el frenético y
desbocado andar del corazón que no sabe a dónde va, el hervor de la
sangre, que dilatándose anhela encontrar heridas por donde salirse.

Este heroísmo lo tuve, sin que trate ahora de alabarme por ello. Lo
mismo que yo hicieron otros muchos también medio muertos de hambre, y su
exaltación no se admiraba porque no había tiempo para admirar. Yo opino
que nadie se bate mejor que los moribundos.

Allí estaba D. Mariano Álvarez, que nos repitió su cantilena: «Sepan los
que ocupan los primeros puestos, que los que están detrás tienen orden
de hacer fuego sobre todo el que retroceda.» Pero no necesitábamos de
este aguijón que el inflexible gobernador nos clavaba en la espalda para
llevarnos siempre hacia adelante, y como muy acostumbrados a ver la
muerte en todas formas, no podíamos temer a la amiga inseparable de
todos los momentos y lugares.

La misma fatiga sostenía nuestros cuerpos hablábamos poco y nos batíamos
sin gritos ni bravatas, como es costumbre hacerlo en las ocasiones
ordinarias. Jamás ha existido heroísmo más decoroso, y a fuerza de ver
el ejemplo, imitábamos el aspecto estatuario de D. Mariano Álvarez, en
cuya naturaleza poderosa y sobrehumana se estrellaban sin conmoverla las
impresiones de la lucha, como las rabiosas olas en la peña inmóvil.

Por mi parte puedo asegurar que lleno el espíritu de angustia, alarmada
hasta lo sumo la conciencia, aborrecido de mí mismo, me echaba con
insensato gozo en brazos de aquella tempestad, que en cierto modo
reproducía exteriormente el estado de mi propio ser. La asimilación
entre ambos era natural, y si en pequenos intervalos yo acertaba a
dirigir mi observación dentro de mí mismo, me reconocía como una
existencia flamígera y estruendosa, parte esencial de aquella atmósfera
inundada de truenos y rayos, tan aterradora como sublime. Dentro de ella
experimentábanse grandes acrecentamientos de vida, o la súbita extinción
de la misma. Yo puedo decirlo: yo puedo dar cuenta de ambas sensaciones,
y describir cómo acrecía el movimiento, o por el contrario, cómo se iban
extinguiendo los ruidos del cañón, cual ecos que se apagaban repetidos
de concavidad en concavidad. Yo puedo dar cuenta de cómo todo,
absolutamente todo, ciudad, campo enemigo, cielo y tierra, daba vueltas
en derredor de nuestra vista, y cómo el propio cuerpo se encontraba de
improviso apartado del bullidor y vertiginoso conjunto que allí formaban
las almas coléricas, el humo, el fuego y los ojos atentos de D. Mariano
Álvarez, que relampagueando entre tantos horrores lo engrandecían todo
con su luz. Digo esto porque yo fui de los que quedaron apartados del
conjunto activo. Me sentí arrojado hacia atrás por una fuerza poderosa y
al caer, bañándome la sangre, exclamé en voz alta:

---¡Gracias a Dios que me he muerto!

Un patriota que por no tener arma se contentaba con arrojar piedras,
arrancó el fusil de mis manos inertes, y ocupando mi puesto gritó con
alegría:

---Acabáramos. ¡Gracias a Dios que tengo fusil!

\hypertarget{xx}{%
\chapter{XX}\label{xx}}

Fui primero hollado y pisoteado, y sobre mi cuerpo algunos patriotas se
empinaban para ver mejor hacia fuera; pero pronto me apartaron de allí y
sentí el contacto de suavísimas manos. Pareciome que unos pájaros del
cielo bajaban a posarse sobre mi cuerpo dolorido, trayéndole milagroso
alivio. Aquellas manos eran las de unas monjas.

Diéronme de beber y me curaron, diciéndose unas a otras:

---El pobrecillo no vivirá.

Ignoro dónde estaba, y no me es posible apreciar el tiempo que
transcurría. Sólo en una ocasión recuerdo haber abierto los ojos
adquiriendo la certidumbre de que me rodeaba oscurísima noche. En el
cielo había algunas tristes estrellas que fulguraban con blanca luz.
Sentía entonces agudísimos dolores; pero todo se extinguió prontamente,
y cayendo en profundo sopor, vivía con largas interrupciones de
sensibilidad. Otra vez abrí los ojos y vi que se estaban batiendo. Las
monjas acudieron de nuevo a mí, y su asistencia me produjo muy vivo
consuelo. Yo no hablaba: no podía hablar; pero un accidente harto
original me obligó poco después a empeñarme en usar la palabra. Entre la
mucha gente que por allí en distintas direcciones discurría, vi un
muchacho en quien hube de reconocer a Badoret.

Badoret llevaba a cuestas el cuerpo de un niño de pocos años, cuyas
piernas y brazos colgaban hacia adelante. Así cargaba comúnmente a su
hermano cuando vivía, y así lo llevaba muerto. Hice un esfuerzo y llamé
al muchacho. Este, que se inclinaba a examinar a los que allí en
diversos puntos yacían, acercose a mí y me dijo:

---Andrés, ¿tú también has muerto?

---¿Por qué llevas a cuestas el cuerpecito de tu hermano?

---¡Ay! Andrés, me mandaron que lo echara al hoyo que hay en la plaza
del Vino; pero no quiero enterrarlo, y lo llevo conmigo. El pobre ya no
llora ni chilla.

---¿Y tu hermana?

---Hermana Siseta no se mueve, ni habla, ni llora tampoco. La llamamos y
no nos responde.

Iba a preguntarle por Josefina; pero me faltó valor, se me extinguió la
facultad de hablar, y nublándose mis ojos, vi desaparecer a Badoret,
saltando con su lúgubre carga sobre los hombros.

La fiebre traumática se apoderó de mí con gran intensidad,
reproduciéndome los hechos que habían precedido a la situación en que me
encontraba. Siseta aparecía a mi lado con su hermano en los brazos, y yo
le decía:---Prenda mía, ya no podemos ir a sentarnos a la sombra de los
olivos que tengo en la Almunia, porque mi conciencia va detrás de mí
acosándome sin cesar, y tengo que huir y correr hasta que encuentre un
sitio lejano a donde ella no pueda seguirme. No volveré a entrar jamás
en tu casa, porque allí junto está, tendido en cruz sobre el suelo, D.
Pablo Nomdedeu, a quien maté porque me quería quitar mi azúcar. Yo me
voy a donde no me vea gente nacida. Dame tu mano. Adiós.

Al decir esto, estaba besando la mano de una señora monja.

Otras veces creía sentir el contacto de un brazo junto al mío, y
exclamaba: ¡Ah!, es usted, Sr.~D. Pablo Nomdedeu. Los dos hemos muerto y
nos juntamos en lo que llamábamos allá la otra vida; sólo que usted
camina hacia el cielo, y yo voy derecho al infierno. Aquí donde estamos,
entre estas oscuras nubes, ya no hay odios ni resentimientos. Me pesa de
haberle matado a usted, y válgame el arrepentimiento. ¿Cómo había de
consentir en darle a usted el azúcar? No, Sr. D. Pablo, no lo consentiré
jamás. ¿Aún insiste usted en quitármela, cuando despojados de la
vestidura corporal, volamos los dos por esta región donde no hay ruido,
ni luz, ni nada? ¿Aún aquí, equivocándonos de caminos, nos encontramos
para reñir? Pero no, siga usted adelante y no se detenga a quitarme lo
mío. Dios me perdonará mi crimen; yo fui atacado por usted, yo me
defendía, y una bestia feroz que se metió dentro de mí, le mató a usted.
Fue sin duda aquel infame Napoleón. ¡Oh! ¿Por qué quise apropiarme el
aparente cuerpo de tan fiero demonio? Sí, ya te estoy viendo delante de
mí\ldots{} Allá voy, no me llames más. Vagando por estos espacios donde
no hay ruido, ni luz, ni nada, yo creí que no te presentarías delante de
mí; pero aquí estás. Cierra esos ojillos negros como cuentas de
azabache, no claves en mí tus dientes más blancos que el marfil, ni
enrosques esa culebra que llevas por cola. Ya sé que te pertenezco desde
que cayó el artesón sobre ti, y tus tramas infernales me pusieron en el
caso de matar a aquel santo varón, buen amigo, excelente padre y honrado
patriota. Iré contigo al infierno, que será mi expiación. No vuelvas el
horrendo hocico hacia atrás, que ya te sigo. Los arcángeles celestiales
me azuzaron como a un perro cuando me acerqué a las puertas del Paraíso,
y ahora camino hacia abajo. Adiós, Nomdedeu, ya te veo allá arriba.
Brillas como una estrella; pero tu resplandor no ilumina esta oscuridad
en que me veo. El calor de las llamas que despides por la boca, infame
Napoleón, me está abrasando, me ahogo en una atmósfera de fuego, y una
sed espantosa seca mi boca. ¿No hay quién me dé un poco de agua?

Un vaso tocó mis labios. Las monjas me daban agua.

Luego tornaba a los mismos delirios, siempre éstos diversos a cada
instante, ora terribles, ora gratos, hasta que un día me reconocí en el
uso completo de mis sentidos y con el entendimiento claro y sin nubes.
Vi el cielo encima, en derredor mucha gente que hablaba, y a mi lado un
fraile. No se oían cañonazos, y el silencio, con serlo, parecía un ruido
indefinible.

---Hijo mío---me dijo el fraile---¿estás mejor? ¿Te sientes bien? Esa
herida del pecho no es mortal. Si hubiera recursos en Gerona y se te
alimentara bien, curarías como otros muchos.

---¿Qué ocurre, padre? ¿Qué día es hoy? ¿A cuántos estamos?

---Hoy es el 9 de Diciembre, y ocurre una inmensa desgracia.

---¿Qué?

---Está enfermo D. Mariano Álvarez, y la ciudad se va a rendir.

---¡Enfermo!---exclamé con sorpresa.---Yo creí que D. Mariano no podía
estar enfermo ni morir. Moriremos nosotros; pero él\ldots{}

---Él también morirá. Hoy le ha entrado el delirio y ha traspasado el
mando al teniente del Rey D. Juan Bolívar. Desde que Álvarez está en
cama, nadie considera posible la defensa. Sólo hay mil hombres
disponibles, y aun estos están también enfermos. A estas horas hay junta
de jefes para ver si se rinde o no la plaza en este día. Me temo que se
saldrán con la suya los pícaros que quieren la rendición. Es una
vergüenza que esto pase. Hay aquí mucha gente que no piensa más que en
comer.

---Padre---dije yo,---si hay algo por ahí, démelo, aunque sea un pedazo
de madera. No puedo resistir más.

El fraile me dio no sé qué cosa; pero yo la devoré sin averiguar lo que
era. Después le hablé así:

---¿Su paternidad está aquí auxiliando a los moribundos? Yo, aunque Dios
en su infinita misericordia me conserve por ahora la vida, quiero
confesar un gran pecado que tengo. Si no me quito de encima este gran
peso, no podré vivir. Por ahí creerán que D. Pablo Nomdedeu ha muerto de
hambre o de miedo. No, yo debo declarar que le he matado porque me quiso
quitar un pedazo de azúcar.

---Hijo mío---repuso el fraile,---o estás aún delirando, o confundiste
con otro al Sr.~Nomdedeu, pues tengo la seguridad de haber visto a este
hoy mismo, si no bueno y sano, al menos con vida. No descansa en lo de
curar a diestro y siniestro.

---¡Cómo! ¿Es posible?---exclamé con estupefacción.---¿Vive el Sr.~D.
Pablo Nomdedeu, ese espejo de los médicos? Padre, tan buena nueva me
devuelve por entero la vida. Yo le dejé por muerto en medio del patio.
No puedo creer sino que ha resucitado para que su hija no quedase
huérfana. Padre, ¿conoce usted a Siseta, la hija del Sr.~Cristòful
Mongat? ¿Sabe por ventura si vive?

---Hijo, nada puedo decirte de esa muchacha. Sólo sé que la casa donde
vivían el Sr.~Mongat y el Sr.~Nomdedeu, ha sido destruida por una bomba
ayer mismo. Tengo idea de que todos sus habitantes se salvaron, excepto
alguno que se ha extraviado, y no se le puede encontrar.

---¡Oh! ¡Si pudiera levantarme y correr allá!---dije.---Pero parece que
me han clavado en esta maldita cama. ¿En dónde estoy?

---Esta es la cama en que murió Periquillo del Roch, asistente del
Sr.~D. Francisco Satué, que es, como sabes, edecán del gobernador.
Cuando murió Periquillo, te pusimos aquí, y ayer dijo Satué que te
tomaría por asistente.

---¿Con que Su Paternidad no me da noticias de la pobre Siseta? El
corazón me dice que no ha muerto, y que no soy por lo tanto viudo.

---¿Eres casado?

---Con el corazón. Siseta será mi mujer si vive\ldots{} ¿Y dice Su
Paternidad que no ha muerto el Sr.~Nomdedeu?

---Así parece, pues se le ve por la ciudad. Verdad es que más bien tiene
aspecto de un muerto que anda, que de persona viva.

---¿Será cierto lo que oigo? ¿Y el Sr.~D. Pablo se mueve?

---Anda, aunque cojo.

---¿Y abre los ojos?

---Sí; sus ojos parduzcos buscan las piernas rotas en la oscuridad de
los escombros.

---¿Y habla?

---Con su voz clueca, que tan buenas cosas sabe decir.

---¿Pero es el mismo, o un remedo de don Pablo, una sombra que viene del
otro mundo a figurar que pone vendas?

---El mismo, aunque de puro desfigurado, apenas se le conoce.

---¡Oh, qué inmensa alegría siento! ¿De modo que ha resucitado?

---No dudes que vive; pero también te aseguro que no doy dos ochavos por
lo que le quede de razón.

En todo aquel día no me pude mover, aunque notaba de hora en hora
bastante mejoría. La curiosidad y el afán me devoraban, anhelando saber
la suerte de los míos, y aunque la certidumbre de no ser matador de
Nomdedeu había dado gran tranquilidad a mi espíritu, el no saber el
paradero de Siseta me entristecía en sumo grado. Sin moverme de allí
supe que la plaza estaba a punto de rendirse, y que había ido a tratar
con el general francés el español D. Blas de Fournás. Esto tenía muy
irritados a los fantasmas que con el nombre de hombres discurrían aún
arma al brazo por las murallas destruidas, y fue preciso a Fournás,
cuando salió de la plaza, ocultar el verdadero motivo de su viaje.

Álvarez, según oí, se agravaba por instantes y recibió los sacramentos
el mismo día 9; pero aun en tal situación insistía en no rendirse,
repitiendo esto con palabras enérgicas, lo mismo dormido que despierto.
Muchos patriotas se resistían a creer que fuera cierto lo de la
rendición, y la posibilidad de entregarse al extranjero causaba más
horror que la muerte y el hambre; verdad es que muchos tenían aún la
loca esperanza de que llegasen socorros.

Por la tarde empezó a susurrarse que al día siguiente entrarían los
\emph{cerdos}, y los patriotas acudieron a casa del gobernador, la cual,
casi por completo arruinada, apenas conservaba en pie los aposentos
donde el heroico paciente residía, y allí entre las ruinas, metiéndose
por los claros de las paredes destruidas, alborotaron largo rato
pidiendo a su excelencia que saliese de nuevo a gobernar la plaza. Dicen
que Álvarez en su delirio oyó los populares gritos, e incorporándose
dispuso que resistiéramos a todo trance. Enfermos o heridos los que aún
vivíamos, con diez mil cadáveres esparcidos por las calles,
alimentándonos de animales inmundos y sustancias que repugna nombrar,
nuestro más propio jefe debía de ser y era un delirante, un insensato,
cuyo grande espíritu perturbado aún se sostenía varonil y sublime en las
esferas de la fiebre.

Al día siguiente pude dar algunos pasos sin alejarme mucho. De buena
gana habría hecho una excursión por la ciudad visitando la casa de
Siseta; pero las señoras monjas que tan cariñosamente me cuidaban
impidiéronmelo. El capitán D. Francisco Satué llegose a mí y me hizo
saber que había resuelto tomarme por asistente en reemplazo de
Periquillo Delroch, y yo, agradecido a su bondad, me tomé la libertad de
decirle:

---Mi capitán: ¿sabe usía por dónde anda Siseta? Supongo que usía conoce
a Siseta, la hija del Sr.~Cristòful Mongat.

Satué no se dignó contestarme, y volvió la espalda, dejándome solo con
mis horrorosas dudas. Yo preguntaba a todos; pero nadie me hablaba sino
de la capitulación. ¡Capitular! Parecía imposible tal cosa cuando
todavía existía pegado a las esquinas el bando de D. Mariano:
\emph{«Será pasada inmediatamente por las armas cualquier persona a
quien se oiga la palabra capitulación u otra equivalente.»}

Según oí decir, los franceses habían dado una hora de tiempo para
arreglar la capitulación; pero nuestra Junta pedía un armisticio de
cuatro días, prometiendo cumplirla si al cabo de dicho plazo no venía el
socorro que desde Noviembre estábamos esperando. El mariscal Augereau no
quiso acceder a esto, y por último, después de muchas idas y venidas de
un campo a otro, firmáronse las condiciones de nuestra rendición a las
siete de la noche del 10.

En este convenio, como en todos los que hicieron los franceses en
aquella guerra, se pactó lo que luego no había de ser cumplido: respetar
a los habitantes, respetar la religión católica y las vidas y haciendas,
etc\ldots{} Todo esto se escribe y se firma sobre un tambor dentro de
una tienda de campaña; pero luego las órdenes expedidas desde París por
la gran rata obligan a poner en olvido lo acordado.

---¡Bonito final!---me dijo el padre Rull, que me había asistido durante
el penoso mal.---¡Y que hayamos venido a esto después de haber resistido
siete meses! ¿Y todo por qué, amigo Andrés? Porque no se reparten dos
pavos por barba al día, y porque alguno se ha visto obligado a
mantenerse chupando el jugo de un pedazo de estera. Dioscórides dice que
el esparto contiene sustancias alimenticias. ¡Oh! Si Álvarez no hubiera
caído enfermo, si aquel hombre de bronce pudiera aún levantarse de su
lecho y venir aquí y alzar el bastón en la mano derecha\ldots{} Ya
sabes, Andrés, que la guarnición debe salir mañana de la plaza con los
honores de la guerra, marchando a Francia prisionera. Creo que os
pondrán a tirar del carro de Napoleón cuando salga a paseo\ldots{} Los
\emph{cerdos} se nos meterán aquí mañana a las ocho y media, y parece
han acordado no alojarse en las casas sino en los cuarteles. ¿Lo crees
tú? Ya verás cómo no lo cumplen. Me parece que los veo echando a los
vecinos a la calle para acomodarse sus señorías en las pocas casas que
han dejado en pie. Y ahora te pregunto yo: ¿qué harán de nosotros, los
pobres frailes? Amigo, con Gerona se acabó España, y con la salud de
Álvarez se acabaron los españoles bravos y dignos. Muchachos, ¡viva D.
Mariano Álvarez de Castro, terror de la Francia!

Durante la noche, los vecinos y los soldados, sabedores ya de las
principales cláusulas de la capitulación, inutilizaron las armas o las
arrojaron al río, y al amanecer los que podían andar, que eran los
menos, salieron por la puerta del Areny para depositar en el glacis unas
cuantas armas si tal nombre merecían algunos centenares de herramientas
viejas y fusiles despedazados. Los enfermos nos quedamos dentro de la
plaza, y tuvimos el disgusto de ver entrar a los señores cerdos. Como no
nos habían conquistado, sino simplemente sometido por la fuerza del
hambre, nosotros los mirábamos de arriba abajo, pues éramos los
verdaderos vencedores, y ellos al modo de impíos carceleros. Si no
existiese el goloso cuerpo, y sólo el alma viviera, ¿pasarían estas
cosas?

En honor de la verdad, debo decir que los franceses entraron sin
orgullo, contemplándonos con cierto respeto, y cuando pasaban junto a
los grupos donde había más enfermos, nos ofrecían pan y vino. Muchos se
resistieron a comerlo; pero al fin la fuerza instintiva era tal que
aceptamos lo que a las pocas horas de su entrada nos ofrecieron. Durante
todo el día estuvieron entrando carros cargados de víveres que
estacionados en las plazas de San Pedro y del Vino, servían de depósito,
a donde todo el mundo iba a recoger su parte. ¡Comer!, ¡qué novedad tan
grande! Sentíamos el regreso del cuerpo que volvía después de la larga
ausencia, a ser apoyo del alma. Se admiraba uno de tener claros ojos
para ver, piernas para andar y manos con que afianzarse en las paredes
para ir de un punto a otro. Los rostros adquirían de nuevo poco a poco
la expresión habitual de la fisonomía humana, y se iba extinguiendo el
espanto que aun después de la rendición causábamos a los franceses.

Dadme albricias, porque al fin, señores míos, me reconocí con bríos para
andar veinte pasos seguidos, aunque apoyándome con la derecha mano en un
palo, y con la izquierda en las paredes de las casas. No creáis que el
andar por las calles de Gerona en aquellos días era cosa fácil, pues
ninguna vía pública estaba libre de hoyos profundísimos, de montones de
tierra y piedras, además de los miles de cadáveres insepultos que
cubrían el suelo. En muchas partes los escombros de las casas destruidas
obstruían la angosta calle, y era preciso trepar a gatas por las ruinas,
exponiéndose a caer luego en las charcas que formaban las fétidas aguas
remansadas. El viaje al través de aquellos montes, lagos y ríos era tan
fatigoso para mí, que a cada poco trecho me sentaba sobre una piedra
para tomar aliento. Mas cuando no era ya posible pensar en batirse, y
cuando estaba aplacado el terrible ardor de la guerra, me producía
indecible espanto la vista de tantos muertos; y al examinar los
horrorosos cuadros que se desarrollaban ante mi vista, cerraba a veces
los ojos temiendo reconocer en una mano helada la mano de Siseta, en la
punta de un vestido, la punta del vestido de Siseta, en una piedrecita
encarnada las cuentas de coral que adornaban las lindas orejas de
Siseta.

\hypertarget{xxi}{%
\chapter{XXI}\label{xxi}}

Al llegar a la calle de Cort-Real, vi allí casi en total ruina la casa
donde se albergaban los míos. Unos vecinos me dijeron que el señor
Nomdedeu y su hija estaban aposentados en la calle de la Neu; pero que
no se sabía dónde habían ido a parar Siseta y sus hermanos. Contristado
con tal noticia, fui en busca del doctor, y la primer persona que salió
a mi encuentro fue la señora Sumta, encargándome que no hiciera ruido
porque el señor dormía.

---Aquí encontrarás todos los papeles cambiados, Andresillo---me
dijo,---porque la señorita Josefina se ha puesto buena, y el amo está
tan malo, que se morirá pronto si Dios no lo remedia.

En esto oímos la voz del doctor, que en aposento cercano sonaba,
diciendo:

---Déjele usted entrar, señora Sumta, que estoy despierto. Andrés, amigo
querido, ven acá.

Entré, pues, y D. Pablo arrojándose de su lecho me abrazó con cariño,
hablándome así:

---¡Qué placer me das, Andrés! ¡Yo creí que habías muerto! ¡Ven acá,
valiente joven, y abrázame otra vez! ¿Cómo va esa salud? ¿Y ese
estómago? No conviene cargarlo después de tanta privación. ¿Hay
apetito?\ldots{} Te recomiendo mucho la sobriedad. ¿Tienes heridas? Las
curaremos\ldots{} Manda lo que gustes, hijo.

Yo, muy confundido, le expresé mi gratitud por tanta benevolencia,
añadiendo que le consideraba como el más generoso y cristiano de los
mortales por pagar con abrazos y cariños los golpes que de mí recibiera.

---Señor---añadí,---yo creí haber muerto al mejor de los hombres, y no
podía vivir con el gran peso de mi conciencia. Veo que usted perdona las
ofensas y abre sus brazos a los que han intentado matarle.

---Todo está perdonado, y si culpa hubo en ti tratándome como me
trataste, mayor fue la mía, que en mi furor, no reparaba en quitarte la
vida por un pedazo de azúcar. Aquellas, amigo Andrés, no deben
considerarse como acciones libres que constituyen verdadera
responsabilidad, y la horrible situación en que ambos nos hallábamos nos
disculpa a los ojos de Dios. En tan triste momento, la ley suprema de la
propia conservación imperaba sobre todas las leyes, nuestro carácter, el
resultado de las facultades ingénitas o cultivadas por el trato y de los
hábitos adquiridos, no existía realmente, y el torpe bruto en que
estamos metidos, rompía salvajemente todos los frenos que se oponían a
la satisfacción de sus necesidades. Por mi parte, puedo decirte que no
me daba cuenta de lo que hacía. El espectáculo de mi pobre hija me
trastornaba el poco sentido que aún me hacía reconocerme como hombre, y
delante de mí no había amigos ni semejantes. Estas relaciones se acaban,
se extinguen cuando el brutal instinto recobra sus dominios, y si veía
un pedazo de pan en boca de otro hombre, parecíame esto un privilegio
irritante, que mi egoísmo no podía tolerar. ¡Ay, qué horroroso
padecimiento! ¡Qué vergonzoso estado de moral y qué degradación del ser
más noble que pisa la tierra! Válgame tan sólo la circunstancia de que
nada quería para mí, sino todo para ella. Tengo la seguridad de que a no
ser por mi idolatrada hija, yo me hubiera recostado en un rincón de la
casa, dejándome morir sin hacer esfuerzo alguno por conservar la vida.

---Y la señorita Josefina ha resistido las privaciones tal vez mejor que
nosotros.

---Mucho mejor---añadió Nomdedeu.---Ya me ves a mí que parezco un
cadáver. Pues ella, completamente transfigurada, parece haberse
apropiado toda la salud que a mí me falta. Esto me tenía contentísimo,
Andrés. Pero verás ahora lo que ha pasado. Cuando me dejaste en el patio
de la casa del canónigo, tardé mucho tiempo en recobrar el uso de los
sentidos a consecuencia del gran golpe y de la mucha extenuación. Por
fin, no sé qué manos caritativas me sacaron a la calle, donde recobré
completo acuerdo. Mi sensación principal era una gran sorpresa de
hallarme con vida. Arrastréme hasta entrar en casa, y en las
habitaciones de Siseta encontré a mi hija. La infeliz casi no me
conocía. Iba a perecer de inanición. ¡Dios mío! Quisiera morir, si la
muerte borrara de mi memoria el recuerdo de aquellas horas. Yo
decía:---Señor, antes de ver tal espectáculo, valiera más que quedara
exánime sobre las baldosas de la casa del canónigo.---¡Ay, amigo
Marijuán, no me preguntes nada sobre esto! Sólo te diré que habiendo
salido en busca de alimentos, al regresar, mi hija ya no estaba allí.

---¿Y Siseta?---pregunté con la mayor inquietud.

---Siseta tampoco---repuso Nomdedeu inmutándose en sumo grado.---Pero ¿a
qué me preguntas por Siseta? Yo no sé nada de ella. Déjame seguir.
Ninguno de los vecinos supo darme razón del paradero de mi hija, y corrí
como un loco por la ciudad buscándola. Felizmente ni ella ni yo
estábamos allí, cuando la casa fue destruida. Pero yo te pregunto: ¿a
dónde creerás que había ido mi idolatrada Josefina? Pues nada menos que
a la torre Gironella, donde contemplaba el horrible fuego con que se
defendió aquel fuerte en sus postrimerías. Te asombrarás de que mi hija
fuera a tal sitio. Pues oye. Encontrándose sola en la casa, la horrible
necesidad obligola a salir a la calle, y discurrió largo tiempo por
Gerona, implorando la caridad pública, pero sin ser atendida por nadie.
Mientras mayor era su desamparo, mayores eran sus esfuerzos por apegarse
a la vida, y aquella naturaleza miserable halló en sí misma suficiente
energía para sobreponerse a la situación. Parece esto imposible, pero es
cierto. Ahora caigo en que a las criaturas de ánimo apocado nada les
conviene tanto como encontrarse lanzadas de improviso a un gran peligro
sin sostén ni ayuda de mano extraña. Pues bien, Josefina, sola en medio
de tantos horrores, huyó por la pendiente que conduce a los fuertes,
creyendo más seguros aquellos sitios. La vista de los cadáveres que
obstruyen el camino prodújole gran espanto, y mayor aún al ver de cerca
la terrible acción que allí se trabara. Cuando quiso retroceder la
pobrecita, le fue imposible, y encontrose envuelta en el fuego, en el
momento de la retirada. ¡Oh, qué incomprensibles son los arcanos de la
Naturaleza! Si yo hubiera sabido por qué lugares andaba mi enferma, y
todo el protomedicato hubiérame pedido mi dictamen sobre su suerte,
habría dicho: «Josefina morirá en el acto de verse próxima a un
combate.» Pues no fue así, Andrés. Según me ha contado ella misma, al
ver aquello, sintiose con inusitada energía, y sus miembros
desentumecidos como por milagro, adquirieron una agilidad que jamás
habían tenido. Sin hallarse libre de miedo, inundaba su alma una
generosa y expansiva inquietud, y abundantes lágrimas corrían de sus
ojos\ldots{} A esto añade que luego volvió dos veces a la ciudad, donde
unas señoras apiadadas de ella la dieron algún alimento; que después,
sin saber cómo, viose arrastrada en el tropel de las que iban a llevar
pólvora a las murallas; añade que durmió dos noches en campo raso; que
la señora Sumta tomándola por su cuenta, la tuvo más de tres horas en
Alemanes, hasta que se retiró de allí la guarnición, y comprenderás si
han sido fuertes los cauterios aplicados por el azar al espíritu de esa
pobre niña. Ahora, Andrés, me resta decirte que si ella ha adquirido
súbitamente bríos y agilidad, yo he perdido radicalmente mi salud, a
consecuencia de los intensos padeceres físicos y morales de esta
temporada, y aquí donde me ves, no doy dos cuartos por lo que pueda
vivir de aquí al domingo que viene. La alegría que me causa el ver cómo
se ha regenerado el organismo de aquella que es todo mi amor y mi
consuelo, ahoga el sentimiento que podría causarme la propia muerte. Lo
que hoy me produce profunda tristeza es el convencimiento adquirido hace
poco de que soy un detestable médico. Sí, Andrés, yo creí saber
bastante, y ahora resulta que todo lo ignoro, todo, todo. Figúrate que
después de adoptar en el tratamiento de Josefina el sistema de
precauciones, de cuidados que me recomendaban en diverso estilo
centenares de libros, salimos con la patochada de que el mejor sistema
es el opuesto al que yo seguí. ¡Y para esto, Dios mío, ha estudiado uno
treinta años! ¡Oh!, medicina, medicina, ¡cuán desdeñosa y esquiva eres!
¡Cómo te ocultas al que más te busca, y qué bien guardas tus encantos!
Cuando parece más fácil tocarte, más rápidamente desapareces, como
sombra que de las ansiosas manos se escapa. ¡Quién me lo había de decir!
Yo intentaba curarla con delicadezas y cuidados y dengues,
resguardándola hasta del aire por temor a que el aire mismo la hiciera
daño, y Dios la ha fortalecido con las crudezas, las molestias, los
golpes, los sustos, con el fuego y el frío, con los peligros y las
muertes. Yo evitaba en ella las grandes impresiones que me parecía
debieran quebrar su naturaleza, como los martillazos rompen el vidrio, y
los fortísimos sacudimientos de la sensibilidad la han repuesto en su
primer ser y estado. Curose como había enfermado, y este misterio y esta
novedad pasmosa confunden mi inteligencia. Hasta ahora no sabía que la
enfermedad curase la enfermedad, y me muero con mil ideas sobre este
oscuro punto\ldots{} porque yo me muero, Andrés: en eso sí que no se
equivocará mi escaso saber.

Diciendo esto, se tendió de largo a largo en la cama, y a cada rato
exhalaba hondísimos suspiros. Yo le hablé así:

---Sr.~D. Pablo, usted, aunque ha padecido bastante, tiene el consuelo
de ver a su hija no sólo con vida, sino con la salud que antes no tenía;
pero yo, ni siquiera puedo asegurar que vive mi adorada Siseta y sus dos
hermanos.

El doctor, al oírme, moviose inquietamente en su lecho con síntomas de
alteración nerviosa, e incorporándose de improviso, me mostró su cara,
muy contrariada y desfigurada de un modo notable.

---No me preguntes por Siseta y sus hermanos---exclamó con torpe lengua
y haciendo ademán de apartar un objeto que inspira desagrado.---Yo no sé
nada de ellos. Andrés, más vale que te marches y me dejes en paz.

La señora Sumta, que entró a la sazón, puso el dedo en la sien, mirando
a su amo con expresión de lástima. Con el gesto y la mirada quería
decirme:

---No hagas caso, que el amo ha perdido el juicio.

Perdiéralo o no, lo cierto es que me llenaban de inexplicables
confusiones sus palabras. Interroguele de nuevo; pero él, cerrando los
ojos y extendiendo brazos y piernas, cual exánime cuerpo, aparentaba no
oírme, o realmente aletargado, no me oía.

Josefina entró en seguida y mostró mucha alegría al verme. Por mi parte
quedeme sorprendido al notar la animación de sus ojos, su color menos
pálido que de ordinario, y al observar la agilidad, la gracia y
desenvoltura que había adquirido en sus movimientos desde que no nos
veíamos. Después de contestar con amables sonrisas a mis cumplidos, que
adivinaba por el movimiento de los labios, me preguntó por Siseta.

---¡Ay!---respondí, expresando con signos mi suprema
aflicción.---Siseta\ldots{} se ha ido, señorita; no sé dónde está.

---Busquémosla---dijo Josefina con resolución.

---¡Ay!, gracias, señorita Josefina\ldots{} Yo no me puedo tener; pero
si usted me acompaña, sacaré fuerzas de flaqueza para recorrer la
ciudad.

En la casa tenían ya comida abundante, que se repartía entre los
diferentes vecinos allegadizos que allí se albergaban, y a mí me dieron
una buena porción. Cuando salí enlazando mi brazo con el de Josefina, me
sentía tan restablecido, que no necesité buscar apoyo en las paredes, ni
arrojarme al suelo cada diez minutos para tomar aliento.

\hypertarget{xxii}{%
\chapter{XXII}\label{xxii}}

¿Dónde buscaremos a Siseta? ¿Dónde?\ldots{} Siseta, gritábamos por todos
lados, en las ruinas, en la puerta de las casas enteras, en las plazas,
en las murallas, en las cortaduras, en los montones de escombros; pero
ninguna voz conocida nos respondía. En diversos puntos de la ciudad, los
franceses se ocupaban en tapar con tierra los hoyos donde habían sido
arrojados los cadáveres, y miles de cuerpos desaparecían de la vista de
los vivos para siempre\ldots{} ¡Oh!---exclamaba yo con la mayor
angustia,---¡si estará ahí Siseta!

Hubiera querido escarbar con mis manos todas las fosas, por cerciorarme
de que no yacía en ellas la persona perdida. Visitamos luego los
hospitales, y en ninguno de ellos aparecieron tampoco Siseta ni sus
hermanos: preguntamos de puerta en puerta a todos los conocidos, a los
vecinos todos, y nadie nos dio razón ni noticia alguna. Pasando a
Mercadal, lo recorrimos todo, y al volver, miré al fondo del río, por
ver si entre sus turbias aguas se distinguía el cuerpo de Siseta.
Pregunté por ella a los españoles y a los franceses que no me
entendieron; pero ambas naciones carecían de noticias acerca de mi
amiga; subí a los tejados, bajé a los sótanos, la busqué en plena luz y
en la profunda oscuridad; pero el rayo de sus ojos, para mí superior a
todas las claridades, no brillaba en ninguna parte.

Por último, cuando llegábamos cerca del puente de San Francisco de Asís,
creí distinguir una lastimosa figura de muchacho, en la cual, aunque con
mucha dificultad, podía reconocer a la persona del buen Manalet. No era
posible determinar la forma de su vestido, que era un andrajo, por cuyas
rasgaduras los brazos y las piernas en completa desnudez asomaban. Su
rostro cadavérico, sus manos negras, su cuello manchado de sangre, sus
pies heridos, su mirar temeroso me causaron profunda pena. Le llamé, con
el alma dividida entre una animosa esperanza y un inmenso dolor, y él
corrió a abrazarme con los ojos llenos de lágrimas. Pasado el primer
momento de su alegría, la presencia de Josefina al lado mío produjo en
el ánimo del pobre chico vivísima inquietud; mirábala con ojos azorados,
e hizo algún movimiento para huir de nosotros. Deteniéndole, tuve valor
para preguntarle por su hermana.

---Hermana Siseta---me dijo,---no está, no la busquen ustedes. Se ha ido
con Gasparó. Los dos\ldots{}

Al decir \emph{los dos} señalaba la tierra.

Yo, poseído de profundo dolor, no me reconocía satisfecho con sus vagas
noticias y quería saber más; seguí tras él, pero mi corto andar no me
permitió alcanzarle y hube de resignarme al terrible padecimiento de la
duda; porque, en efecto, las afirmaciones de Manalet no resolvían mi
perplejidad, y las palabras, el razonamiento, la inquietud del infeliz
chico indicaban que algún misterio para mí ignorado, existía en la
desaparición de Siseta.

---Señorita Josefina---dije a mi acompañante, expresando como me fue
posible el desaliento y la desesperación,---no conseguiremos nada.
Volvámonos a la calle de la Neu.

Ambos muy tristes y desanimados nos detuvimos en el puente, mirando a
los transeúntes, que discurrían sin cesar de un lado a otro y como yo
buscaban personas queridas que el desorden de los últimos días había
hecho desaparecer. Las fosas sobre las cuales se echaba tanta tierra
iban poco a poco destruyendo los rastros que habrían podido guiar en sus
exploraciones a padres, esposas e hijos, y la necesidad de enterrar
pronto hacía que muchas familias se quedasen en completa ignorancia
respecto a la suerte de los suyos.

Estábamos sentados junto al puente. Josefina me miraba en silencio,
compadecida de mi dolorosa perplejidad, y yo interrogaba al cielo,
cansado ya de interrogar a la tierra y a los hombres. De repente, la
hija del doctor diome un ligero golpe en la cabeza y agitando los brazos
en dirección del río, señaló una casa de las que se levantan con los
cimientos dentro del Oñá a espaldas de la plaza de las Coles y de la
calle de la Argentería. Al principio no distinguí nada; pero ella con el
rostro alterado, la mirada chispeante y el índice extendido hacia un
punto fijo, dirigió mi atención al tejado de una de aquellas casas, de
cuyo alero, un muchacho se descolgaba trabajosamente por una cuerda. Era
Badoret. Al instante grité fuertemente: ¡Badoret! ¡Badoret!, y el chico
que oyó mi voz, saludome con la mano en el momento de poner pie firme en
un balcón, desde el cual parecía querer avanzar al puente saltando de
una casa a otra. Los irregulares aleros, balconajes, miradores y cuerpos
salientes de aquella orilla del río, permitían este viaje sin gran
peligro. Por fin, Badoret llegó a donde estábamos, y pude notar que su
aspecto era más lastimoso que el de su hermano.

---Andrés---me dijo---¿han entrado los franceses?

---Sí---le respondí.---¿En dónde estás metido que no lo sabes? ¿Has
resucitado acaso?

---¿De modo que ya hay algo que comer?

---Sí, todo lo que quieras\ldots{} ¿Y Siseta?

---Siseta está durmiendo desde ayer. ¿Quieres verla? La llamamos y no
quiere despertar.

---¿Pero dónde os habéis metido? ¿Dónde está Siseta?

---¿Hay ya qué comer? No hemos vuelto a ver a Napoleón, Andrés. ¿Cuánto
darán ahora por él?

---Anda al diablo con Napoleón. Llévame a donde está tu hermana.

---En el tejado.

---¡En el tejado!

---Sí: la llevamos allá entre todos, porque el Sr.~Nomdedeu la quería
matar.

---¡Matarla! ¡Estás loco!

---Sí; para comérsela.

No pude reprimir la risa, a pesar de que mi ánimo no estaba para burlas.

---El Sr.~Nomdedeu---prosiguió,---se volvió loco y quiso comernos a
todos.

---Estáis tontos sin duda---repliqué.---Llévame donde está Siseta.

---Si no vas por donde yo he venido\ldots{} De la casa del canónigo
donde estamos, se pasa por el tejado a la del droguero de la calle de la
Argentería, pero de esta no se puede salir a la calle porque está
cerrada\ldots{} Por la bodega, se pasa a una casa del otro extremo que
está quemada y por las tejas se baja a los balcones del río. Si puedes
hacer que te abran la puerta de la casa del droguero que está en la
calle de la Argentería junto a la plaza de las Coles, entrarás mejor que
yo he salido.

---Vamos allá---dije con resolución.---Si ese señor droguero no nos
quiere abrir la puerta, la derribaremos a puñetazos.

Por fortuna, no me pusieron obstáculos a que entrara por la casa
indicada, lo cual verifiqué dejando a Josefina en la inmediata de la
calle de la Neu. Subí al tejado, y saltando con grandes esfuerzos y
peligros de techo en techo, llegamos Badoret y yo a las bohardillas de
la casa del canónigo. Allí en un lóbrego aposento del desván, donde
antaño tuvo su vivienda el ama de gobierno del Sr.~Ferragut, yacía la
pobre Siseta sin movimiento ni sentido sobre un miserable colchón. La
llamé con fuertes voces, incorporela en el lecho, y la infeliz abrió los
ojos, pero sin aparentar reconocerme. Mi gozo al ver que vivía fue
inmenso; pero aún dudaba que pudiese tornar a la vida, y no pensé más
que en prodigarle toda clase de socorros. Recorrí la casa aturdidamente
sin darme cuenta de lo que buscaba, y vi en distintas habitaciones hasta
una docena de chicos de ocho a doce años, en quienes reconocí a los
amigos que acompañaban a Badoret y Manalet en todas sus correrías; pero
el estado de aquellos infelices niños era atrozmente lastimoso y
desconsolador. Algunos de ellos yacían muertos sobre el suelo, otros se
arrastraban por la biblioteca sin poderse tener, uno estaba comiéndose
un libro, y otro saboreaba el esparto de una estera.

---¿Qué ha pasado aquí?---pregunté a Badoret.

---Ay ¡Andrés!, no podíamos salir por ninguna parte. Estábamos
encerrados hace dos días. A nuestra casa no se podía pasar, porque siete
paredes llenaron el patio hasta arriba. No teníamos qué comer, ni dónde
buscarlo\ldots{} Esta mañana buscamos Manalet y yo una salida. Él se
descolgó por la calle de Argentería, y yo por donde me viste\ldots{}
pero a mí se me está ya pegando la lengua al cielo de la boca, no puedo
moverme, y me caigo muerto también.

Diciéndolo, Badoret, cerró los ojos y se extendió de largo a largo en el
suelo. Algunos de sus camaradas lloraban, llamando a sus madres, y por
todos lados el espectáculo de aquella desolación infantil contristaba mi
alma. Resuelto a obrar con prontitud, pasé por el tejado a las casas
inmediatas, llamé, pedí socorro, logré que me oyeran y que acudiesen en
mi auxilio algunos vecinos, y bien pronto, reuní en los desiertos
lugares donde se hallaba mi infeliz amiga gran número de víveres y no
pocas personas caritativas.

La primera en quien probamos nuestros recursos fue Siseta, que tardó
mucho en recobrar su acuerdo, inspirándome serias inquietudes; pero al
fin me reconoció, y vencida su repugnancia a tomar los alimentos que le
ofrecíamos, convenciéndose al fin de que no le dábamos animales inmundos
ni horribles manjares, entró en un período de fortalecimiento que
indicaba una enérgica disposición de la naturaleza a recobrar su
primitivo equilibrio y asiento. Badoret cobró sus fuerzas con más
rapidez y a la media hora ya hablaba como una tarabilla arengando a sus
amigos. Para algunos de estos llegó tarde el remedio, y no nos dieron
más trabajo que entregar sus cuerpos a las pobres madres que venían a
recogerlos, después de haberlos buscado inútilmente por toda la ciudad.

---Hermana Siseta ha despertado al fin---me dijo Badoret, tragándose
medio pan.---Yo pensé que íbamos a quedarnos aquí para que se regalaran
con nuestro pellejo Napoleón, Sancir, Agujerón y los demás que andan por
ahí. No estamos todos vivos, Andrés, porque Pauet no resuella, y Sisó,
que estaba tan rabioso contra los cerdos, se ha quedado tieso en la
biblioteca con medio libro en el cuerpo y otro medio en la mano. Así
quisiera yo ver al condenado de D. Pablo Nomdedeu que quiso hacer con
nosotros un guisote. Ya estamos libres de caer al fondo de la cazuela
con sal y agua, y eso de que la señorita Josefina se le almuerce a uno,
no tiene gracia\ldots{} Los marranos están ya dentro de Gerona.
Vaya\ldots{} y decían que D. Mariano no les dejaría entrar. Si es lo que
yo digo\ldots{} mucha facha, mucho boquear, y después nada.

---No desatines, y cuéntame por qué trajisteis aquí a tu hermana.

---Pregúntaselo a D. Pablo y a la señora Sumta. Nosotros le llevamos a
hermana Siseta siete reales que habíamos ganado. Hermana Siseta estaba
llorando con Gasparó en brazos. Un caballero entró en la casa y con
malos modos mandó que enterrásemos al niño. Entonces hermana Siseta le
dio muchos besos y yo le cargué para llevarle a la fosa; pero me daba
lástima y estuve con él a cuestas todo el día, hasta que al fin\ldots{}
Manalet, echaba la tierra y yo la apretaba con las manos para que
quedase bien. Pero luego quisimos volverle a ver, y sacamos la
tierra\ldots{} ¡Ay! Andresillo: después la tornamos a echar, y ya no le
vimos más\ldots{} Al volver a casa, D. Pablo entró suspirando y dando
gemidos, y dijo que traía todos los huesos rotos. Después pidió algo de
comer a la señora Sumta, y la señora Sumta se puso también a echar
suspiros y gemidos. La señorita Josefina, tendida en el suelo, se
chupaba los dedos, D. Pablo empezó a gritar llamando al santo acá y al
santo allá, y luego a todos nos daba con la punta del pie, diciendo:
«Levantaos y salid a buscar algo para mi hija.» Después del entierro
habíamos comprado con los siete reales un pan negro y duro, y se lo
dimos a mi hermana. Si vieras qué ojos le echó D. Pablo. Siseta es más
tonta\ldots{} ¿creerás que no quiso el pan y mandó que se lo diéramos a
la señorita Josefina? Pero yo dije: «sí, para ella está,» y dando la
mitad a Manalet empezamos a comérnoslo. La señora Sumta saltando encima
de mí, me quitó mi parte; pero Manalet se comió toda la suya de un
tragón, atacándosela con los dedos para que le pasara por el gañote.
Entonces, amigo Andrés, el Sr.~Nomdedeu fue arriba y bajando al poco
rato con un gran cuchillo, nos dijo: «Diablillos desvergonzados, puesto
que no servís más que de estorbo, os comeremos.» Yo me reí y Manalet se
puso a temblar y a llorar, pero yo le decía: «no seas burro: primero nos
le comeríamos nosotros a él, si tuviera algo más que huesos. La señora
Sumta sí que está gordita.» Cuando la vieja oyó esto, me amenazó con el
puño, y D. Pablo volvió a decir: «Sí; nos les comeremos, ¿por qué
no?\ldots» Después la señorita Josefina se abrazó a su padre, y este se
puso a llorar soltando lagrimones como balas, y luego la arrullaba en
sus brazos como si ella fuera un chiquillo. ¡Pobre D. Pablo! De veras me
daba lástima\ldots{} Arrullando a su hija le cantaba como a los niños y
después decía: «Señora Sumta, traiga usted una taza de caldo.» Al oír
esto, no podía menos de reírme, y dije: «Pues ya que va a la cocina la
señora Sumta, tráigame a mí un par de perdices porque estoy desganado, y
no quiero más.» Los dos se pusieron furiosos, pero el médico parecía
loco, y todo se le volvía gritar: «Señora Sumta; traiga usted caldo para
mi hija, tráigalo usted pronto o la mato a usted\ldots» ¡Si le hubieras
visto, Andrés! Echaba chispas por los ojos, y con los pelos amarillos
tiesos sobre el casco, parecía nada menos que un demonio\ldots{} En esto
pasaron mis amigos por la calle, llamáronme, yo salí con ellos, y al
poco rato, cuando iba por la calle de Ciudadanos, veo venir a Manalet
corriendo y llorando, que decía: «Hermano Badoret, ven pronto que D.
Pablo nos quiere matar a todos.» Chico, eché a correr con todos mis
amigos hacia casa. ¿Has visto un gato rabioso cómo tira la zarpa, enseña
los dientes, bufa y salta? Pues así estaba D. Pablo. Dejando a su hija
en el suelo, venía hacia nosotros, nos amenazaba con el cuchillo,
golpeaba con el pie a mi hermana, luego parecía querer matarse a él
mismo, y a todo esto gritaba así: «¡Quiero acabar con el género
humano!\ldots» Esto lo dijo muchas, muchísimas veces. Mis amigos estaban
muertos de miedo, y yo cogí unas tenazas para tirárselas a la cabeza.
Pero no me dio tiempo, porque sin soltar su cuchillo salió a la calle
gritando siempre que iba a acabar con todo el género humano, y entonces
Manalet dijo: «Vámonos de aquí y llevémonos a Siseta.» Dicho y hecho:
éramos doce: entre los más grandes cargamos a mi hermana, que estaba
como un cuerpo muerto sin mover ni brazo ni pierna, y la llevamos a la
casa del Canónigo; Manalet, lleno de miedo iba delante chillando: «A
prisa, a prisa, que viene otra vez con el cuchillo\ldots» ¡Ay! Amigo
Andrés, cuando nos vimos en esta casa, respiramos. Luego porque la
pobrecita no estuviera sobre las baldosas del patio la subimos a este
aposento con grandísimo trabajo, poniéndola en la cama donde la ves. La
llamamos, y no nos respondía. Entonces nos ocurrió que debíamos buscarle
algo que comer; pero no hallamos salida más que por los tejados, y antes
nos asparían que pasar otra vez a nuestra casa. Aquí de los apuros,
chico, llegó la noche y nos moríamos de hambre. Pauet y Sisó anduvieron
por los techos comiéndose las yerbas y el musgo que nacen entre las
tejas. Yo bajé a la bodega\ldots{} ni rastro de Napoleón. Se han ido
todos al otro lado del Oñá, corriéndose hacia el campo enemigo\ldots{}
Pues como te iba diciendo, vino después de la noche el día, y después
del día otra noche, y luego amaneció el día de hoy y nosotros sin comer.
Se me olvidaba contarte que oímos caer la bomba en nuestra casa, y yo
dije: «Ahí me las den todas. Si ha cogido a Nomdedeu bien empleado le
está por bruto\ldots» Amigo, desde el tejado nos asomábamos a los patios
de todas las casas de por aquí; llamábamos a la gente para que nos
socorriera; pero no nos hacían caso. Verdad es que muchos de los que
veíamos abajo estaban muertos. Mis amigos se acobardaron ¡pobrecitos!,
como unos gallinas, y Sisó dijo que se iba a comer una de sus manos. Yo
los llevé a la biblioteca, dándoles permiso para que sacaran el vientre
de mal año con los libros, y algunos así fueron tirando. ¡Qué día, qué
noche, Andrés! Mi hermana no nos respondía cuando la llamábamos, y
Manalet me dijo: «Hermano, yo me voy a tirar del tejado a la calle para
traer algo de comida a Siseta\ldots» Estuvimos mirando las rejas y los
balcones para ver si se podía saltar, y por fin Manalet se fue
escurriendo, no sé cómo, sentando los pies en los clavos y las manos en
las rejas, y bajó a la calle por junto a la plaza. Yo bajé también por
donde me viste, y con esto te digo todo, porque ya no hay nada más que
contar.

---Bien, Badoret, veo que acertaste en trasladar aquí a tu hermana, pues
aunque no me parezca cierto, como dijiste, que D. Pablo quisiera
merendarse a tu familia, ese es un hombre a quien la desgracia de su
hija exalta y enfurece, y capaz es de cometer cualquier atrocidad.
Ahora, gracias a Dios, estamos libres de tales horrores, porque el sitio
ha concluido y hay en Gerona víveres abundantes.

Al caer de la tarde, Siseta, sus dos hermanos y los camaradas de estos
que habían escapado a la muerte, no ofrecían cuidado. Al día siguiente
trasladé a mis amiguitos a una casa de la calle de la Barca, donde nos
dieron asilo.

\hypertarget{xxiii}{%
\chapter{XXIII}\label{xxiii}}

Yo no tardé en reponerme, y transcurridos pocos días me presenté a mi
amo don Francisco Satué, quien me dio una malísima noticia.

---Disponte para el viaje---me dijo, dándome uniforme, tahalí y espada,
para que en todo ello comenzase a ejercitar mis altas funciones.

---¿Pues a dónde vamos, mi capitán?

---A Francia, bruto---me respondió con su habitual rudeza.---¿No sabes
que somos prisioneros de guerra? ¿Crees que nos dejan aquí para muestra?

---Señor, yo creí que nadie se metería ya con nosotros.

---Estamos en Gerona como enfermos; pero quieren que vayamos a
convalecer a Perpiñán. Nos detienen tan sólo porque el gobernador no se
halla en situación de poder ser llevado en un carro de municiones.

---¡Ojalá no lo estuviera en cien meses!

---Bárbaro ¿qué dices?---exclamó amenazándome.

---No, mi capitán, no es que yo desee otra cosa que la salud de nuestro
queridísimo gobernador D. Mariano Álvarez de Castro; pero eso de
llevarle a uno a Perpiñán es casi tan malo como lo que hemos pasado.
Pero pues así lo mandan los que pueden más que nosotros, sea, y por mí
no ha de quedar. No a Perpiñán, sino al fin del mundo, iré con mis
jefes, mayormente si llevamos entre nosotros al gran gobernador.

Yo hablaba así, echándomela de bravo; pero en realidad sentía profunda
pena al caer en la cuenta de que era un prisionero de guerra, de cuya
libertad y residencia los franceses disponían a su antojo. ¡Desgraciado
el que en la guerra pone su afición en lugares y personas, que no han de
poder seguir tras él en los frecuentes e inesperados viajes a que
impulsan la victoria o la desdicha!

Cuando fui al lado de Siseta, casi derramando lágrimas me expresé así:

---Prenda mía, ¿ves cuán desgraciado soy?\ldots{} Ahora me llevan a
Francia como prisionero de guerra, con todos los demás militares que
estamos aquí, desde D. Mariano hasta el último ranchero. Si te pudiera
llevar conmigo, Siseta\ldots{} Pero mi capitán, el Sr.~D. Francisco
Satué, es el primer perseguidor de muchachas que hay en toda Cataluña, y
le tengo miedo. Ahora me ocurre, Siseta, que mientras yo tomo el camino
de esa condenada Francia, a quien vería de buena gana comida de lobos,
tú con tus dos hermanos debes marcharte a la Almunia de doña Godina,
donde está mi madre, y esperarme allí cuidándome las haciendas, hasta
que me suelten o Dios disponga de la vida de este pecador.

Siseta me contestó dándome esperanza, y asegurando que convenía aguardar
con serenidad el cumplimiento de nuestro destino, sin desconfiar de la
bienhechora Providencia. Convinimos al fin en que no era una gran
desventura que yo fuese a Francia, y por su parte halló muy prudente
refugiarse en la Almunia, mientras yo volvía. La verdadera dificultad
era la absoluta carencia de medios para vivir dentro de Gerona, lo mismo
que para ausentarse. Éramos pobres hasta el último grado, y después de
pasar tantos y tan penosos trabajos, Siseta y sus hermanos estaban
destinados a sostenerse de la caridad pública. Pero Dios no abandona a
las criaturas desvalidas, y he aquí cómo vino en nuestra ayuda por
inesperados caminos. ¿De qué manera? ¿Cuándo? Esto, los mismos
acontecimientos que voy contando os lo dirán.

Pero déjenme acudir a casa del Sr.~D. Pablo Nomdedeu, de cuya salud me
han dado muy malas noticias al volver de casa del talabartero, donde
llevé el tahalí de mi amo para que le echase una pieza. Déjenme ir allá,
que a pesar de las cuestiones desagradables que tuvimos, no deja de ser
el señor don Pablo un entrañable amigo mío, a quien quiero de todas
veras. Lo malo es que no puedo ir tan pronto como deseara, porque en la
calle de Cort-Real la mucha gente que allí se junta en animados
corrillos, me detiene el paso. ¿Qué ocurre? ¿Tenemos un cuarto sitio? No
es nada; parece que los franceses, cansados de haber cumplido hasta ayer
de mala gana las principales cláusulas de la capitulación, han acordado
solemnemente romperlas. Así me lo dijo el padre Rull, a quien vi muy
sofocado entre el gentío, refiriendo con declamatoria pomposidad los
pormenores del suceso.

---Esto es una desvergüenza---decía,---y un emperador que tales cosas
hace es un pillo\ldots{} nada, un pillo; ¿qué me importa que oigan los
franceses? No bajaré la voz, no, señores. Lo dicho, dicho. En la
capitulación se acordó que los regulares serían respetados, y ahora
salimos con que nos llevan a Francia. ¿Pues qué, las órdenes son cosas
de juego? ¿Somos chicos de escuela, para que hoy se nos diga una cosa y
mañana otra?

---También yo voy a Francia, padre Rull---le dije,---y consolémonos uno
con otro, que frailes y soldados hacen buena miga, y la carga se lleva
mejor en dos hombros que en uno.

---Nada, hijos míos, iremos adonde nos lleven y soportaremos sus
crueldades con paciencia, como nos lo manda Nuestro Señor Jesucristo. Si
así lo habéis querido vosotros, ¿qué se ha de hacer? Ved aquí las
consecuencias de capitular cuando todavía podía haberse tirado una
temporadita más, comiendo lo que había. A Francia, pues, y fíese usted
de palabras de \emph{cerdos}. Nosotros confiábamos ingenuamente en el
cumplimiento de lo pactado, cuando vierais aquí que esta mañana se
presenta en la santa casa un oficialejo, el cual con voces torpes y
destempladas, dijo que nos preparásemos para salir mañana mismo para
Francia, porque S. M. el emperador lo había dispuesto así desde París.
Por lo visto, nos temen tanto como a los soldados. Y díganme ustedes
ahora: ¿qué va a ser de Gerona sin frailes?

Cada uno contestaba al padre Rull, según sus ideas, cuál con enojo, cuál
festivamente; pero al fin todos los que le oímos, convinimos en que lo
del viaje era una grandísima picardía de S. M. el emperador de los
franceses. Cuando me retiré de allí, quedaba el buen fraile sermoneando
a sus amigos sobre la preeminencia que siempre alcanzaron las órdenes
religiosas en los tratados de las naciones.

Llegué a casa del Sr.~Nomdedeu, y desde mi entrada conocí que la salud
del buen médico no debía de ser buena, por las señales de consternación
que noté en el semblante de Josefina lo mismo que en el de la señora
Sumta. Esta me dijo:

---Andresillo, no hables al amo de Siseta ni de los chicos, porque
siempre que se le nombran, le da uno al modo de desmayo.

Josefina me preguntó por los míos, y al instante le comuniqué con la
alegría de mis ojos el infeliz encuentro de mi novia y sus hermanos.

---Todos se salvan, menos mi buen padre---dijo tristemente la muchacha.

Al instante entré a ver al enfermo, quien me recibió con su habitual
bondad. Junto a su lecho estaba un hombre en quien reconocí a uno de los
escribanos de Gerona.

Indudablemente D. Pablo iba a hacer testamento. Su aspecto y figura no
podían ser más tristes, al punto se echaba de ver que aquella lámpara
tenía ya muy poco aceite. La postrimera luz brillaba, sí, como próxima a
extinguirse, con viva claridad, y la irregular llama, tan pronto grande
como chica, espantaba con sus oscilaciones deslumbradoras. Unas veces el
espíritu del buen doctor se empequeñecía con extraordinario
aplanamiento; otras se agrandaba, tomando proporciones superiores a las
de la vida común: y con este variar angustioso, síntoma de todo fuego
que se apaga luchando entre la combustión y la muerte, la lengua del
médico pasaba de un mutismo invencible a una locuacidad mareante.

Cuando entré, respondió a mis cariñosas preguntas con monosílabos, que
salían difícilmente de su sofocado pecho; pero al poco rato se fue
despabilando en términos, que a ninguno de los presentes nos dejaba
meter baza, y él se lo decía todo sin mostrarse cansado.

---¿Con que aseguras tú que no moriré? Ilusión, amigo mío, ilusión de tu
buen deseo. Dios me ha leído ya la sentencia y en esto no hay ni puede
haber duda alguna. Yo cumplí mi misión, ahora estoy demás.

---Señor, anímese usted---exclamé fingiendo entusiasmarme.---Pues qué,
¿ahora que Gerona está libre de hambres y muertes, se ha de ir el hombre
mejor de toda la ciudad? Levántese de esa cama y vamos por ahí a ver las
murallas rotas, los fuertes deshechos, las casas arruinadas, testigos de
tanto heroísmo. Fuera pereza. Eso no es más que pereza, señor don Pablo.

---Pereza es, sí; pero la pereza última y definitiva, aquella del
viajero que habiendo andado toda la jornada, se arroja sin aliento en el
camino, convencido de que no puede más. Pereza es, sí, la mejor de
todas, porque lleva al más dulce, al más placentero de los sueños, la
muerte. ¡Ay, qué postrado me siento! Pues qué, ¿era posible que después
de tan colosales esfuerzos en lo físico y en lo moral, siguiese yo
viviendo? No una vida como la mía, sino cien robustas y vigorosas
habríanse consumido en esta lucha con la naturaleza, que yo sostuve
durante tanto tiempo; porque decirte, Andrés, el sin número de
dificultades que vencí, sería el cuento de nunca acabar. Baste referirte
que en pocos días, busque, fomenté y desarrollé en mí cualidades que no
tenía; en pocos días, trasformado hasta lo sumo, encontreme con
sentimientos y pasiones que antes no tenía, y todo fue como si una serie
de hombres diversos se desarrollaran dentro de mí propio. Yo estoy
asombrado de lo que hice, y ahora comprendo qué inmenso tesoro de
recursos tiene el hombre en sí, si sabe explotarlo. Al fin, Andrés, mi
pobre hija alargó sus días hasta el fin del cerco, y cuando los sanos y
robustos sucumbieron, ella, enferma y endeble se ha salvado. He aquí
premiada dignamente mi amorosa solicitud y mis colosales esfuerzos. Esta
tierna niña, que es todo mi amor, está hoy delante de mí alegrando mi
vista y mi alma con el color de sus mejillas. Basta este espectáculo a
consolarme de todas mis penas, y si me entristece la muerte es porque mi
hija y yo nos separamos ahora. Dios lo permite así, porque ya ella no
necesita de mis constantes cuidados, y la savia vital que milagrosamente
ha adquirido le dará bríos para subsistir por sí sola, sin el apoyo de
estas manos fatigadas, que reclama la tierra, ansiosa de carne.

---Sr.~D. Pablo---le dije dominando mi melancolía,---deseche usted esos
tristes pensamientos, que son la primera y única causa de su mal; mande
a la señora Sumta que traiga y aderece un par de chuletas, que ya las
hay buenas en Gerona, sin ser de gato ni de ratón, y cómaselas en paz y
gracia de Dios, con lo cual, o mucho me engaño, o no habrá muerte que le
entre en largos años.

---Esto no va con chuletas, amigo Andrés. Mi cuerpo rechaza todo
alimento, y no quiere más que morirse. Está echando a voces el alma,
increpándola para que se vaya fuera de una vez.

---Más consumidos y extenuados estaban otros, y sin embargo han vivido,
y por ahí andan hechos unos robles. Y si no, ahí tenemos el ejemplo de
Siseta, a quien dimos todos por muerta, y viva y sana está, gracias a
Dios.

---¿Vive Siseta?---preguntó Nomdedeu con profundo interés y cierta
exaltación que no pudo disimular.

---Sí, señor; tan viva está como sus dos hermanos.

---¿Estás seguro de ello?

---Segurísimo.

---¿Y no tiene heridas en su cuerpo gentil, ni golpes en su cabeza, ni
rasguños en su piel, ni le falta brazo, pierna, dedo u otra parte alguna
de su estimable persona?

---No, señor, nada le falta---repuse jovialmente,---o al menos no tengo
yo noticia de ello.

---¿Y los muchachos, aquellos juguetones y traviesos muchachos, están
vivos y sanos?

---También, señor doctor, y todos muy deseosos de venir a ofrecer a
usted sus respetos con la cortesía que les es propia, saltando y
chillando.

---¡Oh, loado sea Dios!---exclamó con cierto arrobamiento contemplativo
el infortunado doctor.

Dicho esto, permaneció un rato meditando u orando, que ambas funciones
podían deducirse de su recogida y silenciosa actitud, y luego
reposadamente me habló así:

---Me has proporcionado indecible consuelo al darme noticias tan
lisonjeras de la familia del Sr.~Mongat, porque me atormentaba la
sospecha, el recelo, más que sospecha y recelo, la terrible certidumbre
de que yo había ocasionado un gran mal a esos muchachos y a su bondadosa
hermanita, cuando después del lamentable accidente del pedazo de azúcar,
entré en casa de Siseta. Mi hija iba a morir de inanición. Yo pedía a la
señora Sumta que nos diera algo de comer, y la señora Sumta no nos daba
nada. Yo pedí a Dios que me enviase algo del cielo, y Dios tampoco
quería enviarme nada. Siseta estaba allí; sus hermanos entraron haciendo
ruido, y la insolente vitalidad que revelaban sus ágiles cuerpos
despertó en mi alma un sentimiento que no te podré pintar, aunque por
espacio de cien años te hable y agote todos los recursos de todas las
lenguas conocidas. No: aquel sentimiento es una anomalía horrorosa en el
ser humano, y sólo es posible que exista durante cortísimos intervalos
en días que muy rara vez contará el tiempo en su infinita marcha. Yo
miraba a los chicos, yo miraba a su hermana, y sentía un insaciable y
sofocante anhelo de hacerlos desaparecer de entre los seres vivientes.
¿Por qué, amigo mío? Esto sí que no sabré decírtelo, porque yo mismo no
lo entiendo. No creas que conturbaba mi cerebro el repugnante instinto
de la antropofagia: no, no es nada de eso. Era un sentimiento del linaje
de la envidia, Andrés; pero mucho, muchísimo más fuerte; era el egoísmo
llevado al extremo de preferir la conservación propia a la existencia de
todo el resto de la humana familia; era una aspiración brutal a aislarme
en el centro del planeta devastado, arrojando a todos los demás al
abismo, para quedarme solo con mi hija; era un vivísimo deseo de cortar
todas las manos que quisieran asirse a la tabla en que los dos
flotábamos sobre las embravecidas olas. Pintar todo lo que yo odié en
aquel momento a los dos hermanos y a la pobre muchacha, sería más
difícil que pintarte los horrores del infierno, abrazando lo grande y lo
pequeño, el conjunto y los pormenores de la mansión donde el hombre
impenitente expía sus culpas. Cada inhalación de su aliento al respirar,
me parecía un robo; cada átomo de aire que entraba en sus pulmones, un
tesoro arrancado al conjunto de elementos vitales que yo quería reunir
en torno mío y de mi hija. Los malditos se repartían un pedazo de pan,
un pedacito de pan, Andrés, amasado con todo el trigo y con toda el agua
de la creación, para mi regalo. En aquella crisis del egoísmo, yo no
comprendía que el Universo con sus mil mundos, su fauna y su flora, sus
inagotables recursos y prodigios existiese para nadie más que para
Josefina y para mí.

Detúvose el doctor fatigado, y yo, queriendo apartar de su mente ideas
que le hacían más daño que el mal físico, le dije:

---Mande usted a paseo, Sr.~D. Pablo, esas vanas imaginaciones que le
están secando el cerebro. Siseta y sus hermanos están buenos, amigo, y
yo le aseguro a usted que no se los ha comido. ¿A qué pensar más en eso?

---Calla, Andrés, y déjame seguir---dijo reposadamente.---No son vanas
imaginaciones lo que cuento, pues lo que yo sentía real existencia tuvo
dentro de mí. Me faltaba decirte que reconocí la horrible metamorfosis
de mi espíritu, pues no puedo darle otro nombre, y me decía: «No, yo no
soy yo. Dios mío, ¿por qué has consentido que yo sea otro?»
Efectivamente, yo no era yo. ¡Qué horrorosas lobregueces rodeaban los
ojos de mi espíritu así como los de mi cuerpo!\ldots{} Aquellos
condenados muchachos estaban comiendo, Andrés; llevaban a la boca unos
pedazos de pan, y delante de mí, tenían la audacia de ofrecer una parte
a su hermana. ¡Cómo quieres tú que esto viera impasiblemente quien
dentro tenía difundidos por su sangre y haciendo cabriolas en las
sutiles cuerdas de sus nervios los millares de demonios que yo llevaba
conmigo! ¡Al ver cómo mordían con sus insolentes dientecillos; al verles
tragar con tanta desvergüenza, duplicose en mí el furor contra ellos y
les increpé, diciéndoles no estar dispuesto a consentir que nadie
viviese delante de mí! Andrés amigo, Andrés de mi corazón; yo tomé un
cuchillo y lo esgrimía, como quien intenta matar moscas a estocadas;
corría hacia ellos, corría hacia Siseta y la señora Sumta; pero en mi
salvaje insensatez no me faltaba un pensamiento humano que me detuviese
en los arranques brutales de aquel desbordado apetito de matar. Los
chicos, que de improviso salieron, regresaron con otros de su edad, y
sus chillidos y provocativas risas me enardecieron más. Desde entonces
mis ojos nublados no vieron más que sangrientos objetos; entrome un
delirio salvaje, durante el cual sentía detestable complacencia en herir
acaso en el vacío, descargando golpes a todos lados contra cuerpos que
me rodeaban y azuzaban sin cesar. Creo que después de dar vueltas por la
casa, salí a la calle, y mi brazo vengativo iba destruyendo en
imaginarios cuerpos a toda la familia humana. Hablaba mil inconexos
desatinos; contemplaba con gozo a los que creía mis víctimas; buscaba la
soledad, insultando a cuantos se me ofrecían al paso; pero la soledad no
llegaba nunca, pues de cada víctima surgían nuevos cuerpos vivos que me
disputaban el aire respirable, la luz y cuantos tesoros de vida
hermosean y enriquecen el vasto mundo\ldots{} No sé qué habría sido de
mí si unos frailes no me hubieran sujetado en la calle de Ciudadanos,
llevándome a cuestas largo trecho. ¡Ay, amigo mío! En mi cerebro, que
era una masa de bullidoras burbujas, cual si hirviera puesto al fuego,
retumbaron estas palabras: «Es lástima que el Sr.~Nomdedeu se haya
vuelto loco.» Y al recoger esta idea, mi alma pareció disponerse a
recobrar su perdido asiento. Luego los frailes dijeron: «Démosle un poco
de estas lonjas de cuero de sillón que hemos cocido, a ver si se
repone\ldots» Les pregunté por mi hija, y respondiéronme que no tenían
noticia de las hijas de nadie. Encontreme con un poco de fuerza regular,
no exaltada y anómala como la que me había impulsado a tantos
disparates, y quise marchar a mi casa\ldots{} Caí al suelo\ldots{} perdí
el cuchillo\ldots{} una monja me ofreció su brazo y llegué a mi casa. Ni
Siseta, ni sus hermanos, ni Josefina, ni la señora Sumta estaban ya
allí. Las monjas me dieron un poco de corcho frito que no pude comer, y
les pregunté por mi hija. Todo lo que había pasado se me presentó como
los recuerdos de un sueño, pero aunque adquirí el convencimiento de no
haber extinguido todo el linaje de los nacidos, no estaba seguro de la
invulnerabilidad de mis ciegos golpes. «Yo he matado algo,» me dije para
mí; y esta idea me causaba hondísima pena. Me reconocía como yo mismo
exclamando: «Pablo Nomdedeu, ¿fuiste tú quien tal hizo?»

---Basta ya, amigo mío---dije interrumpiéndole, al advertir que los
recuerdos de sus locuras empeoraban al buen doctor.---Más adelante nos
contará usted tan curiosas novedades. Ahora procure descabezar un sueño,
entre tanto que la señora Sumta adereza las chuletas consabidas.

---Calla, Andrés, y no quieras gobernar en mí---repuso.---Yo dormiré
cuando lo tenga por conveniente. Déjame concluir, que ya no falta mucho.
Los enfermeros del hospital fueron los que me proporcionaron algún
alimento que se podía comer, con lo cual me encontré relativamente bien,
y pude salir en busca de mi hija. Ya sabes cómo la encontré al fin, y lo
que le aconteció. Por mi parte, hijo, yo mismo, después de la horrorosa
crisis que había pasado, me espantaba de verme asistiendo enfermos que
sin duda lo estaban menos que yo, y heridos que no tenían llagas tan
terribles en su cuerpo como la que yo tenía en mi alma. ¡Ay, Andrés!
Nomdedeu estaba herido de muerte. Las penas sufridas con tanta paciencia
desde mayo me han labrado este profundo mal que ahora siento y que me
llevará dentro de poco al seno de Dios. Me admiro de haber resistido
tanto, y digo que tuve fuerza de cien hombres. No, uno solo es incapaz
de tanto. D. Mariano Álvarez tenía para resistir el estímulo de la
gloria y del agradecimiento patrio; yo no he tenido ante mí sino
espectáculos lastimosos y un porvenir oscuro. El esfuerzo ha sido
grande; la tensión inmensa; por eso la cuerda se ha roto, y me voy, me
voy, hija mía, Andrés, señora Sumta y demás presentes. Bastante he
hecho. El que crea haber hecho más, que levante el dedo.

Josefina y la señora Sumta lloraban, y yo cuando el enfermo calló,
procuraba consolarle con tiernas palabras. Poco más tarde fueron a verle
Siseta y sus hermanos, con cuya visita pareció muy complacido el
enfermo, y a todos prodigó cariños y congratulaciones, obsequiándoles
con una excelente comida. Después se durmió, y al caer de la noche, hora
en que por encargo suyo, volvió el escribano, acompañado de tres
personas de la intimidad de D. Pablo; este nos llamó a todos diciendo
que iba a dictar su testamento, el cual hizo en regla, nombrando por
heredera de casi todos sus bienes a su hija Josefina, con una cláusula,
sobre la cual debo llamar a ustedes la atención, para que conozcan la
generosidad de aquel ejemplar sujeto. Además de que el doctor dejaba a
Siseta y a sus hermanos los veinticuatro alcornoques que tenía en la
parte de Olot, dispuso que en caso de morir sin sucesión la señorita
Josefina, pasase el total de los bienes a Siseta y a sus hermanos,
recomendando a aquella y a esta que viviesen juntos para perpetuar la
amistad y buenos servicios de que la infeliz enferma había sido objeto
por parte de los míos durante el sitio. La fortuna del doctor era harto
exigua, pues la finca de Castellà, devastada por los franceses, valía
bien poco, y lo demás consistía en diversos grupos de alcornoques
diseminados por la comarca ampurdanesa y en sitios a los cuales los
herederos no se aventurarían a emprender viaje por saber el corcho de
que eran dueños. También a mí y a la señora Sumta nos dejó varias
mandas, aunque la mía más era honorífica que de provecho, por consistir
en el Diario de las peripecias del sitio, redactado de puño y letra por
el mismo doctor. El ama de gobierno pescó todos los muebles y ropas que
de la casa pudieron salvarse.

Luego que el testamento fue hecho, administraron al enfermo el Santo
Viático, y cumplida esta ceremonia, quedose Nomdedeu muy postrado,
hablando poco y con dificultad, mirándonos a ratos con estúpido asombro
y cerrando después los ojos para entregarse a un inquieto sueño.
Exceptuando Manalet, que se durmió en el suelo, todos velamos,
dispuestos a asistirle con la mayor solicitud y esmero; pero el infeliz
D. Pablo no necesitó largo tiempo de nuestra asistencia. Cerca de la
madrugada, abrió los ojos, llamó a su hija, y abrazándola tiernamente le
habló así:

---¿Te quedas tú, hija mía? ¿Te quedas aquí cuando yo me voy? ¿De modo
que no te veré más? Entonces toda la eternidad será infierno para
mí\ldots{} Josefina, ven, sígueme, ponte el manto que nos vamos. Mi hija
no se apartará de mí ni un solo momento\ldots{} Después de pasar juntos
las grandes penas, ¿hemos de separarnos cuando todo ha concluido? No,
Josefina. Vámonos juntos o nos quedaremos aquí en Castellà. Paseemos por
nuestra huerta viendo cómo van saliendo los pepinos, y no nos cuidemos
de lo que pasa en Gerona. Mira qué tomates, hija, y observa cómo van
tomando color esos pimientos\ldots{} ¿Ves? Por ahí viene la señora
Pintada pavoneándose con sus diez y ocho pollos: entre ellos hay seis
patitos, que son los más guapos, los más salados y los más monos de
todos. Llegan al estanque, y sin que la madre pueda impedirlo con
cacareadas amonestaciones\ldots{} ¡zas!, al agua todos. Mira cómo se
asusta la señora Pintada y los llama. Pero ellos\ldots{} sí, que si
quieres\ldots{} Hija mía, los perales no pueden con más peras: algunas
están maduras. ¿Pues y los melocotones? Me parece que la cabra ha
mordido en las matas de estas remolachas\ldots{} ¡pero quia! ¡si es
Dioscórides, el burro de nostramo Mansió! Míralo, allí está haciendo de
las suyas. ¡Eh, fuera! Le llamo Dioscórides por lo grave y sesudo. El
gran sabio de la antigüedad me perdone\ldots{} ¿Has visto las palomas,
Josefina? Veamos si anoche se han comido también las ratas algunos
huevos de los que aquellas están sacando\ldots{} ¡Eh, nostramo Mansió,
que Dioscórides se come la huerta! Amárrelo usted\ldots{} El pobre
hortelano no me oye\ldots{} ¡Qué ha de oír si está limpiándole las babas
a su nieta? Ven acá, Pauleta, toma la mano de Josefina, y vamos a
ordeñar la vaca. ¡Qué hermoso está el ternerillo! No acercarse mucho,
que el otro día dio una cornada a nostramo\ldots{} A ver, Josefina, trae
el cántaro. Mansió dice que yo no sé hacer esta maniobra, y yo le
desafío a él y a todos los nostramos de la comarca a que hagan mejor que
yo esta operación del ordeñar. No temas, Esmeralda, no te hago daño,
pisch, pisch\ldots{} Esta atmósfera del establo te sienta muy bien,
hija, y a mí me agrada en extremo\ldots{} Ya viene tranquila, dulce,
grave, amorosa y callada la incomparable noche, en cuyo seno tan bien
reposa mi alma. ¿Oyes las ranas, que empiezan a saludarse diciéndose:
\emph{¿Cómo estáis? Bien, ¿y vos?} ¿Oyes los grillos disputando esta
noche sobre el mismo tema de anoche? ¿Oyes el misterioso disílabo del
cuco, que parece la imagen musical más perfecta de la serenidad del
espíritu? Ya vienen los labradores del trabajo. ¡Con qué gusto alargan
los bueyes su hocico adivinando la proximidad del establo! Oye los
cantos de esos gañanes y de esos chicos, que vuelven hambrientos a la
cabaña. Ahí los tienes. Mira cómo rodean a la abuela, que ya ha puesto
el puchero a la lumbre. El humo de los techos formando esbeltas columnas
sobre el cielo azul, discurre luego y vaporosamente se extiende a
impulsos del suave viento que viene de la montaña a jugar en las copas
de estos verdes olmos, de estas oscuras encinas, de estos lánguidos
sauces, de estos flacos chopos, cuyas charoladas hojas brillan con las
últimas luces de la tarde\ldots{} La oscuridad avanza poco a poco, y el
cielo profundo ofrece sobre nuestras cabezas un tranquilo mar al revés,
por cuyo diáfano cristal en vano tratamos de lanzar la vista para
distinguir el fondo. ¡Oh!, quedémonos aquí, hija mía, y no nos separemos
ni salgamos más de este lugar delicioso. Todo está tranquilo: los
cencerros de las ovejas suenan con grave música a lo lejos; el cuco, el
grillo y la rana no han acabado aún de poner en claro la cuestión que
les tiene tan declamadores. El viento cesa también, cierra los ojos,
extiende los brazos y se duerme. Ya no humean los techos; Esmeralda se
echa sobre la fresca yerba, y su hijo, abrigándose junto a ella,
hociquea buscando en el seno materno lo que nosotros hemos dejado.
Nostramo Mansió duerme también, y Dioscórides, escondiendo el ojo
brillante bajo la negra ceja, sumerge el cerebro en profundo sopor. Las
palomas han dejado de arrullarse, los conejos se esconden en sus
guaridas, meten los pájaros bajo el ala la inteligente cabeza, y la
señora Pintada se retira pausadamente al corral con sus diez y ocho
hijos, incluso los patos, que van dejando en el suelo la huella de sus
palmas mojadas. El mundo reposa, hija; reposemos nosotros también. El
cielo está oscuro. Todo está oscuro, y no se ve nada. Mi espíritu y el
tuyo anhelaban ha tiempo esta profunda tranquilidad por nadie ni por
nada turbada. Reposemos; no hay sol ni luna en el cielo, y sólo el
lucero nos envía una luz que viene recta hasta nosotros como un hilo de
plata. Míralo, Josefina, y descansa tu frente en mi hombro. Yo reposaré
mi cabeza sobre la tuya, y así nos dormiremos apoyados el uno en el
otro. Todo ha callado y no se ve más que el lucero\ldots{} ¿lo ves?

Después de esto, nada más dijo en este mundo el Sr.~Nomdedeu.

Algún tiempo después de expirar, nos costó gran trabajo desasir de los
brazos helados del doctor a su desconsolada hija, cuyo estado era tan
lastimoso que daba ocasión a augurar una segunda catástrofe.

\hypertarget{xxiv}{%
\chapter{XXIV}\label{xxiv}}

Adiós, señores; me voy a Francia, me llevan. Los sucesos que he referido
habíanme hecho olvidar que era prisionero de guerra, como los demás
defensores de la plaza, y era forzoso partir. Solamente en razón de mi
enfermedad me fue permitido, como a otros muchos, el permanecer allí
desde el 10 hasta el 21, de modo que con el mal acababa la dulce
libertad.

Adiós, señores; me voy, adiós, pues tanta prisa me daba aquella canalla,
que no digo para despedirme de mis caros oyentes, pero ni aun para
abrazar a Siseta y sus hermanos me alcanzaba el breve tiempo de que
disponía. Notificada la marcha, nos señalaron hora, nos recogieron y
haciéndonos formar en fila, camina que caminarás a Francia. Los castigos
impuestos por contravenir el programa de circunspección que nos habían
recomendado, eran: la pena de muerte para el conato de fuga, cincuenta
palos por hablar mal de José Botellas, cantar el \emph{dígasme tú
Girona}, o nombrar a D. Mariano Álvarez.---Adiós, Siseta, adiós, Badoret
y Manalet, cara esposa y hermanitos míos. Cuidado con lo que os he
advertido. El prisionero os escribirá desde Francia, si antes no logra
burlar la vigilancia de sus crueles carceleros. Adiós. No os mováis de
aquí, mientras yo no os lo mande, ni penséis por ahora en tomar posesión
de vuestros alcornoques, que eso y mucho más se hará más adelante.
Acompañad a la desgraciada hija del gran D. Pablo, y alegrad sus tristes
horas. Adiós, dad otro abrazo a Andrés Marijuán, a quien llevan preso a
Francia por haber defendido la patria. Tengo confianza en Dios, y el
corazón me dice que no he de dejar los huesos en la tierra de los
cerdos. Ánimo: no lloréis, que el que ha escapado de las balas, también
escapará de las prisiones; y sobre todo no es de personas valerosas el
lagrimear tanto por un viaje de pocos días. Salud es lo que importa, que
libertad\ldots{} ella sola se viene por sus pasos contados, sin que
nadie lo pueda impedir. Adiós, adiós.

Así les hablaba yo al despedirme, y por cierto que carecía completamente
del ánimo y entereza que a los demás recomendaba, faltándome poco para
dar al traste con mi seriedad; pero convenía en aquella ocasión
echármela de hombre de bronce. Mi gravedad era ficticia y no hay
heroísmo más difícil que aquel que yo intentaba al despedirme de Siseta
y sus hermanos. La verdad es que tenía el corazón oprimido como si mano
gigantesca me lo estrujara para sacarle todo su jugo.

Siseta se quedó en la calle de la Neu, agobiada por su profunda
aflicción. Badoret y Manalet me acompañaron hasta más allá de Pedret, y
no fueron más adelante porque se lo prohibí, temiendo que con la
oscuridad de la noche se extraviaran al regresar. Salimos, pues, en la
noche del 21. Delante iba rodeado de gendarmes a caballo el coche en que
llevaban a D. Mariano Álvarez: seguían los oficiales, entre los cuales
estaba mi amo, y dos o tres asistentes completábamos el primer grupo de
la comitiva. Más atrás marchaba toda la clase de tropa, soldados
convalecientes de heridas o de epidemia en su mayor parte. La procesión
no podía ser más lúgubre, y el coche del gobernador rodaba
despaciosamente. No se oía más que lengua francesa, que hablaban en voz
alta y alegre nuestros carceleros. Los españoles íbamos mudos y tristes.

Hicimos alto en Sarrià, donde se nos agregaron los frailes que habían
salido antes que nosotros con el mismo destino, y con sus paternidades a
la cabeza nada faltó para que la comitiva pareciese un jubileo. Daba
lástima verlos, porque si entre ellos había jóvenes robustos y recios
que resistían el rigor de la penosa jornada, no faltaban ancianos
encorvados y débiles que apenas podían dar un paso. La gendarmería los
arreaba sin piedad, y lo más que se les concedió fue que alguno de
nosotros les ofreciera apoyo llevándolos del brazo. El padre Rull
sofocaba su impetuosa cólera, y marchando delante de todos con resuelto
paso, revolvía sin duda en su mente proyectos de venganza. Los legos,
que cargaban repletas alforjas, repartían graciosamente en cada descanso
raciones de pan, queso, frutas secas y algún vino, de lo cual algo se
rodaba siempre hacia la parte seglar de la caravana, aunque no mucho.
Algunos gendarmes franceses, más humanos que sus jefes, también nos
ofrecían no poca parte de sus víveres.

De este modo llegamos a Figueras a las tres de la tarde del 22, y sin
permitirle descanso alguno, fue el gobernador enviado al castillo de San
Fernando. Frailes y soldados quedaron en el pueblo, y solamente subimos
con aquel los del servicio del propio general o de sus ayudantes.
Marchamos todos tras el coche, y al llegar dentro de la fortaleza, la
debilidad de D. Mariano era tal, que tuvimos que sacarle en brazos para
trasportarle de la misma manera al pabellón que le habían destinado, el
cual era un desnudo y destartalado cuartucho sin muebles. Entró el héroe
con resignación en aquella pieza, y echose sin pronunciar queja alguna
sobre las tablas, que a manera de cama le destinaron. Los que tal
veíamos, estábamos indignados, no comprendiendo tan baja e innoble
crueldad en militares hechos ya de antiguo a tratar enemigos vencidos y
rivales poderosos, pero callábamos por no irritar más a los verdugos,
que parecían disputarse cuál trataba peor a la víctima. Luego que se
instaló, trajeron al enfermo una repugnante comida, igual al rancho de
los soldados de la guarnición; pero Álvarez, calenturiento, extenuado,
moribundo, no quiso ni aun probarla. De nada nos valió pedir para él
alimentos de enfermo, pues nos contestaron bruscamente que allí no había
nada mejor, y que si durante el cerco habíamos sido tan sobrios,
comiésemos entonces lo que había.

Con la resignación y entereza propias de su grande alma, resistió
Álvarez estas miserias y bajas venganzas de sus carceleros; y sólo le
vimos inmutado cuando el gobernador del castillo, que era un soldadote
de mediana graduación, brusco, fatuo y muy soplado, empezó a dirigirle
impertinentes preguntas. La insolencia de aquella canalla nos tenía
ciegos de ira, pues no sólo el gobernador de la plaza, sino oficialejos
de la última escala, se atrevían a hacer preguntas tontas e importunas a
nuestro héroe, que ni siquiera les hacía el honor de mirarles.

Las preguntas eran no sólo contrarias a la cortesía, sino al espíritu
militar, pues en todas ellas se le pedía cuenta a nuestro jefe del gran
crimen de haber defendido hasta la desesperación la ciudad que el
gobierno de su patria le había confiado. No parecían militares los que
con insultos y burlas groseras mortificaban al hombre de más temple que
en todo tiempo se pusiera delante de sus armas. Álvarez, siempre
caballero aun en presencia de gente de tal ralea, les respondió
sencillamente:\emph{---Si ustedes son hombres de honor, hubieran hecho
lo mismo en mi lugar.---} Tan sublime concepto no lo comprendían la
mayor parte, y solamente algunos oficiales distinguidos, penetrándose
del indigno papel que estaban haciendo, se apresuraron después de la
respuesta del general, a poner fin al denigrante interrogatorio.

Mi amo enviome al instante al pueblo en busca de carne para aderezar la
comida del enfermo, y gracias a mi prontitud y diligencia, pronto
pudimos servirle una comida mediana. Delante de los franceses, que nos
negaban todo auxilio, Satué puso el puchero, soplaba el fuego otro
oficial español, y convertidos todos en cocineros, nos disputábamos
chicos y grandes el honor de asistir al enfermo. Pasó bien la noche;
pero serían las dos de la madrugada, cuando con estrépito llamaron a la
puerta del pabellón, diciéndonos que nos dispusiéramos a seguir el viaje
a Francia. Álvarez, que dormía profundamente, despertó al ruido, y
enterado de la continuación de la jornada, dijo sencillamente:---Vamos
allá.---Quiso incorporarse sobre las tablas en que con nuestros capotes
le habíamos arreglado un mal lecho, y no pudo\ldots{} ¡Tan agotadas
estaban sus fuerzas!\ldots{} Pero en brazos le llevamos nosotros al
coche, y con un frío espantoso, azotados por la lluvia de hielo y
pisando la nieve que cubría el camino, emprendimos el de la Junquera.
Una precaución ridícula habían añadido los franceses a las que antes
tomaran para custodiarnos. Esto hace reír, señores. Además de la fuerte
escolta de caballos, sacaron también de Figueras dos piezas de
artillería, que iban detrás de nosotros, amenazándonos constantemente.
Es que su recelo de que nos escapásemos era vivísimo, y con ninguna de
las cautelas ordinarias creían segura la persona de D. Mariano Álvarez,
inválido y casi moribundo. Éramos muy pocos en aquella segunda jornada,
porque los frailes y la tropa quedáronse en Figueras hasta el amanecer.
Ignoro si para tener a raya las fogosidades del padre Rull, se
pertrecharon también con un par de baterías de campaña y algunos
regimientos de línea.

En la Junquera nos detuvimos muy poco tiempo; siguiendo luego por
Francia adelante, llegamos a Perpiñán a las siete de la noche del mismo
día 23, y después de detenemos en casa del gobernador, nos llevaron al
Castillet, fortaleza de ladrillo, de airosa vista, obra del rey D.
Sancho, la cual habrán visto cuantos hayan estado en aquella ciudad. Sin
más ceremonias, destinaron para habitación de Álvarez un tenebroso
aposento a manera de calabozo, con más humedades que muebles, y tan
inmundo y sucio, que el mismo D. Mariano, a pesar de su temple resignado
y fuerte, no pudo contenerse y exclamó con indignación: \emph{¿Es este
sitio propio para vivienda de un general? ¿Y son ustedes los que se
precian de guerreros?} El alcaide, que era un bárbaro, alzó los hombros,
pronunciando algunas palabrotas francesas, que me pareció querían decir
poco más o menos: «es preciso tener paciencia.» Luego, dirigiéndose a
los de la comitiva, aquel caritativo personaje nos dijo que estaba
dispuesto a darnos de comer lo que quisiéramos, pagándolo previamente en
buena moneda española. La moneda española ha sido siempre muy bien
recibida en todo país donde ha habido manos. Dándole las gracias,
pedímosle lo que nos pareció más necesario, y aguardamos la cena,
aposentados todos en la inmunda pocilga. Nuestro primer cuidado fue
improvisar con los capotes una cama para el gobernador, cuya fatiga y
debilidad iban siempre en aumento. El cancerbero volvió al poco rato con
unos manjares tan mal guisados, que no se podían comer, lo cual no fue
parte a impedir que nos los cobrase a peso de oro; pero se los pagamos
con gusto, suplicándole, unos en mal francés y otros en castellano, que
nos hiciera el favor de no honrarnos más con su interesante presencia.

Pero él o no entendió o quiso mostrarnos todo el peso de su
impertinencia, y a cada cuarto de hora venía a visitarnos, poniéndonos
ante los ojos, que en vano querían dormir, la luz de una deslumbradora
linterna. Esto mortificaba a todos; pero principalmente al enfermo, que
por su estado necesitaba reposo y sueño, y así se lo dijimos al alcaide,
añadiéndole que como no pensábamos fugarnos, podía eximirnos de sus
repetidos reconocimientos. Él nos respondía con amenazas soeces;
quedábamos luego a oscuras, nos vencía el dulce sueño; pero no habíamos
trasportado los umbrales de esta rica y apacible residencia del
espíritu, cuando la luz de la linterna volvía a encandilar nuestros
ojos, y el alcaide nos tocaba el cuerpo con su pata para cerciorarse por
la vista y el tacto de que estábamos allí.

Satué, furioso y fuera de sí, me dijo en uno de los pequeños intervalos
en que estábamos solos: «Si ese bestia vuelve con la linterna, se la
estrello en la cabeza.» Pero D. Mariano, calmó su arrebato, condenando
una imprudencia que podía ser a todos funestísima. La noche fue por
tanto, y merced a las visitas del alcaide, penosa y horrible. Por la
mañana nos hizo el honor de visitarnos el comandante de la plaza, el
cual habló largamente con Álvarez, tratándole con cierta benevolencia
cortés que nos agradó; mas luego hizo recaer la conversación sobre un
suceso de que no teníamos noticia y allí dio rienda suelta a las
groserías y los insultos. Parece que algunos oficiales de los
trasladados a Francia inmediatamente después de la rendición de Gerona,
se habían fugado, en lo cual obraron cuerdamente, si padecieron el
martirio de la linterna del señor alcaide. Al hablar de esto, el
comandante les prodigó delante de nosotros vocablos harto denigrantes,
añadiendo: «Pero por fortuna, hemos pescado a once de los prófugos, y
han sido arcabuceados hace dos días. Buscamos a los demás.»

Álvarez se sonrió y dijo: \emph{¿Con que volaron, eh?}\ldots{} y en su
rostro por un instante dibujose ligera expresión festiva. A pesar de que
el comandante de Perpiñán no era hombre de mieles, prometió a Álvarez
dejarle descansar todo aquel día, poniendo freno a las importunidades de
la candileja, y nos dispusimos para dormir; pero ¡ay!, estábamos
destinados a nuevos tormentos, entre los cuales el mayor era presenciar
cómo padecía en silencio sin hallar alivio en sus males ni piedad en los
hombres, el más fuerte y digno de los españoles de aquel tiempo;
estábamos entre gente que hacía punto de honra el mudar las coronas del
heroísmo en coronas de martirio sobre la frente del que no se abatió, ni
se dobló, ni se rompió jamás mientras tuvo un hálito de vida que
sostuviera su grande espíritu.

Serían, pues, las diez de la mañana, cuando el alcaide nos hizo ver su
cara redonda, encendida y brutal, de rubios pelos adornada, y aunque por
la claridad del día venía sin linterna, demostronos desde sus primeras
palabras que no venía a nada bueno. Díjonos aquel simpático pedazo de la
humanidad que nos dispusiéramos a salir todos, y como le indicáramos que
el enfermo a causa de la horrorosa fiebre no podía moverse, repuso que
vendría quien le hiciese mover. D. Mariano nos dio el ejemplo de la
resignación, incorporándose en su lecho, y pidiendo su sombrero. Le
levantamos en brazos; trató de andar por su propio pie, mas no siéndole
posible, le condujimos fuera del aposento, y bajamos todos en triste
procesión, mudos y abrumados de pena. Fuera del castillo vimos dos filas
de gendarmería indicándonos el camino hacia la muralla, y la curiosa
multitud nos contemplaba con lástima. Aquel espectáculo no podía ser más
triste, y con el alma oprimida y llena de angustia dije para mí: «Nos
van a fusilar.»

\hypertarget{xxv}{%
\chapter{XXV}\label{xxv}}

¡Oh, qué trance tan amargo, y qué horrenda hora! Eso de que a sangre
fría le quiten a uno la preciosa existencia, lejos de la patria, ausente
de las personas queridas, sin ojos que le lloren, en soledad espantosa y
entre gente que no ve en ello más que la víctima inmolada a los
intereses militares, es de lo más abrumador que puede ofrecerse a la
contemplación del espíritu humano. Yo miraba aquel cielo, y no era como
el cielo de España; yo miraba a aquella gente, oía su lengua extraña
modulando en conjunto voces incomprensibles, y no era aquella gente
tampoco como la gente de España. Sobre todo, Siseta no estaba allí, y el
vacío formado por su ausencia no lo habrían frenado cien vidas otorgadas
en cambio de la que me iban a quitar. Me ocurrió protestar contra
aquella barbarie, gritando y defendiéndome contra miles de hombres; pero
la realidad de mi impotencia me aplastaba con formidable pesadumbre.
Dejé de ver lo que tenía ante los ojos, y muy intensa congoja me hizo
llorar como una mujer. Mostraban entereza mis compañeros; pero ellos no
habían dejado en Gerona ninguna Siseta.

Al llegar a la muralla vimos formados en fila a los frailes y soldados
que nos habían seguido. Algunos legos y ancianos lloraban; pero el padre
Rull despedía llamas por sus negros y varoniles ojos. En tan supremo
trance, el fraile patriota, rabiando de enojo contra sus verdugos, había
olvidado la principal página del Evangelio. Nos pusieron también a
nosotros en fila, y la persona de Álvarez fue confundida entre los demás
sin consideración a su jerarquía. Estuvimos parados largo rato,
ignorando qué harían de nosotros, en terrible agonía, hasta que apareció
un oficialejo barrigudo, que con un papelito en la mano nos iba
nombrando uno por uno. Tanto aparato, la cruel exhibición ante el
populacho, el despliegue de tan colosales fuerzas contra unos pobres
enfermos muertos de hambre, de cansancio y de sueño, no tenía más objeto
que pasar lista. ¡Ay! Cuando adquirí la certidumbre de que no nos
fusilaban, los franceses me parecieron la gente más amable, más
caritativa y más humana del mundo.

Volvimos al castillo, donde hallamos una gran novedad. El aposento donde
pasamos la noche, se había considerado como un gran lujo de comodidades
para estos pícaros insurgentes y bandidos, que tan heroicamente
defendieron la plaza de Gerona, y nos destinaron a una lóbrega mazmorra
sin aire, empedrada de agudísimos guijarros, entre cuyos huecos se
remansaban fétidas aguas. Doble puerta con cerrojos fuertísimos la
cerraba, y un mezquino agujero abierto en el ancho muro dejaba entrar
sólo al medio día un rayo de luz, insuficiente para que nos
reconociésemos las caras. Protestamos; el mismo Álvarez reprendió
ásperamente al alcaide; pero este ni aun siquiera tuvo la dignación de
contestarnos otra cosa más que la oferta de servirnos una buena comida,
si se la pagábamos bien. El ilustre enfermo se empeoraba de hora en
hora, y desde aquel día comprendimos que se nos iba a morir en los
brazos, si no se instalaba en lugar más higiénico. Haciendo un esfuerzo
el mismo Álvarez, escribió una carta al general Augereau, notificándole
los malos tratamientos de que era objeto; pero no tuvo contestación. Y
seguía lo de la linterna por la noche, en cuya obra caritativa se
esmeraba el maldito francés regordete y rubio, amén de robarnos con la
perversa cena que nos ponía. Si el gobernador necesitaba alguna
medicina, no había fuerzas humanas que la hiciesen traer, por temor de
que se envenenara, y registrándonos escrupulosamente, fuimos despojados
de todo instrumento cortante para evitar que tratásemos de poner fin a
aquella deliciosa vida con que éramos regalados.

En aquella inmunda pocilga estuvimos hasta que concluyó Diciembre y el
funestísimo año 9, enfermos todos, y más que enfermo, moribundo el gran
Álvarez, que al resistir tan grandes padecimientos mostró tener el
cuerpo tan enérgico y vigoroso como el alma. Durante las largas y
tristes horas departía con nosotros sobre la guerra, contábanos su
gloriosa historia militar y nos infundía esperanza y bríos, augurando
con elevado discernimiento el glorioso fin de la lucha con los franceses
y el triunfo de la causa nacional. Su extraordinario espíritu, superior
a cuanto le rodeaba, sabía abarcar los acontecimientos con segura
perspicacia, y oyéndole, oíamos la voz poderosa de la patria que llegaba
al calabozo excavado en extranjero suelo.

Al fin nuestro doloroso encierro en aquella mazmorra donde nos
consumíamos viendo extinguirse la noble vida del defensor de Gerona,
tuvo fin una noche en que el alcaide entró a decirnos que nos
vistiéramos a toda prisa porque nos iban a internar en Francia. Esta
noticia, a pesar de alejarnos de España nos produjo inmensa alegría
porque ponía fin al encierro, y no aguardamos a que la repitiese el
panzudo hombre de la linterna, demostrándole de diversos modos el gran
gusto que sentíamos por perderle de vista lo mismo que a su aparato. Nos
sacaron de Perpiñán con numerosa escolta, y iban los frailes con
nosotros. El jefe de la gendarmería dio orden de fusilar a todo señor
fraile que tratase de huir, y nos pusimos en marcha.

Pero en este viaje la Providencia nos deparó un hombre generoso y
caritativo que a escondidas de los franceses, sus compatriotas, prodigó
al ilustre enfermo solícitos cuidados. Era el mismo cochero que le
conducía, el cual, condolido de sus males e ignorando que fuese un
héroe, mostró sus cristianos sentimientos de diversos modos. Agradecidos
a su bondad quisimos recompensarle; pero no consintió en admitir nada, y
como los gendarmes le mandaran que avivase el paso de las caballerías
para marchar más a prisa, él, sabiendo cuánto daño hacía al paciente la
celeridad de la carrera, fingió enfermedades en el escuálido ganado y
desperfectos en el viejo coche para justificar el tardo paso con que
andaba. Todos los de a pie, que éramos los más, le agradecimos en el
alma la pereza de su vehículo.

Después de descansar un poco en Salces, hicimos noche en Sitjans, y
nunca a tal punto llegáramos, porque haciendo bajar de su coche al
general, le aposentaron con los demás de su séquito en una caballeriza
llena de estiércol, y donde no había cama ni sillas, ni nada que se
pareciese a un mueble, siquiera fuese el más mezquino y pobre. Agotada
la paciencia ante tanta infamia, y viendo cuán poco adecuado era aquel
inmundo sitio para quien por su categoría y además por su lastimoso
estado tenía derecho a todas las consideraciones, no pudimos contener la
explosión de nuestro enojo, y con durísimas palabras increpamos al jefe
de la gendarmería. Este, después de amenazarnos, pareció aplacarse,
comprendiendo sin duda la justicia de nuestra reclamación, y al fin
después de vacilar, vino a decir en suma que el alojamiento no era
cuenta suya. Por fin el cochero, con orden o por simple tolerancia del
jefe de la fuerza, introdujo en la cuadra una cama en que descansó
algunas horas el desgraciado enfermo, cuya prodigiosa resistencia
parecía tocar ya al último límite.

A la mañana siguiente cuando nos íbamos a poner de nuevo en marcha,
aparecieron unos guardias a caballo que traían una orden para el jefe
que nos conducía. Abriendo el pliego en nuestra presencia, nos dio a
conocer su contenido, el cual no era otra cosa sino que monsieur Álvarez
debía volver a España. Esto nos alegró sobre manera, por la esperanza de
ver pronto la patria querida, y hasta sospechamos, si, apiadados de
nuestra desgracia, se dispondrían aquellos caballeros a dejarnos en
libertad luego que traspasásemos la frontera. Los frailes, la gente de
tropa que no pertenecía a la comitiva del enfermo, creyéronse también
destinados a pisar pronto el suelo español, y mostráronse muy alegres;
pero los gendarmes al punto les sacaron de su risueño error, mandándoles
seguir adelante, por Francia adentro. Nos despedimos de ellos
tiernamente recogiendo encargos, recados, cartas y amorosas memorias de
familia, y volvimos la cara al Pirineo. D. Mariano al saber que se
variaba de rumbo, dijo: \emph{«Como no me vuelvan al Castillet de
Perpiñán, llévenme a donde quieran.»}

Excuso enumerar los miserables aposentamientos, los crueles tratos que
se sucedieron desde Sitjans a la frontera española, ni sé cómo por tanto
tiempo y a tan repetidos golpes resistió la naturaleza del hombre contra
quien se desplegaba tan gran lujo de maldad. Por último, señores,
concluiré refiriendo a ustedes la última escena de aquel terrible via
crucis, la cual ocurrió en la misma frontera, y un poco más allá de
Pertús. Es el caso que cuando con el mayor gozo habíamos pisado la
tierra de España, se presentaron unos guardias a caballo con nuevas
órdenes para los gendarmes. El jefe mostrose muy contrariado, y
habiéndose trabado ligera reyerta entre este y uno de los portadores del
oficio, oímos esta frase, que aunque dicha en francés, fácilmente podía
ser comprendida: «Monsieur Álvarez debe volver, pero los edecanes y
asistentes no.»

Al punto comprendimos que se nos quería separar de nuestro idolatrado
general, dejándonos a todos en Francia, mientras a él se le llevaba otra
vez solo, enteramente solo, al castillo de Figueras. Esto causó una
verdadera desolación en la pequeña comitiva. Satué, cerrando los puños y
vociferando como un insensato, dijo que antes se dejaría hacer pedazos
que abandonar a su general; otros, creyendo mal camino para convencer a
nuestros conductores el de la amenaza y la cólera, suplicamos al jefe de
los gendarmes que nos dejase seguir. El mismo enfermo indicó que si se
le separaba de sus fieles compañeros de desgracia, la residencia en
España le sería tan insoportable al menos, como la prisión en el
Castillet. Suplicamos todos en diverso estilo que nos dejasen asistir y
consolar a nuestro querido gobernador, pero esto fue inútil. Como
complemento de los mil martirios que con refinado ingenio habían
aplicado al héroe, quisieron someter su grande alma a la última prueba.
Ni su enfermedad penosísima, ni sus años, ni la presunción de su muerte
que se creía próxima y segura, les movieron a lástima; tanta era la
rabia contra aquel que había detenido durante siete meses frente a una
ciudad indefensa a más de cuarenta mil hombres, mandados por los
primeros generales de la época; que no había sentido ni asomo de
abatimiento ante una expugnación horrorosa en que jugaron once mil
novecientas bombas, siete mil ochocientas granadas, ochenta mil balas, y
asaltos de cuyo empuje se puede juzgar considerando que los franceses
perdieron en todos ellos veinte mil hombres.

Cansados de inútiles ruegos, pedimos al fin que se permitiera ir
acompañando y sirviendo al general a uno de nosotros, para que al menos
no careciese aquel de la asistencia que su estado exigía; pero ni esto
se nos concedió. La agria disputa inspiró al mismo Álvarez las palabras
siguientes: \emph{«Todas estas son estratagemas de que se valen los
franceses para mortificar a aquel a quien no han podido hacer bajar la
espalda.»}

Bruscamente nos quisieron apartar del coche en que iba; pero
atropellando a los que nos lo impedían, nos abalanzamos sobre él, y unos
por un costado otros por el opuesto, le besamos las manos regándolas con
nuestras lágrimas. Satué se metió violentamente dentro del coche, y los
gendarmes lo sacaron a viva fuerza, amenazándole con fusilarle allí
mismo, si no se reportaba en las manifestaciones de su dolor. El
general, despidiéndonos con ánimo sereno, nos dijo que renunciásemos a
una inútil resistencia, conformándonos con nuestra suerte; añadió que él
confiaba en el próximo triunfo de la causa nacional, y que aun
sintiéndose próximo a morir, su alma se regocijaba con aquella idea.
Recomendonos la prudencia, la conformidad, la resignación, y él mismo
dio a sus conductores la orden de partir para poner pronto fin a una
escena que desgarraba su corazón lo mismo que el nuestro. El cupé partió
a escape y nos quedamos en Francia, sujetados por los gendarmes, que nos
ponían sus fusiles en el pecho para impedir las demostraciones de
nuestra ira. Seguimos con los ojos llenos de lágrimas de desesperación
el coche que se perdía poco a poco entre la bruma, y cuando dejamos de
verle, Satué bramando de ira, exclamó: «Se lo llevaron esos perros; se
lo llevan para matarle sin que nadie lo vea.»

\hypertarget{xxvi}{%
\chapter{XXVI}\label{xxvi}}

No puedo pintar a ustedes nuestra profunda consternación al vernos
esclavos de Francia, y considerando la situación del desgraciado
Álvarez, solo, en poder de sus verdugos. Nuestra propia suerte de
prisioneros nos causaba menos pesar que la de aquel heroico veterano,
condenado por su valor sublime a ser juguete de una cruel soldadesca, a
quien lo entregaron para que se divirtiese martirizándole.

Encerráronnos en Pertús en una inmunda cuadra, donde con centinelas de
vista nos tuvieron hasta el día siguiente, en cuya alborada, cuando nos
llevaban fuera del pueblo, verificamos un acto honroso, con el cual
quiero poner fin a mi narración. Allí, sobre unas peñas desde las cuales
se divisaban a lo lejos los cerros y vertientes de España, nos dimos las
manos y juramos todos morir antes que resignarnos a soportar la odiosa
esclavitud que la canalla quería imponernos. Desde aquel instante
principiamos a concertar un hábil plan para fugarnos, cual tantos otros,
que llevados a Francia, habían sabido volver por peligrosos caminos y
medios a la patria invadida.

Amigos míos: por no cansar a ustedes con prolijidades que sólo a mí se
refieren y a mis particulares cuitas, omito los pormenores de nuestra
residencia en Francia, y de los medios que empleamos para regresar a
España. Éramos seis, y sólo tres volvimos. Los demás, cogidos in
fraganti, fueron fusilados, dos en Maurellas y uno en Boulou. ¿Alguno de
los que me oyen no se ha visto en igual caso? ¡Cuántos de los que
estamos aquí desataron sus manos de las cuerdas que los franceses han
llevado a Francia después de la toma de Zaragoza o de Madrid! Con la
relación de los padecimientos que sufrí en la frontera, de las diabluras
y estratagemas que puse en juego para escaparme, y de las mil cosas que
me sucedieron desde que pasé la frontera por Puigcerdà hasta unirme en
el centro de España a esta división de Lacy en que ahora estoy,
emplearía otras dos noches largas, pues todo el sitio de Gerona y las
extravagancias de D. Pablo Nomdedeu no exigen más tiempo y espacio que
los peligros, trapisondas, trabajos y terribles trances en que me he
visto. Concluyo, pues, no sin dirigir una ojeada hacia atrás, como
parecen exigírmelo mis caros oyentes, deseosos de saber qué fue de
Siseta, así como de sus hermanitos Badoret y Manalet.

No estaría mi ánimo tranquilo si en tan largo plazo hubiese vivido sin
saber de personas tan caras para mí. Antes de abandonar a Cataluña con
intención de unirme al ejército del Centro, hallé medios para hacer
llegar a Gerona noticias mías, y Dios me deparó el consuelo de que
también vinieran a mí verdaderas y frescas. Los tres hermanos siguen
allí sanos y buenos en compañía de la señorita Josefina, que en ellos ve
toda su familia, y el único consuelo de sus tristes días. La hija del
doctor no ha recobrado por completo la salud, ni desgraciadamente la
recobrará, según me dicen. Ha tenido inclinación a entrar en un
convento; mas Siseta procura arrancarla sus melancolías y la induce a
que aspire al matrimonio, en la seguridad de encontrar buen esposo. No
demuestra, sin embargo, Josefina disposición a seguir este consejo, y
gusta de embeber su vida en contemplaciones de la Naturaleza y de la
religión, que son sin duda el alimento más apropiado a su pobre espíritu
huérfano y solitario.

Siseta y sus hermanos aguardan a que yo me retire del ejército para
marchar a la Almunia, donde tengo mis tierras, consistentes en dos
docenas de cepas y un número no menor de frondosos olivos, y por mi
parte pido a Dios que nos libre al fin de franceses, para poder soltar
el grave peso de las armas y tornar a mi pueblo, donde no pienso hacer
al tiempo de mi llegada otra cosa de provecho más que casarme.

Con lo que Siseta ha heredado, y lo que yo poseo, tenemos lo suficiente
para pasar con humilde bienestar y felicidad inalterable la vida, pues
no me mortifica el escozor de la ambición, ni aspiro a altos empleos, a
honores vanos ni a la riqueza, madre de inquietudes y zozobras. Hoy
peleo por la patria, no por amor a los engrandecimientos de la milicia,
y de todos los presentes soy quizás el único que no sueña con ser
general.

Otros anhelan gobernar el mundo; sojuzgar pueblos y vivir entre el
bullicio de los ejércitos; pero yo contento en la soledad silenciosa, no
quiero más ejércitos que los hijos que espero ha de darme Siseta.

Así acabó su relación Andresillo Marijuán. La he reproducido con toda
fidelidad en su parte esencial, valiéndome como poderoso auxiliar del
manuscrito de D. Pablo Nomdedeu, que aquel mi buen amigo me regaló más
tarde cuando asistí a su boda. Repito lo que dije al comenzar el libro,
y es que las modificaciones introducidas en esta relación afectan sólo a
la superficie de la misma, y la forma de expresión es enteramente mía.
Tal vez haya perdido mucho la leyenda de Andrés al perder la sencillez
de su tosco estilo; pero yo tenía empeño en uniformar todas las partes
de esta historia de mi vida, de modo que en su vasta longitud se hallase
el trazo de una sola pluma.

Cuando Marijuán calló, algunos de los presentes dieron interpretaciones
diversas al encierro de D. Mariano Álvarez en el castillo de Figueras, y
como ya desde antes de entrar en Andalucía habíamos sabido la misteriosa
muerte del insigne capitán, la figura más grande sin duda de las que
ilustraron aquella guerra, cada cual explicó el suceso de distinto modo.

---Dícese que le envenenaron---afirmó uno,---en cuanto llegó al
castillo.

---Yo creo que Álvarez fue ahorcado---opinó otro,---pues el rostro
cárdeno e hinchado, según aseguran los que vieron el cadáver de Su
Excelencia, indica que murió por estrangulación.

---Pues a mí me han dicho---añadió un tercero,---que lo arrojaron a la
cisterna del castillo.

---Hay quien afirma que le mataron a palos.

---Pues no murió sino de hambre, y parece que desde su llegada fue
encerrado en un calabozo, donde lo tuvieron tres días sin alimento
alguno.

---Y cuando le vieron bien muerto, y se aseguraron de que no volvería
hacer otra como la de Gerona, expusiéronle en unas parihuelas a la vista
del pueblo de Figueras, que subió en masa a contemplar el cuerpo del
grande hombre.

Discutimos largo rato sin poder poner en claro la clase de muerte que
había arrebatado del mundo a aquel inmortal ejemplo de militares y
patriotas; pero como su fin era evidente, convinimos por último en que
el esclarecimiento del medio empleado para exterminar tan terrible
enemigo del poder imperial, afectaba más al honor francés que al
ejército español, huérfano de tan insigne jefe; y si verdaderamente fue
asesinado, como se ha venido creyendo desde entonces acá, la
responsabilidad de los que toleraron sin castigarla tan atroz barbarie
bastaría a exceptuar entonces a Francia de la aplicación de las leyes de
la guerra en lo que antes tienen de humano. Que murió violentamente
parece indudable, y mil indicios corroboran una opinión que los
historiadores franceses no han podido con ingeniosos esfuerzos destruir.
No es creíble que órdenes de París impulsaran este horrible asesinato;
pero un poder que si no disponía, toleraba tan salvajes atentados,
merecía indisputablemente las amarguras y horrendas caídas que
experimentó luego. La soberbia enfatuada y sin freno perpetra grandes
crímenes ciegamente, creyendo realizar actos marcados por ilusorio
destino. Los malvados en grande escala que han tenido la suerte o la
desgracia de que todo un continente se envilezca arrojándose a sus pies,
llegan a creer que están por encima de las leyes morales, reguladoras
según su criterio, tan sólo de las menudencias de la vida. Por esta
causa se atreven tranquilamente y sin que su empedernido corazón palpite
con zozobra, a violar las leyes morales, ateniéndose para ello a las mil
fútiles y movedizas reglas que ellos mismos dictaron llamándolas razones
de estado, intereses de esta o de la otra nación; y a veces si se les
deja, sobre el vano eje de su capricho o de sus pasiones hacen mover y
voltear a pueblos inocentes, a millares de individuos que no quieren
sino el bien. Verdad es que parte de la responsabilidad corresponde al
mundo, por permitir que media docena de hombres o uno solo jueguen con
él a la pelota.

Desarrollados en proporciones colosales los vicios y los crímenes, se
desfiguran en tales términos que no se les conoce; el historiador se
emboba engañado por la grandeza óptica de lo que en realidad es pequeño,
y aplaude y admira un delito tan sólo porque es perpetrado en la
extensión de todo un hemisferio. La excesiva magnitud estorba a la
observación lo mismo que el achicamiento que hace perder el objeto en
las nieblas de lo invisible. Digo esto, porque a mi juicio, Napoleón I y
su efímero imperio, salvo el inmenso genio militar, se diferencian de
los bandoleros y asesinos que han pululado por el mundo cuando faltaba
policía, tan sólo en la magnitud. Invadir las naciones, saquearlas,
apropiárselas, quebrantar los tratados, engañar al mundo entero, a reyes
y a pueblos, no tener más ley que el capricho y sostenerse en constante
rebelión contra la humanidad entera, es elevar al máximum de desarrollo
el mismo sistema de nuestros famosos caballistas. Ciertas voces no
tienen en ningún lenguaje la extensión que debieran, y si despojar a un
viajante de su pañuelo se llama robo, para expresar la tala de una
comarca, la expropiación forzosa de un pueblo entero, los idiomas tienen
pérfidas voces y frases con que se llenan la boca los diplomáticos y los
conquistadores, pues nadie se avergüenza de nombrar los grandiosos
planes continentales, la absorción de unos pueblos por otros\ldots{}
etc. Para evitar esto debiera existir (no reírse) una policía de las
naciones, corporación en verdad algo difícil de montar; pero entre tanto
tenemos a la Providencia, que al fin y al cabo sabe poner a la sombra a
los merodeadores en grande escala, devolviendo a sus dueños los objetos
perdidos, y restableciendo el imperio moral, que nunca está por tierra
largo tiempo.

Perdónenme mis queridos amigos esta digresión. No pensaba hacerla; pero
al hablar de la muerte del incomparable D. Mariano Álvarez de Castro, el
hombre, entre todos los españoles de este siglo, que a más alto extremo
supo llevar la aplicación del sentimiento patrio, no he podido menos de
extender la vista para observar todo lo que había en derredor, encima y
debajo de aquel cadáver amoratado que el pueblo de Figueras contemplaba
en el patio del castillo una mañana del mes de enero de 1810. Aquel
asesinato, si realmente lo fue, como se cree, debía traer grandes
catástrofes a quien lo perpetró o consintió, y no importa que los
criminales, cada vez más orgullosos, se nos presentaran con aparente
impunidad, porque ya vemos que el mucho subir trae la consecuencia de
caer de más alto, de lo cual suele resultar el estrellarse.

\hypertarget{xxvii}{%
\chapter{XXVII}\label{xxvii}}

Oímos el relato de Andrés Marijuán, aposentados en una casa del Puerto
de Santa María, donde moraban, además de nosotros, que pertenecíamos al
ejército de Areizaga, muchos canarios de Alburquerque, que habían
llegado el día antes, terminando su gloriosa retirada. A este general
debió el poder supremo no haber caído en poder de los franceses, pues
con su hábil movimiento sobre Jerez, mientras contenía en Écija las
avanzadas de Víctor y Mortier, dio tiempo a preparar la defensa de la
isla de León, y entretuvo al enemigo en las inmediaciones de Sevilla.
Esto pasaba a principios de Febrero, y en los mismos días se nos dio
orden de pasar a la Isla, porque en el continente, o sea del puente de
Suazo para acá ¡triste es decirlo!, no había ni un palmo de terreno
defendible. Toda España afluyó a aquel pedazo de país, y se juntaban
allí ejército, nobleza, clero, pueblo, fuerza e inteligencia, toda la
vida nacional en suma. De la misma manera, en momentos de repentino
peligro para el hombre de ánimo esforzado, toda la sangre afluye al
corazón, de donde sale después con nuevo brío.

Por mi parte deseaba ardientemente entrar en la Isla. Aquel pantano de
sal y arena invadido por movedizos charcos y surcado por regueros de
agua salada, tenían para mí el encanto del hogar nativo, y más aún las
peñas donde se asienta Cádiz en la extremidad del istmo, o sea en la
mano de aquel brazo que se adelanta para depositarla en medio de las
olas. Yo veía desde lejos a Cádiz, y una viva emoción agitaba mi pecho.
¿Quién no se enorgullece de tener por cuna la cuna de la moderna
civilización española? Ambos nacimos en los mismos días, pues al fenecer
el siglo se agitó el seno de la ciudad de Hércules con la gestión de una
cultura que hasta mucho después no se encarnó en las entrañas de la
madre España. Mis primeros años agitados y turbulentos, fuéronlo tanto
como los del siglo, que en aquella misma peña vio condensada la
nacionalidad española, ansiando regenerarse entre el doble cerco de las
olas tempestuosas y del fuego enemigo. Pero en Febrero de 1810 aún no
había nada de esto, y Cádiz sólo era para mí el mejor de los asilos que
la tierra puede ofrecer al hombre; la ciudad de mi infancia, llena de
tiernísimos recuerdos, y tan soberbiamente bella que ninguna otra podía
comparársele. Cádiz ha sido siempre la Andalucía de las ondas, graciosa
y festiva dentro de un círculo de tempestades. Entonces asumía toda la
poesía del mar, todas las glorias de la marina, todas las grandezas del
comercio. Pero en aquellos meses empezaba su mayor poesía, grandeza y
gloria, porque iba a contener dentro de sus blancos muros el conjunto de
la nacionalidad con todos sus elementos de vida en plena efervescencia,
los cuales expulsados del gran territorio, se refugiaban allí dejando la
patria vacía.

A las puertas de Cádiz comienzan los acontecimientos de mi vida que más
vivamente anhelo contar. Estadme atentos, y dejadme que ponga orden en
tantos y tan variados sucesos, así particulares como históricos. La
historia al llegar a esta isla y a esta peña es tan fecunda, que ni ella
misma se da cuenta de la multitud de hijos que deposita en tan estrecho
nido. Trataré de que no se me olvide nada, ni en lo mío ni en lo ajeno.
Para no perder la costumbre, comienzo por una aventura propia, en que
nada tiene que ver la atisbadora historia, pues hasta hoy no he tenido
empeño en comunicarlo a nadie, ni aunque la comunicara, se
inmortalizaría en láminas de bronce, y fue lo siguiente:

Un amigo mío portugués de los que habían venido de Extremadura con
Alburquerque, rondaba cierta casa en la extremidad de la calle Larga
donde algunos días antes viera entrar desconocida beldad, que él ponía
por las nubes, siempre que tocábamos este punto. Sus paseos diurnos y
nocturnos, en que mostraba un celo, una abnegación superiores a todo
encomio, no dieron más resultado que ver al través de las apretadas
verdes celosías, dos figuras, dos bultos de indeterminada forma, pero
que al punto revelaban ser alegres mujeres por el sordo cuchicheo y las
risas con que parecían festejar la cachaza de mi paseante amigo. Cuanto
menos las veía, más acabadamente hermosas se le figuraban, y con la
dificultad de hablarlas, crecía su deseo de poner fin gloriosamente a
una aventura, que hasta entonces había tenido pocos lances. Una tarde
quiso le acompañase yo en su centinela al pie de la reja, y tuve la
suerte de que mi presencia modificara la monótona esquivez de las bellas
damas, las cuales hasta entonces ni a billetes ni a señas, ni a miradas
lánguidas habían contestado más que con las risas consabidas y los
ceceos burlones. Figueroa había deslizado una esquela, y tuvo la
indecible satisfacción de recibir respuesta en un billete que cayó, cual
bendición del cielo, delante de nosotros. En él decía la hermosa
desconocida que estaba dispuesta a abrir la celosía para expresarle de
palabra su gratitud por los amorosos rendimientos, y añadía que
hallándose en un gran compromiso por causa de un suceso doméstico que no
podía revelar, solicitaba para salir de él la ayuda del galán juntamente
con la de su amigo.

Esto nos llamó grandemente la atención, y de vuelta al alojamiento para
esperar la hora de las siete en que se nos había citado, hicimos mil
comentarios sobre el suceso. Mientras mayor era el misterio, mayor
también el anhelo de descifrarlo, y curiosos ambos por saber si íbamos a
tener una sabrosa aventura o a ser objetos de una broma, acudimos por la
noche al pie de la reja. En cuanto llegamos, abriose esta y una voz de
mujer, cuyo acento aunque dulce no me pareció revelar persona de elevada
clase, dijo a Figueroa con bastante agitación estas palabras:

---Señor militar, si es usted caballero, como creo, espero que no se
negará a conceder a una desgraciada dama la generosa ayuda que solicita.
Mi esposo el señor duque de los Umbrosos Montes duerme a estas horas;
pero no puedo dejarle pisar a usted el recinto de este arcásar, que mi
celoso dueño ha convertido en sepulcro de mi hermosura, en cárcel de mi
libertad y en muerte de mi vida. El más leve rumor despertaría al fiel y
sanguinario Rodulfo, paje de mi señor y carcelero mío. Pues verasté: mi
honra depende de que al punto una persona de confianza atraviese las
saladas ondas y parta a Cádiz a llevar un recado urgentísimo, sin lo
cual mi situación es tal que no esperaré a que venga la rosada aurora,
para arrancalme la vida con un veneno de cien mortíferas plantas
compuesto que tengo aquí en aquesta botellita.

Figueroa estaba perplejo y embobado, aunque algo dispuesto a tomar
aquello en serio, y yo contenía la risa al considerar cómo se reían de
nosotros las dos desconocidas; pero mi amigo aseguró estar resuelto a
prestar a ambas cuantos servicios fáciles o difíciles quisieran pedirle,
y entonces la misma que antes hablara, añadió:

---¡Oh!, gracias, \emph{invito} militar; así lo esperaba yo de su
galantería y caballerosidad nunca desmentida en mil y mil lances, cual
lo prueban las voces de la fama que han traído a mis orejas sus hasañas.
Bueno, pues verasté. Mi criada, que es esta guapa y gallarda donsella,
que a mi lado ve usted, y se llama Soraida, irá a Cádiz en un frágil
esquife que Perico el botero tiene preparado en el muelle; pero como es
grande su cortedad, deseo vaya acompañada de ese vuestro leal amigo, que
está ahí oyéndonos como un marmolejo.

Al punto dije que estaba dispuesto a acompañar a la doncella, y mi
amigo, algo corrido con los discursos de su adorada beldad, no sabía qué
contestar. La desconocida habló así con creciente afectación:

---¡Oh! Gracias, \emph{insine amigo del valiente Otelo}. Ya lo esperaba
yo de su \emph{malanimidad}. Pues \emph{oigasté}, señor militar.
Mientras este fiel amigo va a Cádiz a acompañar a mi donsella en la
difícil comisión que mi amenasado honor le encomienda, nosotros nos
quedaremos aquí pelando la pava en este balcón; con lo cual, ¿usté se
entera?, tendré ocasión de mostrarle el amoroso fuego que inflama mi
pecho.

No había acabado de hablar, cuando abriéndose la puerta de la casa,
apareció una mujer cubierta de la cabeza a los pies con espeso manto
negro, la cual llegándose a mí y tomándome el brazo, me obligó a que
rápidamente la siguiese, diciéndome:

---Señor oficial, vamos, que es tarde.

No tuve tiempo para oír lo que desde la ventana decía la desconocida al
amartelado Figueroa, porque la dama, criada o lo que fuera, no me
permitía detenerme y me impulsaba hacia adelante, repitiendo siempre:

---Señor oficial, siga usted. ¡Qué pesado es usted!\ldots{} No mire
usted atrás ni se detenga, que estoy de prisa.

Quise ver su rostro; pero se lo ocultaba cuidadosamente. Se conocía que
trataba de contener la risa y disimular la voz. Era una mujer arrogante
y que me revelaba con sólo el roce de su mano en mi brazo la alta
calidad a que pertenecía. Desde su aparición había yo sospechado, que no
era criada, y después de oírla y sentir el contacto de su vestido,
ningún hombre se habría equivocado respecto a su clase. Yo estaba algo
aturdido por lo inusitado de la aventura, y una dulce confusión
embargaba mi alma. Venían a mi mente indicios, recuerdos, y aquella
mujer llevaba en los pliegues de su vestido una atmósfera que no era
nueva para mí. Pero al principio ni aun pude formular claramente mis
sospechas. La desconocida me llevaba rápidamente y andábamos a prisa por
las calles del Puerto, hablando de esta manera:

---Señora, ¿insiste usted en ir a Cádiz por mar a estas horas?

---¿Por qué no? ¿Se marea usted? ¿Tiene usted miedo a embarcarse?

---Por bueno que esté el mar, el viaje no será cómodo para una dama.

---Es usted un necio. ¿Cree usted que yo soy cobarde? Si no tiene usted
ánimo iré sola.

---Eso no lo consentiré, y aunque se tratara de ir a América en el
frágil esquife de que hablaba la señora duquesa de los Umbrosos
Montes\ldots{}

La desconocida no pudo contener la risa, y el dulce acento de su voz
resonó en mi cerebro, despertando mil ideas que rápidamente cambiaron en
luz las oscuridades de mi pensamiento, y en certidumbre las nebulosas
dudas.

---Adelante---exclamó al ver que me detenía.---Ya estamos en el muelle.
El botero está allí. La marea sube y nos favorecerá; el mar parece
tranquilo.

Callé y seguimos hasta el malecón. Era preciso bajar por una serie de
piedras puestas en la forma más parecida a una escalera, y el descenso
no carecía de peligro. Tomé en brazos a mi compañera, y la bajé
cuidadosamente al bote. Entonces ni pudo, ni quiso sin duda ocultarme su
rostro, y la conocí. La fuerte emoción no me permitió hablar.

---¡Oh, señora condesa!---exclamé besándole tiernamente las
manos.---¡Qué felicidad tan grande encontrar a usía!\ldots{}

---Gabriel---me contestó,---ha sido realmente una felicidad que me hayas
encontrado, porque vas a prestarme un gran servicio.

---Estoy destinado a ser criado de vuecencia en donde quiera que me
halle.

---Criado no: ya esos tiempos pasaron. ¿Dónde has estado?

---En Zaragoza.

---¿Ves qué fácilmente se van ganando charreteras, y con ellas posición
y nombre en el mundo? Entramos en unos tiempos en que los desgraciados y
los pobres se encaramarán a los puestos que debe ocupar la grandeza.
Gabriel, estoy asombrada de verte caballero. Bien, muy bien. Así te
quería. No me habías dicho nada. ¿Por qué no me has buscado?\ldots{} Ya
no nos quieres.

---Señora, ¿cómo he de olvidar los beneficios que de vuecencia recibí?
Estoy confundido al ver que nuevamente, y cuando menos lo esperaba, se
digna usía servirse de mí.

---No bajes tanto, Gabriel; han cambiado las cosas. Tú no eres el mismo;
no te conozco. Me ves, me hablas, ¿y no me preguntas por Inés?

---Señora---exclamé anonadado,---no me atreví a tanto. Veo que vuecencia
ha cambiado más que yo.

---Tal vez.

---¿Inés vive?

---Sí, está en Cádiz. ¿Deseas verla? Pues no te apures; yo te prometo
que la verás, la verás.

Diciendo esto, Amaranta se expresaba en un tono que me hacía comprender
su anhelo de mortificar a alguien, al permitirme ver a su hija. Su
benevolencia me tenía tan confundido, que ni aun acertaba a darle las
gracias.

---¡En qué momento tan crítico para mí te me has aparecido, Gabriel! Un
suceso que sabrás más tarde me obliga a ir a Cádiz esta noche, sola, sin
que ninguno de mi familia lo sepa. Dios no me podía ofrecer compañero ni
custodio más a propósito.

---Pero señora, ¿usía no considera que las puertas de Cádiz están
cerradas a estas horas?

---Lo están para mí todas menos una. Por eso me aventuro en esta
travesía que podría ser peligrosa. El jefe de guardia en la puerta de
mar es amigo mío y me espera. Yo tenía el bote preparado. Estaba
dispuesta a ir sola, y cuando te presentaste en la calle acompañando al
oficial que nos rondaba, vi el cielo abierto. Gabriel, te juro que estoy
contentísima de verte en la honrosa condición en que ahora te hallas.
Así te deseaba yo. Pero chiquillo, ¿eres tú mismo?\ldots{} ¡Pues no
lleva sus charreteras como un hombre!\ldots{} El muy zarramplín con ese
uniforme, que le sienta bien, tiene aire de persona decente\ldots{} Vaya
usted a hacer creer a la gente que has jugado en la Caleta\ldots{}
chico, bien, bien, así me gusta\ldots{} qué bien te vendría ahora
aquella farsa de tus abolengos\ldots{} No me canso de mirarte,
pelafustán\ldots{} ¡qué tiempos estos! He aquí un gato que quiso zapatos
y que se ha salido con ello\ldots{} Te juro que eres otro. Inés no te va
a conocer\ldots{} ¡Qué a tiempo has venido! Estás muy bien,
hijito\ldots{} Desde que fuiste mi paje conocí tu corazón de oro\ldots{}
¡Ay!, no te faltaba más que el forro, y veo que lo vas teniendo\ldots{}
Gabriel: creo que te alegras de verme, ¿no es verdad? Yo también.
Cuántas veces he dicho: si ahora apareciese ese muchacho\ldots{} Mañana
te contaré todo. Chiquillo, soy la mujer más desgraciada de la tierra.

El bote avanzaba con la proa a Cádiz. El botero fijo en la popa llevaba
el timón, y dos muchachos habían izado la vela latina, con la cual,
merced al viento fresco de la noche, la embarcación se deslizaba
cortando gallardamente las mansas olas de la bahía. La claridad de la
luna nos alumbraba el camino: pasábamos velozmente junto a la negra masa
de los barcos de guerra ingleses y españoles, que parecían correr al
costado en dirección opuesta a la que seguíamos. Aunque el mar estaba
tranquilo, agitábase bastante el bote, y sostuve con mi brazo a la
condesa para impedir que se hiciera daño con las frecuentes cabezadas
del barco. Los tres marinos no pronunciaron una sola palabra en todo el
trayecto.

\hypertarget{xxviii}{%
\chapter{XXVIII}\label{xxviii}}

---¡Cuánto tardamos!---dijo Amaranta con impaciencia.

---El bote va como un rayo. Antes de diez minutos estaremos allá---dije
al ver las luces de la ciudad reflejadas en el agua.---¿Tiene vuecencia
miedo?

---No, no tengo miedo---repuso tristemente,---y te juro que aunque las
olas fueran tan fuertes, que lanzaran el bote a la altura de los topes
de ese navío, no vacilaría en hacer este viaje. Lo habría hecho sola, si
no te hubieras aparecido como enviado del cielo para acompañarme. Cuando
te vi, mi primera idea fue llamarte; pero luego mi criada y yo
discurrimos la invención que oíste, para desorientar al hidalgo
portugués. No quiero que nadie me conozca.

---La señora duquesa de los Umbrosos Montes estará a estas horas
trastornando el seso de mi buen amigo.

---Sí, y lo hará bien. Si mi ánimo estuviera tranquilo, me reiría
recordando la gravedad con que dijo las relaciones que le enseñé esta
tarde. Hace poco, como se empeñara en galantearme un viajero inglés,
Dolores quiso pasar por ama y yo por criada; pero él conoció al punto el
engaño. No nos dejaba ni a sol ni a sombra, y no puedes figurarte las
felices ocurrencias de mi doncella a propósito del caballero británico,
de su aspecto tristón, de sus ardientes arrebatos y de su cojera. Era a
ratos amable y fino, a ratos sombrío y sarcástico y se llamaba lord
Byron.

---No es extraño que vuecencia enloqueciera a ese señor inglés. Pero ya
llegamos, señora condesa, y el bote va a atracar en el muelle. Sale la
guardia a darnos el quién vive.

---No importa; tengo pase. Di que llamen a D. Antonio Maella, jefe de la
guardia.

Presentose el oficial, y nos dio entrada sin dificultad, abriéndonos
luego la puerta, por donde pasamos a la plaza de San Juan de Dios.
Mientras nos acompañaba hasta dicho punto, habló brevemente con
Amaranta.

---Ya la esperaba a usted---dijo.---Las dos señoras marquesas tienen
preparado su viaje para mañana, en la fragata inglesa Eleusis. Piensan
establecerse en Lisboa.

---Su objeto es alejarse de mí---repuso Amaranta.---Felizmente he tenido
aviso oportuno, y me parece que llego a tiempo.

---Tan callado tenían el viaje, que yo mismo no lo he sabido hasta esta
tarde por el capitán de la fragata. ¿Piensa usted partir también con
ellas?

---Partiré si no puedo detenerlas.

Al decir esto, la condesa, sin perder tiempo en contestar a los
cumplidos y finezas del oficial, tomó mi brazo, y obligándome a tomar
paso algo vivo, me dijo:

---Gabriel, no nos detengamos. ¡Cuán inquieta estoy!\ldots{} Ya te lo
contaré todo después. Figúrate que después de que me hacen vivir como en
destierro, separada de lo que más amo en el mundo\ldots{} ¿qué te
parece? Dios mío, ¿qué he hecho yo para merecer tal castigo?\ldots{}
Pues sí\ldots{} Después que me obligan a vivir allá\ldots{} Te
diré\ldots{} hasta se han empeñado en hacerme pasar por
afrancesada\ldots{} Y todo ¿por qué?, dirás tú\ldots{} Pues nada más
sino porque\ldots{} andemos más a prisa\ldots{} porque me opongo a que
la hagan desventurada para siempre\ldots{} Mi tía no tiene sensibilidad,
y nuestra parienta la de Rumblar tiene un rollo de pergaminos en el
sitio donde los demás llevamos el corazón. Además, con los vidrios
verdes de sus espejuelos no ve más que dinero\ldots{} Gabriel, etiqueta
y soberbia en un lado, soberbia y avaricia en otro\ldots{} No puedes
figurarte cuán apenadas y tristes están las tres pobres
muchachas\ldots{} Y ahora quieren llevárselas a Lisboa\ldots{} ¿qué
dices tú a eso?\ldots{} Todo por alejar a Inés de mí\ldots{} ¡Con cuánto
secreto han preparado el viaje!\ldots{} ¡Con qué habilidad me confinaron
en el Puerto, haciendo llegar a los individuos de la Junta falsas
noticias acerca de mí! Por fortuna soy amiga del embajador inglés,
Wellesley\ldots{} que no\ldots{} Pues sí, mi tía y yo nos disputamos
ardientemente el dirigir a la pobre Inés hacia su mejor destino\ldots{}
ella va por una senda, yo por otra\ldots{} lo que yo quiero es más
razonable; y si no, dime tu parecer\ldots{} Pero ya hablaremos mañana.
¿Te quedarás en la Isla o vendrás a Cádiz? Espero que nos veremos,
Gabrielillo. ¿Te acuerdas cuando eras mi paje en el Escorial y yo te
contaba aquellas historias?

---Esos y otros recuerdos de aquel tiempo, señora---le respondí,---son
los más dulces de mi vida.

---¿Te acuerdas cuando te presentaste en Córdoba?---prosiguió
riendo.---Entonces estabas algo tonto. ¿Te acuerdas de cuando en Madrid
fuiste a casa con el padre Salmón?\ldots{} ¿Te acuerdas de cuando te
encontré en el Pardo vestido de duque de Arión?\ldots{} Después me he
acordado mucho de ti, y he dicho: «¡Dónde estará aquel
desgraciado!\ldots» No puedo creer sino que Dios te ha cogido por la
mano para ponerte delante de mí\ldots{} Ya llegamos.

Nos detuvimos junto a una casa de la calle de la Verónica.

---Llama a la puerta---me dijo la condesa.---Esta es la casa de una
amiga mía de toda confianza.

---¿Vive aquí la señora marquesa?---pregunté tirando de la campanilla de
la reja.---Esta casa no me es desconocida.

---Aquí vive doña Flora de Cisniega: ¿la conoces? Entremos. Se ven luces
en la sala. Aún están en la tertulia; es temprano. Ahí estarán Quintana,
Gallego, Argüelles, Gallardo y otros muchos patriotas.

Subimos y en un gabinete interior nos recibió el ama de la casa, en
quien al punto reconocí una amistad antigua.

---¿Está aquí?---le preguntó con ansiedad la condesa.

---Sí; aunque se embarcan mañana de secreto, han venido esta noche sin
duda para que yo no sospeche su determinación. Pero a mí no se me
engaña\ldots{} ¿va usted a la sala? Está muy animada la tertulia. ¡Ay!,
amiga mía, esta noche he ganado al \emph{monte} una buena suma.

---No, no voy a la sala. Haga usted salir a Inés con cualquier pretexto.

---Está en coloquio tirado con el amable inglesito. Pero saldrá. Mandaré
a Juana que la llame.

Después de dar la orden a su doncella, doña Flora me observó
atentamente, queriendo reconocerme.

---Sí, soy Gabriel, señora doña Flora, soy Gabriel, el paje del Sr.~D.
Alonso Gutiérrez de Cisniega.

Doña Flora, no necesitando más, abalanzose a mí con todo el ímpetu de su
sensible corazón.

---Gabrielillo, ¿es posible que seas tú?---exclamó chillonamente
estrechándome entre sus brazos.---Estás hecho un hombre, un
caballero\ldots{} ¡Qué alto estás! Cuánto me alegro de verte\ldots{} ya
te he echado de menos\ldots{} pero ¡qué buen mozo eres!\ldots{} ¿Qué tal
me encuentras?\ldots{} Otro abrazo\ldots{} ¡Ay!\ldots{} ¿Por qué me
dejaste?\ldots{} ¡pobrecito niño!

Mientras era objeto de tan ardientes demostraciones de regocijo, sentí
el rumor propio de un rápido movimiento de faldas hacia el corredor que
conducía a la pieza donde estábamos.

\flushright{Junio de 1874.}

~

\bigskip
\bigskip
\begin{center}
\textsc{fin}
\end{center}

\end{document}
