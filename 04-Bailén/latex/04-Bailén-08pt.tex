\PassOptionsToPackage{unicode=true}{hyperref} % options for packages loaded elsewhere
\PassOptionsToPackage{hyphens}{url}
%
\documentclass[oneside,8pt,spanish,]{extbook} % cjns1989 - 27112019 - added the oneside option: so that the text jumps left & right when reading on a tablet/ereader
\usepackage{lmodern}
\usepackage{amssymb,amsmath}
\usepackage{ifxetex,ifluatex}
\usepackage{fixltx2e} % provides \textsubscript
\ifnum 0\ifxetex 1\fi\ifluatex 1\fi=0 % if pdftex
  \usepackage[T1]{fontenc}
  \usepackage[utf8]{inputenc}
  \usepackage{textcomp} % provides euro and other symbols
\else % if luatex or xelatex
  \usepackage{unicode-math}
  \defaultfontfeatures{Ligatures=TeX,Scale=MatchLowercase}
%   \setmainfont[]{EBGaramond-Regular}
    \setmainfont[Numbers={OldStyle,Proportional}]{EBGaramond-Regular}      % cjns1989 - 20191129 - old style numbers 
\fi
% use upquote if available, for straight quotes in verbatim environments
\IfFileExists{upquote.sty}{\usepackage{upquote}}{}
% use microtype if available
\IfFileExists{microtype.sty}{%
\usepackage[]{microtype}
\UseMicrotypeSet[protrusion]{basicmath} % disable protrusion for tt fonts
}{}
\usepackage{hyperref}
\hypersetup{
            pdftitle={BAILÉN},
            pdfauthor={Benito Pérez Galdós},
            pdfborder={0 0 0},
            breaklinks=true}
\urlstyle{same}  % don't use monospace font for urls
\usepackage[papersize={4.80 in, 6.40  in},left=.5 in,right=.5 in]{geometry}
\setlength{\emergencystretch}{3em}  % prevent overfull lines
\providecommand{\tightlist}{%
  \setlength{\itemsep}{0pt}\setlength{\parskip}{0pt}}
\setcounter{secnumdepth}{0}

% set default figure placement to htbp
\makeatletter
\def\fps@figure{htbp}
\makeatother

\usepackage{ragged2e}
\usepackage{epigraph}
\renewcommand{\textflush}{flushepinormal}

\usepackage{indentfirst}

\usepackage{fancyhdr}
\pagestyle{fancy}
\fancyhf{}
\fancyhead[R]{\thepage}
\renewcommand{\headrulewidth}{0pt}
\usepackage{quoting}
\usepackage{ragged2e}

\newlength\mylen
\settowidth\mylen{...................}

\usepackage{stackengine}
\usepackage{graphicx}
\def\asterism{\par\vspace{1em}{\centering\scalebox{.9}{%
  \stackon[-0.6pt]{\bfseries*~*}{\bfseries*}}\par}\vspace{.8em}\par}

 \usepackage{titlesec}
 \titleformat{\chapter}[display]
  {\normalfont\bfseries\filcenter}{}{0pt}{\Large}
 \titleformat{\section}[display]
  {\normalfont\bfseries\filcenter}{}{0pt}{\Large}
 \titleformat{\subsection}[display]
  {\normalfont\bfseries\filcenter}{}{0pt}{\Large}

\setcounter{secnumdepth}{1}
\ifnum 0\ifxetex 1\fi\ifluatex 1\fi=0 % if pdftex
  \usepackage[shorthands=off,main=spanish]{babel}
\else
  % load polyglossia as late as possible as it *could* call bidi if RTL lang (e.g. Hebrew or Arabic)
%   \usepackage{polyglossia}
%   \setmainlanguage[]{spanish}
%   \usepackage[french]{babel} % cjns1989 - 1.43 version of polyglossia on this system does not allow disabling the autospacing feature
\fi

\title{BAILÉN}
\author{Benito Pérez Galdós}
\date{}

\begin{document}
\maketitle

\hypertarget{i}{%
\chapter{I}\label{i}}

---Me hacen Vds. reír con su sencilla ignorancia respecto al hombre más
grande y más poderoso que ha existido en el mundo. ¡Si sabré yo quién es
Napoleón!, yo que le he visto, que le he hablado, que le he servido, que
tengo aquí en el brazo derecho la señal de las herraduras de su caballo,
cuando\ldots{} Fue en la batalla de Austerlitz: él subía a todo escape
la loma de Pratzen, después de haber mandado destruir a cañonazos el
hielo de los pantanos donde perecieron ahogados más de cuatro mil rusos.
Yo que estaba en el 17.\textsuperscript{o} de línea, de la división de
Vandamme, yacía en tierra gravemente herido en la cabeza. De veras creí
que había llegado mi última hora. Pues como digo, al pasar él con todo
su estado mayor y la infantería de la guardia, las patas de su caballo
me magullaron el brazo en tales términos que todavía me duele. Sin
embargo, tan grande era nuestro entusiasmo en aquel célebre día que
incorporándome como pude, grité: «¡Viva el Emperador!»

Decía estas palabras un hombre para mí desconocido, como de cuarenta
años, no malcarado, antes bien con rasgos y expresión de cierta
hermosura ajada aunque no destruida por la fatiga o los vicios; alto de
cuerpo, de mirada viva y sonrisa entre melancólica y truhanesca, como la
de persona muy corrida en las cosas del mundo y especialmente en las
luchas de ese vivir al par holgazán y trabajoso, a que conducen
juntamente la sobra de imaginación y la falta de dinero; persona de
ademanes francos y desenvueltos, de hablar facilísimo, lo mismo en las
bromas que en las veras; individuo cuya personalidad tenía acabado
complemento en el desaliño casi elegante de su traje, más viejo que
nuevo, y no menos descosido que roto, aunque todo esto se echaba poco de
ver, gracias a la disimuladora aguja que había corregido así las
rozaduras del chupetín como la ortografía de las medias.

Estas eran, si mal no recuerdo, negras, y el pantalón de color de clavo
pasado. Llevaba corto el pelo, con dos mechoncitos sobre ambas sienes,
sin polvo alguno, como no fuera el del camino: su casaca oscura y de un
corte no muy usual entre nosotros, su chaleco ombliguero, forma un poco
extranjera también, y su corbata informemente escarolada, le hacían
pasar como nacido fuera de España aunque era español. Mas por otra
circunstancia distinta de las singularidades de su vestir, causaba
sorpresa la persona de quien me ocupo, y este es un capitalísimo punto
que no debo pasar en silencio. Aquel hombre tenía bigote. Esto fue, ¿a
qué negarlo?, lo que más que otra cosa alguna, llamó mi atención cuando
le vi inclinado sobre la mesa, comiendo ávidamente en descomunal
escudilla unas al modo de sopas, puches o no sé qué endemoniado manjar,
mientras amenizaba la cena, contando entre cucharada y cucharada las
proezas de Napoleón I. Dos personas, ambas de edad avanzada y de
distinto sexo, componían su auditorio: el varón, que desde luego me
pareció un viejo militar retirado del servicio, oía con fruncido ceño y
taciturnamente los encomios del invasor de España; pero la señora
anciana, más despabilada y locuaz que su consorte, contestaba e
interrumpía al panegirista con cierto desenfado tan chistoso como
impertinente.

---Por Dios, Sr.~de Santorcaz---decía la vieja,---no grite Vd. ni hable
tales cosas donde le puedan oír. Mi marido y yo, que ya le conocemos de
antes, no nos espantamos de sus extravagancias; pero ¡ay!, la vecindad
de esta casa es muy entrometida, muy enredadora, y toda ella no se ocupa
más que de chismes y trampantojos. Como que ayer las niñas de la
bordadora en fino, que vive en el cuarto núm. 8, llegaron pasito a
pasito a nuestra puerta para oír lo que Vd. decía cuando nos contaba con
desaforados gritos lo que pasó allá en las Asturias en la batalla de
Pirrinclum, o no sé qué\ldots{} pues esos enrevesados nombres no se han
hecho para mi lengua\ldots{} Esta mañana, cuando Vd. entró de la calle,
la comadre del núm. 3 y la mujer del lañador, dijeron: «Ahí va el pícaro
\emph{flamasón} que está en casa del Gran Capitán. Apuesto a que es
espía de la \emph{canalla}, para ver lo que se dice en esta casa y
contarlo a sus mercedes». El mejor día nos van a dar que sentir, porque
como dice Vd. esas cosas y tiene esos modos, y hace ascos de la comida
cuando tiene azafrán, y siempre saca lo que ha visto en las tierras de
allá, le traen entre ojos, y sabe Dios\ldots{} Como aquí están tan
rabiosos con lo del día 2\ldots{}

---Ya se aplacarán los humos de esta buena gente---dijo Santorcaz,
apartando de sí escudilla y cuchara.---Cuando se organicen bien los
cuerpos de ejército y venga el Emperador en persona a dirigir la guerra,
España no podrá menos de someterse, y esto que es la pura verdad lo digo
aquí para entre los tres, de modo que no lo oigan nuestras camisas.

---España no se somete, no señor, no se somete---exclamó de improviso el
anciano quebrantando el voto de su antes silenciosa prudencia, y
levantándose de la silla para expresar con frases y gestos más
desembarazados los sentimientos de su alma patriota.---España no se
somete, Sr.~D. Luis de Santorcaz, porque aquí no somos como esos
cobardes prusianos y austriacos de que Vd. nos habla. España echará a
los franceses, aunque los manden todos los emperadores nacidos y por
nacer, porque si Francia tiene a Napoleón, España tiene a Santiago, que
es además de general un santo del cielo. ¿Cree Vd. que no entiendo de
batallas? Pues sí: soy perro viejo, y callos tengo en los oídos de tanto
escuchar el redoblar de los tambores y los tiros de cañón.

---No te sofoques, Santiago---dijo apaciblemente la anciana,---que ya
andas en los tres duros y medio y aunque yo creo como tú que España no
bajará la cabeza, no es cosa de que te dé el reuma en la cara por lo que
hable este mala cabeza de Santorcaz.

---Pues lo digo y lo repito---añadió el viejo soldado.---Venir a
hablarme a mí de cuerpos de ejército, y de brigadas de caballería y de
cuadros\ldots{}

---¿En qué batallas se ha encontrado Vd.?---preguntó con sonrisa burlona
Santorcaz.

---¡Que en qué batallas me encontré!---exclamó D. Santiago Fernández
cuadrándose ante su interpelante y mirándole con el desprecio propio de
los grandes genios al ver puesta en duda su superioridad.---¿Pues no
sabe todo el mundo que fui asistente del señor marqués de Sarriá el año
1762 cuando aquella famosa campaña de Portugal, que fue la más terrible
y hábil y estratégica que ha habido en el mundo, así como también digo
que después de Alejandro el Macedonio no ha nacido otro marqués de
Sarriá?\ldots{} ¡Qué cosas tiene este caballerito! ¡Preguntar en qué
acciones me he encontrado! Aquella fue una gran campaña, sí señor;
entramos en Portugal, y aunque al poco tiempo tuvimos que volvernos,
porque el inglés se nos puso por delante, se dieron unas
batallas\ldots{} ¡qué batallitas, mi Dios! Yo era asistente del señor
marqués, y todas las mañanas le hacía los rizos y le empolvaba la
peluca, de tal modo que la cabeza de nuestro general parecía un sol. Él
me decía: «Santiago, ten cuidado de que los rizos vayan parejos, y que
uno de otro no discrepen ni el canto de un duro, porque no hay nada que
aterre tanto al enemigo como la conveniencia y buen parecer de nuestras
personas». ¡Y cuánto le querían los soldados! Como que en toda aquella
guerra apenas murieron tres o cuatro.

Santorcaz al oír esto se desternillaba de risa, haciendo subir de punto
con sus irreverentes manifestaciones el enfado de D. Santiago Fernández,
el cual, dando una fuerte puñada en la mesa, continuó así:

---¿Qué valen todos los generales de hoy, ni los emperadores todos,
comparados con el marqués de Sarriá? El marqués de Sarriá era partidario
de la táctica prusiana, que consiste en estarse quieto esperando a que
venga el enemigo muy desaforadamente, con lo cual este se cansa pronto y
se le remata luego en un dos por tres. En la primera batalla que dimos
con los aldeanos portugueses, todos echaron a correr en cuanto nos
vieron, y el general mandó a la caballería que se apoderara de un hato
de carneros, lo cual se verificó sin efusión de sangre.

---No, no ha habido en el mundo batallas como esas, Sr.~D.
Santiago---dijo Santorcaz moderando su risa;---y si Vd. me las cuenta
todas, confesaré que las que yo he visto son juegos de chicos. Y como
desde aquella fecha ha conservado Vd. los hábitos de campaña, y gusta
tanto de conversar sobre el tema de la guerra, los vecinos le llaman el
Gran Capitán.

---Ese es un mote, y a mí no me gustan motes---dijo doña Gregoria, que
así se llamaba la mujer del valiente expedicionario de
Portugal.---Cuando nos mudamos aquí, y dieron los vecinos en llamarte
Gran Capitán, bien te dije que alzaras la mano y regalaras un bofetón al
primero que en tus propias barbas te dijera tal insolencia; pero tú con
tu santa pachorra, en vez de llenarte de coraje se te caía la baba
siempre que los chicos te saludaban con el apodo, y ahora Gran Capitán
eres y Gran Capitán serás por los siglos de los siglos.

---Yo no me paro en pequeñeces---dijo D. Santiago Fernández,---y aunque
tolero un apodo honroso, no consiento que nadie se burle de mí. A fe, a
fe, que cuando uno ha servido en las milicias del Rey por espacio de
veinte años, cuando uno ha estado en la campaña de Portugal, cuando uno
ha tenido también el honor de encontrarse en la expedición de Argel que
mandó el Sr.~D. Alejandro O'Reilly en 1774; cuando después de tan
gloriosas jornadas se le han podrido a uno las nalgas sentado en la
portería de la oficina del Detall y cuenta y razón del arma de
artillería, viendo entrar y salir a los señores oficiales, y haciéndoles
un recadito hoy y otro mañana, bien se puede alzar la cabeza y decir una
palabra sobre cosas militares.

---Eso mismo digo yo---indicó doña Gregoria.---Bien saben todos que tú
no eres ningún rana, y que has escupido en corro con guardias de Corps y
walonas y generales de aquellos que había antes, tan valientes que sólo
con mirar al enemigo le hacían correr.

---Y no se trate---prosiguió el Gran Capitán,---de embobarnos con
cuentos de brujas como los que desembucha el Sr.~de Santorcaz. A las
niñas del lañador y a doña Melchora, la que borda en fino, les puede
trastornar el seso este caballero contándoles esas batallas fabulosas de
prusianos y rusos, con lo de que si el Emperador fue por aquí o vino por
allí. Hombres como yo no se tragan bolas tan terribles, ni ha estado uno
veinte años mordiendo el cartucho y peinando los rizos del señor marqués
de Sarriá, para dar crédito a tales novelas de caballerías. Conque ¿cómo
fue aquello?---añadió en tono de mofa y sentándose junto a
Santorcaz.---Dijo Vd. que cuatro mil franceses atacaron a la bayoneta a
diez mil rusos y los hicieron caer en un pantano donde se ahogaron la
mitad. Pues ¡y lo de que rompieron el hielo a cañonazos para que se
hundieran los enemigos que estaban encima!\ldots{} ¡Bonito modo de hacer
la guerra! Pero hombre de Dios, si andaban por sobre el hielo se
resbalarían y\ldots{} pobres nalgas del Emperador\ldots{} digo, de los
tres emperadores, pues ahí dice Vd. que eran tres nada menos. ¿Sabes,
Gregoria, que es aprovechada la familia?

El Gran Capitán hizo reír a su digna esposa con estos chistes, hijos de
su inexperta fatuidad, y ambos celebraron recíprocamente sus
ocurrencias.

---Si es novela de caballerías lo que he contado---dijo
Santorcaz,---pronto lo hemos de ver en España, porque pasan de cien mil
los Esplandianes que andan desparramados por ahí esperando que su amo y
señor les mande empezar la función.

---¡Los asesinos de Madrid!---exclamó el Gran Capitán inflamándose en
patriótico ardor.---¿Y cree Vd. que les tenemos miedo? ¡Santa María de
la Cabeza! Ya veo que están fortificando el Retiro, y que no permiten
que vuele una mosca alrededor de sus señorías; pero ya hablaremos. Esto
es ahora, porque estamos sin tropa; pero ¿sabe Vd. lo que se va a formar
en Andalucía?, un ejército. ¿Y en Valencia?, otro ejército. Y en Galicia
y en Castilla, otro y otro ejército. ¿Cuántos españoles hay en España,
Sr.~de Santorcaz? Pues ponga Vd. en el tablero tantos soldados como
hombres somos aquí, y veremos. ¿A que no sabe Vd. lo que me ha dicho hoy
el portero de la secretaría de la Guerra? Pues me ha dicho que mi pueblo
ha declarado la guerra a Napoleón. ¿Qué tal?

---¿Cuál es el pueblo de Vd.?

---Valdesogo de Abajo. Y no es cualquier cosa, pues bien se pueden
juntar allí hasta cien hombres como castillos, no como esos rusos de
alfeñique de que Vd. habla, sino tan fieros, que despacharán un
regimiento francés como quien sorbe un huevo.

---Pues una mujer que ha venido hoy de la sierra---dijo doña
Gregoria,---me ha contado que también mi pueblo va a declarar la guerra
a ese ladrón de caminos, sí, Sr.~de Santorcaz, mi pueblo, Navalagamella.
Y allí no se andarán con juegos, sino al bulto derechitos. Si esos
pueblos que Vd. nombra, las Austrias y las Prusias fueran como
Navalagamella, la \emph{canalla} no los hubiera vencido, y se conoce que
todos los austriacos y prusiacos son gente de mucha facha y nada más.

---No se dice prusiacos, sino prusianos---indicó enfáticamente a su
esposa el Gran Capitán.

---Bien, hombre; los rusos y los prusos, lo mismo da. Lo que digo es que
si Valdesogo de Abajo y Navalagamella, que son dos pueblos como dos
lentejas comparados con la grandeza de todo el Reino, se ponen en ese
pie, los demás lugares y ciudades harán lo mismo, y entonces, áteme esa
mosca el Sr.~de Santorcaz. No, no quedará un francés para contarlo, y la
que hicieron aquí a primeros del mes, la pagarán muy cara. ¿Hase visto
alguna vez bribonada semejante? ¡Fusilar en cuadrilla a tantos
pobrecitos, sin perdonar a sacerdotes ancianos, a inocentes doncellas y
a infelices muchachos como el que está en esa cama! ¡Ay! Vd. no vio
aquello, Sr.~de Santorcaz, porque llegó a Madrid tres días después;
¡pero si Vd. lo hubiera visto! Por esta calle del Barquillo pasaron esas
fieras, y como les arrojaron algunos ladrillos desde los andamios de la
casa que se está fabricando en la esquina, mataron a una pobre mujer que
pasaba con un niño en brazos. Al ver esto, todas las vecinas de la casa
que estábamos en los balcones, empezamos a tirarles cuanto teníamos. Una
les echaba una cazuela de agua hirviendo, otra la sartén con el aceite
frito; yo cogí el puchero que había empezado a cocer, y sin pensarlo
dije allá va, y aunque aquel día nos quedamos sin comer, no me pesó, no
señor. Después entre Juanita la lañadora, las niñas de al lado y yo,
cogimos una cómoda y echándola a la calle aplastamos a uno. Querían
subir a matarnos; pero ¡quia! Todo facha, nada más que facha. Más de
cuarenta mujeres nos apostamos en la escalera, unas con tenedores, otras
con tenacillas, estas con asadores, aquella con un berbiquí, estotra con
una vara de apalear lana. Si llegan a subir les hacemos pedazos. Mi
marido tomó aquella lanza vieja que tiene allí desde las tan famosas
guerras, y poniéndose delante de nosotras en la escalera nos arengó, y
dispuso cómo nos habíamos de colocar. ¡Ah, si llegan a subir esos
perros! Yo era la más vieja de todas, y la más valiente aunque me esté
mal el decirlo. Mi marido quería salir a la calle al frente de todas
nosotras; pero le convencimos de que esto era una locura. Con su carga
de setenta a la espalda, él hubiera partido de un lanzazo a cuantos
mamelucos encontrara en la calle. ¡Ay qué día! Cuando nos retiramos cada
una a nuestro cuarto, en toda la casa no se oía más que «¡viva el Gran
Capitán!»

---¡Qué día!---exclamó melancólicamente Fernández, disimulando el
legítimo orgullo que el recuerdo de sus proezas le causara.---A eso de
las ocho de la mañana vi salir de la oficina al capitán D. Luis Daoíz.
El día anterior me había mandado por unas botas a la zapatería de la
calle del Lobo, y desde allí se las llevé a su casa en la calle de la
Ternera, y cuando volví después de hacer el mandado, viendo que había
cumplido con la puntualidad y el esmero que son en mí peculiar, me dio
dos reales, que guardo en este pañuelo como memoria de hombre tan
valiente.

Diciendo esto, trajo un pañuelo y desdoblando una de las puntas
despaciosamente, y como si se tratara de la más vulnerable y santa
reliquia, sacó una moneda de plata que puso ante la vista de Santorcaz
sin permitirle que la tocara.

---Esto me dió---añadió enjugando con el mismísimo pañuelo las lágrimas
que de improviso corrieron de sus ojos;---esto me dio con sus propias
manos aquel que vivirá en la memoria de los españoles mientras haya
españoles en el mundo. Yo estaba barriendo la oficina cuando entró D.
Pedro Velarde buscándole y le dije: «Mi capitán, hace un rato que salió
con D. Jacinto Ruiz». Después D. Pedro entró y estuvo disputando con el
coronel: al cabo de un cuarto de hora volvió a pasar por delante de mí.
¡Quién me había de decir\ldots!

El Gran Capitán no pudo continuar, porque la pena ahogaba su voz; doña
Gregoria se llevó también la punta del delantal sucesivamente a sus dos
ojos, y Santorcaz más serio y grave que antes respetaba el dolor de sus
dos amigos.

---Me han asegurado---dijo después de una pausa,---que ese D. Pedro
Velarde iba a comer todos los días en casa de Murat. ¿Es que simpatizaba
con los franceses?

---No, no; y quien lo dijere miente---exclamó don Santiago, dejando caer
de plano sobre la mesa sus dos pesadísimas manos.---D. Pedro Velarde
pasaba por un oficial muy entendido en el arma, y como fue de los que el
Rey envió a Somosierra a recibir al \emph{melenudo}, éste le trató, supo
conocer sus buenas dotes y quiso atraérselo. ¡Bonito genio tenía D.
Pedro Velarde para andarse con mieles! Le convidaban a comer,
obsequiábanle mucho; pero bien sabían todos que si nuestro capitán
pisaba las alfombras de aquel palacio era \emph{para conocer más de
cerca a la canalla}, como él mismo decía.

---Él y sus compañeros de Monteleón---dijo Santorcaz,---demostraron un
valor tanto más admirable, cuanto que es completamente inútil. Aquí
están ciegos y locos. Creen que es posible luchar ventajosamente contra
las tropas más aguerridas del mundo, sin otros elementos que un ejército
escaso, mal instruido, y esas nubes de paisanos que quieren armarse en
todos los pueblos. La obstinación ridícula de esta gente hará que sean
más dolorosos los sacrificios, y el número de víctimas mucho más grande,
sin que puedan vanagloriarse al morir de haber comprado con su sangre la
independencia de la patria. España sucumbirá, como han sucumbido Austria
y Prusia, Naciones poderosas que contaban con buenos ejércitos y Reyes
muy valientes.

---¡Esos países no tienen vergüenza!---exclamó con furor D. Santiago
Fernández, levantándose otra vez de su asiento.---En Austria y Prusia
habrá lo que Vd. quiera; pero no hay un Valdesogo de Abajo, ni un
Navalagamella.

Discretísimo lector: no te rías de esta presuntuosa afirmación del Gran
Capitán, porque bajo su aparente simpleza encierra una profunda verdad
histórica.

Santorcaz soltó de nuevo la risa al ver el acaloramiento de su amigo,
cuyas patrióticas opiniones apoyó de nuevo su esposa, hablando así:

---Aquí somos de otra manera, Sr.~de Santorcaz. Usted viviendo por allá
tanto tiempo, se ha hecho ya muy extranjero y no comprende cómo se toman
aquí las cosas.

---Por lo mismo que he estado fuera tanto tiempo, tengo motivos para
saber lo que digo. He servido algunos años en el ejército francés;
conozco lo que es Napoleón para la guerra, y lo que son capaces de hacer
sus soldados y sus generales. Cien mil de aquellos han entrado en España
al mando de los jefes más queridos del Emperador. ¿Saben Vds. quién es
Lefebvre? Pues es el vencedor de Dantzig. ¿Saben Vds. quién es Pedro
Dupont de l'Etang? Pues es el héroe de Friedland. ¿Conocen Vds. al duque
de Istria? Pues es quien principalmente decidió la victoria de Rívoli.
¿Y qué me dicen de Joaquín Murat? Pues es el gran soldado de las
Pirámides, y el que mandó la caballería en Marengo\ldots{}

---No, no le nombre Vd.---dijo doña Gregoria,---porque si todos los
demás son como ese de las \emph{melenas}, buena gavilla de perdidos ha
metido Napoleón en España.

---Sr.~de Santorcaz---añadió con grave comedimiento el Gran
Capitán,---ya sabe Vd. que un hombre como yo, testigo de cien combates,
no se traga ruedas de molino, y todas esas heroicidades del general
Pitos y del general Flautas las vamos a ver de manifiesto ahora, sí
señor. Y supongo que Vd. habrá venido para ponerse de parte de ellos,
pues quien tanto les alaba y admira, es natural que les
ayude.---No---repuso Santorcaz;---yo he vuelto a España para un asunto
de intereses, y dentro de unos días partiré para Andalucía. Cuando
arregle mi negocio, me volveré a Francia.

\hypertarget{ii}{%
\chapter{II}\label{ii}}

---¡Qué mal hombre es Vd.!---exclamó doña Gregoria.---Y su pobre padre,
y toda la familia llorando su ausencia, y muertos de pena sin poder
traer al buen camino a este calaverilla que durante quince años y desde
aquella famosa aventura\ldots{} Pero chitón---añadió volviendo la cara
hacia mí;---me parece que el chico se ha despertado y nos está oyendo.

Los tres me miraron y yo observé claramente cuanto me rodeaba, pudiendo
apreciarlo todo sin mezcla de vagas imágenes, ni mentirosas visiones.
Hallábame en una cama, de cuyo durísimo colchón daban fe las
mortificaciones de mis huesos y la instintiva tendencia de mi cuerpo a
arrojarse fuera de ella, mientras uno de mis brazos, fuertemente vendado
se negaba a prestarme apoyo, tan inmóvil y rígido como si no me
perteneciera. Asimismo rodeaba mi cabeza complicado turbante de trapos
que olían a ungüentos y vinagre, y mi débil y extenuado cuerpo sentía
por aquí y por allí terribles picazones. El lecho en que yacía tan
incómodamente ocupaba el rincón del cuarto, el cual era de ordinarias
dimensiones, con blancos muros y suelo de ladrillos, mal cubiertos por
una vieja y acribillada estera de esparto. Algunas láminas de santos, a
quienes el artista grabador había dado nuevo martirio en sus impíos
troqueles, adornaban la desnuda pared, en uno de cuyos testeros
ostentaba su temerosa longitud la lanza del Gran Capitán. En el centro
de la pieza hallábase la mesa, que sostenía un candil de cuatro
mecheros, y junto a ella sentados en sendas sillas de cuero, que
lastimosamente gemían al menor movimiento, estaban los tres personajes
cuya conversación hirió mis oídos cuando volví de un largo paroxismo.

Todos fijaron en mí la atención, y doña Gregoria, acercándose
maternalmente a mi cama, me habló así:

---¿Estás despierto, niño? ¿Ves y entiendes? ¿Puedes hablar? Pobrecito:
ya se te ha quitado la terrible calentura, y el Santo Ángel de tu Guarda
ha conseguido del Padre Eterno que te otorgue el seguir viviendo. ¿Cómo
estás? ¿Nos ves a los que estamos aquí? ¿Nos conoces? ¿Entiendes lo que
decimos? Debes de estar bien, porque ya no dices desatinos, ni quieres
echarte de la cama, ni nos insultas, ni dices que nos vas a matar, ni
llamas a D. Celestino ni a la doña Inés, que te traían trastornado el
juicio. Estás bien, ya estás fuera de peligro, y vivirás, pobre niño;
pero ¿has perdido la razón, o Dios quiere que te veamos en tu ser
natural, sano y completo y cuerdo, tal y como estabas, antes de que
aquellos caribes\ldots?

---Y en verdad, no sé cómo ha escapado el infeliz---dijo Fernández a
Santorcaz.---Tres balazos tenía en su cuerpecito: uno en la cabeza el
cual no es más que una rozadura, otro en el brazo izquierdo, que no le
dejará manco, y el tercero en un costado, y en parte sensible, tanto que
si no le hubieran sacado la bala, no le veríamos ahora tan despiertillo.

Aquellas bondadosas personas me instaron para que hablase, mostrándoles
que mi razón, como mi cuerpo, se había repuesto de la tremenda crisis a
que estuviera sujeta. También acudió con cariñosa solicitud a darme
alimento la ejemplar doña Gregoria, y tomado aquel ávidamente por mí, me
sentí muy bien. ¿Había resucitado o había nacido en aquella noche?

---Ahora, chiquillo, estate tranquilo---continuó doña Gregoria
sentándose a mi lado.---¡Cuánto se va a alegrar el Sr.~Juan de Dios
cuando te vea!

---¡Cómo!---exclamé con la mayor sorpresa.---¿Juan de Dios vive aquí?
¿Pues en dónde estoy? ¿Y ustedes quiénes son? ¿Qué ha sido de Inés?

---¡Otra vez Inés! Este joven no está todavía bueno. Dejémonos de Ineses
y a descansar.

Santorcaz se llegó a mí, y mostrándome algún interés, me dijo:

---¡Pobrecito!, ¡con que te fusilaron! El gran duque de Berg es hombre
terrible y sabe sentar la mano. Dicen que mataste más de veinte
franceses. Ya me contarás tus hazañas, picarón. Y di, ¿tienes ánimos de
volver a hacer de las tuyas? Me parece que no\ldots{} porque habrás
visto que esa gente gasta unas bromas un poco pesadas.

Dicho esto, Santorcaz, tomando su capa, se marchó.

La sensación que yo experimentaba al verme allí, tornado nuevamente y de
improviso, según mi entender, a la vida; en presencia de personas
desconocidas y volviendo sin cesar al pasado mi pensamiento recién
salido de una sombra profunda; las impresiones de mi alma, a quien el
repentino despertar después de un largo entumecimiento había dado cierta
actividad ansiosa, fueron causa de que no pudiera estar tranquilo como
me rogaban el Gran Capitán y su mujer. Hacíales mil preguntas diversas,
con la curiosidad del que volviendo al mundo después de un siglo de
muerte real, deseara conocer en un instante cuanto ha pasado en el
planeta durante su ausencia. A todo contestaban que me estuviese quieto
y sin cuidarme de nada, para que no me repitiesen los accesos de fiebre;
pero no pude conseguir este objeto, y si descansé un poco, procurando
poner a un lado mis terribles recuerdos y apartar de la vista las
siniestras figuras que se habían hecho compañeras inseparables de mi
espíritu, poco después, cuando, ya avanzada la noche, llegó Juan de
Dios, me sentí tan vivamente inquieto al verle, que a no impedírmelo mi
debilidad, habría saltado del lecho para correr hacia él, arrastrado por
un odio terrible y una curiosidad más fuerte aún que el odio. El antiguo
mancebo de D. Mauro Requejo estaba tan demacrado, tan excesivamente
amarillo y mustio, que parecía haber vivido diez años de penas en el
trascurso de algunos días. Sus ojos encendidos conservaban huellas de
recientes lágrimas, y su desmadejado cuerpo se movía con pesadez, como
si le fatigara su propio peso. Arrojose en una silla junto a mi cama,
cuando los dos ancianos se retiraban a su aposento, y me habló así:

---Gabriel, ¿ya estás bueno? ¿Has recobrado el juicio? ¿Entiendes lo que
se te dice?

---¿Dónde está Inés?---le pregunté con ansiedad.

---¡Oh, desgraciado de mí!---exclamó ocultando el rostro entre las
manos.---Tú estás enfermo todavía, y si te doy la noticia\ldots{} ¿Que
dónde está Inés? Espántate, Gabriel, porque no lo sé. Yo estoy loco, yo
estoy imbécil. Llevo quince días de dolores que a nada son comparables.
Las lágrimas que he derramado podrían agujerar una peña. Ahora
mismo\ldots{} ¿de dónde crees que vengo? Pues vengo de la bóveda de San
Ginés, adonde voy todas las noches a mortificarme el cuerpo con
disciplinazos, por ver si Dios se apiada de mí y me devuelve lo que me
quitó, sin duda en castigo de mis grandes pecados.

Después de enjugar sus lágrimas y sonarse con estrépito, continuó así:

---Yo saqué a Inés de la huerta del Príncipe Pío. ¡Ay!, si no te
salvaste también tú, fue porque no pude, que bien lo intenté; te juro
que lo intenté. Inés se desmayó, y no pudiendo traerla aquí, por ser
esto muy lejos, Lobo me indujo a llevarla a casa de unas que él llamaba
honradísimas señoras, donde permanecería hasta tanto que fuera posible
traerla aquí para casarme con ella\ldots{} ¡Oh, infame legista,
miserable enredador, tramposo y falsario! Inés me abofeteó, Gabriel, al
verse en aquella casa, y me clavó en las mejillas sus deditos. No puedes
formarte idea de las palabras tiernas que le dije para que se calmara,
pero nada podía consolarla de que no os hubierais salvado también tú y
el buen sacerdote. En vano le dije que sería mi mujer; en vano le dije
que la adoraba con profundísimo amor; también le mostré mi dinero,
prometiéndole gastar una buena parte en huir para siempre de Madrid y de
España si así lo deseaba. ¡Infeliz de mí!, a estas irrecusables pruebas
de mi cariño, sólo contestaba llamándome bestia y ordenándome que se le
quitara de delante\ldots{} A cada instante te llamaba, y luego se
deshacía en lágrimas, y quería después arrojarse fuera de la casa para
volver a la Montaña. A pesar de esto yo era feliz, porque la tenía en
mis brazos, apartábale de la frente los desordenados cabellos, y con mi
pañuelo limpiaba sus lágrimas divinas, con las cuales se refrescarían,
si las bebieran, los condenados del infierno\ldots{} El pérfido Lobo no
se apartaba de allí, y desde luego me parecieron sospechosos el esmero y
solicitud con que la atendía. Inés no cesaba un momento de gemir, y
tanto a mi compañero como a mí nos mostraba mucha repugnancia,
ordenándonos que la dejáramos sola, porque no quería vernos, y que la
matáramos, porque no quería vivir. Su desesperación llegó a tal punto
que no la podíamos contener, y se nos escapaba de entre los brazos,
diciendo que pues no le era posible salvaros la vida, quería ir a daros
a entrambos sepultura. Por último, a fuerza de ruegos logramos calmarla
un poco, prometiéndole yo acudir al lugar del suplicio a cumplir tan
triste obligación. Cuando esto le dije, me miró con tanta ternura, y
después me lo ordenó de un modo tan persuasivo, tan elocuente, que no
vacilé un instante en hacer lo prometido y salí dejándola al cuidado de
Lobo. ¡Nunca tal hiciera y maldito sea el instante en que me separé de
aquel tesoro de mi vida, de aquel imán de mi espíritu! Gabriel, corrí a
la Moncloa, me acerqué a los grupos en que eran reconocidos los
cadáveres, y anduve de un lado para otro esperando encontrarte entre
aquellos que, abandonados hasta en tan triste ocasión, no tenían quien
formara a su alrededor concierto de llantos y exclamaciones\ldots{} Al
fin encontré al sacerdote; pero tú no estabas a su lado, pues unas
mujeres compasivas, habiendo notado que vivías, te habían llevado a un
paraje próximo para prodigarte algunos cuidados. Grande fue mi alegría
cuando te vi abrir los ojos, cuando te oí pronunciar algunas frases
oscuras, y observé que tus heridas no parecían de mucha gravedad; así es
que en cuanto dimos sepultura a tu buen amigo, me ocupé de los medios de
traerte a mi casa. Rogué a aquellas mujeres que te cuidaran un momento
más, mientras yo volvía con una camilla, y al salir de la huerta, me
regocijaba con la idea de participar a Inés que estabas vivo. «¡Cuánto
se va a alegrar la pobrecita!» decía para mí, y yo me alegraba también,
porque había comprendido por sus palabras que aquella flor de Jericó te
apreciaba bastante ¿no es verdad? ¡Ay!, Gabriel, tú hubieras sido
nuestro criado, tú nos hubieras servido fielmente, ¿no es
verdad?\ldots{} Pues bien, hijo, como te iba diciendo, corrí desalado a
comunicarle la feliz nueva de tu salvación, y cuando entré en la casa
donde la había dejado, Inés ya no estaba allí. Aquellas señoras
desconocidas dijéronme que Lobo se había llevado a la muchacha, y como
yo les manifestara mi extrañeza e indignación, llamáronme estúpido y me
arrojaron de su casa. Volé a la de ese miserable ladrón; mas no le pude
ver ni en todo aquel día ni en los siguientes. Figúrate mi
desesperación, mi agonía, mi locura; yo no sé cómo no entregué el alma a
Dios en aquellos días, porque además de mi gran pena, me consumía una
fuerte calentura, a consecuencia de la herida de esta mano, pues bien
viste que perdí dedo y medio en la calle de San José\ldots{} ¿Crees que
me curaba? Ni por pienso. Después que el boticario de la Palma Alta me
vendó la mano, no volví a acordarme de tal cosa, y no digo yo dedo y
medio, ¡sino los cinco de cada mano me hubiera yo arrancado con los
dientes, con tal de hallar a mi idolatrada Inés, a aquella rosa
temprana, a aquel jazmín de Alejandría! Durante este tiempo no me olvidé
de ti, pues el mismo día 3 te hice conducir a esta casa, que es la mía,
en la cual has permanecido hasta hoy, y donde, gracias a los cuidados de
tan buena gente, has recobrado la salud.

---¿Pero Lobo ha desaparecido también?---pregunté con afanoso
interés.---Si no ha desaparecido, ¿no puede obligársele a decir qué ha
hecho de Inés?

---Al cabo de diez días lo encontré al fin en su casa. ¿Sabes tú lo que
me dijo el muy embustero? Pues verás. Después de reírse de mí,
llamándome bobo y mentecato, me dijo que no pensara en volver a ver a
Inés, porque la había entregado a sus padres. «¿Pues acaso Inés tiene
padres?» le dije. Y él me contestó: «Sí, y son personas de las
principales de España, por lo cual he creído de mi deber entregarles la
infeliz muchacha, desde tanto tiempo condenada a vivir fuera de su rango
y entre personas de inferior condición». Me quedé atónito; pero al punto
comprendí que esto era invención de aquel inicuo tramposo embaucador, y
en mi cólera le dije las más atroces insolencias que han salido de estos
labios\ldots{} ¿No crees tú como yo que lo de entregarla a sus
desconocidos padres es pura fábula de Lobo, para ocultar así su crimen?
Gabriel, ¿no te estremeces de espanto como yo? ¿Dónde estará Inés?
¿Dónde la tendrá ese monstruo? ¿Qué habrá hecho de ella? ¡Ay! Yo la he
buscado sin cesar por todo Madrid, he pasado noches enteras junto a la
casa de la calle de la Sal examinando quién entraba y quién salía; he
dado dinero a los criados, aguadores, lavanderas, a los escribientes del
licenciado, a cuantas personas visitaban la casa; pero nadie me ha
sabido dar razón: nadie, nadie. ¿Es esto para desesperarse? ¿Es esto
para morirse de pena? ¡Trabajar tanto, cavilar tanto para sacarla del
poder de sus tíos, cometer grandes pecados, y exponer uno su alma a las
horribles penas del infierno, para ver desvanecida como el humo aquella
esperanza encantadora, aquella soñada dicha y suprema felicidad!\ldots{}
¿Será castigo de Dios por mis culpas, Gabriel? ¿Lo crees tú así?
¿Apruebas lo que estoy haciendo ahora, que es rezar mucho y pedir a Dios
que me perdone, o que me devuelva a Inés, aunque no me perdone? ¿Crees
tú que concurriendo a la bóveda de San Ginés con gran constancia y
devoción, podré alcanzar de Dios alguna misericordia? ¡Ay! Si las
lágrimas que he derramado hubiesen caído todas en el corazón de ese
infame Lobo, habríanle atravesado de parte a parte haciendo el efecto de
un puñal. ¿Dónde está Inés? ¿Qué es de ella? ¿Vive o muere? Gabriel, tú
tienes ingenio, y Dios ha querido que recobres tu preciosa vida para que
desbarates los inicuos planes de ese monstruo, y devuelvas a Inés su
libertad, así como a mí la paz del alma que he perdido quizás para
siempre.

Así habló el afligido hortera, y oyéndole no pude menos de compadecerle
por los tormentos de su alma tan apasionada como inocente. No se cansó
de hablar hasta muy avanzada la noche, siempre sobre el mismo tema y con
iguales demostraciones dolorosas. Al fin, su voz se perdió para mí en el
vacío de un silencio profundo, porque me quedé dormido, cediendo mi
atención y curiosidad a la fatiga y flaqueza de ánimo que me consumían
aún.

\hypertarget{iii}{%
\chapter{III}\label{iii}}

A la mañana siguiente la primera persona que vieron mis ojos fue doña
Gregoria, a quien ya había empezado a tomar cariño, pues tan propio de
la caridad es inspirarlo en poco tiempo. La mujer del Gran Capitán
limpiaba la sala, procurando mover los trastos lentamente para no hacer
ruido, cuando desperté, y al punto lo dejó todo para correr a mi lado.

---Esa cara está respirando salud---me dijo.---Veremos lo que dice hoy
D. Pedro Nolasco cuando te vea.

---¿Y quién es ese D. Pedro Nolasco?---pregunté sospechando fuera el
citado varón algún médico afamado de la vecindad.

---¿Quién ha de ser, hijo? El albéitar, que vive en el cuarto número 14.
Aquí no gastamos médico, porque es bocado de príncipes. Y cuando
Fernández padece del reuma, le ve D. Pedro Nolasco, que es un gran
doctor. A él debes la vida, chiquillo, y él te sacó del costado la bala;
que si no, a estas horas estarías en el otro mundo.

Oído esto, le hice varias preguntas acerca de su condición y la calidad
de la casa, a las que satisfizo bondadosamente diciendo que su esposo
era portero en una oficina del ramo de la Guerra, y que con su sueldo, y
lo que el Sr.~Juan de Dios les daba por su modesto pupilaje, pasaban la
vida pobres y contentos.

---Esta no es casa de huéspedes, porque nosotros no queremos
barullo---añadió,---pero hace mucho tiempo que conocemos al Sr.~de
Arroiz y por eso le tenemos aquí. Este Sr.~de Santorcaz que has visto
anoche y que no ha de tardar en venir, es un joven a quien conocimos en
Alcalá, cuando estábamos allí establecidos, y él corría la tuna en
aquella célebre Universidad. Ha sido muy calavera, y sus padres no le
han vuelto a ver desde que se marchó a Francia hace quince años, huyendo
de una persecución muy merecida, a consecuencia de sus barrabasadas y
viciosas costumbres. ¡Desgraciado joven! Allá ha sido soldado, y cuando
nos cuenta sus trabajos y penalidades nos quedamos como si oyéramos leer
la novela \emph{El asombro de la Francia, Marta la Romarantina}, aunque
Santiago dice que todo lo que cuenta es mentira. A pesar de es un
tarambana, nosotros apreciamos a este mala cabeza de Santorcaz, y él no
nos quiere mal; así es que cuando se aparece por España, siempre viene a
parar a nuestra casa, donde le damos hospitalidad por bien poco dinero.
¡Ay!, sí, por bien poco dinero: verdad es que si le pidiéramos mucho, el
infeliz no podría dárnoslo, porque no lo tiene. Y no es porque haya
nacido de las yerbas del campo, pues su familia a un buen solar de
tierra de Salamanca pertenece: sólo que como no es primogénito\ldots{}
su padre se empeñó en dedicarle a la Iglesia, y el pobre chico no tenía
afición de misacantano\ldots{}

Estábamos doña Gregoria y yo enfrascados en este coloquio que no dejaba
de interesarme, cuando volviendo de su oficina D. Santiago Fernández,
quitose gravemente el pesado uniforme, que su consorte colgó en la
percha no lejos de la amenazadora lanza, y se dispuso a comer.

---Grandes noticias te traigo, mujer---dijo con retozona sonrisa,
sentado ya en el sillón de cuero y con ambas manos posadas en las
respectivas rodillas, mientras con lento compás movía el cuerpo.---Te
vas a poner más contenta\ldots{}

---No puede ser sino que el Gran Duque ha reventado ya de los cólicos
que padecía.

---No, no es eso, mujer. ¿Quién te dijo que Navalagamella le había
declarado la guerra a la \emph{canalla}? No es Navalagamella sólo,
mujer, es Asturias, León, Galicia, Valencia, Toledo, Burgos, Valladolid,
y se cree que también Sevilla, Badajoz, Granada y Cádiz. En la oficina
lo han dicho, y si vieras cómo están todos bailando de contento. Oficial
conozco que no ha dormido en toda la noche esperando el correo, y si
supieras, mujer\ldots{} A ti te lo puedo decir, y no importa que lo oiga
este chico. Oye, oíd los dos: muchos oficiales se han fugado, sin que en
los cuarteles, ni en sus casas se sepa dónde están. Y dirás tú, «¿pues
dónde están?» Yo lo sé, sí señora, yo lo sé: se han ido a unirse a los
ejércitos españoles que se están formando\ldots{} ¿a que no sabes dónde
se están formando? Pues yo lo sé, sí señora, yo lo sé: uno se está
formando en Valladolid, y lo mandará D. Gregorio de la Cuesta: otro en
Asturias y Galicia, que corre a cargo de Blake\ldots{} y el
tercero\ldots{} Esta es la más gorda de todas: ¿te la digo?

---Hombre sí, dila: no nos dejes a media miel.

---Pues se dice por ahí que las tropas de Andalucía se sublevarán, sí
señor, se sublevarán. Pues no se han de sublevar. Si en cuanto uno dé la
voz empieza a desfilar nuestra gente, y ni un ranchero español quedará a
las órdenes de Murat, ni de la Junta.

---Veo que lo van a pasar mal, Santiago. Pero siento golpes en la
puerta. Son los vecinos que vienen a saber noticias\ldots{} Pase Vd.,
Sr.~D. Roque; pasen ustedes niñas; pase Vd. Sr.~de Cuervatón.

Abrió doña Gregoria la puerta y penetraron en ordenada falange como una
docena de personas de uno y otro sexo, y de diferentes edades y fachas,
las cuales personas eran los vecinos más adictos a la simpática persona
del Gran Capitán, y además entusiastas creyentes de sus noticias, por lo
cual acudían todas las mañanas cuando aquel regresaba de la oficina, con
el anhelo de saciar en la fuente más pura y cristalina la ardorosa
curiosidad que entonces devoraba a los habitantes de Madrid. ¿Debo
detenerme en enumerar a tan dignas personas? ¿Para qué, si el lector no
necesita conocer al lañador, ni al talabartero, ni tampoco a D. Roque,
el arruinado comerciante, ni al Sr.~de Cuervatón, ni menos a las niñas
de la bordadora en fino? Dejémosles envueltos en el velo de su discreto
incógnito, y oigamos a Fernández, que desbordándose de su propio ser, a
causa de la exorbitante hinchazón de su orgulloso júbilo, iba contando
lo que oyera, sin dejar de aderezar sus relatos con la sal y pimienta de
la exageración.

---Pues en Andalucía---dijo,---en Andalucía\ldots{} ya saben Vds. dónde
está Andalucía; como si dijéramos en Cádiz\ldots{} pues. Dicen que la
Junta de Sevilla ha armado un gran ejército, con las tropas que estaban
en San Roque. ¿Saben Vds. lo que es San Roque? Pues es como si
dijéramos\ldots{} supongan Vds. que aquí está Gibraltar, pues aquí
abajito está San Roque.

---Este D. Santiago lo sabe todo.

---Ya, como quien ha visto tantas tierras, y ha estado en tantas
batallas.

---En San Roque están las mejores tropas de España, tanto en infantería
como en artillería y caballos; de modo que si se forma ese ejército, y
viene sobre Madrid\ldots{} ¡Jesús!

---¡Jesús!---repitió un coro de diez voces.

---¿Vd. cree que vendrá sobre Madrid?---preguntó uno de los
concurrentes.

---Eso es lo que no puedo asegurar---repuso con énfasis el Gran
Capitán.---Pero a lo que yo entiendo y según la experiencia que adquirí
en aquellas terribles guerras, me atrevo a decir que el ejército de
Andalucía viene sobre Madrid, y si hace lo mismo el de don Gregorio de
la Cuesta, juzguen Vds. el susto que pasarán los franceses. Hay que
guardar el secreto: mucho cuidado, señores, y Vds., niñas, guárdense muy
bien de ir contando estas cosas cuando vayan a la costura, porque puede
llegar a oídos del gran duque de Berg\ldots{} Yo creo que pasará lo
siguiente. El ejército de Andalucía vendrá a la Mancha: los franceses
irán a batirlos, dejando libre a Madrid, donde entrará D. Gregorio de la
Cuesta, el cual si sigue después hacia el Mediodía, les picará la
retaguardia por Tarancón, y como al mismo tiempo los de allí le harán
retroceder hacia el Tajo, viéndose los franceses atacados por todos
lados, por fuerza tendrán que caer en el río, donde se ahogarán.

---¡Cuánto sabe este hombre! Es un asombro que de esa manera pueda
anunciar los movimientos del enemigo. Y no hay duda, así tiene que
suceder.

---Y como la sublevación es general---añadió Fernández,---no podrán
acudir a todos lados. Además no pueden contar con un solo soldado
español que les ayude, porque todos desertan; de modo que si Napoleón
quiere continuar la guerra en España, ya puede mandar gente.

---Y como de los que vienen, la mitad mueren de borrachera\ldots{}

---El mismo Murat está padeciendo unos cólicos que se lo llevarán al
otro mundo.

---¡Quia! Si lo que tiene es una enfermedad vergonzosa.

---Así pagará las que ha hecho. ¿Pues qué puede ser eso, sino castigo de
Dios por su barbarie y crueldad?

---No es eso, señora; es que según dicen es aficionado a la bebida.

---¡Menudas borracheras habrá tomado desde que está aquí! ¿Y se marchará
o no se marchará?---Yo creo que sí---dijo Fernández.---Tengo entendido
que está muy disgustado, porque Napoleón no le quiere hacer rey de
España.

---Angelito; pues no pide poco que digamos.

---Y como parece que mandan de rey al que lo es de Nápoles, un D. José,
al cual según dicen también le gusta aquello\ldots{}

---Se conoce que es afición de familia.

---Lo que debiera hacer el Sr.~Fernández---dijo el lañador,---es irse a
cualquiera de esos ejércitos, donde sin duda se había de lucir, y quién
sabe si nos lo harían general de la noche a la mañana.

---Yo no sirvo para nada---contestó el Gran Capitán.---Yo tuve mi época,
y ahora que trabajen otros como trabajamos los de entonces. Aquellas sí
eran guerras, señores\ldots{} Esto de ahora es una bobería, y sino, ya
verán Vds. cómo en menos que canta un gallo se acaba todo.

---Pero lo del ejército de Andalucía, ¿es cierto o es puro barrunto de
Vd.? Sepámoslo de una vez.

---Es cierto, señores. Me parece que Santiago Fernández tiene motivos
para saber lo que hace un ejército y lo que deja de hacer. Cuando
empiecen nuestros generales a decir «por aquí te doy», ya les tendré a
Vds. al tanto de todo día por día.

A este punto llegaba, cuando entró Santorcaz, y no bien le vieron las
honradas personas que formaban el auditorio del buen Fernández,
empezaron todos a desfilar de muy mal talante, porque la presencia del
citado \emph{flamasón} era harto desagradable a todos los habitantes de
la casa.

---Grandes noticias, grandes noticias traigo, señor D. Gonzalo Fernández
de Córdoba---exclamó desde la puerta.---Aguárdense todos, si quieren
saber la verdad pura. ¿Pero se van estas niñas? ¿Por qué me tienen
miedo? ¿Y Vd., D. Roque, no quiere escuchar?\ldots{} Vayan noramala,
pues, y Vds. se lo pierden, porque no saben lo que ocurre\ldots{} La
lanza, Sr.~Fernández, tome Vd. al punto la lanza, y prepárese al
combate, porque se acerca lo tremendo, y ahora verá quiénes son buenos
patriotas y quiénes no lo son.

---No tomemos a broma estas graves cosas, señor D. Luis---dijo algo
amoscado el que podremos llamar vencedor de Cerinola,---ni nos
escandalice a la vecindad con sus endemoniados aspavientos.

---¿A que no sabe Vd. lo que yo sé?---añadió Santorcaz.---¿A que no sabe
Vd. que el general Dupont, que estaba en Toledo, ha recibido orden de
marchar a Andalucía, y que Moncey sale mañana de aquí para Valencia, y
que Lefebvre, que está en Pamplona, irá pronto sobre la capital de
Aragón; que Duhesme se extenderá por Cataluña y que Bessières baja hacia
Valladolid a toda prisa con las divisiones de Lasalle y de Merle?

---¡Cómo se conoce que Vd. escupe en corro con la canalla! ¿Y cómo están
sus mercedes del estómago? ¿Se han hecho al fin al vino de España? Y el
gran duque de Berg, ¿cómo anda de sus calenturas? ¿Hay mieditis? Porque
yo tengo para mí que si a esos señores se les caen los calzones es
porque, como dijo el otro, al que mal vive, el miedo le sigue. Yo, en
verdad, no sabía lo que Vd. acaba de decir; pero allá en la oficina oí
decir otras cosillas que no sé si sonarán bien en las orejas de la
canalla. ¿Por qué no va mi Sr.~D. Luis a contárselas, a ver si con el
gusto se les quita el destemple?

---¿Qué noticias son esas?

---Nada, poca cosa. Cuando el francés las sepa, verá Vd. qué contento se
pone\ldots{} Que en todas las ciudades se han nombrado o se van a
nombrar Juntas, las cuales no harán caso de lo que se mande en Bayona,
sino que\ldots{}

---Pero si Fernando VII no es ya Rey de España, porque ha cedido sus
derechos al Emperador, lo mismo que Carlos IV. ¿Qué son esas Juntas más
que cuadrillas de insurgentes?

---Sí\ldots{} pues que las quiten: es cosa fácil. ¡Demonios de Juntas! Y
los muy simples están formando unos ejércitos\ldots{} cosa de juego,
Sr.~de Santorcaz; cuatro gatos que estaban ahí en el Campo de San Roque
con unos cuantos cañoncillos\ldots{} Y también han dado en armarse los
paisanos, lo mismo en Castilla que en Cataluña, que en Valencia, que en
Andalucía\ldots{} pero eso no vale nada; son hombres de alfeñique y
alcorza , y no digo yo con balas, con saliva los destruirán los
franceses.

---¿Y todo lo que sabe Vd. se reduce a que la Junta de Sevilla está
formando un ejército con las tropas de San Roque que manda Castaños, y
las de Granada que están a las órdenes de Reding? Pues eso lo sabe todo
Madrid.

---Mira, Fernández---dijo oficiosamente doña Gregoria,---haces mal en
revelar lo que sabes por tan buen conducto, porque yo no soy lerda para
conocer que lo que hace nuestro ejército no se debe decir. Y sino, pongo
por caso: si tú que estás enterado de todo, a causa de tu gran tino para
la guerra, descubres lo que hace el ejército de Andalucía y llega a
oídos del francés, puede aprovecharse de la noticia y entonces\ldots{}

---¡Qué ha de aprovecharse, mujer, ni qué entiendes tú de estas cosas!
Al contrario, yo quiero que el señor de Santorcaz vaya con el cuento. Y
también en Castilla\ldots{}

---Otro ejército, sí, compuesto de guardias de corps, acostumbrados a
hacer la guerra en los palacios, de estudiantes, de paletos y
contrabandistas ¡Ah!---exclamó Santorcaz, dando tregua a las bromas y
hablando con completa seriedad.---Es una desgracia para nosotros el
tener que confesar que no podemos batirnos con los franceses. ¿Qué
importa que se armen multitud de paisanos, si esas turbas
indisciplinadas antes que ayuda serán elemento de desconcierto para el
escaso ejército español? ¿Qué obstáculo pueden ofrecer a los que han
sometido la Europa entera, esos infelices alucinados, a quienes engaña
su ignorancia? ¿Han visto alguna vez un campo de batalla? ¿Tienen idea
de lo que significa la previsión, la táctica, el genio de un jefe
experto para decidir la victoria? Es una triste cosa haber llegado a
este extremo por las torpezas de nuestros Reyes; pero una vez aquí, no
hay más remedio que someterse a lo que la Providencia ha querido hacer
de nosotros. España no puede resistir la invasión, porque si la
resistiera haría un milagro, una hazaña sobrenatural nunca vista.
Condenada a ser de Napoleón y a ver sentado en su trono a un Rey de la
familia imperial, lo más cuerdo es resignarse a este resultado con la
conciencia de haberlo merecido.

---¡Que España será francesa, que España será de Napoleón!---exclamó el
Gran Capitán encendido en violenta ira.---Sr.~de Santorcaz, Vd. es un
insolente, usted es un deslenguado, Vd. no tiene respeto a mis canas. Ya
¿qué se puede esperar de un trapisondista calavera como Vd. que abandonó
a su familia por irse al extranjero a aprender malas mañas? ¡Decir que
España ha de ser francesa! Salga Vd. de mi casa, y no ponga más los pies
en ella. ¿Qué te parece, Gregoria? Mujer, ¿te estás con esa calma y no
bufas de cólera como yo?

Y levantándose de su asiento, indicó a Santorcaz con majestuoso gesto la
puerta de la sala; mas como D. Luis no tuviera humor de marcharse,
porque todos los días se repetía la misma escena sin resultado alguno,
preparábase a comer tranquilamente, dejando que se desvaneciera, como
efectivamente se desvaneció sin efusión de sangre, la ira de su honrado
amigo. Durante la comida, D. Santiago gruñó un poco; pero la prudencia y
discreción de su esposa evitó un choque que pudiera haber tenido
calamitosas consecuencias.

\hypertarget{iv}{%
\chapter{IV}\label{iv}}

Lo que he contado pasaba el 20 de Mayo, si no me engaña la memoria. Poco
a poco fui avanzando en mi convalecencia, y en pocos días me hallé ya
con fuerzas suficientes para levantarme y dar algunos paseos por los
grandes corredores de la casa, pues la vivienda del Gran Capitán tenía
como único desahogo el largo pasillo, en cuya pared se abrían hasta
veinte puertas numeradas, albergues de otras tantas familias. Peor que
mi cuerpo se hallaba mi alma, llena de turbaciones, de sobresaltos y
congojas, tan apenada por terribles recuerdos como por angustiosas
presunciones, de tal modo que mi pensamiento corría a refugiarse
alternativamente de lo pasado a lo futuro, buscando en vano un poco de
paz.

La muerte del cura de Aranjuez, sin dejar de formar en mi alma un gran
vacío, me era menos sensible de lo que a primera vista pudiera parecer,
porque conceptuándola yo como tránsito que había llevado un nuevo santo
a las falanges del Paraíso, consideraba a mi amigo en su verdadero
lugar, y no tan lejos de nosotros que pudiera desampararnos si le
invocábamos.

En cuanto a Inés, no dudaba que existía en poder de alguien que la
protegiera por encargo de los parientes de su madre, y aunque para esta
creencia no tenía más dato que la relación del alucinado Juan de Dios,
yo me confirmaba cada vez más en ella, fundándome en antecedentes que
omito por ser de mis lectores conocidos, y en la sórdida avaricia del
licenciado Lobo, a cuyo carácter correspondía perfectamente una buena
recompensa, a quien deseaba poseerla.

Todo mi afán consistía en hallarme completamente restablecido para poder
salir a la calle, y cuando lo conseguí, tuve el gusto de darme a conocer
a todos mis amigos como un verdadero resucitado, o alma del otro mundo,
que vuelve con forma corporal a cobrar deudas atrasadas.

No tendrán Vds. idea del aspecto que ofrecía entonces Madrid, si no les
digo que la gente toda andaba azorada y aturdida, a veces llena de miedo
y a veces haciendo esfuerzos para disimular su alegría. El odio a los
franceses no era odio, era un fanatismo de que no he conocido después
ningún ejemplo; era un sentimiento que ocupaba los corazones por entero
sin dejar hueco para otro alguno, de modo que el amar a los semejantes,
el amarse a sí mismo, y hasta me atrevo a decir el amar a Dios se
adoptaban y sometían como fenómenos secundarios al gran aborrecimiento
que inspiraban los verdugos del pueblo de Madrid.

A estos se les veía solos en todos los sitios: su presencia hacía
detener o apresurar a los transeúntes, y era tan extraordinario este
desvío, que hasta parecían ellos mismos afectados de profundo pesar, y
se les observaba taciturnos y foscos, sintiendo que el suelo les quemaba
las plantas de los pies. Habían llenado de trincheras y baterías el
Retiro, y para ver en todo su orgullo y presunción a los invasores, no
había más que dirigir el paseo hacia Oriente, y se les encontraba en
grandes grupos alrededor de las cantinas, o paseando por la carretera de
Aragón. Ningún español se encaminaba hacia allí, a no ser los granujas
que entonces, como ahora, gustaban de meter las narices en todas partes.
Yo, llevado de mi curiosidad, me acerqué al Retiro, y también recorrí
otros sitios hacia el Mediodía, igualmente ocupados como posiciones
ventajosas.

En el interior de Madrid las tiendas estaban desiertas, pues todas las
personas que se juntaban para pedir o comunicar noticias se reunían en
parajes ocultos, siendo de notar que ya entonces comenzaban a dar sus
primeras señales de vida las sociedades secretas, aunque yo no vi
ninguna, y digo esto sólo con referencia a vagos rumores. Como el afán
por tener noticias relativas al levantamiento de las provincias, era una
fiebre de que no estaban exentos ni los niños, ni los ancianos, ni las
mujeres, cuando se sabía que D. Fulano de Tal había recibido una carta
de Andalucía o de Galicia o de Cataluña, la casa se llenaba de amigos, y
hasta los desconocidos se permitían invadirla ruidosamente para no
esperar a que se les contara el gran suceso. Sacábanse copias de las
cartas que hablaban de la Junta de Sevilla y de la sublevación de las
tropas de San Roque, y aquellas copias circulaban con una rapidez que
envidiaría la moderna prensa periódica.

Todos los días y a todas horas se hablaba de los oficiales que habían
huido de Madrid para unirse a los ejércitos de Cuesta o de Blake, y
cuando se tropezaba con un militar o con algún joven paisano de buen
porte y bríos, no se le hacía otra pregunta que esta: «¿Usted cuándo se
va?» Las familias de las víctimas se habían olvidado ya de rezar por los
muertos, y pensaban en equipar a los vivos. Escaseaban los jornaleros y
menestrales, porque de los barrios bajos partían diariamente muchos
hombres a engrosar las partidas de Toledo y la Mancha, y a pesar de los
brutales bandos del general francés, ni faltaban armas en las casas, ni
los fugitivos partían con las manos vacías.

Los invasores, que vigilaban el odio de la capital con la suspicacia
medrosa del que ha padecido sus terribles efectos, no permitían, siendo
tan grande su número y fuerza, que se manifestara lo que los madrileños
pensaban y sentían; pero aun así, ¡cuántos cantares, cuántas jácaras,
romances y décimas brotaron de improviso de la vena popular, ya
amenazando con rencor, ya zahiriendo con picantes chistes a los que
nadie conocía sino por el injurioso nombre de la \emph{canalla}!

En el fondo de aquella grande agitación, y entre tantos recelos, había
un júbilo secreto, pues como un día y otro llegaban noticias de nuevos
levantamientos, todos consideraban a los franceses como puestos en el
vergonzoso trance de retirarse. Aquel júbilo, aquella confianza, aquella
fe ciega en la superioridad de las heterogéneas y discordes fuerzas
populares, aquel esperar siempre, aquel no creer en la derrota, aquel
\emph{no importa} con que curaban el descalabro, fueron causa de la
definitiva victoria en tan larga guerra, y bien puede decirse que la
estrategia, y la fuerza y la táctica, que son cosas humanas, no pueden
ni podrán nunca nada contra el entusiasmo, que es divino.

Como era natural, las noticias del levantamiento se exageraban mucho, y
el entusiasmo popular veía miles de hombres donde no había sino
centenares. Cuando las noticias venían de Bayona, eran objeto de
sistemático desprecio, y las disposiciones del palacio de Marrás, así
como la convocatoria de irrisorias Cortes en la ciudad del Adour, y el
pleito homenaje por algunos grandes tributado a Bonaparte, daban pábulo
a las sátiras sangrientas. Cuando alguno decía que vendría de Rey a
Madrid el hermano de Napoleón, daba pie para las más ingeniosas
improvisaciones del género epigramático.

Todas las tertulias, que entonces eran muchas, pues la sociedad no se
desparramaba aún por los cafés, eran, digámoslo así, verdaderos clubs
donde latía sorda y terrible la conspiración nacional. Se conspiraba con
el deseo, con las noticias, con las sospechas, con las exageraciones,
con las sátiras, con verdades y mentiras, con el llanto tributado a los
muertos y las oraciones por el triunfo de los vivos.

\hypertarget{v}{%
\chapter{V}\label{v}}

Tal era Madrid a fines de Mayo de 1808, antes de que sonaran los
primeros cañonazos de Cabezón y los primeros tiros del Bruch. Dicho
esto, se me permitirá que hable un poco de mi persona, pues atendiendo a
que la desgracia halla siempre eco en las personas discretas y
sensibles, creo que no soy saco de paja a los ojos de mis lectores, y
que algún interés les inspiran los penosos trances de mi borrascosa
existencia. Necesito, además, explicar por qué causas emprendí mi viaje
a Andalucía entre Mayo y Junio; y si de buenas a primeras me presentara
camino de Despeñaperros en compañía del desconocido Santorcaz, Vds. no
acertarían a explicarse ni los móviles de jornada tan peligrosa, ni mi
repentino acomodamiento con aquel hombre singular.

Es, pues, el caso que no satisfecho con las noticias que acerca de Inés
me dio Juan de Dios, traté de averiguar la verdad y tuve la feliz
ocurrencia, mejor dicho, la inspiración, de presentarme en casa de la
marquesa, a quien no hallé; mas quiso la Divina Providencia que un
criado, conocido mío desde la famosa noche de la representación, me
saliera al encuentro, y después de mostrarse muy obsequioso,
satisficiera mi curiosidad sobre aquel punto. Según me dijo, el mismo
día 3 de Mayo se presentó allí un hombre de antiparras verdes, el cual
conducía dentro de una litera a cierta joven llorona y al parecer
enferma. No encontrando a la señora, preguntó por su hermano, con el
cual hubo de conferenciar más de dos horas, después de cuyo tiempo
despidiose, dejando a la madamita en la casa.

El hermano de la marquesa, que no era otro que aquel simpático
diplomático a quien conocimos en Octubre de 1807, partió el día 4 para
Córdoba a unirse con su hermana y sobrina, y ¡cosa rara!---decía aquel
curioso servidor,---se llevó consigo a la jovenzuela.

---¿De modo que ahora están todos en Córdoba?---le pregunté.

---Sí, y según noticias, no piensan venir hasta que no se acaben estas
cosas. Eso de la muchacha que trajeron en la litera ha dado mucho que
hablar a la servidumbre, y según dice mi mujer\ldots{} más vale callar.
El hombre aquel de las antiparras verdes había estado ya algunos días
aquí, y unas veces la señora condesa, otras su tía, le recibían. Mal
hombre parece.

---¿Y la muchacha no hizo resistencia cuando se la quisieron llevar?

---Si parecía muerta; ¿qué resistencia podía hacer? Si tuvimos que
cargarla entre dos para ponerla en el coche\ldots{}

Ignoro si esto que oí y puntualmente refiero, llamará la atención de
Vds., pero lo que sí les ha de causar sorpresa ¡qué digo sorpresa!,
asombro grandísimo, es el saber que me atreví a desafiar las iras del
licenciado Lobo, del mismo Lobo de marras, no vacilando en arriesgarlo
todo por esclarecer más aún que tan hondamente me inquietaba. No
queriendo aparecer ni aun en sombra por la aborrecida calle de la Sal,
busquelo allá por la alcaldía de Casa y Corte, donde con toda seguridad
pensaba encontrarle, y al punto que me vio\ldots{} No, no es verosímil,
no lo van ustedes a creer. ¿Necesitaré jurarlo? Pues lo juro: juro que
es la pura verdad\ldots{} Pues bien: al pronto que me vio, echome los
brazos al cuello, demostrando gran interés por mi persona, y no sólo me
pidió nuevas acerca de mi salud, sino que me rogó le contase algunos
pormenores acerca de mi fusilamiento y para él milagrosa resurrección.

Esto me dejó atónito, aunque no tranquilo, pues presumí que tan
desusadas blanduras serían obra de su refinada astucia, y preparación de
algún nuevo golpe contra mí; pero cuando le pregunté por el estado en
que se hallaba el proceso célebre, respondiome que ya no se pensaba en
tal cosa, porque como los franceses eran amigos del Príncipe de la Paz,
no convenía molestar a los servidores y amigos de este.

---No quiero---añadió,---que S. A. el Gran Duque se amosque. Aquello fue
una broma, y de haberte prendido, al punto hubieras sido puesto en
libertad. Pero di, picarón\ldots{} ¿conque tú eras galán de D.ª Inés?
Cuéntame todo: ¿dónde la conociste? ¡Ah, bien comprendía Requejo que
guardaba un tesoro en su casa! Yo lo sabía todo\ldots{} ¿y tú?, sospecho
que también, perillán. Lo que sí no sabías es que a fines del mes de
Abril se acordó en consejo de familia recoger e identificar a esa
jovencita para darle la posición que le corresponde. Como yo estaba al
tanto de todo, y además tenía el honor de conocer a la señora marquesa,
comprometime a entregarla, haciéndoles creer que había grandes
dificultades para arrancarla de casa de los parientes de su supuesta
madre. Hijo, es preciso hacer algo por la vida: a fe que es un pobre con
mujer, nueve hijos, dos suegras y tres cuñadas; dos suegras, sí señor,
la madre y la abuela de mi mujer, y si uno no se da maña para mantener a
este familión\ldots{} La verdad es que a todos les di cordelejo, a D.
Mauro, al papanatas de Juan de Dios, y a ti mismo, que ahora resucitas
para pedirme a Inesita. ¿Pero la amabas tú? Anda, zanguango, cortéjala,
a ver si logras casarte con ella, lo cual aunque difícil, no es
imposible\ldots{} la niña tendrá una dote regular y quizás pueda heredar
el mayorazgo y el título, lo cual será según el tenor de las
escrituras\ldots{} ¡Ah pelafustán! Me parece que tú traes un proyectillo
entre ceja y ceja. ¿Vas a Córdoba? Oye: recuerdo que la palomita te
llamaba con exclamaciones muy tiernas, cuando medio muerta la
conducíamos en la litera mi pasante y yo. ¡Ja, ja, ja! ¿Sabes de qué me
río? De ese ganso de Juan de Dios, que estuvo aquí el otro día, y
poniéndose de rodillas delante de mí, me dijo: «¡Deme Vd. a Inés porque
me muero sin ella! ¡Démela Vd. hoy y máteme mañana!» Fue una comedia,
Gabriel, y aunque nos reímos mucho, al fin nos cansó tanto que tuvimos
que echarle a palos de la escribanía.

Atención sostenida presté yo a estas y otras muchas razones del
licenciado Lobo, el cual para que nada faltara en su inexplicable
benignidad y cortesanía, al tiempo de despedirme me dijo que quizás
pudiera proporcionarme algunas lecciones de latín, si me hallaba con
ánimos, puesto que era tan gran humanista, de ganarme el pan con la
enseñanza. Dile las gracias y me retiré tan satisfecho del resultado de
mis investigaciones, que el mismo día decidí marchar a Córdoba cuando
estuviera restablecido.

¿Me seguirán Vds., o fatigados de estas aventuras dejarán que marche
solo a resolver cuestiones que a nadie interesan más que al que esto
escribe? No; espero que no nos separaremos tan a deshora, y cuando
parece probable que siguiéndome asistan Vds. a algún espectáculo que les
haga más llevadero el fastidio de mis personales narraciones. Vamos,
pues, y tengan en cuenta que nos acompaña el Sr.~de Santorcaz, a quien
llevan a Andalucía asuntos de familia. Yo le manifesté que deseaba me
llevase como escudero; mas él dijo que no tenía con qué pagar mis
servicios, porque su bolsa no estaba en disposición de atender a gastos
de servidumbre, y que harto se congratularía de llevarme como compañero
y amigo. Así fue, en efecto, y como yo necesitara algunos días más de
restablecimiento, él me esperó, y en uno de los últimos de Mayo o de los
primeros de Junio, luego que me despedí de mis obsequiosos protectores,
correspondiéndoles como pude, y de Juan de Dios, a quien oculté el
objeto de mi expedición, nos pusimos en marcha.

\hypertarget{vi}{%
\chapter{VI}\label{vi}}

Como Santorcaz era pobre, y yo más pobre todavía, nuestro viaje fue tan
irregular, cual los que en antiguas novelas vemos descritos. No
adoptamos sistemáticamente ninguna de las clases de incómodos vehículos
conocidos en nuestra España; así es que en varias ocasiones marchábamos
en galera, otras en macho, si nos franqueaban sus caballerías los
arrieros que tornaban a la Mancha de vacío, y las más veces a pie.
Hacíamos noche en las posadas y ventas del camino, donde Santorcaz lucía
su prodigiosa habilidad en el no gastar, logrando siempre que se le
sirviese bien. Para estas y otras picardías, mi compañero se hacía pasar
por un insigne personaje, mandándome que le llamase Su Excelencia, y que
me descubriese ante él siempre que nos mirase el mesonero. Yo lo cumplía
puntualmente; y con tal artificio, más de una vez, además de no
cobrarnos nada, salían a despedirnos humildemente rogándonos que les
dispensáramos el mal servicio.

Más allá de Noblejas y Villarrubia de Santiago, y cuando después de una
larga jornada sesteábamos, apartados del camino, junto a la ermita del
\emph{Santo Niño}, se nos agregó un mozo que nos dijo llevaba el mismo
camino que nosotros, y que desde entonces fue nuestro inseparable
compañero. Tenía como veinte años; llamábase Andresillo Marijuán, y
aunque era natural de Aragón, iba a servir de mozo de mulas a un pueblo
de Andalucía, en casa de la señora condesa de Rumblar, su ama y señora,
pues en las fincas que esta poseía en tierra de Almunia de Doña Godina,
había nacido aquel mancebo. Al punto su genio franco y alegre simpatizó
con el mío, y nos hicimos muy amigos. Santorcaz nos trataba con
superioridad aunque sin tiranía. Cuando al llegar a una posada
cabalgando él en perverso macho y nosotros a pie, íbamos a tenerle el
estribo y después a quitarle las espuelas, deshaciéndonos en cumplidos y
cortesías, teníamos que apretar los dientes para no soltar la risa.
Marijuán, que mejor que yo sabía fingir, era el encargado de ordenar al
ventero que le diese al amo lo mejor de la despensa, porque Su
Excelencia que iba de Regente a Sevilla, era hombre terrible, y
castigaba con fiereza a los posaderos que no le servían bien.

Así atravesamos la Mancha, triste y solitario país donde el sol está en
su reino, y el hombre parece obra exclusiva del sol y del polvo; país
entre todos famoso desde que el mundo entero se ha acostumbrado a
suponer la inmensidad de sus llanuras recorrida por el caballo de D.
Quijote. Es opinión general que la Mancha es la más fea y la menos
pintoresca de todas las tierras conocidas, y el viajero que viene hoy de
la costa de Levante o de Andalucía, se aburre junto al ventanillo del
wagon, anhelando que se acabe pronto aquella desnuda estepa, que como
inmóvil y estancado mar de tierra, no ofrece a sus ojos accidente, ni
sorpresa, ni variedad, ni recreo alguno. Esto es lo cierto: la Mancha,
si alguna belleza tiene, es la belleza de su conjunto, es su propia
desnudez y monotonía, que si no distraen ni suspenden la imaginación, la
dejan libre, dándole espacio y luz donde se precipite sin tropiezo
alguno. La grandeza del pensamiento de don Quijote, no se comprende sino
en la grandeza de la Mancha. En un país montuoso, fresco, verde, poblado
de agradables sombras, con lindas casas, huertos floridos, luz templada
y ambiente espeso, D. Quijote no hubiera podido existir, y habría muerto
en flor, tras la primera salida, sin asombrar al mundo con las grandes
hazañas de la segunda.

D. Quijote necesitaba aquel horizonte, aquel suelo sin caminos, y que,
sin embargo, todo él es camino; aquella tierra sin direcciones, pues por
ella se va a todas partes, sin ir determinadamente a ninguna; tierra
surcada por las veredas del acaso, de la aventura, y donde todo cuanto
pase ha de parecer obra de la casualidad o de los genios de la fábula;
necesitaba de aquel sol que derrite los sesos y hace locos a los
cuerdos, aquel campo sin fin, donde se levanta el polvo de imaginarias
batallas, produciendo al transparentar de la luz, visiones de ejércitos
de gigantes, de torres, de castillos; necesitaba aquella escasez de
ciudades, que hace más rara y extraordinaria la presencia de un hombre,
o de un animal; necesitaba aquel silencio cuando hay calma, y aquel
desaforado rugir de los vientos cuando hay tempestad; calma y ruido que
son igualmente tristes y extienden su tristeza a todo lo que pasa, de
modo que si se encuentra un ser humano en aquellas soledades, al punto
se le tiene por un desgraciado, un afligido, un menesteroso, un
agraviado que anda buscando quien lo ampare contra los opresores y
tiranos; necesitaba, repito, aquella total ausencia de obras humanas que
representen el positivismo, el sentido práctico, cortapisas de la
imaginación, que la detendrían en su insensato vuelo; necesitaba, en
fin, que el hombre no pusiera en aquellos campos más muestras de su
industria y de su ciencia que los patriarcales molinos de viento, los
cuales no necesitaban sino hablar, para asemejarse a colosos inquietos y
furibundos, que desde lejos llaman y espantan al viajero con sus gestos
amenazadores.

\hypertarget{vii}{%
\chapter{VII}\label{vii}}

Tal es la Mancha. Al atravesarla no podía menos de acordarme de D.
Quijote, cuya lectura estaba fresca en mi imaginación. Durante nuestras
jornadas nos aburríamos bastante, menos cuando Santorcaz nos contaba
algún extraordinario suceso de los muchos que en lejanos países había
presenciado. Una vez nos dejó con la boca abierta contándonos la fiesta
de la coronación de Bonaparte, con todos sus pelos y señales, y otra nos
puso los pelos de punta refiriendo la más famosa batalla de las muchas
en que se había encontrado. Cuando nos hizo el cuento, íbamos caballeros
en sendos machos que nos facilitaron por poco dinero unos arrieros de
Villarta, y no estoy seguro si habíamos traspasado ya el término de
Puerto Lápice o íbamos a entrar en él. Lo que sí recuerdo es que por
huir del calor, emprendimos nuestra jornada mucho antes de la salida del
sol, y que la noche estaba brumosa, el cielo encapotado y sombrío, la
tierra húmeda, a consecuencia del fuerte temporal de agua que descargara
el día anterior.

Debo indicar el paisaje que teníamos delante, porque no menos que la
pintoresca relación de Santorcaz, contribuyó aquel a impresionar mis
sentidos. El camino seguía en línea recta ante nosotros: a la izquierda
elevábanse unos cerros cuyas suaves ondulaciones se perdían en el
horizonte formando dilatadas curvas: en el fondo y muy lejos se
alcanzaba a ver una colina más alta, en cuya falda parecían distinguirse
las casas de un pueblo: a la derecha el suelo se extendía completamente
llano, y en su inmensa costra la tarda corriente de un arroyo y el agua
de la lluvia, formaban multitud de pequeños charcos, cuyas superficies,
iluminadas por la luna, ofrecían a la vista la engañosa perspectiva de
una gran laguna o pantano. He hablado de la luna, y debo añadir que
aquel astro, desfigurador de las cosas de la tierra, prestaba imponente
solemnidad al desnudo y solitario paisaje, esclareciéndolo o dejándolo a
oscuras alternativamente, según que daban paso o no a sus pálidos rayos,
los boquetes, desgarrones y acribilladuras de las nubes.

Santorcaz, después de un rato de silencio y meditación, contuvo su
cabalgadura, parose en mitad del camino y contemplando con cierto
arrobamiento el horizonte lejano, las colinas de la izquierda y los
charcos de la derecha, habló así:

---Estoy asombrado, porque nunca he visto dos cosas que tanto se
parezcan como este país a otro muy distante donde me encontraba hace
tres años a esta misma hora, en la madrugada del 2 de Diciembre. ¿Es mi
imaginación la que me reproduce las formas de aquel célebre lugar, o por
arte milagrosa nos encontramos en él? Gabriel, ¿no hay enfrente y hacia
la derecha unos grandes pantanos? ¿No se ven a la izquierda unos cerros
que terminan en lo alto con un pequeño bosque? ¿No se eleva delante una
colina en cuya falda blanquea un pueblecillo? Y aquellas torres que
distingo al otro lado de dicha colina ¿no son las del castillo de
Austerlitz?

Marijuán y yo nos reímos, diciéndole que se le quitaran de la cabeza
tales cosas, y que si bien lo de los charcos era cierto, por allí no
había ningún castillo de Terlín ni nada parecido. Pero él poniendo al
paso la cabalgadura y mandándonos que le siguiéramos uno a cada lado,
continuó hablando así:

---Muchachos, no puedo olvidar aquella célebre jornada, que llamamos de
los Tres Emperadores, y que es sin duda la más sangrienta, la más
gloriosa, la más hábil con que ha ilustrado su nombre el gran tirano,
ese hombre casi divino, a quien ahora puedo nombrar a boca llena, porque
no nos oyen más que el cielo y la tierra. Os contaré, muchachos, para
que sepáis lo que es el hacha de la guerra en manos de ese leñador de
Europa. Yo me hallaba en París sin recursos después de haber sido
sucesivamente maestro de latín, pintor de muestras, corista en
Ventadour, espadachín, servidor de los emigrados de Coblenza, postillón
de diligencias, carbonero y cajista de imprenta, cuando senté plaza en
el ejército de Boulogne, destinado a dar un golpe de mano contra
Inglaterra\ldots{} Cuando el Emperador nos trasladó de improviso y sin
revelar su pensamiento al centro de Europa, estábamos un tanto amoscados
porque las violentas marchas nos mortificaban mucho, y como éramos unos
zopencos, no comprendíamos los grandes planes de nuestro jefe. Pero
después de la capitulación de Ulm, nos creíamos los primeros soldados
del mundo, y al hablar de los austriacos, de los prusianos y de los
rusos, nos reíamos de ellos, juzgándolos hasta indignos de nuestras
balas. Cuando pasamos el Inn ya presumíamos que se preparaban grandes
cosas: al internarnos en la Moravia, después de la acción de Hollabrünn,
comprendimos que el ejército ruso-austriaco nos iba a presentar batalla
formal. Lo que no estaba reservado a nuestras cabezas era el discurrir
si tomaríamos la ofensiva o si operaríamos a la defensiva. Pero la gran
cabeza, aquella que tiene un mechón en la frente y el rayo en el
entrecejo, lo iba a decidir bien pronto.

A este punto llegaba, cuando el camino por que marchábamos torció hacia
la derecha describiendo una gran vuelta, de modo que formaba ángulo
recto con su primitiva dirección. Santorcaz, nuevamente alucinado, con
aquello que parecía para él extraordinaria coincidencia, prosiguió así:

---¿Pero no es este el camino de Olmutz? Gabriel: o esto es aquello
mismo, o se le parece como una gota a otra gota. Mira, ahora tenemos
enfrente los pantanos de Satzchan y a nuestra izquierda la colina de
Pratzen. Mira hacia allá. ¿No se oye ruido de tambores? ¿No se ven
algunas luces? Pues allí están los rusos y los austriacos. ¿Sabes cuál
es su intención? Pues quieren cortarnos el camino de Viena, para lo cual
tendrán que bajar de la colina de Pratzen y situarse entre nuestra
derecha y los pantanos. ¡Mira si son estúpidos! Eso precisamente es lo
que quiere el Emperador y todo lo dispone de modo que parezca que nos
retiramos hacia Viena. Figúrate que aquí está nuestro ejército,
compuesto de setenta mil hombres, cuyo inmenso frente ocupa todas las
colinas de la izquierda, el camino y parte de la llanura que hay a la
derecha. El Emperador, después de llenarse las narices de tabaco, sale a
media noche a recorrer el campo, y observar los movimientos del enemigo.
¿Veis?, por allí va. ¿No se oyen las pisadas de su caballo, y los gritos
de entusiasmo con que le saludan los soldados? ¿No se ve el resplandor
de las hogueras que encienden a su paso? ¿Pero Vds. no ven todo esto?
Bah. Es ilusión mía, pero de tal modo aviva mis recuerdos la similitud
del paisaje, que me parece ver y oír lo que estoy contando\ldots{} Pero
querréis saber cómo fue que vencimos a los rusos y a los austriacos, y
os lo voy a referir. Al amanecer ¡oh chiquillos!, los rusos bajaban
maquinalmente por aquella alta colina de enfrente, con objeto de venir
hacia nuestra derecha para cortarnos el camino. No olvidéis que aquí
delante tenemos un arroyo que viene serpenteando de izquierda a derecha
hasta perderse en los pantanos. El Emperador manda que la derecha pase
el arroyo, y verificado esto, los rusos la atacan. El centro, mandado
por Soult y la izquierda por Lannes, ansiaban entrar en fuego; pero el
Emperador contenía el ardor de aquellos generales, para aguardar a que
los rusos acabasen de cometer el desatino de bajar de las alturas de
Pratzen para meterse en la madre del arroyo de Golbasch. Os explicaré
bien. Allá en lontananza y al pie de la loma están las aldeas de Telnitz
y Sokolnitz\ldots{}

---Si aquí no hay tales aldeas, señor---interrumpió Marijuán, indócil a
la mistificación.

---Necio, ¿querrás callar?---continuó el francmasón.---Yo sé lo que me
digo, y es que todo el afán de Napoleón después que vio bajar a los
rusos, consistía en tomar aquellas aldeas para luego apoderarse de la
loma que tenemos enfrente. ¿No le veis? Pues bien; los generales Soult y
Lannes partieron al galope para dirigir las operaciones del centro y de
la izquierda. Yo pertenecía al centro, y estaba en el
17.\textsuperscript{o} de línea y a las órdenes de Vandamme. Avanzamos
hacia el arroyo: ¿veis?, fuimos por aquí a toda prisa.

---Si aquí no hay tal arroyo---dijo Marijuán riendo.---Vd. sí que tiene
la cabeza llena de arroyos y aldeas, y derechas e izquierdas.

---Llegamos a la aldea de Telnitz y allí comenzó el ataque---continuó
imperturbablemente Santorcaz.---En la loma quedaban todavía veintisiete
batallones de infantería rusa y austriaca, mandados en persona por los
dos Emperadores y por el general en jefe ruso Kutusof. ¡Ah, muchachos,
si hubierais visto aquello! Mirad hacia enfrente, pues desde aquí se
distingue muy bien la posición que respectivamente teníamos, ellos
encima, nosotros debajo\ldots{} Al principio nos acribillaban; pero
Soult nos manda subir a todo trance, y subimos desafiando la lluvia de
balas. Para ayudarnos, el general Thiebault, que pertenecía a la
división de Saint-Hilaire, refuerza nuestra derecha con doce piezas de
artillería que bien disparadas hacen grandes claros en las filas
enemigas. Estas tienen al fin que retroceder al otro lado de la loma.
¿Veis aquel repecho que hay a la izquierda? Pues allí fue el
17.\textsuperscript{o} de línea. Piquemos nuestras caballerías y nos
hallaremos en el mismo sitio. Estúpidos, ¿no os entusiasmáis con estas
cosas? Mira, Gabriel, ya estamos subiendo: esta es la loma que veíamos
desde lejos: este repecho que miráis a la izquierda es el repecho de
Stari-Winobradi, a donde el general Vandamme nos condujo. ¿Pero creéis
que era cosa de juego? El repecho estaba defendido por numerosas tropas
rusas, y una formidable artillería. La cosa era peliaguda; pero cuando
los generales dicen \emph{adelante}, \emph{adelante}, no es posible
resistir, y aunque del 17.\textsuperscript{o} de línea no quedamos más
que la tercera parte para contarlo, ayudados por el
24.\textsuperscript{o} de ligeros, tomamos al fin el repecho,
apoderándonos de la artillería. Los rusos se desbandaron por el otro
lado de la loma, dirigiéndose hacia aquel caserío que a lo lejos clarea
a la luz de la luna y que no es otro que el castillo de Austerlitz.

Marijuán reventaba de hilaridad. Yo, a mi vez, no pude menos de hacer
alguna observación al narrador, diciéndole:

---Señor de Santorcaz, allá no se ve ningún castillo, como no sea que se
le antoje fortaleza la cabaña de algún pastor de carneros, únicos rusos
que andan por estos lugares.

---Tú sí que no sabes lo que te dices---prosiguió Santorcaz deteniendo
su macho en medio del camino.---Os seguiré contando. Mientras los del
centro hacíamos lo que habéis oído, allá por la izquierda, en esa tierra
llana que tenemos a este lado, la caballería cargaba portentosamente al
mando de Lannes y Murat. Francamente, rapaces, de esto poco os puedo
hablar, porque caí herido: por un buen rato se me pusieron ciertas
telarañas ante los ojos, y mis oídos no percibían sino un vago zumbido.
Pero ahí hacia la derecha se remataba a los rusos y austriacos del modo
más admirable. ¿No veis los pantanos de Satzchan? A lo lejos brilla su
engañosa superficie: están helados, y los rusos, impelidos por Soult, se
precipitan sobre ellos. En el acto el Emperador manda que la artillería
de la guardia dispare algunos cañonazos sobre el hielo para que se
hunda, y entre los desmenuzados cristales, caen al agua dos mil rusos
con sus cañones, caballos, pertrechos, armas, municiones y carros,
precipitándose confusamente, sin que sus compañeros les prestaran
socorro, porque no pensaban más que en huir, y huyendo se ahogaban, y
quedándose morían barridos por la metralla francesa. ¡Qué espantoso
desastre para aquella pobre gente, y qué gran victoria para nosotros!
Estábamos locos de entusiasmo. ¡Pero qué veo! Gabriel, y tú, Marijuán,
¿no os entusiasmáis? Sois unos gaznápiros. Aquello fue prodigioso. Sólo
entramos en fuego cuarenta mil hombres, y merced a las hábiles
disposiciones del gran tirano, derrotamos a noventa mil aliados,
matándoles o ahogando quince mil, cogiendo veinte mil prisioneros y
ciento veinte cañones. ¿No había motivo para que nos volviéramos locos
con nuestro jefe? ¡Ah, muchachos, si hubierais estado allí cuando
recorrió el campo de batalla mandando recoger los heridos! Creo que
hasta los muertos se levantaban para gritar «¡viva el Emperador!», y
cuando a la noche siguiente encendimos una gran hoguera, en este mismo
sitio donde ahora estamos, y vino él a situarse allí enfrente para
recibir al emperador de Austria, parecía un dios rodeado de aureola de
fuego y teniendo al alcance de su mano los rayos con que destruía tronos
y reyes, imperios y coronas.

Marijuán y yo nos reíamos; pero pronto nos fue forzoso disimular nuestra
hilaridad, porque habiendo preguntado el joven aragonés con mucha sorna
que cuál fue la ventaja sacada de tal lucha, Santorcaz se amoscó, y
amenazando castigarnos si no nos entusiasmábamos como él, nos dijo:

---Mentecatos, podencos; ¿acaso la paz y tratado de Presburgo es paja?
Prusia quedó aliada de Francia, perdiendo Austria el apoyo de su
hermana. Austria abandonó a Francia el estado de Venecia y cedió el
Tirol a Baviera, reconociendo al mismo tiempo la soberanía de los
electores de Baviera, Wurtemberg y Baden, después de pagar a Francia
cuarenta millones de indemnización de guerra. Al mismo tiempo, pedazos
de alcornoque, por el tratado de Schœnbrunn, Francia cedió a Prusia el
Hannover, Prusia cedió a Baviera el marquesado de Anspach y a Francia el
principado de Neufchatel y el ducado de Cleves.

Marijuán y yo volvimos a mirarnos y nos volvimos a reír, lo cual,
advertido por Santorcaz, fue causa de que este nos sacudiera un par de
latigazos, que a ser repetidos, nos habrían obligado a defendernos,
haciendo allí mismo un segundo Austerlitz. Más bien estábamos para
burlas que para veras, y Marijuán especialmente, no dejaba pasar
coyuntura alguna en que pudiera zaherir a nuestro compañero; así es, que
habiendo acertado a encontrar un rebaño de ovejas y cabras, dijo el
aragonés:

---Apartémonos aquí junto al charco para ver de derrotar a estos
austriacos y rusiacos, que vienen mandados por el tío Parranclof,
emperador del Zurrón y rey de los guarros, y subamos a la loma de la
Panza para quitarles la artillería y hacerles meter en el castillo.

Yo en tanto, acordándome de D. Quijote, contemplaba el cielo, en cuyo
sombrío fondo las pardas y desgarradas nubes, tan pronto negras como
radiantes de luz, dibujaban mil figuras de colosal tamaño y con esa
expresión que sin dejar de ser cercana a la caricatura, tiene no sé qué
sello de solemne y pavorosa grandeza. Fuera por efecto de lo que acababa
de oír, fuera simplemente que mi fantasía se hallase por sí dispuesta a
la alucinación que siempre produce un bello espectáculo en la solitaria
y muda noche, lo cierto es que vi en aquellas irregulares manchas del
cielo veloces escuadrones que corrían de Norte a Sur; y en su revuelta
masa las cabezas de los caballos y sus poderosos pechos, pasando unos
delante de otros, ya blancos, ya negros, como disputándose el mayor
avance en la carrera. Las recortaduras, varias hasta lo infinito, de las
nubes, hacían visajes de distintas formas, de colosales sombreros o
morriones con plumas, penachos, bandas, picos, testuces, colas, crines,
garzotas; aquí y allí se alzaban manos con sables y fusiles, banderas
con águilas, picas, lanzas, que corrían sin cesar; y al fin, en medio de
toda esa barahúnda, se me figuró que todas aquellas formas se deshacían,
y que las nubes se conglomeraban para formar un inmenso sombrero
apuntado de dos candiles, bajo el cual los difuminados resplandores de
la luna como que bosquejaban una cara redonda y hundida entre las altas
solapas, desde las cuales se extendía un largo brazo negro, señalando
con insistente fijeza el horizonte.

Yo contemplaba esto, preguntándome si la terrible imagen estaba
realmente ante mis ojos, o dentro de ellos, cuando Santorcaz exclamó de
improviso:

---Miradle, miradle allí. ¿Le veis? ¡Estúpidos!, ¡y queréis luchar con
este rayo de la guerra, con este enviado de Dios que viene a transformar
a los pueblos!

---Sí, allí lo veo---exclamó Marijuán, riendo a carcajadas.---Es D.
Quijote de la Mancha que viene en su caballo, y seguido de Sancho Panza.
Déjenlo venir, que ahora le aguarda la gran paliza.

Las nubes se movieron, y todo se tornó en caricatura.

\hypertarget{viii}{%
\chapter{VIII}\label{viii}}

El sol no tardó en salir aclarando el país y haciendo ver que no
estábamos en Moravia, como vamos de Brunn a Olmutz, sino en la Mancha,
célebre tierra de España.

El pueblo donde paramos a eso de las ocho de la mañana era Villarta, y
dejando allí nuestros machos, tomamos unas galeras que en nueve horas
nos hicieron recorrer las cinco leguas que hay desde aquel pueblo a
Manzanares: ¡tal era la rapidez de los vehículos en aquellos felices
tiempos! Cuando entrábamos en esta villa al caer de la tarde,
distinguimos a lo lejos una gran polvareda, levantada al parecer por la
marcha de un ejército, y dejando los perezosos carros, entramos a pie en
el pueblo para llegar más pronto, y saber qué tropas eran aquellas y a
dónde iban.

Allí supimos que eran las del general Ligier-Belair que iba a auxiliar
el destacamento de Santa Cruz de Mudela, sorprendido y derrotado el día
anterior por los habitantes de esta villa. En la de Manzanares reinaba
gran desasosiego, y una vez que los franceses desaparecieron, ocupábanse
todos en armarse para acudir a auxiliar a los de Valdepeñas, punto donde
se creía próximo un reñido combate. Dormimos en Manzanares, y al
siguiente día, no encontrando ni cabalgaduras ni carro alguno, partimos
a pie para la venta de la Consolación, donde nos detuvimos a oír las
estupendas nuevas que allí se referían.

Transitaban constantemente por el camino paisanos armados con escopetas
y garrotes, todos muy decididos, y según la muchedumbre de gente que
acudía hacia Valdepeñas, en Manzanares, y en los pueblos vecinos de
Membrilla y la Solana no debían de quedar más que las mujeres y los
niños, porque hasta algunos inútiles viejos acudían a la guerra. Por
último, resolvimos asistir nosotros también al espectáculo que se
preparaba en la vecina villa, y poniéndonos en marcha, pronto recorrimos
las dos leguas de camino llano: mucho antes de llegar divisamos una gran
columna de negro humo que el viento difundía en el cielo. La villa de
Valdepeñas ardía por los cuatro costados.

Apretando el paso, oímos ya cerca del pueblo prolongado rumor de voces,
algunos tiros de fusil, pero no descargas de artillería. Bien pronto nos
fue imposible seguir por el arrecife, porque la retaguardia francesa nos
lo impedía, y siguiendo el ejemplo de los demás paisanos, nos apartamos
del camino, corriendo por entre las viñas y sembrados, sin poder
acercarnos a la villa. En esto vimos que la caballería francesa se
retiraba del pueblo, ocupando el llano que hay a la izquierda, y al
mismo tiempo el incendio tomaba tales proporciones, que Valdepeñas
parecía un inmenso horno. Los gritos, los quejidos, las imprecaciones
que salían de aquel infierno, llenaban de espanto el ánimo más
esforzado.

Al punto comprendimos que el interior del pueblo se defendía
heroicamente, y que el plan de los franceses consistía en apoderarse de
los extremos, incendiando todas las casas que no pudieran ocupar. De vez
en cuando un estruendo espantoso indicaba que alguno de los endebles
edificios de adobes había venido al suelo, y el polvo se confundía en
los aires con el humo. Los escombros sofocaban momentáneamente el fuego;
pero este surgía con más fuerza, cundiendo a las casas inmediatas. Al
fin pareció que todo iba a cesar, y, según dijeron los que estaban más
cerca, habían salido del pueblo algunos hombres a conferenciar con el
general francés. Mucho tiempo debieron de durar las conferencias, porque
no vimos que estos se retiraran ni que concluyese el ruido y algazara en
el interior; pero al cabo de largo rato un movimiento general de la
multitud nos indicó que algo importante ocurría. En efecto, los
franceses, replegando sus caballos en la calzada, retrocedían hacia
Manzanares.

Cuando entramos en Valdepeñas, el espectáculo de la población era
horroroso. Parece increíble que los hombres tengan en sus manos
instrumentos capaces de destruir en pocas horas las obras de la
paciencia, de la laboriosidad, del interés acumuladas por el brazo
trabajador de los años y los siglos. La calle Real, que es la más grande
de aquella villa, y, como si dijéramos, la columna vertebral que sirve a
las otras de engaste y punto de partida, estaba materialmente cubierta
de jinetes franceses y de caballos. Aunque la mayor parte eran
cadáveres, había muchos gravemente heridos, que pugnaban por levantarse;
pero clavándose de nuevo en las agudas puntas del suelo, volvían a caer.
Sabido es que bajo las arenas que artificiosamente cubrían el pavimento
de la vía, el suelo estaba erizado de clavos y picos de hierro, de tal
modo que la caballería iba tropezando y cayendo conforme entraba, para
no levantarse más.

A la calle se habían arrojado cuantos objetos mortíferos se creyeron
convenientes para hostilizar a los dragones, y aun después del combate
surcaban la arena turbios arroyos de agua hirviendo, que, mezclada con
la sangre, producía sofocante y horrible vapor. En algunas ventanas
vimos cadáveres que pendían medio cuerpo fuera y apretando aún en sus
crispados dedos el trabuco o la podadera. En el interior de las casas
que no eran presa de las llamas, el espectáculo era más lastimoso,
porque no sólo los hombres, sino las mujeres y los niños, aparecían
cosidos a bayonetazos en las cuevas, y a veces cuando se trataba de
entrar en alguna casa por dar auxilio a los heridos que lo habían
menester, era preciso salir a toda prisa, abandonándolos a su
desgraciada suerte, porque el fuego, no saciado con devorar la
habitación cercana, penetraba en aquella con furia irresistible.

En resumen, franceses y españoles se habían destrozado unos a otros con
implacable saña; pero al fin aquellos creyeron prudente retirarse, como
lo hicieron, no parando hasta Madridejos. Cuando Santorcaz, Marijuán y
yo seguimos nuestra marcha, para hacer noche en Santa Cruz de Mudela, el
espíritu de los valerosos paisanos de Valdepeñas no había decaído, y
tratando de reparar los estragos de aquella sangrienta jornada, parecían
capaces de repetirla al siguiente día.

De lejos y al caer de la tarde distinguíamos la columna de humo,
cubriendo el cielo de vagabundas y sombrías ráfagas, y el aragonés y yo
no pudimos menos de maldecir en voz alta y expresivamente al tirano
invasor de España. Contra lo que esperábamos, Santorcaz no nos contestó
una palabra, y seguía su camino profundamente pensativo.

\hypertarget{ix}{%
\chapter{IX}\label{ix}}

Al pasar la sierra, me reconocí completamente sano de mi anterior
enfermedad. La influencia sin duda de aquel hermoso país, el vivo sol,
el viaje, el ejercicio equilibraron al punto las fuerzas de mi cuerpo, y
respiraba con desahogo, andaba con energía, sin sentir malestar alguno
en mis heridas. Todo rastro de dolor o debilidad desapareció, y me
encontré más fuerte que nunca. Nada de particular hallamos durante
nuestro tránsito por las nuevas poblaciones, a no ser la alarma, la
inquietud y los preparativos de defensa. En la Carolina y en Santa Elena
escaseaban mucho los hombres, porque la mayor parte habían ido a
incorporarse a la legión formada por D. Pedro Agustín de Echévarri,
legión cuya base fueron los valerosos contrabandistas del país. Quedaba,
no obstante, en los desfiladeros de Despeñaperros bastante gente para
detener todos o la mayor parte de los correos, y en varios puntos,
apostadas las mujeres o los chiquillos en lo escabroso de aquellas
angosturas, avisaban la proximidad del convoy para que luego cayeran
sobre él los hombres. También advertimos gran abandono en los primeros
campos de pan que se ofrecieron a nuestra vista; y en algunos sitios las
mujeres se ocupaban en segar a toda prisa los trigos todavía lejos de
sazón. Cerca de Guarromán vimos grandes sementeras quemadas, señal de
que había comenzado allí su oficio la horrible tea invasora.

Hasta entonces no había ocurrido ninguna colisión sangrienta entre los
imperiales y los andaluces. Estos, al ver que de improviso por entre los
romeros y lentiscos de la sierra a aquellos soldados de la fábula, tan
hermosos y al mismo tiempo tan justamente engreídos de su valor, no
volvieron de su asombro sino cuando los vieron desaparecer camino de
Córdoba, y sólo entonces, sintiendo requemadas sus mejillas por generosa
vergüenza, cayeron en la cuenta de que el suelo patrio no debía ser
hollado por extranjeras botas. Los franceses encontraron el país
tranquilo, y creyeron llegar felizmente a Cádiz; pero bajo las
herraduras de sus caballos iba naciendo la yerba de la insurrección.
Aquellos caballos no eran como el de Atila, que imprimía sello de muerte
a la tierra, sino que por el contrario, sus pisadas, como un toque de
rebato, iban despertando a los hombres y convocándolos detrás de sí.

Llegamos por último a Bailén, y explicaré por qué nos detuvimos en esta
villa algunos días. Allí residía el ama de Marijuán, quien al
presentarse a ella nos rogó que le acompañásemos, y esta apreciable
señora que era doña María Castro de Oro, de Afán de Ribera, condesa de
Rumblar, nos recibió con tanto agasajo, nos ponderó de tal modo la
ruindad de las posadas y ventas de la villa, que no tuvimos por
conveniente hacernos de rogar, y aceptamos la hospitalidad que se nos
ofrecía. La casa era grandísima y no faltaba hueco para nosotros, ni
tampoco excelente comida y bebida de lo más selecto de Montilla y
Aguilar.

---A estas horas---nos dijo la condesa,---los franceses deben de haber
empeñado una acción con el ejército de paisanos que dicen salió de
Córdoba para defender el paso del puente de Alcolea. Si ganan los
españoles, los franceses retrocederán hacia Andújar, y como han de estar
muy rabiosos, cometerán mil atrocidades en el camino. No conviene que
salgan ustedes de aquí, a no ser que tengan intención, como mi hijo, de
incorporarse al ejército que se está formando en Utrera.

No eran necesarias tantas razones para convencernos. Nos quedamos, pues,
en la ilustre casa; y ahora, señores míos, con todo reposo voy a
contaros puntualmente lo que recuerdo de aquella mansión y de sus
esclarecidos habitantes, destinados a figurar bastante en la historia
que voy refiriendo.

El palacio de Rumblar era un caserón del siglo pasado, de feísimo
aspecto en su exterior, pero con todas las comodidades interiores que
alcanzaban los tiempos. Las altas paredes de ladrillo, las rejas
enmohecidas y rematadas en pequeñas cruces, los dos escudos de piedra
oscura que ocupaban las enjutas de la puerta, cuyo marco apainelado y
con vuelta de cordel, parecía remontarse a fecha más antigua que el
resto de la casa; las dos ventanas angreladas junto a un mirador
moderno; el farol sostenido por pesada armadura de hierro dulce, en cuyo
centro se retorcían algunas letras iniciales y una corona dibujadas con
las vueltas del lingote; las guarniciones jalbegadas alrededor de los
huecos; sus pequeños vidrios, sus celosías, y la diversidad y variedad
de aberturas practicadas en el muro, según las exigencias del interior,
le asemejaban a todas las antiguas mansiones de nuestros grandes,
bastante desprendidos siempre para gastar en la fábrica de los conventos
el gusto y el dinero que exigían las fachadas de sus palacios. Por
dentro resplandecía el blanco aseo de las casas de Andalucía. Tenía gran
sala baja, capilla, patio con flores, habitaciones con zócalo de
azulejos amarillos y verdes, puertas de pino lustradas y chapeadas, gran
número de arcones, muchas obras de talla, cuadros viejos y nuevos,
algunas jaulas de pájaros, finísimas esteras, y sobre todo, una
tranquilidad, un reposo y plácido silencio que convidaban a residir allí
por mucho tiempo.

Hablemos ahora de la familia de Afán de Ribera, o Perafán de Ribera, que
en esto no están acordes los cronistas. Ocupará el primer lugar en esta
reverente enumeración la señora condesa viuda doña María Castro de Oro
de Afán, etc., aragonesa de nacimiento, la cual era de lo más severo,
venerando y solemne que ha existido en el mundo. Parecía haber pasado de
los cincuenta años, y era alta, gruesa, arrogante, varonil: usaba para
leer sus libros devotos o las cuentas de la casa, unos grandes
espejuelos engastados en gruesa armazón de plata, y vestía
constantemente de negro, con traje que a las mil maravillas convenía a
su cara y figura. Aquella y esta eran de las que tienen el privilegio de
no ser nunca olvidadas, pues su curva nariz, sus cabellos entrecanos, su
barba echada hacia afuera y la despejada y correcta superficie de su
hermosa frente, hacían de ella un tipo cual no he visto otro. Era la
imagen del respeto antiguo, conservada para educar a las presentes
generaciones.

Tendrá el segundo lugar su hijo, joven de veinte años, niño aún por sus
hábitos, su lenguaje, sus juegos y su escasa ciencia. Era el único
varón, y por tanto el mayorazgo de aquella noble casa, cuyo origen, como
el del majestuoso Guadalquivir, se remontaba a las fragosidades de la
Sierra de Cazorla, donde los primeros Afán de Ribera hicieron no sé qué
hazañas durante la conquista de Jaén. El joven D. Diego Hipólito Félix
de Cantalicio había sido educado conforme a sus altos destinos en el
mundo, bajo la dirección de un ayo, de que después hablaremos, y aunque
era voluntarioso y propenso a sacudir el cascarón de la niñez, así como
a arrastrar por el polvo de la travesura juvenil el purpúreo manto de la
primogenitura, su madre lo tenía metido en un puño, como suele decirse,
y ejercía sobre él todos los rigores de su carácter. Verdad es que el
muchacho, con su instinto y buen ingenio, había descubierto un medio
habilísimo para atacar la severidad materna, y era que cuando su ayo o
la condesa no le hacían el gusto en alguna cosa, poníase los puños en
los ojos, comenzaba a regar con pueriles lágrimas los veinte años de su
cuerpo y exclamaba: «Señora madre, yo me quiero meter fraile». Estas
palabras, esta resolución del muchachuelo, que de ser llevada adelante,
troncharía implacablemente el frondoso árbol mayorazguil, difundía el
pánico por todos los ámbitos de la casa. Procuraban todos aplacarle, y
la madre decía: «No seas loco, hijo mío. Vaya, puedes montarte a caballo
en la viga del patio, y te permito que le pongas al gato las cáscaras de
nuez en sus cuatro patitas».

A estos dos personajes seguirán forzosamente las dos hijas de la
marquesa; dos pimpollos, dos flores de Andalucía, lindas, modestas,
pequeñas, frescas, sonrosadas, alegres, sin pretensiones a pesar de su
nobleza, rezadoras de noche y cantadoras por la mañana; dos avecillas
que encantaban la vista con el aleteo de su inocente frivolidad y de
cierta ingenua coquetería, de ellas mismas ignorada. Eran pequeñas como
el reseda; pero como el reseda tenían la seducción de un perfume que se
anuncia desde lejos, pues al sentirles los pasos se alegraba uno, y su
proximidad era aspirada con delicia. Asunción y Presentación eran dos
angelitos con quienes se deseaba jugar para verles reír y para reírse
uno mismo del grave gesto con que enmascaraban sus lindas facciones
cuando su madre les mandaba estar serias. La de menor edad era destinada
al claustro, y mientras halagaba a doña María la grandiosa idea de
ponerla en las Huelgas de Burgos, se acordó que tomara las lecciones
necesarias para ser doctora, por lo cual el ayo de su hermano le había
empezado a enseñar la primera declinación latina, que aprendió en un
periquete, encontrando aquello muy bonito. La primera, esto es,
Asunción, no tenía necesidad de aprender nada, porque era destinada al
matrimonio.

Y por último, no quiero dejar en la oscuridad al ayo del joven D. Diego.
Llamábanle comúnmente don Paco y era un varón de gran sencillez y
moderación en sus costumbres, aunque algo pedante. Estaba él convencido
de que sabía latín, y citaba a veces los autores más célebres,
aplicándoles lo que estos desgraciados no pensaron nunca en decir. ¡A
tales imputaciones calumniosas está expuesta la celebridad! También se
preciaba D. Paco de enseñar acertadamente la historia antigua y moderna
a sus discípulos, aunque nosotros sabemos por documentos de autenticidad
incontestable que en sus explicaciones nunca pasó más acá del arca de
Noé. Era, sí, muy fuerte en la vida de Alejandro el Grande, y podemos
asegurar que poseía en altísimo grado un arte, que no a todos los
mortales es dado cultivar con regular acierto. Don Paco era un gran
pendolista, que pudiera competir con esos colosos de la caligrafía,
Torío el sublime y Palomares el divino, y hasta con el moderno
Iturzaeta; habilidad que en parte había transmitido a su discípulo, pues
las planas del heredero de Rumblar llenaban de admiración al señor
obispo de Guadix, cuando iba a pasar unos días en la casa. Además, D.
Paco era un hombre excelente, y temblaba de miedo delante de la condesa,
cuando esta le achacaba las faltas del niño. Vestía de negro y siempre
en traje ceremonioso, aunque no nuevo, usando asimismo peluca blanca,
rematada en descomunal bolsa. A los forasteros huéspedes nos trataba con
mucha dulzura porque la hospitalidad---decía---fue don particular de los
pueblos antiguos, y debe ser practicada por los presentes para enseñanza
de los venideros.

\hypertarget{x}{%
\chapter{X}\label{x}}

El patrimonio de aquella casa era bueno, aunque muy inferior al de otras
familias de Andalucía y de Castilla; pero doña María contaba con que
sería de los primeros de España luego que su hijo heredase el mayorazgo
de unos parientes por línea colateral, que carecían de sucesión directa.
Para facilitar esto, doña María concibió un proyecto gigantesco, del
cual dependía, como el lector verá, la perpetuidad de aquella casa y
linaje y solar ilustre por el largo discurso de los siglos; trató de
casar a su hijo con una hembra de la familia de aquellos sus parientes,
a la sazón poseedores del mayorazgo, y residentes en Córdoba, aunque su
habitual morada era Madrid. No era obstáculo para esto la niñez más bien
moral que física de don Diego, pues siendo entonces costumbre emparentar
lo más pronto posible a los mayorazgos, los casaban fresquitos y antes
que tuvieran tiempo de asomar las narices por las rehendijas de la
puerta del mundo, donde al decir de D. Paco, no había sino perdición y
desvanecimiento para la juventud, porque las dulzuras de la copa de los
placeres duraban breves instantes, mientras que sus amargas heces
trascendían por luengos años.

Pero alguien desconcertó o aplazó al menos los planes sabiamente
trazados por doña María y sus ilustres primas; desconcertolos Napoleón,
emperador de los franceses, al poner sus ojos en esta joya del
continente y al invadirla. La guerra, aquella santa guerra de que no nos
muestra otro ejemplo la historia en tiempos cercanos, obligó a suspender
este como otros proyectos, y doña María, que era aragonesa y muy
patriota, hubo de llamar a D. Diego, y desde lo alto de su sitial le
aterró con estas palabras, confiadas después a mi discreción por D.
Paco:

---Hijo mío, mucho te quiero. Tu muerte no sólo nos mataría de pena,
sino que aniquilaría nuestra casa y linaje. Eres mi único varón, eres el
alma de esta casa, y sin embargo, es preciso que vayas a la guerra.
Sangre valerosa corre por tus venas y estoy bien segura de que a pesar
de tus pocos años dejarás en buen lugar el nombre que llevas. Todos los
jóvenes se deben a su rey y a su patria en estos terribles días en que
un miserable extranjero se atreve a conquistar a España. Hijo mío, mucho
te amo; pero prefiero verte muerto en los campos de batalla y pisoteado
por los caballos franceses, a que se diga que el hijo del conde de
Rumblar no disparó un tiro en defensa de su patria. Los hijos de todas
las familias nobles de Andalucía se han alistado ya en el ejército de
Castaños; tú irás también, con un séquito de criados, que armaré y
mantendré a mis expensas mientras dure la guerra.

Al decir esto, la marmórea cara de doña María no se inmutó; pero
Asunción y Presentación lloraron a moco y baba. El joven palpitó de
entusiasmo al verse enviado a tomar parte en un juego que no conocía, y
que visto de lejos es muy bonito.

Nosotros llegamos precisamente cuando se estaban haciendo los
preparativos y el equipo de guerra del mayorazgo. Todos trabajaban en
aquella casa, y no eran las menos atareadas las hermanitas del señor
conde, porque a más de la delicadísima ropa blanca que con sus propias
manos y bajo la inspección de su madre aparejaron, poniéndola con mucho
orden en las gruperas, se ocupaban a toda prisa en arreglar unos muy
lindos escapularios, no sólo para él, sino para todos los de la
comitiva.

No sé qué tenían aquellos preparativos de semejante con los que se hacen
para mandar a un chico al colegio: verdad es que nada hay tan
instructivo y despabilador como un campamento, y por eso decía D. Paco
que la guerra es maestra del ingenio y domeñadora de las impetuosidades
juveniles.

Marijuán fue destinado a acompañar al señorito. Con él y otros criados
formose una legioncilla de cinco hombres; mas sabedora doña María de que
otros jóvenes de familias ricas de Baeza, Bujalance y Andújar habían
llevado hasta diez, mandó que se aumentara aquel número, fijándose al
instante en Santorcaz y en mí. Se nos ofrecía una peseta diaria, además
de lo que cayera si volvíamos con vida y salud; así es que mi compañero
y yo nos miramos, consultando con elocuente silencio el aspecto de
nuestras respectivas fachas. Hallábamonos ambos muy derrotados; y con
aquella escrutadora penetración que da la carencia de posibles, cada
cual conoció la escualidez y vanidad de la bolsa del otro. Santorcaz
opinó que yo debía aceptar el enganche, y yo fui del mismo dictamen
respecto a mi amigo; doña María ofreció equiparnos, mudando nuestras
ropas por otras nuevas y mejores, y además comprometíase a mantener por
algún tiempo a los que ya comenzaban a abrigar algunas dudas acerca del
pan que comerían al llegar a Córdoba. No vacilamos, y henos convertidos
en soldados de caballería, prontos a incorporarnos al pequeño pero
brillante ejército de San Roque. Comprendí que aquel era mi destino, y
que para el fin que a Córdoba me llevaba, más me convenía penetrar en
esta ciudad como soldado oscuro que como desalmado y andrajoso
vagabundo. Santorcaz se decidió después de meditarlo mucho, dando paseos
en la habitación donde se nos había albergado. Una vez resuelto a ello,
pareció muy alegre, y le oí pronunciar algunas palabras que me
demostraban la agitación de su alma por causas para mí desconocidas
entonces. Luego expuso a doña María que no partiría de Bailén hasta no
recibir unas cartas que esperaba de Córdoba y de Madrid, relativas a sus
intereses, a lo cual accedió la señora, diciéndole que permaneciese en
la casa hasta cuando quisiera con la condición de incorporarse después a
la escolta de D. Diego si esta salía antes.

No tardó mucho el día de la partida. El joven mayorazgo estaba vestido
del modo siguiente. Una ancha faja de seda color de amaranto le ceñía el
cuerpo. Sus calzones de ante se ataban bajo la rodilla, y sobre las
medias de seda llevaba gruesas botas de cordobán con espuelas de plata.
El marsellés de paño pardo fino con adornos rojos y azules daba singular
elegancia a su cuerpo, así como el ladeado sombrero portugués, con moña
de felpa negra y cordón de oro. Guarnecía su cintura sobre el fajín, lo
que llamaban charpa, y era un ancho cinturón de cuero con diversos
compartimientos ocupados por dos pistolas, un puñal y un cuchillo de
monte, de modo que aquello equivalía a llevar en los lomos un completo
arsenal, propio para hacer frente a todas las circunstancias
imaginables.

Ocupábanse la madre y las hijas en arreglar los últimos pormenores del
vestido, esta cosiendo el último botón, aquella poniendo un alfiler a la
cinta del sombrero, la otra calzando la espuela al mozo, cuando doña
María dijo con la viveza propia del que recuerda de improviso la cosa
más importante:

---Falta lo principal, falta la espada.

Al punto las miradas de todos fijáronse con cierto respeto en un
venerable armario de añejo roble que en el testero principal de la
habitación desde largos años existía. Acercose a él la señora condesa, y
abriéndolo, sacó una espada larguísima con su vaina y tahalí, las tres
piezas muy marcadas con el sello de honrosa antigüedad. Desenvainó el
acero la propia doña María con gesto majestuoso aunque sin ninguna
afectación de brío varonil, y luego que lo hubo contemplado un instante,
volvió a esconderlo en la vaina entregándolo después a su hijo. Era
aquella espada una hermosa hoja toledana de cuatro mesas y de una vara y
seis pulgadas de largo. En la cazoleta o taza cabía holgadamente una
azumbre, y sus gavilanes nielados de oro, lo mismo que el arriaz, daban
aspecto artístico y lujoso a la empuñadura. Tenía en las dos fachadas
del puño el escudo de los Rumblares, y en el pomo una cabeza con la
empresa del armero toledano Sebastián Hernández. En la hoja, algo
roñosa, se podía deletrear, aunque con trabajo, la inscripción grabada
en uno de sus lados, \emph{Pro Fide et Patria. Pro Christo et Patria.
Pro Aris et Focis}. \emph{Inter Arma silent Leges}.

Colgose al cinto esta poderosa e ilustre tizona el joven D. Diego, para
cuyas manos era exorbitante peso; mas él, orgulloso de llevarlo, hizo un
gesto poco favorable a los propósitos del invasor de España, y se
preparó a salir. Prorrumpieron en copioso llanto Asunción y
Presentación, lo cual dio al traste con la forzada entereza del
condesito, destinado a ser el terror de la Francia, y pasando de los
pucheros a los hipidos y de los hipidos a una violenta explosión de
lágrimas, atronó la casa por espacio de un cuarto de hora. Ni por esas
perdió doña María su serenidad, hablando a su hijo de asuntos extraños a
la guerra.

---Lo primero que has de hacer cuando llegues a Córdoba, es visitar a
mis primas y entregarles estas cartas. Mira, aquí van las señas de su
palacio. Harto sentimos que no pueda celebrarse la boda concertada; pero
Dios lo quiere así, y la patria es lo primero. Algún día será. Di a esas
señoras que si vuelven pronto a Madrid, como me dicen en su última
carta, no les perdono que pasen sin detenerse algunos días en esta su
casa.

Luego tomando distinto tono, habló así:

\emph{---Hijo mío, cuidado con lo que haces. Observa la mejor conducta:
mira que vas a combatir al enemigo y a defender la religión, la patria,
el Estado y el Rey. Si cobarde vuelves la espalda, no vuelvas jamás a mi
casa, ni te acuerdes nunca de tu madre, ni cuentes ya con su tierno
cariño\ldots{} Su indignación, su aborrecimiento eterno, he aquí la
recompensa que te aguarda.}

He subrayado estas palabras, porque son puntualmente históricas; y si no
están en la historia, constan en papeles impresos de aquel tiempo, que
puedo mostrar al que desee verlos. La mujer que las pronunciara (pues no
fue doña María, y el atribuirlo a esta es de mi exclusiva
responsabilidad), añadió lo siguiente, dirigiéndose a otras madres que
despedían a sus hijos en las puertas del pueblo:---«\emph{Compañeras, si
en las batallas llegan a morir todos los hombres, triunfaremos
nosotras}\footnote{Esto pasó en Mérida en 23 de Junio.}».

Salimos de la casa, tomando cada cual la cabalgadura que se le había
destinado, juntamente con un sable y dos pistolas. El bagaje se repartió
entre todos. Un criado antiguo se había encargado del dinero, otro
llevaba las ropas del señorito; Marijuán llenaba sus alforjas con
abundantes provisiones, y en mi grupera pusimos varios encargos y las
cartas que D. Diego debía entregar en Córdoba. Cuando yo las acomodaba
entre mi equipaje, pude de soslayo ver los sobres y me quedé frío de
sorpresa y casi diré de terror; leí los nombres de Amaranta, de la
marquesa su tía y del señor diplomático.

Santorcaz, que hasta entonces no había recibido lo que aguardaba, se
quedó, prometiendo juntarse con nosotros al día siguiente o a los dos
días. Yo le vi muy pensativo y tétrico con las manos a la espalda,
paseando por el portal de la casa cuando salíamos de ella. Hasta fuera
de la villa fue en nuestra compañía D. Paco, el cual recordaba a su
discípulo las máximas de Alejandro sobre la guerra, recomendándole una y
otra vez que las pusiera en práctica al pelear contra los franceses, y
que cuidase de sostener siempre el orden oblicuo disponiendo una segunda
línea para asegurar las espaldas y los flancos, porque a
esto---decía,---debió el gran Macedonio que siempre quedaran victoriosas
sus difalangarquías y tetrafalangarquías.

Con tan sabia máxima que el heredero de Rumblar juró cumplir al pie de
la letra, despidiose don Paco, y seguimos nuestra marcha muy contentos.
No tomamos el camino real desde Bailén a Córdoba por no tropezar con la
retaguardia del general Dupont o con los muchos destacamentos que había
dejado en todos los pueblos, y en vez de las diez y ocho leguas y media
de que consta aquella vía, tuvimos que andar unas veinticuatro, pues en
nuestro rodeo fuimos a Mengíbar; desde allí por Torre Jimeno, siguiendo
un detestable camino de herradura, pasamos a Martos, y de Martos, por
Alcaudete y Baena, fuimos a buscar en Castro del Río la margen derecha
del Guadajoz, que nos condujo a las inmediaciones de Córdoba.

Al salir de Bailén supimos la derrota de los paisanos y soldados de
regimientos provinciales en el puente de Alcolea, y en Alcaudete nos
dieron otra terrible noticia, referente a la entrada de los franceses en
Córdoba y al saqueo de aquella hermosa ciudad. Esto y el encuentro de
algunos hombres dispersados de la partida de Echévarri nos inclinó a
tomar el camino de Écija; pero el día 16 supimos que los franceses
habían evacuado a Córdoba; y adoptando nuestro primitivo itinerario,
divisamos en la mañana del 18 un inmenso caserío blanco, que destacaba
sobre el verde-azul de la lejana sierra infinidad de torres, minaretes,
espadañas y cimborrios.

\hypertarget{xi}{%
\chapter{XI}\label{xi}}

Era Córdoba, la ciudad de Abdherrahmán, la Meca de Occidente, la que fue
maestra del género humano, la vieja andaluza, que aún se engalana con
algunos restos de su antigua grandeza; todavía hermosa, a pesar de los
siglos guerreros que han pasado por ella; ya sin Zahara, sin Academias,
sin pensiles, sin aquellas doscientas mil casas de que hablan los
cronistas árabes; sin califa, sin sabios, pero orgullosa aún de su
mezquita catedral, la de las ochocientas columnas; triste y religiosa,
habiendo sustituido el bullicio de sus bazares con el culto de sus
sesenta iglesias y sus cuarenta conventos; siempre poética y no menos
rica en la decadencia cristiana que en el apogeo musulmán; ciudad que
hasta en los más pequeños accidentes lleva el sello de los siglos;
tortuosa, arrugada, defendiéndose de la luz como si quisiera ocultar su
vejez; escondida en sus interiores donde guarda innumerables maravillas,
y siempre asustada al paso del transeúnte; protectora de los enamorados
para quienes ha hecho sus mil rejas y ha oscurecido sus calles; devota y
coqueta a la vez, porque cubre con sus joyas las imágenes sagradas, y se
engalana y perfuma aún con los jazmines de sus patios. Tal era la ciudad
que había estado entregada por tres días a la brutal y salvaje codicia
de los soldados de Dupont. Este desgraciado general, que desde entonces
comenzó a sentir aquel aturdimiento e indecisión que lo acompañaron
hasta capitular, temeroso de ser sorprendido allí por las tropas de
Castaños, se retiró el 16 de Junio, dirigiéndose a Andújar, desde donde
pidió refuerzos a Madrid.

El 18 entramos nosotros en la ciudad saqueada, aún llena de mortal
espanto. Todavía no había sido lavada la sangre que manchaba sus calles,
ni sabían exactamente los cordobeses a ciencia cierta el dinero y
cantidad de alhajas que se les habían robado. Antes que en contar lo que
les quedaban pensaron en armarse, y si antes habían ido a la lucha,
además de los regimientos provinciales y las milicias urbanas, los
paisanos del campo, después del saqueo todas las clases de la sociedad
se apercibieron para lo que más que guerra era un ciego plan de
exterminio, pues no se decía \emph{vamos a la guerra}, sino a
\emph{matar franceses}.

Desde que entré en la desgraciada ciudad, a la emoción producida por el
espectáculo del reciente desastre se unía la que experimentaba por
asuntos de mi propia cuenta, y por la supuesta proximidad a quien era el
faro de mi vida. Así es que luego que el conde y los de la comitiva nos
arreglamos en una de las mejores posadas, salí con objeto de buscar la
casa de la señora Amaranta y de su tía, lo cual me era sumamente fácil,
por haber visto los sobres de las cartas que traíamos para aquellas
personas. Llegué a eso de las doce a la calle de la Espartería, donde
era su residencia. En lo sucesivo y para evitar confusiones, ya que no
puedo nombrar a la tía de Amaranta con su verdadero nombre, usaré el
título convencional de marquesa de Leiva.

Cuando di los primeros aldabonazos en la puerta, parecíame que golpeaba
en mi propio corazón. ¿Estaría allí Inés? ¿Estaría allí, ya olvidada de
que existiera antes en el mundo un chico llamado Gabriel, arcabuceado
por los franceses? Y si estaba y de improviso me veía, ¿no era posible
que se me presentara deslumbrada por los esplendores de su nueva
posición, y que a la palidez de la primera sorpresa sucediera en su
rostro el rubor de haberme amado? ¿Se acercaba el momento de que yo
cayese de la inconmensurable altura de mi fatuidad amorosa, encontrando
una sonrisa de desdén y la mano de un criado que me pusiera en la calle?
¿Por ventura el trance que me esperaba era hermano gemelo de aquella
otra gran caída ocurrida en el Escorial, cuando por el favor de Amaranta
soñaba con los primeros puestos de la Nación? ¿Bajaría mi alma desde
príncipe a lacayo, como poco antes bajó mi ambición?

Abriome la puerta un criado conocido, a quien rogué me llevase a
presencia de mi antigua ama la señora condesa. Mientras atravesábamos el
patio, buscaba afanosamente algún objeto que me indicase la proximidad
de Inés. Como olfatea el perro buscando el rastro de su amo, así
aspiraba yo las emanaciones de la casa, buscando el aire que había sido
aliento de aquella naturaleza querida. No oí su voz, ni sentí sus pasos,
ni vi cosa alguna que tuviera las huellas de su mano. A mí se me
antojaba que en cualquier objeto podía notar un sello especial que
indicara pertenecerle. En nada de lo que vieron mis ojos encontré la
huella indefinible que debía tener todo aquello en que Inés pusiera los
suyos. Esto se comprende y no se explica. El corazón es el único
adivino, y el mío me dijo que Inés no estaba allí.

El patio era fresco y risueño, como todos los de las buenas casas de
Andalucía. Entre los jazmines reales, que abrazándose a una columna
ostentaban sus mil florecillas llenas del perfume más grato a los
enamorados; entre los naranjos de la China, graciosas miniaturas del
naranjo común; entre los rosales de la tierra y esos claveles indígenas
cuya imperial hermosura no ha logrado eclipsar ninguna de las elegantes
flores modernas; entre los tiestos de reseda, de mejorana, de albahaca y
de sándalo, saltaban los chorros de una fuente habladora, con cuyo
monólogo se concertaba el canto de algunos pájaros prisioneros en
doradas jaulas. El pavimento era de mármol y los zócalos de azulejos;
sobre estos, y cubriendo gran parte de la pared, había cuadros al óleo
de aquella escuela andaluza que ha llevado a los lienzos el tono
caliente de la tierra, la esplendidez de la inflamada atmósfera y la
agraciada melancolía de los semblantes.

Afortunadamente para mí, Amaranta se dignó recibirme. Estaba en una sala
baja, fresca y oscura, y cuando yo entré se ocupaba en armar unas flores
de altar. ¿Se había entregado a la devoción? Vestía completamente de
blanco, y a la exigencia de la moda se había unido el rigor de la
estación para que aquel ligero traje fuera nada más que lo absolutamente
necesario para cubrir su hermoso cuerpo. Entonces entre las miradas de
fuera y el pudor interno no se ponía tan gran baluarte de telas como se
pone hoy.

Amaranta estaba abrumadoramente hermosa, y sus ojos negros, que eran,
como otra vez he dicho, los primeros ojos del mundo, es decir, los
Bonapartes de la mirada humana, conquistaban al punto todo aquello a que
dirigían su pupila. Sentí en su presencia mucha cortedad, mucha
turbación; sentime sin ideas y sin palabra.

---¿Qué vienes a buscar aquí?---me dijo.

---Señora, he venido a Córdoba para afiliarme en el ejército del general
Castaños, y sabiendo que Su Excelencia y apreciable familia estaban en
esta población, he querido visitar a mi antigua y querida ama.

---Eres tan hipócrita como intrigantuelo y trapisondista---repuso entre
severa y amable.---¿Conque me tienes ley? ¿Por qué te portaste tan mal
conmigo?

---Señora---exclamé haciendo aspavientos de respeto.---¡Yo portarme mal!
Si no puedo olvidar lo bien que estaba al servicio de Su Excelencia.

---¿Quieres ser otra vez mi criado?---me preguntó.

Esta proposición cayó sobre mí como un rayo. Pensé en Inés, en el
repentino engrandecimiento de la que había juzgado compañera de mi vida,
y al considerarme criado de aquella casa, temblé de indignación.

---No señora, no quiero servir más. Soy soldado---repuse.---Sin embargo,
estoy a las órdenes de Vuecencia para lo que guste mandarme.

---¿Conque soldado? ¿Y vas a la guerra? Dentro de un mes serás
general---dijo con punzante ironía.

---No aspiro a tanto. Quiero servir a mi país, y nada más. Con tal de
que mañana pueda decir: «contribuí a echar de España a la canalla»,
quedaré satisfecho.

---¿Y crees que España podrá echar fuera a la canalla? ¡Ah!, yo no
participo de la ilusión de esta buena gente. ¿Qué pasó el día 9 en el
puente de Alcolea? Aquellos pobres paisanos, a quienes no se puede negar
el valor, huyeron ante las tropas disciplinadas del general Dupont. En
Córdoba tampoco se les puso resistencia, y ¡qué horror, Dios mío!, ¡qué
tres días de angustia! Todos creíamos que los franceses entrarían con
bandera de paz, porque la gente de Echévarri abandonó la ciudad, y los
de aquí no trataban de hacer resistencia. Llegaron los franceses a la
Puerta Nueva, y mientras las autoridades hablaban con ellos para darles
entrada, de una casa cercana salieron algunos tiros. Furiosos los
enemigos, después de derribar la puerta a cañonazos, desparramáronse por
las calles de Córdoba asesinando a cuantos encontraban al paso y
metiéndose en las casas para coger cuanto había. No puedes figurarte lo
que era aquello. Mudos de espanto y ansiedad estábamos todos aquí,
atento el oído a los rumores de la calle, cuando sentimos que las
puertas caían a golpes, y penetraba aquella soldadesca bestial, diciendo
que se les entregasen todos los objetos de valor. El miedo nos impidió
andar en contestaciones con ellos, y al punto les dimos alhajas, dinero,
plata de mesa y cuanto había, deseando que se lo llevasen todo de una
vez para no escuchar sus insultos. Mas luego bajaron a la bodega
sedientos de vino: no contentos con echar fuera las cubas pequeñas,
bebían en las llaves de las pipas grandes, y dejándolas luego abiertas,
corría el Montilla de setenta y cinco años inundando las cuevas. Uno de
aquellos salvajes pereció ahogado en vino. Pero al fin se fueron de casa
sin cometer atrocidades de otra clase, y nos vimos libres de semejante
chusma. En otras partes los horrores no pueden contarse. Robaron todo el
dinero de la administración, toda la plata de los conventos, los vasos
sagrados, los cálices, las custodias, las alhajas de las imágenes;
penetraron también en los conventos de frailes, muchos de los cuales
murieron asesinados; convirtieron en lupanar la iglesia de Fuensanta, y
por tres días Córdoba no fue una ciudad, fue un infierno, porque todos
los demonios, todas las maldades y abominaciones cayeron sobre ella. Por
las calles se les encontraba borrachos, llenos de inmundicia, y se
revolcaban en el lodo, engullendo vorazmente la comida que sacaban a
viva fuerza de las casas. Los generales franceses, avergonzados de tanta
bajeza, querían someterlos a palos; pero fue preciso emplear mucho
rigor, y algunos hubieron de ser fusilados para hacer entrar en razón a
los demás. Por último, saliendo de Córdoba para Andújar, esos cafres nos
han dejado en paz por algún tiempo. ¡Qué espantoso estado el de España!
Y lo peor es que sucumbirá. ¡Qué horrores, qué días terribles nos
aguardan! ¿Y en Madrid qué tal se vive?

---¿Piensa usía volver a la corte?

---¡Oh! Sí\ldots{} Pensamos marcharnos pronto, porque nos llama un
asunto en que está interesada toda la familia. A ser por mí, ya
estaríamos allá. No puedo vivir en Córdoba, y menos en el estado actual
de las cosas. Esto no es vivir. Si en Madrid no hubiese tranquilidad,
nos iríamos a Bayona con toda la familia.

---¿Y ninguna de las personas de esta casa fue maltratada por la
soldadesca francesa?---pregunté deseando saber qué personas había en la
casa.

---Ninguna: sólo mi tío el marqués tuvo una contusión en la cabeza; pero
recibiola al esconderse debajo de una cama, y lo hizo con tanto ímpetu
que se dio un golpe muy fuerte contra el suelo. Un amigo de casa, que
nos visita todos los días, D. José María de Malespina, también recibió
un ligero rasguño en la mano derecha al ocultarse detrás de un armario.

---¿Y las señoras? Oí decir que una sobrinita de la señora
marquesa\ldots{} o sobrinita de Su Excelencia, no estoy bien seguro,
había venido de Madrid a acompañarlas.

---No,---contestó Amaranta mirando al suelo.

---Pues entonces lo confundo yo con otra cosa. Paréceme que en Madrid lo
oí decir en Madrid al señor licenciado Lobo, aquel famoso
escribano\ldots{} pero no, seguramente se equivocó.

---¿Conoces tú al Sr.~de Lobo?---me preguntó con inquietud.

---Ya lo creo: somos muy amigos. Le conocí cuando yo servía en casa de
D. Mauro Requejo\ldots{} y por cierto que el señor licenciado y yo
tuvimos una cuestión con motivo de cierta muchacha\ldots{} una infeliz,
señora, una desgraciada chiquilla, huérfana de padre y madre.

---A ver, cuéntame eso---dijo con interés.

---Pues los señores de Requejo que eran dos puerco-espines, martirizaban
a la damisela. Yo tenía lástima de ella, y quise sacarla de allí\ldots{}
pero me fusilaron los franceses.

---¡Te fusilaron!

---Sí señora; y el Sr.~de Lobo\ldots{} pues\ldots{} lo cierto fue que la
muchacha desapareció.

---Ya\ldots{} Cuéntamelo todo.

Con el mayor afán, con el interés más grande que durante mi vida he
sentido por cosa alguna, empezaba a contar a Amaranta lo que sabía,
cuando la entrada de dos personas me interrumpió.

Eran el diplomático y D. José María de Malespina, aquel por tantos
títulos famoso aunque retirado coronel de artillería de quien hablé
cuando lo de Trafalgar. El primero me reconoció y tuvo la bondad de
dirigirme algunas bromas.

\hypertarget{xii}{%
\chapter{XII}\label{xii}}

---Sobrina---dijo el marqués,---ya pronto tendremos aquí las tropas de
Castaños. ¿Sabes lo que ahora le decía al Sr.~de Malespina? Pues le
decía que si la Junta de Sevilla me comisionara para entrar en
negociaciones con los franceses, tal vez lograría poner fin a esta
desastrosa guerra.

---¿Qué negociaciones, ni qué ocho cuartos?---dijo con desprecio
Malespina.---¡Oh! ¡Si la Junta de Sevilla siguiera el plan que he
imaginado estos días! Mientras no demos a la artillería el lugar que le
corresponde, no es posible alcanzar ventaja alguna. Mis recientes
estudios sobre cyclodiatomía y catapéltica, me han hecho descubrir
importantes principios que ahora debieran llevarse a la práctica.

---Reniego de la ciencia que inventa medios de destrucción---exclamó con
gesto elocuente el marqués,---cuando por las vías diplomáticas pudieran
las Naciones resolver todas sus querellas. ¡La guerra! ¿De qué sirve la
guerra? ¿Vale la pena de que perezcan miles de seres humanos por una
cuestión que podría arreglarse con un pedazo de papel y una pluma mojada
en tinta, puesta en manos de alguna persona que yo me sé?

---Hombre de Dios, sin la guerra ¿qué sería del mundo? Y sobre todo,
¿qué sería del mundo sin la artillería? Montecúculi dice que las
batallas \emph{dan y quitan las coronas, concluyen las guerras e
inmortalizan al vencedor}.

---¡Sangre y luto y desolación! Pero no disputemos sobre el volcán,
amigo. La guerra es un mal, pero existe hoy entre nosotros. Lo que
conviene es buscar alianzas en Europa. Por eso desde que llegué a
Andalucía sugerí a la Junta Suprema la idea de pedir auxilio a
Inglaterra. Magnífico pensamiento, que ni a Saavedra, ni al padre Gil se
le había ocurrido.

---Y ¡Vd. se atribuye la invención!---dijo con sorna Malespina.---Pero
hombre de Dios, si los asturianos fueron los primeros que en tal cosa
pensaron, y desde el 30 de Mayo salieron de Gijón mis queridísimos
amigos D. Andrés Ángel de la Vega y el vizconde de Matarrosa, hijo del
conde de Toreno\ldots{} ¡Bah, bah!\ldots{} Si estos diplomáticos han
perdido la chaveta. Nada, amigo mío, yo le dije al padre Gil que cuidara
de aumentar la artillería, adoptando los adelantos que yo quiero
introducir en el arma. Pues qué, ¿cree usted que Napoleón no tiene
noticia de ellos? Yo he descubierto que antes de invadir a España, mandó
una comisión secreta para que averiguara si estaba yo aquí. Como
entonces mi familia hizo correr la voz de que yo había pasado a América,
Napoleón dijo: «Pues no hay cuidado ninguno», y ordenó la invasión. Ya,
ya me conoce él de muy antiguo.

---¡Qué vanaglorioso es Vd.!---dijo el diplomático con mayor fatuidad
que la de su amigo.---Eso lo dice usted por obligarme a hablar, por
obligarme a que revele\ldots{} no: es secreto de Estado, del cual quizás
depende la paz de España y de Europa, no saldrá de mis labios, ni soy
hombre que cede fácilmente a las sugestiones de la curiosa e imprudente
amistad.

---Todo eso es pura farsa. Sepamos de una vez esos secretos.

---¡Farsa!---exclamó con enojo el diplomático.---Pero ya comprendo el
juego. Lo mismo hace mi sobrina cuando quiere obligarme a que revele los
secretos de Estado. No, callaré, callaré, aunque Vd. me insulte, aunque
Vd. aparente dudar de mi veracidad, para que la indignación me haga
romper el secreto. ¡Pues qué!, si yo dijera que un elevado personaje, el
más poderoso que hoy existe en el mundo, se decidió al fin a transigir
conmigo, después de una enemistad que data desde la paz de Luneville; si
yo dijera que los preliminares de negociación que entablé para evitar a
España los horrores de la guerra, comenzaban a dar resultado, cuando
algunos hombres pérfidos\ldots{} si yo dijera esto\ldots{} pero no: mi
sobrina me mira como para incitarme a seguir hablando, y Vd. Sr.~de
Malespina, me mira también\ldots{} mas no, punto en boca, y cesen las
impertinentes preguntas que en vano amenazan el inexpugnable alcázar de
mi discreción.

---Todo eso es pura fábula---afirmó D. José María con
desenfado.---Aborrezco la falsedad y la jactancia, pues soy hombre que
se dejaría hacer picadillo antes que decir una palabra contraria a la
rigurosa verdad. Por tanto basta de fingidas diplomacias y de tratados
que no han existido sino en la cabeza de Vd. En estos momentos seamos
soldados, y dejemos a un lado los protocolos. Veremos si ahora, cuando
en Bayona se sepa que yo sigo en España y que no pienso en partir a
América, se retiran los franceses de nuestro país, porque
francamente\ldots{} Napoleón me conoce.

---Hombre, eso es demasiado fuerte---exclamó el diplomático soltando la
risa.---Conque Napoleón\ldots{}

---No extraño esas risas---dijo muy amoscado el artillero.---¿Qué ha de
hacer quien no conoce el peligro personal; qué ha de hacer un hombre que
cuando entraron los franceses a saquear esta casa, se escondió debajo de
la cama?

---Yo\ldots---contestó con turbación el marqués,---si penetré en aquel
apartado sitio, bien saben todos la causa, que no fue miedo ni mucho
menos. En aquel instante me ocupaba mentalmente en buscar los términos
más propios de un arreglo y transacción con aquella gente, y como el
ruido no me dejaba pensar, busqué la soledad de aquel lugar recogido y
pacífico, donde sin estorbo pudiera entregarme a mis sutilísimas
disquisiciones. Lo incomprensible es que un militar viejo como Vd.
buscase asilo detrás de un armario, mientras los franceses insultaban a
las señoras.

---Nada, lo que he dicho siempre---repuso Malespina.---Es inútil esperar
que los profanos hagan nunca justicia a las combinaciones de la ciencia.
Todo lo ven bajo el aspecto vulgar, y lanzan al público las acusaciones
más irreverentes. Hombre de Dios, ¿necesitaré explicar mi conducta?
¿Necesitaré decir que, convencido desde el principio de la imposibilidad
de establecer en el patio un campo atrincherado, tuve que retirarme a
esta sala, y apoyar mi centro de retaguardia en aquel armario, para
operar con el ala derecha? Viendo que se acercaban con ímpetu formidable
los franceses, hice un movimiento envolvente sobre mi ala izquierda, y
me metí tras el armario, dirigiendo el raso de metales de la terrible
arma de fuego que llevaba en mi bolsillo hacia el marco de la puerta,
para que la trayectoria fuese directamente al patio. El enemigo, al ver
mi actitud, retrocedió lleno de espanto, y he aquí cómo sin efusión de
sangre se les obligó a la retirada.

Amaranta no podía contener la risa oyendo la disputa entre su tío y su
amigo. Antes de que esta concluyera, entró la marquesa de Leiva y dijo:

---Acaba de llegar la \emph{Gaceta Ministerial de Sevilla}. Creo que hoy
trae la noticia de que ha muerto Napoleón.

---¡Jesús! ¿Qué dice Vd.?

---¿Dónde está, dónde está esa \emph{Gaceta}?

Al punto corrieron el marqués y D. José María a la habitación inmediata.
La marquesa, que no había parado mientes en mi persona, aunque le hice
reverencias muy profundas, acercose a Amaranta, y mostrándole un
medallón que en la mano traía, le dijo:

---¿Te gusta? ¿No es verdad que está parecido? El pintor ha hecho un
hermoso retrato.

---Está muy bonito y se parece mucho---dijo mi antigua ama.---Veremos
qué le parece a ese barbilindo cuando lo vea.

---Es extraño que no haya llegado ya. Su madre me decía que para el 12
pasaría por aquí.

El diplomático y Malespina aparecieron de nuevo, trayendo cada cual una
hoja de papel impreso.

---Efectivamente, aquí está en letras de molde---dijo con grandes
aspavientos el diplomático preparándose a leer.---Oigan Vds.:
\emph{Madrid 6 de Junio. El descontento de las tropas enemigas parece
general, y corre muy válida la voz de que en Bayona hay insurrección y
de que el Emperador está oculto, añadiendo algunos que herido}.

---Hombre, eso es importantísimo---exclamó Malespina,---aunque no me
coge de nuevo, porque ya tenía noticias detalladas de este suceso.

---¿Que los franceses se sublevan contra Napoleón?---dijo la
marquesa.---Dios les habrá tocado el corazón.

---Pero oigan Vds. estotra noticia---añadió Malespina.---\emph{Toledo 4.
Dícese que cerca de Gallur los franceses han sido derrotados por
Palafox, dejando en el campo de batalla 12.000 muertos y un número
infinito de heridos. Los españoles les tomaron 48 cañones y 12 águilas}.

---Hombre, magnífica victoria---exclamó el diplomático---¿Pero qué dice
aquí? ¡Oh, esta sí que es gorda!\emph{Reus 8 de Junio. Aquí se habla de
la muerte de Josef Napoleón, de los varios partidos que dividen la
Francia y de la sublevación del Rosellón. Si estas noticias salen
ciertas, podemos asegurar que llegó ya el día de la venganza y de la
libertad de España.}

---Vienen muy satisfactorios estos dos números de la
\emph{Gaceta}---dijo Amaranta.

---Ya sabía yo todo eso---afirmó con aplomo el marqués.---¡Pero que veo,
santos cielos! Este sí que es notición. Oigan todos, oiga Vd., Sr.~D.
José María: \emph{Valencia 10 de Junio. El ejército de Duhesme ha sido
derrotado. Corren voces de que el castillo de Figueras está en nuestro
poder; se repite la noticia del levantamiento del Rosellón y de la
indignación con que ha visto toda la Francia la conducta de su Emperador
con la España.}

Los sueltos que oí leer en aquella ocasión pueden verse en la
\emph{Gaceta Ministerial de Sevilla}, periódico oficial de la Junta
Suprema. En sus breves columnas se insertaban diariamente despachos y
noticias que remitían de todas partes. Dictábalas el entusiasmo y las
devoraba la credulidad, y como nadie las discutía, el efecto era
inmenso. Según la \emph{Gaceta Ministerial}, todos los días era
derrotado un ejército francés, y todos los días ocurría en Francia una
insurrección para destronar al azotador de Europa. ¡Ah!, entonces
corrían unas bolas, junto a las cuales son flor de cantueso las
equivocaciones del moderno telégrafo.

---Oigan Vds.---exclamó la marquesa, que había tomado el periódico de
manos del marqués;---esta sí que es noticia extraordinaria. Y no digan
Vds. que la sabían, porque hasta ahora no se ha hablado en España ni en
el mundo de semejante cosa. Atención: \emph{Cádiz 14. Corre muy válida
la voz de que la Francia está dividida en tres partidos: borbónico,
republicano y bonapartista}. También dice que han desembarcado en Rosas
11.000 hombres con armas que vienen de Mallorca.

---¡Tres partidos!---exclamó el diplomático mirando a D. José María.

---¡Tres partidos! Ya lo sabía.

---¡Y yo también!\ldots{} Pero corro a comunicar esta nueva a nuestros
amigos---dijo el marqués levantándose.

---Aguarda---le indicó su hermana.---No olvides que esta tarde tienes
que pasar por allí.

---¡Otra vez!---exclamó el diplomático.---Si no hay quien la haga salir.
Le he prometido, le he rogado, le he amenazado, le he dicho mil finezas
y ternuras, y nada, no quiere salir. ¿Por qué no vais vosotras?

---Sí, esta tarde iremos---afirmó detenidamente la marquesa.---Es
preciso hacerla salir; porque sin ella no podemos volver a Madrid.

---¡Oh!, picarón\ldots{} ya sabemos el secreto---dijo Malespina
dirigiéndose con maliciosa expresión al marqués.---Ayer me hablaron del
caso en varias tertulias\ldots{} Ya sabía yo que había Vd. sido un
terrible seductor\ldots{} ¿Pero ahora salimos con eso?

---Amigo, es preciso reparar de algún modo los extravíos de una
borrascosa juventud. Ya sabe usted que hasta hace quince años me
llamaban el \emph{azote de las familias}. Pero ya pasaron aquellos
tiempos, y ahora\ldots{}

---¿De modo que no vas esta tarde?

---Francamente---dijo el marqués,---en estos días me gusta salir a la
calle lo menos posible. Suele haber tumultos\ldots{} ¡la gente anda tan
excitada!\ldots{} ¡Qué susto me llevé la otra tarde en el barrio de San
Lorenzo!\ldots{} y como a causa de la gota no puedo correr\ldots{}

---Y como en la calle no se encuentran camas para esconderse debajo de
ellas\ldots{} Vamos, vamos, señor marqués, y leeremos a los amigos estas
estupendas novedades.

~

Salieron la artillería y la diplomacia, y como la marquesa había salido
de la habitación un momento antes, quedamos solos otra vez Amaranta y
yo.

---Sigue contando---me dijo.---Y ese señor tendero con quien servías,
¿ha venido contigo a Córdoba?

---No señora, yo no he vuelto más a aquella casa. Salí de Madrid
acompañando al Sr.~de Santorcaz.

---¡Santorcaz!---exclamó la dama, poniéndose encarnada y después pálida
como una difunta.---¿Quién? ¿Quién has dicho?

---D. Luis de Santorcaz, señora, un caballero castellano que ha venido
ahora de Francia. Amaranta parecía experimentar una conmoción profunda.
Para disimularla se levantó fingiendo buscar algo, dio media vuelta,
sentose de nuevo, después se puso la mano sobre los ojos, y finalmente,
rompió una flor de trapo que tenía entre sus manos.

---¿Qué estabas diciendo, que no te oí\ldots?---me preguntó.

---Que el Sr.~de Santorcaz\ldots{}

---Deja a ese hombre\ldots{} no hables de lo que no me interesa. ¿Conque
antes decías que los tenderos de la calle de la Sal martirizaban a la
joven\ldots?

---Sí señora, mucho. Aquello me desgarraba el corazón---contesté sin
cuidarme de disimular los tiernos sentimientos de mi alma.

---Era natural que te interesaras por la desgracia.

---Es que yo había conocido a Inés antes de que fuera a aquella casa. La
había conocido cuando estaba con su tío el buen D. Celestino del Malvar.
Nos conocíamos los dos, señora, y como ella era tan buena, y yo
también\ldots{} porque yo era muy bueno\ldots{} En fin, señora, yo no
puedo ocultar a usía la verdad.

---Dímela de una vez.

Dejándome llevar de la impetuosa pena que pugnaba por desbordarse en mi
afligido pecho, y olvidando toda consideración, todo tacto, toda
prudencia, con el acento de la verdad y de un dolor inmenso, dije lo
siguiente, sin reflexión ni cálculo alguno:

---Señora, Inés y yo éramos novios\ldots{} Yo la amo, yo la
adoro\ldots{} ella también\ldots{}

Amaranta se levantó rápidamente, y en su semblante observé señales de
repentina cólera. Mandándome callar, después de decirme que era un
desvergonzado y un truhán, agitó con inquieta mano una campanilla.

¡Altos cielos! ¡Por qué no os hundisteis sobre mí! Entró un criado, y
Amaranta le mandó que me pusiera al instante en la puerta de la calle.

\hypertarget{xiii}{%
\chapter{XIII}\label{xiii}}

El criado, cumplidor de la ignominiosa orden, era un segundo mayordomo
llamado Román, que desde su niñez servía en la casa. Desde que le conocí
en el Escorial, aquel hombre me había inspirado inexplicable antipatía,
y digo esto y además le nombro, para que mis lectores le tengan
presente, por si casualmente figurase después un poco en los raros
sucesos de esta historia.

¿Será preciso que hable de mis tormentos morales en los días siguientes
a aquel suceso? ¡Dios mío! Voy a aburrir a mis lectores, abusando de la
gentil cortesía que les movió a fijar sus ojos en estas relaciones. No,
más vale que devore en silencio mis penas y les hable de otros asuntos,
que así alcanzaré la doble ventaja de proporcionarles útil
entretenimiento, y de calmar mis pesares, adormeciéndoles con el beleño
de patriótico entusiasmo.

En Córdoba reinaba gran impaciencia por la tardanza del ejército de
Castaños. Entonces, como ahora y como siempre, los profanos en el arte
de la guerra arreglaban fácilmente las cuestiones más arduas, charlando
en cafés y en tertulias, y para ellos era muy fácil, como lo es hoy,
organizar ejércitos, ganar batallas, sitiar plazas y coger prisionero a
medio mundo. A los profanos se unían los bullangueros y voceadores que
entonces ¡santo Dios!, pululaban tanto como en nuestros felices días, y
entre aquellos y estos y el torpe vulgo, armaban tal algazara, que no sé
cómo las Juntas y los generales podían resistirla.

Principiaron a hacerse comentarios muy diversos sobre la lentitud con
que Castaños organizaba sus tropas; unos aseguraban que tenía miedo;
otros que estaba decidido a dar la batalla, pero que seguro de perderla,
tenía tomadas sus medidas para retirarse a Cádiz y huir a América con lo
más granado de sus tropas; otros, en fin, se atrevieron a más, y
pronunciaron la palabra traidor. Esta palabra no era entonces palabra,
era un puñal: víctimas de ella fueron Solano en Cádiz, Filangieri en
Galicia, Cevallos en Valladolid, Ordóñez en Palencia, el conde del
Águila en Sevilla, Trujillo en Granada, Torre del Fresno en Badajoz, el
barón de Albalat en Valencia. Inútil era decir a los impacientes de
Córdoba que un ejército no se instruye, arma y equipa en cuatro días:
nada de esto entendían. Aunque al través del tiempo nos parezca lo
contrario, entonces se chillaba mucho, y también había quien tomara muy
a pechos los asuntos de la guerra sólo por el simple placer de meter
ruido, y también para hacerse notar. Todos los días oíamos decir:
«mañana viene el ejército» o «ya ha salido de Utrera, ya está en
Carmona\ldots» Pero pasaban días y el ejército no venía.

En tanto en Córdoba no cesaban los trabajos. Si no tienen Vds. idea de
lo que es el delirio de la guerra, entérense de aquello. En estos
tiempos modernos, si ocurre una guerra, las señoras, llevadas de sus
humanitarios sentimientos, se ocupan en hacer hilas. ¡Ay!, entonces las
señoras tenían alma para ocuparse en fundir cañones. Cuando tal era el
espíritu de las mujeres, figúrense Vds. cómo estarían los hombres.
¡Hilas! Allí nadie pensaba en tales morondangas.

Los voluntarios y cuerpos francos se uniformaban según el gusto
indumentario de cada uno, y aquí de la imaginación de las hembras de la
familia, para galonar marselleses, para emplumar sombreros, y guarnecer
charpas y polainas. Se hicieron muchos uniformes; pero no bastaban para
equipar los dos regimientos, uno de caballería y otro de infantería que
organizó la Junta de Córdoba. Sin embargo, este inconveniente se obvió,
disponiendo que con cada prenda de vestir se cubriesen dos: el uno
llevaba los calzones, casaca y sombrero, y el otro el pantalón, chaqueta
y gorra de cuartel. El correaje también servía para dos: uno llevaba la
bayoneta en la cartuchera y el otro en el porta-bayoneta, y no
alcanzando las cartucheras y cananas, se suplían con saquillos de
lienzo. Más adelante, cuando tenga el gusto de describiros en su
conjunto el ejército de Andalucía, daré completa idea de su abigarrada
conformación y aspecto. Francamente, señores, era aquel un ejército que
movía a risa.

Durante los días que aguardamos la llegada de Castaños para incorporamos
a él (y necesariamente tengo que volver a hablar de mí), yo hacía una
vida vagabunda y holgazana. Como el servicio del joven D. Diego no
exigía más que presentarme en la posada a la hora de comer, pasaba el
día y parte de la noche discurriendo por aquellas tortuosas calles, que
convidan al transeúnte a perderse por ellas, entregándose al azar, a lo
aventurero, a lo desconocido, sin saber a dónde se va, ni de dónde se
viene. Por ser la soledad mi mayor gusto, rechazaba la compañía de mis
camaradas, buscando errante y solo aquellos lugares donde más pronto me
perdía.

El único sitio adonde iba deliberadamente todos los días era la casa de
Amaranta, y pasaba largas horas contemplando su puerta, con los ojos
fijos en las desnudas paredes, como si quisiese leer en ellas alguna mal
escrita página de mi destino. Sus cerradas ventanas, sus espesas
celosías, no daban paso a ninguna esperanza. Sin embargo, aquella
fachada era tan elocuente, que no podía dejar de mirarla. Al apartarme
de allí, el viejo muro con su puerta, sus ventanas, sus aleros y sus
miradores, quedaba tan presente en mi imaginación como si fuese una
fisonomía. ¡Cara funesta que nunca tuvo una sonrisa para mí! Los criados
de la casa, a quienes impacientemente preguntaba por Inés, no sabían o
no querían darme noticia alguna.

Pero un día, precisamente el 1º de Julio, cambió repentinamente la
situación de mi espíritu. Atiendan ustedes que esto es de suma
importancia. Por fin, tras larga espera llegó el ejército del general
Castaños, y al anochecer debía partir para el Carpio. Entre los paisanos
armados que se juntaron con Echévarri, existía un grupo compuesto de
contrabandistas de Sierra-Morena, de Villamanrique y de Pozo Alcón, con
los cuales fraternizaron bien pronto formando amistosa cuadrilla, los
licenciados de Málaga, batallón que se formó con alguna gente condenada
por faltas, y que la Junta tuvo a bien indultar. Estos caballeros para
cuya domesticación emplearon grandes rigores los jefes militares, tuvo
una reyerta en Córdoba con los suizos de Reding. Fue cuestión de vino,
prontamente aplacada; pero que, sin embargo, alarmó el barrio de Santa
Marina durante media hora, produciendo sustos, algunas corridas, tal
cual desmayo de sensibles mujeres, las que al oír los dos o tres tiros
disparados en la colisión creyeron que los franceses estaban otra vez
sobre Córdoba, y así lo gritaban corriendo desordenadamente por las
calles. La parte mayor de la ciudad no se enteró de este suceso, que
insignificante en las páginas de la historia patria, fue para mí de
trascendencia suma, y más digno de mención que si hubiese derribado
añejos tronos y alterado la geografía del continente. Así los granos de
arena pesan a veces como montañas en el destino de un ser humano, y lo
que es gota de agua en el cauce de la generalidad, es río impetuoso en
el de uno solo, o viceversa, según lo que nosotros llamamos antojos de
allá arriba, y no es sino concierto sublime, que no podemos comprender,
como no puede una hormiga tragarse el sol.

Pues bien: algunas horas antes de la que señalaron para la partida, salí
a la calle, impulsado por un sentimiento de amor hacia los laberintos de
aquella ciudad que en sus repliegues escondidos había dado un asilo a mi
tristeza. Sentía salir de Córdoba, como siente el ermitaño dejar su
cueva. Me había acostumbrado tanto a pasear mi aburrimiento y soledad
por aquellos callejones, a quienes en cierto modo había hecho
confidentes de mi pesar; hallaba tantas perspectivas amigas en un
recodo, en una torre, en un ajimez, en una encrucijada, en un poste, en
una reja, en una piedra corroída por el tiempo, en un zócalo garabateado
por los chicos, que no pude menos de salir a dar el último adiós a todas
aquellas mudas compañías de mi tristeza. Aquel día estaba más triste que
nunca.

Era de tarde: pasé por una plazuela irregular solitaria e irregular, de
esas que son la desesperación de los arquitectos modernos: a un lado
muros de ladrillo, en los cuales por la disposición de este material se
ha querido imitar una decoración greco-romana, con jambas, dentículas,
capiteles, metopas y triglifos; a otro una pared sin puertas ni
ventanas, luego un descomunal portalón, una esquina cargada de escudos,
un farol, un santo, torres medio caídas y machones que se van a caer;
una plazuela, en fin, de esas que nos salen al paso cuando visitamos
cualquier vieja metrópoli, tal como Toledo, Granada, Valladolid, León,
etc\ldots{} Al atravesarla sentí el ruido que cerca producía la citada
reyerta entre los licenciados y los suizos: oíase lejana algazara, y al
extremo de largo callejón vi algunas mujeres que corrían gritando. Esto
despertó mi curiosidad y marché hacia allí; pero no había dado dos
pasos, cuando me detuve asombrado y estremecido, porque en el fondo de
la plazuela, y en el ángulo que esta formaba con una calle, vi una mano
que me hacía señas; sí, una mano blanca que me llamaba.

Dirigime allá y en unos cuantos segundos se disipó la ilusión. Me reí de
mi torpeza al observar que en el ángulo mencionado había una imagen de
la Virgen de esas que la devoción de los españoles ha puesto en las
antiguas calles. La Virgen tenía una corona de hierro, en cuyos picos
debió de haberse enredado una cometa de algún chico de la vecindad, pues
un jirón de papel, todavía suspendido junto al cuerpo de la sagrada
estatua, se movía a impulsos del viento. Aquello fue lo que a mí me
pareció un brazo que se movía y una mano que me llamaba. Tal alucinación
en pleno día era señal de mi estupidez, por lo cual burlándome de mí
propio, seguí mi camino.

Pasando bajo la imagen, contemplaba el jirón de la cometa, cuando me
detuve de nuevo, porque un objeto rozó mi cara produciéndome cierto
escalofrío. El jirón de papel se había desprendido de la imagen cayendo
sobre mí. ¡Vean Vds. lo que es el estado del ánimo! Aquel hecho
insignificante, tan insignificante como el aplastamiento de un grano de
arena con nuestro pie, me hizo detener el paso, me hizo temblar, me hizo
mirar a todos lados, puso en mis labios esta pregunta que me dirigí
lleno de confusión:---Pero Gabriel, ¿te has vuelto bobo, o lo has sido
toda tu vida?

Seguí andando hacia la acera de enfrente, cuando de nuevo me detuve, me
quedé helado, absorto, estupefacto, porque detrás de mí había sonado
claramente mi nombre. ¿Quién me llamaba? Volvime y nada vi. La plazuela
estaba enteramente desierta y muda: sólo a lo lejos se oían apenas
algunas voces del altercado, que de ningún modo podían confundirse con
la que a mi espalda había dicho: «Gabriel».

Al volverme, mis ojos se fijaron en una puerta; era la puerta de una
iglesia. Abiertas de par en par las hojas de madera chapeada, se veía el
cancel de mugriento cuero, con dos puertecillas laterales. Una vieja, al
salir, puso en movimiento las mohosas bisagras, y al ruido de la
herrumbre, un sonido lastimero llegó a mis oídos, modulando aquella voz
que a mí me había parecido mi nombre. Esta vez no me reí, sino que entré
decididamente en la iglesia. Vi muchos santos pintados o de escultura, y
¡cosa singular!, pareciome que todas las imágenes sonreían
apaciblemente. La iglesia era modesta, blanca, oscura. En los lustrosos
bancos se sentaban algunas señoras de edad: las luces del altar, al
reflejarse en los oropeles de un luengo cortinón rojo que servía de
dosel a la Virgen, brillaban, estrellas tembladoras de aquella dulce
oscuridad, indicando a dónde debían dirigirse los piadosos ojos. Al poco
rato de estar allí, pareciome aquel interior menos oscuro, y comencé a
ver distintamente todos los objetos. En el fondo de la iglesia, frente
al altar, había una gran reja que se alzaba desde el suelo al techo;
tras esta reja percibíanse vagas claridades movibles y un murmullo
sordo, de cuyo conjunto se destacaba de rato en rato una sílaba o una
tos que repetían los ecos de la bóveda. Acercándome a aquella reja, pude
fácilmente distinguir tras ella varios bultos blancos y negros, entre
los cuales algunos desfilaron pausadamente y sin ruido hacia una puerta
que se abría en el ángulo del fondo, y otros permanecían inmóviles y de
rodillas. Eran las monjas.

Contemplando la tranquilidad de aquellas santas mujeres, su apacible
recogimiento, la aparente vaguedad de sus formas corpóreas, aquel
silencio de sus pasos que las asemejaba a simples creaciones de la luz,
discurriendo por el fondo de la cámara oscura; contemplando aquella
calma de sus rezos que nadie oía, sentí envidia de los que sumergen su
vida en la dulce sombra de un claustro. Yo no apartaba mis ojos del
coro, observando indiscretamente los movimientos de las buenas madres, y
mientras mayor era mi atención, con más claridad se me iban presentando
los distintos objetos de aquel recinto, y vi poco a poco los sillones,
el facistol, el órgano, los cuadros. Tan lentamente salían de la
oscuridad los perfiles de estos objetos, que mi propia imaginación podía
creerse autora de aquel espectáculo.

El día iba descendiendo, y la iglesia se oscurecía por grados; pero una
de las madres, tirando de unas cuerdas, descorrió la cortina negra de la
alta ventana del coro, y entonces entró la luz crepuscular, dando a todo
su verdadera forma. Retiráronse algunas monjas: yo sentí el tenue chocar
de las medallas de sus rosarios cuando levantaban la rodilla, y luego
algunos besos. Era fácil contar el número de las que salían por el
número de los suaves estallidos que resonaban en aquel espacio, porque
todas al salir besaban los pies de un Cristo colgado junto a la puerta.
Yo atendía a esto cuando de las figuras que aún quedaban de rodillas en
el centro del coro, se levantó una dirigiéndose a la reja y al mismo
lugar en que yo estaba. Mi impresión al verla, al ver su cara, al ver
sus ojos que me miraban, fue tan viva, tan aterradora que hube de quedar
petrificado, me quedé con la sangre helada, la vida en suspenso, hecho
una estatua de plomo. Lo que estaba viendo, ¿qué era? ¿Era una
aberración, un delirio, una imagen del sueño, un juguete fantástico,
obra de los ángeles traviesos para burlarse de los que con sus mundanas
tristezas van a profanar la casa de Dios? La miré fijamente, atónito
ante aquel enigma, ante aquel misterio; pero la visión no duró más que
algunos segundos, porque la monja, llamada por otra, se apartó de la
reja, y salió rápidamente del coro sin besar el pie del Santo Cristo.

Al hallarme solo reuní todos, absolutamente todos los rayos de mi razón,
y juntándolos los dirigí a la confusa y negra oscuridad de aquel
fenómeno. Quise desvanecer el celaje que envolvía mi inteligencia
haciéndome estúpido, y me pregunté si lo que acababa de presenciar era
reproducción de aquella burla de mis sentidos que poco antes me había
hecho ver una mano en un pedazo de papel y oír mi nombre en el chirrido
de una puerta. Me di golpes en la cabeza, busqué un sitio más solitario,
donde, serenándome, pudiera poner en claro cuestión tan ardua, y sin
saber cómo, di conmigo en el fondo de una capilla. En un cuadro que se
ofreció de improviso a mis ojos vi una falange de ángeles, mil
encantadoras criaturas de esas que sin más naturaleza corporal que una
cabeza y dos alas, han creado los artistas para regocijar los lienzos de
la pintura ascética. Atrajeron mi atención aquellos seres juguetones y
enredadores: todos se reían con infantiles carcajadas y entremezclándose
volaban, rasgando nubes, esparciendo flores con el batir de sus alas de
pollo y dándose de coscorrones al chocar unas con otras las rubias
cabecitas. Por momentos me parecía que avanzaba sobre mí aquella bandada
de rostros voladores, y luego retrocedían haciendo con alegre algazara
movimientos de miedo, para esconderse después tras una nube, y hacerme
desde allí guiños con sus ojuelos, y encantadoras muecas con sus bocas.

A tal situación habían llegado mis sentidos cuando el sacristán,
agitando un grueso manojo de llaves con cencerril estruendo, me hizo
salir de la iglesia, pues yo era la única persona que quedaba en ella.
Salí, y la luz de la calle pareció devolverme el sentido común, que,
según mi propia opinión, había perdido. El tumulto de que poco antes
hablé, continuaba más reciamente, y algunas personas atravesaron
corriendo la plazuela. Entre estas vi un hombre, un caballero que corría
azorado y con miedo, volviendo la vista atrás, deteniéndose a cada dos
pasos, y vacilando luego sobre qué dirección tomaría. Fijose en mí, y al
punto, llamándome por mi nombre, se me acercó con muestras de alegría
por haberme encontrado. Era el diplomático.

\hypertarget{xiv}{%
\chapter{XIV}\label{xiv}}

---Gabriel---me dijo con voz temblorosa y sin dejar de mirar hacia el
sitio del tumulto,---vas a hacerme un favor\ldots{} ¡Los franceses!
¡Están ahí los franceses! Sí\ldots{} yo he visto pasar por esa calle las
gorras de pelo de a dos varas de alto\ldots{} Bien lo decía yo\ldots{}
Mi sobrinita y mi hermana tienen unas cosas\ldots{} a ellas solas se les
ocurre mandarme con esta comisión, sin reparar que la pierna gotosa no
me deja correr. Pero no doy un paso más\ldots{} me retiro a casa\ldots{}
tú te encargarás de llevar las flores, la carta y el recado\ldots{} ¿No
oíste un tiro? Me parece que vienen por ese lado. ¡Jesús, esto es atroz!
Si viene una bala perdida\ldots{} Adiós, me voy; toma, chiquillo:
encárgate tú de esto. Es muy fácil. Ahí está el convento. Mira, en aquel
callejón está la puerta del torno. Entras, preguntas por la señorita
Inés, la novicia\ldots{} pues. Dices que vas de parte de la señora
marquesa de Leiva. ¿Lo olvidarás?\ldots{} ¡Dios mío! ¡Esas mujeres que
pasan corriendo! Sin duda los muy tunantes intentan deshonrarlas. Me
voy\ldots{} Toma: entra tú en el locutorio. ¡Para qué vendría yo a estos
malditos barrios! Toma el ramo de flores contrahechas\ldots{} toma la
carta, que darás a la señorita Inés\ldots{} le dices que la señora
marquesa está enojada con ella, y que es preciso que se decida a salir
del convento\ldots{} insiste mucho en esto, ¿eh?, dile que nos vamos
para Madrid, y que en la corte del nuevo rey José I\ldots{} ¡Demonio,
eso que ha sonado es un tiro de obús!\ldots{} Me parece que ahora cayó
una granada en el techo de esa casa.

---¿Una granada? Lo menos cincuenta van disparadas ya---dije yo,
atizando el fuego de su miedo para que se marchara pronto y me dejase
tan sublime comisión.

---Conque, chiquillo---continuó, temblando como un azogado,---¿lo harás
bien? Si te dan contestación la llevas a casa. Ve pronto. Yo me escaparé
corriendo por esta calle donde no se siente ruido\ldots{} adiós.

Desapareció el diplomático, llevado por su miedo, y al punto entré en la
portería del convento con febril alegría, y di fuertes porrazos en el
torno. Una voz regañona me contestó:

---\emph{Deo gratias}---dije.---Vengo de parte de mi ama la señora
marquesa de Leiva a traer un recado a la señorita Inés.

La portera me dijo que esperara en el locutorio, y al poco rato de estar
allí corriose la cortina de éste y vi dos monjas. No sé cómo me pude
mantener en pie. Una de ellas era Inés.

No me cabía duda alguna, era ella misma: en su semblante, adelgazado y
pálido, habían impreso terribles huellas los sesenta días de incesantes
pesares transcurridos desde el 2 de Mayo; pero la reconocí, a pesar de
la escasísima luz del locutorio, y la hubiera reconocido en la oscuridad
de las entrañas de la tierra. Pareciome que al verme cerró los ojos, y
que asió las rejas con sus dos manos para sostenerse. Cuando me dirigió
la primera pregunta su voz temblaba de tal modo, que era imposible
entender sus palabras. Sin poder decir una sola, incapaz de discurso y
de movimiento, permanecí yo breve rato con la cara apoyada en la reja.

La monja que la acompañaba me obligó por fin a hablar.

---La señora marquesa me ha dado este ramo de flores y esta carta---dije
introduciendo ambas cosas para que las tomara Inés.

---¡Ah, el ramo para el Santo Niño de la Enfermería!---dijo la monja
vieja.---La señora condesa no se olvida de nosotras.

---También me ha dado un recado de palabra para la señorita
Inés---continué,---y es que se prepare a salir del convento para partir
con ella a Madrid dentro de algunos días.

---¡Oh!---exclamó la vieja.---La señora condesa y la señora marquesa
hacen mal en contrariar la decidida vocación de esta niña. ¡Por qué ese
empeño de llevarla al siglo, cuando ella quiere dejar sus maldades y
abominaciones! La pobrecita no quiere cuentas con nadie más que con su
prometido esposo, que es nuestro Señor Jesucristo.

---Madre Transverberación---dijo Inés con voz más entera,---el chocolate
y los bollos que han hecho sus mercedes ayer para la señora condesa,
¿dónde están? ¿Los ha traído su merced?

---No por cierto.

---¡Si tuviera su merced la bondad de ir a buscarlos para que los lleve
este mozo!

---Bien pudo Vd. haberlos traído---dijo gruñendo la vieja.

---Si la señora condesa no lo recibe esta tarde, se enojará mucho, y me
será difícil convencerla de que no quiero dejar nunca más esta santa
morada.

---Voy por él\ldots{} ¡Qué niñas éstas!

Dejonos solos la madre Transverberación, y entonces hablé así:

---Inés mía, estoy vivo, he resucitado. Salí vivo de aquel montón de
victimas, donde perdimos para siempre a nuestro buen amigo D. Celestino.
Al verme vivo y sin ti, pensé que Dios me había devuelto la vida para
castigarme; pero ahora que te encuentro, alabo a Dios porque veo que no
una, sino dos veces me ha devuelto la vida.

---¿Debo salir de aquí? ¿Debo hacer lo que me mandan esas señoras?---me
preguntó Inés con impaciencia, porque temía la vuelta de la madre
Transverberación.

---Sí, Inés, sal de aquí. Haz lo que te mandan esas señoras. ¿Qué dicen
en esa carta?

---Toma, léela---dijo, alargándola al través de la reja.

A la escasa luz del locutorio pude leer la carta, que decía, entre otras
cosas relativas al ramo y al chocolate, lo siguiente: «Esperamos que
cesará tu obstinación en profesar. Nos oponemos resueltamente a ello, y
no queremos que tu ingreso en el seno de esta familia sea señal de
aniquilamiento de nuestra casa. Ya te dijimos que habíamos determinado
casarte con un joven de alto linaje, proyecto en el cual estriba la
felicidad y grandeza y lustre de la familia a que perteneces. Todo está
concertado, y aunque se aplace por motivo de la guerra, al fin tiene que
ser; de modo que si persistes en profesar, nos llenarás de dolor. ¿No
anhelas servirnos de consuelo en nuestra soledad? ¿No correspondes al
mucho amor que te profesamos? ¿No deseas ocupar el puesto que te
pertenece en nuestro corazón y en nuestra casa? Mi sobrina y yo iremos a
convencerte, y en tanto disponemos el viaje a Madrid, adonde nos
acompañarás, porque tu presencia es indispensable a las diligencias de
tu legitimación».

---Sí, saldré---dijo Inés cuando acabé de leer la carta.---Ya no quiero
estar más aquí.

---¿Pues qué, estabas decidida a profesar?

---Sí, muy decidida. Nada me consolaba sino la idea de encerrarme aquí
para siempre. Cuando me trajeron a Córdoba\ldots{} ¡qué días y qué
viaje!, yo no sabía lo que era de mí. Me encerraron en este
convento\ldots{} luego vinieron esas señoras a decirme que era su
sobrina\ldots{} me besaron\ldots{} lloraron mucho las dos\ldots{} luego
dijeron que me iban a casar, y cuando les contesté: «Pues ya que me han
puesto aquí, aquí me he de quedar toda la vida», ambas se afligieron
mucho\ldots{} Me visitan con frecuencia, acompañadas de un señor de edad
que me hace mil caricias, y asegura quererme mucho; pero siempre me he
negado a ceder a sus ruegos para salir.

---¿Y ahora?

---Las paredes del convento se me caen encima, y anhelo salir.

---¡Pero te van a casar!---exclamé indignado.---Te quieren casar y no se
hunde el mundo.

Entonces se rió, creo que por primera vez después de mucho tiempo, y
aquella espontánea alegría me pareció expresión de una renaciente vida.
Inés salía del seno del claustro como yo del montón de muertos de la
Moncloa, y al contestar con una sonrisa a mis amorosas quejas, sacaba
del sepulcro de la Orden el pie que tan impremeditadamente había metido
dentro. Viéndola reír, reíme yo también, y al punto olvidando la
situación, nos hablamos con la confianza de aquellos tiempos en que de
nuestras penas hacíamos una sola.

---¡Ay, chiquilla! Ahora que eres archiduquesa y archipámpana, ¿no
tienes vergüenza de quererme?

---¿Pero qué quieren hacer de mí?---dijo Inés poniéndose triste otra
vez.

---Mira, princesa; haz lo que te mandan esas señoras: obedécelas en
todo. Ya habrás conocido el parentesco que tienes con ellas. Dios te ha
puesto en sus manos: acepta lo que Dios te da, y El arreglará lo demás.

---Saldré del convento---afirmó ella.---¡Ay! Las madres se van a asustar
cuando me lo oigan decir. Pero ya Dios no quiere que yo sea monja.

---No lo serás, no; y cuando yo vuelva de la guerra\ldots{}

---¿Pero vas tú a la guerra? Chiquillo, ¿quién te ha metido en guerras?

---¿Pues qué he de hacer? ¿Quieres que toda la vida sea criado? Escucha,
Inés, lo que me pasó hace días en casa de la señora condesa. Fui a
visitarla, y habiendo cometido la indiscreción de decirle que te amaba,
se enfureció de tal modo que me hizo poner en la puerta de la calle.

Inés cruzó las manos, dejándolas caer luego con desaliento sobre su
falda, mientras elevaba sus ojos al cielo, sin decir nada.

---No soy más que un criado, Inés---exclamé agarrándome con fuerza a la
reja y sacudiéndola, como si quisiera hacerla pedazos;---no soy más que
un miserable chico de las calles, indigno de ser mirado por personas de
tu clase. Después que nos separamos, mira qué distantes estamos uno de
otro. Pero no creas que lo siento; me gusta verte donde debes estar.

---¿Y tú?---me preguntó con perplejidad.

---Yo haré lo que deba, Inesilla. Sal de este convento, ve con esas
señoras, y espérame tranquila, con la seguridad de que iré a buscarte.
Si para entonces no has variado\ldots{} si te encuentro la misma\ldots{}

Inés me contestó al instante pasando su dedo índice por uno de los
huecos de la reja. Yo se lo besé, se lo mordí tan sin pensarlo, que ella
no pudo contener un pequeño grito, a punto que la madre Transverberación
regresaba con el chocolate y los bollos.

---¿Qué es eso, niña?---exclamó la vieja asombrada de oírla chillar.

---Nada, madre Transverberación. Esta reja tiene unos picos\ldots{} Al
mover la mano me lastimé un dedo---repuso Inés chupándose la coyuntura
del dedo índice y sacudiéndolo después para aparentar el dolor del
supuesto rasguño.

---Aquí están el chocolate y los bollos---añadió la monja.---Vaya, ya es
tiempo de que se marche ese mocito, porque oscurece y no es ésta hora de
tener abierto el locutorio.

---Rabiando estoy por marcharme---dije.---Vengan acá esos bollos y ese
chocolate, que la señora marquesa ha de estar con el alma en un hilo,
aguardando tan buenas cosas. ¿Y qué le digo a su merced en contestación
al recado que tuve el honor de traer?

---Que está muy bien---contestó Inés apretando su cara contra la
reja.---Que haré lo que me mandan, y que cuando quieran venir por mí,
estoy dispuesta a salir del convento.

---¿Cómo es eso, niña?---dijo alarmada la monja.---¡Que quiere Vd.
salir! ¡Qué pensará su futuro esposo Jesucristo si llega a sus oídos lo
que Vd. ha dicho! Y tiene que saberlo forzosamente, porque Él está en
todas partes y todo lo oye. Nada, nada---añadió arrimando su hocico a la
verja.---Rapaz, a la señora marquesa dirá Vd. que la niña persiste en su
ejemplar vocación, y que si quieren verla enfadada y bufando de rabia,
que le hablen del siglo y sus tentaciones.

Inés prorrumpió en una carcajada tan natural, tan graciosa, tan fresca,
tan jovial, que hasta las paredes del convento parecían regocijarse con
tan alegre música.

---¿Qué risas tan mundanas son esas?---dijo la madre
Transverberación.---Es la primera vez que se ríe Vd. de ese modo en esta
casa. ¿Qué pasa para tanta alegría?\ldots{} Adentro, niña, adentro y
daremos parte de este inaudito desenfado a la madre abadesa.

Cerrose el locutorio y salí a la calle. Sentíame con nueva vida, con
centuplicadas fuerzas en mi espíritu y en mi cuerpo; sentíame capaz de
todo, de la abnegación, de la lucha, hasta del heroísmo, porque la
presencia y las palabras de Inés habían abierto desconocidos horizontes,
inmensos espacios delante de mí.

\hypertarget{xv}{%
\chapter{XV}\label{xv}}

Antes de llegar a la posada, fuerte ruido de tambores y cornetas me
anunció la salida del ejército. Corrí a buscar mis armas y mi caballo, y
antes de que se notara mi falta, ya estaba en fila con el señorito conde
de Rumblar, Marijuán y los demás de la partida. Era ya de noche cuando
salimos, y el pueblo todo tomó parte en aquella espontánea fiesta de
nuestra despedida: millares de luces se encendieron a nuestro paso en
balcones y puertas; ninguna mujer dejó de saludarnos desde la reja, ya
sin galán, y todos los chicos engendrados por aquella fecunda
generación, salieron delante de los tambores acompañándonos hasta más
allá de la Puerta Nueva.

Anduvimos toda la noche, y al día siguiente, al salir del Carpio, nos
desviamos del camino real de Andalucía tomando a la derecha en dirección
a Bujalance. Durante esta primera jornada encontramos a Santorcaz, que
había salido de Bailén para incorporarse a su cuadrilla, y a todos nos
dio mucho gusto el verle.

---Aquí traigo varios regalitos que le manda a usted su señora
mamá---dijo a mi amo, entregándole unos paquetes.---La señora estaba
desazonada por no haber tenido noticias de Vd., y me encargó que le
cuidase bien. ¿Hizo el señor conde las visitas que doña María le
encargó?

---Puntualmente---contestó mi amo.---Y Vd., ¿por qué no ha venido antes?

---¡Qué demonio!---exclamó Santorcaz.---Con estas cosas ni tenemos
posta, ni quien lleve una carta. Sin embargo, yo recibí las que
esperaba, y aquí estoy al fin, deseando, como los demás, que tropecemos
con los franceses.

Desde entonces fue Santorcaz el principal personaje de la cuadrilla
después del amo, lugar que supo conquistarse con su desenvoltura y la
amenidad subyugadora de su conversación. Él ponía todo su esmero en
agradar a D. Diego, cosa fácil de conseguir; y siempre fijo al lado de
este, cautivó prontamente el ánimo del buen chico, ya contándole hazañas
y extraordinarios hechos, ya sugiriéndole con su fértil imaginación
ideas y conceptos propios para enloquecer a un joven de chispa, pero muy
atrasado en su desarrollo intelectual.

Y a todas estas, señores míos, ni una palabra os he dicho de aquel
ejército, ni de su extraña composición; pero atended ahora, que lejos de
ser tarde, es esta la ocasión propicia de hacerlo, según el refrán que
dice: «cada cosa en su tiempo y los nabos en Adviento.»

La base del ejército de Andalucía estaba en las tropas del campo de San
Roque mandadas por Castaños, y en las que después trajo D. Teodoro
Reding de Granada. Componíase de lo más selecto de nuestra infantería de
línea, con algunos caballos y muy buena artillería, no excediendo su
número de trece a catorce mil hombres. Agregáronse a aquellas fuerzas
algunos regimientos provinciales y los paisanos que espontáneamente o
por disposición de las Juntas, se engancharon en las principales
ciudades de Andalucía. Difícil es conocer la cifra exacta a que se
elevaron las fuerzas de paisanos armados; pero seguramente eran muchos,
porque la convocatoria había llamado a todos los mozos de diez y seis a
cuarenta y cinco años, solteros, casados y viudos sin hijos, de cinco
pies menos una pulgada, medidos descalzos. Además de los notoriamente
inútiles, como cojos, mancos, ciegos, etc., se exceptuaba a los que
tenían su mujer embarazada o ejercían cargos públicos, así como a los
ordenados de Epístola; pero no había excepción por razón de cosecha o
labores del campo. Los únicos rechazados de las filas, sin tener
aquellos reparos, eran los \emph{negros, mulatos, carniceros, verdugos y
pregoneros}. Con paisanos, pues, creó Sevilla cinco batallones y dos
regimientos de caballería; Cádiz mandó el batallón de tiradores que
llevaba su nombre, y las ciudades y villas de Utrera, Jerez, Osuna,
Carmona, Jaén, Montoro y Cabra, enviaron cuerpos de infantería y
caballería de número irregular.

Esto aumentó el ejército; pero aún debía crecer un poco más aquel que
empezó enano y debía ser gigante terrible, si no por su tamaño, por su
fuerza. Los militares españoles que el Gobierno de Madrid incorporaba a
las divisiones de Moncey, de Vedel o de Lefebvre iban huyendo de sus
traidoras filas en cuanto se les presentaba ocasión para ello, de tal
modo que al verificar sus marchas aquellos ejércitos por parajes
montuosos y accidentados, veían que los españoles se les escapaban por
entre los dedos, como suele decirse. Los desertores acudían a engrosar
las tropas del ejército de Blake, del de Cuesta o del de Castaños; y a
Carmona y a Córdoba llegaron muchos, escapados de las filas de Moncey,
así como casi todos los que hacían la campaña de Portugal con Junot.
Aquellos oficiales y soldados al romper la disciplina literal que los
sujetaba a la Francia invasora para acudir al llamamiento de la
disciplina moral de su patria oprimida, hacían el viaje disfrazados,
traspasaban a pie las altas montañas y los ardientes llanos, hasta
encontrar un núcleo de fuerza española. Daba lástima verles llegar
rotos, descalzos y hambrientos, aunque su gozo por hallarse al fin en
tierra no invadida les hacía olvidar todas las penas. Con estos
desertores, entre quienes había guardias de corps, walones, ingenieros,
y artilleros, aumentó un poco nuestro ejército.

Pero aún creció algo más. La Junta de Sevilla había indultado el 15 de
Mayo a todos los contrabandistas y a los penados que no lo fueran por
los delitos de homicidio, alevosía o lesa majestad divina o humana, y
esto trajo una legión, que si no era la mejor gente del mundo por sus
costumbres, en cambio no temía combatir, y fuertemente disciplinada, dio
al ejército excelentes soldados. Ibros, lugar célebre en los fastos del
contrabando; Jandulilla, Campillo de Arenas, y otras localidades,
entregadas más tarde al sable de la guardia civil y de los carabineros,
enviaron respetables escuadrones, con la particularidad de que por venir
armados hasta los dientes, y ser todos unos caballeros de muy buen
temple, que sabían dónde echaban la boca del trabuco, se les reputó como
auxiliares muy eficaces del ejército. Cuerpos reglamentados españoles,
con algunos suizos y walones; regimientos de línea que eran la flor de
la tropa española; regimientos provinciales que ignoraban la guerra,
pero que se disponían a aprenderla; honrados paisanos que en su mayor
parte eran muy duchos en el arte de la caza, y por lo general tiraban
admirablemente; y por último, contrabandistas, granujas, vagabundos de
la sierra, chulillos de Córdoba, holgazanes convertidos en guerreros al
calor de aquel fuego patriótico que inflamaba el país; perdidos y
merodeadores, que ponían al servicio de la causa nacional sus malas
artes; lo bueno y lo malo, lo noble y lo innoble que el país tenía,
desde su general más hábil hasta el último pelaire del Potro de Córdoba,
paisano y colega de los que mantearon a Sancho, tales eran los elementos
del ejército andaluz.

Se formó de lo que existía; entraron a componer aquel gran amasijo la
flor y la escoria de la Nación; nada quedó escondido, porque aquella
fermentación lo sacó todo a la superficie, y el cráter de nuestra
venganza esputaba lo mismo el puro fuego, que las pestilentes lavas.
Removido el seno de la patria, echó fuera cuanto habían engendrado en él
los gloriosos y los degenerados siglos; y no alcanzando a defenderse con
un solo brazo, trabajó con el derecho y el izquierdo, blandiendo con
aquel la espada histórica y con este la navaja.

En cuanto a uniformes y trajes, los había de todas las formas conocidas.
Es prodigioso cómo se equipó aquel ejército de paisanos en diez y seis
días. La administración actual, con todos sus recursos, es un sastre de
portal comparada con aquel confeccionador que puso en movimiento
millones de agujas en dos semanas. En cierto estado que la historia no
ha creído digno de sus páginas, pero que existe aún, aunque en el
olvido, se consigna el número de piezas de vestuario que hicieron
gratuitamente las monjas y señoras de Sevilla. Dice así: «Por las
comunidades y señoras de distinción se han hecho 3.335 camisas, 1.768
pantalones y 167 casacas de soldado: 1.001 camisas, 312 pantalones y 700
chalecos de sargento: 374 botines de paño, 149 sacos de caballería, 16
mochilas y 1.684 escarapelas». Las señoras de Alcolea, las de Carmona,
Lora del Río y otros pueblos figuran en la cuenta con cifras parecidas.

Esta diversidad de manos en la hechura del vestuario indica que la voz
\emph{uniforme}, en lo tocante a voluntarios, era una palabra. Al lado
de las casacas blancas con solapa negra, carmesí o azul que vestían la
mayor parte de los regimientos de línea; al lado de las levitas azules
con bandolera que vestían los walones y los suizos, veíamos los
chaquetones de paño pardo con que se cubría la gente colecticia. Entre
los altos morriones de la artillería y las gorras de los granaderos,
llamaban la atención nuestros blancos sombreros portugueses y las gorras
de cuartel y los tocados de innumerables clases con que cubrían sus
chollas los tiradores y voluntarios de los pueblos. Como antes he dicho,
aquel ejército hacía reír.

¿Y el dinero para la guerra? Causa risa ver cómo se da hoy de
calabazadas un ministro de Hacienda para \emph{arbitrar} con destino a
otra guerra unos cuantos millones que nadie quiere darle si no hipoteca
hasta el último pingajo de la Nación. Aprended, generaciones egoístas.
Leed las listas de donativos hechos por los gremios, por los
comerciantes, por los nobles y hasta por los mendigos. ¡Aquel sí era
llover de dinero, y reunirlo a montones, sin que ni un realito de vellón
se escapase por entre los agujeros del cesto administrativo! En la lista
de donaciones hay una partida conmovedora que dice así: «La señora
condesa viuda de Montelirios ha entregado su \emph{toaleta} de plata,
manifestando el sentimiento de que sus medios no alcancen tanto como su
voluntad».

¿Habrá hoy quien dé su \emph{toaleta}?\ldots{}

\hypertarget{xvi}{%
\chapter{XVI}\label{xvi}}

Nuestra marcha por Cañete de las Torres en dirección al río Salado era
un verdadero paseo triunfal, mejor dicho, casi no parecía que
marchábamos, porque la gente de los pueblos, incluso mujeres, ancianos y
chicos, nos seguían a un lado y otro del camino, improvisando fiestas y
bailes en todas las paradas. Cuando el ejército se detenía, se
eclipsaban en apariencia todos los males de la patria, porque la tropa,
recobrando el buen humor, convertía el campamento en una especie de
feria. Yo no sé de dónde salían tantas guitarras; no pude comprender de
qué estaban hechos aquellos cuerpos tan incansables en el baile como en
el ejercicio, ni de qué metal durísimo eran las gargantas, para ser tan
constantes en el gritar y cantar.

Durante la primera semana del mes de Julio no nos faltaron víveres
abundantes, así es que lo pasábamos perfectamente; y como tampoco
tropezamos con los franceses, que estaban establecidos, aunque muy
inquietos, al otro lado del río, a todos, especialmente a los
inexpertos, nos parecía la guerra una ocupación dulcísima. Sobre todo el
condesito de Rumblar no cabía en su pellejo de puro alborozado; y como
con el roce de tanta y tan diversa gente se iba despabilando por
extremo, llegó a adquirir con la nueva vida un desembarazo, un dominio
de su propia persona que antes no tenía. Santorcaz, como dije, había
logrado en poco tiempo gran ascendiente sobre D. Diego, de tal modo que
cuanto nuestro mozalbete ponía por obra, lo consultaba con aquel.
Marijuán en cambio hacía buenas migas con un servidor de Vds., y siempre
juntos en las marchas y en los descansos, nos contábamos nuestras cosas,
compadeciéndonos y consolándonos mutuamente. Nosotros dos solos y sin
dar parte a nadie nos comimos el divino chocolate y los bollos de la
madre Transverberación.

Todo el ejército tenía gran impaciencia por venir a las manos con la
\emph{canalla}. Como existen en todo campamento, además del supremo
consejo que se celebra en la tienda del general, tantos consejillos como
grupos de soldados se escalonan aquí y allí en la cantina o en el campo
raso, para echar una caña o tirar un par de cartas, nosotros estábamos
dilucidando siempre en pequeños cónclaves la eterna cuestión de nuestro
encuentro con los franceses. ¡Cuántas veces reunidos junto a un tambor
donde había un jarro de vino, dispusimos el paso del río, el ataque del
enemigo en su posición de Andújar, u otra hazaña de la misma harina!

Un día hallándonos en Porcuna, y después que se nos unió el ejército de
Reding, resolvimos, después de ardiente discusión, que nuestros
generales estaban atolondrados, y sin saber qué plan adoptarían. El
conde de Rumblar dijo que iba a escribir a su maestro D. Paco, para que
le dijera lo que más convenía hacer; pero como todos se rieron de esta
ocurrencia, nuestro generalito se amoscó y fue a que le consolara con
sus adulaciones interminables el lugarteniente Santorcaz.

Por último, tras largo consejo celebrado por los generales, se dijo que
iban a ser distribuidas las divisiones para tomar la ofensiva
inmediatamente. Aquel día, que fue si no recuerdo mal el 12 o el 13 de
Julio, vi por primera vez al general Castaños, cuando nos pasó revista.
Parecía tener cincuenta años, y por cierto que me causó sorpresa su
rostro, pues yo me lo figuraba con semblante fiero y ceñudo, según a mi
entender debía tenerlo todo general en jefe puesto al frente de tan
valientes tropas. Muy al contrario, la cara del general Castaños no
causaba espanto a nadie, aunque sí respeto, pues los chascarrillos y las
ingeniosas ocurrencias que le eran propias las guardaba para las
intimidades de su tienda. Montaba airosamente a caballo, y en sus
modales y apostura había aquella gracia cortés y urbana, que tan común
ha sido en nuestros Césares y Pompeyos. Es preciso confesar que a
caballo y en las paradas hemos tenido grandes figuras. Esto no es decir
que Castaños fuera simplemente un general de parada, pues en 1808, y
antes de inmortalizar su nombre tenía muy buenos antecedentes militares,
aunque había hecho su carrera con rapidez grande, si no desusada en
aquellos tiempos. A los doce años de edad obtuvo el mando de una
compañía; a los veintiocho le hicieron teniente coronel y a los treinta
y tres coronel. Si en su juventud no asistió a ninguna campaña, en 1794,
y cuando tenía treinta y ocho años y la faja de mariscal de campo,
estuvo en la del Rosellón a las órdenes del general Caro, y allí le
hirieron gravemente en el lado izquierdo del cuello. Cuentan que la
ligera inclinación de su cabeza hacia aquel lado provenía de la tal
herida.

Voy a decir de qué manera nos distribuyeron. La primera división la
mandaba Reding, la segunda Coupigny y la tercera Jones: la reserva
estaba a las órdenes de D. Juan de la Peña, y mandaban destacamentos
sueltos compuestos poco más o menos de mil hombres, y en calidad de
tropas volantes para mortificar al enemigo, D. Juan de la Cruz, el
marqués de Valdecañas y D. Pedro Echévarri, que después fue uno de los
más famosos polizontes de la reacción. Trescientos escopeteros que
habían salido Dios sabe de dónde, eran capitaneados por el presbítero D.
Ramón de Argote. ¿No es verdad que hubiera estado mejor diciendo misa?

A caballo éramos tres mil, fuerza no muy grande si se considera que
íbamos a operar en país entrellano y contra jinetes muy aguerridos; pero
en cambio nuestra artillería era de primer orden. Teníamos veinticuatro
piezas, servidas por el Real Cuerpo, con lo más florido de aquella
oficialidad a quien estaba reservada la mayor gloria de la guerra, desde
el 2 de Mayo hasta la batalla de Vitoria.

Nosotros nos extendíamos por la izquierda del Guadalquivir, ocupando los
pueblos de Porcuna y Lopera; y alargando una de nuestras alas por el
camino de Arjonilla, observábamos la orilla derecha, mientras la otra
ala se extendía hacia Higuera de Arjona buscando a Mengíbar. El francés
ocupaba a Andújar con las fuerzas que primitivamente trajo a Andalucía,
y que habían vencido en el puente de Alcolea y saqueado a Córdoba. La
división de Vedel, fuerte de diez mil hombres, ocupaba a Bailén, y la
pequeña división de Ligier-Belair, el mismo general a quien vimos
batirse con los vecinos de Valdepeñas en los primeros días de Junio,
estaba en Mengíbar guardando el paso del río por aquella parte. Andújar,
Bailén, Mengíbar. Del primero al segundo punto corría la carretera
general de Andalucía, desde Bailén a Mengíbar el camino que iba a Jaén,
y desde Mengíbar a Andújar el río. Conserven Vds. en la memoria la
disposición de este triángulo para comprender la importancia de los
movimientos de ambos ejércitos.

Cualquiera que fuese el pensamiento de nuestros generales, lo cierto es
que la primera división recibió orden inmediata de ponerse en marcha,
mientras Castaños con la tercera y la reserva se dirigía hacia el puente
de Marmolejo para pasarlo y atacar a Dupont en Andújar. Ya he dicho que
mandaba D. Teodoro Reding la primera división: lo que aún no ha sido
escrito por la historia ni dicho por mí, es que yo formaba parte de
ella, porque toda la caballería voluntaria había sido incorporada, mejor
dicho, fundida en los batallones del ejército, que apenas contaban con
la mitad del contingente. A mi amo y a los que le seguían nos tocó
formar en las filas del regimiento de Farnesio, mientras que los
lanceros de Sevilla fueron casi todos incorporados al regimiento de
España.

El día 13 nos separamos de nuestros compañeros y tomamos el camino,
mejor dicho, las veredas y trochas que conducían a Mengíbar. No
llegábamos a seis mil; pero éramos buena gente aunque me esté mal el
decirlo. El regimiento de guardias walones, los suizos, el de la Corona,
el de Irlanda, el de Jaén, los granaderos provinciales, los fusileros de
Carmona, la caballería de Farnesio y las seis bocas de fuego que mandaba
D. Antonio de la Cruz, eran piezas respetables, orgullosas de sí mismas.
Teníamos por general a un hombre impetuoso, de más arrojo que prudencia,
mediano táctico; pero incansable en las marchas. Nuestro jefe de Estado
Mayor, D. Francisco Javier Abadía, era un militar muy entendido, quizás
de los mejores que entonces tenía el ejército español, y el coronel
puesto al frente de la artillería pasaba por un oficial de mucho
entendimiento en su arma. Nosotros le llamábamos el sainetero por ser
hijo de D. Ramón de la Cruz.

Adelante, pues. Al llegar a Mengíbar, encontramos la población muy
alborotada, porque un destacamento francés enviado a Jaén en busca de
víveres, después de saquear horriblemente esta ciudad, había retrocedido
a su cuartel general asolando a su paso la comarca. De Jaén se contaban
atrocidades que apenas son creíbles en militares de un país europeo.
Dijéronnos que mujeres y niños habían sido inhumanamente degollados y
que igual muerte padecieron dentro de sus mismos hospitales varios
frailes agustinos y dominicos enfermos. La consternación de aquellos
pueblos era excesiva, y al aproximarse las tropas acudían en tropel a
nuestro encuentro, derramando lágrimas de ira, suplicándonos que no
dejáramos vivo un francés, y pidiendo los viejos aún fuertes y los
rapaces de doce años que se les dejase marchar entre las filas para
ayudarnos. Según nos decían, después del saqueo, en los caseríos
inmediatos al tránsito, Almenara, Fuente del Rey, Grañena y otros no
habían dejado ni un grano de trigo, ni un azumbre de vino, ni un puñado
de paja. Hasta las medicinas de las boticas y de los hospitales de Jaén
fueron robadas, y al propio tiempo ni un carro ni una mula quedaron en
todos aquellos contornos.

Muchas familias expoliadas habían acudido a Mengíbar. En la plaza del
pueblo dos frailes escapados a las carnicerías de Jaén, predicaban el
exterminio de los franceses. Al ver la indignación de aquella infeliz
gente robada y vejada, al ver las mujeres que acudían frenéticas y
rabiosas pidiéndonos que vengáramos a sus inocentes hijos degollados sin
piedad en la cuna, comprendí las crueldades de que por su parte
empezaban a ser víctimas los franceses, cuando se rezagaban.

\hypertarget{xvii}{%
\chapter{XVII}\label{xvii}}

Antes de decidirse a pasar el río, nuestro general mandó una pequeña
fuerza en reconocimiento de la situación de las tropas de Coupigny.
Algunos jinetes de Farnesio tomaron parte en esta expedición, y Marijuán
que fue en ella, nos contó a su regreso en la tarde del 15, que habían
encontrado la división del marqués hacia Villanueva de la Reina, donde
le entregaron los pliegos de Reding. Desde el campamento de Coupigny se
había visto una gran polvareda en la orilla derecha, y parecía que la
división de Vedel marchaba desde Bailén a Andújar, para reforzar a
Dupont, que ya había trabado la lucha con Castaños. La gente venida de
Arjonilla aseguraba haber oído fuerte cañoneo hacia la parte de los
Visos.

---A estas horas---decía Marijuán,---o ellos o los de Castaños han de
estar derrotados.

---¿Y qué esperaba el marqués en Villanueva de la Reina?---preguntó
Santorcaz con aquella suficiencia estratégica que le hiciera tan digno
de admiración a los ojos del joven D. Diego.

---Allí se estaba tan quieto---repuso Marijuán.---Parece que está de
acuerdo con nuestro general para operar en combinación y atacar juntos a
Bailén.

---¿Pero qué estrategia es esa, ni a qué conduce atacar a Bailén?---dijo
Santorcaz, atrayendo en su alrededor un círculo de soldados.---¿No dices
que la división Vedel salió de Bailén y está ya sobre Andújar?

---Sí: así lo decían en Villanueva.

---Pues si no hay enemigos en Bailén, ¿qué es eso de atacar a Bailén? Se
tratará de ocuparlo para luego avanzar por el arrecife y embestir a
Dupont y a Vedel por la espalda, mientras Castaños, Jones y Peña lo
atacan de frente.

---Eso, eso será---dijimos todos.---De ese modo les cogeremos entre dos
fuegos y no escapará ni una patena de las que han robado en Córdoba.

---Pero si ese es el plan, ya debía estar puesto en ejecución. Si se
están batiendo en Andújar, a estas horas deberíamos estar nosotros
cayendo sobre la retaguardia francesa; mientras que si nos ponemos en
marcha esta noche y llegamos mañana, sabe Dios\ldots{}

Al anochecer se nos puso en movimientos río arriba, lo cual no
comprendimos ni poco ni mucho hasta que algunos compañeros que eran del
país y conocían el terreno nos dijeron que íbamos buscando el vado del
Rincón para pasar al otro lado. Por la noche algunas fuerzas de
infantería y dos piezas pasaron por junto a la barca, mientras el grueso
del ejército con la caballería nos disponíamos a hacerlo media legua más
arriba. Antes de amanecer sentimos algunos tiros del otro lado, y
diósenos orden de hacer el menor ruido posible, y de no encender lumbre.
La noche era calurosa: habíamos comido poco y mal el día anterior, y con
esto y el no dormir no estábamos del mejor humor; pero la guerra tiene
mil contrariedades, y ojalá fueran todas como aquella. Entramos al fin
en el río, cuya frescura era agradable a nuestros cuerpos, secos e
irritados por el calor y el polvo, y algún tiempo después, cuando
comenzaban a iluminar el horizonte los primeros vislumbres de la aurora,
ya éramos dueños de la orilla derecha. El mayor general Abadía, que
había dirigido el paso, nos mandó replegarnos a un sitio bajo, donde
casi toda la fuerza podía permanecer oculta, y allí aguardamos más de
media hora. No se veían los enemigos por ningún lado; pero allá lejos
hacia la barca continuaba cada vez más vivo el tiroteo de fusil.

El terreno es por allí bastante quebrado, abundando los matorrales y
chaparros; y entre estos designaron un camino de trocha por donde avanzó
la infantería, mientras a los de a caballo se nos mandó caminar por
terreno más alto. Habíamos tomado tan al pie de la letra la orden de no
hacer ruido, que avanzamos despacio y silenciosamente con el alma en
suspenso y los ojos atentamente fijos en el último término del terreno
hacia la izquierda, punto donde se había trabado la acción. Vimos al fin
a los franceses tiroteándose con nuestros compañeros, con aquellos que
habían pasado la barca durante la noche, y luchaban en un campo bajo
salpicado de espesos matorrales.

En una pequeña loma, y como a dos tiros de fusil de aquel sitio,
brillaba inmóvil e imponente una cosa que desde el primer momento atrajo
nuestras miradas, infundiéndonos cierto recelo. Era un escuadrón de
coraceros, la mejor caballería del ejército de Dupont. Todos los jinetes
contemplamos el resplandor de las bruñidas corazas, en cuyos petos el
sol naciente producía plateados reflejos; y después de mirar aquello sin
decir nada, nos miramos unos a otros, como si nos contáramos. Ni una voz
se oía en nuestras filas: a todos se nos había cambiado el color, y
temblábamos aunque cada cual hiciera esfuerzos por disimularlo. El único
rumor que turbaba el profundo silencio de nuestro regimiento, donde
hasta los caballos parecían contener el aliento y explorar el campo con
atónitos ojos, era un ligero y casi imperceptible son metálico producido
por las estrellas de las espuelas. Aquel temblor de piernas es un
accidente que la caballería observa siempre en el comienzo de toda
batalla.

El combate, principiado en guerrillas, arreciaba desde que empezó la
infantería a desplegar un frente compacto de consideración. Pero casi
toda la tropa española se mantenía en reserva, esperando a saber
fijamente si los franceses ocultaban una gran fuerza en la carretera de
Bailén. Mientras el frente español aumentaba sus tiros, resistiendo a
las innumerables guerrillas francesas, que al abrigo de sus posiciones
medio atrincheradas hacían fuego mortífero, la artillería continuaba a
retaguardia, y la caballería, asimismo fuera de acción, recibió orden de
ocupar un cerro a mano derecha. Fijos allí, no quitábamos los ojos de la
tremenda fila de corazas que resplandecían en la loma de enfrente,
quietas y confiadas en su valor y pesadumbre. Aquella fuerza era muy
superior a la nuestra por su organización y la marcialidad de cada uno
de sus soldados; pero nosotros teníamos sobre ella, además de la ventaja
numérica, que no era de gran valor, dada nuestra impericia, la siguiente
ventaja moral: puestos ellos en la vertiente anterior de una loma, todo
su poder y su número se presentaban a nuestra vista: no había más
coraceros que aquéllos, y podíamos contarlos uno por uno. Nosotros, en
cambio, estábamos sabiamente colocados por el mayor general en otra
altura parecida; pero sólo una quinta parte del regimiento ocupaba la
parte culminante de la loma, mientras que todo lo demás se extendía en
la vertiente posterior, permaneciendo completamente oculto a la vista
del enemigo; de modo que si nosotros les contábamos perfectamente a
ellos, los franceses, engañados por la apariencia, se reirían de los
treinta o cuarenta jinetes sin uniforme, enseñoreados del cerro con aire
de perdona vidas.

Nosotros teníamos sobre ellos la ventaja de lo desconocido, que es el
genio tutelar de las batallas, de eso que no se ve y que en el momento
apurado y crítico sale inopinadamente de lo hondo de un camino, del
respaldo de una loma, de la espesura de un bosque; combatiente de última
hora que la tierra echa de su seno, y se presenta fresco, sin heridas ni
cansancio a decidir la victoria.

Nuestras filas habían desalojado a los franceses de sus posiciones. Les
vimos replegarse en desorden y entonces cesó la inmovilidad de los
coraceros. Los resplandecientes petos despedían múltiples reflejos, y
ordenadamente descendieron de su colina en perfecta fila. Relincharon
sus caballos, y los nuestros relincharon también, aceptando el reto.
Pero entonces ocurrió uno de esos cambios de escena tan frecuentes en la
guerra, y cuyo artificio, si cae en buenas manos, basta a decidir la
victoria. Arrojadas nuestras filas sobre las guerrillas enemigas,
clareado el terreno y puestas en juego algunas piezas de artillería,
viose que los franceses vacilaban, agrupándose y retrocediendo como si
buscaran nuevas posiciones. Se nos dio orden de avanzar bajando, y una
vez en llano, convertimos sobre nuestro flanco, para formar un largo
frente de batalla. La infantería francesa estaba delante de nosotros,
resguardada por sus coraceros: pero estos observando nuestro movimiento
y reconociendo al instante su indudable inferioridad, invadieron
precipitadamente la carretera. La retirada era cierta. Se nos formó en
columnas, dándonos orden de cargar, y el regimiento se puso rápidamente
al galope. Parecía que la misma tierra, sacudiéndose bajo las herraduras
de nuestros caballos, nos echaba hacia adelante. Aquellos primeros pasos
tras un ideal de gloria, acompañaron voces de guerra mezcladas con
piadosas invocaciones.

---¡Madre nuestra, Santa Virgen de Araceli, ven con nosotros!

---¡Viva España, Fernando VII, y la Virgen de la Fuensanta!

Ya nadie pensaba en tener miedo: muy lejos de esto, todos los de mi fila
rabiábamos por no estar en las de vanguardia, en aquellas filas dichosas
que acometían a sablazos a los franceses de a pie, ya pronunciados en
completa dispersión. Tal era nuestro furor bélico en aquella fácil
victoria, que D. Diego, Marijuán y yo, no encontrando a derecha e
izquierda francés alguno, hacíamos grande estrago con nuestros sables en
los arbustos del camino, diciendo: «Perros, canallas, ya sabréis cómo
las gastamos los españoles».

La gloria de cargar sobre la infantería francesa perteneció tan sólo a
las primeras filas, aunque no les duró mucho el regocijo, porque los
enemigos, convencidos ya de que no tenían fuerza bastante para hacernos
frente, tomaban a toda prisa el camino de Bailén. Una vez posesionados
del camino, seguimos adelante; pero los caballos enemigos corrían a todo
escape, y la infantería se puso en salvo por las veredas, dispersándose
a un lado y otro de la carretera. Sobre las diez nos detuvimos, y
puestas en orden las columnas, avanzamos despacio, porque recelábamos de
ser atacados por una división entera. Entretanto nuestras pérdidas
habían sido nulas en la caballería, y escasas, aunque sensibles, en la
infantería, que perdió un capitán del regimiento de la Reina y bastantes
soldados.

Después de haber perdido de vista a los enemigos, continuamos la marcha
hacia Bailén, si bien con mucha cautela, pues había la presunción de que
los franceses, reforzados con gran número de tropas y caballos y
artillería, se nos presentarían de nuevo en mitad del camino,
sorprendiéndonos en nuestra triunfal carrera. Así fue en efecto. A eso
del medio día nuestras columnas avanzadas recibieron el fuego de los
imperiales, que rehechos con un destacamento que había llegado de
Linares, trataban de ganar lo perdido.

Furiosos por el reciente desastre, acometieron briosamente a nuestra
vanguardia. Tomamos posiciones, y las tropas ligeras, ayudadas de un
enjambre de paisanos, se diseminaron por las escabrosidades colindantes,
desde cuyos matorrales mortificaban a los franceses con fuego menudo. La
caballería entretanto continuaba muy lejos de la acción, y aunque
nuestro deseo hubiera sido que se nos enviara a lo más recio para
desahogar la furia de nuestro enardecido pecho, Dios quiso por fortuna
que no llegase esta ocasión, pues la escaramuza terminó de improviso;
cesaron los tiros, y vimos con sorpresa que los franceses, como poseídos
de súbito pavor, retrocedían a la desbandada hacia Bailén, recogiendo
precipitadamente sus heridos.

¿Qué ocurría? Según después supimos, los franceses había tenido una
pérdida funesta, la de su general Gobert, el cual cayó mortalmente
herido por una de esas balas de invisible guerrero, que salían de entre
las malezas para taladrar el corazón del Imperio. Aquel valiente militar
murió pocas horas después en Guarromán. Dueños nosotros del campo, y sin
enemigos a la vista, parecía natural que fuéramos sobre Bailén; pero el
ejército volvió hacia Mengíbar para repasar el río, movimiento que no
fue por nosotros comprendido. Todos estábamos muy orgullosos, y
especialmente los paisanos inexpertos no cabíamos en el pellejo.

---¡Hoy es día del Carmen!---exclamó D. Diego.---¡Viva la Virgen del
Carmen, y mueran los franceses!

Ruidosas exclamaciones alegraron y conmovieron nuestras filas. Era el 16
de Julio: en este día la Iglesia celebra, además de la advocación del
Carmen, el Triunfo de la Santa Cruz, fiesta conmemorativa de la gran
batalla de las Navas de Tolosa, ganada contra los infieles por
castellanos, aragoneses y navarros, en aquellos mismos sitios donde
nosotros perseguíamos a los franceses, y en el mismo 16 del mes de
Julio. Habían pasado quinientos noventa y seis años. La coincidencia del
lugar y la fecha nos inflamaba más, y añadido a nuestro patriotismo una
profunda fe religiosa, nos creímos héroes, aunque hasta entonces no
habíamos tenido ocasión de probarlo.

Antes de cruzar el río, descansamos para llevar algo a la boca. ¡Oh, qué
desengaño! Estábamos muertos de hambre y cansancio, y se nos dijo que no
había más que un tercio de ración. Pero nosotros éramos buenos chicos y
nos conformamos, supliendo los dos tercios restantes con la sustancia
moral del entusiasmo.

---Pero Sr.~de Santorcaz---pregunté a mi compañero, cuando con el agua
al estribo vadeábamos el Guadalquivir,---¿nos quiere Vd. decir por qué
no se nos ha llevado adelante? ¿Por qué después de esta victoria
desandamos lo andado?

---¡Zopenco!---me contestó.---Esto no ha sido más que una fiestecilla de
pólvora, y todavía no ha empezado lo bueno. ¿Crees que no hay más
franceses que esos cuatro gatos de Ligier-Belair? ¿Qué sabes tú si a
estas horas, Vedel, que fue a Andújar en auxilio de Dupont, habrá
regresado a Bailén? Ahora, o yo me engaño mucho, o vamos en busca del
marqués de Coupigny para reunirnos y emprender juntos un nuevo ataque.
¿Estás al tanto de lo que digo? ¿Ves cómo no en vano ha mordido uno el
cebo en Hollabrünn, en Austerlitz y en Jena?

Efectivamente, la intención de nuestro general era reunirse con
Coupigny; pero esto no se verificó hasta la noche del 17 al 18.

\hypertarget{xviii}{%
\chapter{XVIII}\label{xviii}}

Se nos acampó en una altura a espaldas de Mengíbar, y supimos con gusto
que aquella noche no haríamos movimiento alguno. Nuestro gozo, como
nuestra fatiga, necesitaba descanso; necesitábamos dar desahogo al
efervescente alborozo, no sólo renovando en la memoria todos los
incidentes de la acción de aquel día, sino también refiriendo cuanto
cada uno hizo y cuanto dejó de hacer para que la batalla fuese
completamente ganada. Los suizos y los soldados de línea no estaban tan
engreídos como nosotros los paisanos, que creíamos haber asistido a la
más grande y gloriosa batalla de los modernos tiempos. Mirábamos con
desdeñosa indiferencia a los que quedaron de reserva, y al contarles lo
que pasó, hacíamos subir a cifras fabulosas el número de franceses
segados por nuestros cortadores sables en la refriega.

Largas horas pasamos sobre el campo saboreando los deliciosos recuerdos
de tanta gloria, que como dejos de un manjar muy rico nos renovaban el
placer del vencimiento. La noche era como de verano y como de Andalucía,
serena, caliente, con un cielo inmenso y una atmósfera clara, donde
fluctúa algo sonoro, cuya forma visible buscamos en vano en derredor
nuestro. Tendidos sobre la caldeada tierra a orillas del río, cuyas
frescas emanaciones buscábamos con anhelo, entreteníamos las horas
hablando, cantando, o haciendo eruditas disertaciones sobre la campaña
tan felizmente emprendida. En un grupo se jugaba a las cartas, en otro
se decía un romance de héroes o de santos, en este algunos cantaores
echaban al vuelo las más románticas endechas de la tierra, pues desde
entonces era romántica Andalucía; en aquel se narraban cuentos de
brujas, y en algunos, finalmente, se dormía sin inquietud por el día
venidero.

Nuestro D. Diego, siempre al arrimo de Santorcaz; Marijuán, yo y algunos
más formábamos un grupo bastante animado, en el cual no cesó el ruido
hasta muy alta la noche. Después de cantar, no escasearon los cuentos,
acertijos y adivinanzas, y por último, la conversación recayó en tema de
mujeres:

---Yo---dijo D. Diego con su natural ingenuidad,---me voy a casar. A
todos les convido a mi boda. «¿Y quién es la novia?» dirán Vds. Pues
sepan que no la he visto. Mi señora madre lo ha arreglado todo con otras
dos señoras de Córdoba, y según me han dicho, es más bonita que el sol,
aunque ahora le ha dado por no salir del convento.

---Será para cuando acabe la guerra, porque ahora no está el horno para
bollos---dijo Marijuán.---Yo también voy a casarme con una muchacha de
Almunia, que tiene siete parras, media casa y burro y medio de hijuela.
También será cuando acabe la guerra, y a todos les convido a mi boda. ¿Y
tú, Gabriel?

---Pues yo para no ser menos---contesté,---diré que cuando se acabe la
guerra me pienso casar también. ¿Y con quién?, dirán Vds. Pues me caso
con una condesa.

---¡Con una condesa!

---Sí señores, con una condesa que posee todas estas tierras que estamos
viendo y otras más allá, y tiene dos escudos con ocho lobos sobre plata
y catorce calderos, con media cabeza de moro y un letrero que
dice\ldots{}

\emph{---Toma casa con hogar y mujer que sepa hilar}---dijo Marijuán
interrumpiéndome.---¿Pues no dice que se casa con una condesa? Será con
alguna duquesa del estropajo. Pero di, ¿en qué alcázares reales está tu
novia?

---Este es un bobalicón que no sabe lo que se habla---dijo D.
Diego.---¡Buena condesa será ella! Pues, como os decía, muchachos, mi
novia está muy desazonada esperando a que se acabe la guerra para
casarse conmigo. Así me lo han dicho, y lo creo. Apuesto que están Vds.
rabiando por saber quién es y cómo se llama; pero eso no lo he de
mentar, porque mi señora madre y D. Paco me dijeron que si hablaba de
esto antes de llegar la ocasión me castigarían no dejándome montar en el
potro. ¡Qué guapa es, señores! Sus ojos son dos luceros, como aquel
grande y muy claro que está sobre el tejado de esa casa; su boca se
compone de dos hojas de rosa; sus dientes hacen que todas las perlas
echen a correr de envidia; sus mejillas son claveles abiertos, y cuando
llora sus lágrimas son diamantes. Yo no la he visto más que en figura;
porque han de saber Vds. que cuando fui a visitar a sus tías en Córdoba
me dieron un medalloncito con el retrato de la que ha de ser mi mujer,
el cual retrato, por temor a que se me perdiera, lo he dado a guardar al
señor de Santorcaz.

---Eso se parece---dijo uno de los oyentes,---a la historia de la
princesa Laureola, por quien vinieron de La Meca los tres reyes moros, y
dice el cuento que tenía los ojos de azabache ardiendo, la boca de flor
de granado, y las orejas de caracolitos del mar. ¿Lo sabes tú?

---Eso está en el romance de la Reina mora, bruto. ¿Qué tiene eso que
ver con la princesa Laureola?

---Yo sé el romance de la Reina mora---gritó don Diego batiendo
palmas.---¿Lo echo? ---Venga.

---No; el del \emph{Barandal del cielo}, que es más bonito y habla de la
Virgen---añadió el condesito gozoso de hallarse a punto de lucir sus
habilidades.---Me lo enseñó mi hermana Presentación, que sabe
veintisiete y los dijo todos arreo delante del señor obispo de Guadix,
cuando su ilustrísima paró en casa el mes pasado.

Y sin esperar a que le rogasen, el mayorazguito de Rumblar, con
sonsonete de escuela, voz agridulce y amanerados gestos dio principio a
la siguiente retahíla:

\small
\newlength\mlena
\settowidth\mlena{Nuestra madre, buen San Juan,   }
\begin{center}
\parbox{\mlena}{   Por el barandal del cielo                \\
                se pasea una doncella                       \\
                blanca, rubia y encarnada,                  \\
                que alumbra como una estrella.              \\
                San Juan le dice a Jesús:                   \\
                ¿quién es aquella doncella?                 \\
                Nuestra madre, buen San Juan,               \\
                nuestra madre linda y bella;                \\
                la Virgen no viene sola,                    \\
                ángeles vienen con ella;                    \\
                no viene vestida de oro,                    \\
                ni de plata, ni de seda;                    \\
                viene vestida de grana...}                  \\
\end{center}
\normalsize

Y como al concluir fuera acogida esta relación con una salva de
aplausos, animose el recitador y nos endilgó otra, no menos famosa, que
empezaba:

\small
\newlength\mlenb
\settowidth\mlenb{Nuestra madre, buen San Juan,   }
\begin{center}
\parbox{\mlenb}{   Allá arriba en aquel alto                \\
                hay una fuente muy clara,                   \\
                donde se lava la Virgen                     \\
                sus santos pechos y cara...}                \\
\end{center}
\normalsize

---¡Basta de romances!---exclamó de improviso Santorcaz, asustándonos a
todos con su interrupción.---Eso es cosa de chiquillos, y no de hombres
formales. ¿No sabe Vd. más que eso?

---Sé muchos más---dijo tímidamente el joven.---D. Paco me ha enseñado
muchos, y me los hace aprender de memoria para que los diga en las
tertulias.

---¿Y nada más le ha enseñado a Vd. ese señor D. Paco, a quien desde el
primer momento tuve y diputé por un gran zopenco?

---También me ha enseñado historia, sí señor. Y sé lo de nuestro padre
Adán y aquello de Alejandro cuando fue a dar batallas a los persas como
ahora vamos nosotros a dárselas a los franceses.

---¿Y nada más?

---¡Toma: también latín!, pero mi señora madre mandó que no me
atarugasen la cabeza de latín, puesto que no era necesario, y por último
D. Paco dijo que con saber un poquito de \emph{Musa musæ} bastaba.

---¿Y qué libros ha leído Vd.?

---Nada más que la \emph{Guía de Pecadores}, donde está aquello del
infierno. Ese libro es muy feo, y mi señora madre no me dejaba leer más
que lo del infierno, que da mucho miedo, y sueña uno con ello. Pero mi
señora madre tiene otros libros en el cofre, y cuando iba a misa, yo con
mucha cautela los sacaba para leerlos. Uno se titula La \emph{farfulla o
la cómica convertida}, novela escrita por un fraile de mínimos, y otra,
\emph{Princesa, ramera y mártir}, \emph{Santa Afra}. Ambos libros son
muy bonitos y traen un aquel de amores y besos que me daba mucho gusto
cuando los leía a escondidas.

Santorcaz sonreía. Después de una pausa, dijo con cierta petulancia:

---¿De modo que no ha leído Vd. la \emph{Enciclopedia}?

---¿Qué es eso?

---La \emph{Cincopedia}---exclamó uno.---¡Eh!, ¿sabes tú a dónde cae la
\emph{Cincopedia}?

Esta palabra, que adquirió fortuna aquella noche, fue pasando de boca en
boca, y más de cien la repitieron entre zumbas y chacota.

---Veo que son Vds. unos animales---dijo Santorcaz un poco
avispado.---De todos modos, Sr.~D. Diego, la educación que Vd. ha
recibido no puede ser más deplorable en un joven mayorazgo, que por lo
mismo que ha de sobresalir entre los demás en la sociedad, debe cultivar
su entendimiento.

---A ver, amigo---dijo Rumblar,---hábleme Vd. de esas cosas que me
gustan. Todo lo que Vd. me decía anteayer, cuando íbamos de camino por
aquí, me tenía encantado, y le juro que si no estuviera en vísperas de
casarme y fuera preciso seguir con ayo, le diría a mi señora madre que
me le pusiera a Vd. en lugar de D. Paco, el cual bien se me alcanza que
no me ha enseñado más que gansadas y tonterías.

---Pues repito que un joven destinado a ocupar tan alta posición en el
mundo, debe saber algo más que el romance del \emph{Barandal del cielo}.
Verdad es que, o mucho me equivoco, o todo eso de los mayorazgos se lo
llevará la trampa, y tarde o temprano se pondrán las cosas de manera que
cada cual sea hijo de sus obras.

---Así debe ser---dijo Marijuán.---¿No somos todos hijos de Dios?

---Vengan Vds. acá y respondan---dijo Santorcaz excitando la curiosidad
de sus oyentes.---¿No les parece que el mundo está muy mal arreglado?

Abriéronse varias bocas con estupefacción, y no se oyó ninguna
respuesta.

---Pues yo que no he leído ningún libro---afirmó al fin uno de los
circunstantes---digo que Dios tiene que volver a hacer el mundo, porque
eso de que se lo lleve todo el que primero salió del vientre de la madre
y los demás se queden bailando el pelao, no está bien. Mi hermano el
mayor, sólo porque le dio la gana de nacer antes que yo, tiene tres
dehesas y dos casas; y los demás\ldots{} uno hubo de meterse fraile,
otro se fue al Perú, otro está muerto de hambre en un hospital de
Sevilla, y yo, señores, tuve que meterme en el contrabando para que no
se me helara el cielo de la boca.

---Oye, tú, Marijuán---dijo otro,---¿sabes lo que contaban en Sevilla?
Pues decían que la Junta se iba a poner de compinche con las otras
Juntas para ver de quitar muchas cosas malas que hay en el gobierno de
España, lo cual podemos hacer nosotros, \emph{sin necesidad de que
vengan los franceses a enseñárnoslo}\footnote{Palabras textuales de la
  Junta de Sevilla.}.

---Así ha de ser---observó Santorcaz.---Me han dicho que en Sevilla hay
sociedades secretas.

---¿Qué es eso?

---Ya sé---dijo uno.---Tiene razón D. Luis. En Sevilla hay lo que llaman
\emph{flamasones}, hombres malos que se juntan de noche para hacer
maleficios y brujerías.

---¿Qué estás diciendo? No hay tales maleficios. Mi amo iba también a
esas Juntas, y cuando su mujer se lo echaba en cara, respondía que los
que allí iban eran al modo de filósofos, y no hacían mal a nadie.

---Pues en Madrid las sociedades secretas están todavía en la
infancia---añadió Santorcaz.---En Francia las hay a miles, y todo el
mundo se apresura a inscribe en ellas.

---Pues si voy a Madrid---dijo con énfasis el mayorazguito,---lo primero
que haré será meterme en una de esas sociedades, donde sin duda se han
de aprender muy buenas cosas. ¿No es verdad, D. Luis? Yo no tengo nada
de torpe: me lo conozco, sí, señores. ¿Creerá Vd., Sr.~de Santorcaz, que
eso que Vd. ha dicho de los mayorazgos se me había ocurrido a mí muchas
veces cuando jugaba en el patio de casa con las gallinas? Pero ya que me
enseña Vd. lo que ignoro, contésteme a una duda: ¿Por qué tenemos
nosotros en nuestras casas tantos papelotes llenos de garabatos, y por
qué usamos esos escudos con sapos y culebras? El de mi casa tiene cuatro
lagartos y un tablero de ajedrez con dos calderitos muy monos.

---Si esos signos representan algo---repuso Santorcaz,---es referente al
primero que los usó, a sus hazañas si las hizo, y a sus privilegios si
los tuvo; pero hoy, amiguito, tales pinturas no valen de nada, y dentro
de algunos años, los que las posean sin dinero, serán unos pobres
pelagatos, a quienes nadie se arrimará, así como todo aquel que haya
hecho una fortuna con su trabajo o la haya heredado de sus padres, o
descuelle por su talento, será bien quisto en el mundo, aunque no tenga
ni un adarme de lagartija en su escudo.

---¿De modo---preguntó el mozalbete,---que yo seré un pelagatos, si
llego a perder mi patrimonio o soy un bruto? Esto sí que es bueno.

---Nada, nada---dijo uno.---Fuera mayorazgos, y que todos los hermanos
varones y hembras entren a heredar por partes iguales.

---Eso no puede ser---observó Marijuán,---porque entonces no habría las
grandes casas que dan lustre al reino.

---Eso no puede ser---afirmó un tercero.---Pues qué, ¿el Rey iba a ser
tan tonto que quitara los mayorazgos? Nada, nada; los dejará siempre por
la cuenta que le tiene.

---Es que si el Rey no quiere quitarlos, no faltará quien los
quite---afirmó Santorcaz.

Todos se rieron al oír sostener la idea de que existe alguna voluntad
superior a la voluntad del Rey.

---¿Cómo puede ser eso? Si el Rey no quiere\ldots{} ¿Hay quien esté por
cima del Rey? El Rey manda en todas partes, y digan lo que quieran, no
hay más que su sacra real voluntad. ¡Muchachos, viva Fernando VII!

---Pero vengan acá, zopencos---dijo Santorcaz.---¿Dicen Vds. que nadie
manda más que el Rey?

---Nadie más.

---Y si todos los españoles dijeran a una voz: «queremos esto, señor
Rey, nos da la gana de hacer esto», ¿qué haría el Rey?

Abriéronse de nuevo todas las bocas, y nadie supo contestar.

\hypertarget{xix}{%
\chapter{XIX}\label{xix}}

---Gaznápiros, animales: si Vds. están probando lo que digo---añadió con
energía D. Luis.---Lo que pasa en España ¿qué es? Es que el Reino ha
tenido voluntad de hacer una cosa y la está haciendo, contra el parecer
del Rey y del Emperador. Hace tres meses había en Aranjuez un mal
ministro, sostenido por un rey bobo, y Vds. dijeron: «No queremos ese
ministro ni ese Rey», y Godoy se fue y Carlos abdicó. Después, Fernando
VII puso sus tropas en manos de Napoleón, y las autoridades todas, así
como los generales y los jefes de la guarnición, recibieron orden de
doblar la cabeza ante Joaquín Murat; pero los madrileños dijeron: «No
nos da la gana de obedecer al Rey ni a los Infantes ni al Consejo ni a
la Junta ni a Murat», y acuchillaron a los franceses en el parque y en
las calles. ¿Qué pasa después? El nuevo y el viejo Rey van a Bayona,
donde les aguarda el tirano del mundo. Fernando le dice: «La corona de
España me pertenece a mí; pero yo se la regalo a Vd., Sr.~Bonaparte». Y
Carlos dice: «La coronita no es de mi hijo, sino mía; pero para acabar
disputas, yo se la regalo a Vd., señor Napoleón, porque aquello está muy
revuelto y usted sólo lo podrá arreglar». Y Napoleón coge la corona y se
la da a su hermano, mientras volviéndose a Vds. les dice:
\emph{Españoles, conozco vuestros males y voy a remediarlos}. Pero Vds.
se encabritan con aquello, y contestan: «No, camarada, aquí no entra Vd.
Si tenemos sarna, nosotros nos la rascaremos: no reconocemos más Rey que
a Fernando VII». Fernando VII se dirige entonces a los españoles, y les
dice que obedezcan a Napoleón; pero entretanto, muchachos, un señor que
se titula alcalde de un pueblo de doscientos vecinos, escribe un
papelucho, diciendo que se armen todos contra los franceses: este
papelucho va de pueblo en pueblo, y como si fuera una mecha que prende
fuego a varias minas esparcidas aquí y allí, a su paso se va levantando
la Nación desde Madrid hasta Cádiz. Por el Norte pasa lo propio, y los
pueblos grandes lo mismo que los pequeños forman sus Juntas, que dicen:
«No, si aquí no manda nadie más que nosotros. Si no reconocemos las
abdicaciones, ni admitiremos de Rey a ese D. José, ni nos da la gana de
obedecer al Emperador, porque los españoles mandamos en nuestra casa, y
si los reyes se han hecho para gobernarnos, a nosotros no nos han parido
nuestras madres para que ellos nos lleven y nos traigan como si fuéramos
manadas de carneros\ldots» ¿Están Vds.? ¿Lo comprenden Vds.? Pues esto
ni más ni menos es lo que está pasando aquí. Y ahora contéstenme los
alcornoques que me oyen: ¿Quién manda, quién dispone las cosas, quién
hace y deshace, el Rey o el Reino?

El estupor que produjeron estas palabras reveladoras en el atento
concurso, compuesto de muchachos rudos e ignorantes, pero de gran viveza
de imaginación, fue tan extraordinario que por un corto rato no se oyó
la más insignificante voz, señal cierta de que las ideas vertidas por
Santorcaz, entrando de improviso en los oscuros cacúmenes de sus
oyentes, habían armado allí gran zipizape y polvareda, dejándolos
aturdidos, confusos y sin palabra. El primero que rompió el silencio fue
Rumblar, diciendo:

---Todo eso está muy bien dicho. ¿Querrán ustedes creer que hace días me
ocurrió una idea parecida cuando estaba cazando moscas y poniéndoles
rabos en cierta parte, para que al volar hicieran reír a mis dos
hermanas que estaban rezando? Sólo que yo no sabía cómo decir aquello
que pensaba.

---Sí, señores, ¡vivan las Juntas!---exclamó uno levantándose.---Yo me
sé de memoria aquel papel que echó a la calle la de Córdoba,
diciendo\ldots{} Oigan ustedes: «¡Cordobeses: los reinos de Andalucía se
ven acometidos por los asesinos del Norte; vuestra patria va a verse
oprimida bajo el yugo de un tirano; vosotros mismos seréis arrancados de
vuestros hogares y de vuestras casas! ¡Cuarenta argollas está labrando
el lascivo Murat para conduciros al Norte como a los animales más
inmundos!\ldots{} ¡Soldados: gemid de rabia y furor!\ldots{} Doce
millones de hombres os están mirando y envidiando vuestra gloria, y aun
la Francia misma ansía por vuestros triunfos».

Ruidosos aplausos y gritos acogieron esta proclama, fielmente recitada
con dramáticos gestos por el muchacho.

---Pues si los españoles---continuó luego Santorcaz,---pueden hacer lo
que están haciendo, no pueden también decir el día de mañana: «Vamos, no
queremos que haya más inquisición, ni más vinculaciones»\ldots{} pongo
por caso\ldots{} O que digan: «En lugar de mil conventos, que haya tan
sólo la mitad, con lo cual basta y sobra», o «no me da la gana de que
haya diezmos»\ldots{}

---Eso sí que estaría bueno---dijo Marijuán.---Pero si todos los
españoles van a hacer eso, y cada uno empieza a gritar por su lado
diciendo lo que quiere, se armará tal laberinto que no podrán
entenderse.

---Vaya unos zotes---añadió Santorcaz.---Pero venid acá: ¿no veis que
hay en Sevilla una Junta que es la que dispone? ¿No veis que hay otra en
Granada, otra en Córdoba y otra en Málaga, etc.? Pues en lugar de todas
esas Juntas pequeñas que gobiernan en cada pueblo, ¿no puede haber una
muy grande que se reúna en Madrid y acuerde lo que se ha de hacer?

Miráronse los oyentes unos a otros, y los monosílabos de aquiescencia y
aun de admiración corrieron de boca en boca, demostrando la prontitud
con que aquellas juveniles inteligencias desplegaban sus alas, aún
entumecidas y vacilantes, para intentar describir los primeros círculos
en el espacio del pensamiento.

---Estas conversaciones me enamoran---dijo el condesito de Rumblar.---Me
estaría toda la noche oyendo a este hombre, sin cansarme. Ya, ya voy
aprendiendo muchas cosas que no sabía.

Así aquella fantasía encerrada en el capullo de una educación mezquina,
agujeraba con entusiasmo su encierro, porque había vislumbrado fuera
alguna cosa que tenía la fascinación de lo nuevo. Así aquel germen de
pasión y de inteligencia, guardado en un huevo, se reconocía con vida,
se reconocía con fuerza, y empezaba a dar picotazos en su cárcel,
anhelando respirar fuera de ella otros aires, y calentarse con calores
más enérgicos. Así aquella ceguera abría sus párpados, gozándose en la
desconocida luz.

La conversación terminó en el punto en que la he dejado, porque la noche
estaba muy avanzada y casi todos empezaron a rendirse al sueño, excepto
el mayorazguito, cuyo despabilamiento era casi febril a causa del
organismo de su imaginación. Largo tiempo continuaron él y Santorcaz
hablando en diálogo animadísimo, y como si discutieran planes y
expusieran proyectos de gran trascendencia para los dos. Yo me aparté
del grupo, fingiendo retirarme a dormir; pero con ánimo de satisfacer
una imperiosa exigencia de mi alma, que a voces me pedía soledad y
meditación. Todos los ruidos habían cesado en el campamento: las
guitarras y castañuelas, así como las cajas y las cornetas, estaban
mudas, porque el ejército dormía. Lejos del grupo de mis amigos, echeme
sobre el suelo, aguardando la aurora, sin poder ni querer cerrar los
ojos; y allí me puse a meditar sobre lo que desde mi salida de Madrid
había visto y oído. ¡Cuántas personas nuevas para mí había encontrado en
aquella breve jornada de mi vida! ¡Con cuánto afán, meditando a solas y
mirándolas al lado, preguntaba a aquellos caminantes si tenían alguna
noticia de lo que me reservaba el destino! De todas aquellas personas,
ninguna estaba tan enérgicamente fija en mi pensamiento como Santorcaz,
hombre para mí incomprensible y sospechoso, y que empezaba a inspirarme
secreta antipatía, sin que acertara a explicarme por qué.

\hypertarget{xx}{%
\chapter{XX}\label{xx}}

Al siguiente día hicimos un movimiento por la orilla izquierda, río
arriba, hasta un punto mucho más alto que Mengíbar. Nada entendíamos;
pero Santorcaz, o por petulancia o porque realmente había penetrado la
intención de Reding, nos dijo:

---Nuestro general sabe lo que se hace, y es hombre que conoce la
filosofía de las marchas.

Haciendo alto a orillas del Guadalimas, parte del ejército se entretuvo
en marchas incomprensibles, y empleando en esto más de un día, nos
encontramos de nuevo sobre Mengíbar al anochecer del 18, punto al cual
había llegado horas antes la división del marqués de Coupigny. Reunidos
ambos ejércitos, no hubo allí más parada que la precisa para recoger las
provisiones de que estábamos tan escasos, y ya muy de noche emprendimos
el camino de Bailén. Éramos catorce mil hombres. Todo anunciaba que
íbamos a tener un encuentro formal con el ejército francés.

Según nuestras noticias, Dupont continuaba en Andújar, reforzado por la
división de Vedel. ¿Habían trabado acción con nuestro tercer cuerpo y el
de reserva que, pasando el río por Marmolejo, estaban situados en la
orilla derecha? Nosotros creíamos que sí, a menos que Castaños no
aguardase para atacar enérgicamente a que la primera y segunda división
cayeran sobre la espalda del ejército de Dupont, bajando desde Bailén.
¿Era este el objeto que nos guiaba en nuestra marcha? Parecíanos que sí.

Mientras llegaba el momento del drama, lejos de nosotros y en los
flancos del ejército imperial, mil dramáticas peripecias debían
precipitar la catástrofe, irritando paulatinamente al enemigo. Los
cuerpos y columnas de guerrilleros, mandados por D. Juan de la Cruz, el
conde de Valdecañas y el clérigo Argote, se habían desparramado como
enjambre mortífero por los pueblos y caseríos que dominaban el cuartel
general francés en las primeras estribaciones de la sierra al Norte de
Andújar. De tal modo perseguían aquellos ardorosos paisanos a los
franceses y con tanta rapidez se dispersaban para evitar ser atacados,
que a los invasores les era de todo punto imposible estar tranquilos un
solo momento. El poderoso gigante sacudía de una manotada aquellos
moscones venenosos; pero estos volvían a zumbar en derredor suyo, le
molestaban con sus terribles picaduras y huían incólumes, sin temer la
espada ni el cañón, pues estas armas no se han hecho para mosquitos.

No podían apartarse los franceses de su cuartel general como no fuera en
grandes destacamentos: frecuentemente iban mil hombres a llenar en la
fuente próxima unas cuantas alcarrazas de agua. Si por acaso salían a
merodear pelotones de poca fuerza, eran despachados por los guerrilleros
en menos que se reza un credo. Antes que consentir que se apoderasen de
una panera, la quemaban: las fuentes eran enturbiadas con lodo y
estiércol, para que no pudieran beber: los molinos desmontados y
enterradas sus piedras para que no molieran un solo grano. ¡Ay de aquel
francés que se rezagara en las marchas de su destacamento! ¡Sentíase de
improviso asido por mil coléricas manos, sentíase arrastrado por las
mujeres, pellizcado por los chicos y acuchillado por los hombres, hasta
que su existencia se apagaba con horrible choque en la fría profundidad
de un pozo! El invasor no encontraba asilo en ninguna parte, y
forzosamente encerrado en los límites del cuartel general, veía
conjurados contra sí hombres y naturaleza. Por esto, rabioso y
desesperado, anhelaba batirse en función campal, seguro de su destreza y
costumbre de guerrear; y lamentando la estupefacción del general en
jefe, exclamaba: «Demos una batalla, y aunque muera la mitad del
ejército, la otra mitad conquistará un charco en que beber y un puñado
de trigo seco que llevar a la boca».

Habían dejado los franceses en Montoro un destacamento de setenta
hombres, para custodiar un molino donde fabricaban con dificultad harina
malísima. El alcalde de aquella villa, donde no había quedado ni una
sola arma de fuego, se atreve, sin embargo, a dar cuenta de los setenta
franceses, para lo cual era preciso despachar primero a los veinticinco
que a todas horas estaban de guardia en el puente. Reúne, pues, algunos
paisanos decididos, y usando la arma blanca, ataca con furia a la
guardia; los veinticinco son exterminados; apodérase de sus fusiles la
valiente cuadrilla, sorprende el resto del destacamento en la casa donde
se albergaba, hace prisioneros a soldados y jefes, y les manda a la isla
de León. El parte en que se notificó este suceso a la Junta Suprema
decía que todo se hizo con las \emph{varas de los harrieros} (conservo
la ortografía del original); pero esto ha de ser una hipérbole andaluza.

Sintiéndose llamado a más grandes acciones, don José de La Torre (que
así se nombraba aquel alcaldito), sale al encuentro de un convoy que
venía de Córdoba, y de los cincuenta y nueve franceses que custodiaban
este, los cincuenta quedan tendidos en el camino, y los nueve restantes
corren a contar a Dupont lo que ha pasado. Entonces Dupont envía mil
hombres a Montoro con encargo de que incendien el pueblo y lleven vivo o
muerto al alcalde. Arde Montoro, y La Torre, conducido vivo, va a ser
pasado por las armas: pero un general francés, a quien poco antes había
dado hospitalidad, intercede por él; es puesto en libertad, y aquel
\emph{petit caporal} de las guerrillas marcha a Sevilla y recibe de la
Junta los galones de capitán de ejército.

Pues bien; lo que pasaba en Montoro, ocurría en todos los pueblos de la
carretera de Andalucía desde Córdoba hasta Santa Elena. El gigante que
incendiaba lugares y destrozaba ejércitos no podía dar un paso sin
encontrar un avispero, y frenético con aquel zumbido, envenenado por los
aguijones, maldecía la hora de la invasión. El águila, devorada por los
insectos, graznaba a orillas del Guadalquivir con hambre y calentura,
afilando sus garras en el tronco de los olivos, con el ansia de que
llegara pronto la ocasión de destrozar alguna cosa.

\hypertarget{xxi}{%
\chapter{XXI}\label{xxi}}

Al entrar en Bailén, ya muy avanzada la noche, nos sorprendió mucho el
no ver ninguna fuerza francesa a la entrada del pueblo para disputarnos
el paso. ¿A dónde habían ido los franceses? ¿Qué les pasaba, cuando ni
por precaución dejaron allí un par de batallones para guardar punto tan
importante? Pronto salimos de dudas, porque de boca de los habitantes de
Bailén, que salieron en masa a recibirnos, supimos que la división Vedel
había pasado por allí en dirección a la Carolina.

---Nosotros les hacíamos a Vds. en Linares---dijo D. Paco, que también
salió a nuestro encuentro, rebosando de júbilo.---¡Oh!, señor conde,
niño mío\ldots{} ¿Está por ventura herido Vuestra Excelencia? Vamos un
rato a casa, donde la señora marquesa y las niñas están rezando por el
buen éxito de la guerra. ¿No darán un descanso a las tropas?

Nuestro general había determinado salir en seguida para Andújar; pero
como ocupábamos todo el pueblo, pudimos llegarnos a la casa de nuestro
amo en cuya sala baja se nos dio un tente-tieso muy confortante.

---Es un milagro que podamos daros estos cuantos panes y estas onzas de
chocolate crudo---nos dijo don Paco al ofrecernos aquellos
artículos.---Los franceses no han dejado nada. ¡Qué horroroso saqueo! Y
gracias que quedamos con vida. ¡Ay!, la señora condesa salió a
recibirlos con una serenidad que me espantó. Yo temblaba y tuve que
esconderme en el oratorio, porque delante de ellos hubiera perdido la
dignidad de mi carácter. ¿Qué modo de saquear?\ldots{} En una palabra,
la paja de los caballos, las gallinas del corral, los huevos, hasta unos
tomates que tenía yo guardaditos en mi escritorio para hacer un
gazpachito\ldots{} todo, todo se lo llevaron. El pueblo está muerto de
miseria, y yo sé de mucha gente que echó la harina en los muladares para
que ellos no se la llevaran. ¿No lo creéis? ¿Pues y el Sr.~Salvador, que
sacó al campo los doscientos pellejos de aceite y ciento de vino que
tenía en su cueva, y destapándolos dejó correr aquel precioso caldo
hasta que todo se lo chupó la tierra? Otros hicieron una grande hoguera
con los carros y la paja. Las alhajas de las imágenes y la plata de las
iglesias están todas enterradas, porque esto parece que es lo que más
les abre el ojo a esos señores. Así estaban ellos de rabiosos, cuando
vieron que no sacaban de aquí gran cosa. El día 16, después de haber
pasado un gran miedo, gozamos lo indecible cuando les vimos llegar de la
barca de Mengíbar, derrotados y con su general muerto. ¡Cómo corrían por
esas calles, y qué gritos daban, y qué cosas tan atroces e indecentes
echaron por aquellas bocazas! ¡Así se vengaban los muy perros! ¿Pues qué
creéis? Dieron muerte a muchas personas que no les hacían daño, lo cual
creo yo que no se vio en ninguna de las guerras de Alejandro. Pero
también se les molió de firme. Unos cuantos pasaron por la calle de
enfrente echando bravatas y detuviéronse en la puerta de la posada de
Gil, donde tenían encendido el horno para cocer la loza. ¡Ay! Mis
francesitos se ponen a decir no sé qué insolencias obscenas a la mujer
de Gil, cuando salen los mozos, me los agarran y con morriones y
todo\ldots{} plaf\ldots{} al horno\ldots{} Pero ahí viene la señora
condesa, que estaba en el oratorio con las niñas.

En efecto, vimos desfilar gravemente, cubierta de negro manto, a la
señora de la casa, seguida de los dos tiernos pimpollitos de sus hijas,
las cuales arrojáronse llorando en los brazos de su hermano. Doña María
abrazó a su hijo sin perder ni por un instante su solemne y estirado
empaque, y luego saludonos a todos con mucho afecto, nombrándonos uno
por uno. Cuantos componían la cuadrilla estaban presentes, menos
Santorcaz, el cual desde nuestra llegada había pedido con mucha prisa a
D. Paco recado de escribir, y puéstose a trazar unas cartas en el
despacho de este.

La marquesa, después de saludarnos, tomó asiento y dirigió a D. Diego
estas palabras dignas de la historia:

---Hijo mío: sé todo lo que pasó en la acción del 16, y nadie me ha
dicho que hicieras algo notable. ¿Has tenido miedo?

---¡Miedo!---exclamó el muchacho riendo.---No señora. He cumplido con mi
deber en las filas, y nada más hasta ahora; pero su merced no se
impaciente, porque aunque no soy más que soldado espero lucirme.

---¡Nada más que soldado!---dijo la condesa.---Tú no eres soldado,
aunque así parezca. Cualquiera que sea el puesto que se ocupe, cada cual
debe obrar conforme a su nombre y a la posición que tiene en el mundo.
¿Qué se diría de ti, de mí, de esta casa, de tu difunto padre, si en
estas guerras no hicieras algo superior a lo que corresponde a un simple
soldado?

---Señora---repuso el mozo con un desenfado que sorprendió a su
familia,---yo haré lo que pueda, y según lo que haga, así seré más o
menos que los demás. Y ya que hablo de esto, señora madre, yo quiero
seguir en el ejército, yo quiero que su merced pida al Rey, ¿qué digo al
Rey?, a la Junta, una bandolera.

---Tú no estás destinado a ser militar sino en esta ocasión suprema, en
que la patria necesita de todos sus hijos desde el más alto al más bajo.

---Pero, señora madre, no soy nada y quiero ser algo---insistió el
muchacho, mostrando una energía que nadie hasta entonces le había
conocido.

---¡Que no eres nada!---exclamó la madre con sorpresa primero, después
con cólera, y mirándonos a todos como para preguntarnos si su hijo se
había vuelto loco durante la campaña.

---Yo no soy nada, no soy más que un papamoscas---repuso el chico.---¿De
qué me valen esos papeluchos viejos y esos escudos de armas, si todos se
ríen de mí desde que abro la boca, porque no digo más que necedades?

La marquesa se puso encendida como la grana, y sin decir palabra, miró a
D. Paco, el cual confuso, absorto, aterrado por lo que acababa de oír,
revolvía sus espantados ojos de un lado para otro.

---Este joven---dijo al fin el ayo,---parece que ha perdido el juicio.
Señora, cuando vuelva de cumplir sus deberes de caballero en los campos
de batalla, le haremos que se penetre bien de las máximas contenidas en
la historia de Alejandro el Grande.

Doña María, cuya dignidad no podía consentir que semejante asunto se
tratara delante de personas extrañas, hizo callar a D. Paco, y también
impuso silencio a su hijo con gesto aterrador. Asunción y Presentación,
después de registrar los bolsillos de su hermano, examinaban las
polainas, el sombrero y la charpa, por ver, según dijeron, si aquellas
prendas estaban agujeradas por alguna bala de cañón.

Pero el D. Diego, sintiendo sin duda en su cabeza un hervidero de
palabras, que atropelladamente se le ocurrían conforme a la repentina
fecundidad de su entender, no pudo estar callado mucho tiempo, y habló
para poner en mayores cuidados a la señora de Rumblar. Estábamos, como
he dicho, en una sala baja, donde la condesa había hecho traer para
nuestro regalo un par de zaques, milagrosamente salvados de la rapacidad
francesa. D. Diego, luego que tal vio, volviose a nosotros que
permanecíamos respetuosamente detenidos en la puerta, y con gesto de
campechana confianza, nos dijo:

---Ea, muchachos, entrad todos aquí. ¿Por qué estáis en la puerta? Vaya,
poneos los sombreros, que aquí todos somos iguales, todos somos
compañeros de armas, y lo mismo puede matarme a mí una bala que a
vosotros. Ea, bebamos juntos. ¿Tenéis vergüenza, porque soy noble y
mayorazgo, y vosotros unos pobres hambrones? Fuera necedades; que hoy o
mañana las Juntas quitarán todas esas antiguallas, y entonces cada cual
valdrá según lo que tenga y lo que sepa.

D. Paco se puso verde al oír tales despropósitos, y llevándose la mano
al corazón, miró a la condesa con semblante dolorido y contristado, como
para manifestarla en la sola elocuencia de una mirada que él no había
enseñado tales cosas al joven discípulo. Doña María encerraba su enojo
en lo más hondo del pecho, y aunque harto se le conocían la inquietud y
la ira en el furtivo centellear de sus negros ojos, nada dijo que
comprometiera su dignidad, y deseando que su hijo variase de
conversación, le preguntó si había hecho en Córdoba las visitas a la
señora marquesa de Leiva y su sobrina.

---Sí señora---contestó el rapaz.---Las vi; la señora condesa me dio
muchos dulces, y la marquesa me preguntó si sabía ayudar a misa. Una y
otra me dijeron que la joven con quien está concertado mi matrimonio, se
obstina en no salir del convento, asegurando que antes quería casarse
con Jesucristo que conmigo. ¡Qué ranciedades, señora madre!---añadió con
nuevo arrebato.---Yo quiero seguir en el ejército, yo quiero ir a Madrid
para tratar a la gente que sabe, y a los filósofos, y leer la
\emph{Enciclopedia}, y ver las sociedades secretas, si las hay para
entonces, y aprender lo que no sé, pues D. Paco no me ha enseñado más
que esa sandez de \emph{Por el barandal del cielo}.

El ayo volvió a mirar compungidamente a la condesa, pintando en sus
húmedos ojos la persuasión de que no había instruido al mayorazgo en
tales iniquidades, y doña María reprendió a su hijo con majestad
verdaderamente regia, diciéndole con pausa y aplomo estas amargas
palabras:

---Hijo mío, recordarás que te entregué una espada que fue de tus
abuelos. Honra da al que la ciñe, esa arma antigua; pero también ella la
recibe de las manos de su poseedor, si este es persona que sabe
adquirirla en los campos de batalla. ¿Deshonrarás tú esa espada que
llevó el tatarabuelo de tu padre en el sitio de Maestrich, cuando medio
mundo se llamaba España?

---¡La espada!---exclamó el chico con sorpresa.---Ya no me acordaba de
la dichosa espada. Si ya no la tengo.

---¿Que no la tienes?---preguntó doña María con estupefacción.

---No señora. Si no sirve para nada. Cuando dimos el primer ataque en
Mengíbar, yo saqué mi espada, y a los primeros golpes que di en unas
yerbas observé que no cortaba.

---¡Que no cortaba!

---No señora. Era una hoja mellada, llena de garabatos, letreros, sapos
por aquí, culebras por allí, y cubierta de moho desde la punta a la
empuñadura. ¿Para qué me servía? Como no tenía filo, la cambié por un
sable nuevo que me dio un sargento.

---¡Y diste la espada, la espada!\ldots---exclamó la condesa
levantándose de su asiento.

La señora estaba sublime en su indignación. Parecía la imagen de la
historia levantándose de su sepulcro a pedir cuentas a la generación
contemporánea.

---Sí señora; se la di al sargento---añadió el mozo sacando de la vaina
un sable nuevo, reluciente y de agudísimo filo.---Si aquello no servía
para nada. Muy bonita, eso sí, toda llena de dibujos de plata y oro;
pero, señora madre, si no cortaba\ldots{} si estaba llena de
orín\ldots{} Vea Vd. este sable: no tiene letrero ni cabecitas, ni
garrapatos: pero corta que es un gusto.

Observamos que la condesa dio un paso hacia su hijo; que su semblante
hermosamente venerable se contrajo, desfigurado por la ira; que extendió
sus brazos; que comenzó a balbucir con locución atropellada, cual si su
indignada lengua no acertara a encontrar una palabra bastante dura,
bastante enérgica para tal situación; la vimos después llevarse ambas
manos a la cabeza, retroceder, vacilar, apoyarse en el hombro de D.
Paco, y por último, reponerse, dominarse, erguirse, serenarse, mirar a
su hijo con desdén, señalar a la calle, donde de improviso empezaba a
oírse fuerte redoblar de tambores, y decir:

---El ejército se va. Marcha, corre. Cuando se acabe la guerra te
ajustaremos cuentas. Si eres valiente y vuelves vivo, a palmetazos te
enseñaré quién eres. Pero si eres cobarde, no vuelvas acá.

Salimos a toda prisa, y montando en nuestras cabalgaduras, ocupamos las
filas. Al punto se nos unió Santorcaz. D. Paco no quiso salir a
despedirnos, porque estaba traspasado de dolor, al ver---según dijo
después,---cómo en una semana se torciera al soplo de las malas
compañías el derecho arbolito criado con tanto esmero en el apacible
huerto de sus lecciones.

Las dos muchachas salieron a las ventanas, y nos despedían agitando los
mismos pañuelos con que secaban sus lágrimas. Ninguna de las dos, ni la
destinada al matrimonio, que era, por lo tanto, ignorante, ni la
consagrada al claustro, que era ya medio doctora, habían entendido la
conversación de que he hecho mérito.

Las pobrecillas veían desaparecer un mundo y nacer otro nuevo sin darse
cuenta de ello.

\hypertarget{xxii}{%
\chapter{XXII}\label{xxii}}

Era la madrugada cuando las columnas de vanguardia comenzaron a salir de
Bailén. Mi regimiento debía salir de los últimos, y mientras se puso en
movimiento la artillería y los cuerpos de a pie, estuvimos más de media
hora formados a la salida del pueblo y a mano derecha del camino,
esperando la orden de marcha. Íbamos a Andújar, resueltos a tomar la
ofensiva contra el ejército francés, que al mismo tiempo debía ser
atacado por Castaños, del lado de Marmolejo. ¿Y la división de Vedel,
cuyos movimientos eran la clave de aquel problema estratégico? La
división de Vedel estaba en Andújar el día 16, cuando ocurrió la acción
de Mengíbar, que antes he descrito. Al saber Dupont la derrota de
Ligier-Belair, y la muerte de Gobert, dispuso que Vedel marchase sobre
Bailén, con intención de seguirle él al día siguiente.

Mientras este avanzaba a Andújar, Ligier-Belair, al vernos retirar y
pasar el río, creyó que las tropas de Reding, unidas con las de
Coupigny, intentaban extenderse cautelosamente por la orilla izquierda,
río arriba, tomando el camino de Linares a Guarromán, para ocupar luego
la Carolina y cortar el paso de la sierra. Persuadido de esto, y sin
hacer averiguaciones, emprendió la marcha hacia el Norte, creyendo
anticiparse a lo que creía un rasgo de ingenio estratégico del general
Reding. Llega Vedel a Bailén creyendo encontrarnos, y los franceses que
quedaron allí le dicen: «Quia, los \emph{insurgentes} han repasado el
río y van por Linares a ocupar el paso de la sierra; pero el general
Ligier-Belair, que ha comprendido el juego, ha marchado en seguida a
ocupar a la Carolina, de modo que cuando lleguen los españoles, creyendo
haber hecho un movimiento de primer orden, se lo encontrarán allí».
Vedel oye esto y dice: «Han ido a cortar el paso de la sierra para
impedirnos la retirada y matarnos aquí de hambre y sed. Pues corramos a
la Carolina. Vamos; en marcha». Manda un emisario a Dupont, diciéndole:
«Señor general en jefe, los \emph{insurgentes} han ido a cortar el paso
de la sierra. Corro a la Carolina: venga Vd. tras mí, y acabaremos con
ellos».

Esto pasaba en los días 17 y 18. En tanto los \emph{insurgentes},
replegados a la orilla izquierda, como he dicho, fingíamos un movimiento
hacia Linares; pero en cuanto cerró la noche, los \emph{insurgentes}
caminamos a marchas forzadas hacia Bailén. Por eso en este pueblo nos
decían: «Por aquí pasó Vedel esta mañana en dirección a la Carolina,
para impedirles a Vds. que cortaran el paso de la sierra. ¿No ibais
hacia Linares?»

No; nosotros íbamos a Andújar a atacar a Dupont. En virtud de los
torpísimos movimientos de los generales franceses, una gran parte de la
fuerza imperial corría hacia la sierra, buscando un fantasma. Los
\emph{insurgentes} que ellos creían en marcha hacia la Carolina, estaban
en Bailén, en marcha para Andújar. He aquí la verdadera y exacta
situación de las divisiones españolas y francesas en la noche del 18 al
19 de Julio.

Íbamos a luchar con Dupont, sólo con Dupont. Pero ¿y si Vedel,
conociendo a tiempo su error, retrocedía velozmente para caer de
improviso sobre nuestra espalda durante el combate? Esta funesta
probabilidad estaba compensada con el hecho seguro de que el ejército
francés de Andújar tendría que defenderse al mismo tiempo de nosotros y
de la reserva que le amenazaba del lado de Poniente. De todos modos,
nuestra posición era arriesgada; por lo cual, deseando Reding
cerciorarse de la verdadera distancia a que se hallaba Vedel, camino
arriba había despachado desde Mengíbar al teniente de ingenieros D. José
Jiménez con encargo de averiguarlo. Este valiente oficial, cuyo nombre
no está en la historia, se disfrazó de arriero, y en una fatigosa
jornada supo desempeñar muy bien su comisión, volviendo por la noche a
decir que Vedel había pasado ya más allá de la Carolina.

Así andaban las cosas cuando nos preparábamos a salir de Bailén al
amanecer del 19. Pero no lo habíamos previsto todo; no habíamos previsto
que Dupont, muy receloso de aquella ilusoria ocupación de la sierra por
los insurgentes, había levantado su campo en la misma noche, y
silenciosamente, sofocando los ruidos de su tropa, abandonaba la funesta
y para ellos maldita ciudad de Andújar.

Era cerca de la madrugada cuando nuestros jefes disponían las columnas
para la marcha. Si al comienzo de aquella misma noche, que ya se iba a
extinguir, una mirada humana hubiera podido escudriñar desde la altura
de los cielos lo que pasaba en aquella larga faja de sementeras y
olivares que se extiende a la vera de los montes, entre estos y el
Guadalquivir, habría visto que del oscuro caserío de Andújar se
destacaba cautelosamente, escurriéndose por detrás de las casas una
hilera de hombres y caballos; que esta hilera se iba alargando por la
carretera en interminable procesión, y serpenteaba con lento paso y sin
ruido y sin luces; habría visto cómo se iba extendiendo aquella raya
negra, destacándose a ratos sobre la tierra blanquecina, a ratos
confundiéndose con los oscuros olivos, sin dejar de seguir paso a paso
como si no quisiera ser vista y anhelara apagar en el polvo el ruido de
las cureñas; habría visto que iban delante unos tres mil hombres de
infantería, después un escuadrón de caballos, después seis cañones,
después un número inmenso de carros, tantos, tantos carros, que ocupaban
dos leguas; detrás de los carros nuevos grupos de infantería y muchos
generales; después otros seis cañones, dos regimientos de coraceros,
luego cuatro cañones, y al fin otro grupo de jefes, seguidos de
quinientos hombres de a pie. Esta raya no se detenía en parte alguna, y
avanzaba despacio y con precaución, custodiando sus dos leguas de
convoy. Los hombres que la formaban, mudos y cabizbajos, presagiando sin
duda funestos acontecimientos, dirían para sí: «Llegaremos a la
Carolina, donde ya ha de estar Vedel, y batiendo a los
\emph{insurgentes}, nos abriremos paso por desfiladeros para abandonar
esta tierra maldita, a la cual el Emperador ha tenido la mala ocurrencia
de mandarnos\ldots{} ¡Oh! ¡Cuándo os veremos tierras de la Turenne, del
Poitou, de la Charente, de los Vosgos, del Artois, del Limosin!\ldots»

\hypertarget{xxiii}{%
\chapter{XXIII}\label{xxiii}}

Mientras aguardábamos la salida, nuestras lenguas no estaban ociosas, y
aunque Marijuán me entretenía por un lado con sus donaires y chuscadas,
por el otro era de tanto interés un diálogo entablado entre Santorcaz y
D. Diego, que a las palabras de estos dirigí toda mi atención. No puedo
menos de copiarlo íntegro y tal cual lo oí, por si mis lectores quieren
meditar un poco sobre el mismo tema.

---Lo que me indicó Vd. hace poco---decía Santorcaz,---acerca de que esa
linda joven que se le destina para esposa no quiere salir del convento,
debe tenerle sin cuidado. Esas son gazmoñerías de las muchachas
españolas que, engañadas por su fantasía, se creen enamoradas de
Jesucristo, cuando lo que sienten es verdadera pasión por un ideal
mundano.

---Y si no quiere salir, que no salga---respondió el joven.---Si yo no
la he visto, si yo no comprendo por qué razón he podido pensar en ella
una sola vez.

---¿Pero la quiere Vd.?

---Confesaré a Vd. lo que me pasa. Cuando mi madre me llamó un día, y
después de darme dos palmetazos porque tenía las manos manchadas de
tinta, me dijo que había determinado casarme, sentí mucha alegría, y al
volver a mi cuarto rompí todas las planas de escritura, diciendo a D.
Paco que yo era un hombre y no me daba la gana de obedecerle. A todas
horas pensaba en mi mujercita y en las delicias del matrimonio. Mi madre
escribía cartas y más cartas para concertar mi boda, y cuando yo le
preguntaba con la mayor curiosidad: «Señora madre, ¿cómo va eso?» me
respondía: «Anda a estudiar, mocoso. Ahora con la novelería del
casamiento no coges un libro en la mano». Por fin mi mamá, a fuerza de
cartas lo arregló todo. Cuando fui a Córdoba creí que me la enseñarían;
pero aquellas señoras dijéronme que la discreta joven no quería salir
del convento; y por último, me dieron el medallón que Vd. tiene
guardado. Después la sobrina me regaló unos dulces y su tía un pito para
que fuera pitando por las calles, y en mi segunda y tercera visita pasó
lo mismo, excepto que no me dieron más pitos. Cuando vi el retrato me
gustó tanto la muchacha, que por la calle le iba dando besos, y por la
noche lo acosté conmigo en mi cama. Estoy prendado de ella; mejor dicho,
lo estuve estos días atrás, porque ya, habiendo discurrido sobre la
necedad de prendarme de un retrato, me río de mí mismo y digo: «¡Si de
carne y hueso encontraré tantas, a qué volverme loco por una pintura!»

---Pues no, Sr.~D. Diego---dijo Santorcaz.---Puesto que la señora
condesa le escogió a Vd. esa esposa, sin duda es un gran partido, y Vd.
debe insistir en casarse con ella.

---¿Sí? Pues vaya Vd. a sacarla del convento---añadió Rumblar.---Vamos,
que según me dijeron, no hay quien le hable de otro esposo que
Jesucristo.

---Ya lo he dicho: esas son gazmoñerías de las españolas, por lo general
mujeres nerviosas, muy extremadas en sus pasiones, y dispuestas siempre
a confundir en un mismo sentimiento la voluptuosidad y el misticismo.
Cuidado con las monjitas de quince años, que reniegan del siglo y juran
que han de morir de viejas en el claustro. Yo conocí una joven y linda
novicia que tampoco quería tener más esposo que Jesucristo, y que se
ponía furiosa cuando la hablaban de salir del convento, hasta que un
viernes santo vio a cierto joven al través de la verja del coro. A los
quince días la hermosa novicia abrió por la noche una de las rejas del
convento, y se arrojó a la calle, donde le esperaba su amante y hoy
feliz esposo.

---¡Oh! ¡Bonitísimo suceso!---exclamó con entusiasmo D. Diego.---¡Cuánto
daría porque a mí me pasase uno semejante!

---¿Ella le ha visto a Vd.?

---No.

---Pues en cuanto le vea, apuesto a que la muchacha se apresura a salir
por la puerta, sin exponerse a los peligros de arrojarse por la ventana.
Pero ahora que me ocurre, Sr.~D. Diego, si Vd. en vez de ser un muchacho
apocadito, educado a la antigua y sencillo como un fraile motilón, fuera
un hombre atrevido, arrojado\ldots{} pues\ldots{} como somos todos
aquellos que no hemos recibido la educación de Grandes de España; si Vd.
echara de una vez fuera el cascarón de huevo en que le ha empollado la
ciencia de D. Paco y los mimos de sus hermanitas, ahora podríamos
lanzarnos a una aventura deliciosa.

---¿Cuál, amigo Santorcaz?

---Mire Vd. Después de la batalla y cuando volvamos a Córdoba, sacar a
esa muchacha del convento.

---¿Cómo?

---Demonio, ¿cómo se hacen las cosas? ¡Si viera usted! Eso es muy
divertido. ¿Ve Vd. este rasguño que tengo en la mano derecha? Me lo hice
saltando las tapias de un convento. Son cinco los que he escalado, por
trapicheos con otras tantas novicias y monjas. ¡Ay, Sr.~D. Diego de mi
alma! El recuerdo de estas y otras cosillas es lo que le alegra a uno,
cuando se siente ya en las puertas de la triste vejez.

---Hombre, eso me parece muy bonito---dijo don Diego saltando sobre la
silla.---Pues yo quiero hacer lo mismo, yo quiero rasguñarme saltando
tapias de convento. Con que diga Vd. ¿qué hacemos? ¿Nos entramos de
rondón en el convento y cogiendo a la muchacha me la llevo a mi casa?
Sí: y habrá que pegarle un par de sablazos a alguien, y romper puertas y
apagar luces. Hombre, ¡magnífico! ¡Si dije que usted es el hombre de las
grandes ideas! ¡Qué cosas tan nuevas y tan preciosas me dice! Estoy
entusiasmado, y me parece que antes de venir al ejército era yo un
zoquete. Cabalmente recuerdo que he pensado alguna vez en eso que Vd. me
dice ahora\ldots{} sí\ldots{} allá\ldots{} cuando iba a misa con mi
madre al convento de dominicas.

---Estas cosas, D. Diego, son la vida---añadió Santorcaz;---son la
juventud y la alegría.

---¡Soberbia idea! ¿Conque vamos a buscar a esa muchachuela, mi futura
esposa? ¡Qué preciosa ocurrencia! Verá ella si yo soy hombre que se deja
burlar por niñerías de novicia. Nada, nada, mi esposa tiene que ser
quiera o no quiera. Pero oiga Vd., ¿y si nos descubren los alguaciles y
nos llevan presos?

---Por eso hay que andar con cuidado; pero en ese mismo cuidado, en las
precauciones que es preciso tomar consiste el mayor gusto de la empresa.
Si no hubiera obstáculos y peligros, no valía la pena de intentarla.

---Efectivamente. A mí me gustan los peligros, señor D. Luis. A mí me
gusta todo aquello que no se sabe a dónde va a parar. Siga Vd.
hablándome del mismo asunto. ¿Qué precauciones tomaremos?

---¡Oh! Cuando llegue el caso\ldots{} Yo soy muy corrido en esas cosas.
Ya no estoy para fiestas, es verdad, y por cuenta mía no intentaría
aventuras de esta especie; pero son tan grandes las disposiciones que
descubro en Vd. para ser hombre a la moderna, para ser hombre de ideas
atrevidas y para echar a un lado las ranciedades y rutinas de España,
que volveré a las andadas y entre los dos haremos alguna cosa.

---Pero hombre, ¿cuándo se dará esa batalla, cuándo volveremos a
Córdoba, para enseñarle yo a mi señorita cómo se portan los caballeros
de ideas modernas, que han recibido un desaire de las novias de
Jesucristo? Pero diga Vd. Santorcaz, si perdemos la batalla, si nos
matan\ldots{}

---Todavía no se ha hecho la bala que me ha de matar. Y Vd., ¿qué
presentimientos tiene?

---Creo que tampoco he de morir por ahora. ¡Ay! Si viera Vd., tengo un
fuego dentro de la cabeza; me hierven aquí tantos pensamientos nuevos,
tantas aventuras, tantos proyectos, que se me figura he de vivir lo
necesario para que sepa el mundo que existe un D. Diego Afán de Ribera,
conde de Rumblar.

---¡Bueno, magnífico! Lo mismo era yo cuando niño. Fui después a
Francia, donde aprendí muchísimas cosas que aquí ignoraban hasta los
sabios. Al volver, he encontrado a esta gente un poco menos atrasada.
Parece que hay aquí cierta disposición a las cosas atrevidas y nuevas.
En Madrid se han fundado varias sociedades secretas\ldots{}

---¿Para asaltar conventos?

---No, no son sociedades de enamorados. Si algún día se ocupan de
conventos, será para echar fuera a los frailes y vender luego los
edificios\ldots{}

---Pues yo no los compraría.

---¿Por qué?

---Porque esas casas son de Dios, y el que se las quite se condenará.

---¿Qué es eso de condenarse? Me río de vuestras simplezas. Pues hijo,
adelantado estáis.

---Estemos en paz con Dios---dijo D. Diego.---Por eso creo que antes de
robar del convento a mi novia, debemos confesar y comulgar, diciéndole
al Señor que nos perdone lo que vamos a hacer, pues no es más que una
broma para divertirnos, sin que nos mueva la intención de ofenderle.

Santorcaz rompió a reír desahogadamente.

---¿Conque Vd. es de los que encienden una vela a Dios y otra al diablo?
Robamos a la muchacha, ¿sí o no?

---Sí, y mil veces sí. Ese proyecto me tiene entusiasmado. Y me marcharé
con ella a Madrid; porque yo quiero ir a Madrid. Dicen que allí suele
haber alborotos. ¡Oh! Cuánto deseo ver un alboroto, un motín, cualquier
cosa de esas en que se grita, se corre, se pega. ¿Ha visto Vd. alguno?

---Más de mil.

---Eso debe de ser encantador. Me gustaría a mí verme en un alboroto; me
gustaría gritar con los demás, diciendo: abajo esto o lo otro. ¡Ay!
¡Cómo me alegraba cuando mi señora madre reñía a D. Paco, y este a los
criados, y los criados unos con otros! No pudiendo resistir el alborozo
que esto me causara, iba al corral, ponía cañutillos de pólvora a los
gatos, y encerrándolos en un cuarto con las gallinas, me moría de risa.

Santorcaz, lejos de reír con esta nueva barrabasada de su discípulo,
estaba con la mirada fija en el horizonte, completamente abstraído de
todo y meditando sin duda sobre graves asuntos de su propio interés. No
sé cuál será la opinión que el lector forme de las ideas de aquel
hombre; pero no se les habrá ocultado que sus ingeniosas sugestiones
encerraban segundo intento. El atolondrado rapaz, lanzado a las filas de
un ejército sin tener conocimiento alguno del mundo, con mucha
imaginación, arrebatado temperamento y ningún criterio; igualmente
fascinado por las ideas buenas y las malas con tal que fueran nuevas,
pues todas echaban súbita raíz en su feraz cerebro, acogía con júbilo
las lecciones del astuto amigo; y su lenguaje, su nervioso entusiasmo,
sus planes entre abominables e inocentes, todo anunciaba que don Diego
se disponía a cometer en el mundo mil disparates.

Santorcaz después de permanecer por algunos minutos indiferente a las
preguntas de su discípulo, reanudó la conversación; pero apenas
comenzada esta, oímos un tiro, en seguida otro y luego otro y otro.

\hypertarget{xxiv}{%
\chapter{XXIV}\label{xxiv}}

Todos callamos: detuviéronse las columnas que habían comenzado a
marchar, y desde el primero al último soldado prestamos atención al
tiroteo, que sonaba delante de nosotros a la derecha del camino y a
bastante distancia. Corrieron por las filas opiniones contradictorias
respecto a la causa del hecho. Yo me alzaba sobre los estribos
procurando distinguir algo; pero además de ser la noche oscurísima, las
descargas eran tan lejanas, que no se alcanzaba a ver el fogonazo.

---Nuestras columnas avanzadas---dijo Santorcaz,---habrán encontrado
algún destacamento francés, que viene a reconocer el camino.

---Ha cesado el fuego---dije yo.---¿Echamos a andar? Parece que dan
orden de marcha.

---O yo estoy lelo, o la artillería de la vanguardia ha salido del
camino.

Oyose otra vez el tiroteo, más vivo aún y más cercano; y en la
vanguardia se operaron varios movimientos, cuyas oscilaciones llegaron
hasta nosotros. Sin duda pasaba algo grave, puesto que el ejército todo
se estremeció desde su cabeza hasta su cola. Un largo rato permanecimos
en la mayor ansiedad, pidiéndonos unos a otros noticias de lo que
ocurría; pero en nuestro regimiento no se sabía nada: todos los
generales corrieron hacia la izquierda del camino, y los jefes de los
batallones aguardaban órdenes decisivas del estado mayor. Por último, un
oficial que volvía a escape en dirección a la retaguardia, nos sacó de
dudas, confirmando lo que en todo el ejército no era más que halagüeña
sospecha. ¡Los franceses, los franceses venían a nuestro encuentro!
Teníamos enfrente a Dupont con todo su ejército, cuyas avanzadas
principiaban a escaramucear con las nuestras. Cuando nosotros nos
preparábamos a salir para buscarle en Andújar, llegaba él a Bailén de
paso para la Carolina, donde creía encontrarnos. De improviso unos
cuantos tiros les sorprenden a ellos tanto como a nosotros: detienen el
paso; extendemos nosotros la vista con ansiedad y recelo en la oscura
noche; todos ponemos atento el oído, y al fin nos reconocemos, sin
vernos, porque el corazón a unos y otros nos dice: «Ahí están».

Cuando no quedó duda de que teníamos enfrente al enemigo, el ejército se
sintió al pronto electrizado por cierto religioso entusiasmo. Algunos
vivas y mueras sonaron en las filas, pero al poco rato todo calló. Los
ejércitos tienen momentos de entusiasmo y momentos de meditación:
nosotros meditábamos.

Sin embargo, no tardó en producirse fuertísimo ruido. Los generales
empezaron a señalar posiciones. Todas las tropas que aún permanecían en
las calles del pueblo, salieron más que de prisa, y la caballería fue
sacada de la carretera por el lado derecho. Corrimos un rato por terreno
de ligera pendiente; bajamos después, volvimos a subir, y al fin se nos
mandó hacer alto. Nada se veía, ni el terreno ni el enemigo: únicamente
distinguíamos desde nuestra posición los movimientos de la artillería
española, que avanzaba por la carretera con bastante presteza. Entonces
sentimos camino abajo, y como a distancia de tres cuartos de legua, un
nuevo tiroteo que cesó al poco rato, reproduciéndose después a mayor
distancia. Las avanzadas francesas retrocedían, y Dupont tomaba
posiciones.

---¿Qué hora es?---nos preguntábamos unos a otros, anhelando que un rayo
de sol alumbrase el terreno en que íbamos a combatir.

No veíamos nada, a no ser vagas formas del suelo a lo lejos; y las
manchas de olivos nos parecían gigantes, y las lomas de los cerros el
perfil de un gigantesco convoy. Un accidente noté que prestaba extraña
tristeza a la situación: era el canto de los gallos que se oía a lo
lejos, anunciando la aurora. Nunca he escuchado un sonido que tan
profundamente me conmoviera como aquella voz de los vigilantes del
hogar, desgañitándose por llamar al hombre a la guerra.

Nuevamente se nos hizo cambiar de posición, llevándonos más adelante a
espaldas de una batería, y flanqueados por una columna de tropa de
línea. Gran parte de la caballería fue trasladada al lado izquierdo;
pero a mí con el regimiento de Farnesio me tocó permanecer en el ala
derecha.

De repente una granada visitó con estruendo nuestro campo, reventando
hacia la izquierda por donde estaban los generales. Era como un saludo
de cortesanía entre dos guerreros que se van a matar, un tanteo de
fuerzas, una bravata echada al aire para explorar el ánimo del
contrario. Nuestra artillería, poco amiga de fanfarronadas, calló. Sin
embargo, los franceses, ansiando tomar la ofensiva, con ánimo de
aterrarnos, acometieron a una columna de la vanguardia que se destacaba
para ocupar una altura, y la lóbrega noche se iluminó con relampagueo
horroroso, que interrumpiéndose luego, volvió a encenderse al poco rato
en la misma dirección.

Por último, aquellas tinieblas en que se habían cruzado los resplandores
de los primeros tiros, comenzaron a disiparse; vislumbramos las
recortaduras de los cerros lejanos, de aquel suave e inmóvil oleaje de
tierra, semejante a un mar de fango, petrificado en el apogeo de sus
tempestades; principiamos a distinguir el ondular de la carretera,
blanqueada por su propio polvo, y las masas negras del ejército,
diseminado en columnas y en líneas; empezamos a ver la azulada masa de
los olivares en el fondo y a mano derecha; y a la izquierda las colinas
que iban descendiendo hacia el río. Una débil y blanquecina claridad
azuló el cielo antes negro. Volviendo atrás nuestros ojos, vimos la
irradiación de la aurora, un resplandecimiento que surgía detrás de las
montañas; y mirándonos después unos a otros, nos vimos, nos reconocimos,
observamos claramente a los de la segunda fila, a los de la tercera, a
los de más allá, y nos encontramos con las mismas caras del día
anterior. La claridad aumentaba por grados, distinguíamos los rastrojos,
las yerbas agostadas, y después las bayonetas de la infantería, las
bocas de los cañones, y allá a lo lejos las masas enemigas, moviéndose
sin cesar de derecha a izquierda. Volvieron a cantar los gallos. La luz,
única cosa que faltaba para dar la batalla, había llegado, y con la
presencia del gran testigo, todo era completo.

Ya se podía conocer perfectamente el campo. Prestad atención, y sabréis
cómo era. El centro de la fuerza española ocupaba la carretera con la
espalda hacia Bailén, de allí poco distante: a la derecha del camino por
nuestra parte se alzaban unas pequeñas lomas, que a lo lejos subían
lentamente hasta confundirse con los primeros estribos de la sierra: a
la izquierda también había un cerro; pero este cerro caía después en la
margen del río Guadiel, casi seco en verano, y que emboca en el
Guadalquivir cerca de Espelúy. Ocupaba el centro a un lado y otro del
camino una poderosa batería de cañones, apoyada por considerables
fuerzas de infantería: a la izquierda estaba Coupigny con los
regimientos de Bujalance, Ciudad-Real, Trujillo, Cuenca, Zapadores y la
caballería de España; y a la derecha estábamos además de la caballería
de Farnesio, los tercios de Tejas, los suizos, los walones, el
regimiento de Órdenes, el de Jaén, Irlanda y voluntarios de Utrera.
Mandábanos el brigadier D. Pedro Grimarest. Los franceses ocupaban la
carretera por la dirección de Andújar, y tenían su principal punto de
apoyo en un espeso olivar situado frente a nuestra derecha, y que por
consiguiente servía de resguardo a su ala izquierda. Asimismo ocupaban
los cerros del lado opuesto con numerosa infantería y un regimiento de
coraceros, y a su espalda tenían el arroyo de Herrumblar, también seco
en verano, que habían pasado. Tal era la situación de los dos ejércitos,
cuando la primera luz nos permitió vernos las caras. Creo que entrambos
nos encontramos respectivamente muy feos.

---¿Qué le parece a Vd. esta aventura, Sr.~D. Diego?---dijo Santorcaz.

---Estoy entusiasmado---repuso el mozuelo,---y deseo que nos manden
cargar sobre las filas francesas. ¡Y mi señora madre empeñada en que
conservara aquella espada vieja sin filo ni punta!\ldots{}

---¿Está usía sereno?---le preguntó Marijuán.

---Tan sereno que no me cambiaría por el emperador Napoleón---repuso el
conde.---Yo sé que no me puede pasar nada, porque llevo el escapulario
de la Virgen de Araceli que me dieron mis hermanitas, con lo cual dicho
se está que me puedo poner delante de un cañón. ¿Y Vd., Sr.~de
Santorcaz, está sereno?

---¿Yo?---repuso D. Luis con cierta tristeza.---Ya sabe Vd. que he
estado en Hollabrünn, en Austerlitz y en Jena.

---Pues entonces\ldots{}

---Por lo mismo que he estado en tan terribles acciones de guerra, tengo
miedo.

---¡Miedo! Pues fuera de la fila. Aquí no se quiere gente medrosa.

---Todos los soldados aguerridos---dijo Santorcaz,---tienen miedo al
empezar la batalla, por lo mismo que saben lo que es.

Oído esto, casi todos los bisoños que poco antes reíamos a carcajada
tendida, saludándonos con bravatas y dicharachos, conforme a la guerrera
exaltación de que estábammos poseídos, callamos, mirándonos unos a
otros, para cerciorarse cada cual de que no era él solo quien tenía
miedo.

---¿Sabéis lo que dijo mi señora madre que hiciera al comenzar la
batalla?---indicó Rumblar.---Pues me dijo que rezara un Ave-María con
toda devoción. Ha llegado el momento. \emph{Dios te salve, María\ldots,
etc}.

El mayorazguito continuó en voz baja el Ave-María que había empezado en
alta voz, y todos los que estaban en la fila le imitaron, como si
aquello en vez de escuadrón fuera un coro de religioso rezo; y lo más
extraño fue que Santorcaz, poniéndose pálido, cerrando los ojos, y
quitándose el sombrero con humilde gesto, dijo también \emph{Santa
María}\ldots{}

Aún resonaba en el aire aquella fervorosa invocación, cuando un
estruendo formidable retumbó en las avanzadas de ambos ejércitos. Las
columnas francesas del ala derecha se desplegaron en línea y rompieron
el fuego contra nuestra izquierda.

\hypertarget{xxv}{%
\chapter{XXV}\label{xxv}}

He empleado mucho tiempo en describir la posición de los ejércitos, la
configuración del terreno y el principio del ataque; pero no necesito
advertir que todo esto pasó en menos tiempo del empleado por mi tarda
pluma en contarlo. Nuestras fuerzas no estaban convenientemente
distribuidas cuando tuvo lugar la primera embestida de los imperiales.
Verificada esta, no pueden Vds. figurarse qué precipitados movimientos
hubo en el centro del ejército español. Las de retaguardia, que aún
llenaban la carretera, corrían velozmente a sostener la izquierda: los
cañones ocupaban su puesto; todo era atropellarse y correr, de tal modo,
que por un instante pareció que el primer ataque de los franceses había
producido confusión y pánico en las filas de Coupigny. En tanto, los de
la derecha permanecíamos quietos, y los de a caballo que ocupábamos
parte de la altura, podíamos ver perfectamente los movimientos del
combate, que en lugar más bajo y a bastante distancia se había acabado
de trabar.

Tras las primeras descargas de las líneas francesas, estas se
replegaron, y avanzando la artillería disparó varios tiros a bala rasa.
Ellos ponían en ejecución su táctica propia, consistente en atacar con
mucha energía sobre el punto que juzgaban más débil, para desconcertar
al enemigo desde los primeros momentos. Algo de esto lograron al
principio; pero nosotros teníamos una excelente artillería, y disparando
también con bala rasa las seis piezas puestas en la carretera y a sus
flancos, el centro francés se resintió al instante, y para reforzarle,
tuvo que replegar su ala derecha, produciendo esto un pequeño avance de
la división de Coupigny. Entretanto, todos teníamos fija la vista en el
otro extremo de la línea y hacia la carretera, y olvidábamos la espesura
del olivar que estaba delante. De pronto, las columnas ocultas entre los
árboles salieron y se desplegaron, arrojando un diluvio de balas sobre
el frente del ala derecha. Desde entonces, el fuego, corriéndose de un
extremo a otro, se hizo general en el frente de ambos ejércitos. La
caballería, brazo de los momentos terribles, no funcionaba aún y
permanecía detrás, quieta y relinchante, conteniéndose con sus propias
riendas.

Pero a pesar de generalizarse la lucha, en aquel primer período de la
batalla todo el interés continuaba, como he dicho, en el ala izquierda.
Atacada por los franceses con una valentía pasmosa, nuestros batallones
de línea retrocedieron un momento. Casi parecía que iban a abandonar su
posición al enemigo; pero bien pronto se repusieron tomando la ofensiva
al amparo de dos bocas de fuego y de la caballería de España, que cargó
a los franceses por el flanco. Vacilaron un tanto los imperiales de
aquella ala, y gran parte de las fuerzas que habían salido del olivar se
transportaron al otro lado. Su artillería hizo grandes estragos en
nuestra gente; mas con tanta intrepidez se lanzó esta sobre las lomas
que ocupaba el enemigo entre el camino y el río Guadiel; con tanta
bravura y desprecio de la vida afrontaron los soldados de línea la
mortífera bala rasa y las cargas de la caballería del general Privé, que
llegaron a dominar tan fuerte posición.

Antes que esto se verificara ocurrieron mil lances de esos que ponen a
cada minuto en duda el éxito de una batalla. Se clareaban nuestras
líneas, especialmente las formadas con voluntarios; volvían a verse
compactas y formidables, avanzando como una muralla de carne; oscilaban
después y parecían resbalar por la pendiente cuando las patas delanteras
de los caballos de los coraceros principiaban a martillar sobre los
pechos de nuestros soldados; luego estos rechazaban a los animales con
sus haces de bayonetas; caían para levantarse con frenético ardor o no
levantarse nunca, hasta que, por último, el ala francesa se puso en
dispersión, replegándose hacia la carretera.

Mientras esto pasaba, los de la derecha se sostenían a la defensiva, y
el centro cañoneaba para mantener en respeto al enemigo, porque casi
gran parte de la fuerza había acudido a la izquierda; pero una vez que
se oyeron los gritos de júbilo de los soldados de esta, posesionados de
la altura, antes en poder de los franceses, y cuando se vio a estos
aglomerarse sobre su centro, diose orden de avance a las seis piezas del
nuestro, y por un instante el pánico y desorden del enemigo fueron
extraordinarios. Para concertarse de nuevo y formar otra vez sus
columnas tuvieron que retroceder al otro lado del puente del Herrumblar.
Viéndoles en mal estado, se trató de lanzar toda la caballería en su
persecución; pero varias de sus piezas, desmontadas por nuestras balas,
obstruían el camino, también entorpecido con los espaldones que habían
empezado a formar. El sol esparcía ya sus rayos por el horizonte.
Nuestros cuerpos proyectaban en la tierra y hacia adelante larguísimas
sombras negras. Cada animal, con su jinete, dibujaba en el suelo una
caricatura de hombre y caballo, escueta, enjuta, disparatada, y todo el
suelo estaba lleno de aquellas absurdas legiones de sombras que harían
reír a un chico de escuela.

Ustedes se reirán de verme ocupado en tan triviales observaciones; pero
así era, y no tengo por qué ocultarlo. En aquel momento estábamos en una
pequeña tregua, aunque la cosa no pareciera muy próxima a concluir.
Hasta entonces sólo habíamos sido atacados por una parte de las fuerzas
enemigas, pues la división de Barbou, algo rezagada, no estaba aún en el
campo francés. Entretanto, y mientras se tomaban disposiciones para
rechazar un segundo ataque, que no sabíamos si sería por la derecha o
por el centro, retiraban los españoles sus heridos, que no eran pocos,
mas no ciertamente en mi división, la cual estuviera hasta entonces a la
defensiva, tiroteándose ambos frentes a alguna distancia. Mi regimiento
permanecía aún intacto y reservado para alguna ocasión solemne.

Los franceses no tardaron en intentar la adquisición del puente perdido.
Su primer ataque fue débil, pero el segundo violentísimo. Oíd cómo fue
el primero. La infantería española, desplegándose en guerrillas a un
lado y a otro del camino, les azotaba con espeso tiroteo. Lanzaron ellos
sus caballos por el puente; pero con tan poca fortuna, que tras de una
pequeña ventaja obtenida por el empuje de aquella poderosa fuerza,
tuvieron que retirarse, porque pasada la sorpresa, nuestros infantes les
acribillaron a bayonetazos, dejando un sinnúmero de jinetes en el suelo
y otros precipitados por sobre los pretiles al lecho del arroyo. No
tuvimos tan buena suerte en el segundo ataque, porque renunciando ellos
a poner en movimiento la caballería en lugar angosto, atacaron a la
bayoneta con tanta fiereza, que nuestros regimientos de línea, y aun los
valientes walones y suizos, retrocedieron aterrados. Yo oí contar en la
tarde de aquel mismo día a un soldado de los tiradores de Utrera,
presente en aquel lance, que los franceses, en su mayor parte militares
viejos, cargaron a la bayoneta con una furia sublime, que producía en
los nuestros, además del desastre físico, una gran inferioridad moral.
Me dijo que se espantaron, que en un momento viéronse pequeños, mientras
que los franceses se agrandaban, presentándose como una falange de
millones de hombres; que los vivas al Emperador y los gritos de cólera
eran tan furiosamente pronunciados, que parecían matar también por el
solo efecto del sonido; y que, por último, sintiendo los de acá
desfallecer su entusiasmo y al mismo tiempo un repentino e invencible
cariño a la vida, abandonaron aquel puente mezquino, ardientemente
disputado por dos Naciones, y que al fin quedó por Francia. El efecto
moral de esta pérdida fue muy notable entre nosotros. Advirtiose
claramente en todo el ejército como un estremecimiento de desasosiego,
como una inquietud que, partiendo de aquel gran corazón compuesto de
diez y ocho mil corazones, se transmitía a la temblorosa bayoneta, asida
por la indecisa mano.

Entonces pude observar cómo se individualiza un ejército, cómo se hace
de tantos uno solo, resumiendo de un modo milagroso los sentimientos lo
mismo que se resume la fuerza; pude observar cómo aquella gran masa
recibe y transmite las impresiones del combate con la presteza y
uniformidad de un solo sistema nervioso; cómo todos los movimientos del
organismo físico, desde la mano del general en jefe hasta la pezuña del
último caballo, obedecen a la alegría de un momento, a la pena de otro
momento, a las angustiosas alternativas que en el discurso de cuantas
horas consiente y dispone Dios, espectador no indiferente de estas
barbaridades de los hombres.

La pérdida del puente sobre el Herrumblar, que al amanecer se había
ganado, hizo que el ala derecha retrocediera buscando mejor posición.
Casi todas las posiciones se variaron. Los generales conocían la
inminencia de un ataque terrible, los soldados viejos la preveían, los
bisoños la sospechábamos, y nuestros caballos, reculando y estrechándose
unos contra otros, olían en el espacio, digámoslo así, la proximidad de
una gran carnicería.

Eran las seis de la mañana y el calor principiaba a hacerse sentir con
mucha fuerza. Comenzamos a sentir en las espaldas aquel fuego que más
tarde había de hacernos el efecto de tener por médula espinal una barra
de metal fundido. No habíamos probado cosa alguna desde la noche
anterior, y una parte del ejército, ni aun en la noche anterior había
comido nada. Pero este malestar era insignificante comparado con otro
que desde la mañana principió a atormentarnos, la sed, que todo lo
destruye; alma y cuerpo, infundiendo una rabia inútil para la guerra,
porque no se sacia matando. Es verdad que desde Bailén salían en
bandadas multitud de mujeres con cántaros de agua para refrescarnos;
pero de este socorro apenas podía participar una pequeña parte de la
tropa, porque los que estaban en el frente no tenían tiempo para ello.
Algunas veces aquellas valerosas mujeres se exponían al fuego,
penetrando en los sitios de mayor peligro, y llevaban sus alcarrazas a
los artilleros del centro. En los puntos de mayor peligro, y donde era
preciso estar con el arma en el puño constantemente, nos disputábamos un
chorro de agua con atropellada brutalidad: rompíanse los cántaros al
choque de veinte manos que los querían coger, caía el agua al suelo, y
la tierra, más sedienta aún que los hombres, se la chupaba en un
segundo.

\hypertarget{xxvi}{%
\chapter{XXVI}\label{xxvi}}

¿Por qué sitio pensaban atacarnos los franceses? Conociendo que el
centro era inexpugnable por entonces; siendo el principal objeto de
Dupont abrirse camino hacia Bailén, y considerando que era peligroso
intentarlo por el ala izquierda, no sólo porque allí la posición de los
españoles era excelente, sino porque les ofrecía un gran peligro la
cuenca del Guadiel, determinaron atacar nuestra ala derecha, esperando
abrir en ella un boquete que les diera paso. Su artillería no cesaba de
arrojar bala rasa, protegiendo la formación de las poderosas columnas
que bien pronto debían hostilizarnos. Al punto se reforzó el ala
derecha, se desplegaron en línea varios batallones y sin esperar el
ataque marcharon hacia el enemigo, amparados por dos piezas de
artillería. El primer momento nos fue favorable. Pero el olivar vomitó
gente y más gente sobre nuestra infantería. Por un instante confundidas
ambas líneas en densa nube de polvo y humo, no se podía saber cuál
llevaba ventaja. Caían los nuestros sobre los imperiales, y la metralla
enemiga les hacía retroceder; avanzaban ellos y adquiríamos a nuestra
vez momentánea inferioridad.

Por largo tiempo duró este combate, tanto más cruel, cuanto era más
proporcionado el empuje de una y otra parte, hasta que al fin observamos
síntomas de confusión en nuestras filas; vimos que se quebraban aquellas
compactas líneas, que retrocedían sin orden, que chocaban unos con otros
los grupos de soldados. La división se conmovió toda, y dos batallones
de reserva avanzaron para restablecer el orden. Gritaban los jefes hasta
perder la voz, y todos se ponían a la cabeza de las columnas,
conteniendo a los que flaqueaban y excitando con ardorosas palabras a
los más valientes. Los tercios de Tejas y el regimiento de Órdenes se
lanzaron al frente, mientras se restablecía el concierto en los cuerpos
que hasta entonces habían sostenido el fuego. Sobre todo, el regimiento
de Órdenes, uno de los más valientes del ejército, se arrojó sobre el
enemigo con una impavidez que a todos nos dejó conmovidos de entusiasmo.
Su coronel D. Francisco de Paula Soler, parecía dar fuego a todos los
fusiles con la arrebatadora llama de sus ojos, con el gesto de su mano
derecha empuñando la espada que parecía un rayo, con sus gritos que
sobresalían entre el granizado tiroteo, sublimando a los soldados.

La metralla y la fusilería enemiga se recrudecieron de tal modo, que
casi toda la primera fila del valiente regimiento de Órdenes cayó, cual
si una gigantesca hoz la segara. Pero sobre los cuerpos palpitantes de
la primera fila pasó la segunda, continuando el fuego. Como si los tiros
franceses persiguieran con inteligente saña las charreteras, el
regimiento vio desaparecer a muchos de sus oficiales.

Reforzáronse también los imperiales, y desplegando nueva línea con gente
de reserva, avanzaron a la bayoneta, pujantes, aterradores,
irresistibles. ¡Momento de incomparable horror! Figurábaseme ver a dos
monstruos que se baten mordiéndose con rabia, igualmente fuertes y que
hallan en sus heridas, en vez de cansancio y muerte, nueva cólera para
seguir luchando.

Cuando las bayonetas se cruzaban, el campo ocupado por nuestra
infantería se clareó a trozos; sentimos el crujido de poderosas cureñas
rebotando en el suelo de hoyo en hoyo al arrastre de las mulas
castigadas sin piedad; los cañones de a 12 enfilaron el eje de sus
ánimas hacia las líneas enemigas; los botes de metralla penetraron en el
bronce, se atacaron con prontitud febril, y un diluvio de puntas de
hierro, hendiendo horizontalmente el aire, contuvo la marcha del frente
francés. A un disparo se sucedía otro: la infantería, rehecha,
flanqueaba los cañones, y para completar el acto de desesperación, un
grito resonó en nuestro regimiento. Todos los caballos patalearon,
expresando en su ignoto lenguaje que comprendían la sublimidad del
momento; apretamos con fuerte puño los sables, y medimos la tierra que
se extendía delante de nosotros. La caballería iba a cargar.

Vimos que a todo escape se nos acercó un general, seguido de gran número
de oficiales. Era el marqués de Coupigny, alto, fuerte, rubio, colorado
de suyo, y en aquella ocasión encendido, como si toda su cara despidiera
fuego. Era Coupigny hombre de pocas palabras; pero suplía su escasez
oratoria con la llama de su mirar, que era por sí una proclama. Nosotros
pusimos atención esperando que nos dijera alguna cosa; pero el general
dispuso con un gesto la dirección del movimiento, y después nos miró. No
necesitamos más.

«¡Viva España! ¡Viva el Rey Fernando! ¡Mueran los franceses!» exclamamos
todos, y el escuadrón se puso en movimiento.

Estábamos formados en columna, y nos desplegamos en batalla sobre los
costados, bajando a buen paso, pero sin precipitación, de la altura
donde habíamos estado. Maniobramos luego para tener a nuestro frente el
flanco enemigo; las tropas que por allí atacaban dicho flanco doblaron
por cuartas para darnos paso por los claros; el jefe gritó: «A la
carga»; picamos espuela, y ciegamente caímos sobre el enemigo como
repentina avalancha. Yo, lo mismo que Santorcaz, el mayorazgo y los
demás de la partida, íbamos en la segunda fila. Penetraron
impetuosamente los de la primera, acuchillando sin piedad; los caballos
bramaban de furor, sintiéndose heridos a fuego y a hierro. Algunos
caían, dejando morir a sus jinetes, y otros se arrojaban con más fuerza
destrozando cuanto hallaban bajo sus poderosas manos. Los de la primera
fila hicieron gran destrozo; pero a los de la segunda nos costó más
trabajo, porque avanzando demasiado los delanteros, quedamos envueltos
por la infantería, lo cual atenuaba un poco nuestra superioridad. Sin
embargo, destrozábamos pechos y cráneos sin piedad.

Yo vi a Rumblar, ciego de ira, luchando cuerpo a cuerpo con un francés;
vi a Santorcaz dando pruebas de tener un puño formidable para el manejo
del sable; uselo yo mismo con toda la destreza que me era posible, y lo
mismo yo que mis amigos y otros muchos jinetes de mi fila nos internamos
locamente por el grueso de la infantería contraria. Otro escuadrón daba
nueva carga por el mismo flanco, lo cual, observado por nosotros, nos
reanimó. No íbamos mal; pero los franceses eran muchos, estaban muy
hechos a tales embestidas y sabían defenderse bien de la pesadumbre de
los caballos, así como de los sablazos.

Sin embargo, no retrocedían delante de nosotros. Ya se sabe que siendo
el objeto de la caballería producir un gran sacudimiento y pavor en las
filas enemigas por la violencia del primer choque, cuando este no da
aquellos resultados y se empeñan combates parciales entre los caballos y
una numerosa infantería, los primeros corren gran riesgo de desaparecer,
brutales masas devoradas en aquel hervidero de agilidad y de destreza.
Aunque en la carga les hicimos gran daño, no les pusimos en dispersión:
los combates parciales se entablaron pronto, y fue preciso que la
caballería de España, a escape traída del ala izquierda; nos reforzase,
para no ser envueltos y perdidos sin remedio. Hubo un momento en que me
vi próximo a la muerte. A mi lado no había más que dos o tres jinetes,
que se hallaban en trance tan apurado como yo: nos miramos, y
comprendiendo que era preciso hacer un supremo esfuerzo, arremetimos a
sablazos con bastante fortuna. Con esto y el pronto auxilio de la carga
hecha en el mismo instante por la caballería de España, salimos del
apuro. Revolviendo atrás, hundí las espuelas, y mi caballo de un salto
se puso en la nueva fila. No vi a mi lado más cara conocida que la de
Marijuán. El conde y Santorcaz habían desaparecido.

En el mismo instante mi caballo flaqueó de sus cuartos traseros. Intenté
hacerle avanzar, clavándole impíamente las espuelas: el noble animal,
comprendiendo sin duda la inmensidad de su deber y tratando de
sobreponerle a la agudeza de su dolor, dio algunos botes; pero cayó al
fin escarbando la tierra con furia. El desgraciado había recibido una
violenta herida en el vientre, y falto de palabra para expresar su
padecimiento, bramaba, aspirando con ansia el aire inflamado, sacudía el
cuello, parecía dar a entender que hallando un charco de agua en que
remojar la lengua sus dolores serían menos vivos, y al fin se abandonó a
su suerte, tendiéndose sobre el campo, indiferente al ruido del cañón y
al toque de degüello.

\hypertarget{xxvii}{%
\chapter{XXVII}\label{xxvii}}

Hallándome desmontado, me dirigí a buscar un puesto entre las escoltas
de la artillería o en el servicio de municiones que se hacía
precipitadamente por los tambores entre los carros y las piezas. Al dar
los primeros pasos, advertí el extraordinario decaimiento de mis fuerzas
físicas; no podía tenerme en pie, y el ardor de mi sangre llegado a su
último extremo, me paralizaba cual si estuviese enfermo. No es propio
decir que hacía calor, porque esta frase común al verano de todos los
países europeos es inexpresiva para indicar la espantosa inflamación de
aquella atmósfera de Andalucía en el día infernal que presenció la
batalla de Bailén. El efecto que hacía en nuestros cuerpos era el de una
llamarada que los azotaba por todos lados: la cara se nos abrasaba como
cuando nos asomamos a un horno encendido, y deshechos en sudor, nuestros
cuerpos hervían, descomponiéndose la economía entera, desde el instante
en que fuertes excitaciones del espíritu dejaban de sostenerla.

Cuando me encontré a pie y a alguna distancia del combate, que seguía
con ventaja para los españoles, empecé a sentir vivamente y de un modo
irresistible el aguijón candente de la sed que horadaba mi lengua, y la
corriente de fuego que envolvía mi cuerpo. Esto me daba tal
desesperación, que de prolongarse mucho hubiérame impelido a beber la
sangre de mis propias venas. Por ninguna parte alcanzaba a ver la gente
del pueblo que antes trajera cántaros con agua, y al buscar con ansiosa
inspiración en el seco aire una partícula de agua, bebía y respiraba
oleadas de polvo abrasador.

Por un rato perdí la exaltación guerrera y el furor patriótico que antes
me dominaban, para no pensar más que en la probabilidad de beber,
previendo las delicias de un sorbo de agua, y anhelando apagar aquellas
ascuas pegajosas que revolvía en mi boca. Con este deseo caminé largo
trecho ante las filas de retaguardia del centro: los soldados de los
regimientos que allí se rehacían para salir de nuevo al frente, clamaban
también pidiendo agua. Vimos con alegría que desde el pueblo venían
corriendo algunos soldados con cubos; pero al punto se nos dijo que
aquella agua no era para nosotros; era para otros sedientos, cuyas bocas
necesitaban refrescarse antes que las nuestras, si el combate había de
tener buen éxito; era para los cañones.

La resistencia enérgica de las dos piezas del ala derecha, combinadas
con las seis de la batería central, y el auxilio de la caballería
atacando por el flanco la línea enemiga, hizo que esta fuese rechazada,
a pesar de su frente compacto e incomparable bravura. Los franceses se
retiraron, dejándose perseguir y desposicionar por la infantería y
caballos de nuestra derecha. Harto se conocía este resultado en los
gritos de alegría, en aquel concierto de injurias con que el vencedor
confirma la catástrofe del vencido, cuando este vuelve la espalda. El
sitio donde yo estaba se vio despejado por el avance de nuestras tropas,
y en casi todos los jefes que allí había observé tal expresión de gozo
que sin duda consideraban asegurada la victoria. ¡Oh momento feliz! Ya
se podía pensar en beber. ¿Pero dónde?

Después del avance de nuestras tropas, que no ocuparon enteramente las
posiciones francesas por ofrecer esto algún peligro, los soldados del
regimiento de Órdenes divisaron una noria, en el momento en que los
franceses que durante la acción la habían ocupado se hallaban en el caso
de abandonarla. Vieron todos aquel lugar como un santuario cuya
conquista era el supremo galardón de la victoria, y se arrojaron sobre
los defensores del agua escasa y corrompida que arrojaban unos cuantos
arcaduces en un estanquillo. Los enemigos, que no querían desprenderse
de aquel tesoro, le defendían con la rabia del sediento. Apenas
disparados los primeros tiros, otros muchos franceses, extenuados de
fatiga, y encontrándose ya sin fuerzas para combatir si no les caía del
cielo o les brotaba de la tierra una gota de agua, acudieron a beber, y
viéndola tan reciamente disputada, se unieron a los defensores.

Yo oí decir: «¡Allí hay agua, allí se están disputando la noria!» y no
necesité más. Lanceme y conmigo se lanzaron otros en aquella dirección;
tomé del suelo un fusil que aún apretaba en sus manos un soldado muerto,
y corrí con los demás a todo escape en dirección a la noria. Penetramos
en un campo a medio segar, a trechos cubierto de altos trigos secos, a
trechos en rastrojo. La lucha en la noria se hacía en guerrillas;
acerqueme a la que me pareció más floja, y desprecié la vida lleno mi
espíritu del frenético afán de conquistar un buche de agua. Aquel
imperio compuesto de dos mal engranadas ruedas de madera, por las cuales
se escurría un miserable lagrimeo de agua turbia, era para nosotros el
imperio del mundo. La hidrofagia, que a veces amilana, a ratos también
convierte al hombre en fiera, llevándole con sublime ardor a desangrarse
por no quemarse.

Los franceses defendían su vaso de agua, y nosotros se lo disputábamos;
pero de improviso sentimos que se duplicaba el calor a nuestras
espaldas. Mirando atrás, vimos que las secas espigas ardían como yesca,
inflamadas por algunos cartuchos caídos por allí, y sus terribles
llamaradas nos freían de lejos la espalda. «O tomar la noria o morir»,
pensamos todos. Nos batíamos apoyados contra una hoguera, y la
hambrienta llama, al morder con su diente insaciable en aquel pasto,
extendía alguna de sus lenguas de fuego azotándonos la cara. La
desesperación nos hizo redoblar el esfuerzo porque nos asábamos,
literalmente hablando; y por último, arrojándonos sobre el enemigo
resueltos a morir, la gota de agua quedó por España al grito de «¡Viva
Fernando VII!»

Por un momento dejamos de ser soldados, dejamos de ser hombres, para no
ser sino animales. Si cuando sumergimos nuestras bocas en el agua,
hubiera venido un solo francés con un látigo, nos habría azotado a
todos, sin que intentáramos defendernos. Después de emborracharnos en
aquel néctar fangoso, superior al vino de los dioses, nos reconocimos
otra vez en la plenitud de nuestras facultades. ¡Qué inmensa alegría!,
¡qué rebosamiento de fuerza y de orgullo!

¿Pero habíamos vencido definitivamente a los franceses? Cuando se disipó
aquella lobreguez moral con que la horrible sequedad del cuerpo había
envuelto el espíritu, nos vimos en situación muy difícil. Corriendo
hacia la noria nos habíamos apartado de nuestro campo, y adviértase que
si el ejército francés fue rechazado con grandes pérdidas, conservaba
aún sus posiciones. ¿Iba a emprenderse nuevo ataque, haciendo el último
esfuerzo de la desesperación? Creíamos que sí, y señales de esto notamos
en el campo enemigo que teníamos tan cerca. Al punto corrimos
desbandamente hacia el nuestro, que estaba algo lejos, y saltando por
junto a los trigos incendiados, abandonamos la noria, por temor a que
fuerzas más numerosas que las nuestras nos hicieran prisioneros.

Verdad que los franceses, no dando ya ninguna importancia a las acciones
parciales, se ocupaban en organizar el resto y lo mejor de su fuerza
para dar un golpe de mano, última estocada del gigante que se sentía
morir. Corrimos, pues, hacia nuestro campo. Ya cerca de él, pasó
rápidamente por delante de mí un caballo sin jinete, arrogante,
vanaglorioso, con la crin al aire, entero y sin heridas, algo azorado y
aturdido. Era un animal de pura casta cordobesa, lo mismo que el mío. Le
seguí, y apoderándome de sus bridas, cuando volvía me monté en él:
después de ser por un rato soldado de a pie, tornaba a ser jinete.
Busqué con la vista el escuadrón más próximo, y vi que a retaguardia del
centro se formaba en columna con distancias el de España. Entré en las
primeras filas, a punto que dijeron junto a mí:

---Los generales franceses van a hacer el último esfuerzo. Dicen que hay
unas tropas que todavía no han entrado en fuego, y son las mejores que
Napoleón ha traído a España.

Efectivamente, el centro se preparaba a una defensa valerosa, y
guarnecía sus baterías, distribuía los regimientos a un lado y otro,
agrupando a retaguardia fuerzas considerables de caballería a
retaguardia. Cuando esto pasaba, sentí un vivo clamor de la naturaleza
dentro de mí, sentí hambre, pero ¡qué hambre!\ldots{} Francamente, y sin
ruborizarme, digo que tenía más ganas de comer que de batirme. ¿Y qué?
¿Este miserable hijo de España no había hecho ya bastante por su Rey y
por su patria, para permitir llevarse a la boca un pedazo de pan?

Haciendo estas reflexiones, registré primero la grupera de mi
cabalgadura allegadiza, donde no había más que alguna ropa blanca, y
después las pistoleras, donde encontré un mendrugo. ¡Hallazgo
incomparable! No satisfecho, sin embargo, con tan poca ración, llevé mis
exploraciones hasta lo más profundo de aquellos sacos de cuero, y mis
dedos sintieron el contacto de unos papeles. Saquelos, y vi un pequeño
envoltorio y tres cartas, la una cerrada y las otras dos abiertas, todas
con sobrescrito. Leí el primer sobre que se me vino a la mano, y decía
así: «Al Sr.~D. Luis de Santorcaz, en Madrid, calle de\ldots»

Había montado en el caballo de Santorcaz.

\hypertarget{xxviii}{%
\chapter{XXVIII}\label{xxviii}}

Olvidándome al instante de todo, no pensé mas que en examinar bien lo
que tenía en las manos. El sobrescrito de la primera carta que saqué y
que estaba abierta, era de letra femenina, que reconocí al momento. El
de la carta cerrada, que sin duda no estaba ya en la estafeta por
detención involuntaria, era de hombre, y decía: «\emph{Señora condesa
de}\ldots{} (aquí el título de Amaranta) \emph{en Córdoba, calle de la
Espartería}». El tercer sobre, también de carta abierta, era de letra de
hombre y dirigido a Santorcaz. Desenvolví en seguida el envoltorio de
papeles, que guardaba un bulto como del tamaño de un duro, y al ver lo
que contenía, una luz vivísima inundó mi alma y sentí dolorosa punzada
en el corazón. Era el retrato de Inés.

Aquella aparición en el campo de batalla, en medio del zumbido de los
cañones y del choque de las armas; la inesperada presencia ante mí de
aquella cara celestial, fielmente reproducida por un gran artista; la
sonrisa iluminada que creí observar sobre la placa, cuando fijé en ella
mis ojos; aquella repentina visita, pues no era otra cosa, de mi fiel
amiga, cuando yo hacía tan vivos esfuerzos para hacerme digno de ella,
me regocijaron de un modo inexplicable. Para iluminar los rasgos y
colores de aquel retrato que sonreía, valía la pena de que saliese el
sol, de que existiese el mundo, de que la serie del tiempo trajera aquel
día, aunque deslustrado por los horrores de una batalla.

Estreché aquella Inés de dos pulgadas contra mi corazón y la guardé en
mi pecho, resuelto a no darla, aunque la materialidad del pedazo de
cobre pintado no me pertenecía\ldots{} Pero era preciso leer aquellos
papeles que podían esclarecer alguna de mis dudas. Detúvome al principio
la vergüenza de leer cartas ajenas, lo cual es cosa fea; pero consideré
que Santorcaz habría muerto, fundándome en la dispersión de su caballo
abandonado, y además, como la curiosidad me empezaba a picar, a escocer,
a quemar de un modo muy vivo, me decidí a leer la carta abierta, porque
el deseo de hacerlo era más fuerte que todas las consideraciones.

Yo estaba completamente absorbido por aquel asunto de interés íntimo: yo
no atendía a la batalla; yo no hacía caso de los cañonazos; yo no me
fijaba en los gritos; yo no apartaba la cabeza del papel, aunque sentía
correr por junto a mis oídos el estrepitoso aliento de la lucha. En
aquel instante, entre los veinte mil hombres que formando dos grandes
conjuntos, se disputaban unas cuantas varas de terreno, yo era quizás el
único que merecía el nombre de individuo. Átomo disgregado
momentáneamente de la masa, se ocupaba de sus propias batallas.

La carta abierta, que llevaba la firma de Amaranta, decía así, después
de las fórmulas de encabezamiento:

«¿Eres un malvado o un desgraciado? En verdad no sé qué creer, pues de
tu conducta todo puede deducirse. Después de una ausencia de muchos
años, durante los cuales nadie ha logrado traerte al buen camino, ahora
vuelves a España sin más objeto que hostigarme con pretensiones absurdas
a que mi dignidad no me permite acceder. Harto he hecho por ti, y ahora
mismo cuando me has manifestado tu situación, te he propuesto un medio
decoroso de remediarla. ¿Qué más puedo hacer? Pero no te satisface lo
que en la actualidad y siempre bastaría a calmar la ambición de un
hombre menos degradado que tú; te rebelas contra mis beneficios, y
aspiras a más, amenazándome sin miramiento alguno. A todo esto contesto
diciéndote que desprecio tus amenazas, y que no las temo. No, no es
posible que por la amenaza consiga nadie de mí lo que me impelen a negar
mi dignidad, mi categoría, mi familia y mi nombre. Nunca creí que
aspiraras a tanto, y siempre pensé que te conceptuarías muy feliz con lo
que otras veces has alcanzado de mí, y hoy te ofrezco, haciendo un
verdadero sacrificio, porque el estado del Reino ha disminuido nuestras
rentas\ldots»

Al llegar aquí el golpe de un peso que cayó chocando con mi rodilla, me
hizo levantar la vista de la carta. El soldado que formaba junto a mí,
herido mortalmente por una bala perdida, había rodado al suelo. En aquel
intervalo vi hacia enfrente, envueltas en espeso humo las columnas
francesas que venían a atacar el centro. Pero mi ánimo no estaba para
fijar la atención en aquello. Pude notar que la caballería avanzaba un
poco, que después retrocedía y oscilaba de flanco; pero dejándome llevar
por el caballo, con los ojos fijos en el papel que sostenía a la altura
de las riendas, no puse ni un desperdicio de voluntad en aquellos
movimientos de la máquina en que estaba engranado. La carta continuaba
así:

«\ldots En vano para conmoverme finges gran interés por aquel ser
desgraciado que vino al mundo como testimonio vivo de la funesta
alucinación y del fatal error de su madre. ¿A qué ese sentimiento
tardío? ¿A qué acusarme de su abandono? No, esa niña no existe; te han
engañado los que te han dicho que yo la he recogido. Mal podría
recogerla cuando ya es un hecho evidente que Dios se la llevó de este
mundo. ¿A qué conduce el amenazarme con ella, haciéndola instrumento de
tus malas artes para conmigo? No pienses en esto. Por última vez te
aconsejo que desistas de tus locas pretensiones, y te presentes ante mí
con bandera de paz. ¿Eres un malvado o un desgraciado? Yo sería muy
feliz si me probaras lo segundo, porque uno de mis mayores tormentos
consiste en suponer tan profundamente corrompido el corazón que hace
años sólo existía para amarme\ldots»

Con esto y la firma de Amaranta terminaba la epístola, cuya lectura,
absorbiendo mi atención, me distraía de la batalla. El fragor de esta
zumbaba en mis oídos como el rumor del mar, a quien generalmente no se
hace caso alguno desde tierra. ¿Es tal vuestra impertinencia que queréis
obligarme a contaros lo que allí pasaba? Pues oíd. Cuando la tropa
francesa de línea retrocedió por tercera vez, extenuada de hambre, de
sed y de cansancio; cuando los soldados que no habían sido heridos se
arrojaban al suelo maldiciendo la guerra, negándose a batirse e
insultando a los oficiales que les llevaran a tan terrible situación, el
general en jefe reunió la plana mayor, y expuesto en breve consejo el
estado de las cosas, se decidió intentar un último ataque con los
marinos de la guardia imperial, aún intactos, poniéndose a la cabeza
todos los generales.

Por eso, cuando leída la carta alcé los ojos, vi delante de las primeras
filas de caballería algunas masas de tropa escoltando los seis cañones
de la carretera, cuyo fuego certero y terrible había sido el nudo
gordiano de la batalla. Servidos siempre con destreza y al fin con
exaltación, aquellos seis cañones eran durante unos minutos la pieza de
dos cuartos arrojada por España y Francia, por la usurpación y la
nacionalidad en un corrillo de veinte mil soldados. ¿Cara o cruz? ¿Las
tomarían los franceses? ¿Se dejarían quitar los españoles aquellos seis
cañones? ¿Quién podría más, nuestros valientes y hábiles oficiales de
artillería, o los quinientos marinos?

Yo vi a estos avanzar por la carretera, y entre el denso humo
distinguimos un hombre puesto al frente del valiente batallón y
blandiendo con furia la espada; un hombre de alta estatura, con el
rostro desfigurado por la costra de polvo que amasaban los sudores de la
angustia; de uniforme lujoso y destrozado en la garganta y seno como si
se lo hubiera hecho pedazos con las uñas para dar desahogo al oprimido
pecho. Aquella imagen de la desesperación, que tan pronto señalaba la
boca de los cañones como el cielo, indicando a sus soldados un alto
ideal al conducirles a la muerte, era el desgraciado general Dupont que
había venido a Andalucía, seguro de alcanzar el bastón de mariscal de
Francia. El paseo triunfal de que habló al partir de Toledo había tenido
aquel tropiezo.

Los repetidos disparos de metralla no detenían a los franceses.
Brillaban los dorados uniformes de los generales puestos al frente, y
tras ellos la hilera de marinos, todos vestidos de azul y con grandes
gorras de pelo, avanzaba sin vacilación. De rato en rato, como si una
manotada gigantesca arrebatase la mitad de la fila, así desaparecían
hombres y hombres. Pero en cada claro asomaba otro soldado azul, y el
frente de columna se rehacía al instante, acercándose imponente y
aterrador. Acelerábase su marcha al hallarse cerca; iban a caer como
legión de invencibles demonios sobre las piezas para clavarlas y
degollar sin piedad a los artilleros.

Los que asistían a aquel espectáculo, sin ser actores de él, estaban
mudos de estupor, con el alma y la vida en suspenso, cual si aguardaran
el resultado del encuentro para dejar de existir o seguir existiendo.
Sin embargo esto, ¿creerán mis lectores que algo ocupaba mi espíritu más
de lleno que la última peripecia? Pues sí: yo tenía en mi mano la carta
cerrada, y la curiosidad por leerla no era curiosidad, era una sed moral
más terrible que la sed física que poco antes me había atormentado.
Incapaz de resistirla, sintiendo que todo se eclipsaba ante la
inmensidad del interés despertado en mí por los asuntos de dos o tres
personas que no habían de decidir la suerte del mundo, tomé la carta, la
abrí sin reparar en lo vituperable de esta acción, y al punto la devoré
con los ojos, leyendo lo siguiente:

«Señora condesa: Vuestra carta me anuncia que nada puedo esperar de vos
por los honrados medios que os he propuesto. Lo comprendo todo, y si en
la última que me dirigisteis, dictada sin duda por vuestro propio
corazón, mostrabais bastante generosidad, en esta reconozco las ideas de
vuestra tía la señora marquesa, que otro tiempo os dijo que antes quería
veros muerta que casada con un hombre inferior a vuestra clase.
Preguntáis que si soy un malvado o un desgraciado: y contesto que ya que
os alcanza la responsabilidad de lo segundo, a vos también os tocará sin
duda la triste gloria de lo primero. Esta será la última que os escriba
el que en algún tiempo no hubiera cambiado por todas las delicias del
Paraíso el gozo de leer una letra de vuestra mano. Quizás por mucho
tiempo no oigáis hablar de mí; quizás disfrutéis la inefable
satisfacción de creer que he muerto; pero en la oscuridad y lejos de
vos, yo me ocuparé de lo que me pertenece. ¿Quién es el culpable, vos o
yo? Cuando supe en Madrid que habíais recogido a nuestra hija después de
largo abandono, os prometí legitimarla por subsiguiente matrimonio, como
correspondía a personas honradas. Primero me contestasteis indecisa y
luego furiosa, rechazando una proposición que calificabais de absurda e
irreverente, y llamándome jacobino, francmasón, calavera, perdido,
tramposo, con otras injurias que quisiera oír en tan linda boca. Yo
acepto el bofetón de vuestro orgullo. Lo que no me explico es la
desfachatez con que negáis haber recogido a vuestra hija. ¿Y decís que
esto no me importa? Ya veréis si me importa o no. Yo sé que la habéis
recogido; yo sé que está en un convento; yo sé que su boda con el conde
de Rumblar está concertada; yo sé que para llevarla a cabo se han tenido
en cuenta poderosos intereses de ambas familias, que la hacen
imprescindible; yo sé que para llevar a efecto la legitimación, se ha
consumado una superchería poco digna de personas como\ldots»

Una inmensa conmoción, un estrépito indescriptible me obligaron a
apartar la atención de la carta. Los marinos llegaban a la boca de los
cañones, y un combate terrible, en que parecíamos llevar lo mejor, se
había trabado. Esto era sin duda sublime; esto sacaba de quicio y
conmovía el alma en su fundamento; pero ¿no había algo más en el mundo?
Inés, su madre, su padre, su porvenir, su casamiento, y yo con mi
desmedido y leal amor: yo, preguntándome si podría subir hasta ella, o
si era preciso hacerla descender hasta mí\ldots{} ¡Oh!, esta sí que era
batalla; esta sí que era lucha, señores. Su campo estaba dentro de mí, y
sus fuerzas terribles chocaban dentro del espacio silencioso de mi
pensamiento. ¿Cómo no atender a ella más que a otra alguna? El corazón,
tirano indiscutible, agrandando inconmensurablemente las proporciones de
mi batalla, la había hecho mayor que aquella de que tal vez dependían
los destinos del mundo.

Yo vi los marinos próximos ya, muy próximos a nuestros cañones; sentí
gritos de júbilo y de victoria pronunciados en española lengua, y aunque
todo esto me conmovía mucho, la carta no concluida me quemaba la mano.
Decid que yo era un estúpido egoísta; pero señores, ¿y la carta, y aquel
\emph{casamiento imprescindible}, y aquella \emph{superchería
misteriosa}?\ldots{} ¿Se ganaba la batalla? Creo que sí, y la faz de
Europa iba a variar sin duda. ¿Pero qué me importaba el desconcierto del
Imperio, el júbilo de Inglaterra, el estupor de Rusia, los preparativos
de la coalición, el descrédito del Grande Ejército?

¿Hemos de sobreponer el interés de los conjuntos lanzados a bárbaras
guerras, al interés del inocente individuo que lucha a solas por el bien
y por el amor? ¿Hemos de sobreponer el interés de la guerra, que
destruye, al del amor que crea y aumenta y embellece lo creado? Reíos de
mí; pero al mismo tiempo pensad en el modo de probarme que un corazón
ocupa menos espacio en la totalidad del universo que los quinientos diez
millones de kilómetros cuadrados de la pelota de tierra en que
habitamos.

Si es egoísmo, confieso mi egoísmo, y declaro a la faz de mi auditorio
que en el punto en que se eclipsaba la estrella que por diez años había
iluminado la Europa, volví a fijar los ojos en la carta para continuar
leyendo. Si no quieren Vds. enterarse de ello, no se enteren; pero es mi
deber decir que la carta concluía así:

«\ldots una superchería poco digna de personas como vos. Segura estáis y
con razón de que nada puedo contra vos. En efecto, yo sé que si algo
intentara sería vencido. Pobre, sin recursos, sin valimiento, ¿qué
podría contra la justicia que sólo defiende a los poderosos? Pero mi
hija me pertenece, y si hoy no está en mi poder, os aseguro que lo
estará mañana. Entretanto guardaos vuestro dinero».

No decía más. Pero cuando acabé de leerla, ¡qué nueva y terrible fase
tomaba la refriega entre los marinos y nuestros soldados! ¡Santo Dios!
¿La batalla se perdería? Los franceses, destrozados en el primer ataque,
lo repetían sacando el último resto de bravura de sus corazones
resecados por el calor, y volvían a la carga resueltos a dejarse hacer
trizas en la boca de los cañones, o tomarlos. Nuestros soldados sacaban
fuerzas de su espíritu, porque en el cuerpo ya no las tenían. Hasta los
artilleros empezaban a desfallecer, y heridos casi todos los primeros de
derecha e izquierda, atacaban los segundos, daban fuego los terceros, y
el servicio de municiones era hecho por paisanos. Los franceses medio
resucitados con la valentía de los marinos, pudieron habilitar dos
piezas y desde lejos tomando por punto en blanco la masa de nuestra
caballería, disparaban bastantes tiros. Su larga trayectoria, pasando
por encima de la batería española, hería las primeras filas de mi
regimiento. Este se encabritó como si fuera un solo caballo; chocamos
unos con otros, y el espectáculo de dos compañeros muertos sin combatir
nos llenó de terror. Al mismo tiempo oímos decir que escaseaban las
municiones de cañón. ¡Terrible palabra! Si nuestros cañones llegaban a
carecer de pólvora, si en sus almas de bronce se extinguía aquella
indignación artificial, cuyo resoplido conmueve y trastorna el aire,
estremece el suelo y arrasa cuanto encuentra por delante, bien pronto
serían tomados por los valientes marinos, y les aguardaba el morir
inutilizados por el denigrante clavo, fruslería que destruye un gigante,
alfiler que mata a Aquiles.

Esta consideración ponía los pelos de punta. ¿Sucumbiría España? ¿No le
reservaba Dios la gloria de dar el primer golpe en el pedestal del
tirano de Europa?\ldots{} No, no es posible asistir indiferente al
espectáculo de tan supremo esfuerzo, oh patria; pero te confieso que yo
rabiaba por conocer el autor de aquella tercera carta que tenía en mi
mano, y cuando sin desatender a tu admirable heroísmo, miré la firma y
vi el nombre de Román, segundo mayordomo de mi inolvidable ama; cuando
consideré que aquel papel contendría revelaciones importantes, me dominó
de tal modo la curiosidad, que por un instante desapareciste de mi
espíritu, ¡oh sublime rincón de tierra, destinado más de una vez a ser
equilibrio del mundo! Adiós España, adiós Napoleón, adiós guerra, adiós
batalla de Bailén. Como borra la esponja del escolar el problema escrito
con tiza en la pizarra, para entregarse al juego, así se borró todo en
mí para no ver más que lo siguiente:

«Sr.~D. Luis de Santorcaz: Voy a deciros puntualmente lo ocurrido. Todo
está resuelto, y por ahora os dan con la puerta en los hocicos. La
señora marquesa de Leiva, al recoger a la señorita Inés, pensó en el
modo de legitimarla. Advierto a Vd. que desde que la trataron, ambas la
quieren mucho, y se desviven por decidirla a que salga del convento.
Cuando la señora condesa recibió la carta de Vd. en que le proponía la
legitimación por subsiguiente matrimonio, mostrola a su tía, y ésta
furiosa y fuera de sí preguntó si quería deshonrarse para siempre siendo
esposa de semejante perdido. Lloró un poco la condesa, lo cual es
indicio de que aún le queda algo de aquel amor; y por último, después de
muchas reconvenciones, convinieron las dos en no admitirle a Vd. en su
familia por ningún caso. Ya sabe Vd. que según consta en la fundación de
este gran mayorazgo, uno de los principales de España, no habiendo
herederos directos, pasa a los de segundo grado en línea recta, por lo
cual ahora correspondería al primogénito del conde Rumblar. La actual
condesa de Rumblar, enterada de la aparición de una heredera, anunció a
mi ama que entablaría un pleito, y vea Vd. aquí el motivo de que en casa
se haya trabajado tanto por la legitimación. Por fin, las dos familias
acordaron evitar la ruina de un pleito y se han puesto de acuerdo sobre
esta base: casar a la señorita Inés con D. Diego de Rumblar, previa
legitimación de aquella, por lo que llaman autorización del Rey, con lo
cual, ambos derechos se funden en uno solo, evitando cuestiones. En
cuanto al punto más difícil, la señora marquesa lo ha resuelto al fin de
un modo ingenioso y seguro. La niña ha entrado al fin con pie derecho en
la familia. No pudiendo legitimar la madre, porque a ello se oponen las
leyes; no pudiendo aceptarse la fórmula del subsiguiente matrimonio, ni
conviniendo tampoco la adopción, por no dar esto derecho a la herencia
del mayorazgo, se acordó lo que voy a decir a Vd., y que sin duda le
llenará de admiración. Este sesgo del asunto tiene para la familia la
ventaja de que mi señora la condesa no pasará ningún bochorno. La
señorita Inés ha sido reconocida por aquel\ldots»

Un violento golpe arrebató el papel de mis manos. Encabritose mi
caballo, y al avanzar siguiendo el escuadrón, sentí la estrepitosa risa
de un soldado que decía: «Aquí no se viene a leer cartas». Corrimos
fuera de la carretera, y todos mis compañeros proferían exclamaciones de
frenética alegría. Vi los cañones inmóviles y delante una espesa cortina
de humo, que al disiparse permitía distinguir los restos del batallón de
marinos. En el frente francés flotaba una bandera blanca, avanzando
hacia nuestro frente. La batalla había concluido.

Nuestros soldados se abrazaban con delirio. Confundíanse los diversos
regimientos, y los paisanos advenedizos con la tropa. La gente del
vecino pueblo de Bailén acudía con cántaros y botijos de agua.
Agrupábanse hombres y mujeres junto a los heridos para recogerlos. Los
caballos recorrían orgullosos la carretera, y los generales confundidos
con la gente de tropa, demostraban su alegría con tanta llaneza como
esta. Los gritos de ¡viva España!, ¡viva Fernando VII! parecían un
concierto que llenaba el espacio como antes el ruido del cañón; y el
mundo todo se estremecía con el júbilo de nuestra victoria y con el
desastre de los franceses, primera vacilación del orgulloso Imperio. En
tanto yo recorría el campamento, miraba al suelo, miraba las manos de
todos, las cureñas de los cañones, los charcos de sangre, los mil
rincones del suelo, junto al cuerpo de un herido y bajo la cabeza del
caballo moribundo. Marijuán se llegó a mí con los brazos abiertos y
gritó:

---Les vencimos, Gabriel. ¡Viva España y los españoles, y la Virgen del
Pilar a quien se debe todo! Pero ¿qué buscas, que así miras al suelo?

---Busco un papel que se me ha perdido---le contesté.

\hypertarget{xxix}{%
\chapter{XXIX}\label{xxix}}

---Déjate de papeles---me dijo Marijuán---¡Qué demonios de marinos!
¿Viste cómo atacaban?

---La hacen hija legítima por autorización real.

---¿Qué estás diciendo? Ya no queda duda que hemos vencido a Napoleón, y
como este ha vencido a todo el mundo, resulta que nosotros hemos vencido
al mundo entero. ¿Pero chico, no te vuelves loco? Mira cómo alzan los
brazos gritando, aquellos generales que vienen por el llano. ¡Benditas
penas, benditos golpes, bendito calor y bendita sed, puesto que al fin
hemos salido vencedores! ¡Viva España!

---De esa manera---le dije yo, preocupado con mis guerras---entra a
disfrutar el mayorazgo, casándose con D. Diego, para evitar un litigio
que arruinaría a las dos familias.

---¿Qué hablas ahí, muchacho?---exclamó con sorpresa.---Ya sabes que los
franceses se van a entregar todos. ¡Qué vergüenza! ¡Que vuelva Napoleón
a meterse con los españoles! Chico; nos vamos a comer el mundo, y digo
que la Junta de Sevilla es una remilgada si no nos manda conquistar a
París. ¡Viva España!

---Y nuestro amo, ¿dónde está?---pregunté intranquilo.---¿Qué ha sido
del señorito de Rumblar?

---¡Creo que ha muerto!---me contestó lacónicamente Marijuán, picando
espuelas y alejándose de mí.

Tan estupenda noticia dio nueva dirección a mis alborotados
pensamientos. El aspecto de la refriega interior que me sacudía el alma
cambió de improviso y por completo. Todo vino abajo, todo se puso de
otro color, y el mundo fue distinto a mis ojos. Ignoro si en aquel
momento sentí la muerte de mi amo, o si por el contrario, desbordado el
corruptor egoísmo en mi alma, acepté con regocijo la desaparición de
quien interponiéndose entre mi ideal y yo, alteraba a mis ojos el
equilibrio del universo, más que Napoleón el de Europa\ldots{} En medio
del delirio de aquella gran victoria, una de las más trascendentales que
han ocurrido en el mundo, yo permanecía mudo, y mi caballo me
transportaba de un lado para otro según su albedrío. En mi derredor la
efervescencia de aquella patriótica alegría, de aquel entusiasmo febril
causaba estrepitoso oleaje. Allí la persona humana había desaparecido
fundiéndose en el hermoso conjunto de la sociedad o la Nación, que era
sin duda la que conmovía la tierra con sus gritos de gozo. El único que
se conservaba aislado, y podía llamarse hombre, era el egoísta Gabriel,
grano de arena no conglomerado con la montaña, y que rodaba solo
haciendo por su propia cuenta las revoluciones establecidas por la
armonía del mundo.

---Es preciso averiguar si realmente ha muerto Rumblar\ldots{} ¿Entrará
al fin Inés en la familia de su madre? ¿La perderé para siempre? ¿Debo
reírme de mi necia y ridícula aspiración? ¿Un hombre como yo puede subir
a tanta altura? ¿La misteriosa oscuridad de los tiempos venideros
ocultará alguna cosa que destruya este nivel espantoso? ¿Puedo esperar,
o resignarme desde ahora, bendiciendo la mano de la Providencia que me
arroja en el polvo de donde nunca debí intentar salir?

Estas preguntas me hacía, cuando un acontecimiento no previsto vino a
alterar repentinamente la situación de las cosas fuera de mí. El
ejército corría a ocupar sus posiciones; la corneta y el tambor
convocaban a todos los soldados, y gran número de gentes del pueblo,
hombres y mujeres, corrían hacia las calles de Bailén. Nuestros
destacamentos habían divisado las columnas avanzadas del general Vedel
que venía de Guarromán en auxilio de Dupont, y ya a poca distancia, un
cañonazo nos anunció la presencia de un nuevo enemigo. ¡Ay!, ¡si Vedel
hubiese llegado un momento antes, poniéndonos entre dos fuegos! Pero
Dios, protector en aquel día de la España oprimida y saqueada, permitió
que Vedel llegase cuando estaba convenida ya la tregua, y se había
principiado a negociar la capitulación.

Al instante mandó Reding un oficio al general francés dándole cuenta de
lo ocurrido, y los enemigos se detuvieron más allá de una ermita que
llaman de San Cristóbal, situada a mano izquierda del camino real, yendo
de Bailén a Guarromán. Al poco rato vimos un oficial francés que llegó
al pueblo con un oficio para Reding y otro para Dupont, y como en el
cuartel general de este se estaban ya negociando las bases de la
capitulación, nos consideramos seguros de ser atacados por la parte alta
del camino, a causa de que la acordada suspensión de armas debía afectar
a todas las fuerzas que componían el ejército imperial de Andalucía.

A pesar de esta confianza, varios regimientos, entre ellos el de Irlanda
y el famosísimo de Órdenes Militares que tanto se había distinguido en
la batalla, ocuparon el camino frente a las tropas de Vedel, las cuales
iban llegando por momentos y tomaban posiciones. Mi regimiento fue
colocado en la entrada oriental del pueblo. Sería poco más de la una
cuando los franceses de Vedel, sin aguardar a que les contestara Dupont,
rompieron el fuego contra Irlanda, sorprendiéndoles con fuerzas
considerables. Gran efervescencia y algazara y tumulto en nuestras
filas. Todos querían ir no a combatir con los franceses, sino a pasarlos
a cuchillo, por violar las leyes de la guerra. Pero nosotros teníamos,
para sojuzgar a los traidores, rehenes preciosos, cuales eran los restos
del ejército de Dupont, que estaban en nuestro poder, como una víctima
maniatada y con la cabeza sobre el tajo. Durante la confusión que siguió
al ataque, algunas tropas acudieron a cercar el campo francés vencido, y
otras corrieron en auxilio de los regimientos de Irlanda y Órdenes,
puestos en gran compromiso.

A pesar de la inferioridad de número y de posición de nuestras tropas,
todo anunciaba que se iba a trabar un combate tan encarnizado como el
primero, y los valerosos paisanos lo mismo que los soldados de línea
ardían en generoso anhelo de morir si era preciso por rematar con una
tarde épica la gloriosa mañana.

Pero la Providencia, como he dicho, estaba de nuestra parte. Casi
juntamente con los primeros tiros de la embestida de Vedel, sonaron
cañonazos lejanos, que al principio no supimos a qué dirección referir.

---¿Qué es eso? ¿Hacen fuego por el Herrumblar o es la gente de
Mengíbar? ---preguntaban allí.

---Es la división de D. Manuel de la Peña, que viene por la Casa del
Rey---contestó uno que a todo escape venía del primer campo de batalla.

La tercera división, enviada al amanecer desde Andújar por Castaños en
seguimiento de Dupont, había llegado, y se anunciaba al enemigo con
disparos de pólvora seca. Aterrado con este nuevo refuerzo, que
aniquilaría los restos del ejército, si Vedel no se sometía al
armisticio, Dupont dio enérgicas órdenes para que cesara el fuego de la
división recién venida de Guarromán, y el fuego cesó. Con esto, los
nueve mil hombres de Vedel se sometieron de antemano al pacto que
ajustaba su General en Jefe.

Seguimos, sin embargo, sobre las armas, y las entradas de la villa
continuaron custodiadas por numerosas fuerzas, que se relevaban para
proporcionarnos algún descanso. Cuando me tocó dejar la guardia,
dirigime a una de las muchas casas del pueblo en que curaban heridos,
para que me pusieran algo en la mano izquierda, donde había recibido una
contusión que aunque ligera, me escocía bastante. Regresaba luego a pie
en busca de mi puesto, cuando, sintiendo una mano en mi hombro, miré y
tuve el gusto de encontrarme cara a cara con D. Paco, el maestro y ayo
de D. Diego.

---¿Qué ha sido del niño?, ¿dónde está? No ha venido por casa---me dijo
con tono angustiado y poniéndose pálido.

---Sr.~D. Paco---le contesté,---francamente, no sé dónde está el señor
conde, aunque me parece que debe de estar vivo.

---¡Qué miedo, qué pavor! ¡La santa Virgen de Araceli, la de Fuensanta,
la del Pilar y la del Tremedal todas juntas nos favorezcan! Las piernas
me tiemblan, Gabriel, y si mi señor y discípulo no parece, yo no me
atrevo a decírselo a la señora.

---Ya parecerá; yo le vi poco antes de concluir la batalla. Andará por
cualquier lado---dije para calmar su inquietud.

---Es raro que estando sano y salvo no viniese a casa, o mandara un
recado. ¿En dónde hay caballería?

---En San Cristóbal, en donde estaba la batería, en la noria, en los
altos de la derecha, en los del Gaudiel, hacia el Herrumblar, en muchas
partes. Ya andará el Sr.~D. Diego por ahí.

---Dios lo quiera. Voy, corro a buscarlo. ¿Dime tú\ldots{} ya no harán
fuego, eh? ¿Habrá peligro en andar por aquí? Si quisieras acompañarme.
¡Diantre con el niño, y si supiera él qué buenas noticias le traigo cómo
se apresuraría a venir a mi encuentro!

---¿Qué noticias, Sr.~D. Francisco? ¿Se pueden saber?---pregunté
disponiéndome a acompañar al ayo por el campo de batalla.

---¡Noticias estupendas y que le harán saltar de gozo! Esta mañana
recibió la señora un propio de la marquesa de Leiva, anunciando que su
Excelencia, con la condesa, con la señorita Inés y el señor marqués,
salen de Córdoba para Madrid, a donde los llama un negocio de mucho
interés para las dos familias.

---El camino no está para viajes, Sr.~D. Paco.

---Vienen por Mengíbar, y anuncian que de esta noche a mañana llegarán a
casa, donde piensan detenerse algunos días, no sólo para tomar descanso,
sino para que ambas familias se conozcan y traten, pues son ramas que
van a injertarse, formando un solo árbol frondoso que eche profundas
raíces en el suelo de la Nación y dé sombra a numerosa e ilustre prole.

---Sí---dije,---ya sé que el señorito se casa\ldots{}

---¡Ay! ¡Dónde estará ese Juan enreda de D. Diego!\ldots{} Sí, se casa.
He visto el retrato de la señorita Inés, que es un portento de
hermosura. Pues sí: la niña no quería salir del convento, aunque se lo
predicaran frailes teatinos; pero yo no sé; algo pasó allá a principios
del mes, o sin duda la joven al ver el retrato de D. Diego, sintió la
flecha del dios ceguezuelo en su corazón. Lo cierto es que ha pedido
salir del convento, con gran regocijo de sus parientes, y ahora marchan
todos a Madrid para las diligencias de la legitimación, porque ya sabes
tú que\ldots{}

---Sí, había entendido que esa joven era hija de la señora condesa.

---¡Calla, deslenguado procaz! ¡Qué has dicho! La señora condesa, prima
de mi señora, había de tener semejantes tapujos. No hay tal cosa,
chiquillo desvergonzado. La señorita Inés es hija de una dama
extranjera, que ya no existe y que floreció hace quince años en la
corte, dando que hablar por sus amores con un célebre caballero de esta
ilustre familia. ¿Sabes quién es el padre de doña Inés? Pues no es otro
que ese espejo de los diplomáticos, ese discretísimo hermano de la
señora marquesa de Leiva, el cual ha reconocido a la muchacha por hija
suya, y ahora se apresura a legitimarla por autorización real para que
entre en posesión del mayorazgo cuando Dios se sirva llamar a su seno a
la señora marquesa de Leiva.

---¡Qué bien lo han compuesto todo!---exclamé sin poder contener la
expresión de mi asombro.

---¿Cómo compuesto? Mi señora me ha participado esta mañana lo que acabo
de decir. ¡Ah! Ese sin par diplomático, que tanta fama tiene en todas
las cortes de Europa, ha dado una prueba de caballerosidad, poniendo su
nombre a ese fruto de sus iracundas fogosidades juveniles, abandonado
hasta hoy, y que en lo sucesivo descollará cual arbusto lozano en el
pensil de la sociedad española\ldots{} Pero ese D. Diego\ldots{} ¿En
dónde está D. Diego? Hablemos al general en jefe\ldots{} preguntemos a
esos soldados\ldots{} Diga Vd., héroe de este día, que se anotará en los
fastos de la historia con piedra blanca, \emph{albo notanda lapillo};
oiga Vd., ¿ha visto Vd. por casualidad a D. Diego?

Y así iba preguntando a todos, sin que nadie le diese razón.

\hypertarget{xxx}{%
\chapter{XXX}\label{xxx}}

Vino la noche. Los franceses, muertos de fatiga y de hambre en su
campamento, aguardaban con anhelo a que la capitulación estuviese
firmada. Los que menos paciencia tenían eran los suizos afiliados en el
ejército imperial, y así que oscureció empezaron a pasarse a nuestro
campo. Un historiador francés, queriendo atenuar el desastre de los
suyos, ha escrito que la defección ocurrió durante la batalla; pero esto
es falso. Lo peor es que otro historiador, no francés sino español, lo
ha repetido con lamentable ligereza, faltando así a su patria y a la
verdad, que es superior a todo.

La capitulación iba despaciosamente, porque los parlamentarios se habían
juntado en Andújar, residencia del general en jefe, y en Bailén no
teníamos noticia de lo que allí pasaba. Temiendo que los enemigos
intentaran escaparse, nuestros generales tomaron acertadas precauciones,
y la artillería ocupó, mecha encendida, los puestos convenientes. Al
mismo tiempo millares de paisanos, discurriendo por cerros y alturas,
hostigaban de tal modo a los franceses en todas partes, que no les era
posible moverse. Esta vigilancia permitía descansar a una parte del
ejército; y especialmente los heridos, aunque sólo lo fueran muy
levemente como yo, teníamos libertad para estar en el pueblo, donde nos
ocupábamos en reunir víveres y llevarlos a los del campamento, así como
en acomodar a los heridos graves en las principales casas.

Salía yo de Bailén con un cesto de víveres para unos jefes de artillería
cuando tropecé con Santorcaz, que volvía seguido de algunos voluntarios
de Utrera y licenciados de Málaga.

---¡Oh, Sr.~de Santorcaz!---exclamé con la mayor sorpresa.---¿Está Vd.
vivo? Yo le hacía en el otro barrio.

---No, muchacho, vivo estoy---me respondió.---Dios quiere que todavía el
que está dentro de esta camisa dé mucho que hacer en el mundo.

---¿Pero tampoco está Vd. herido?

---Aquí tengo un par de rasguños; pero esto no es nada para un hombre
como yo. Ya sabes que me han hecho sargento. No vine aquí para ganar
charreteras; pero puesto que me las dan, las tomo.

---Grandes hazañas habrá hecho el Sr.~D. Luis.

---Poca cosa. Caí del caballo, y a pie defendime rabiosamente contra
tres o cuatro franceses. Reventé a uno, descalabré a otro, y me volví a
nuestro campo con un águila que entregué al marqués de Coupigny. Al
recoger de mis manos la bandera, el general, después de preguntarme si
era licenciado de presidio, me dijo: «Es Vd. sargento». ¿Ves? Me han
puesto al frente de este pelotón de buenos muchachos; ¿quieres venirte
con nosotros?

Diciendo esto señaló a los esclarecidos varones que le seguían, los
cuales, o yo me engaño mucho o eran la flor y nata de Ibros, Sierra de
Cazorla y Despeñaperros, todos gente de ligerísimas piernas y manos.
Dile las gracias por el ofrecimiento, y seguí mi camino.

---¡Ah! ¿Qué sabe Vd. de D. Diego?---le pregunté volviendo atrás.

---Pues qué---dijo retrocediendo,---¿no se sabe dónde está D. Diego? ¿Ha
muerto? ¿Se ha extraviado? Es preciso averiguarlo. Y di, ¿tú has visto
por casualidad mi caballo? ¿Sabes si alguien lo recogió?

---No sé nada de tal caballo---repuse alejándome.

Ya avanzada la noche regresé a Bailén, donde me causó sorpresa ver una
triste procesión compuesta de tres mujeres vestidas de negro, a las
cuales seguían hasta media docena de hombres, llevando por delante dos
criados con sendos farolillos para alumbrar el camino. Acerqueme y
reconocí a doña María, con sus dos hijas, las tres cubiertas con negros
mantones y muy afligidas y llorosas. Digo mal, porque si las dos
muchachas se deshacían en lágrimas, la señora condesa conservaba seco el
rostro, aunque visiblemente alterado, la mirada fija y valerosa y el
andar muy firme. Al instante me presenté a ella, saludándola con el
mayor respeto y ofreciéndola mi ayuda si, como parecía, iban en busca de
D. Diego.

---¿Conque no parece el niño? ¿Cuándo le perdiste de vista durante la
batalla?---me preguntó.

---Señora, desde la gran carga que dimos sobre el ala izquierda de los
franceses dejé de ver a D. Diego.

---Yo creí que estuviera entre los heridos; pero no está. ¿Todos los
muertos han sido recogidos del campo de batalla?

---Sí señora; sólo quedan los desconocidos, los paisanos que no estaban
afiliados a ningún regimiento.

---Vamos a verlo---dijo con un aplomo, con una firmeza que me
asombraron, pues no suponía tanto valor en el alma de una mujer.

---Yo acompañaré a usía con mucho gusto.

---¿Y qué tal se ha portado mi hijo?---me preguntó cuando marchábamos
juntos.

---Señora, se ha portado como un héroe; se ha portado como quien es.

---¿Los jefes advirtieron su valor? ¿Elogiaron su bizarría, recordando
el linaje de mi hijo?

---Sí señora, los jefes estaban con la boca abierta presenciando las
hazañas de D. Diego---repuse por halagar el amor propio de la noble
señora, cuyo dolor se atenuaría sabiendo que su vástago había honrado el
nombre de Rumblar.

---¿Y amabais vosotros a mi hijo?

---¡Oh!, sí señora. D. Diego es tan bueno\ldots{} y nos trata como si
fuéramos todos iguales.

---¡Como si todos fuerais iguales!---exclamó doña María con ligeras
muestras de enfado.

---No\ldots{} vamos al decir\ldots---indiqué corrigiendo mi
\emph{lapsus}.---D. Diego es un caballero y nosotros unos
badulaques\ldots{} quiero decir que nos trataba sin tiranía\ldots{}
¡Pobre D. Diego! Pero le hemos de encontrar, señora. D. Diego está sano
y salvo. Me lo dice el corazón.

---Tú eres un buen muchacho. Ayúdanos a buscar a mi hijo y te
recompensaré. Si parece, yo te prometo que serás su paje cuando se case.

---¡Ah, gracias señora!, muchas gracias---contesté con viveza.

---Eres modesto. ¿Crees que no mereces este honor? Aunque no lo merezcas
yo te lo concedo.

Llegamos a un punto en que se distinguía un cuerpo tendido boca abajo
sobre el suelo. Nos estremecimos todos, y Asunción y Presentación se
abrazaron llorando a gritos. La curiosidad luchó un instante en nosotros
con el temor, pues deseábamos acercamos al cadáver por ver si era D.
Diego, y temíamos llegar a él por si acaso era. Doña María fue la
primera que dio un paso y la seguimos todos. Aquel cadáver solitario de
un hombre muerto por la patria, no había encontrado todavía ni un
pariente, ni un amigo, ni un camarada que se cuidase de él. No era D.
Diego.

La condesa después de examinarlo alzó los ojos al cielo, cruzó las manos
y rezó en voz alta el \emph{Padre nuestro}, a cuya oración contestamos
todos muy devotamente con \emph{El pan nuestro}\ldots{}

Seguimos andando, y en otro sitio encontramos algunos cadáveres, que la
condesa con heroísmo sobrenatural examinaba cara a cara hasta
convencerse de que su hijo no estaba allí. Si nos acontecía llegar en el
momento de abrir a alguno la sepultura, todos echábamos un puñado de
tierra en la fosa del patriota, que bien pronto desaparecía en la vasta
superficie del campo, no quedando huella ni marca alguna en el suelo,
como no queda noticia del heroísmo individual en la historia.

Nuestras pesquisas por todo el campamento no dieron resultado alguno.
Las dos hermanitas no podían tenerse en pie, ni cesaban de rezar en
castellano y en latín, recitando con fervorosa declamación cuantas
oraciones sabían. Tales eran la confusión y anonadamiento de D. Paco,
que más de una vez se cayó al suelo. Sólo doña María conservaba una
entereza heroica y casi bárbara que hacía creer en la superioridad del
temple moral de algunos linajes sobre el plebeyo vulgo. No en vano tenía
aquella señora por su línea materna la sangre de Guzmán el Bueno.

Era muy tarde cuando volvimos a la casa. Mientras reinaba en ella la
desolación, ni una lágrima brotó de los ojos de doña María.

---Si Dios ha querido disponer de la vida de mi hijo---exclamó
sentándose en el clásico sillón de cuero,---concédame al menos el
consuelo de saber que ha muerto con honor.

---D. Diego ha de parecer, señora---dije yo con movido.---Si hubiera
muerto, ¿no habríamos encontrado su cuerpo?

Esta razón devolvió a D. Paco su perdida fuerza dialéctica, y habló así:

---¿Pero no hubo también un pequeño combate por donde estaba Vedel?
¡Quién sabe si cogerían prisionero al niño!

---Los prisioneros fueron devueltos esta tarde por orden de
Dupont---repuso doña María.

---¿Y si el niño estaba herido y lo metieron en el hospital
francés?\ldots{}

---Yo lo he de averiguar, señora---exclamé.---Mañana mismo pediremos un
salvo-conducto para ir al campo enemigo. Me parece que allí le
encontraremos.

---Ya sabes que te he prometido una gran recompensa. Si haces lo que
dices, y encuentras a mi hijo y le traes---me dijo la de Rumblar,---la
recompensa será aún mayor. Dios dispone de todo, y las glorias de la
tierra a veces son trocadas en miseria, en tristeza, en nada por su mano
poderosa. Si mi hijo no parece, ¿qué soy, qué me queda, qué resta a mi
casa y a mi nombre? Dios habrá decidido que todo perezca y que las
grandezas de ayer sean hoy ruinas, donde nos ocultemos para llorar. ¿La
victoria se había de alcanzar sin desgracias? Napoleón es vencido en
España, y ante la salvación de nuestro país, ¿qué significa una vida por
noble que sea?, ¿qué una familia, por grande que sea su lustre?

La enérgica entereza de aquella mujer de acero me llenó de asombro.
Después continuó así:

---Yo creí que este sería un día de júbilo en mi casa. Después de la
victoria alcanzada, hubiéramos sido muy felices teniendo aquí a mi hijo,
y recibiendo a la prometida esposa que con mis primas debe de llegar
aquí esta noche\ldots{} ¿No ha llegado? Cuide usted, don Paco, de que
nada les falte. ¿Está todo preparado, las camas, la cena, las
habitaciones? Niñas, ¿qué hacéis ahí mano sobre mano?

Asunción y Presentación lloraron con más fuerza al oírse nombrar por su
madre. Pareciome que esta también comenzaba a sentir vacilante su
varonil espíritu, y que apagándose la llama de sus ojos, se desmayaban
sus enérgicos brazos, cayendo con desaliento sobre los del sillón. Pero
sin duda no quería perder su dignidad de gran señora delante de
nosotros, y mandándonos salir a todos, a sus hijas, a D. Paco, a los
criados y a mí, se quedó sola.

Un rato después sentí ruido de coches y mulas en la calle; luego una
gran algazara en el patio, y al oír esto, diome un gran vuelco el
corazón. Escondido tras uno de los pilares vi descender de los coches y
subir pausadamente a las personas que eran esperadas, y al mirar al
diplomático que cargaba en sus brazos a una mujer para bajarla del
carruaje, reconocí a la monjita de Córdoba.

Yo temía ser visto de Amaranta; pero como esta y su tía habíanse
adelantado y estaban ya arriba, me aventuré a seguir al diplomático, que
subió detrás de todos con Inés, sosteniéndola por la cintura. Delante
iban los criados con hachas, detrás yo solo. Inés se envolvía en un gran
manto, chal o cabriolé que tenía larguísimos flecos en sus orillas.
Subíamos lentamente, ellos delante, yo detrás, y aquellos menudos hilos
de seda pendientes de la espalda y de la cintura de Inés flotaban
delante de mis ojos. Como quien llega a la puerta del cielo y tira del
cordón de la campanilla para que le abran, así cogí yo entre mis dedos
uno de aquellos cordoncitos rojos y tiré suavemente. Inés volvió la
cabeza y me vio.

\hypertarget{xxxi}{%
\chapter{XXXI}\label{xxxi}}

Una vez arriba, el ayo informó a los viajeros de lo que ocurría, y
pasando adentro las tres señoras, el diplomático se quedó con D. Paco en
el comedor.

---Aquí estamos consternados, Sr.~D. Felipe---dijo el ayo.---Y si mi amo
no parece el mundo habrá perdido en el fragor de horripilante batalla a
un joven que prometía ser gran filósofo, y que ya era gran calígrafo.

---¡Demonio de contrariedad!---dijo el diplomático, sacando su caja de
tabaco y ofreciendo un polvo al ayo, después de tomarlo él.---Lo
siento\ldots{} a nuestra edad nos gusta tener quien nos suceda y herede
nuestras glorias para desparramar su luz por los venideros siglos. Vea
Vd. la razón por qué me apresuré a reconocer a mi querida hija\ldots{}
¡Ah! Sr.~D. Francisco: yo he tenido una juventud muy borrascosa, como
todo el mundo sabe, y hartas noticias tendrá Vd. de mis aventuras, pues
no había en las cortes de Europa dama alguna, casada ni soltera, que no
se me rindiese. Después de todo es una desgracia haber nacido con tal
fuerza de atracción en la persona, Sr.~D. Francisco; tanto que
todavía\ldots{} pero dejemos esto. Ahora no me ocupo más que del
bienestar de mi idolatrada niña. Y a fe que si es cierto que no existe
D. Diego, no por eso se quedará soltera; pues cartas tengo aquí del
príncipe de Lichenstein, del archiduque Carlos Eugenio, del conde de
Schöenbrunn y de otros esclarecidos jóvenes de sangre real pidiéndomela
en matrimonio. Como yo tengo tantos amigos en las cortes de Europa, y en
España mismo, pues\ldots{} ya he sabido que las principales familias
acogidas en Bayona o residentes en Madrid, se disputan la mano de mi
hija. ¿La ha visto Vd., Sr.~D. Francisco? ¿Ha observado usted en su cara
los rasgos que indican la noble sangre mía y la de aquella hermosísima,
cuanto desgraciada señora extranjera\ldots? ¡Oh!, me enternezco, señor
D. Francisco\ldots{} Pero hablemos de otra cosa, cuénteme Vd. cómo ha
sido esa batalla. ¿Conque hemos ganado? ¿Y hay capitulación? De modo que
he llegado a tiempo. ¡Oh! Sr.~D. Francisco, temo que hagan un desatino,
si no les asisto con mis luces, porque los militares son tan legos en
esto de tratados\ldots{} Yo traigo un proyectillo, mediante el cual la
Rusia ocupará Despeñaperros, España pasará a guarnecer las orillas del
Don y de la Moscowa, y Prusia\ldots{}

Cuando me marché, el diplomático continuaba calentando los cascos al
buen D. Paco, que le ofreció algunos manjares y vino de Montilla para
reparar sus fuerzas. Al salir de la casa, vi en la puerta de la calle a
varios hombres, no de muy buena facha por cierto, uno de los cuales
llegose a mí, y tomándome por el brazo, me dijo:

---¿Conoces tú a esa gente que acaba de llegar?

---No, Sr.~de Santorcaz---repuse.---No sé qué gente es esa, ni me
importa saberlo.

Apartámonos todos de la casa, y por el camino me dijo otra vez D. Luis
que tendría mucho gusto en verme en las filas de su compañía.

Al día siguiente, que era el 20, nos ocupamos Marijuán y yo en buscar
otra vez a nuestro amo. Uniósenos D. Paco, y el general español escribió
un oficio a Dupont, rogándole que nos permitiera hacer indagaciones en
el campamento francés, para ver si se encontraba allí a D. Diego, herido
o muerto. Visitamos el hospital enemigo, y entre los heridos no había
ningún español, lo cual nos desconsoló sobremanera. Yo no era el que
menos se acongojaba con esta contrariedad, aunque sabía el casamiento de
Inés. ¿Qué significaba aquel generoso sentimiento mío? ¿Era pura bondad,
era puro interés por la vida del semejante, aunque fuese enemigo, o era
un sentimiento mixto de benevolencia y orgullo, en virtud del cual yo,
convencido de que Inés no amaba sino a mí, quería proporcionarme el gozo
de ver a D. Diego despreciado por ella? Francamente, yo no lo sabía, ni
lo sé aún.

Cuando recorrimos el campo francés, pudimos observar la terrible
situación de nuestros enemigos. Los carros de heridos ocupaban una
extensión inmensa, y para sepultar sus tres mil muertos, habían abierto
profundas zanjas donde los iban arrojando en montón, cubriéndoles luego
con la mortaja común de la tierra. Algunos heridos de distinción estaban
en las Ventas del Rey; pero la mayor parte, como he dicho, tenían su
hospital a lo largo del camino, y allí los cirujanos no daban paz a la
mano para vendar y amputar, salvando de la muerte a los que podían. Los
soldados sanos sufrían los horrores del hambre, alimentándose muy mal
con caldos de cebada y un pan de avena, que parecía tierra amasada.

Todos anhelaban que se firmase de una vez la capitulación para salir de
tan lastimoso estado; pero la capitulación iba despacio, porque los
generales españoles querían sacar el mejor partido posible de su
triunfo. Según oí decir aquel día cuando regresamos a Bailén, ya estaba
acordado que se concediese a los franceses el paso de la sierra para
regresar a Madrid, cuando se interceptó un oficio en que el
lugarteniente general del Reino mandaba a Dupont replegarse a la Mancha.
Comprendieron entonces los españoles que conceder a los franceses lo
mismo que querían, era muy desairado para nuestras armas, y acordaron
considerarles como prisioneros de guerra, obligándoles a entregar las
armas. Pero aún el día 21 los contratantes del lado francés, generales
Chabert y Marescot, y los del lado español, Castaños y conde de Tilly,
no habían llegado a ponerse de acuerdo sobre las particularidades de la
rendición.

También alcanzamos a ver a lo largo del camino la interminable fila de
carros donde los imperiales llevaban todo lo cogido en Córdoba.
¡Funestas riquezas! Dicen algunos historiadores que si los franceses no
hubieran llevado botín tan numeroso, habrían podido salvarse retirándose
por la sierra; pero que el afán de no dejar atrás aquellos quinientos
carros llenos de riquezas les puso en el aprieto de rendirse, con la
esperanza de salvar el convoy. Yo no creo que los franceses hubieran
podido escaparse con carros ni sin carros, porque allí estábamos
nosotros para impedírselo; pero sea lo que quiera, lo cierto es que
Napoleón dijo algún tiempo después a Savary en Tolosa, hablando de aquel
desastre tan funesto al Imperio:

«\emph{Más hubiera querido saber su muerte que su deshonra. No me
explico tan indigna cobardía sino por el temor de comprometer lo que
había robado\footnote{«Je ne m'explique cette indigne lâcheté que par la
  crainte de compromettre ce que l'on avait volé.» (\emph{Mem}. Duc
  Rovigo, vol.~IV.)}»}.

No nos atrevimos a volver a la casa con la mala noticia de que el niño
no parecía, y seguimos visitando todos los contornos, para preguntar a
la gente del campo. D. Paco estaba tan fatigado, que no pudiendo dar un
paso más se arrojó al suelo; pero al fin pudimos reanimarle, y firmes en
nuestra santa empresa, nos dirigimos al campamento de Vedel, con otro
oficio del general Reding. Mas vino la noche y los centinelas no nos
dejaron pasar, viéndonos por esto obligados a diferir nuestra expedición
para el día siguiente muy temprano. Ni Marijuán, ni D. Paco ni yo
teníamos esperanza alguna, y considerábamos al mayorazgo perdido para
siempre.

Desde que amaneció corrían voces de que la capitulación estaba firmada,
y más nos lo hacía creer la circunstancia de que varios oficiales
pasaron frecuentemente de un campo a otro, trayendo y llevando
despachos.

No distábamos mucho de la ermita de San Cristóbal, cuando advertimos
gran movimiento en el ejército de Vedel. Apretando el paso hasta que les
tuvimos muy cerca, observamos que camino abajo venía hacia nosotros un
joven saltando y jugando, con aquella volubilidad y ligereza propia de
los chicos al salir de la escuela. Corría a ratos velozmente, luego se
detenía y acercándose a los matorrales sacaba su sable y la emprendía a
cintarazos con un chaparro o con una pita; luego parecía bailar,
moviendo brazos y piernas al compás de su propio canto, y también echaba
al aire su sombrero portugués para recogerlo en la punta del sable.

---¡Qué veo!---exclamó D. Paco con súbita exaltación.---¿No es aquel
mozalbete el propio D. Diego, no es mi niño querido, la joya de la casa,
la antorcha de los Rumblares\ldots? Eh\ldots{} D. Dieguito, aquí
estamos\ldots{} venid acá.

En efecto, cuando estuvimos cerca, no nos quedó duda de que el mozuelo
bailarín era D. Diego en persona. Él nos vio y al punto vino corriendo
para abrazarnos a todos con mucha alegría.

---Venid acá, venid a mis brazos, esperanza del mundo---exclamó D. Paco,
loco de contento.---¡Si supiera Vd. cómo está mamá! ¡Buen susto nos ha
dado el picaroncillo!\ldots{} ¿Pero qué ha sido eso, niño? ¿Estaba usía
prisionero?

---Me cogieron prisionero junto a la ermita---dijo D. Diego.---¿Pero
estás vivo, Gabriel, y tú también, Marijuán? Yo creí que os habían
matado en aquella furiosa carga. ¿Y Santorcaz?\ldots{} Pero os contaré
lo que me pasó. Después de la carga, y cuando entró la caballería de
España, quedé a retaguardia del regimiento; se me murió el caballo y
corrí a las filas del regimiento de Irlanda. Cuando vinimos aquí nos
cogieron prisioneros los franceses, y yo les dije tantas picardías que
quisieron fusilarme.

---¡Qué horror!---exclamó D. Paco.---Pero veo que es Vd. un héroe, oh mi
niño querido. Creo que la mamá piensa dirigir una exposición a la Junta
para que le den a Vd. la faja de capitán general.

---Me iban a fusilar---continuó el rapaz,---cuando un oficial francés
tuvo lástima de mí y me salvó la vida. Después lleváronme a sus tiendas
donde me dieron vino, y\ldots{}

---Vamos, vamos pronto a casa, y allí contará Vd. todo---dijo D.
Paco.---¡Qué alegría! Volemos, señores. ¡Cuando la señora condesa sepa
que le hemos encontrado!\ldots{} ¡Ah! ¿No sabe Vd. que está ahí su
novia?\ldots{} ¡Qué guapísima es!\ldots{} La pobre no cesa de llorar la
ausencia del niño, y si no hubiese Vd. parecido, creo que la tendríamos
que amortajar. Vamos, vamos al punto.

Corrimos todos a Bailén muy contentos. Al llegar al pueblo, uno de
nosotros propuso anticiparse para anunciar a doña María la fausta nueva;
pero no permitió D. Paco que nadie sino él en persona se encargase de
tan dulce comisión, y con sus piernas vacilantes corrió hasta entrar en
la casa diciendo con desaforados gritos:---¡Ya pareció, ya pareció!
Cuando nosotros llegamos con el joven, todos salieron a recibirle,
excepto Amaranta, a quien un fuerte dolor de cabeza retenía en su
cuarto. Era de ver cómo los criados, las hermanitas y la misma doña
María, sin poder contener en los límites de la dignidad su maternal
cariño, le abrazaban y besaban a porfía; y uno le coge, otro le deja,
durante un buen rato le estrujaron sin compasión. Al fin reuniéndose
todos, inclusos los huéspedes en la sala baja, don Diego fue
solemnemente presentado a su novia. No puedo olvidar aquella escena que
presencié desde la puerta con otros criados, y voy a referirla.

\hypertarget{xxxii}{%
\chapter{XXXII}\label{xxxii}}

Inés, confusa y ruborosa, no contestó nada, cuando el diplomático se fue
derecho a ella llevando de la mano a D. Diego, y le dijo:

---Hija mía, aquí tienes al que te destinamos por esposo: mi sobrino,
varón ilustre, a quien veremos general dentro de poco como siga la
guerra.

---Hijo mío---añadió doña María,---las altas prendas de la que va a ser
irremisiblemente tu mujer no necesitan ser ponderadas en esta ocasión,
porque harto las conocemos todos. Ahora, con el trato, se avivará el
inmenso cariño que os profesáis desde hace algunos años, señal evidente
de que Dios tenía decidida ya vuestra unión en sus altos designios.

---Bonito es el retrato---dijo D. Diego con un desenfado impropio de la
situación;---pero Vd., Inés, lo es más todavía. ¿Y en qué consistía el
no querer salir del maldito convento? Sin duda las pícaras monjas la
retenían a Vd. por fuerza, esperando que al profesar les llevara un buen
dote. Pero no, yo juro que estaba decidido a sacar de allí a mi monjita,
y ya discurría el modo de saltar por las tapias de la huerta y romper
rejas y celosías para conseguir mi objeto.

Doña María, al escuchar esto, palideció, y luego las centellas de la ira
brillaron en sus ojos. Pero con disimulo habló de otro asunto,
procurando que el noble concurso y discreto senado olvidara las palabras
del incipiente chico.

---Pero cuéntanos de una vez lo que te ha pasado en el campamento
francés---dijo a D. Diego.

---Pues me querían fusilar---repuso el mayorazgo sentándose.---Ya me
tenían puesto de rodillas, cuando un oficial mandó suspender la
ejecución.

---¿Y por qué te querían asesinar esos cafres?

---Porque les dije mil perrerías. Después, cuando me llevaron a la
tienda, todos se reían de mí. Luego me dieron vino, obligándome a
beberlo, y yo mientras más bebía más charlaba, diciendo atroces
disparates y frases graciosas, hasta que me quedé como un cuerpo muerto.

---¿Y no sabes tú---exclamó doña María sin poder disimular su
indignación,---que las personas de buena crianza no beben sino poquito?

---Es verdad; pero aquel vino tenía un saborcillo que me gustaba, y los
franceses se reían mucho conmigo. Todos iban a verme, llamándome
\emph{le petit espagnol}.

---Lo cual, en la lengua de las Galias, quiere decir \emph{el pequeño
español}---dijo D. Paco.

---Pero no debió Vd. dejarse emborrachar, joven---indicó el
diplomático.---Juro que si eso hubiera pasado conmigo, de un sablazo
descalabro a todos los oficiales de la división de Vedel.

Doña María, profundamente indignada, silenciosa, ceñuda, parecía una
sibila de Miguel Ángel.

---Pero si todos aquellos señores me querían mucho\ldots---continuó D.
Diego.---Por la tarde, y luego que desperté de aquel largo sueño, me
dijeron que si sabía yo lidiar un toro. Díjeles que sí, y poniéndose muy
contentos, me mandaron que diese al punto una corrida. No quería yo más
para divertirme; así es que, poniendo una silla en lugar de toro, le
capeé, le puse banderillas y le di muerte con mi sable, pasándole de
parte a parte. ¡Cuánto se rieron aquellos condenados! Hasta el general
acudió a verme.

---Veo que has aprovechado el tiempo en el campamento francés---dijo la
señora madre con tremenda ironía.

---Si no me querían dejar venir. Después me dijeron que les cantara el
jaleo, y lo canté de pie sobre una banqueta. ¡Ave-María purísima! Hasta
los soldados se acercaban a la tienda para oír. Entre los oficiales
había dos que no me dejaban de la mano, y me decían que si me pasaba al
ejército francés, me tomarían por ayudante, llevándome a Francia, a
París, y de París a recorrer toda la Europa.

---¡Y no les distes una bofetada!---exclamó doña María clavando sus
dedos en el cuero del sillón.

---¡Quia! Me eché a reír y les dije que ya pensaba ir a Francia con el
Sr.~de Santorcaz, que es mi amigo y ha de ser mi ayo y maestro cuando me
case.

Esta vez no fue doña María la que se estremeció de sorpresa e
indignación; fue la marquesa de Leiva, quien mudando el color y con
absortos ojos miró sucesivamente a su prima, a su sobrino y al ayo.

---Pero ¿qué está diciendo el niño?---preguntó este mirando a la
condesa.---¿Quién dice que es su maestro y su amigo?

---Cualquiera menos Vd.---contestó insolentemente el heredero.---¡Vaya
un maestro, que no sabe enseñar sino mentecatadas y simplezas!

---¡Jesús! Diego, repara que estás\ldots---dijo doña María conteniendo
con grandes esfuerzos los gestos amenazadores, natural expresión de su
ira.

D. Paco se llevó el pañuelo a los ojos para enjugar una lágrima. Inés
atendía a todo discretamente y sin hablar. ¡Ah! Mientras allí la
juzgaban indiferente al peligroso diálogo, ¡qué admirables
observaciones, qué exactos juicios haría en aquellos momentos ante
semejante escena! Su talento y alto criterio dominarían sobre las
pasiones, los errores y las querellas de la histórica familia como el
sol inmutable sobre la volteadora tierra.

Asunción y Presentación, que aguardaban coyuntura para dar expansión al
comprimido gozo de sus almas, hubieran querido reír como su hermano,
pero la seriedad de su madre las tenía mudas de terror.

---Esta predisposición de Vd.---dijo el marqués,---a visitar las cortes
europeas me indica que se siente el niño con inclinaciones a la
diplomacia. Hija mía---añadió dirigiéndose a Inés,---cada vez descubro
más eminentes cualidades en el que te destinamos por esposo, y veo
justificado el amor que desde hace tiempo en silencio le profesas, y
que, en tu castidad y delicadeza, procuras disimular hasta el último
instante.

---¡Ah!, se me olvidaba decir---exclamó D. Diego riendo a
carcajadas,---que los franceses me han enseñado a decir algunas palabras
en su lengua.

Y levantándose al punto, hizo profundas reverencias ante Inés,
diciéndole:

---\emph{Ponchú, madama. ¿Como la porta bú}?

Asunción y Presentación después de mirarse una a otra creyeron que había
llegado el momento de reír, y rieron dando desahogo a sus oprimidos
corazones; pero como doña María no desplegó sus labios, las dos
muchachitas tuvieron que ponerse serias otra vez.

---¡Oh! ¡\emph{Tres bien}!---dijo el diplomático.---Señor D. Francisco,
su alumno de Vd. demuestra las luces y copiosa doctrina del eruditísimo
maestro.

Hizo D. Paco una graciosa reverencia, y su rostro compungido y lloroso
se esclareció con una sonrisa.

Doña María callaba; pero en su pecho rugía iracunda y atormentadora la
tempestad. Ella y su prima la de Leiva se miraban de vez en cuando,
transmitiéndose una a otra el fuego de sus coléricos sentimientos.

---Otras muchas palabras sé---continuó el rapaz;---como \emph{Crenom de
Dieu}, \emph{Sacrebleu}, exclamaciones que se dicen cuando uno está
rabioso, en vez de \emph{¡Caracoles!} \emph{¡Canastos!}

Doña María se levantó de su asiento\ldots{} y se volvió a sentar.

---¡Cómo me querían aquellos demonios de franceses! Uno de ellos sabía
español y hablaba a ratos conmigo. Me dijo que los españoles eran muy
valientes y muy honrados; pero que hacían mal en defender a Fernando
VII, porque este príncipe es un farsantuelo que engañó a su padre y
ahora está engañando a la Nación y al Emperador.

Doña María se llevó la mano a los ojos.

---Yo le aseguré que los españoles les echaríamos de España, y él me
contestó que parecía probable, porque la guerra iba tomando mal aspecto;
pero que esto sería un mal para nosotros, porque de venir otra vez
Fernando VII, España seguiría con su mal Gobierno, y con las muchas
cosas perversas, injustas y anticuadas que hay aquí.

---¡Oh! ¿Y no se le ocurrió a Vd. la contestación a tan atrevido y
antipatriótico aserto?---preguntó con énfasis el diplomático.

---Yo le dije que aquí íbamos ahora a arreglar todas esas cosas, y a
quitar la santa Inquisición, y los diezmos, y los mayorazgos, como me
decía el Sr.~de Santorcaz.

Doña María aferró sus manos a los brazos de la silla como si quisiera
estrujar la madera entre sus dedos.

---Sobre todo los mayorazgos---prosiguió Rumblar.---También le dije al
francés que yo soy mayorazgo y que después de casado tendré dos
vinculaciones. ¡Cómo se reía cuando le dije que era Grande de España!
Todos acudían a verme y me volvieron a dar de beber, y me caí otra vez
al suelo cantando que me las pelaba.

¡Ay! Doña María se llevó las manos a la cabeza, doña María cerró los
ojos, doña María golpeó el suelo con su pie derecho, doña María semejaba
la imponente imagen de la tradición aplastando la hidra revolucionaria.

---Esta mañana me preguntaron si yo tenía hermanas guapas. Díjeles que
eran muy bonitas, y luego me dijeron que vendrían a verlas, y que si se
las quería dar para casarse con ellas, puesto que también serían
mayorazgas. Yo les contesté que mayorazgo era el que había nacido
primero.

Y luego dirigiéndose a sus hermanitas, les dijo:

---Os fastidiasteis, chicas, por haber nacido hembras y después que yo.
Una de Vds. se casará con cualquier pelele, y la otra se meterá en un
conventito a rezar por nosotros los pecadores, a no ser que algún día
vea un galán por la reja, y se enamore, y luego se tire por la ventana a
la calle.

Doña María no podía resistir más. Iba a estallar su furibunda cólera;
pero aún era mayor el caudal de su prudencia que el caudal de su
enojo\ldots{} se contuvo y cerró otra vez los ojos ya que no podía
cerrar los oídos.

---Después---siguió el mancebo,---me dijeron si mis hermanas usaban
navaja, si tocaban la guitarra, si iban a los toros y si yo era familiar
de la Inquisición. ¡Cómo se reían aquellos condenados! Lo gracioso es
que no me dejaban salir de allí, y a cada rato me decían \emph{so},
\emph{so}, \emph{so}.

---\emph{Un sot}---dijo el diplomático.---Pues sospecho que os llamaron
tonto. ¡Oh iniquidad de la Nación francesa! ¡Vea Vd., Sr.~D. Paco, lo
que es un pueblo carcomido por el jacobinismo!\ldots{} ¿Y no les dio Vd.
un par de sablazos?

---Si me querían mucho. Anoche me tuvieron toda la noche bailando el
bolero y la cachucha, en medio de un corrillo donde había más de
cuarenta oficiales.

Asunción y Presentación seguían esperando con ansia la ocasión de reír;
pero esta ocasión no llegaba, y consultando el rostro de su madre,
veíanle cada vez más borrascoso. Así es que las dos estaban muertas de
miedo.

D. Paco, conociendo que se preparaba un cataclismo, quiso conjurarlo y
dijo a su discípulo:

---Vamos, basta de franceses, D. Diego. Hable Vd. de otra cosa. Si no
fuera demasiado largo, os mandaría que recitarais aquel capítulo sobre
la batalla del Gránico que os hice aprender de memoria; mas para que tan
escogido concurso, y especialmente este fresco azahar de Andalucía,
vuestra prometida; para que todos, en una palabra, puedan apreciar la
buena pronunciación de Vd. y su cadencioso oído, échenos cualquiera de
esos romances que sabe\ldots{} vamos. Atención, señores.

---El del \emph{Barandal del cielo}---dijo Asunción respirando con
alegría.

---El de los \emph{Santos pechos}---dijo Presentación.

---Vamos, no se haga Vd. de rogar.

---Pues voy a echarles una canción que me enseñaron los franceses.

---No, nada de franceses.

---Si es muy bonita, aunque a decir verdad, yo no la entiendo.

Y sin esperar más, púsose en pie D. Diego, y accionando como un cómico,
con voz fuerte y exaltado acento, cantó así:

Allons enfants de la patrie le jour de gloire est arrivé! Contre nous de
la tyrannie l'etendart sanglant est levé!

\small
\newlength\mlenc
\settowidth\mlenc{   Allons enfants de la patrie}
\begin{center}
\parbox{\mlenc}{\textit{   Allons enfants de la patrie     \\
                le jour de gloire est arrivé!              \\
                Contre nous de la tyrannie                 \\
                l'étendard sanglant est levé!}}            \\
\end{center}
\normalsize

Asunción y Presentación reían como locas, y doña María no dijo nada.
Ninguno de la familia había entendido una palabra.

---Es bonita la canción---dijo D. Paco,---pero no la comprendemos.

Entonces el diplomático levantose ceremoniosa y gravemente, y tomando un
tono de hombre severo habló así:

---¿Sabe Vd. lo que está cantando? Pues está cantando la
\emph{Marsellesa}, esa canción impía y sanguinaria, señores, esa canción
que acompañó al suplicio a todos los mártires de la revolución, incluso
Luis XVI, mi querido amigo\ldots{} porque han de saber Vds. que Luis XVI
y yo teníamos muchas bromas y nos echábamos el brazo por el hombro
paseándonos por Versalles\ldots{} ¡La \emph{Marsellesa}, señores, la
\emph{Marsellesa}! También acompañó al cadalso a María Antonieta\ldots{}
¡y qué buena era aquella señora! ¡Cuántas veces la vi marcando pañuelos
en una ventana baja del pequeño Trianon! ¡Cómo me quería!\ldots{} En
fin, este joven me ha horripilado con la tal tonadilla\ldots{} Señora
condesa, ¿está Vd. indispuesta? ¿Y tú, hermana? El caso no es para
menos. Hija mía, ¿estás nerviosa? ¿Te has puesto mala? ¿Te causa miedo
esa canción?

Inés le contestó que no tenía ni pizca de miedo. En tanto doña María, no
pudiendo resistir más salió del cuarto con sus niñas. Desconcertose al
punto aquella ilustre reunión, y luego no quedó en la sala más que la
familia de Inés con D. Diego. Al poco rato tuvo lugar una escena
lamentable, y fue que doña María, ciega de furor, y necesitando
desahogar aquella tormenta de su espíritu sobre alguien, descargó su
enojo al fin; ¿pero sobre quién, santo Dios?, ¿sobre quién?, dirán
Vds\ldots{} Sobre las dos inocentes muchachas, sobre los dos angelitos
celestiales, Asunción y Presentación. ¿Y todo por qué? Porque
entusiasmadillas con la llegada de su hermano, habían dejado de hacer no
sé qué cosa encomendada a sus tiernas manos. ¡Pobres pimpollitos! La
dignidad impedía a mi señora la condesa castigar al primogénito delante
de la novia y del suegro, y era forzoso que pagaran el pato las dos
niñas desheredadas. Yo las vi llorando como unas Magdalenas y soplándose
las palmas de las manos, escaldadas por aquel fatídico instrumento de
cinco agujeros que pendía de fatal espetera en el despacho de D. Paco.
Las pobrecillas estuvieron a moco y baba todo el día.

\hypertarget{xxxiii}{%
\chapter{XXXIII}\label{xxxiii}}

Este libro va a concluir, queridísimos lectores, a quienes adoro y
reverencio; va a concluir, y los notables y jamás vistos sucesos que me
acontecieron en virtud del proyectado matrimonio de Inés y del encuentro
de aquellas dos familias en el tortuoso y difícil camino de mis amores,
serán escritos, por no caber en este volumen, en otro que pondré a
vuestra disposición lo más pronto posible. Tened, pues, un adarme de
paciencia, y mientras aquellas distinguidas personas se preparan para
ponerse en camino hacia Madrid, a donde con vuestra venia pienso
acompañarlas, atended un poco más.

El mismo día 22 encontré a Santorcaz puesto ya al frente de su
partidilla, la cual, como he dicho, estaba formada de lo mejorcito del
país. Les digo a Vds. que tropa más escogida que aquella no la
capitanearon los famosos caballistas José María y Diego Corrientes.

---¿Va Vd. ya de marcha?---le dije.

---Sí; dispusieron que fuera alguna fuerza de paisanos a guardar el paso
de Despeñaperros, y yo solicité esta comisión que me agrada mucho. Allá
voy con mi gente. ¿Quieres venir? ¿Has estado en casa de Rumblar?

---De allá vengo.

---¿Y esa familia que está ahí es la de la novia de D. Diego?

---Justamente.

---Creo que van todos para Madrid.

---Así parece.

---¿No sabes cuándo?

---Según he oído, pasado mañana. Esperan saber lo de la capitulación
para llevar la noticia.

---¿Conque pasado mañana? Bien\ldots{} adiós. ¿Quieres venir en mi
partida?

---Gracias; adiós.

Les vi partir, y todo el día y toda la noche estuve pensando en aquella
gente.

Yo no vi el triste desfile de los ocho mil soldados de Dupont cuando
entregaron sus armas ante el general Castaños, porque esto tuvo lugar en
Andújar. A pesar de que la primera y segunda división habían sido las
vencedoras de los franceses, la honra de presenciar la rendición fue
otorgada a la tercera y a la de reserva, por una de esas injusticias tan
comunes en nuestra tierra, lo mismo en estos días de vergüenza que en
aquellos de gloria. Por delante de nosotros desfilaron las tropas de
Vedel, en número de nueve mil trescientos hombres, y dejando sus armas
en pabellón, nos entregaron muchas águilas y cuarenta cañones.

Les mirábamos y nos parecía imposible que aquellos fueran los vencedores
de todo el mundo. Después de haber borrado la geografía del continente
para hacer otra nueva, clavando sus banderas donde mejor les pareció,
desbaratando imperios, y haciendo con tronos y reyes un juego de
titiriteros, tropezaban en una piedra del camino de aquella remota
Andalucía, tierra casi olvidada del mundo desde la expulsión del
islamismo. Su caída hizo estremecer de gozosa esperanza a todas las
Naciones oprimidas. Ninguna victoria francesa resonó en Europa tanto
como aquella derrota, que fue sin disputa el primer traspiés del
Imperio. Desde entonces caminó mucho, pero siempre cojeando.

España, armándose toda y rechazando la invasión con la espada y la tea,
con la navaja, con las uñas y con los dientes, iba a probar, como dijo
un francés, que los ejércitos sucumben, pero que las Naciones son
invencibles.

---¡Cuánto siento que no esté aquí el Sr.~de Santorcaz!---me dijo
Marijuán al ver pasar por delante de nosotros a aquellos hermosos
soldados, medio muertos de fatiga y de vergüenza.---¿Te acuerdas de las
grandes bolas que nos contaba cuando veníamos por la Mancha y nos
refería las batallas ganadas por estos contra todo el mundo?

---Lo que nos contaba Santorcaz---respondí,---era pura verdad; pero esto
que ahora vemos, amigo Marijuán, también es verdad.

\hypertarget{xxxiv}{%
\chapter{XXXIV}\label{xxxiv}}

Y ahora consideren Vds. lo que pasaba del otro lado de Sierra-Morena en
aquel mismo mes de Julio. El día 7 había jurado José en Bayona la
Constitución hecha por unos españoles vendidos al extranjero. El día 9
el mismo José traspasaba la frontera para venir a gobernarnos. El día 15
ganaba Bessières en los campos de Rioseco una sangrienta batalla, y al
tener de ella noticia Napoleón, decía lleno de gozo: «La batalla de
Rioseco pone a mi hermano en el trono de España, como la de Villaviciosa
puso a Felipe V».

Napoleón partió para París el 21, creyendo que lo de España no ofrecía
cuidado alguno. El 20, un día después de nuestra batalla, entró José en
Madrid, y aunque la recepción glacial que se le hizo le causara suma
aflicción, aún le parecía que el buen momio de la corona duraría
bastante tiempo.

Pero hacia los días 25, 26 y 27 se esparce por la capital un rumor
misterioso que conmueve de alegría a los españoles y llena de terror a
los franceses; corre la voz de que los paisanos andaluces y algunas
tropas de línea han derrotado a Dupont, obligándole a capitular. Este
rumor crece y se extiende; pero nadie lo quiere creer, los españoles por
parecerles demasiado lisonjero, y los franceses por considerarlo
demasiado terrible. El absurdo se propaga y parece confirmarse; pero la
corte de José se ríe y no da crédito a aquel cuento de viejas. Cuando no
queda duda de que semejante imposible es un hecho real, la corte que aún
no había instalado sus bártulos, huye despavorida; las tropas de Moncey,
que rechazadas de Valencia se habían replegado a la Mancha, se unen a
las de Madrid, y todos juntos, soldados, generales y Rey intruso, corren
precipitadamente hacia el Norte, asolando el país por donde pasan. Aquel
fantasma de reino napoleónico se disipaba como el humo de un cañonazo.

Y ahora os he de hablar de cómo la guerra que parecía próxima a
concluir, se trabó de nuevo con más fuerzas; os he de hablar de aquel
infeliz y bondadoso rey José y de su corte, y de su hermano, y del paso
de Somosierra con la famosa carga de los lanceros polacos, y del sitio
de Madrid, y de otras muchas curiosísimas cosas; pero todo se ha de
quedar para el libro siguiente, donde estos históricos sucesos han de
tener feliz consorcio con los no menos dramáticos de mi vida, y todo lo
mucho y bueno que ocurrió en el matrimonio de Inés. Por ahora guardaré
prudente silencio sobre estos sucesos, pues decidido estoy a seguir al
pie de la letra la reservadísima escuela del diplomático; y así os digo:

«No, no me obliguéis a hablar, no me obliguéis, abusando de la dulce
amistad, a que revele estos secretos de que tal vez depende la suerte
del mundo. No me seduzcáis con ruegos y cariñosas sugestiones que en
vano atacan el inexpugnable alcázar de mi discreción».

A pesar de esto, ¿insistís, importunos amigos? Nada más os digo por
ahora, sino que la familia de Inés salió para Madrid hacia fin de mes y
en los días en que el ejército vencedor marchaba también hacia la
capital de España. Esta circunstancia me permitió ir en la escolta que
por el camino debía custodiar a tan esclarecida comitiva; así es que
formé con los diez de a caballo que galopaban a la zaga de los dos
coches. ¡Ay! Por la portezuela de uno de ellos solía asomarse durante
las paradas una linda cabeza, cuyos ojos se recreaban en la marcial
apostura del pequeño escuadrón.

---Estos valerosos muchachos, hija mía---le decía su padre,---son los
que en los campos de Bailén echaron por tierra con belicosa furia al
coloso de Europa. Veo que les miras mucho, lo cual me prueba tu
entusiasmo por las glorias patrias.

Basta con esto, señores, y no digo más. En vano me hacen Vds. señas,
excitándome a hablar; en vano fingen conocer mentirosos hechos, para que
yo les cuente los verídicos. ¿A qué conduce el anticipar la relación de
lo que no es de este lugar? A los impacientes les diré que nada ocurrió
hasta que llegamos al desfiladero de Despeñaperros. Lo pasábamos en una
noche muy oscura, cuando de pronto detuviéronse los coches, oímos
gritos, sonó un tiro, y algunos hombres de muy mal aspecto, saltando
desde los cercanos matorrales, se arrojaron al camino. Al instante
corrimos sable en mano hacia ellos\ldots{} pero basta ya, y déjenme
dormir, pues ni con tenazas me han de sacar una palabra más.

FIN

\flushright{Octubre-Noviembre de 1873.}

~

\bigskip
\bigskip
\begin{center}
\textsc{Fin de Bailén}
\end{center}

\end{document}
