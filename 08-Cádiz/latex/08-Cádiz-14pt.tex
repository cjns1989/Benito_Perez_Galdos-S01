\PassOptionsToPackage{unicode=true}{hyperref} % options for packages loaded elsewhere
\PassOptionsToPackage{hyphens}{url}
%
\documentclass[oneside,14pt,spanish,]{extbook} % cjns1989 - 27112019 - added the oneside option: so that the text jumps left & right when reading on a tablet/ereader
\usepackage{lmodern}
\usepackage{amssymb,amsmath}
\usepackage{ifxetex,ifluatex}
\usepackage{fixltx2e} % provides \textsubscript
\ifnum 0\ifxetex 1\fi\ifluatex 1\fi=0 % if pdftex
  \usepackage[T1]{fontenc}
  \usepackage[utf8]{inputenc}
  \usepackage{textcomp} % provides euro and other symbols
\else % if luatex or xelatex
  \usepackage{unicode-math}
  \defaultfontfeatures{Ligatures=TeX,Scale=MatchLowercase}
%   \setmainfont[]{EBGaramond-Regular}
    \setmainfont[Numbers={OldStyle,Proportional}]{EBGaramond-Regular}      % cjns1989 - 20191129 - old style numbers 
\fi
% use upquote if available, for straight quotes in verbatim environments
\IfFileExists{upquote.sty}{\usepackage{upquote}}{}
% use microtype if available
\IfFileExists{microtype.sty}{%
\usepackage[]{microtype}
\UseMicrotypeSet[protrusion]{basicmath} % disable protrusion for tt fonts
}{}
\usepackage{hyperref}
\hypersetup{
            pdftitle={CÁDIZ},
            pdfauthor={Benito Pérez Galdós},
            pdfborder={0 0 0},
            breaklinks=true}
\urlstyle{same}  % don't use monospace font for urls
\usepackage[papersize={4.80 in, 6.40  in},left=.5 in,right=.5 in]{geometry}
\setlength{\emergencystretch}{3em}  % prevent overfull lines
\providecommand{\tightlist}{%
  \setlength{\itemsep}{0pt}\setlength{\parskip}{0pt}}
\setcounter{secnumdepth}{0}

% set default figure placement to htbp
\makeatletter
\def\fps@figure{htbp}
\makeatother

\usepackage{ragged2e}
\usepackage{epigraph}
\renewcommand{\textflush}{flushepinormal}

\usepackage{indentfirst}

\usepackage{fancyhdr}
\pagestyle{fancy}
\fancyhf{}
\fancyhead[R]{\thepage}
\renewcommand{\headrulewidth}{0pt}
\usepackage{quoting}
\usepackage{ragged2e}

\newlength\mylen
\settowidth\mylen{...................}

\usepackage{stackengine}
\usepackage{graphicx}
\def\asterism{\par\vspace{1em}{\centering\scalebox{.9}{%
  \stackon[-0.6pt]{\bfseries*~*}{\bfseries*}}\par}\vspace{.8em}\par}

 \usepackage{titlesec}
 \titleformat{\chapter}[display]
  {\normalfont\bfseries\filcenter}{}{0pt}{\Large}
 \titleformat{\section}[display]
  {\normalfont\bfseries\filcenter}{}{0pt}{\Large}
 \titleformat{\subsection}[display]
  {\normalfont\bfseries\filcenter}{}{0pt}{\Large}

\setcounter{secnumdepth}{1}
\ifnum 0\ifxetex 1\fi\ifluatex 1\fi=0 % if pdftex
  \usepackage[shorthands=off,main=spanish]{babel}
\else
  % load polyglossia as late as possible as it *could* call bidi if RTL lang (e.g. Hebrew or Arabic)
%   \usepackage{polyglossia}
%   \setmainlanguage[]{spanish}
%   \usepackage[french]{babel} % cjns1989 - 1.43 version of polyglossia on this system does not allow disabling the autospacing feature
\fi

\title{CÁDIZ}
\author{Benito Pérez Galdós}
\date{}

\begin{document}
\maketitle

\hypertarget{i}{%
\chapter{I}\label{i}}

En una mañana del mes de Febrero de 1810 tuve que salir de la Isla,
donde estaba de guarnición, para ir a Cádiz, obedeciendo a un aviso tan
discreto como breve que cierta dama tuvo la bondad de enviarme. El día
era hermoso, claro y alegre cual de Andalucía, y recorrí con otros
compañeros, que hacia el mismo punto si no con igual objeto caminaban,
el largo istmo que sirve para que el continente no tenga la desdicha de
estar separado de Cádiz; examinamos al paso las obras admirables de
Torregorda, la Cortadura y Puntales, charlamos con los frailes y
personas graves que trabajaban en las fortificaciones; disputamos sobre
si se percibían claramente o no las posiciones de los franceses al otro
lado de la bahía; echamos unas cañas en el figón de Poenco, junto a la
Puerta de Tierra, y finalmente, nos separamos en la plaza de San Juan de
Dios, para marchar cada cual a su destino. Repito que era en Febrero, y
aunque no puedo precisar el día, sí afirmo que corrían los principios de
dicho mes, pues aún estaba calentita la famosa respuesta: «La ciudad de
Cádiz, fiel a los principios que ha jurado, no reconoce otro rey que al
señor D. Femando VII. 6 de Febrero de 1810.»

Cuando llegué a la calle de la Verónica, y a la casa de doña Flora, esta
me dijo:

---¡Cuán impaciente está la señora condesa, caballerito, y cómo se
conoce que se ha distraído usted mirando a las majas que van a alborotar
a casa del señor Poenco en Puerta de Tierra!

---Señora---le respondí,---juro a usted que fuera de Pepa Hígados, la
Churriana, y María de las Nieves, la de Sevilla, no había moza alguna en
casa de Poenco. También pongo a Dios por testigo de que no nos detuvimos
más que una hora y esto porque no nos llamaran descorteses y malos
caballeros.

---Me gusta la frescura con que lo dice---exclamó con enfado doña
Flora.---Caballerito, la condesa y yo estamos muy incomodadas con usted,
sí señor. Desde el mes pasado en que mi amiga acertó a recoger en el
Puerto esta oveja descarriada, no ha venido usted a visitarnos más que
dos o tres veces, prefiriendo en sus horas de vagar y esparcimiento la
compañía de soldados y mozas alegres, al trato de personas graves y
delicadas que tan necesario es a un jovenzuelo sin experiencia. ¡Qué
sería de ti---añadió reblandecida de improviso y en tono de
confianza,---tierna criatura lanzada en tan temprana edad a los
torbellinos del mundo, si nosotras, compadecidas de tu orfandad, no te
agasajáramos y cuidáramos, fortaleciéndote a la vez el cuerpecito con
sanos y gustosos platos, el alma con sabios consejos! Desgraciado
niño\ldots{} Vaya se acabaron los regaños, picarillo. Estás perdonado;
desde hoy se acabó el mirar a esas desvergonzadas muchachuelas que van a
casa de Poenco y comprenderás todo lo que vale un trato honesto y
circunspecto con personas de peso y suposición. Vamos, dime lo que
quieres almorzar. ¿Te quedarás aquí hasta mañana? ¿Tienes alguna herida,
contusión o rasguño, para curártelo en seguida? Si quieres dormir, ya
sabes que junto a mi cuarto hay una alcobita muy linda.

Diciendo esto, doña Flora desarrollaba ante mis ojos en toda su
magnificencia y extensión el panorama de gestos, guiños, saladas muecas,
graciosos mohínes, arqueos de ceja, repulgos de labios y demás signos
del lenguaje mudo que en su arrebolado y con cien menjurjes albardado
rostro servía para dar mayor fuerza a la palabra. Luego que le di mis
excusas, dichas mitad en serio mitad en broma, comenzó a dictar órdenes
severas para la obra de mi almuerzo, atronando la casa, y a este punto
salió conteniendo la risa la señora condesa que había oído la anterior
retahíla.

---Tiene razón---me dijo después que nos saludamos;---el Sr.~D. Gabriel
es un chiquilicuatro sin fundamento, y mi amiga haría muy bien en
ponerle una calza al pie. ¿Qué es eso de mirar a las chicas bonitas?
¿Hase visto mayor desvergüenza? Un barbilindo que debiera estar en la
escuela o cosido a las faldas de alguna persona sentada y de libras que
fuera un almacén de buenos consejos\ldots{} ¿cómo se entiende? Doña
Flora, siéntele usted la mano, dirija su corazón por el camino de los
sentimientos circunspectos y solemnes, e infúndale el respeto que todo
caballero debe tener a los venerandos monumentos de la antigüedad.

Mientras esto decía, doña Flora había traído luengas piezas de damasco
amarillo y rojo y ayudada de su doncella empezó a cortar unas como
dalmáticas o jubones a la antigua, que luego ribeteaban con galón de
plata. Como era tan presumida y extravagante en su vestir, creí que doña
Flora preparaba para su propio cuerpo aquellas vestimentas; pero luego
conocí, viendo su gran número, que eran prendas de comparsa de teatro,
cabalgata o cosa de este jaez.

---¡Qué holgazana está usted, señora condesa!---dijo doña Flora,---y
¿cómo teniendo tan buena mano para la aguja no me ayuda a hilvanar estos
uniformes para la \emph{Cruzada del Obispado de Cádiz}, que va a ser el
terror de la Francia y del Rey José?

---Yo no trabajo en mojigangas, amiguita---repuso mi antigua ama,---y de
picarme las manos con la aguja, prefiero ocuparme, como me ocupo, en la
ropa de esos pobrecitos soldados que han venido con Alburquerque de
Extremadura, tan destrozados y astrosos que da lástima verlos. Estos y
otros como estos, amiga doña Flora, echarán a los franceses, si es que
les echan, que no los monigotes de la Cruzada, con su D. Pedro del
Congosto a la cabeza, el más loco entre todos los locos de esta tierra,
con perdón sea dicho de la que es su tiernísima Filis.

---Niñita mía, no diga usted tales cosas delante de este joven sin
experiencia ---indicó con mal disimulada satisfacción doña Flora;---pues
podría creer que el ilustre jefe de la Cruzada, para quien doy estos
puntos y comas, ha tenido conmigo más relaciones que la de una afición
purísima y jamás manchadas con nada de aquello que D. Quijote llamaba
\emph{incitativo melindre}. Conociome el Sr. D. Pedro en Vejer en casa
de mi primo D. Alonso y desde entonces se prendó de mí de tal modo, que
no ha vuelto a encontrar en toda la Andalucía mujer que le interesara.
Ha sido desde entonces acá su devoción para mí cada vez más fina,
espiritada y sublime, en tales términos que jamás me lo ha manifestado
sino en palabras respetuosísimas, temiendo ofenderme; y en los años que
nos conocemos ni una sola vez me ha tocado las puntas de los dedos.
Mucho ha picoteado por ahí la gente suponiéndonos inclinados a contraer
matrimonio; pero sobre que yo he aborrecido siempre todo lo que sea obra
de varón, el señor D. Pedro se pone encendido como la grana cuando tal
le dicen, porque ve en esas habladurías una ofensa directa a su pudor y
al mío.

---No es tampoco D. Pedro---dijo Amaranta riendo,---con sus sesenta años
a la espalda, hombre a propósito para una mujer fresca y lozana como
usted, amiga mía. Y ya que de esto se trata, aunque le parezcan
irrespetuosas y tal vez impúdicas mis palabras, usted debiera
apresurarse a tomar estado para no dejar que se extinga tan buena casta
como es la de los Gutiérrez de Cisniega; y de hacerlo, debe buscar varón
a propósito, no por cierto un jamelgo empedernido y seco como D. Pedro,
sino un cachorro tiernecito que alegre la casa, un joven, pongo por
caso, como este Gabriel, que nos está oyendo, el cual se daría por muy
bien servido, si lograra llevar a sus hombros carga tan dulce como
usted.

Yo, que almorzaba durante este gracioso diálogo, no pude menos de
manifestarme conforme en todo y por todo con las indicaciones de
Amaranta; y doña Flora sirviéndome con singular finura y amabilidad,
habló así:

---Jesús, amiga, qué malas cosas enseña usted a este pobrecito niño, que
tiene la suerte de no saber todavía más que la táctica de cuatro en
fondo. ¿A qué viene el levantarle los cascos con\ldots? Gabriel, no
hagas caso. Cuidado con que te desmandes, y mal instruido por esta
pícara condesa, vayas ahora a deshacerte en requiebros, y desbaratarte
en suspiros y fundirte en lágrimas\ldots{} Los niños a la escuela. ¡Qué
cosas tiene esta Amaranta! Criatura, ¿acaso el muchacho es de
bronce?\ldots{} Su suerte consiste en que da con personas de tan buena
pasta como yo, que sé comprender los desvaríos propios de la juventud, y
estoy prevenida contra los vehementes arrebatos lo mismo que contra los
lazos del enemigo. Calma y sosiego, Gabriel, y esperar con paciencia la
suerte que Dios destina a las criaturas. Esperar sí, pero sin
fogosidades, sin exaltaciones, sin locuras juveniles, pues nada sienta
tan bien a un joven delicado y caballeroso, como la circunspección. Y si
no aprende de ese Sr.~D. Pedro del Congosto, aprende de él; mírate en el
espejo de su respetuosidad, de su severidad, de su aplomo, de su
impasible y jamás turbado platonismo; observa cómo enfrena sus pasiones;
como enfría el ardor de los pensamientos con la estudiada urbanidad de
las palabras; cómo reconcentra en la idea su afición y pone freno a las
manos y mordaza a la lengua y cadenas al corazón que quiere saltársele
del pecho.

Amaranta y yo hacíamos esfuerzos por contener la risa. De pronto oyose
ruido de pasos, y la doncella entró a anunciar la visita de un
caballero.

---Es el inglés---dijo Amaranta.---Corra usted a recibirle.

---Al instante voy, amiga mía. Veré si puedo averiguar algo de lo que
usted desea.

Nos quedamos solos la condesa y yo por largo rato, pudiendo sin testigos
hablar tranquilamente lo que verá el lector a continuación si tiene
paciencia.

\hypertarget{ii}{%
\chapter{II}\label{ii}}

---Gabriel---me dijo,---te he llamado para decirte que ayer, en una
embarcación pequeña, venida de Cartagena, ha llegado a Cádiz el sin par
D. Diego, conde de Rumblar, hijo de nuestra parienta, la monumental y
grandiosa señora doña María.

---Ya sospechaba---respondí,---que ese perdido recalaría por aquí. ¿No
trae en su compañía a un majo de las Vistillas o a algún cortesano de
los de la tertulia del Sr.~Mano de Mortero?

---No sé si viene solo o trae corte. Lo que sé es que su mamá ha
recibido mucho gusto con la inesperada aparición del niño, y que mi tía,
ya sea por mortificarme, ya porque realmente haya encontrado variación
en el joven, ha dicho ayer delante de toda la familia: «Si el señor
conde se porta bien y es hombre formal, obtendrá nuestros parabienes y
se hará acreedor a la más dulce recompensa que pueden ofrecerle dos
familias deseosas de formar una sola.»

---Señora condesa, yo a ser usted me reiría de don Diego y de las
mortificaciones de cuantas marquesas impertinentes peinan canas y
guardan pergaminos en el mundo.

---¡Ah, Gabriel; eso puede decirse; pero si tú comprendieras bien lo que
me pasa!---exclamó con pena.---¿Creerás que se han empeñado en que mi
hija no me tenga amor ni cariño alguno? Para conseguirlo han principiado
por apartarla perpetuamente de mí. Desde hace algunos días han resuelto
terminantemente que no venga a las tertulias de esta casa, y tampoco me
reciben a mí en la suya. De este modo, mi hija concluirá por no amarme.
La infeliz no tiene culpa de esto, ignora que soy su madre, me ve poco,
las oye a ellas con más frecuencia que a mí\ldots{} ¡Sabe Dios lo que le
dirán para que me aborrezca! Di si no es esto peor que cuantos castigos
pueden padecerse en el mundo; di si no tengo razón para estar muerta de
celos, sí, y los peores, los más dolorosos y desesperantes que pueden
desgarrar el corazón de una mujer. Al ver que personas egoístas quieren
arrebatarme lo que es mío, y privarme del único consuelo de mi vida, me
siento tan rabiosa, que sería capaz de acciones indignas de mi categoría
y de mi nombre.

---No me parece la situación de usted---le dije,---ni tan triste ni tan
desesperada como la ha pintado. Usted puede reclamar a su hija,
llevándosela para siempre consigo.

---Eso es difícil, muy difícil. ¿No ves que aparentemente y según la ley
carezco de derechos para reclamarla y traerla a mi lado? Me han jurado
una guerra a muerte. Han hecho los imposibles por desterrarme, no
vacilando hasta en denunciarme como afrancesada. Hace poco, como sabes,
proyectaron marcharse a Portugal sin darme noticia de ello, y si lo
impedí presentándome aquella noche en tu compañía, me fue preciso
amenazar con un gran escándalo para obligarlas a que se detuvieran. La
de Rumblar me cobró un aborrecimiento profundo, desde que supo mi
oposición a que Inés se desposase con el tunantuelo de su hijo. Mi tía
con su idea del decoro de la casa y de la honra de la familia me
mortifica más que la otra con su enojo, que tiene por móvil una
desmedida avaricia. Si me encontrara en Madrid, donde mis muchas
relaciones me ofrecen abundantes recursos para todo, tal vez vencería
estos y otros mayores obstáculos; pero nos hallamos en Cádiz, en una
plaza que casi está rigurosamente sitiada, donde tengo pocos amigos,
mientras que mi tía y la de Rumblar, por su exagerado españolismo
cuentan con el favor de todas las personas de poder. Suponte que me
obliguen a embarcarme, que me destierren, que durante mi forzada
ausencia engañen a la pobre muchacha y la casen contra su voluntad;
figúrate que esto suceda, y\ldots{}

---¡Oh!, señora---exclamé con vehemencia,---eso no sucederá mientras
usted y yo vivamos para impedirlo. Hablemos a Inés, revelémosle lo que
ya debiera saber\ldots{}

---Díselo tú, si te atreves\ldots{}

---¿Pues no me he de atrever?\ldots{}

---Debo advertirte otra cosa que ignoras, Gabriel; una cosa que tal vez
te cause tristeza; pero que debes saber\ldots{} ¿Tú crees conservar
sobre ella el ascendiente que tuviste hace algún tiempo y que
conservaste aun después de haber mudado tan bruscamente de fortuna?

---Señora---repuse,---no puedo concebir que haya perdido ese
ascendiente. Perdóneseme la vanidad.

---¡Desgraciado muchacho!---me dijo en tono de dulce compasión.---La
vida consiste en mil mudanzas dolorosas, y el que confía en la
perpetuidad de los sentimientos que le halagan, es como el iluso que
viendo las nubes en el horizonte, las cree montañas, hasta que un rayo
de luz las desfigura o un soplo de viento las desbarata. Hace dos años,
mi hija y tú erais dos niños desvalidos y abandonados. El apartamiento
en que vivíais y la común desgracia, aumentando la natural inclinación,
hicieron que os amarais. Después todo cambió. ¿Para qué repetir lo que
sabes tan bien? Inés en su nueva posición no quiso olvidar al fiel
compañero de su infortunio. ¡Hermoso sentimiento que nadie más que yo
supo apreciar en su valor! Aprovechándome de él, casi llegué hasta
tolerarle y autorizarle, impulsada por el despecho y por mortificar a mi
orgullosa parienta; pero yo sabía que aquella corazonada infantil
concluiría con el tiempo y la distancia, como en efecto ha concluido.

Oí con estupor las palabras de la condesa, que iban esparciendo densas
oscuridades delante de mis ojos. Pero la razón me indicaba que no debía
dar entero crédito a las palabras de mujer tan experta en ingeniosos
engaños, y esperé aparentando conformarme con su opinión y mi desaire.

---¿Te acuerdas de la noche en que nos presentamos aquí viniendo del
Puerto de Santa María? En esta misma sala nos recibió doña Flora.
Llamamos a Inés, te vio, le hablaste. La pobrecita estaba tan turbada
que no acertó a contestar derechamente a lo que le dijiste.
Indudablemente te conserva un noble y fraternal afecto; pero nada más.
¿No lo comprendiste? ¿No se ofreció a tus ojos o a tus oídos algún dato
para conocer que ya Inés no te ama?

---Señora---respondí con perplejidad,---aquel instante fue tan breve y
usted me suplicó con tanta precipitación que saliese de la casa, que
nada observé que me disgustara.

---Pues sí, puedes creerlo. Yo sé que Inés no te ama ya---afirmó con una
entereza tal que se me hizo aborrecible en un momento mi hermosa
interlocutora.

---¿Lo sabe usted?

---Yo lo sé.

---Tal vez se equivoque.

---No: Inés no te ama.

---¿Por qué?---pregunté bruscamente y con desabrimiento.

---Porque ama a otro---me respondió con calma.

---¡A otro!---exclamé tan asombrado que por largo rato no me di cuenta
de lo que sentía.---¡A otro! No puede ser, señora condesa. ¿Y quién es
ese otro? Sepámoslo.

Diciendo esto, en mi interior se retorcían dolorosamente unas como
culebras, que me estrujaban el corazón mordiéndolo y apretándolo con
estrechos nudos. Yo quería aparentar serenidad; pero mis palabras
balbucientes y cierta invencible sofocación de mi aliento descubrían la
flaqueza de mi espíritu caído desde la cumbre de su mayor orgullo.

---¿Quieres saberlo? Pues te lo diré. Es un inglés.

---¿Ese?---pregunté con sobresalto señalando hacia la sala donde
resonaba lejanamente el eco de las voces de doña Flora y de su
visitante.

---¡Ese mismo!

---¡Señora, no puede ser!, usted se equivoca---exclamé sin poder
contener la fogosa cólera que desarrollándose en mí como súbito
incendio, no admitía razón que la refrenara, ni urbanidad que la
reprimiera.---Usted se burla de mí; usted me humilla y me pisotea como
siempre lo ha hecho.

---Qué furioso te has puesto---me dijo sonriendo.---Cálmate y no seas
loco.

---Perdóneme usted si la he ofendido con mi brusca respuesta---dije
reponiéndome;---pero yo no puedo creer eso que he oído. Todo cuanto hay
en mí que hable y palpite con señales de vida, protesta contra tal idea.
Si ella misma me lo dice, lo creeré; de otro modo no. Soy un ciego
estúpido tal vez, señora mía, pero yo detesto la luz que pueda hacerme
ver la soledad espantosa que usted quiere ponerme delante. Pero no me ha
dicho usted quién es ese inglés ni en qué se funda para pensar\ldots{}

---Ese inglés vino aquí hace seis meses, acompañando a otro que se llama
lord Byron, el cual partió para Levante al poco tiempo. Este que aquí
está, se llama lord Gray. ¿Quieres saber más? ¿Quieres saber en qué me
fundo para pensar que Inés le ama? Hay mil indicios que ni engañan ni
pueden engañar a una mujer experimentada como yo. ¿Y eso te asombra?
Eres un mozo sin experiencia, y crees que el mundo se ha hecho para tu
regalo y satisfacción. Es todo lo contrario, niño. ¿En qué te fundabas
para esperar que Inés estuviera queriéndote toda la vida, luchando con
la ausencia, que en esta edad es lo mismo que el olvido? ¡Pues no pedías
poco en verdad! ¿Sabes que eres modestito? Que pasaran años y más años,
y ella siempre queriéndote\ldots{} Vamos, pide por esa boca. Es preciso
que te acostumbres a creer que hay además de ti, otros hombres en el
mundo, y que las muchachas tienen ojos para ver y oídos para escuchar.

Con estas palabras que encerraban profunda verdad, la condesa me estaba
matando. Parecíame que mi alma era una hermosa tela, y que ella con sus
finas tijeras me la estaba cortando en pedacitos para arrojarla al
viento.

---Pues sí. Ha pasado mucho tiempo---continuó.---Ese inglés se apareció
en Cádiz; nos visitó. Visita hoy con mucha frecuencia la otra casa, y en
ella es amado\ldots{} Esto te parece increíble, absurdo. Pues es la cosa
más sencilla del mundo. También creerás que el inglés es un hombre
antipático, desabrido, brusco, colorado, tieso y borracho como algunos
que viste y trataste en la plaza de San Juan de Dios cuando eras niño.
No: lord Gray es un hombre finísimo, de hermosa presencia y vasta
instrucción. Pertenece a una de las mejores familias de Inglaterra, y es
más rico que un perulero\ldots{} Ya\ldots{} ¡tú creíste que estas y
otras eminentes cualidades nadie las poseía más que el Sr.~D. Gabriel de
Tres-al-Cuarto! Lucido estás\ldots{} Pues oye otra cosa. Lord Gray
cautiva a las muchachas con su amena conversación. Figúrate, que con ser
tan joven, ha tenido ya tiempo para viajar por toda el Asia y parte de
América. Sus conocimientos son inmensos; las noticias que da de los
muchos y diversos pueblos que ha visto, curiosísimas. Es hombre además
de extraordinario valor; hase visto en mil peligros luchando con la
naturaleza y con los hombres, y cuando los relata con tanta elocuencia
como modestia, procurando rebajar su propio mérito y disimular su
arrojo, los que le oyen no pueden contener el llanto. Tiene un gran
libro lleno de dibujos, representando paisajes, ruinas, trajes, tipos,
edificios que ha pintado en esas lejanas tierras; y en varias hojas ha
escrito en verso y prosa mil hermosos pensamientos, observaciones y
descripciones llenas de grandiosa y elocuente poesía. ¿Comprendes que
pueda y sepa hacerse amar? Llega a la tertulia, las muchachas le rodean;
él les cuenta sus viajes con tanta verdad y animación, que vemos las
grandes montañas, los inmensos ríos, los enormes árboles de Asia, los
bosques llenos de peligros; vemos al intrépido europeo defendiéndose del
león que le asalta, del tigre que le acecha; nos describe luego las
tempestades del mar de la China, con aquellos vientos que arrastran como
pluma la embarcación, y le vemos salvándose de la muerte por un esfuerzo
de su naturaleza ágil y poderosa; nos describe los desiertos de Egipto,
con sus noches claras como el día, con las pirámides, los templos
derribados, el Nilo y los pobres árabes que arrastran miserable vida en
aquellas soledades; nos pinta luego los lugares santos de Jerusalén y
Belén, el sepulcro del Señor, hablándonos de los millares de peregrinos
que le visitan, de los buenos frailes que dan hospitalidad al europeo;
nos dice cómo son los olivares a cuya sombra oraba el Señor cuando fue
Judas con los soldados a prenderle, y nos refiere punto por punto cómo
es el monte Calvario y el sitio donde levantaron la santa Cruz. »Después
nos habla de la incomparable Venecia, ciudad fabricada dentro del mar,
de tal modo, que las calles son de agua y los coches unas lanchitas que
llaman góndolas; y allí se pasean de noche los amantes, solos en aquella
serena laguna, sin ruido y sin testigos. También ha visitado la América,
donde hay unos salvajes muy mansos que agasajan a los viajeros, y donde
los ríos, grandísimos como todo lo de aquel país, se precipitan desde lo
alto de una roca formando lo que llaman cataratas, es decir, un salto de
agua como si medio mar se arrojase sobre el otro medio, formando mundos
de espuma y un ruido que se oye a muchísimas leguas de distancia. Todo
lo relata, todo lo pinta con tan vivos colores, que parece que lo
estamos viendo. Cuenta sus acciones heroicas sin fanfarronería, y jamás
ha mortificado el orgullo de los hombres que le oyen con tanta atención,
si no con tanta complacencia como las mujeres. Ahora bien, Gabriel,
desgraciado joven, ¿por lo que digo comprendes que ese inglés tiene
atractivos suficientes para cautivar a una muchacha de tanta
sensibilidad como imaginación, que instintivamente vuelve los ojos hacia
todo lo que se distingue del vulgo enfatuado? Además, lord Gray es
riquísimo, y aunque las riquezas no bastan a suplir en los hombres la
falta de ciertas cualidades, cuando estas se poseen, las riquezas las
avaloran y realzan más. Lord Gray viste elegantemente; gasta con
profusión en su persona y en obsequiar dignamente a sus amigos, y su
esplendidez no es el derroche del joven calavera y voluntarioso, sino la
gala y generosidad del rico de alta cuna, que emplea sabiamente su
dinero en alegrar la existencia de cuantos le rodean. Es galante sin
afectación, y más bien serio que jovial. ¡Ay, pobrecito! ¿Lo comprendes
ahora? ¿Llegarás a entender que hay en el mundo alguien que puede
ponerse en parangón con el Sr.~D. Gabriel Tres-al-Cuarto? Reflexiona
bien, hijo; reflexiona bien quién eres tú. Un buen muchacho y nada más.
Excelente corazón, despejo natural, y aquí paz y después gloria. En
punto a posición oficialito del ejército\ldots{} bien ganado, eso
sí\ldots{} pero ¿qué vale eso? Figura\ldots{} no mala; conversación,
tolerable; nacimiento humildísimo, aunque bien pudieras figurarlo como
de los más alcurniados y coruscantes. Valor, no lo negaré; al contrario,
creo que lo tienes en alto grado, pero sin brillo ni lucimiento.
Literatura, escasa\ldots{} cortesía, buena\ldots{} Pero, hijo, a pesar
de tus méritos, que son muchos, dada tu pobreza y humildad, ¿insistirás
en hacerte indestronable, como se lo creyó el buen D. Carlos IV que
heredó la corona de su padre? No, Gabriel; ten calma y resígnate.

El efecto que me causó la relación de mi antigua ama fue terrible.
Figúrense ustedes cómo me habría quedado yo, si Amaranta hubiera cogido
el pico de Mulhacén, es decir, el monte más alto de España\ldots{} y me
lo hubiese echado encima. Pues lo mismo, señores, lo mismo me quedé.

\hypertarget{iii}{%
\chapter{III}\label{iii}}

¿Qué podía yo decir? Nada. ¿Qué debía hacer? Callarme y sufrir. Pero el
hombre aplastado por cualquiera de las diversas montañas que le caen
encima en el mundo, aun cuando conozca que hay justicia y lógica en su
situación, rara vez se conforma, y elevando las manecitas pugna por
quitarse de encima la colosal peña. No sé si fue un sentimiento de noble
dignidad, o por el contrario un vano y pueril orgullo, lo que me impulsó
a contestar con entereza, afectando no sólo conformidad sino
indiferencia ante el golpe recibido.

---Señora condesa---dije,---comprendo mi inferioridad. Hace tiempo que
pensaba en esto, y nada me asombra. Realmente, señora, era un
atrevimiento que un pobretón como yo, que jamás he estado en la India ni
he visto otras cataratas que las del Tajo en Aranjuez, tenga
pretensiones nada menos que de ser amado por una mujer de posición. Los
que no somos nobles ni ricos, ¿qué hemos de hacer más que ofrecer
nuestro corazón a las fregatrices y damas del estropajo, no siempre con
la seguridad de que se dignen aceptarlo? Por eso nos llenamos de
resignación, señora, y cuando recibimos golpes como el que usted se ha
servido darme, nos encogemos de hombros y decimos: «paciencia.» Luego
seguimos viviendo, y comemos y dormimos tan tranquilos\ldots{} Es una
tontería morirse por quien tan pronto nos olvida.

---Estás hecho un basilisco de rabia---me dijo la condesa en tono de
burla,---y quieres aparecer tranquilo. Si despides fuego\ldots{} toma mi
abanico y refréscate con él.

Antes que yo lo tomara, la condesa me dio aire con su abanico
precipitadamente. Sin ninguna gana me reía yo, y ella después de un rato
de silencio, me habló así:

---Me falta decirte otra cosa que tal vez te disguste; pero es forzoso
tener paciencia. Es que estoy contenta de que mi hija corresponda al
amor del inglés.

---Lo creo señora---respondí apretando con convulsa fuerza los dientes,
ni más ni menos que si entre ellos tuviera toda la Gran Bretaña.

---Sí---prosiguió,---todo suceso que me dé esperanzas de ver a mi hija
fuera de la tutela y dirección de la marquesa y la condesa, es para mí
lisonjero.

---Pero ese inglés será protestante.

---Sí---repuso,---mas no quiero pensar en eso. Puede que se haga
católico. De todos modos, ese es punto grave y delicado. Pero no reparo
en nada. Vea yo a mi hija libre, hállese en situación tal que yo pueda
verla, hablarla como y cuando se me antoje, y lo demás\ldots{} ¡Cómo
rabiaría doña María si llegara a comprender\ldots! Mucho sigilo,
Gabriel; cuento con tu discreción. Si lord Gray fuera católico, no creo
que mi tía se opusiera a que se casase Inés con él. ¡Ay!, luego nos
marcharíamos los tres a Inglaterra, lejos, lejos de aquí, a un país
donde yo no viera pariente de ninguna clase. ¡Qué felicidad tan grande!
¡Ay! Quisiera ser Papa para permitir que una mujer católica se casara
con un hombre hereje.

---Creo que usted verá satisfechos sus deseos.

---¡Oh!, desconfío mucho. El inglés aparte de su gran mérito es bastante
raro. A nadie ha confiado el secreto de sus amores, y sólo tenemos
noticias de él por indicios primero y después por pruebas irrecusables
obtenidas mediante largo y minucioso espionaje.

---Inés lo habrá revelado a usted.

---No, después de esto, ni una sola vez he conseguido verla. ¡Qué
desesperación! Las tres muchachas no salen de casa, sino custodiadas por
la autoridad de doña María. Aquí doña Flora y yo hemos trabajado lo que
no es decible para que lord Gray se franquease con nosotras, y nos lo
revelara; pero es tan prudente y callado, que guarda su secreto como un
avaro su tesoro. Lo sabemos por las criadas, por la murmuración de
algunas, muy pocas personas de las que van a la casa. No hay duda de que
es cierto, hijo mío. Ten resignación y no nos des un disgusto. Cuidado
con el suicidio.

---¿Yo?---dije afectando indiferencia.

---Toma, toma aire, que te incendias por todos lados---me dijo agitando
delante de mí su abanico.---Don Rodrigo en la horca no tiene más orgullo
que este general en agraz.

Cuando esto decía, sentí la voz de doña Flora y los pasos de un hombre.
Doña Flora dijo:

---Pase usted milord, que aquí está la condesa.

---Mírale\ldots{} verás---me dijo Amaranta con crueldad,---y juzgarás
por ti mismo si la niña ha tenido mal gusto.

Entró doña Flora seguida del inglés. Este tenía la más hermosa figura de
hombre que he visto en mi vida. Era de alta estatura, con el color
blanquísimo pero tostado que abunda en los marinos y viajeros del Norte.
El cabello rubio, desordenadamente peinado y suelto según el gusto de la
época, le caía en bucles sobre el cuello. Su edad no parecía exceder de
treinta o treinta y tres años. Era grave y triste pero sin la pesadez
acartonada y tardanza de modales que suelen ser comunes en la gente
inglesa. Su rostro estaba bronceado, mejor dicho, dorado por el sol,
desde la mitad de la frente hasta el cuello, conservando en la huella
del sombrero y en la garganta una blancura como la de la más pura y
delicada cera. Esmeradamente limpia de pelo la cara, su barba era como
la de una mujer, y sus facciones realzadas por la luz del Mediodía
dábanle el aspecto de una hermosa estatua de cincelado oro. Yo he visto
en alguna parte un busto del Dios Brahma, que muchos años después me
hizo recordar a lord Gray.

Vestía con elegancia y cierta negligencia no estudiada, traje azul de
paño muy fino, medio oculto por una prenda que llamaban sortú, y llevaba
sombrero redondo, de los primeros que empezaban a usarse. Brillaban
sobre su persona algunas joyas de valor, pues los hombres entonces se
ensortijaban más que ahora, y lucía además los sellos de dos relojes. Su
figura en general era simpática. Yo le miré y observé ávidamente,
buscándole imperfecciones por todos lados; pero ¡ay!, no le encontré
ninguna. Mas me disgustó oírle hablar con rara corrección el castellano,
cuando yo esperaba que se expresase en términos ridículos y con yerros
de los que desfiguran y afean el lenguaje; pero consolome la esperanza
de que soltase algunas tonterías. Sin embargo no dijo ninguna.

Entabló conversación con Amaranta, procurando esquivar el tema que
impertinentemente había tocado doña Flora al entrar.

---Querida amiga---dijo la vieja,---lord Gray nos va a contar algo de
sus amores en Cádiz, que es mejor tratado que el de los viajes por Asia
y África.

Amaranta me presentó gravemente a él, diciéndole que yo era un gran
militar, una especie de Julio César por la estrategia y un segundo Cid
por el valor; que había hecho mi carrera de un modo gloriosísimo, y que
había estado en el sitio de Zaragoza, asombrando con mis hechos heroicos
a españoles y franceses. El extranjero pareció oír con suma complacencia
mi elogio, y me dijo después de hacerme varias preguntas sobre la
guerra, que tendría grandísimo contento en ser mi amigo. Sus refinadas
cortesanías me tenían frita la sangre por la violencia y fingimiento con
que me veía precisado a responder a ellas. La maligna Amaranta reíase a
hurtadillas de mi embarazo, y más atizaba con sus artificiosas palabras
la inclinación y repentino afecto del inglés hacia mi persona.

---Hoy---dijo lord Gray,---hay en Cádiz gran cuestión entre españoles e
ingleses.

---No sabía nada---exclamó Amaranta.---¿En esto ha venido a parar la
alianza?

---No será nada, señora. Nosotros somos algo rudos, y los españoles un
poco vanagloriosos y excesivamente confiados en sus propias fuerzas,
casi siempre con razón.

---Los franceses están sobre Cádiz---dijo doña Flora,---y ahora salimos
con que no hay aquí bastante gente para defender la plaza.

---Así parece. Pero Wellesley---añadió el inglés,---ha pedido permiso a
la Junta para que desembarque la marinería de nuestros buques y defienda
algunos castillos.

---Que desembarquen; si vienen, que vengan---exclamó Amaranta.---¿No
crees lo mismo, Gabriel?

---Esa es la cuestión que no se puede resolver---dijo lord
Gray,---porque las autoridades españolas se oponen a que nuestra gente
les ayude. Toda persona que conozca la guerra ha de convenir conmigo en
que los ingleses deben desembarcar. Seguro estoy de que este señor
militar que me oye es de la misma opinión.

---Oh, no señor; precisamente soy de la opinión contraria---repuse con
la mayor viveza, anhelando que la disconformidad de pareceres alejase de
mí la intolerable y odiosísima amistad que quería manifestarme el
inglés.---Creo que las autoridades españolas hacen bien en no consentir
que desembarquen los ingleses. En Cádiz hay guarnición suficiente para
defender la plaza.

---¿Lo cree usted?---me preguntó.

---Lo creo---respondí procurando quitar a mis palabras la dureza y
sequedad que quería infundirles el corazón.---Nosotros agradecemos el
auxilio que nos están dando nuestros aliados, más por odio al común
enemigo que por amor a nosotros; esa es la verdad. Juntos pelean ambos
ejércitos; pero si en las acciones campales es necesaria esta alianza,
porque carecemos de tropas regulares que oponer a las de Napoleón, en la
defensa de plazas fuertes harto se ha probado que no necesitamos ayuda.
Además, las plazas fuertes que como esta son al mismo tiempo magníficas
plazas comerciales, no deben entregarse nunca a un aliado por leal que
sea; y como los paisanos de usted son tan comerciantes, quizás gustarían
demasiado de esta ciudad, que no es más que un buque anclado a vista de
tierra. Gibraltar casi nos está oyendo y lo puede decir.

Al decir esto, observaba atentamente al inglés, suponiéndole próximo a
dar rienda suelta al furor, provocado por mi irreverente censura; pero
con gran sorpresa mía, lejos de ver encendida en sus ojos la ira, noté
en su sonrisa no sólo benevolencia, sino conformidad con mis opiniones.

---Caballero---dijo tomándome la mano,---¿me permitirá usted que le
importune repitiéndole que deseo mucho su amistad?

Yo estaba absorto, señores.

---Pero milord---preguntó doña Flora;---¿en qué consiste que aborrece
usted tanto a sus paisanos?

---Señora---dijo lord Gray,---desgraciadamente he nacido con un carácter
que si en algunos puntos concuerda con el de la generalidad de mis
compatriotas, en otros es tan diferente como lo es un griego de un
noruego. Aborrezco el comercio, aborrezco a Londres, mostrador
nauseabundo de las drogas de todo el mundo; y cuando oigo decir que
todas las altas instituciones de la vieja Inglaterra, el régimen
colonial y nuestra gran marina tienen por objeto el sostenimiento del
comercio y la protección de la sórdida avaricia de los negociantes que
bañan sus cabezas redondas como quesos con el agua negra del Támesis,
siento un crispamiento de nervios insoportable y me avergüenzo de ser
inglés. El carácter inglés es egoísta, seco, duro como el bronce,
formado en el ejército del cálculo y refractario a la poesía. La
imaginación es en aquellas cabezas una cavidad lóbrega y fría donde
jamás entra un rayo de luz ni resuena un eco melodioso. No comprenden
nada que no sea una cuenta, y al que les hable de otra cosa que del
precio del cáñamo, le llaman mala cabeza, holgazán y enemigo de la
prosperidad de su país. Se precian mucho de su libertad, pero no les
importa que haya millones de esclavos en las colonias. Quieren que el
pabellón inglés ondee en todos los mares, cuidándose mucho de que sea
respetado; pero siempre que hablan de la dignidad nacional, debe
entenderse que la quincalla inglesa es la mejor del mundo. Cuando sale
una expedición diciendo que va a vengar un agravio inferido al orgulloso
leopardo, es que se quiere castigar a un pueblo asiático o africano que
no compra bastante trapo de algodón.

---¡Jesús, María y José!---exclamó horrorizada doña Flora.---No puedo
oír a un hombre de tanto talento como milord hablando así de sus
compatriotas.

---Siempre he dicho lo mismo, señora---prosiguió lord Gray,---y no ceso
de repetirlo a mis paisanos. Y no digo nada cuando quieren echársela de
guerreros y dan al viento el estandarte con el gato montés que ellos
llaman leopardo. Aquí en España me ha llenado de asombro el ver que mis
paisanos han ganado batallas. Cuando los comerciantes y mercachifles de
Londres sepan por las Gacetas que los ingleses han dado batallas y las
han ganado, bufarán de orgullo creyéndose dueños de la tierra como lo
son del mar, y empezarán a tomar la medida del planeta para hacerle un
gorro de algodón que lo cubra todo. Así son mis paisanos, señoras. Desde
que este caballero evocó el recuerdo de Gibraltar, traidoramente ocupado
para convertirle en almacén de contrabando, vinieron a mi mente estas
ideas, y concluyo modificando mi primera opinión respecto al desembarco
de los ingleses en Cádiz. Señor oficial, opino como usted: que se queden
en los barcos.

---Celebro que al fin concuerden sus ideas con las mías, milord---dije
creyendo haber encontrado la mejor coyuntura para chocar con aquel
hombre que me era, sin poderlo remediar, tan aborrecible.---Es cierto
que los ingleses son comerciantes, egoístas, interesados, prosaicos;
pero ¿es natural que esto lo diga exagerándolo hasta lo sumo un hombre
que ha nacido de mujer inglesa y en tierra inglesa? He oído hablar de
hombres que en momentos de extravío o despecho han hecho traición a su
patria; pero esos mismos que por interés la vendieron, jamás la
denigraron en presencia de personas extrañas. De buenos hijos es ocultar
los defectos de sus padres.

---No es lo mismo---dijo el inglés.---Yo conceptúo más compatriota mío a
cualquier español, italiano, griego o francés que muestre aficiones
iguales a las mías, sepa interpretar mis sentimientos y corresponder a
ellos, que a un inglés áspero, seco y con un alma sorda a todo rumor que
no sea el son del oro contra la plata, y de la plata contra el cobre.
¿Qué me importa que ese hombre hable mi lengua, si por más que charlemos
él y yo no podemos comprendernos? ¿Qué me importa que hayamos nacido en
un mismo suelo, quizás en una misma calle, si entre los dos hay
distancias más enormes que las que separan un polo de otro?

---La patria, señor inglés, es la madre común, que lo mismo cría y
agasaja al hijo deforme y feo que al hermoso y robusto. Olvidarla es de
ingratos; pero menospreciarla en público indica sentimientos quizás
peores que la ingratitud.

---Esos sentimientos, peores que la ingratitud, los tengo yo, según
usted---dijo el inglés.

---Antes que pregonar delante de extranjeros los defectos de mis
compatriotas, me arrancaría la lengua---afirmé con energía, esperando
por momentos la explosión de la cólera de lord Gray.

Pero este, tan sereno cual si se oyese nombrar en los términos más
lisonjeros, me dirigió con gravedad las siguientes palabras:

---Caballero, el carácter de usted y la viveza y espontaneidad de sus
contradicciones y réplicas, me seducen de tal manera, que me siento
inclinado hacia usted, no ya por la simpatía, sino por un afecto
profundo.

Amaranta y doña Flora no estaban menos asombradas que yo.

---No acostumbro tolerar que nadie se burle de mí, milord---dije,
creyendo efectivamente que era objeto de burlas.

---Caballero---repuso fríamente el inglés,---no tardaré en probar a
usted que una extraordinaria conformidad entre su carácter y el mío ha
engendrado en mí vivísimo deseo de entablar con usted sincera amistad.
Óigame usted un momento. Uno de los principales martirios de mi vida, el
mayor quizás, es la vana aquiescencia con que se doblegan ante mí todas
las personas que trato. No sé si consistirá en mi posición o en mis
grandes riquezas; pero es lo cierto que en donde quiera que me presento,
no hallo sino personas que me enfadan con sus degradantes cumplidos.
Apenas me permito expresar una opinión cualquiera, todos los que me oyen
aseguran ser de igual modo de pensar. Precisamente mi carácter ama la
controversia y las disputas. Cuando vine a España, hícelo con la ilusión
de encontrar aquí gran número de gente pendenciera, ruda y primitiva,
hombres de corazón borrascoso y apasionado, no embadurnados con el vano
charol de la cortesanía. Mi sorpresa fue grande al encontrarme atendido
y agasajado, cual lo pudiera estar en Londres, sin hallar obstáculos a
la satisfacción de mi voluntad, en medio de una vida monótona, regular,
acompasada, no expuesto a sensaciones terribles, ni a choques violentos
con hombres ni con cosas, mimado, obsequiado, adulado\ldots{} ¡Oh, amigo
mío! Nada aborrezco tanto como la adulación. El que me adula es mi
irreconciliable enemigo. Yo gozo extraordinariamente al ver frente a mí
los caracteres altivos, que no se doblegan sonriendo cobardemente ante
una palabra mía; gusto de ver bullir la sangre impetuosa del que no
quiere ser domado ni aun por el pensamiento de otro hombre; me cautivan
los que hacen alarde de una independencia intransigente y enérgica, por
lo cual asisto con júbilo a la guerra de España. Pienso ahora internarme
en el país, y unirme a los guerrilleros. Esos generales que no saben
leer ni escribir, y que eran ayer arrieros, taberneros y mozos de
labranza, exaltan mi admiración hasta lo sumo. He estado en academias
militares y aborrezco a los pedantes que han prostituido y afeminado el
arte salvaje de la guerra, reduciéndolo a reglas necias, y decorándose a
sí mismos con plumas y colorines para disimular su nulidad. ¿Ha militado
usted a las órdenes de algún guerrillero? ¿Conoce usted al Empecinado, a
Mina, a Tabuenca, a Porlier? ¿Cómo son? ¿Cómo visten? Se me figura ver
en ellos a los héroes de Atenas y del Lacio. Amigo mío, si no recuerdo
mal, la señora condesa dijo hace un momento que usted debía sus rápidos
adelantamientos en la carrera de las armas a su propio mérito, pues sin
el favor de nadie ha adquirido un honroso puesto en la milicia. ¡Oh,
caballero!, usted me interesa vivamente, usted será mi amigo, quiéralo o
no. Adoro a los hombres que no han recibido nada de la suerte ni de la
cuna, y que luchan contra este oleaje. Seremos muy amigos. ¿Está usted
de guarnición en la Isla? Pues venga a vivir a mi casa siempre que pase
a Cádiz. ¿En dónde reside usted para ir a visitarle todos los
días\ldots?

Sin atreverme a rechazar tan vehementes pruebas de benevolencia, me
excusé como pude.

---Hoy, caballero---añadió,---es preciso que venga usted a comer
conmigo. No admito excusas. Señora condesa, usted me presentó a este
caballero. Si me desaíra, cuente usted como que ha recibido la ofensa.

---Creo---dijo la condesa,---que ambos se congratularán bien pronto de
haber entablado amistad.

---Milord, estoy a la orden de usted---dije levantándome cuando él se
disponía a partir.

Y después de despedirnos de las dos damas, salí con el inglés. Parecía
que me llevaba el demonio.

\hypertarget{iv}{%
\chapter{IV}\label{iv}}

Lord Gray vivía cerca de las Barquillas de Lope. Su casa, demasiado
grande para un hombre solo, estaba en gran parte vacía. Servíanle varios
criados, españoles todos a excepción del ayuda de cámara que era inglés.

Dábase trato de príncipe en la comida, y durante toda ella no tenían un
momento de sosiego los vasos, llenos con la mejor sangre de las cepas de
Montilla, Jerez y Sanlúcar.

Durante la comida no hablamos más que de la guerra, y después, cuando
los generosos vinos de Andalucía hicieron su efecto en la insigne cabeza
del mister, se empeñó en darme algunas lecciones de esgrima. Era gran
tirador según observé a los primeros golpes; y como yo no poseía en tal
alto grado los secretos del arte y él no tenía entonces en su cerebro
todo aquel buen asiento y equilibrio que indican una organización
educada en la sobriedad, jugaba con gran pesadez de brazo, haciéndome
más daño del que correspondía a un simple entretenimiento.

---Suplico a milord que no se entusiasme demasiado---dije conteniendo
sus bríos.---Me ha desarmado ya repetidas veces para gozarse como un
niño en darme estocadas a fondo que no puedo parar. ¡Ese botón está mal
y puedo ser atravesado fácilmente!

---Así es como se aprende---repuso.---O no he de poder nada, o será
usted un consumado tirador.

Después que nos batimos a satisfacción, y cuando se despejaron un tanto
las densas nubes que oscurecían y turbaban su entendimiento, me marché a
la Isla, a donde me acompañó deseoso, según dijo, de visitar nuestro
campamento. En los días sucesivos casi ninguno dejó de visitarme. Su
afectuosidad me contrariaba, y cuanto más le aborrecía, más desarmaba él
mi cólera a fuerza de atenciones. Mis respuestas bruscas, mi mal humor,
y la terquedad con que le rebatía, lejos de enemistarle conmigo,
apretaban más los lazos de aquella simpatía que desde el primer día me
manifestó; y al fin no puedo negar que me sentía inclinado hacia hombre
tan raro, verificándose el fenómeno de considerar en él como dos
personas distintas y un solo lord Gray verdadero, dos personas, sí, una
aborrecida y otra amada; pero de tal manera confundidas, que me era
imposible deslindar dónde empezaba el amigo y dónde acababa el rival.

Érale sumamente agradable estar en mi compañía y en la de los demás
oficiales mis camaradas. Durante las operaciones nos seguía armado de
fusil, sable y pistolas, y en los ratos de vagar iba con nosotros a los
ventorrillos de Cortadura o Matagorda, donde nos obsequiaba de un modo
espléndido con todo lo que podían dar de sí aquellos establecimientos.
Más de una vez se hizo acompañar al venir desde Cádiz por dos o tres
calesas cargadas con las más ricas provisiones que por entonces traían
los buques ingleses y los costeros del Condado y Algeciras; y en cierta
ocasión en que no podíamos salir de las trincheras del puente Suazo,
transportó allá con rapidez parecida a la de los tiempos que después han
venido, al Sr.~Poenco con toda su tienda y bártulos y séquito mujeril y
guitarril, para improvisar una fiesta.

A los quince días de estos rumbos y generosidades no había en la Isla
quien no conociese a lord Gray; y como entonces estábamos en buenas
relaciones con la Gran Bretaña, y se cantaba aquello de

\small
\newlength\mlena
\settowidth\mlena{\qquad La trompeta de la Gloria }
\begin{center}
\parbox{\mlena}{La trompeta de la Gloria                  \\
                dice al mundo \textit{Velintón...}}       \\
\end{center}
\normalsize

(lo mismo que está escrito) nuestro misterera popularísimo en toda la
extensión que inunda con sus canales el caño de Sancti-Petri.

Su mayor confianza era conmigo; pero debo indicar aquí una
circunstancia, que a todos llamará la atención, y es que aunque
repetidas veces procuré sondear su ánimo en el asunto que más me
interesaba, jamás pude conseguirlo. Hablábamos de amores, nombraba yo la
casa y la familia de Inés, y él, volviéndose taciturno, mudaba la
conversación. Sin embargo, yo sabía que visitaba todas las noches a doña
María; pero su reserva en este punto era una reserva sepulcral. Sólo una
vez dejó traslucir algo y voy a decir cómo.

Durante muchos días estuve sin poder ir a Cádiz, a causa de las
ocupaciones del servicio, y esta esclavitud me daba tanto fastidio como
pesadumbre. Recibía algunas esquelas de la condesa suplicándome que
pasase a verla, y yo me desesperaba no pudiendo acudir. Al fin logré una
licencia a principios de Marzo y corrí a Cádiz. Lord Gray y yo
atravesamos la Cortadura precisamente el día del furioso temporal que
por muchos años dejó memoria en los gaditanos de aquel tiempo. Las olas
de fuera, agitadas por el Levante, saltaban por encima del estrecho
istmo para abrazarse con las olas de la bahía. Los bancos de arena eran
arrastrados y deshechos, desfigurando la angosta playa; el horroroso
viento se llevaba todo en sus alas veloces, y su ruido nos permitía
formar idea de las mil trompetas del Juicio, tocadas por los ángeles de
la justicia. Veinte buques mercantes y algunos navíos de guerra
españoles e ingleses estrelláronse aquel día contra la costa de
Poniente; y en el placer de Rota, la Puntilla y las rocas donde se
cimenta el castillo de Santa Catalina aparecieron luego muchos cadáveres
y los despojos de los cascos rotos y de las jarcias y árboles deshechos.

Lord Gray, contemplando por el camino tan gran desolación, el furor del
viento, los horrores del revuelto cielo, ora negro, ora iluminado por la
siniestra amarillez de los relámpagos, la agitación de las olas verdosas
y turbias, en cuyas cúspides, relucientes como filos de cuchillos, se
alcanzaban a ver restos de alguna nave que se hundía luego en los
cóncavos senos para reaparecer después; contemplando lord Gray, repito,
aquel desorden, no menos admirable que la armonía de lo creado, aspiraba
con delicia el aire húmedo de la tempestad y me decía:

---¡Cuán grato es a mi alma este espectáculo! Mi vida se centuplica ante
esta fiesta sublime de la Naturaleza, y se regocija de haber salido de
la nada, tomando la execrable forma que hoy tiene. Para esto te han
criado ¡oh mar! Escupe las naves comerciantes que te profanan, y prohíbe
la entrada en tus dominios al sórdido mercachifle, ávido de oro,
saqueador de los pueblos inocentes que no se han corrompido todavía y
adoran a Dios en el ara de los bosques. Este ruido de invisibles
montañas que ruedan por los espacios, chocándose y redondeándose como
los guijos que arrastra un río; estas lenguazas de fuego que lamen el
cielo y llegan a tocar el mar con sus afiladas puntas; este cielo que se
revuelca desesperado; este mar que anhela ser cielo, abandonando su
lecho eterno para volar; este hálito que nos arrastra, esta confusión
armoniosa, esta música, amigo, y ritmo sublime que lo llena todo,
encontrando eco en nuestra alma, me extasían, me cautivan, y con fuerza
irresistible me arrastran a confundirme con lo que veo\ldots{} Esta
alteración se repite en mi alma; esta rabia y desesperado anhelo de
salir de su centro, propiedad es también de mi alma; este rumor, donde
caben todos los rumores de cielo y tierra, ha tiempo que también
ensordece mi alma; este delirio es mi delirio, y este afán con que
vuelan nubes y olas hacia un punto a que no llegan nunca, es mi propio
afán.

Yo pensé que estaba loco, y cuando le vi bajar del calesín, acercarse a
la playa e internarse por ella hasta que el agua le cubrió las botas,
corrí tras él lleno de zozobra, temiendo que en su enajenación se
arrojase, como había dicho, en medio de las olas.

---Milord---le dije,---volvámonos al coche, pues no hay para qué
convertirse ahora en ola ni nube, como usted desea, y sigamos hacia
Cádiz, que para agua bastante tenemos con la que llueve, y para viento,
harto nos azota por el camino.

Pero él no me hacía caso, y empezó a gritar en su lengua. El calesero,
que era muy pillo, hizo gestos significativos para indicar que lord Gray
había abusado del Montilla; pero a mí me constaba que no lo había
probado aquel día.

---Quiero nadar---dijo lacónicamente lord Gray, haciendo ademán de
desnudarse.

Y al punto forcejeamos con él el calesero y yo, pues aunque sabíamos que
era gran nadador, en aquel sitio y hora no habría vivido diez minutos
dentro del agua. Al fin le convencimos de su locura, haciéndole volver a
la calesa.

---Contenta se pondría, milord, la señora de sus pensamientos si le
viera a usted con inclinaciones a matarse desde que suena un trueno.

Lord Gray rompió a reír jovialmente, y cambiando de aspecto y tono,
dijo:

---Calesero, apresura el paso, que deseo llegar pronto a Cádiz.

---El lamparín no quiere andar.

---¿Qué lamparín?

---El caballo. Le han salido callos en la \emph{jerraúra. ¡Ay sé!} Este
caballo es muy respetoso.

---¿Por qué?

---Muy respetoso con los amigos. Cuando se ve con Pelaítas, se hacen
cortesías y se preguntan cómo ha ido de viaje.

---¿Quién es Pelaítas?

---El violín del Sr.~Poenco. \emph{¡Ay sé!} Si usted le dice a mi
caballo: «vas a descansar en casa de Poenco, mientras tu amo come una
aceituna y bebe un par de copas,» correrá tanto, que tendremos que darle
palos para que pare, no sea que con la fuerza del golpe abra un boquete
en la muralla de Puerta Tierra.

Gray prometió al calesero refrescarle en casa de Poenco, y al oír esto
¡parecía mentira!, el lamparín avivó el paso.

---Pronto llegaremos---dijo el inglés.---No sé por qué el hombre no ha
inventado algo para correr tanto como el viento.

---En Cádiz le aguarda a usted una muchacha bonita. No una, muchas tal
vez.

---Una sola. Las demás no valen nada, señor de Araceli\ldots{} Su alma
es grande como el mar. Nadie lo sabe más que yo, porque en apariencia es
una florecita humilde que vive casi a escondidas dentro del jardín. Yo
la descubrí y encontré en ella lo que hombre alguno no supo encontrar.
Para mí solo, pues, relampaguean los rayos de sus ojos y braman las
tempestades de su pecho\ldots{} Está rodeada de misterios encantadores,
y las imposibilidades que la cercan y guardan como cárceles inaccesibles
más estimulan mi amor\ldots{} Separados nos oscurecemos; pero juntos
llenamos todo lo creado con las deslumbradoras claridades de nuestro
pensamiento.

Si mi conciencia no dominara casi siempre en mí los arrebatos de la
pasión, habría cogido a lord Gray y le habría arrojado al mar\ldots{}
Hícele luego mil preguntas, di vueltas y giros sobre el mismo tema para
provocar su locuacidad; nombré a innumerables personas, pero no me fue
posible sacarle una palabra más. Después de dejarme entrever un rayo de
su felicidad, calló y su boca cerrose como una tumba.

---¿Es usted feliz?---le dije al fin.

---En este momento sí---respondió.

Sentí de nuevo impulsos de arrojarle al mar.

---Lord Gray---exclamé súbitamente---¿vamos a nadar?

---¡Oh! ¿Qué es eso? ¿Usted también?

---¡Sí, arrojémonos al agua! Me pasa a mí algo de lo que a usted pasaba
antes. Se me ha antojado nadar.

---Está loco---contestó riendo y abrazándome.---No, no permito yo que
tan buen amigo perezca por una temeridad. La vida es hermosa, y quien
pensase lo contrario, es un imbécil. Ya llegamos a Cádiz. Tío Hígados,
eche aceite a la lamparilla, que ya estamos cerca de la taberna de
Poenco.

~

Al anochecer llegamos a Cádiz. Lord Gray me llevó a su casa, donde nos
mudamos de ropa, y cenamos después. Debíamos ir a la tertulia de doña
Flora, y mientras llegaba la hora, mi amigo, que quise que no, hubo de
darme nuevas lecciones de esgrima. Con estos juegos iba, sin pensarlo,
adiestrándome en un arte en el cual poco antes carecía de habilidad
consumada, y aquella tarde tuve la suerte de probar la sabiduría de mi
maestro dándole una estocada a fondo con tan buen empuje y limpieza, que
a no tener botón el estoque, hubiéralo atravesado de parte a parte.

---¡Oh, amigo Araceli!---exclamó lord Gray con asombro.---Usted adelanta
mucho. Tendremos aquí un espadachín temible. Luego, tira usted con mucha
rabia\ldots{}

En efecto; yo tiraba con rabia, con verdadero afán de acribillarle.

\hypertarget{v}{%
\chapter{V}\label{v}}

Por la noche fuimos a casa de doña Flora; pero lord Gray, a poco de
llegar, despidiose diciendo que volvería. La sala estaba bien iluminada,
pero aún no muy llena de gente, por ser temprano. En un gabinete
inmediato aguardaban las mesas de juego el dinero de los apasionados
tertuliantes, y más adentro tres o cuatro desaforadas bandejas llenas de
dulces nos prometían agradable refrigerio para cuando todo acabase.
Había pocas damas, por ser costumbre en los saraos de doña Flora que
descollasen los hombres, no acompañados por lo general más que de una
media docena de beldades venerables del siglo anterior, que, cual
castillos gloriosos, pero ya inútiles, no pretendían ser conquistables
ni conquistadas. Amaranta representaba sola la juventud unida a la
hermosura.

Saludaba yo a la condesa, cuando se me acercó doña Flora, y
pellizcándome bonitamente con todo disimulo el brazo por punto cercano
al codo, me dijo:

---Se está usted portando, caballerito. Casi un mes sin parecer por
aquí. Ya sé que se divirtió usted en el puente de Suazo con las buenas
piezas que llevó allí el Sr.~Poenco hace ocho días\ldots{} ¡Bonita
conducta! Yo empeñada en apartarle a usted del camino de la perdición, y
usted cada vez más inclinado a seguir por él\ldots{} Ya se sabe que la
juventud ha de tener sus trapicheos; pero los muchachos decentes y bien
nacidos desfogan sus pasiones con compostura, antes buscando el trato
honesto de personas graves y juiciosas que el de la gentezuela maja y
tabernaria.

La condesa afectó estar conforme con la reprimenda y la repitió, dándola
más fuerza con sus irónicos donaires. Después, ablandándose doña Flora y
llevándome adentro, me dio a probar de unos dulces finísimos que no se
repartían sino entre los amigos de confianza. Cuando volvimos a la sala,
Amaranta me dijo:

---Desde que doña María y la marquesa decidieron que no viniera Inés,
parece que falta algo en esta tertulia.

---Aquí no hacen falta niñas, y menos la condesa de Rumblar, que con sus
remilgos impedía toda diversión. Nadie se había de acercar a la niña, ni
hablar con la niña, ni bailar con la niña, ni dar un dulce a la niña.
Dejémonos de niñas: hombres, hombres quiero en mi tertulia; literatos
que lean versos, currutacos que sepan de corrido las modas de París,
diaristas que nos cuenten todo lo escrito en tres meses por las Gacetas
de Amberes, Londres, Augsburgo y Rotterdam; generales que nos hablen de
las batallas que se van a ganar; gente alegre que hable mal de la
regencia y critique la cosa pública, ensayando discursos para cuando se
abran esas saladísimas Cortes que van a venir.

---Yo no creo que haya tales Cortes---dijo Amaranta,---porque las Cortes
no son más que una cosa de figurón, que hace el rey para cumplir un
antiguo uso. Como ahora estamos sin rey\ldots{}

---¿Pues no ha de haber? Nada; vengan esas Cortes. Cortes nos han
prometido, y Cortes nos han de dar. Pues poco bonito será este
espectáculo. Como que es un conjunto de predicadores, y no baja de ocho
a diez sermones los que se oyen por día, todos sobre la cosa pública,
amiga mía, y criticando, criticando, que es lo que a mí me gusta.

---Habrá Cortes---dije yo,---porque en la Isla están pintando y
arreglando el teatro para salón de sesiones.

---¿Pero es en un teatro? Yo pensé que en una iglesia---dijo doña Flora.

---El estamento de próceres y clérigos se reunirá en una
iglesia---indicó Amaranta,---y el de procuradores en un teatro.

---No, no hay más que un estamento, señoras. Al principio se pensó en
tres; pero ahora se ha visto que uno solo es más sencillo.

---Será el de la nobleza.

---No, hija, serán todos clérigos. Esto parece lo más propio.

---No hay más estamento que el de procuradores, en que entrarán todas
las clases de la sociedad.

---¿Y dices que están pintando el teatro?

---Sí, señora. Le han puesto unas cenefas amarillas y encarnadas que
hacen una vista así como de escenario de titiriteros en feria\ldots{} En
fin, monísimo.

---Para esta festividad quiere sin duda el Sr.~D. Pedro los cincuenta
uniformes amarillos y encarnados que le estamos haciendo, todos
galoneados de plata y cortados en forma que llaman de española antigua.

---Me temo mucho---dijo Amaranta riendo,---que D. Pedro y otros tan
extravagantes y locos como él, pongan en ridículo a Cortes y
procuradores, pues hay personas que convierten en mojiganga todo aquello
en que ponen la mano.

---Ya principia a venir gente. Aquí está Quintana. También vienen Beña y
D. Pablo de Xérica.

Quintana saludó a mis dos amigas. Yo le había visto y oído hablar en
Madrid en las tertulias de las librerías, pero sin tener hasta entonces
el placer de tratar a poeta tan insigne. Su fama entonces era grande, y
entre los patriotas exaltados gozaba de mucha popularidad, conquistada
por sus artículos políticos y proclamas patrióticas. Era de fisonomía
dura y basta, moreno, con vivos ojos y gruesos labios, signo claro esto,
así como su frente lobulosa, de la viril energía de su espíritu. Reía
poco, y en sus ademanes y tono, lo mismo que en sus escritos, dominaba
la severidad. Tal vez esta severidad, más que propia, fuera atribuida y
supuesta por los que conocían sus obras, pues en aquella época ya habían
salido a luz las principales odas, las tragedias y algunas de las
\emph{Vidas}; Píndaro, Tirteo y Plutarco a la vez, estaba orgulloso de
su papel, y este orgullo se le conocía en el trato.

Quintana era entusiasta de la causa española y liberal ardiente con
vislumbres de filósofo francés o ginebrino. Más beneficios recibió de su
valiente pluma la causa liberal que de la espada de otros, y si la
defensa de ciertas ideas, que él enaltecía con todas las galas del
estilo y todos los recursos de un talento superior y valiente cual
ninguno; si la defensa de ciertas ideas, repito, no hubiera corrido
después por cuenta de otras manos y de gárrulas plumas, diferente sería
hoy la suerte de España.

Más simpático en el trato que Quintana, por carecer de aquella
grandílocua y solemne severidad, era D. Francisco Martínez de la Rosa,
recién llegado entonces de Londres, y que no era célebre todavía más que
por su comedia \emph{Lo que puede un empleo}, obra muy elogiada en
aquellos inocentes tiempos. Las gracias, la finura, la encantadora
cortesía, la amabilidad, el talento social sin afectación, amaneramiento
ni empalago, nadie lo tenía entonces, ni lo tuvo después, como Martínez
de la Rosa. Pero hablo aquí de una persona a quien todos han conocido, y
a quien vida tan larga no imprimió gran mudanza en genio y figura. Lo
mismo que le vieron ustedes hacia 1857, salvo el detrimento de los años,
era Martínez de la Rosa cuando joven. Si en sus ideas había alguna
diferencia, no así en su carácter, que fue en la forma festivamente
afable hasta la vejez, y en el fondo grave, entero y formal desde la
juventud.

No sé por qué me he ocupado aquí de este eminente hombre, pues la verdad
es que no concurrió aquella noche a la tertulia de doña Flora, que estoy
con mucho gusto describiendo.

Fueron, sí, como he dicho, Xérica y Beña, poetas menores de que me
acuerdo poco, sin duda porque su fama problemática y la mediocridad de
su mérito hicieron que no fijase mucho en ellos la atención. De quien me
acuerdo es de Arriaza, y no porque me fuera muy simpático, pues la
índole adamada y aduladora de sus versos serios y la mordacidad de sus
sátiras me hacían poca gracia, sino porque siempre le vi en todas
partes, en tertulias, cafés, librerías y reuniones de diversas clases.
Este llegó más tarde a la tertulia.

Después de los que he mencionado, vimos aparecer a un hombre como de
unos cincuenta años, flaco, alto, desgarbado y tieso. Tenía como D.
Quijote los bigotes negros, largos y caídos, los brazos y piernas como
palitroques, el cuerpo enjutísimo, el color moreno, el pelo entrecano,
aguileña la nariz, los ojos ya dulces, ya fieros, según a quien miraba,
y los ademanes un tanto embarazados y torpes. Pero lo más singular de
aquel singularísimo hombre era su vestido, a la manera de los de
Carnaval, consistente en pantalones a la turquesca, atacados a la
rodilla, jubón amarillo y capa corta encarnada o herreruelo, calzas
negras, sombrero de plumas como el de los alguaciles de la plaza de
toros y en el cinto un tremendo chafarote, que iba golpeando en el
suelo, y hacía con el ruido de las pisadas un compás triple, cual si el
personaje anduviese con tres pies.

Parecerá a algunos que es invención mía esto del figurón que pongo a los
ojos de mis lectores; pero abran la historia, y hallarán más al vivo que
yo lo hago pintadas las hazañas de un personaje, a quien llamo D. Pedro,
para no ridiculizar como él lo hizo, un título ilustre, que después han
llevado personas muy cuerdas. Sí; vestido estaba como he pintado, y no
fue él solo quien dio por aquel tiempo en la manía de vestir y calzar a
la antigua; que otro marqués, jerezano por cierto, y el célebre Jiménez
Guazo y un escocés llamado lord Downie, hicieron lo mismo; pero yo por
no aburrir a mis lectores presentándoles uno tras otro a estos tipos tan
característicos como extraños, he hecho con las personas lo que hacen
los partidos, es decir, una fusión, y me he permitido recoger las
extravagancias de los tres y engalanar con tales atributos a uno solo de
ellos, al más gracioso sin disputa, al más célebre de todos.

Al punto que entró D. Pedro, oyéronse estrepitosas risas en la sala;
pero doña Flora salió al punto a la defensa de su amigo, diciendo:

---No hay que criticarle, pues hace muy bien en vestirse a la antigua; y
si todos los españoles, como él dice, hicieran lo mismo, con la
costumbre de vestir a la antigua vendría el pensar a la antigua, y con
el pensar el obrar, que es lo que hace falta.

D. Pedro hizo profundas reverencias y se sentó junto a las damas, antes
satisfecho que corrido por el recibimiento que le hicieron.

---No me importan burlas de gente afrancesada---dijo mirando de soslayo
a los que le contemplábamos,---ni de filosofillos irreligiosos, ni de
ateos, ni de francmasones, ni de \emph{democratistas}, enemigos
encubiertos de la religión y del rey. Cada uno viste como quiere, y si
yo prefiero este traje a los franceses que venimos usando hace tiempo, y
ciño esta espada que fue la que llevó Francisco Pizarro al Perú, es
porque quiero ser español por los cuatro costados y ataviar mi persona
según la usanza española en todo el mundo, antes de que vinieran los
franchutes con sus corbatas, chupetines, pelucas, polvos, casacas de
cola de abadejo y demás porquerías que quitan al hombre su natural
fiereza. Ya pueden los que me escuchan reírse cuanto quieran del traje,
si bien no lo harán de la persona porque saben que no lo tolero.

---Está muy bien---dijo Amaranta.---Está muy bien ese traje, y sólo las
personas de mal gusto pueden criticarlo. Señores, ¿cómo quieren ustedes
ser buenos españoles sin vestir a la antigua?

---Pero señor marqués (D. Pedro era marqués, aunque me callo su
título)---dijo Quintana con benevolencia---¿por qué un hombre formal y
honrado como usted, se ha de vestir de esta manera, para divertir a los
chicos de la calle? ¿Ha de tener el patriotismo por funda un jubón, y no
ha de poder guarecerse en una chupa?

---Las modas francesas han corrompido las costumbres---repuso D. Pedro
atusándose los bigotes,---y con las modas, es decir, con las pelucas y
los colores, han venido la falsedad del trato, la deshonestidad, la
irreligión, el descaro de la juventud, la falta de respeto a los
mayores, el mucho jurar y votar, el descoco e impudor, el atrevimiento,
el robo, la mentira, y con estos males los no menos graves de la
filosofía, el ateísmo, el democratismo, y eso de la soberanía de la
nación que ahora han sacado para colmo de la fiesta.

---Pues bien---repuso Quintana,---si todos esos males han venido con las
pelucas y los polvos, ¿usted cree que los va a echar de aquí vistiéndose
de amarillo? Los males se quedarán en casa, y el señor marqués hará reír
a las gentes.

---Sr.~D. Manolo, si todos fueran como usted que se empeña en combatir a
los franceses, imitándolos en usos y costumbres, lucidos estábamos.

---Si las costumbres se han modificado, ellas sabrán por qué lo han
hecho. Se lucha y se puede luchar contra un ejército por grande que sea;
pero contra las costumbres hijas del tiempo, no es posible alzar las
manos, y me dejo cortar las dos que tengo, si hay cuatro personas que le
imiten a usted.

---¿Cuatro?---exclamó con orgullo D. Pedro.---Cuatrocientas están ya
filiadas en la \emph{Cruzada del obispado de Cádiz}, y aunque todavía no
hay uniformes para todos, ya cuento con cincuenta o sesenta, gracias al
celo de respetables damas, alguna de las cuales me oye. Y no nos
vestimos así, señores míos, para andar charlando en los cafés y metiendo
bulla por las calles, ni imprimiendo papeles que aumenten la
desvergüenza e irrespetuosidad del pueblo hacia lo más sagrado, ni para
convocar Cortes ni cortijos, ni para echar sermones a lo dómine Lucas,
sino para salir por esos campos hendiendo cabezas de filósofos y
acuchillando enemigos de la Iglesia y del rey. Ríanse del traje en buena
hora, que en cuanto sean despachados los mosquitos que zumban más allá
del caño de Sancti-Petri, volveremos acá y haremos que los redactores
del \emph{Semanario Patriótico} se vistan de papel impreso, que es la
moda francesa que más les cuadra.

Dicho esto, D. Pedro celebró mucho con risas su propio chiste, y luego
tomó Beña la palabra para sostener la conveniencia de vestir a la
antigua. ¿Verdad que era graciosa la manía? Para que no se dude de mi
veracidad, quiero trasladar aquí un párrafo del Conciso que conservo en
la memoria:

«Otro de los medios indirectos,---decía---pero muy poderoso, para
renovar el entusiasmo, sería volver a usar el antiguo traje español. No
es decible lo que esto podría influir en la felicidad de la nación. ¡Oh,
padres de la patria, diputados del augusto congreso! A vosotros dirijo
mi humilde voz: vosotros podéis renovar los días de nuestra antigua
prosperidad; vestíos con el traje de nuestros padres, y la nación entera
seguirá vuestro ejemplo.»

Esto lo escribía poco después aquel mismo Sr.~Beña, poeta de
circunstancias, a quien yo vi en casa de doña Flora. ¡Y recomendaba a
los padres de la patria que imitasen en su atavío al gran D. Pedro,
pasmo de los chicos y alboroto de paseantes! ¡Qué bonitos habrían estado
Argüelles, Muñoz Torrero, García Herreros, Ruiz Padrón, Inguanzo, Mejía,
Gallego, Quintana, Toreno y demás insignes varones, vestidos de
arlequines!

Y aquel Beña era liberal y pasaba por cuerdo; verdad es que los
liberales como los absolutistas, han tenido aquí desde el principio de
su aparición en el mundo ocurrencias graciosísimas.

Quintana preguntó a D. Pedro si la \emph{Cruzada del obispado de Cádiz}
pensaba presentarse a las futuras Cortes en aquel talante el día de la
apertura.

---Yo no quiero nada con Cortes---repuso.---¿Pero usted es de los bolos
que creen habrá tal novedad? La regencia está decidida a echar la tropa
a la calle para hacer polvo a los vocingleros que ahora no pueden
pasarse sin Cortes. ¡Angelitos! Déseles la novedad de este juguete para
que se diviertan.

---La regencia---repuso el poeta---hará lo que la manden. Callará y
aguantará. Aunque carezco de la perspicacia que distingue al señor D.
Pedro, me parece que la nación es algo más que el señor obispo de
Orense.

---Verdaderamente, Sr.~D. Manuel---dijo Amaranta,---eso de la soberanía
de la nación que han inventado ahora\ldots{} anoche estaban explicándolo
en casa de la Morlá, y por cierto que nadie lo entendía; eso de la
soberanía de la nación si se llega a establecer va a traernos aquí otra
revolución como la francesa, con su guillotina y sus atrocidades. ¿No lo
cree usted?

---No, señora; no creo ni puedo creer tal cosa.

---Que pongan lo que quieran con tal que sea nuevo---dijo doña
Flora;---¿no es verdad, Sr.~de Xérica?

---Justo, y afuera religión, afuera rey, afuera todo---vociferó D.
Pedro.

---Denme trescientos años de soberanía, de la nación---dijo
Quintana,---y veremos si se cometen tantos excesos, arbitrariedades y
desafueros como en trescientos años que no la ha habido. ¿Habrá
revolución que contenga tantas iniquidades e injusticias como el solo
período de la privanza de D. Manuel Godoy?

---Nada, nada, señores---dijo D. Pedro con ironía.---Si ahora vamos a
estar muy bien; si vamos a ver aquí el siglo de oro; si no va a haber
injusticias, ni crímenes, ni borracheras, ni miserias, ni cosa mala
alguna, pues para que nada nos falte, en vez de padres de la Iglesia;
tenemos periodistas; en vez de santos, filósofos; en vez de teólogos,
ateos.

---Justamente; el Sr.~de Congosto tiene razón---replicó Quintana.---La
maldad no ha existido en el mundo hasta que no la hemos traído nosotros
con nuestros endiablados libros\ldots{} Pero todo se va a remediar con
vestirnos de mojiganga.

---Pero en último resultado---preguntó la condesa---¿hay Cortes o no?

---Sí, señora, las habrá.

---Los españoles no sirven para eso.

---Eso no lo hemos probado.

---¡Ay, qué ilusión tiene usted, Sr.~D. Manuel! Verá usted qué escenas
tan graciosas habrá en las sesiones\ldots{} y digo graciosas por no
decir terribles y escandalosas.

---El terror y el escándalo no nos son desconocidos, señora, ni los
traerán por primera vez las Cortes a esta tierra de la paz y de la
religiosidad. La conspiración del Escorial, los tumultos de Aranjuez,
las vergonzosas escenas de Bayona, la abdicación de los reyes padres,
las torpezas de Godoy, las repugnantes inmoralidades de la última Corte,
los tratados con Bonaparte, los convenios indignos que han permitido la
invasión, todo esto, señora amiga mía, que es el colmo del horror y del
escándalo, ¿lo han traído por ventura las Cortes?

---Pero el rey gobierna, y las Cortes, según el uso antiguo, votan y
callan.

---Nosotros hemos caído en la cuenta de que el rey existe para la nación
y no la nación para el rey.

---Eso es---dijo D. Pedro,---el rey para la nación, y la nación para los
filósofos.

---Si las Cortes no salen adelante---añadió Quintana,---lo deberán a la
perfidia y mala fe de sus enemigos; pues estas majaderías de vestir a la
antigua y convertir en sainete las más respetables cosas, es vicio muy
común en los españoles de uno y otro partido. Ya hay quien dice que los
diputados deben vestirse como los alguaciles en día de pregón de Bula, y
no falta quien sostiene que todo cuanto se hable, proponga y discuta en
la Asamblea, debe decirse en verso.

---Pues de ese modo sería precioso---afirmó doña Flora.

---En efecto---dijo Amaranta,---y como se reúnen en un teatro la ilusión
sería perfecta. Prometo asistir a la inauguración.

---Yo no faltaré. Sr.~de Quintana, usted me proporcionará un palco o un
par de lunetas. ¿Y se paga, se paga?

---No, amiga mía---dijo Amaranta burlándose.---La nación enseña y pone
al público gratis sus locuras.

---Usted---le dijo Quintana sonriendo,---será de nuestro partido.

---¡Ay, no, amigo mío!---repuso la dama.---Prefiero afiliarme a la
\emph{Cruzada del obispado}. Me espantan los revolucionarios, desde que
he leído lo que pasó en Francia. ¡Ay, Sr.~Quintana! ¡Qué lástima que
usted se haya hecho estadista y político! ¿Por qué no hace usted versos?

---No están los tiempos para versos. Sin embargo, ya usted ve cómo los
hacen mis amigos; Arriaza, Beña, Xérica, Sánchez Barbero no dejan
descansar a las prensas de Cádiz.

Beña y Xérica se habían apartado del grupo.

---¡Ay, amigo mío!, que no oiga yo aquello de

\small
\newlength\mlenb
\settowidth\mlenb{\quad ¡Oh! Velintón, nombre amable }
\begin{center}
\parbox{\mlenb}{\quad ¡Oh! \textit{Velintón}, nombre amable,        \\
                grande alumno del Dios Marte.}                      \\
\end{center}
\normalsize

---Es horrible la poesía de estos tiempos, porque los cisnes callan,
entristecidos por el luto de la patria, y de su silencio se aprovechan
los grajos para chillar. ¿Y dónde me deja usted aquello de

\small
\newlength\mlenc
\settowidth\mlenc{veloces marchemos...?}
\begin{center}
\parbox{\mlenc}{\quad Resuene el tambor;                            \\
                veloces marchemos...?}                              \\
\end{center}
\normalsize

---Arriaza---indicó Quintana,---ha hecho últimamente una sátira
preciosa. Esta noche la leerá aquí.

---Nombren al ruin\ldots---dijo Amaranta, viendo aparecer en el salón al
poeta de los chistes.

---Arriaza, Arriaza---exclamaron diferentes voces salidas de distintos
lados de la estancia.---A ver, léanos usted la oda \emph{A Pepillo}.

---Atención, señores.

---Es de lo más gracioso que se ha escrito en lengua castellana.

---Si el gran Botella la leyera, de puro avergonzado se volvería a
Francia.

Arriaza, hombre de cierta fatuidad, se gallardeaba con la ovación hecha
a los productos de su numen. Como su fuerte eran los versos de
circunstancias y su popularidad por esta clase de trabajos
extraordinaria, no se hizo de rogar, y sacando un largo papel, y
poniéndose en medio de la sala, leyó con muchísima gracia aquellos
versos célebres que ustedes conocerán y cuyo principio es de este modo:

«Al ínclito Sr.~Pepe, Rey (en deseo) de las Españas y (en visión) de sus
Indias.

\small
\newlength\mlend
\settowidth\mlend{\quad Salud, gran rey de la rebelde gente,}
\begin{center}
\parbox{\mlend}{\quad Salud, gran rey de la rebelde gente,          \\
                salud, salud, Pepillo, diligente                    \\
                protector del cultivo de las uvas                   \\
                y catador experto de las cubas.»                    \\
                \null \dotfill \null  }                             \\
\end{center}
\normalsize

A cada instante era el poeta interrumpido por los aplausos, las
felicitaciones, las alabanzas, y vierais allí cómo por arte mágico
habíanse confundido todas las opiniones en el unánime sentimiento de
desprecio y burla hacia nuestro rey pegadizo. Por instantes hasta el
gran D. Pedro y D. Manuel José Quintana parecieron conformes. La
composición de Pepillo corrió manuscrita por todo Cádiz. Después la
refundió su autor, y fue publicada en 1812.

Dividiose después la tertulia. Los políticos se agruparon a un lado, y
el atractivo de las mesas de juego llevó a la sala contigua a una buena
porción de los concurrentes. Amaranta y la condesa permanecieron allí, y
D. Pedro, como hombre galante no las dejaba de la mano.

\hypertarget{vi}{%
\chapter{VI}\label{vi}}

---Gabriel---me dijo Amaranta,---es preciso que te decidas a trocar tu
uniforme a la francesa por este español que lleva nuestro amigo. Además,
la orden de la \emph{Cruzada} tiene la ventaja de que cada cual se
encaja encima el grado que más le cuadra, como por ejemplo D. Pedro, que
se ha puesto la faja de capitán general.

En efecto, D. Pedro no se había andado con chiquitas para subirse por
sus propios pasos al último escalón de la milicia.

---Es el caso---dijo sin modestia el héroe,---que necesita uno
condecorarse a sí propio, puesto que nadie se toma el trabajo de
hacerlo. En cuanto a la entrada de este caballerito en la orden, venga
en buen hora; pero sepa que los nuestros hacen vida ascética durmiendo
en una tarima y teniendo por almohada una buena piedra. De este modo se
fortalece el hombre para las fatigas de la guerra.

---Me parece muy bien---afirmó Amaranta,---y si a esto añaden una comida
sobria, como por ejemplo, dos raciones de obleas al día, serán los
mejores soldados de la tierra. Ánimo, pues, Gabriel, y hazte caballero
del obispado de Cádiz.

---De buena gana lo haría, señores, si me encontrara con fuerzas para
cumplir las leyes de un instituto tan riguroso. Para esa \emph{Cruzada}
del obispado se necesitan hombres virtuosísimos y llenos de fe.

---Ha hablado perfectamente---repuso con solemne acento D. Pedro.

---Disculpas, hijo---añadió Amaranta con malicia.---La verdadera causa
de la resistencia de este mozuelo a ingresar en la orden gloriosa es no
sólo la holgazanería, sino también que las distracciones de un amor tan
violento como bien correspondido, le tienen embebecido y trastornado. No
se permiten enamorados en la orden, ¿verdad, Sr.~D. Pedro?

---Según y conforme---respondió el grave personaje tomándose la barba
con dos dedos y mirando al techo.---Según y conforme. Si los catecúmenos
están dominados por un amor respetuoso y circunspecto hacia persona de
peso y formalidad, lejos de ser rechazados, con más gusto son admitidos.

---Pues el amor de este no tiene nada de respetuoso---dijo Amaranta,
mirando con picaresca atención a doña Flora.---Mi amiga, que me está
oyendo, es testigo de la impetuosidad y desconsideración de este
violento joven.

D. Pedro fijó sus ojos en doña Flora.

---Por Dios, querida condesa---dijo esta,---usted con sus imprudencias
es la que ha echado a perder a este muchacho, enseñándole cosas que aún
no está en edad de saber. Por mi parte la conciencia no me acusa palabra
ni acción que haya dado motivo a que un joven apasionado se
extralimitase alguna vez. La juventud, Sr. D. Pedro, tiene arrebatos;
pero son disculpables, porque la juventud\ldots{}

---En una palabra, amiga mía---dijo Amaranta dirigiéndose a doña
Flora.---Ante una persona tan de confianza como el Sr.~D. Pedro, puede
usted dejar a un lado el disimulo, confesando que las ternuras y
patéticas declaraciones de este joven no le causan desagrado.

---Jesús, amiga mía---exclamó mudando de color la dueña de la
casa,---¿qué está usted diciendo?

---La verdad. ¿A qué andar con tapujos? ¿No es verdad, señor de
Congosto, que hago bien en poner las cosas en su verdadero lugar? Si
nuestra amiga siente una amorosa inclinación hacia alguien, ¿por qué
ocultarlo? ¿Es acaso algún pecado? ¿Es acaso un crimen que dos personas
se amen? Yo tengo derecho a permitirme estas libertades por la amistad
que les tengo a los dos, y porque ha tiempo que les vengo aconsejando se
decidan a dejar a un lado los misterios, secreticos y trampantojos que a
nada conducen, sí señor, y que por lo general suelen redundar en desdoro
de la persona. En cuanto a mi amiga, harto la he exhortado, condenando
su insistente celibato, y se me figura que al fin mis prédicas no serán
inútiles. No lo niegue usted. Su voluntad está vacilante, y en aquello
de si caigo o no caigo; de modo que si una persona tan respetable como
el Sr.~D. Pedro uniera sus amonestaciones a las mías\ldots{}

D. Pedro estaba verde, amarillo, jaspeado. Yo, sin decir nada, procuraba
al mismo tiempo que contenía la risa, corroborar con mis actitudes y
miradas lo que la condesa decía. Doña Flora, confundida entre la
turbación y la ira, miraba a Amaranta y al esperpento, y como viera a
este con el color mudado y los ojos chispeantes de enojo, turbose más y
dijo:

---Qué bromas tiene la condesa, Sr.~D. Pedro ¿quiere usted tomar un
dulcecito?

---Señora---repuso con iracunda voz el estafermo,---los hombres como yo
se endulzan con acíbar la lengua, y el corazón con desengaños.

Doña Flora quiso reír, pero no pudo.

---Con desengaños, sí señora---añadió D. Pedro,---y con agravios
recibidos de quien menos debían esperarse. Cada uno es dueño de dirigir
sus impulsos amorosos al punto que más le conviene. Yo en edad temprana
los dirigí a una ingrata persona, que al fin\ldots{} mas no quiero afear
su conducta, ni pregonar su deslealtad, y guardareme para mí solo las
penas como me guardé las alegrías. Y no se diga para disculpar esta
ingratitud, que yo falté una sola vez en veinticinco años al respeto, a
la circunspección, a la severidad que la cultura y dignidad de entrambos
me imponía, pues ni palabra incitativa pronunciaron mis labios, ni gesto
indecoroso hicieron mis manos, ni idea impúdica turbó la pureza de mi
pensamiento, ni nombré la palabra matrimonio, a la cual se asocian
imágenes contrarias al pudor, ni miré de mal modo, ni fijé los ojos en
las partes que la moda francesa tenía mal cubiertas, ni hice nada, en
fin, que pudiera ofender, rebajar o menoscabar el santo objeto de mi
culto. Pero ¡ay!, en estos tiempos corrompidos no hay flor que no se
aje, ni pureza que no se manche, ni resplandor que no se oscurezca con
alguna nubecilla. Está dicho todo, y con esto, señoras, pido a ustedes
licencia para retirarme.

Levantábase para partir, cuando doña Flora le detuvo diciendo:

---¿Qué es eso, Sr.~D. Pedro? ¿Qué arrebato le ha dado? ¿Hace usted caso
de las bromas de Amaranta? Es una calumnia, sí señor, una calumnia.

---¿Pero qué es esto?---dijo Amaranta fingiendo la mayor
estupefacción.---¿Mis palabras han podido causar el disgusto del Sr.~D.
Pedro? Jesús, ahora caigo en que he cometido una gran imprudencia. Dios
mío, ¡qué daño he causado! Sr.~D. Pedro, yo no sabía nada, yo
ignoraba\ldots{} Desunir por una palabra indiscreta dos
voluntades\ldots{} Este mozalbete tiene la culpa. Ahora recuerdo que mi
amiga le está recomendando siempre que le imite a usted en las formas
respetuosas para manifestar su amor.

---Y le reprendo sus atrevimientos---dijo doña Flora\ldots{}

---Y le tira de las orejas cuando se extralimita de palabra u obra, y le
pellizca en el brazo cuando salen juntos a paseo.

---Señoras, perdónenme ustedes---dijo don Pedro,---pero me retiro.

---¿Tan pronto?

---Amaranta con sus majaderías le ha amoscado a usted.

---Tengo que ir a casa de la señora condesa de Rumblar.

---Eso es un desaire, Sr.~D. Pedro. Dejar mi casa por la de otra.

---La condesa es una persona respetabilísima que tiene alta idea del
decoro.

---Pero no hace vestidos para los \emph{Cruzados}.

---La de Rumblar tiene el buen gusto de no admitir en su casa a los
politiquillos y diaristas que infestan a Cádiz.

---Ya.

---Allí no se juega tampoco. Allí no van Quintana el fatuo, ni Martínez
de la Rosa el pedante, ni Gallego el clerizonte ateo, ni Gallardo el
demonio filosófico, ni Arriaza el relamido, ni Capmany el loco, ni
Argüelles el jacobino, sino multitud de personas deferentes con la
religión y con el rey.

Y dicho esto, el estafermo hizo una reverencia que medio le descoyuntó,
marchándose después con paso reposado y ademán orgulloso.

---Amiga mía---dijo doña Flora,---¡qué imprudente es usted! ¿No es
verdad, Gabriel, que ha sido muy imprudente?

---¡Ya lo creo; contarlo todo en sus propias barbas!

---Yo temblaba por ti, niñito, temiendo que te ensartara con el
chafarote.

---La condesa nos ha comprometido---afirmé con afectado enojo.

---Es un diablillo.

---Amiga mía---dijo Amaranta,---lo hice con la mayor inocencia. Después
de lo que he descubierto, me pongo de parte del desairado don Pedro. La
verdad, señora doña Flora; es una gran picardía lo que ha hecho usted.
Trocarle, después de veinticinco años, por este mozuelo sin
respetabilidad\ldots{}

---Calle usted, calle usted, picaruela---repuso la dueña.---Por mi parte
ni a uno ni a otro. Si usted no hubiera incitado a este joven con sus
provocaciones\ldots{}

---De aquí en adelante---dije yo,---seré respetuoso, comedido y
circunspecto, como don Pedro.

Doña Flora me ofreció un dulce, pero viose obligada a poner punto en la
cuestión, porque otras damas, que como ella pertenecían a la clase de
plazas desmanteladas y con artillería antigua, intervinieron
inoportunamente en nuestro diálogo.

He referido la anterior burlesca escena, que parece insignificante y
sólo digna de momentánea atención, porque con ser pura broma, influyó
mucho en acontecimientos que luego contaré, proporcionándome sinsabores
y contrariedades. De este modo los más frívolos sucesos, que no parecen
tener fuerza bastante para alterar con su débil paso la serenidad de la
vida, la conmueven hondamente de súbito y cuando menos se espera.

\hypertarget{vii}{%
\chapter{VII}\label{vii}}

Poco después entró en la sala el memorable D. Diego, conde de Rumblar y
de Peña Horadada, y con gran sorpresa mía, ni saludó a la condesa, ni
esta tuvo a bien dirigirle mirada alguna. Reconociéndome al punto,
llegose a mí, y con la mayor afabilidad me saludó y felicitó por mi
rápido adelantamiento en la carrera de las armas, de que ya tenía
noticias. No nos habíamos visto desde mi aventura famosa en el palacio
del Pardo. Yo le encontré bastante desfigurado, sin duda por recientes
enfermedades y molestias.

---Aquí serás mi amigo, lo mismo que en Madrid---me dijo entrando juntos
en la sala de juego.---Si estás en la Isla, te visitaré. Quiero que
vengas a las tertulias de mi casa. Dime, cuando vienes a Cádiz, ¿paras
aquí en casa de la condesa?

---Suelo venir aquí.

---¿Sabes que mi parienta aprecia la lealtad de los que fueron sus
pajes?\ldots{} Ya sabrás que de esta me caso.

---La condesa me lo ha dicho.

---La condesa ya no priva. Hay divorcio absoluto entre ella y los demás
de la familia\ldots{} ¡oh!, ahora me acuerdo de cuando te encontramos en
el Pardo\ldots{} Cuando le preguntaron a Amaranta que qué hacías allí,
no supo contestar. Lo que hacías, tú lo podrás decir\ldots{} ¿Juegas, o
no?

---Jugaremos.

---Aquí al menos se respira, chico. Vengo huyendo de las tertulias de mi
casa, que más que tertulias son un cónclave de clérigos, frailucos y
enemigos de la libertad. Allí no se va más que a hablar mal de los
periodistas y de los que quieren Constitución. No se juega, Gabriel, ni
se baila, ni se refresca, ni se hablan más que sosadas y
boberías\ldots{} De todos modos, es preciso que vengas a mi casa. Mis
hermanas me han dicho que quieren conocerte; sí, me lo han dicho. Las
pobres están muy aburridas. Si no fuese porque lord Gray distrae un poco
a las tres muchachas\ldots{} Vendrás a casa. Pero cuidado con echártela
de liberal y de jacobino. No abras la boca sino para decir mil pestes de
las futuras Cortes, de la libertad de la imprenta, de la revolución
francesa, y ten cuidado de hacer una reverencia cuando se nombre al rey,
y de decir algo en latín al modo de conjuro siempre que citen a
Bonaparte, a Robespierre o a otro monstruo cualquiera. Si así no lo
haces, mi mamá te echará al punto a la calle, y mis hermanas no podrán
rogarte que vuelvas.

---Muy bien; tendré cuidado de cumplir el programa. ¿En dónde nos
veremos?

---Yo iré a la Isla o nos veremos aquí, aunque la verdad\ldots{} Tal vez
no vuelva. Mi mamá me tiene prohibido poner los pies en esta casa. Vete
a la mía, y pregunta por tu amigo don Diego, el que ganó la batalla de
Bailén. Yo le he hecho creer a mi mamá que entre tú y yo ganamos aquella
célebre batalla.

---¿Y Santorcaz?

---En Madrid sigue de comisario de policía. Nadie le puede ver; pero él
se ríe de todos y cumple con su obligación. Con que juguemos. Yo voy al
caballo.

El juego, antes frío y mal sostenido por personas sin entusiasmo, se
animó con la presencia de Amaranta, que fue a poner su dinero en la
balanza de la suerte. Para que todo marchase a pedir de boca, llegó en
aquel crítico punto lord Gray, de quien dije había desaparecido al
comienzo de la tertulia. Como de costumbre, el espléndido inglés reclamó
para sí las preeminencias de banquero, y tallando él con serenidad,
apuntando nosotros con zozobra y emoción, le desvalijamos a toda prisa.
Sobre todo Amaranta y yo tuvimos una suerte loca. Doña Flora, por el
contrario, veía mermados con rapidez sus exiguos capitales y D. Diego se
mantuvo en tabla con vaivenes de desgracia y fortuna.

Indiferente a su ruina el inglés, más sacaba cuanto más perdía, y todo
lo que de sus bolsillos se trasegó al montón, venía después del montón a
visitar los míos, que se asombraban de una abundancia jamás por ellos
conocida. La función no concluyó sino cuando lord Gray no dio más de sí,
acabándose la tertulia. Los políticos, sin embargo, continuaban
disputando en la sala vecina, aun después de retirada la última moneda
de la mesa de juego.

Cuando salimos para continuar el monte en casa de lord Gray, D. Diego me
dijo:

---Mi mamá cree a estas horas que duermo como un talego. En casa nos
retiramos a las diez. Mi mamá, después de cenar, nos echa la bendición,
rezamos varias oraciones y nos manda a la cama. Yo me retiro a la
alcoba, fingiendo tener mucho sueño, apago la luz y cuando todo está en
silencio, escápome bonitamente a la calle. Muy de madrugada vuelvo, abro
mis puertas con llaves a propósito, y me meto en el lecho. Sólo mis
hermanitas están en el secreto y favorecen la evasión.

Lord Gray nos obsequió en su casa con una espléndida cena; sacamos luego
el libro de las cuarenta hojas y con sus textos pasamos febrilmente
entretenidos la noche. D. Diego en tabla, el inglés perdiendo las
entrañas, y yo ganando hasta que cansados los tres y siempre invariable
y terca la fortuna, dimos por terminada la partida. ¡Oh!, en los
gloriosos años de 1810, 1811 y 1812 se jugaba mucho, pero mucho.

Desde aquella noche no pude volver a Cádiz hasta la tarde del 28 de
Mayo, formando parte de las fuerzas que se enviaron para hacer los
honores a la Regencia, que al día siguiente debía instalarse en el
palacio de la Aduana. Esta ceremonia de la instalación fue muy divertida
y animada tanto el día 29 como el 30, por ser en este los de nuestro
señor rey D. Fernando VII. Cuando estábamos en la Aduana, haciendo
guardia de honor a la Regencia, reunida dentro en sesión solemne, oímos
decir que en aquel mismo día se presentarían en Cádiz al pie de cien
coraceros a la antigua que querían ofrecer sus respetos al poder
central. Al punto que tal oí, acordeme del insigne D. Pedro, y no dudé
que él fuese autor de la diversión que se nos preparaba.

Las doce serían, cuando una gran turba de chicos desembocando por las
calles de Pedro Conde y de la Manzana, anunció que algo muy
extraordinario y divertido se aproximaba; y con efecto, tras el infantil
escuadrón, que de mil diversos modos y con variedad de chillidos
manifestaba su regocijo, vierais allí aparecer una falange de cien a
caballo vestidos todos con el mismo traje amarillo y rojo que yo había
visto en las secas carnes del gran D. Pedro. Este venía delante con faja
de capitán general sobre el arlequinado traje, y tan estirado,
satisfecho y orgulloso, que no se cambiara por Godofredo de Bouillón
entrando triunfante en Jerusalén. Ni él ni los demás llevaban corazas,
pero sí cruces en el pecho; y en cuanto a armas, cuál llevaba sable,
cuál espadín de etiqueta. Como diversión de Carnestolendas, aquello
podía tolerarse; pero como \emph{Cruzada del obispado de Cádiz} para
acabar con los franceses, era de lo más grotesco que en los anales de la
historia se puede en ningún tiempo encontrar.

La multitud les victoreaba, por la sencilla razón de que se divertía;
ellos, con los aplausos, se creían no menos dignos de admiración que las
huestes de César o Aníbal; y por fortuna nuestra, desde el Puerto de
Santa María, donde estaban los franceses, no podía verse ni con
telescopio semejante fiesta, que si la vieran, de buena gana habrían
hecho más ruido las risas que los cañones.

Llegaron a la Aduana, pidió permiso el que los mandaba para entrar a
saludar a la Regencia, se lo negamos, creyendo que los de la Junta no
habrían perdido el juicio; insistió D. Pedro, golpeando el suelo con el
sable y profiriendo amenazas y bravatas; entramos a notificar a los
señores qué clase de estantiguas querían colarse en el palacio del
gobierno, y este al fin consintió en ser felicitado por los caballeros a
la antigua, temiendo despopularizarse si no lo hacía. ¡Debilidad propia
de autoridades españolas!

Entró, pues, Congosto, seguido de cinco de los suyos, escogidos entre
los más granados, atravesó el salón de corte, y al encarar con los de la
Regencia hizo una profunda cortesía, irguiose después, paseó su
orgullosa vista de un confín a otro de la sala, metió la mano en el
bolsillo de los gregüescos y con gran sorpresa de todos los que le
veíamos, sacó unos anteojos de gruesa armadura, que se caló sobre la
martilluda nariz. Tal facha y vestido con anteojos era de lo más
ridículo que puede imaginarse. Los de la Regencia fluctuaban entre el
enojo y la risa, y los extraños que presenciaban aquello, no disimulaban
su contento por disfrutar de escena tan chusca.

Luego que se ensartó los espejuelos y los acomodó bien, enganchados en
las orejas y apoyados en la nariz, metió la otra mano en el otro
bolsillo y saco un papel, ¡pero qué papel! Lo menos tenía una vara.
Todos creímos que sería un discurso; pero no, señores, eran unos versos.
Entonces, para hablar al Rey o al público o a las autoridades, privaban
los malos versos sobre la mala prosa. Desdobló, pues, el luengo papel,
tosió limpiando el gaznate, se atusó los largos bigotes, y con voz
cavernosa y retumbante dio principio a la lectura de una sarta de
endecasílabos cojos, mancos y lisiados, tan rematadamente malos como
obra que eran del mismo personaje que los leía. Siento no poder dar a
mis amigos una muestra de aquella literatura, porque ni se imprimieron
ni puedo recordarlos; pero si no la forma, tengo presente el sentido,
que se reducía a encomiar la necesidad de que todo el mundo se vistiera
a la antigua, único modo de resucitar el ya muerto y enterrado heroísmo
de los antiguos tiempos.

Durante la lectura había sacado D. Pedro la espada, y todas las frases
fuertes las acompañaba de tajos, mandobles y cuchilladas en el aire,
volteando el arma por encima de su cabeza, lo cual remató el grotesco
papel que estaba haciendo. Luego que acabara de leer los malhadados
versos, guardó el cartapacio, descolgó de la nariz los anteojos, y
envainando la espada, hizo otra profunda reverencia y salió del salón
seguido de los suyos.

¡Señores, que es verdad lo que digo! Me ofenden esas muestras de
incredulidad de los que me escuchan. Ábrase la historia, no las que
andan en manos de todos, sino otras algo íntimas, y que testigos
presenciales dictaron. Pues qué, ¿se ha olvidado ya la condición
sainetesca y un tanto arlequinada de nuestros partidos políticos en el
período de su incubación? Verdad purísima, santa verdad es lo que he
referido, aunque parece inverosímil, y aún me callo otras cositas por no
ofender el decoro nacional.

Después, la graciosa procesión recorrió las calles de Cádiz con grande
alegría de todo el pueblo, que se regocijaba con tal motivo
extraordinariamente, sin decidirse por eso a vestir a la antigua\ldots{}
¡Tan grande era su buen sentido! Los balcones y miradores se poblaban de
damas, y en la calle la multitud seguía a los \emph{cruzados}. Sobre
todo los chicos tuvieron un día felicísimo. No faltó más para que
aquello se pareciese a la entrada de D. Quijote en Barcelona, sino que
los muchachos aplicaran a ciertas partes del caballo que montaba don
Pedro las célebres aliagas, y aun creo que algo de esto aconteció al fin
del triunfal paseo y cuando se volvían a la Isla.

~

Después del acontecimiento referido, ciertos sucesos tristísimos
determinan un paréntesis no corto en esta parte de la historia de mi
vida que voy refiriendo. El 1º de Junio sentíame enfermo y caí con la
fiebre amarilla, cual otros tantos que en aquella temporada fueron
víctimas del terrible tifus, con menos suerte que un servidor de
ustedes, el cual escapó de las garras de la muerte, después de verse en
estado tal que vislumbraba los horizontes del otro mundo.

Mi mal (ya me había atacado en la niñez con distinto carácter) no fue
muy largo. Yo estaba en la Isla. Asistiéronme mis amigos cariñosamente;
visitábame lord Gray todos los días, y Amaranta y doña Flora hicieron
largas guardias y vigilias en la cabecera de mi lecho. Cuando me vieron
fuera de peligro las dos lloraban de alegría.

Durante la convalecencia, D. Diego fue a visitarme, y me dijo:

---Mañana mismo vendrás a mi casa. Mis hermanas y mi novia me preguntan
por ti todos los días. ¡Qué susto se han llevado!

---Iré mañana---le respondí.

Pero yo estaba muy lejos de esperar la orden militar e inapelable que
por algún tiempo me desterrara de mi ciudad querida. Es el caso que D.
Mariano Renovales, aquel soldado atrevido que tan heroicas hazañas
realizó en Zaragoza, fue destinado a mandar una expedición que debía
salir de Cádiz para desembarcar en el Norte. Renovales era un hombre muy
bravo; pero con esta bravura salvaje de nuestros grandes hombres de
guerra: valor desnudo de conocimientos militares y de todos los demás
talentos que enaltecen al buen general. Había publicado el guerrillero
una proclama extravagantísima, en cuya cabeza se veía un grabado
representando a Pepe Botellas cayéndose de borracho y con un jarro de
vino en la mano, y el estilo del tal documento correspondía a lo innoble
y ridículo de la estampa. Sin embargo, por esto mismo le elogiaron mucho
y le dieron un mando. ¡Achaques de España! Estos majaderos suelen hacer
fortuna.

Pues señor, como decía, diose a Renovales un pequeño cuerpo de ejército,
y en este cuerpo de ejército me incluyeron a mí, obligándome, casi
enfermo todavía, a seguir al loco guerrillero en su más loca expedición.
Obedecí y embarqueme con él, despidiéndome de mis amigos. ¡Oh, qué
aventura tan penosa, tan desairada, tan funesta, tan estéril! Fiad
empresas delicadas a hombres ignorantes y populacheros que no tienen más
cualidad que un valor ciego y frenético.

No quiero contar los repetidos desastres de la expedición. Sufrimos
tempestades, aguantamos todo género de desdichas, y para colmo de
desgracia, lejos de hacer cosa alguna de provecho, parte de las tropas
desembarcadas en Asturias cayeron en poder de los franceses. Gracias
dimos a Dios los pocos que después de tres meses y medio de angustiosas
penas, pudimos regresar a Cádiz, avergonzados por el infausto éxito de
la aventura. Yo comparé a mis compañeros de entonces con los individuos
de la \emph{Cruzada} en la falta de sentido común.

Regresamos a Cádiz. Algunos fueron a recibirnos con júbilo creyendo que
volvíamos cubiertos de gloria, y en breves palabras contamos lo
ocurrido. La gente entusiasta y patriotera no quería creer que el
valiente Renovales fuese un majadero. Por desgracia, de esta clase de
héroes hemos tenido muchos.

Luego que descansamos un poco, después de poner el pie en tierra, fuimos
a presentarnos a las autoridades de la Isla. Era el 24 de Setiembre.

\hypertarget{viii}{%
\chapter{VIII}\label{viii}}

Una gran novedad, una hermosa fiesta había aquel día en la Isla.
Banderolas y gallardetes adornaban casas particulares y edificios
públicos, y endomingada la gente, de gala los marinos y la tropa, de
gala la Naturaleza a causa de la hermosura de la mañana y esplendente
claridad del sol, todo respiraba alegría. Por el camino de Cádiz a la
Isla no cesaba el paso de diversa gente, en coche y a pie; y en la plaza
de San Juan de Dios los caleseros gritaban, llamando viajeros:---¡A las
Cortes, a las Cortes!

Parecía aquello preliminar de función de toros. Las clases todas de la
sociedad concurrían a la fiesta, y los antiguos baúles de la casa del
rico y del pobre habíanse quedado casi vacíos. Vestía el poderoso
comerciante su mejor paño, la dama elegante su mejor seda, y los
muchachos artesanos, lo mismo que los hombres del pueblo, ataviados con
sus pintorescos trajes salpicaban de vivos colores la masa de la
multitud. Movíanse en el aire los abanicos, reflejando en mil rápidos
matices la luz del sol, y los millones de lentejuelas irradiaban sus
esplendores sobre el negro terciopelo. En los rostros había tanta
alegría, que la muchedumbre toda era una sonrisa, y no hacía falta que
unos a otros se preguntasen a dónde iban, porque un zumbido perenne
decía sin cesar:---¡A las Cortes, a las Cortes!

Las calesas partían a cada instante. Los pobres iban a pie, con sus
meriendas a la espalda y la guitarra pendiente del hombro. Los chicos de
las plazuelas, de la Caleta y la Viña, no querían que la ceremonia
estuviese privada del honor de su asistencia, y arreglándose sus
andrajos, emprendían con sus palitos al hombro el camino de la Isla,
dándose aire de un ejército en marcha, y entre sus chillidos y bufidos y
algazara se distinguía claramente el grito general:---¡A las Cortes, a
las Cortes!

Tronaban los cañones de los navíos fondeados en la bahía; y entre el
blanco humo las mil banderas semejaban fantásticas bandadas de pájaros
de colores arremolinándose en torno a los mástiles. Los militares y
marinos en tierra ostentaban plumachos en sus sombreros, cintas y
veneras en sus pechos, orgullo y júbilo en los semblantes. Abrazábanse
paisanos y militares congratulándose de aquel día, que todos creían el
primero de nuestro bienestar. Los hombres graves, los escritores y
periodistas, rebosaban satisfacción, dando y admitiendo plácemes por la
aparición de aquella gran aurora, de aquella luz nueva, de aquella
felicidad desconocida que todos nombraban con el grito placentero
de:---¡Las Cortes, las Cortes!

En la taberna del Sr.~Poenco no se pensaba más que en libaciones en
honor del gran suceso. Los majos, contrabandistas, matones, chulos,
picadores, carniceros y chalanes, habían diferido sus querellas para que
la majestad de tan gran día no se turbara con ataques a la paz, a la
concordia y buena armonía entre los ciudadanos. Los mendigos abandonaron
sus puestos corriendo hacia la Cortadura que se inundó de mancos, cojos
y lisiados, ganosos de recoger abundante cosecha de limosnas entre la
mucha gente, y enseñando sus llagas, no pedían en nombre de Dios y la
caridad, sino de aquella otra deidad nueva y santa y sublime,
diciendo:---¡Por las Cortes, por las Cortes!

~

Nobleza, pueblo, comercio, milicia, hombres, mujeres, talento, riqueza,
juventud, hermosura, todo, con contadas excepciones, concurrió al gran
acto, los más por entusiasmo verdadero, algunos por curiosidad, otros
porque habían oído hablar de las Cortes y querían saber lo que eran. La
general alegría me recordó la entrada de Fernando VII en Madrid en Abril
de 1808, después de los sucesos de Aranjuez.

Cuando llegué a la Isla, las calles estaban intransitables por la mucha
gente. En una de ellas la multitud se agolpaba para ver una procesión.
En los miradores apenas cabían los ramilletes de señoras; clamaban a voz
en grito las campanas y gritaba el pueblo, y se estrujaban hombres y
mujeres contra las paredes, y los chiquillos trepaban por las rejas, y
los soldados formados en dos filas pugnaban por dejar el paso franco a
la comitiva. Todo el mundo quería ver, y no era posible que vieran
todos.

Aquella procesión no era una procesión de santas imágenes, ni de reyes
ni de príncipes, cosa en verdad muy vista en España para que así llamara
la atención: era el sencillo desfile de un centenar de hombres vestidos
de negro, jóvenes unos, otros viejos, algunos sacerdotes, seglares los
más. Precedíales el clero con el infante de Borbón de pontifical y los
individuos de la Regencia, y les seguía gran concurso de generales,
cortesanos antaño de la corona y hoy del pueblo, altos empleados,
consejeros de Castilla, próceres y gentileshombres, muchos de los cuales
ignoraban qué era aquello.

La procesión venía de la iglesia mayor donde se había dicho solemne misa
y cantado un \emph{Te Deum}. El pueblo no cesaba de gritar \emph{¡Viva
la nación!,} como pudiera gritar ¡viva el rey!, y un coro que se había
colocado en cierto entarimado detrás de una esquina entonó el himno, muy
laudable sin duda, pero muy malo como poesía y música; que decía:

\small
\newlength\mlene
\settowidth\mlene{\quad que España está sufriendo,}
\begin{center}
\parbox{\mlene}{\quad Del tiempo borrascoso                         \\
                que España está sufriendo,                          \\
                va el horizonte viendo                              \\
                alguna claridad.                                    \\
                                                                    \\
                \null \quad La aurora son las Cortes                \\
                que con sabios vocales                              \\
                remediarán los males                                \\
                dándonos libertad.}                                 \\
\end{center}
\normalsize

El músico había sido tan inhábil al componer el discurso musical, y tan
poco conocía el arte de las cadencias, que los cantantes se veían
obligados a repetir cuatro veces \emph{que con sabios, que con sabios},
etc. Pero esto no quita su mérito a la inocente y espontánea alegría
popular.

Cuando pasó la comitiva encontré a Andrés Marijuán, el cual me dijo:

---Me han magullado un brazo dentro de la iglesia. ¡Qué gentío! Pero me
propuse ver todo y lo vi. Lindísimo ha estado.

---¿Pero ya empezaron los discursos?

---Hombre no. Dijo una misa muy larga el cardenal narigudo, y luego los
regentes tomaron juramento a los procuradores, diciéndoles:---¿Juráis
conservar la religión católica? ¿Juráis conservar la integridad de la
nación española? ¿Juráis conservar en el trono a nuestro amado rey D.
Fernando? ¿Juráis desempeñar fielmente este cargo?, a lo cual ellos iban
contestando que sí, que sí y que sí. Después echaron un golpe de órgano
y canto llano y se acabó. Gabriel, a ver si podemos entrar en el salón
de sesiones.

Yo no creí prudente intentarlo; pero fui hacia allá, codeando a diestro
y siniestro, cuando al llegar junto al teatro, ante cuyas puertas se
agolpaban masas de gente y no pocos coches, sentí que vivamente me
llamaban, diciendo:---Gabriel, Araceli, Gabriel, señor D. Gabriel,
Sr.~de Araceli.

Miré a todos lados, y entre el gentío vi dos abanicos que me hacían
señas y dos caras que me sonreían. Eran las de Amaranta y doña Flora. Al
punto me uní a ellas, y después que me saludaron y felicitaron
cariñosamente por mi feliz llegada, Amaranta dijo:

---Ven con nosotras, tenemos papeletas para entrar en la galería
reservada.

Subimos todos, y por la escalera pregunté a la condesa si algún
acontecimiento había modificado la situación de nuestros asuntos,
durante mi ausencia, a lo que me contestó:

---Todo sigue lo mismo. La única novedad es que mi tía padece ahora un
reumatismo que la tiene baldada. Doña María la domina completamente y es
quien manda en la casa y quien dispone todo\ldots{} No he podido ni una
vez sola ver a Inés, ni ellas salen a la calle, ni es posible
escribirle. Yo esperaba con ansia tu llegada, porque D. Diego prometió
llevarte allá. Cuando vayas espero grandes resultados de tu celosa
tercería. A lord Gray no hay quien le saque una palabra; pero los
indicios de lo que te dije aumentan. Por la criada sabemos que doña
María está con una oreja alta y otra baja, y que el mismo D. Diego, con
ser tan estúpido, lo ha descubierto y rabia de celos. Mañana mismo es
preciso que vayas allá, aunque yo dudo mucho que la de Rumblar quiera
recibirte.

No hablamos más del asunto porque el Congreso Nacional ocupó toda
nuestra atención. Estábamos en el palco de un teatro; a nuestro lado en
localidades iguales veíamos a multitud de señoras y caballeros, a los
embajadores y otros personajes. Abajo en lo que llamamos patio, los
diputados ocupaban sus asientos en dos alas de bancos: en el escenario
había un trono, ocupado por un obispo y cuatro señores más y delante los
secretarios del despacho. Poco habían unos y otros calentado los
asientos, cuando los de la Regencia se levantaron y se fueron como
diciendo: «Ahí queda eso.»

---Esta pobre gente---me dijo Amaranta,---no sabe lo que trae entre
manos. Mírales cómo están desconcertados y aturdidos sin saber qué
hacer.

---Se ha marchado el venerable obispo de Orense---dijo doña Flora.---Por
ahí se susurra que no le hacen maldita gracia las dichosas Cortes.

---Por lo que oigo, están eligiendo quien las presida---dije.---Hay aquí
un traer y llevar de papeletas que es señal de votación.

---Buenas cosas vamos a ver hoy aquí---añadió Amaranta con el regocijo
que da la esperanza de una diversión.

---Yo lo que quiero es que prediquen pronto---añadió doña
Flora.---Prontito, señores. Veo que hay muchos clérigos, lo cual es
prueba de que no faltarán picos de oro.

---Pero estos clérigos filósofos son torpes de lengua---afirmó
Amaranta.---Aquí hablarán más los seglares, y será tal el barullo, que
veremos escenas tan graciosas como las de un concejo de pueblo con
fuero. Amiga, preparémonos a reír.

---Ya parece que tienen presidente. Oigamos lo que lee aquel caballerito
que está en el escenario y que parece un mal actor que no sabe el papel.

---Está conmovido por la majestad del acto---repuso Amaranta.---Me
parece que estos señores darían algo ahora porque les mandasen a sus
casas. Verdaderamente las fachas no son malas.

---Desde aquí veo al vizconde de Matarrosa\footnote{Después Conde de
  Toreno.}---indicó doña Flora.---Es aquel mozalbete rubio. Le he visto
en casa de Morlá, y es chico despejado\ldots{} Como que sabe inglés.

---Ese angelito debiera estar mamando, y le van a dispensar la edad para
que sea diputado---repuso la condesa.---Como que no tiene más años que
tú, Gabriel. Vaya unos legisladores que nos hemos echado. Aquí tenemos
Solones de veinte abriles.

---Querida condesa---dijo la otra,---desde aquí veo todas las narices y
toda la boca de D. Juan Nicasio Gallego. Está abajo entre los diputados.

---Sí, allí está. De un bocado se tragará Cortes y Regencia. Es el
hombre de mejores ocurrencias que he visto en mi vida, y de seguro ha
venido aquí a reírse de sus compañeros de procuraduría. ¿No es aquel que
está a su lado D. Antonio Capmany? ¡Miren qué facha! No se puede estar
quieto un instante y baila como una ardilla.

---Ese que se sienta en este momento es Mejía.

---También veo la cara seráfica de Agustinito Argüelles. Dicen que este
predica muy bien. ¿Ve usted a Borrull? Cuentan que este no quiere
Cortes. Pero empiece de una vez la función ¡qué pesados son!

---Aquí como no se paga la entrada, no hay derecho a impacientarse.

---Ya está dispuesta la presidencia. ¿Tocarán un pito para empezar?

---Yo tengo una curiosidad por oír lo que digan\ldots{}

---Y yo.

---Será un disputar graciosísimo---dijo Amaranta,---porque cada cual
pedirá esto y lo otro y lo de más allá.

---Conque salga uno diciendo: «Yo quiero tal cosa,» y otro responda:
«Pues no me da la gana,» se animará esta desabrida reunión.

---¡Cuándo las habrán visto más gordas! Será gracioso oír a los clérigos
gritar: «Fuera los filósofos,» y a los seglares: «Fuera los curas.» Veo
con sorpresa que el presidente no tiene látigo.

---Es que guardarán las formas, amiga mía.

---¿En dónde han aprendido ellos a guardar formas?

---Silencio, que va a hablar un diputado.

---¿Qué dirá? Nadie lo entiende.

---Se vuelve a sentar.

---En el escenario hay uno que lee.

---Se levantarán algunos de sus asientos.

---Ya. Acaban de decir que quedan enterados.

---Nosotros también. Tanto ruido para nada.

---Silencio, señores, que vamos a oír un discurso.

---¡Un discurso! Oigamos. ¡Qué ruido en los palcos!

Si no calla el público, el presidente mandará bajar el telón.

---¿Es aquel clérigo que está allí enfrente quien va a hablar?

---Se ha levantado, se arregla el solideo, echa atrás la capa. ¿Le
conoce usted?

---Yo no.

---Ni yo. Oigamos qué dice.

---Dice que sería prudente adoptar una serie de proposiciones que tiene
escritas en un papelito.

---Bueno: léanos usted ese papelito, señor cura.

---Parece que hablará primero.

---¿Pero quién es?

---Parece un santo varón.

En los palcos inmediatos corría de boca en boca un nombre que llegó
hasta el nuestro. El orador era D. Diego Muñoz Torrero.

Señores oyentes o lectores, estas orejas mías oyeron el primer discurso
que se pronunció en asambleas españolas en el siglo XIX. Aún retumba en
mi entendimiento aquel preludio, aquella voz inicial de nuestras glorias
parlamentarias, emitida por un clérigo sencillo y apacible, de ánimo
sereno, talento claro, continente humilde y simpático. Si al principio
los murmullos de arriba y abajo no permitían oír claramente su voz, poco
a poco fueron acallándose los ruidos y siguió claro y solemne el
discurso. Las palabras se destacaban sobre un silencio religioso,
fijándose de tal modo en la mente que parecían esculpirse. La atención
era profunda, y jamás voz alguna fue oída con más respeto.

---¿Sabe usted, amiga mía---dijo en un momento de descanso doña
Flora,---que este cleriguito no lo hace mal?

---Muy bien. Si todos hablaran así, esto no sería malo. Aún no me he
enterado bien de lo que propone.

---Pues a mí me parece todo lo que ha dicho muy puesto en razón. Ya
sigue. Atendamos.

El discurso no fue largo, pero sí sentencioso, elocuente y erudito. En
un cuarto de hora Muñoz Torrero había lanzado a la faz de la nación el
programa del nuevo gobierno, y la esencia de las nuevas ideas. Cuando la
última palabra expiró en sus labios, y se sentó recibiendo las
felicitaciones y los aplausos de las tribunas, el siglo décimo octavo
había concluido.

El reloj de la historia señaló con campanada, no por todos oída, su
última hora, y realizose en España uno de los principales dobleces del
tiempo.

\hypertarget{ix}{%
\chapter{IX}\label{ix}}

---Atención, que van a leer el papelito.

D. Manuel Luxán leyó.

---¿Se ha enterado usted, amiga doña Flora?

---¿Acaso soy sorda? Ha dicho que en las Cortes reside la
\emph{Soberanía de la Nación}.

---Y que reconocen, proclaman y juran por rey a Fernando VII\ldots{}

---Que quedan separadas las tres potestades\ldots{} no sé qué
terminachos ha dicho.

---Que la Regencia que representa al Rey o sea poder ejecutivo preste
juramento.

---Que todos deben mirar por el bien del Estado. Eso es lo mejor, y con
decirlo, sobraba lo demás.

---Ahora se levanta gran tumulto entre ellos, amiga mía.

---Van a disputar sobre eso. Pues no levantará mal cisco el cleriguito.
¿Cómo se llama?\ldots{}

---D. Diego Muñoz Torrero.

---Parece que vuelve a hablar.

En efecto, Muñoz Torrero pronunció un segundo discurso en apoyo de sus
proposiciones.

---Ahora me ha gustado más, mucho más, señora condesa---dijo la de
Cisniega.---A este hombre le haría yo obispo. ¿No es justo y razonable
lo que ha dicho?

---Sí, que las Cortes mandan y el rey obedece.

---De modo, que según la Soberanía de la Nación, el gobierno del reino
está dentro de este teatro.

---Ahora le toca a Argüelles, amiga mía. Lo que me gusta es que todos
dicen que están de acuerdo. ¿Para cuándo dejan el disputar?

---Al principio todo es mieles. Repare usted que estamos en el primer
acto.

---Ahora habla Argüelles.

---¡Oh, qué bien! ¿Ha conocido usted muchos predicadores que se expresen
con esa elegancia, esa soltura, esa majestad, ese elevado tono, el cual
nos sorprende y embelesa de tal modo que no podemos apartar la atención
del orador, encantándose igualmente con su presencia y voz, la vista y
el oído?

---¡Cosa incomparable es esta!---expresó con entusiasmo doña
Flora.---Diga usted lo que quiera, han hecho muy bien en traer a España
esta novedad. Así todas las picardías que cometan en el gobierno se
harán públicas, y el número de los tunantes tendrá que ser menor.

---Sospecho que esto va a ser más brillante que útil---repuso la
condesa.---Oradores creo que no faltarán. Hoy todos han hablado bien;
¿pero acaso es tan fácil la obra como la palabra?

Y de este modo iban comentando los discursos que sucedieron al de Muñoz
Torrero, los cuales alargaban tanto la sesión, que bien pronto se hizo
de noche y el teatro fue encendido. No por la tardanza se cansaron las
dos damas, quienes, como el resto de la concurrencia, permanecieron en
sus asientos hasta entrada la noche, gozando de un espectáculo que hoy a
pocos cautiva por ser muy común, pero que entonces se presentaba a la
imaginación con los mayores atractivos. Los discursos de aquel día
memorable dejaron indeleble impresión en el ánimo de cuantos los
escucharon. ¿Quién podría olvidarlos? Aún hoy, después que he visto
pasar por la tribuna tantos y tan admirables hombres, me parece que los
de aquel día fueron los más elocuentes, los más sublimes, los más
severos, los más superiores entre todos los que han fatigado con sus
palabras la atención de la madre España. ¡Qué claridad la de aquel día!
¡Qué oscuridades después, dentro y fuera de aquel mismo recinto, unas
veces teatro, otras iglesia, otras sala, pues la soberanía de la nación
tardó mucho en tener casa propia! Hermoso fue tu primer día, ¡oh, siglo!
Procura que sea lo mismo el último.

Ya avanzada la noche, corrió un rumor por las tribunas. Los regentes
iban a jurar, obligados a ello por las Cortes. Era aquello el primer
golpe de orgullo de la recién nacida soberanía, anhelosa de que se le
hincaran delante los que se conceptuaban reflejo del mismo Rey. En los
palcos unos decían: «Los regentes no juran:» y otros: «Vaya si jurarán.»

---Yo creo que unos jurarán y otros no---dijo Amaranta.---Ellos han
intentado tener de su parte el pueblo y la tropa; pero no han encontrado
simpatías en ninguna parte. Los que tengan un poco de valor, mandarán a
las Cortes a paseo. Los débiles se arrastrarán en ese escenario, donde
me parece que resuena todavía la voz del gracioso Querol y de la
Carambilla, y besarán el escabel donde se sienta ese vejete verde, que
es, si no me engaño, don Ramón Lázaro de Dou.

---¡Que juren! Con eso no habrá conflictos. Parece que hay tumulto
abajo.

---Y también arriba, en el paraíso. El pueblo cree que está viendo
representar el sainete de Castillo \emph{La casa de vecindad}, y quiere
tomar parte en la función. ¿No es verdad, Araceli?

---Sí señora. Ese nuevo actor que se mete donde no le llaman, dará
disgustos a las Cortes.

---El pueblo quiere que juren---dijo Flora.

---Y querrá también que se les ponga una soga al cuello y se les cuelgue
de las bambalinas.

---Y fuera también hay marejadita.

---Me parece que esos que han entrado en el escenario son los regentes.

---Los mismos. ¿No ve usted a Castaños, al viejo Saavedra?

---Detrás vienen Escaño y Lardizábal.

---¡Cómo!---exclamó la condesa con asombro.---¿También jura Lardizábal?
Ese es el más orgulloso enemigo de las Cortes, y andaba por ahí diciendo
a todo el mundo que él se guardaría las Cortes en el bolsillo.

---Pues parece que jura.

---Ya no hay vergüenza en España\ldots{} Pero no veo al obispo de
Orense.

---El obispo de Orense no jura---murmuraron las tribunas en rumoroso
coro.

Y en efecto, el obispo de Orense no juró. Hiciéronlo humildemente los
otros cuatro, con mala gana sin duda. La opinión pública en general
estaba muy pronunciada contra ellos. Levantose la sesión, y salimos
todos, oyendo a nuestro paso las opiniones del público sobre el suceso
que había puesto fin al solemne día. Casi todos decían:

---¡Ese testarudo vejete no ha querido jurar! Pero el juramento con
sangre entra.

---Que lo cuelguen. No acatar el decreto que se llamará de 24 de
Setiembre, es dar a entender que las Cortes son cosa de broma.

---Yo me quitaba de cuentos, y al que no bajara la cabeza, le mandaría
prender, y después\ldots{}

---Si esos señores no quieren más que gobierno absoluto\ldots{}

En cambio otros, los menos por cierto, se expresaban así:

---¡Magnífico ejemplo de dignidad ha dado el obispo a sus compañeros!
Humillar el poder real ante cuatro charlatanes\ldots{}

---Veremos quién puede más---decían unos.

---Veremos quién más puede---respondían los otros.

Los dos bandos que habían nacido años antes y crecían lentamente, aunque
todavía débiles, torpes y sin brío, iban sacudiendo los andadores,
soltaban el pecho y la papilla y se llevaban las manos a la boca,
sintiendo que les nacían los dientes.

\hypertarget{x}{%
\chapter{X}\label{x}}

Despedime de Amaranta y su amiga, prometiendo visitarlas al día
siguiente, como en efecto lo hice. En un café de Cádiz juntóseme D.
Diego, quien al punto renovó sus promesas de llevarme a la casa materna,
en lo cual le di tanta prisa, que fijamos para el próximo día la visita.
También hice una a lord Gray, al cual hallé sin variación alguna, y como
le dijese que yo pensaba ir a casa de doña María, se sorprendió,
asegurándome después que él iba todas las noches.

Cuando llegó el anochecer del día indicado, fuimos Rumblar y yo, previa
repetición de las advertencias que el caso requería.

---Ten mucho cuidado---me dijo,---de fingirte mojigato, si no quieres
que te echen a la calle. Mis hermanas, a quien dije que estabas aquí,
desean que vayas; pero no te la eches de galante con ellas. Mucho
cuidado con aludir a mis salidas de noche, porque lo hago a escondidas
de mi señora mamá. A los señores que veas allí, trátales cual si fueran
lumbreras de la patria y prodigios de talento y virtudes. En fin, confío
en tu buen sentido.

Llegamos a la casa, que estaba en la calle de la Amargura y era de
hermosa apariencia. Vivía en el piso alto la de Leiva y en el principal
la de Rumblar, quien por el reciente reumatismo de su ilustre parienta,
ejercía el cargo de jefe y director supremo de la familia con toda la
extensión propia de su carácter. Al entrar y subir detúvonos un lejano y
solemne rumor de rezos, y D. Diego dijo:

---Aguardemos aquí; que están rezando el rosario con Ostolaza, Tenreyro
y D. Paco. A este ya le conoces. Los otros son diputados, que vienen
aquí todas las noches.

Mientras aguardábamos observé la casa, que era alegre y bonita como
todas las de Cádiz. Espaciosas vidrieras cerraban el corredor por el
patio, y en las paredes no se veía un palmo de superficie desocupado de
cuadros al óleo, representando asuntos diversos, y confundidos los
religiosos con los profanos. Al fin, concluido el rezo, tuve el honor de
entrar en la sala, donde estaba doña María con sus dos niñas, D. Paco y
tres caballeros más que yo no conocía. Recibiome la de Rumblar con
cierta cortesanía ceremoniosa y un tanto finchada, pero afablemente y
mostrándome benevolencia de alto a bajo, es decir, entre generosa y
compasiva. Las niñas, observando el ritual a que estaban acostumbradas,
me hicieron una reverencia, sin desplegar los labios; D. Paco, tan
pedante en Cádiz como en Bailén, hízome grandilocuentes cumplidos y los
demás personajes miráronme con recelosa prevención, sin mostrarme
urbanidad más que con algunas rígidas inclinaciones de cabeza.

---Has llegado tarde al rosario---dijo doña María a D. Diego después que
me indicó un asiento.

---¿Pero no dije a usted---respondió el joven,---que lo rezaba esta
tarde en el Carmen Calzado? De allí vengo ahora, junto con Gabriel, que
volvía de confesarse con el padre Pedro Advíncula.

---¡Qué excelente sujeto es el padre Pedro Advíncula!---me dijo en tono
sumamente ponderativo doña María.

---No existe otro en toda la redondez de Cádiz---respondí,---con
especialidad para lo tocante al confesonario. ¿Pues y en el púlpito? ¿Y
quién le echará la zancadilla a cantar una epístola?

---Es verdad.

---A mí me cautiva oírle cantar la epístola---repitió D. Diego.

---Yo celebro mucho---me dijo doña María,---los grandes adelantamientos
que ha hecho usted en su carrera.

Me incliné ante la matrona con el mayor respeto.

---Toda persona de rectitud y caballerosidad, atenta al buen servicio de
la religión y del rey---continuó,---no puede menos de encontrar premio a
su trabajo. Yo sentí mucho que mi hijo no siguiese en el ejército algún
tiempo más\ldots{}

---Harto trabajamos Gabriel y yo junto al puente de Herrumblar---dijo D.
Diego.---Verdaderamente, señora madre, si no es por nosotros\ldots{}
Ello fue que hicimos un movimiento con nuestro escuadrón en tales
términos que\ldots{} ¿te acuerdas, Gabriel? Francamente, si no es por
nosotros\ldots{}

---Calla, vanidoso---dijo doña María.---Más ha hecho el señor que tú y
no se alaba de ello. La propia alabanza es cosa ruin e indigna de
personas bien nacidas. ¿Estará mucho en Cádiz el Sr.~D. Gabriel?

---Hasta que concluya el sitio, señora. Después pienso dejar las armas y
seguir en mi ardiente vocación, que me impele a la carrera de la
Iglesia.

---Alabo mucho su resolución, y esclarecidos santos tiene el cielo, que
primero fueron valientes soldados, como San Ignacio de Loyola, San
Sebastián, San Fernando, San Luis y otros.

---¿Ha estudiado usted teología?---me preguntó un señor de los
presentes.

---Mi maleta de campaña no contiene más que libros de teología, y desde
que tengo un rato de vagar, entre batalla y batalla, me harto de leer
una materia que es para mí más grata que las mejores novelas. Las
tristes horas de la guardia me dan espacio y tiempo para mis
meditaciones.

---Asunción, Presentación---dijo doña María con entusiasmo,---aquí
tenéis un ejemplo que debe sorprenderos y admiraros.

Asunción y Presentación, al oír que yo era una especie de santo, me
contemplaron con admiradas. Yo las miré también. Estaban tan bonitas,
más bonitas que en Bailén; pero oprimidas bajo la exagerada pesadumbre
de la autoridad materna, sus hermosos ojos estaban llenos de tristeza.
Sin que su madre lo advirtiera, dijéronse algunas palabras por lo bajo.

---¿Y qué nuevas nos trae usted de la Isla?---me preguntó doña María.

---Señora, ayer se inauguró esa jaula de locos. Ya sabrá usted que el
señor obispo de Orense se ha negado, con pretexto de enfermedad, a jurar
ante las Cortes.

---Y ha hecho perfectamente. En verdad no se concibe que haya gente tan
loca\ldots{} Antes del rosario nos explicaba el Sr.~Ostolaza lo que
entienden ellos por la soberanía de la nación, y nos hemos horripilado.
¿Verdad, niñas?

---¡Dios nos tenga en su mano!---exclamé yo.---Y ahora se susurra que
nos van a dar lo que llaman \emph{libertad de la imprenta}, que consiste
en permitir a cada uno escribir todas las maldades que quiera.

---Y luego hablan de vencer al francés.

---Los excesos de nuestros políticos---dijo Ostolaza,---excederán con
mucho a los de la revolución francesa. Acuérdese usted de lo que le
digo.

Observé entonces a aquel hombre, el mismo que tanto figuró después en la
camarilla del rey, durante la segunda época constitucional, y puedo
decir que era grueso, de cara redonda, coloradota y reluciente, mirar
provocativo, hablar chillón y ademanes desembarazados y casi siempre
descompuestos. Junto a él estaba el llamado Teneyro, diputado también,
cura de Algeciras, hombre con pretensiones y fama de gracioso, aunque
más que a la agudeza de los conceptos, debía esta al ceceo con que
hablaba; de cuerpo mezquino, de ideas estrafalarias, tan pronto demagogo
furibundo, como absolutista rabioso; sin instrucción, sin principios ni
más conocimientos que los del toque del órgano, cuyo arte medianamente
poseía. El tercero, D. Pablo Valiente, no era ridículo, ni en el trato
ordinario se distinguía por cosa alguna chocante, en maneras o en
lenguaje.

Contestando a Ostolaza, dije yo con el acento más grave que me era
posible:

---¡El cielo se apiade de nuestra infortunada nación, y nos traiga
pronto a nuestro amado monarca D. Fernando el VII!

El nombre del soberano lo acompañé de una reverencia tan exagerada que
casi hube de besarme las rodillas.

---Pues se dice por ahí---indicó Teneyro,---que van a procesar al obispo
de Orense.

---No se atreverán a ello---repuso Valiente, sacando su caja de tabaco y
ofreciendo del oloroso polvo a los circunstantes.

---¿A qué no se atreverá, señores\ldots{} señores, a qué no se atreverá
esta desalmada grey de filósofos y ateístas?---exclamé yo mirando al
techo.

---Señor oficial---me dijo doña María,---es indudable que ustedes los
militares tienen la culpa de que los \emph{cortesanos}\ldots{} así los
llamo yo\ldots{} estén tan ensoberbecidos. Dicen que la Regencia tanteó
a la tropa para dar un golpe, pero la tropa no quiso ponerse de su
parte.

---La tropa---dijo Ostolaza,---ha cometido la falta de inclinarse al
populacho.

---Lo que no se ha hecho, señores---dije yo con profético tono,---se
hará.

Y repetí varias veces, mirando a todos lados, el enérgico «se hará.»

---Si todos fueran como tú, Gabriel---me dijo don Diego,---pronto
acabarían las picardías que estamos viendo.

---¿Durarán las Cortes hasta el mes que viene, señor de
Valiente?---preguntó la de Rumblar.

---Durarán algo más, señora. A no ser que los franceses envalentonados
con nuestras discordias, entren en Cádiz, y hagan con todos los que aquí
estamos un picadillo. Yo he dicho que la soberanía de la nación por un
lado y la libertad de la imprenta por otro, son dos obuses cargados de
horrorosos proyectiles que nos harán más daño que los que ha inventado
Villantroys.

---Caballero---dije yo afeminadamente,---esa comparacioncita es exacta y
procuraré retenerla en la memoria.

---Deploro tantos errores---dijo la dueña de la casa.---Pero aquí,
Sr.~D. Gabriel, no tomamos a pecho la política, y los que en casa se
reúnen no hacen más que departir discretamente sobre el mal gobierno y
los filosofastros. Yo no me ocupo más que del matrimonio de mi querido
hijo, que se efectuará en breve, y de completar la educación religiosa
de mi hija---señaló a Asunción,---que debe entrar muy pronto en un
convento de Recoletas, siguiendo su decidida e inquebrantable
inclinación. Ocupaciones son estas que llenan alegremente mi cansada
vida, y a las que me consagro con el mayor celo.

Asunción había bajado los ojos, y Presentación me miraba, queriendo leer
en mi cara el efecto que me producían las palabras de su mamá.

---¿Enviasteis recado a Inés?---preguntó doña María.---Diego, tu futura
esposa estará sin duda enojada contigo, por tu mal comportamiento y
desaplicación. Necesario es que varíes de conducta. Ahora, cuando baje,
puedes manifestarle con palabras tiernas tu propósito de no ofenderla
más, como lo has hecho saliendo a la calle por las tardes en la hora que
tengo dispuesto hables con ella y le recites alguna fábula bonita o
poesía instructiva. Yo, señor D. Gabriel---y se dirigió a mí de
nuevo,---no gusto de tiranizar a la juventud. Conozco que es preciso ser
tolerante con los muchachos, sobre todo cuando llegan a cierta edad, y
sé muy bien que los tiempos presentes exigen algo más de holgura que los
pasados en los lazos que atan a los jóvenes con sus familias. Con estos
principios, permito a mi nuera que baje a la tertulia y platique con
personas finas y juiciosas sobre asuntos profanos, porque una muchacha
destinada al siglo y a dar lustre a una gran casa como la suya, no debe
ser criada con aquel encogimiento y estrechez que tan bien sienta en la
que sólo ha de vivir en su casa, bien reducida a un decoroso celibato,
bien instruyéndose para servir a Dios en el mejor y más perfecto de los
estados. Mis dos niñas viven aquí gozosas sin apetecer bailes, ni
paseos, ni teatros. No soy yo enemiga tampoco de que se diviertan, ni
crea usted que estoy siempre con el rosario en la mano, haciéndolas
rezar y aburriéndolas con un excesivo manoseo de las cosas santas, no.
También aquí se habla de cosas mundanas, siempre con el debido
comedimiento. A veces tengo que imponer silencio, mandando que cesen las
controversias sobre teología, porque lord Gray, que viene aquí muy a
menudo, gusta de tratar con desenvoltura asuntos muy delicados.

---Como que anoche---dijo D. Paco inoportunísimamente,---dio en afirmar
que no comprendía el misterio de la Encarnación, para que la señorita
Asunción se lo explicara.

---Estoy hablando yo, Sr.~D. Paco---dijo con firmeza y enojo la
condesa.---Nada importa ahora lo que lord Gray hiciera o dejase de hacer
anoche\ldots{} Pues como decía, aquí viene lord Gray, un sujeto
respetabilísimo y tan formal y circunspecto, que no hay otro que se le
iguale. Ellas se entretienen oyéndole contar sus aventuras. ¿Conoce
usted a lord Gray?

---Sí, señora. Es un hombre muy digno y temeroso de Dios. ¿Pero no saben
ustedes que parece inclinado a convertirse al catolicismo?

---¡Jesús y qué me dice usted!---exclamó con asombro y júbilo doña
María.---Aquí se ha tratado algunas veces este punto, y las niñas y yo
le hemos exhortado a que tome tan saludable determinación.

---Como suelo pasarme las horas muertas en el Carmen Calzado---dije
yo,---he visto entrar varias veces a lord Gray en busca del padre
Florencio, que es el mejor catequizador de ingleses que hay en todo
Cádiz.

---Lord Gray no ha de faltar esta noche---dijo doña María.---Y usted,
Sr.~D. Gabriel, ¿no nos acompañará algunos ratitos?

---Señora---respondí,---de buen grado lo haría; pero mis ocupaciones
militares y la necesidad que tengo de despachar de una vez todo el
capítulo de \emph{prescientia}, que es el más difícil de todos, me
retendrán en la Isla.

---¿Y qué opina usted de la \emph{prescientia}?---me preguntó Ostolaza
cuando yo estaba muy lejos de esperar semejante embestida.

---¿Qué opino yo de la \emph{prescientia}?---dije tratando de no
turbarme para contestar alguna ingeniosa vulgaridad que me sacase del
compromiso.

---Opinará lo mismo que San Agustín, \emph{secundum Augustinus}---indicó
oficiosamente D. Paco, que anhelaba mostrar su erudición.

---Ya están las niñas con cada ojo\ldots---dijo doña María observando
que sus hijas atendían a la planteada discusión con demasiado
interés.---Niñas, dejad a los hombres que debatan estas cosas tan
intrincadas. Ellos se sabrán lo que se dicen. No abrir tales ojazos, y
miren los cuadros y las pinturas del techo, o hablen conmigo,
preguntándome si se me alivia el dolor del hombro.

---Lo mismo que San Agustín---indicó don Diego.---Opinará como San
Agustín y como yo.

---Según y conforme---dije recapacitando.---¿Ustedes piensan como San
Agustín?

Ostolaza, Teneyro y D. Paco se desconcertaron.

---Nosotros\ldots{}

---Supongo que conocerán los nuevos tratados\ldots{}

A este punto llegaba la controversia, cuando entró lord Gray a sacarme
del apuro. No pudiera llegar en mejor ocasión. Recibiéronle doña María y
sus tertulios con la mayor cordialidad y agasajo, y él saludó a todos
con afectado encogimiento. Tal vez extrañará alguno de los que me oyen o
me leen, que con tan buena amistad fuera recibido un extranjero
protestante en casa donde imperaban ciertas ideas con absoluto dominio;
pero a esto les contestaré que en aquel tiempo eran los ingleses objeto
de cariñosas atenciones, a causa del auxilio que la nación británica nos
daba en la guerra; y como era opinión o si no opinión, deseo de muchos,
que los ingleses, y mayormente los hermanos Wellesley, no veían con
buenos ojos la novedad de la proyectada Constitución, de aquí que los
partidarios del régimen absoluto trajeran y llevaran con palio a
nuestros aliados. Lord Gray además con su ingeniosísima labia, su
simpático carácter, y también poniendo en práctica estudiadas artimañas
y mojigaterías, como yo, había conseguido hacerse respetar y querer
vivamente de doña María. Además solía ridiculizar con gran desenfado las
ceremonias protestantes.

Mientras lord Gray respondía a ciertas enfadosas preguntas que le hizo
Ostolaza, doña María llamó a sus hijas y dijo a Asunción, no tan por lo
bajo que yo dejase de oírlo:

---Mira, Asunción, habla con lord Gray un ratito; coge con disimulo el
tema de la religión y sondéale, a ver si es cierto que está dispuesto a
abjurar sus errores, por abrazarse a nuestra santa doctrina.

En aquel instante sentí ruido de pasos y entró Inés. ¡Dios mío, qué
guapa estaba, pero qué guapa! No recuerdo si en el libro anterior hablé
a ustedes de la soltura, de la elegancia, de la armoniosa
proporcionalidad que el completo desarrollo había dado a su bella
figura. Además de esto, encontrábale mayor animación en el rostro, y una
grata expresión de conformidad y satisfacción, no menos simpática que su
antigua tristeza, resto de la miserable y ruin vida de la infancia.
Observándola, consideré cuánto había ganado en encantos y atractivos
aquella criatura, añadiendo a sus bellezas naturales, a su discreción e
ingénito saber, la dulce cortesanía y las gracias que infunde el trato
frecuente con personas distinguidas y superiores. En su cara advertí el
extraño realce que da la conciencia del propio mérito, lo cual no es lo
mismo que vanidad.

No parecía haber perdido la hermosa modestia que la hacía tan simpática;
pero sí aquella especie de encogimiento, aquel desmedido amor a la
oscuridad, que emanaban del malestar hallado en su repentino cambio de
fortuna. Había adquirido lo que le faltaba cuando la vi en Córdoba y en
el Pardo, el perfecto conocimiento de su posición y las mil menudencias
personales, accidentes casi imperceptibles de la voz, del gesto, de la
mirada con que el individuo da a entender claramente que se halla donde
debe hallarse. Estaba más alta, un poco más gruesa, con el color menos
pálido, la boca más risueña, los ojos no menos seductores y
arrebatadores que los de su madre, célebres en toda la redondez de
España, la voz más segura, sonora y grave, y el conjunto de su persona
respirando firmeza, vida, soltura y nobleza. ¡Oh imagen tan perfecta
vista como soñada! ¿Fue suerte o desgracia haberte conocido?

\hypertarget{xi}{%
\chapter{XI}\label{xi}}

Inés, no indiferente a mi presencia, según comprendí, pero tampoco
sorprendida, debía saber que yo estaba allí.

---¡Ah!---exclamé con despecho para mis adentros.---La muy pícara aunque
la llamaron, no bajó hasta que vino el maldito inglés.

Doña María me presentó ceremoniosamente a ella diciendo:

---A este caballero le conocimos en nuestra casa de Bailén cuando la
célebre batalla. Es amigo del que va a ser tu marido; allí pelearon
juntos con tan buena suerte, que, según afirma Diego, si no es por
ellos\ldots{}

---Gabriel es un gran militar---dijo don Diego.---¿Pero no le conoces
tú? Es amigo de tu prima la condesa.

Doña María frunció el ceño.

---En efecto---dije yo,---tuve el honor de conocer en Madrid a la señora
condesa. Ambos teníamos un mismo confesor. Yo solicité de la señora
condesa que me consiguiese una beca en el arzobispado de Toledo; pero
después me vi obligado a servir al rey, y salí de la corte.

---Este joven---añadió doña María,---nos acompañará algunas noches,
robando tal cual rato a sus estudios religiosos y a las meditaciones
místicas que le traen tan absorbido. Hoy el servicio de las armas le
obliga a sofocar su ardiente vocación; pero cantará misa después de la
guerra. ¡Noble ejemplo que debieran imitar la mayor parte de los
militares! Yo me complazco, hija mía, en que se reúnan aquí personas
formales y de excelentes y sólidos principios. Caballero---añadió
encarando conmigo,---esta damisela es mi futura nuera, prometida esposa
de este mi amado hijo don Diego.

Inés me hizo una profunda reverencia. Se sonrió al mismo tiempo,
comprendiendo el astuto ardid de mi fingida religiosidad.

¿En tanto dónde estaba lord Gray? Extendí la vista y le vi tras el
respaldo del monumental sillón de doña María, muy enfrascado en estrecha
plática con Asunción, que sin duda le estaba convenciendo de la
superioridad del catolicismo con respecto al protestantismo. A cada paso
apartaba él los ojos de su interlocutora para mirar a Inés.

---Bien decía el tunante---observé para mí,---que se valía de las
discretas amigas. La otra con su santidad es quien les lleva y trae los
recaditos.

Inés me dijo con dulce ironía:

---Celebro mucho que esté usted tan decidido a seguir la carrera
eclesiástica. Hace usted bien, porque hoy no hacen falta militares, sino
buenos clérigos. El mundo está tan pervertido, que no lo curarán las
espadas sino las oraciones.

---Esta afición la tengo desde muy niño---repuse,---y nadie puede
apartarla de mí porque sobrevive a todas mis alternativas y desgracias.

Inés miraba a cada instante el grupo formado por el inglés y Asunción.
También doña María volvió allá los ojos, y dijo:

---Hija, basta ya. No marees al buen lord Gray. Ven a mi lado.

La muchacha acudió al lado de su madre, y al mismo tiempo Inés, por
indicación muda de la condesa, pasó al lado del inglés. Yo estaba
asombrado de aquel ir y venir y del incomprensible diálogo de expresivas
miradas que las muchachas tenían constantemente, trabado entre sí. Me
propuse observar atentamente, para descubrir los misterios que allí
pudieran existir; pero doña María distrajo mi atención, diciéndome:

---Sr.~D. Gabriel, usted, como persona casi divorciada del siglo, aunque
en su continente y rostro no se advierte nada que lo indique,
comprenderá que en estas recatadas tertulias de mi casa no se puede
tener con las muchachas la licenciosa tolerancia que madres inadvertidas
y ciegas tienen con sus hijas en otras familias. Por eso verá usted que
apenas permito a mis niñas hablar un poco con Ostolaza, con lord Gray o
con usted, si bien ha habido noches en que les he consentido
conversaciones de quince minutos en distintas horas. Comprendo que mi
sistema, aunque no es riguroso, será criticado por los que dan rienda
suelta a los impulsos naturales de la juventud. Pero no me importa.
Usted me hace justicia sin duda y alaba la prudencia de mi proceder.

---Seguramente, señora---respondí con afectación y pedantería---¿qué
cosa más sabia, ni más prudente puede haber que prohibir en absoluto a
las niñas toda conversación, diálogo, mirada o seña con hombre que no
sea su confesor? ¡Oh, señora condesa, parece que ha adivinado usted mi
pensamiento! Como usted, yo he observado la corrupción de las
costumbres, hija de la desenvoltura francesa; como usted, he observado
el descuido de las madres, la ceguera de los padres, la malicia de las
tías, la complicidad de las primas y la debilidad de las abuelas; y he
dicho: «orden, rigor, cautela, reclusión, tiranía, o si no dentro de
poco la sociedad se precipitará en los abismos del pecado.» Nada, nada,
señora condesa, yo lo aconsejo a todas las madres de familia que
conozco, y les digo: «mucho cuidado con las niñas mientras sean
solteras. Después de casadas, allá se entiendan ellas, y si quieren
tener dos docenas de cortejos, háganlo.»

---En todo estamos de acuerdo---dijo doña María,---menos en esto último,
pues ni de solteras ni de casadas, les tolero la inmoralidad. ¡Ay, yo
tengo ideas muy raras, Sr.~D. Gabriel! Me asombro de ver por ahí madres
muy cristianas, que celando hasta lo sumo las hijas solteras, ven con
indiferencia los pecadillos de las casadas. Yo no soy así; por eso no
quiero que se casen mis niñas; no, jamás, jamás. Casadas estarían libres
de mi autoridad, y aunque no las creo capaces de nada malo, la idea de
que pueden cometer una falta, siéndome imposible castigarla, me
horripila.

---El gran sistema es el mío, señora; este sistema que no ceso de
recomendar a todas las madres que conozco. Orden, rigor, silencio,
encierro perpetuo y esclavitud constante. Mis lecturas y meditaciones me
han inspirado estas ideas.

---Son también las mías. Mi hija Asunción entrará pronto en un convento,
y Presentación está destinada a ser soltera, porque así lo he resuelto
yo.

---Cosa justísima y naturalísima que usted haya resuelto eso.

---Siendo el destino de la una el claustro y de la otra el celibato, ¿a
qué viene el consentirles conversaciones con los jóvenes?

---Es claro\ldots{} a qué viene\ldots{} No aprenderían más que cosas
malas, pecados\ldots{} ¡y qué pecados!

---Pero como es preciso transigir un poquito con las costumbres, que
exigen cierta licencia, suele írseme la mano en esto del rigor. Ya ve
usted, a casa suelen venir algunas personas muy distinguidas, honestas y
prudentes, sí, pero de mundo. Necesito contemporizar con ellas, por no
aparecer gazmoña, intolerante y extremada. Felizmente baja todas las
noches a mi tertulia, Inés, a quien como muy próxima a ser mujer casada,
puede permitirse que sostenga coloquios tirados con tal cual persona
decente y bien nacida. Si no fuera por ella, lord Gray se aburriría
grandemente en casa. ¿No cree usted, que a una muchacha que va a ser
mayorazga y que ocupará posición muy encumbrada en la corte, se le debe
dar cierta libertad?

---Todas las libertades, señora, todas. ¡Una mayorazga! Pues digo; si me
la hacen camarista de reina, o dama de honor de emperatrices, ¿qué ha de
hacer sin la desenvoltura, el desenfado, la astucia que el buen servicio
y concierto de los palacios exige?

---Cierto; a cada cual se le debe educar según su destino. En posiciones
elevadísimas no puede sostenerse todo el rigor de los principios, según
dice la gente, aunque ciertas leyes sí deben regir en todas partes. Sin
embargo, como así viene de atrás, debemos respetar la obra de nuestros
mayores, quienes harto supieron lo que se hacían.

---Justamente.

---Pero me parece que se prolonga demasiado la conversación de Inés con
lord Gray, y voy a hacer que hablen en corrillo donde les oigamos todos.
Sr.~D. Gabriel, ni un momento debe abandonarse el ejercicio de la
prolija autoridad materna. ¡La autoridad! ¿Qué sería del mundo sin la
autoridad?

---En efecto, ¿qué sería? ¡El caos, el abismo!

Doña María, que reglamentaba los diálogos de sus tertulias como mueve y
ordena un general experto los movimientos de una batalla campal, dispuso
que Inés continuase hablando con lord Gray, y que Presentación pegase la
hebra con Ostolaza. En tanto Asunción charlaba en voz bastante alta con
su hermano, diciéndole cosas cuyo sentido no pude entender. Ostolaza,
Teneyro y D. Paco estaban muy metidos en lenguas disertando sobre los
grandes males de la educación a la moderna, y forzosamente me enredaron
en su coloquio, teniendo ocasión de lucir mi intolerancia, y un poco de
cierta erudicioncilla trasnochada que yo tenía para el caso. Poco
después volví al lado de doña María a punto que don Diego, apartándose
de su hermana, hacía lo mismo, y le oí decir:

---Señora madre, a ser usted, yo no permitiría a Inés tantas intimidades
con lord Gray. Francamente, señora, esto no me gusta, y menos cuando veo
que la que va a ser mi mujer, se está los minutos de Dios oyéndole y
contestándole sin pestañear.

---Diego---manifestó doña María con severo acento.---Me enfada la bajeza
de tus conceptos, que indican la ruindad de tus juicios. Si Inés fuera
tu hermana, podrías tener esos escrúpulos; pero siendo tu futura esposa,
cuanto has dicho es ridículo. Una gran señora, ¿ha de ser encogida y
corta de genio como una novicia de convento?

D. Diego, oído esto, se acercó de muy mal talante a sus hermanas.

---Sr.~de Araceli---me dijo doña María,---la juventud es así. Comprendo
los celillos de mi hijo. Verdaderamente Inés se alarga demasiado con
lord Gray. Aunque le supongo a usted poco aficionado a perder el tiempo
conversando con muchachas frívolas, hágame el favor de departir un rato
con mi futura nuera.

Doña María miró a Inés con enojo, y dirigiéndose luego a lord Gray, le
llamó con afectuosa súplica.

Inés quedó sola y acudí hacia ella. Por primera vez durante la tertulia
hallaba ocasión de poderle hablar lejos de los demás, y la aproveché con
presteza. Ella, anticipándose al afán con que yo iba a hablarle, me
dijo:

---¿Mi prima te ha mandado aquí? ¿Me traes algún recado de ella?

---No---respondí.---No me ha mandado tu prima. No he venido por traerte
recado alguno. He venido porque he querido, y por el deseo de verte y de
saber por mí mismo que me has olvidado.

---Por Dios---me contestó disimulando su emoción.---Repara dónde estás.
La condesa no cesa de observarme. Aquí es preciso fingir a todas horas,
y disimular los pensamientos. ¿Por qué no has venido antes? Pero di: ¿mi
prima no te ha dado ningún recado?

---¿Qué me importa tu prima?---exclamé con enfado.---Tú no sospechabas
que viniera a sorprenderte.

---¿Pero estás loco?, doña María no me quita los ojos.

---Vaya al diantre doña María. Respóndeme, Inés, a lo que te pregunto, o
gritaré y escandalizaré para que nos oigan hasta los sordos.

---Pero si no me has preguntado nada.

---Sí te he preguntado. Pero tú haces que no oyes, y no quieres
responderme.

---No nos entendemos---repuso llena de confusiones, y mortificada por la
observación tenaz de doña María.---¿Vendrás todas las noches? Aquí es
preciso mucha cautela. Para respirar necesito pedir la venia a la
señora. Ten prudencia, Gabriel; también D. Diego nos mira. Haz de modo
que doña María y los murciélagos crean que estamos a hablando de
religión, o de los cuadros de la pared o de esa gran grieta que hay en
el techo. Aquí es preciso hacerlo todo así. No te expreses con
vehemencia. Ponte risueño y mira a las paredes diciendo: «¡Qué bonitas
láminas! Allí están Dafne y Apolo.»

---Pero ¿es preciso ser cómico para entrar aquí?

---Sí; es preciso estar siempre sobre las tablas, Gabriel; fingiendo y
enredando. Esto es muy triste.

---Pues lord Gray no disimula.

---¿Eres amigo de lord Gray?

---Sí, y me lo ha contado todo.

---Te lo ha dicho\ldots---exclamó confusa.---¡Qué hombre tan indiscreto!
Y yo le había encargado la mayor prudencia\ldots{} Por Dios, Gabriel, no
pronuncies una palabra, ni un gesto que puedan dar a conocer lo que te
ha contado lord Gray. ¡Qué indiscreción! Hazme el favor de olvidar lo
que te ha dicho. ¿Él te ha traído aquí?

---No; he venido con D. Diego. He querido saber por ti misma que ya no
me amas.

---¿Qué estás diciendo?

---Lo que oyes. Ya lo sabía; pero a mí me hacía falta oírlo de tus
propios labios.

---Pues no lo oirás.

---Ya lo he oído.

---Por Dios, disimula. Ahora, Gabriel, alza la vista y di: «¡Qué
terrible grieta se ha abierto en el techo!» ¿Con que no te quiero yo?
¿Sabes que no lo había advertido? Y en tanto tiempo ¿qué has hecho tú?
¿Has estado en el sitio de Zaragoza? Aquello sería un paraíso; no estaba
allí doña María.

---No he vivido más que para ti; y si alguna vez he hecho un esfuerzo
para subir un peldaño en la escala del mundo, hícelo sólo con el deseo
de llegar, si no a valer tanto como tú, al menos a ponerme en condición
tal, que no se rieran de mí cuando te miraba.

---Mentiroso, tú también has aprendido a disimular. Ni una sola vez te
has acordado de mí en tanto tiempo\ldots{} Pero no te acerques tanto.
Cuidado, no me tomes la mano. Parece que tienes fuego dentro de los
guantes. Doña María nos observa.

---Yo no sé disimular como tú. Te he querido con toda mi alma, Inesilla,
y con veinte almas más, porque una sola no basta para quererte como te
quiero\ldots{} Dime con la mano puesta sobre el corazón si lo mereces
tú; dímelo.

---Pues no lo he de merecer---me contestó sonriendo.---Merezco eso y
mucho más, porque me lo tengo ganado y pagado con interés y
anticipación. ¿Pero no ve usted, Sr.~D. Gabriel---añadió alzando la
voz,---qué hendidura tan grande es esa que hay en el techo?

---Inés, si es verdad lo que me dices, dímelo otra vez, y alza la voz.
Quiero que lo oigan doña María, D. Diego y los murciélagos.

---Calla; por haber estado tanto tiempo sin verme, merecerías\ldots{} a
ver, ¿que merecerías?

---Bastante castigado estoy por los celos, por unos terribles celos que
me han estado mordiendo el corazón, y me lo muerden todavía.

---¡Celos! ¿De quién?

---¿Me lo preguntas tú? De lord Gray.

---Tú has perdido el juicio---dijo con precipitación y atropellándose en
sus labios frases rápidas y confusas.---¡Él lo dice!\ldots{} Tal
vez\ldots{} Ese hombre me causará grandes pesadumbres.

---¿Tú le amas?

---Por Dios, habla bajo, disimula.

---Yo no puedo disimular. Yo no estoy, como tú, educado en esta escuela
de los fingimientos. Yo no puedo decir más que la verdad.

---¿Has dicho que yo amo a lord Gray? Jamás he pensado en tal cosa.

---¡Oh! ¿Qué haré para creerlo? Bajo la autoridad de doña María has
aprendido de tal modo a disfrazar los pensamientos, que hasta se ocultan
a mis ojos, tan acostumbrados, no sólo a leerlos, sino a adivinarlos. Ha
desaparecido aquella claridad que te rodeaba, y que te hacía doblemente
hermosa ante mí. Ya no hablas aquella palabra divina que ningún mortal,
y menos yo, podía poner en duda. Ahora, Inés, me asegurarás una cosa, me
la jurarás, y\ldots{} ¿qué quieres tú?, no lo creeré. ¡Maldita sea mil
veces doña María que te ha enseñado a disimular!

---Si te alteras de ese modo, no podremos hablar---repuso con agitación
en voz baja; y luego, en voz alta, añadió:---Sr.~D. Gabriel, estas
estampas de Dafne y Apolo, de Júpiter y Europa son indecorosas, y hemos
encargado a Sevilla una colección de santos para sustituirlas. Pero ¿qué
has dicho de lord Gray?--- prosiguió quedamente.---¿Que le amo yo? ¡Oh,
ese hombre me traerá alguna desgracia! No repara en nada. ¡Qué loca he
sido! ¡Me encuentro comprometida! Gabriel, te suplico que olvides lo que
te haya dicho lord Gray. Olvídalo, y a nadie, ni a tu confesor, hables
de eso. Tú reconocerás que está lleno de seducciones y que no es extraño
que su fantasía acalore y agite el alma de una\ldots{} Pero no hables de
eso. Calla, por favor.

---¿De veras no le amas?

---No.

---¿Ama a alguna otra de esta casa?

---No sé\ldots{} calla\ldots{} no, a nadie de esta casa---respondió
turbada.---Pero ¿no merezco que me creas?

---No, casi no.

---¿Me has conocido mentirosa?

---No sé qué tiene esta casa y todos los que la viven. Me parece que en
esta morada del disimulo y la mentira, ninguna cosa es como aparece.
Mienten los que aquí moran; mienten los que aquí viven, y hasta yo he
necesitado mentir para que me admitieran. Esta atmósfera está formada de
falsedad y engaño. Los corazones, oprimidos por una autoridad
insoportable, necesitan desfigurarse para que se les permita vivir. Esta
casa, esta familia, a quien preside desde su sillón doña María, como el
genio de la tristeza, no es para mí. Me ahogo, y deseo huir de este
sitio. Veo aquí mil misterios, y sobre todos mis sentimientos domina
uno, que es el más antipático y desagradable de todos: la desconfianza.
El corazón se me oprime cuando considero que tú, Inesilla, tú me dices
una cosa, me la juras y yo no la puedo creer.

---Ten calma. Doña María no nos quita los ojos. D. Diego tampoco. Yo me
muero de pena\ldots{} Pero, por Dios, Sr.~D. Gabriel---añadió en voz
alta.---Un hombre que va a tomar el hábito cuando acabe la guerra, no
debe entusiasmarse tanto al hablar de una batalla.

Doña María, desde su trono, me interpeló pomposísimamente de esta
manera:

---Pero, Sr.~D. Gabriel, que oigamos todos esas maravillas que está
usted contando con tanta vehemencia, con tanto ardor.

---Me contaba---dijo Inés con una naturalidad que me asombró,---que en
cierta ocasión, estando él en una casa del arrabal de Zaragoza, los
franceses abrieron una mina, pusieron no sé cuántos barriles de pólvora,
¿no fue así?, y luego pegaron fuego.

---¿Y luego, Sr.~D. Gabriel?

---Y luego volamos todos hasta el quinto cielo---repuse.---Siento que
usted no hubiera estado allí\ldots{} pues\ldots{} para que lo hubiera
visto.

---Gracias.

Los vencejos me tomaron por su cuenta para que les explicase cómo fue
aquello de mis vuelos y cabriolas por el aire, y en tanto llegose Inés
junto al sillón de doña María, llamado por esta; y yo con disimulo
(también aprendía) presté atención a lo que dijeron.

---Ha sido demasiado larga tu conversación con el militarcito---le dijo
con desabrimiento la señora.---¡Veinte minutos! ¡Has estado en coloquio
con él veinte minutos!

---Señora madre---repuso Inés,---si se empeñó en contarme sus
hazañas\ldots{} Yo buscaba ocasión de poner punto; pero él, dale que
dale. Me refirió siete sitios, cinco batallas y no sé cuántas
escaramuzas.

---¡Cómo finge, cómo miente, cómo engaña!---exclamé para mí ciego de
rabia.---¡La ahogaría!

Lord Gray se juntó después con Inés y hablaron largamente. Mi rabia,
motivada por una duda cruel, era tanta, que apenas podía disimularla,
hablando pestes de las Cortes ante doña María, Ostolaza y Valiente.

Avanzaba la hora y doña María indicó con majestuosa gravedad el fin de
la tertulia. Despedime de Inés, que a hurtadillas me dijo:

---Cuidado con lo que te he encargado.

Y luego tardó en despedirse de lord Gray más de diez minutos. Por mi
parte anhelaba salir para no volver más a aquella casa, y saludando a la
condesa, echeme fuera, juntándose conmigo en la escalera lord Gray, que
salió un poco después.

---Amigo---le dije cuando estábamos en la calle,---en todas partes es
usted el favorecido de las damas.

No se dignó contestarme. Iba con la cabeza inclinada, fruncido el ceño y
mudo como una estatua. Repetidas veces me esforcé por hacerle hablar;
pero sus labios no articularon una sílaba, y sólo en la calle Ancha, al
despedirse de mí, me dijo sombríamente:

---El amigo que sorprende un secreto mío y usa de él sin mi licencia, no
es mi amigo. ¿Usted me conoce?

---Un poco.

---Pues suelo reñir con los amigos.

---Antes de reñir nosotros, ¿quiere usted acabar de perfeccionarme en la
esgrima?

---Con mucho gusto. Adiós.

---Adiós.

\hypertarget{xii}{%
\chapter{XII}\label{xii}}

Pasaron días, muchos días. Yo tan pronto deseaba volver a casa de
Rumblar, como hacía intención de no poner más los pies en aquella casa,
porque me repugnaban los artificios que hacían de las tertulias una
completa representación de teatro. Durante algún tiempo no vi a lord
Gray ni en la Isla ni en Cádiz, y cuando pregunté por él en su casa, el
criado me negó la entrada, diciéndome que su amo no quería recibir a
nadie.

Ocurrió esto el día de la bomba. ¿Saben ustedes lo que quiero decir?
Pues me refiero a un día memorable porque en él cayó sobre Cádiz y junto
a la torre de Tavira la primera bomba que arrojaron contra la plaza los
franceses. Ha de saberse que aquel proyectil, como los que le siguieron
en el mismo mes tuvo la singular gracia de no reventar; así es que lo
que venía a producir dolor; llanto y muertes, produjo risas y burlas.
Los muchachos sacaron de la bomba el plomo que contenía y se lo
repartían llevándolo a todos lados de la ciudad. Entonces usaban las
mujeres un peinado en forma de saca-corchos, cuyas ensortijadas guedejas
se sostenían con plomo, y de esta moda y de las bombas francesas que
proveían a las muchachas de un artículo de tocador, nació el famosísimo
cantar:

\small
\newlength\mlenf
\settowidth\mlenf{\quad Con las bombas que tiran}
\begin{center}
\parbox{\mlenf}{\quad Con las bombas que tiran                      \\
                \null \qquad \quad los fanfarrones,                 \\
                hacen las gaditanas                                 \\
                \null \qquad \quad tirabuzones.}                    \\
\end{center}
\normalsize

Pues como decía, el día de la bomba, después de tocar inútilmente a la
puerta del noble inglés, llevome el destino segunda vez a casa de la
señora doña María, disponiéndose las cosas de modo que cuando me
encaminaba a casa de dona Flora, tropezase con el señor D. Diego, el
cual me habló así:

---¿Vienes de casa de lord Gray? Dicen que está con la morriña. Nadie le
ve por ninguna parte. Por fin, he conseguido de mi madre que no le
reciba más en casa.

---¿Por qué?

---Porque es muy aficionado a las muchachas, y no me gusta verle hablar
con mi novia. Mamá no quería; pero me planté, chico. «O lord Gray o
yo»---dije,---y no hubo más remedio.

---Según eso, le han puesto en la puerta de la calle.

---Con cortesía y disimulo. Mi mamá ha dicho que hallándose un poco
enferma, suspende por ahora las tertulias.

---¿Y no salen?

---A misa van las cuatro los domingos muy temprano. Pero puedes ir a
casa cuando gustes. Mamá te aprecia y siempre está preguntando por ti.
Ahora precisamente, te ruego vengas conmigo para servirme de testigo.

---¿De testigo?

---Sí. Mi mamá quiere castigarme porque le han dicho que me vieron ayer
en un café. Es verdad que estaba, pero yo lo he negado, y para dar más
fuerza a mis argumentos he dicho: «Pregúntele usted al Sr.~D. Gabriel, y
como no diga que estuvimos juntos viendo sacar agua de la noria\ldots»

---Pues vamos allá.

Entramos, pues, y en la reja del patio, el criado nos dijo que la señora
doña María había salido.

---¡Viva la libertad!---exclamó D. Diego haciendo un par de
cabriolas.---Gabriel, estamos solos. Hermanillas, alegrémonos y
regocijémonos.

La chillona algazara que desde los aposentos vino a mis oídos, indicome
que las hembras estaban libres también de la ominosa esclavitud. Cuando
entramos en la estancia de D. Diego, al punto se nos presentó D. Paco,
aturdido, sofocado, balbuciente, con unas disciplinas en la mano, el
vestido menos puesto en orden que de ordinario, y ostentando algunas
desgreñaduras en lo alto de su peluquín.

---Señorito D. Diego---exclamó con furia semejante a la de esos
perrillos que ladran mucho sin que jamás el transeúnte se detenga a
mirarlos,---la señora mandó que no saliese usted de casa. Se lo diré
cuando venga.

El condesito tomó un palo que frontero a la cama y en lugar medio oculto
tenía, y esgrimiéndolo de un modo alarmante por las costillas del ayo,
gritó:

---Canalla, pedantón\ldots{} Si dices una palabra\ldots{} no te dejaré
un hueso en su lugar.

---Esto no puede tolerarse---dijo D. Paco, no ya enfurecido sino
lloroso.---¡Dios eterno, y tú, Virgen Santísima del Carmen, tened
compasión de mí! Este niño y sus hermanas van a quitarme los pocos días
que me restan de vida. Si les permito hacer su gusto, la señora me riñe,
y más quisiera ver al sol apagado que a la señora colérica. Si quiero
sujetarlos, palos, rasguños, arañazos, tijeretazos y otros mil martirios
espantosos\ldots{} Pues sí, señor D. Dieguito: se lo diré a la señora,
yo no puedo aguantar más\ldots{} ¡Pues no digo nada de lo de las
saliditas por las noches! Yo no puedo acallar la voz de mi conciencia
que me dice: ¡Malvado!, ¡servidor desleal!, ¡traidor!\ldots{} No; se lo
diré a la señora, se lo diré al ama, y entre tanto, orden, silencio,
obediencia, todo el mundo a su sitio.

D. Diego, ciego de enojo, enarboló el palo, y a compás con los
movimientos de su brazo que apuntaban impíamente a las costillas del
pobre ayo, iba diciendo:

---Orden, silencio, obediencia.

Tuve que imponerme para que no acabara con el desdichado perceptor, que
aun vapuleado de aquel modo, tenía la prudencia de no gritar, porque no
se enterase la vecindad del escándalo, y con voz sofocada decía
llorando:

---¡Que me mata este caribe! ¡Favor, señor D. Gabriel, favor!

Huyó D. Paco por el pasillo adelante buscando refugio, y siguiendo tras
él, dimos los tres en una gran pieza, desde la cual se pasaba a otra con
espaciosas rejas a la calle, donde vimos el espectáculo de la más
horrenda anarquía que pueden ofrecer en el interior de una honesta casa
las demasías de la libertad. Asunción, Presentación, Inés, las tres
estaban allí, libres, sueltas, en posesión completa de sus gracias,
donaires, iniciativa y travesura. Pero antes de deciros lo que hacían
aquellos pajaritos aprisionados a quienes se permitía por un momento dar
vueltas holgadamente por la jaula, voy a indicaros cómo era esta.

Varias cestas de labores y algunos bastidores de bordados indicaban que
allí tenía la señora condesa el taller de educación y trabajo de sus
niñas. Una pequeña pero anchísima silla, de fondo hundido por el peso
constante de corpulenta humanidad, denotaba el lugar de la presidencia.
También había una mesilla con libros, al parecer devotos, y en las
paredes no cabían ya más estampas y láminas bordadas, entre las cuales
el mayor número era una variada serie de perritos con el rabo tieso y
los ojos de cuentas negras.

Un pequeño altar ostentaba mil figuras de bulto y realce, alternando con
estampas que sin duda habían pertenecido a libros, y en la delantera
algunos pares de candelabros de plata antigua, sostenían velas de picada
y filigranada cera, adornadas con papelitos, festones y otros primores
de tijera. Pomposos ramos de flores de trapo, que a cien mil leguas
declaraban haber sido hechos por manos de monjas, completaban el ajuar
del altarejo, juntamente con algunos pequeñísimos objetos de plomo,
representando sagrados adminículos, tales como cálices y custodias,
lámparas y misales. Estos juguetes los hacían entonces los veloneros
para los niños buenos y que no lloraban.

Vi asimismo objetos de un orden enteramente distinto, es decir, trajes
hermosísimos de mujer, arrojados en desorden por el suelo, y también
escofietas, moños, lazos, abanicos, quirotecas, zapatillas de raso y
luengos encajes de aquellos finísimos y hereditarios, que eran, como los
diamantes, orgullo y riqueza de las familias. Los bordados, las cestas
de costura, rellenas de fastidiosas telas blancas de indiana y cotonía,
pertenecían a Presentación; los libros, el altar con todo lo que en él
había de místico e infantil, eran de Asunción; y los lujosos trajes y
adornos eran de Inés, que los había bajado para que los viesen sus
primas.

Estaban las tres vestidas según lo que entonces el vulgo, no menos
galicista que ahora, llamaba un \emph{savillé}. Con semejante traje, que
era, por exigirlo la moda, la menos cantidad posible de traje, y lo
absolutamente necesario para que las lindas personas no anduvieran
desnudas, ni la madre más tolerante y descuidada habría permitido que se
presentasen delante de un hombre, aunque fuese pariente cercano. Estaban
las tres, como digo, graciosísimas y sin comparación más guapas que en
las tertulias. La libertad permitiéndoles una alegre y bulliciosa
agitación, había impreso en sus mejillas frescos y risueños colores, y
las lenguas charlatanas de las dos hermanitas llenaban con dulce y
picotera música el ámbito de la estancia. La voz de Inés apenas se oía.

Os diré lo que hacían y esto es reservado, reservadísimo, pues si doña
María supiese que ojos humanos habían visto a sus niñas en tales arreos,
y que orejas de varón habían oído cantar seguidillas a una de ellas,
reventara de pesadumbre, o se sepultaría para siempre, antes avergonzada
que muerta en el sarcófago de sus mayores. Pero seamos indiscretos y
contemos lo que vimos, ocultos en la estancia inmediata y sin ser vistos
por ellas. Inés, en quien primeramente se fijaron mis ojos desde la
puerta, estaba en la reja, como en acecho, mirando ora a la calle, ora
adentro, sin duda para dar la voz de alarma en cuanto el pomposo perfil
y los pomposos y temidos espejuelos de doña María volviesen la esquina
de la calle Ancha. Le oí decir claramente:

---No seáis locas\ldots{} que va a venir.

Presentación, la más pequeña de las dos hermanas, estaba en medio de la
pieza. ¿Creerán ustedes que rezando, cosiendo u ocupada en algún otro
grave menester? Nada de eso, pues no estaba sino bailando, sí, señores,
bailando. ¡Y qué zorongo, qué zapateado tan hechicero! Quedeme absorto
al ver cómo aquella criatura había aprendido a mover caderas, piernas y
brazos con tanta sal y arte tan divino cual las más graciosas majas de
Triana. Agitada por la danza, chasqueando los dedos para imitar el ruido
de las castañuelas, su vocecita sonora y dulce decía con lánguida y
soñolienta música:

\small
\newlength\mleng
\settowidth\mleng{\quad que he cogido de mi huerto:}
\begin{center}
\parbox{\mleng}{\quad Toma, niña, esta naranja                      \\
                Que he cogido de mi huerto:                         \\
                No la partas con cuchillo,                          \\
                Que está mi corazón dentro.}                        \\
\end{center}
\normalsize

Asunción, que era la mayor, de una hermosura menos picante y graciosa
que su hermana, pero más acabada, más interesante, más seria, digámoslo
así, en una palabra, mucho más hermosa, se había puesto algunas de las
joyas y preseas de Inés. Cogió una gran rosa de papel de las que
adornaban el altar, y púsosela orgullosamente en el moño; tomó después
tres varas de aquellos encajes finísimos de Brujas, de tan sutil
urdimbre que parecen hechos por moscas o arañas, pálidos ya y
amarilleados por el tiempo, y agitándolos en las manos, los echó hacia
arriba, dejándolos caer sobre su cabeza y hombros, con tanta, con
tantísima gracia, señores, cual si toda su vida hubiese estado midiendo
en las tardes de primavera las baldosas de la calle Ancha, plaza de San
Antonio y alameda del Carmen.

Yo estaba asombrado contemplando tales transformaciones y me sorprendía
su extraordinaria belleza de la muchacha, cuando la vi realzada con los
atractivos que el arte presta tan hábilmente a la hermosura. ¡Y qué bien
sabía ella aplicarlos a su persona! ¡Qué singular talento el suyo para
poner cada objeto en el sitio donde debía estar, y donde las leyes más
rigurosas de la estética querían y mandaban que estuviese!

Después de rodear su cabeza con las blondas, colgose de las orejitas los
más hermosos pendientes que creo han salido de manos de artífice
platero. Luego estuvo mirándose un rato en el vidrio que cubría cierta
estampa del Purgatorio, llena toda de ánimas, diablos, llamas,
culebrones, sapos, cocodrilos, ruedas, sartenes, peroles, etc\ldots, y
contempló allí su imagen confusa, por no haber en la estancia espejo, ni
vidrio azogado que hiciese sus veces. Después volvió la cabeza para
verse la caída de faldas por detrás, tomó un abanico, dio el meneo a las
varillas, que chillaron desarrollando un vasto paisaje poblado de
amorcitos, y echándose aire con él, comenzó a pasear por la habitación,
riéndose de sí misma y de la risa que a las otras dos causaba.

Viendo tal profanación, escándalo y desacato, penetró el insigne D. Paco
en la pieza, y exclamó:

---¿Qué alboroto es este? Asuncioncita, Presentacioncita, todo se lo
contaré a mamá cuando venga, todo, todito.

Presentación cesó de cantar, y tomando al preceptor por un brazo, le
dijo:

---Sr.~D. Paquito mío, si no le dices nada a mamá, te doy un beso.

Y en el acto se lo dio en sus secas y arrugadas mejillas.

---A mí no se me seduce con besitos, niñas---repuso el viejo vacilando
entre el rigor y la tolerancia.---Cada una a su puesto, a leer, a coser.
Asuncioncita de todos los demonios, ¿qué descaro es ese?

---Calle usted, so bruto---dijo Asunción con muchísima sal.

---Si es un animal---añadió Presentación dándole un sopapo con su suave
manecita.

---Más respeto a mis canas, niñas---exclamó afligido el anciano.---Si no
fuera porque las he visto nacer, porque las he criado a mis pechos,
porque las he cantado el ro-ro\ldots{}

Presentación haciendo gestos de delicada urbanidad, remedando a una
persona que durante el paseo encuentra en la calle a un conocido, parose
ante D. Paco, hizo una graciosa reverencia y le dijo:

---¡Oh! Sr.~D. Protocolo, ¿usted por aquí? ¿Cómo está la señora doña
Circunspecta? ¿Va usted al baile del barón de Simiringande? ¿Qué dice
hoy la \emph{Gaceta} de Pliquisburgo?\ldots{}

---Eh\ldots{} eh\ldots---exclamó D. Paco, queriendo contener la risa que
le embobaba.---Miren la mocosa cómo habla, haciéndose la señora mayor.
Buena pieza tenemos en casa. ¡Qué escándalo, qué profanidad! ¿De dónde
habrá sacado esta niña tales picardías?

Y luego insistiendo ella en llevar adelante el chistoso papel que estaba
desempeñando, llegose a Inés, que también se moría de risa, y le dijo:

---¡Ola, madama! ¿Cómo la porta bu\ldots? ¿Ha visto bu a la condesa?
¡Qué magnífico ha estado el concierto y la ópera de Mitrídates! ¡Oh!,
madama\ldots{} andiamo a tocare il forte piano\ldots{} Aquí viene il
maestro siñor D. Paquitini\ldots{} tan, taralá, tan tin, tan.

Y se puso a bailar un minueto.

---Vaya---exclamó D. Paco, echándosela de benévolo, pero afectando mucha
seriedad,---les perdono lo que ha pasado si se acaba este jaleo, y va
cada una a su puesto. La señora viene.

Inés continuaba en la reja atisbando afuera, y también a ratos decía:

---¡Que va a llegar!

Presentación volvió a cantar, y luego dijo:

---Paquito de mi alma, si bailas conmigo te doy otro beso.

Y sin esperar respuesta del anciano, le tomó por los brazos, haciéndole
dar rápidas vueltas.

---Que me atonta, que me mata esta condenada---exclamaba el maestro,
describiendo curvas sin poderse defender, ni soltar.

---¡Ay, Paquito de mi alma y de mi vida, cuánto te quiero!---decía
Presentación.

El preceptor, abandonado de los ágiles brazos de su pareja, cayó al
suelo, pidiendo al cielo justicia; la muchacha le enredó una flor entre
las blancas guedejas de su peluca de ala de pichón, y dijo así:

---Toma, amor mío, esta flor en memoria de lo que te quiero.

Quiso levantarse, y empujado por Asunción, cayó al suelo. Quiso tirar de
él Presentación y quedose con un pedazo de solapa en la mano. Levantose
al fin, y persiguiéndole las dos con risas y festejo, trató una de ellas
de darle un latigazo con una varita de sacudir telas; mas lo hizo con
tan mala suerte que dando un cachiporrazo al altar, toda la máquina de
santos, velas y juguetes se vino al suelo con estrépito. Mientras acudía
a remediar el desperfecto, D. Paco estaba en tierra de rodillas, con los
brazos en cruz y la mirada fija en el techo y con voz compungida y
entrecortada, mientras gruesos lagrimones lustraban sus mejillas, decía:

---¡Señor Omnipotente y Misericordioso: que estas agonías sean en
descargo de mis pecados! Mucho padeciste en la cruz; ¿pero y esto,
Señor, esto no es cruz, estos no son clavos?, ¿estas no son espinas?,
¿estos no son bofetones y hiel y vinagre? Castigo es este del gran
pecado que cometí ocultando a mi señora las travesuras de estas niñas, y
las mil picardías que han aprendido sin que nadie se las enseñase; pero
por la lanzada que te dieron, Señor, juro que seré leal y fiel con mi
querida ama, y que no he de ocultarle ni tanto así de lo que pasa.

D. Diego y yo, que habíamos permanecido observando aquel espectáculo sin
ser vistos, quisimos entrar; pero vimos que Inés se apartó vivamente de
la reja, y en el mismo instante pasó por la calle una figura, una
sombra, en quien reconocimos a lord Gray. Apenas habíamos tenido tiempo
de reconocerle, cuando un objeto, entrando por la reja, vino a caer en
medio de la sala. Al punto se abalanzó hacia el pequeño bulto D. Paco, y
observándolo y recogiéndolo, dijo:

---¿Una cartita, eh? La ha arrojado un hombre.

Inés, que se acercó de nuevo a la reja, exclamó con terror:

---¡Doña María, doña María viene ya!

\hypertarget{xiii}{%
\chapter{XIII}\label{xiii}}

Se quedaron muertas, petrificadas; pero con presteza extraordinaria las
tres empezaron a ordenar los objetos, para que cada cosa estuviese en su
sitio. Arreglaron el altar atropelladamente; despojose la una de los
atavíos que se había puesto; compuso la otra su vestido en desorden;
pero por más prisa que se daban, tales eran la confusión y desconcierto
producidos allí por la anarquía, que no había medio de volverlo todo a
su primitivo estado. D. Diego me dijo, al ver que las muchachas iban a
ser sorprendidas antes de poder borrar las huellas de su rebelión:

---Amigo, huyamos.

---¿A dónde?

---A la Patagonia, a las Antípodas. ¿Tú no adivinas lo que va a pasar
aquí?

---Quedémonos, amigo, y tal vez hagamos una buena obra defendiendo a
estas infelices, si el preceptor las delata.

---¿Viste que pasó un hombre y arrojó dentro un billete?

---Era lord Gray. Veamos en qué para esto.

---Pero mi madre viene; y si te ve aquí en acecho\ldots{}

Ni esta consideración me hizo apartar de la estancia que nos servía de
observatorio; pero afortunadamente doña María no entró por allí, y
pasando primero a su alcoba, penetró por esta a la funesta habitación
donde ocurriera el sainete que iba a terminar en tragedia. Nosotros nos
pusimos en disposición de poder oírlo todo sin ser vistos, aunque
también sin ver nada. Sepulcral silencio reinó por breve tiempo en la
pieza, y al fin interrumpiole la condesa, diciendo con la mayor
severidad:

---¿Qué desorden es este? Inés, Asunción, Presentación\ldots{} ese altar
destrozado, esos vestidos por el suelo\ldots{} Niñas, ¿por qué estáis
tan sofocadas, por qué tenéis tan encendido el rostro?\ldots{}
Tembláis\ldots{} Vamos a ver; Sr.~D. Paco, ¿qué ha pasado aquí?\ldots{}
¿Pero qué veo? Señor D. Paco, señor preceptor, ¿por qué tiene usted
destrozada la ropa?\ldots{} ¡Pues y ese gran cardenal en el
carrillo\ldots? ¿Ha estado usted quitando telarañas con la peluca?

---Se\ldots{} se\ldots{} señora doña María de mi alma---dijo el ayo con
voz trémula y cierto hipo producido por su gran zozobra y la lucha que
diversos sentimientos sostenían sin duda entonces en su pobre alma,---yo
no puedo callar más\ldots{} Mi conciencia no me lo permite. Yo\ldots{}
hace cuarenta años que co\ldots{} co\ldots{} como el pan de esta
casa\ldots{} y no puedo\ldots{}

No pudiendo seguir, prorrumpió en llanto copiosísimo.

---Pero ¿a qué vienen esos lloros?\ldots{} ¿Qué han hecho las niñas?

---Señora---dijo al fin D. Paco entre sollozos, hipidos y babeos;---me
han pegado, me han arrastrado, me han\ldots{} Asuncioncita se puso a
imitar a la gente de los paseos. Presentacioncita bailó el zorongo, el
bran de Inglaterra y la zarabanda\ldots{} Luego pasó por la calle un
caballerito, miró adentro y les arrojó este billete.

Hubo un momento de silencio, de esos silencios angustiosos como el que
precede al cañonazo, después que se ha visto la mecha próxima al cebo.
Durante aquel intervalo de mudo terror, que desde la escena donde tal
drama pasaba se comunicó a nosotros, haciéndonos temblar como quien
aguarda un terremoto, se sintieron los tenues chasquidos de un papel que
se desdobla, y luego una exclamación de sorpresa, asombro o no sé si de
fiereza inaudita, que salió del tempestuoso seno de doña María.

---Esta letra es de lord Gray\ldots---exclamó.---¡Qué desvergonzado
atrevimiento! ¿A quién de vosotras se dirige la carta? Dice: «Idolatrado
amor mío: si tus promesas no son vanas\ldots» ¡Pero una persona como yo
no puede leer tales indecencias!\ldots{} ¿A quién de vosotras dirige
lord Gray esta esquela?

Continuó el silencio, uno de esos silencios que parecen anunciar el
desplome del mundo.

---Presentación, ¿es a ti? Asunción, ¿es a ti? Inés, ¿es a ti? Responded
al momento. ¡Señor misericordioso! ¡Si alguna de mis hijas, si alguien
nacido de mis entrañas ha dado motivo para que un hombre le dirija estas
palabras, prefiero que muera ahora mismo, y yo detrás, antes que tolerar
tal deshonra!

La imprecación retumbó en la sala como una voz de los pasados siglos que
clamaba en defensa de cien generaciones ultrajadas. Oyéronse luego
llantos comprimidos y el resoplido de D. Paco, que así desfogaba los
ardores de su corazón, inflamado ya por nobles impulsos de generosidad.

---Señora---dijo moqueando y babeando,---perdone usía a las niñas. Eso
no habrá sido nada. Tal vez un tuno que pasó por la calle. Ellas se han
estado muy calladitas.

---Se me figura---dijo doña María sin perder la dignidad en su
cólera,---que no tendré que hacer grandes averiguaciones para saber
quién ha motivado esta amorosa epístola. Tú, Inés, tú has sido. Hace
tiempo que sospechaba esto\ldots{}

Nuevo silencio.

---Responde---prosiguió doña María.---Yo tengo derecho a saber en qué
emplea su tiempo la que va a casarse con mi hijo.

Entonces oí la voz de Inés, que claramente y no muy turbada respondía:

---Sí, señora doña María. Lord Gray escribió para mí. Perdóneme usted.

---¡De modo que tú!\ldots{}

---Yo no tengo culpa\ldots{} Lord Gray\ldots{}

---Te ha trastornado el juicio---dijo doña María.---¡Bonita y ejemplar
conducta de una niña de tu condición, que representa una de las más
principales casas de España! ¡Inés, vuelve en ti, por Dios, repara quién
eres! ¿Es posible que una joven destinada?\ldots{} Yo he observado que
es tu natural de suyo profano a las mundanidades. Ya supieron lo que se
hacían destinándote a ser casada y a ocupar alto puesto en la corte, que
si por arte del demonio hubiérante consagrado al claustro o a un
decoroso celibato\ldots{} ¡pobre criatura!, tiemblo de pensarlo.

Mi ansiedad y zozobra no me permitieron reflexionar sobre las peregrinas
ideas de doña María.

---No has sido tú educada por mí---prosiguió esta,---que de haberlo
sido\ldots{} otra sería tu conducta\ldots{}

---Señora madre---dijo Asunción llorando.---Inés no volverá a faltar
más.

---Calla tú, necia. Después os ajustaré a vosotras dos las cuentas, pues
dijo D. Paco que habíais bailado y cantado.

---No, señora, no ha habido nada de baile ni de canto: fue broma
mía---exclamó muy sofocado el pobre preceptor, cuyo espíritu se afligía
con los crueles alardes de justicia de su señora.

---¿Y para qué has bajado estas ropas?---preguntó la condesa a Inés.

---Para que ellas las vieran. Las subiré, señora, y no las volveré a
bajar más--- repuso Inés con humildad.

---¡Qué fundamento de niña! ¿No conoces que si a ti te cuadran estos
trapos y adornos, a ellas ni aun debe permitírseles el mirarlos? Tu
conducta no puede ser más contraria al decoro.

---Señora doña María---dijo D. Paco,---permítame usía que la diga que la
señora doña Inesita en lo íntimo de su corazón deplora el disgusto que
la ha dado. ¿No es verdad, señora doña Inesita? Vaya, señora doña María,
perdón al canto, y todo se acabó.

---No se meta usted en lo que no le importa, Sr.~D. Paco---dijo la
condesa.---Y tú, Inés, ten entendido que serás perdonada, si las cosas
no siguen adelante. Y no digo más sobre el particular. Ya saben ustedes
que soy benévola hasta la exageración, tolerante hasta la debilidad.
Ciérrense esas rejas al punto, y vamos a trabajar y a rezar\ldots{}
Inés, te lo repito, respira tranquilamente. Con tal que no vuelva a
repetirse\ldots{}

Oyéronse voces de las muchachas, que si no de alegría y completa
bonanza, indicaban que el temporal iba pasando.

D. Diego me dijo:

---Vámonos, no sea que mi madre quiera salir por aquí y nos sorprenda.

Nos apartamos de allí.

---¿Qué te parece lo que hemos oído?

---Una infamia, una alevosía, un crimen sin ejemplo---exclamé no
pudiendo contener la cólera que me dominaba.

---¿Qué te parece la Inesita?\ldots{} Buena pieza en verdad\ldots{}

---Ese inglés de los demonios, ese monstruo que nos ha enviado aquí la
Gran Bretaña es el ser más odioso, más abominable que existe en la
tierra. Por mi parte, digo que le aborrezco, que le abomino; que sin
piedad le mataría, que me bebería su sangre\ldots{} Adiós, me voy.

---¿Te vas?

---Sí: no quiero estar más en esta casa.

---Pero hombre, tú estás tonto. Si te he traído aquí para que me
ampares. Tú no sabes que ahora mi señora mamá, después que ponga fin a
la justiciada de allá, ha de venir a emprenderla conmigo por la
escapatoria de ayer tarde. ¿Olvidas, hombre ligero y frívolo, que has de
atestiguar que me viste ayer ocupado en dar vueltas a la noria?

---No quiero farsas, ni falsos testimonios, ni tengo para qué ver a doña
María\ldots{} Adiós.

---Hombre cruel, detente. Mi madre sale.

En efecto, en el corredor atrapome la señora condesa, la cual después de
mostrarse sorprendida y no muy agradablemente con mi presencia, me
saludó, obligándome a pasar a la sala.

---¿Estabas aquí?---preguntó a su hijo.

---Sí, señora: Gabriel y yo estábamos en mi cuarto leyendo unos libros
de aritmética, y él me enseñaba a encontrar la quinta parte por un medio
nuevo; y como ayer cuando estuvimos viendo dar vueltas a la noria, yo
aposté a que no podía ser tal cosa, vino hoy a demostrármelo.

---¿Con que estuvieron ustedes ayer tarde en la noria?

---Sí, señora; dando vueltas a la noria\ldots{} quiero decir, viendo.

---Es un entretenimiento inofensivo\ldots{}

---Sí, señora\ldots{} e instructivo.

---Propio de jóvenes de cabeza sentada---dijo doña María.---Sin embargo,
he oído que a la noria va mucha gente de mal vivir.

---No señora, de ninguna manera. Canónigos, militares de coronel para
arriba, señoras mayores, frailes\ldots{}

---Mi hijo es algo distraído, y por eso temo\ldots{} Pronto será libre y
dueño de sus acciones, porque en los asuntos de un hombre casado, sobre
todo si está en cierta posición, no deben entrometerse las madres.

---Exactamente. ¿Y cuándo se casa D. Diego?

---Ya no hay día seguro---respondió doña María, con firmeza.

---Y en verdad, Sr.~D. Diego---dije yo volviéndome hacia mi amigo,---que
se lleva usted la más hermosa muchacha que hay en todo Cádiz.

---Lo que es eso\ldots---dijo la condesa con afectación,---mi hijo puede
estar satisfecho de la suerte que le ha cabido en su elección, mejor
dicho, en nuestra elección, pues nosotras lo hemos arreglado todo. Para
que nada falte a esa muchacha, tiene hasta aquellas sutiles cualidades
de ingenio y amabilidad que la harán uno de los más bellos adornos de la
corte, cuando la haya. Y no se diga que a una joven mayorazga, destinada
a casarse con otro mayorazgo, se la debe sujetar y comprimir para que ni
hable, ni trate con personas de mundo. Eso no; eso sería ridículo, y
nada hay más contrario a la alteza y sonoridad de ciertas familias que
verlas representadas en la corte por una damisela encogida, vergonzosa,
que se asusta de la gente y no sabe decir más que \emph{buenas tardes y
buenas noches}.

---Pues maldita la gracia que me hace---dijo D. Diego con
desabrimiento,---ver a mi novia muy amartelada con lord Gray en este
salón.

Doña María se puso encendida.

---Este joven---dije yo,---no eleva su entendimiento hasta los altos
principios de la educación castiza. ¿Pues acaso su mujer va a ser monja?
A las que van a ser monjas o solteras, bueno que se las enseñe a no
levantar los ojos del suelo; pero a las que van a casarse y a ser
grandes señoras\ldots{} Pero hombre, ¿está usted loco? Mi amigo es un
necio, un caviloso, señora. ¿Apostamos a que por estas y otras
imaginaciones ridículas va a dar en la flor de decir que no se casa?

---¡Cómo!---exclamó la dama.---Mi hijo no será capaz de tal simpleza.

---Sí, señora, sí seré capaz---dijo D. Diego sin poder contener el
ímpetu de sus celos.

---¡Diego, hijo mío!

---Sí, señora, lo que dice Gabriel es verdad, no quiero casarme, al
menos hasta ver\ldots{}

---No puede darse necedad mayor---dije.---Porque lord Gray haya
conseguido con su buena apostura, sus finos modales, su talento\ldots{}

---Mi hijo no me dará tan gran pesadumbre.

La condesa, por hallarse en presencia de un extraño, no soltó la ira que
a borbotones quería escapársele del pecho, al ver en su hijo la
obstinada genialidad, que amenazaba echar por tierra todos sus
proyectos; mas conociendo yo que aquel volcán necesitaba cumplido
desahogo por el cráter de la boca y quizás por el de las manos, juzgué
prudente retirarme.

---¿Se marcha usted?---me dijo.---Ya, una persona discreta no puede
soportar las bachillerías y antojos de este inconsiderado niño.

---Señora---repuse,---D. Diego es un niño obediente y hará lo que su
madre le mande. Beso a usted los pies.

Quiso D. Diego salir conmigo; pero la condesa le detuvo, diciendo con
enojo:

---Caballerito, tenemos que hablar.

Yo anhelaba respirar fuera de aquella casa.

\hypertarget{xiv}{%
\chapter{XIV}\label{xiv}}

Al encontrarme en la calle miré a las rejas y las vi cerradas.
Atormentado por el recuerdo de lo que había visto y oído, revolviendo en
mi cabeza pensamientos de venganza, proyectos de barbarie, y no sé qué
ideas impías y locas, dije para mí:

---Ya no me queda duda. Mataré a ese maldito inglés.

En las mil alternativas y vicisitudes de mi vida, bajé, subí, caí y
levanteme; creí tocar con mis manos fatigadas el fondo de aquel mar de
la borrascosa desventura, donde transcurrió mi niñez, y fuerzas
ignoradas me sacaron de nuevo a la superficie; luché y padecí, deseé la
muerte y amé la vida; grandes vaivenes y sacudidas experimenté; pero
cuando subía, y bajaba, y luchaba, y vivía, y moría, jamás dejé de
percibir aquella luz, encendida ante la desgracia, lejana estrella a
quien consideraba como expresión de lo divino y sobrenatural que hay en
la existencia. Pero ya la luz se había apagado, y volviendo los ojos en
derredor, yo no veía sino espantosas oscuridades. Lo que yo creía
perfecto ya no lo era; lo que yo juzgué mío, tampoco era mío, y pensando
en esto no cesaba de exclamar:

---Mataré a ese condenado lord Gray. Ahora comprendo la satisfacción de
matar a un hombre.

Turbado por los celos, mi corazón, que hasta entonces había como
florecido, despidiendo un sentimiento apacible y contemplativo cual el
de la religión, ardía ahora con apasionado centelleo, y lo que había
amado, por extraordinaria contradicción más digno de ser amado le
parecía. Sentía ansia de destrucción, y mi amor propio, mi orgullo
herido clamaban al cielo, haciendo a toda la creación solidaria de mi
agravio. Yo creía que el universo entero estaba ofendido, y que cielo y
tierra respiraban anhelo de venganza. Crucé varias calles, repitiendo:

---Mataré a ese inglés, le mataré.

Al volver una esquina creí distinguirle y apresuré el paso. Sí, era él.
Dios me lo ponía delante; le vi de espaldas y corrí; mas cuando estaba
junto a él y antes que me viera, pensé que no era prudente precipitar un
hecho que debía tener justificación completa. Procurando serenarme, dije
para mí:

---Tengo la seguridad de sorprenderle dentro de la casa. Entretanto,
esperemos.

Le toqué en el hombro, y él, al volverse, me miró impasible, sin mostrar
ni alegría ni desagrado.

---Lord Gray---le dije,---ha tiempo que estoy esperando la última
lección de esgrima.

---Hoy no tengo humor para lecciones.

---La necesitaré pronto.

---¿Va usted a batirse? ¡Qué felicidad! ¡Hoy tengo yo un humor!\ldots{}
Deseo atravesar a cualquiera.

---Yo también, lord Gray.

---Amigo mío, proporcióneme usted un hombre con quien romperme el alma.

---¿Tiene usted \emph{spleen}?

---Horroroso.

---Y yo. Los españoles también solemos padecer esa enfermedad.

---Es muy raro. En buena ocasión me ha salido usted hoy al encuentro.

---¿Por qué?

---Porque tenía una mala tentación. Estaba en lo más negro de la negrura
del \emph{spleen}, y pasó por mí la idea de pegarme un tiro o de
arrojarme de cabeza al mar.

---Todo por un amor desgraciado. Cuénteme usted eso y le daré buenos
consejos.

---No me hacen falta. Yo me entiendo solo.

---Yo conozco a la mujer que le trae a usted a tan lastimoso estado.

---Usted no conoce nada. Dejemos esa cuestión y no hablemos más de ella.

Aquella vez, como otras muchas, lord Gray esquivaba tratar el asunto.

---¿Con que quiere usted que le dé una lección?---me dijo después.

---Sí; pero tal, que con ella aprenda de una vez todo lo que encierra el
noble arte de la esgrima; porque, milord, tengo que matar a uno.

---Es cosa fácil. Le matará usted.

---¿Vamos a casa de milord?

---No; vamos al ventorrillo de Poenco. Beberemos un poco. ¿Y cuándo va
usted a matar a ese hombre?

---Cuando tenga la certeza de su alevosía. Hasta hoy tengo indicios que
casi son datos evidentes; de los cuales resultan sospechas que casi son
la misma certidumbre. Pero necesito más, porque mi alma, crédula hasta
lo sumo, forja sutilezas y escrúpulos. La pícara quiere prolongar su
felicidad.

Él calló y yo también. Silenciosamente llegamos a Puerta de Tierra.

Había en casa del señor Poenco gran remesa de majas y gente del bronce,
y las coplas picantes, con el guitarreo y las palmadas, formaban
estrepitosa música dentro y fuera de la casa.

---Entremos---me dijo lord Gray.---Esta graciosa canalla y sus
costumbres me cautivan. Poenco, llévanos al cuarto de dentro.

---Aquí viene lo güeno---exclamó Poenco.---Desapartarse todo el mundo.
Abran calle; calle, señores\ldots{} espejen, que pasa su majestad
miloro.

---Muchachos, ¡viva miloro y las cortes de la Isla!---gritó el tío
Lombrijón levantándose de su asiento y saludándonos, sombrero en mano,
con aquel garbo majestuoso que es tan propio de gente andaluza.---Y en
celebración del santo del día, que es la santísima libertad de la
imprenta, señó Poenco, suelte usted la espita y que corra un mar de
manzanilla. Todo lo que beba miloro y la compaña lo pago yo, que aquí
está un caballero pa otro caballero.

El tío Lombrijón era un viejo robusto y poderoso, de voz bronca y gestos
gallardos y caballerescos. Era traficante en vinos y gozaba opinión de
hombre rico, así como de gran galanteador y mujeriego, a pesar de la
madurez de sus años.

Lord Gray le dio las gracias, pero sin imitarle ni en el tono ni en los
movimientos, diferenciándose en esto de la mayor parte de los ingleses
que visitan las Andalucías, los cuales tienen empeño en hablar y vestir
como la gente del país.

---Oigasté, tío Lombrijón---dijo otro a quien llamaban Vejarruco, y que
era joven y curtidor en el Puerto.---A mí no me falta ningún hombre
nacío.

---¿Por qué lo dices, camaraíya, y en qué te he faltado?---dijo
Lombrijón.

---Bien lo sabes, camaraíya,---repuso Vejarruco.---En que asina que vi
venir a miloro y la compañía, dije al señor Poenco: «Lo que beba miloro
y la compañía, corre de mi cuenta; que aquí hay un caballero pa otro
caballero.»

---¡Zorongo!---exclamó Lombrijón.---Pero di, Vejarruco, ¿eso es conmigo?

---¡Cachirulo!, contigo es.

---Estira más esa estampa, que no te veo bien.

---Alarga el jocico pa que te tome el molde de él.

---¡Carambita! ¿Usté no sabe que cuando me pica un mosquito le
desmondongo al momento?

---¡Sonsoniche! ¿Usté no sabe que cuando le pego un pezco a un hombre
tiene que pedir prestaos dientes y muelas para comer?

---Basta ya, que se me van regolviendo los sentidos garrofales---dijo
Lombrijón.---Señores, empiecen a cantar el requieternam por ese
probesito Vejarruco.

---Alentaíto está el viejo.

---Pues allá va la lezna.

Lombrijón se llevó la mano al cinturón en ademán de sacar la navaja, y
todos los presentes, principalmente las mujeres, empezaron a gritar.

---Señores, no temblar---indicó Vejarruco.

---No se batirán---me dijo lord Gray.---Todos los días hacen lo mismo y
después no hay nada.

---No he traído el escarbador de dientes---dijo Lombrijón, encontrándose
sin armas.

---Pues ni yo tampoco---añadió Vejarruco.

---Camaraíya, por eso no ha de quedar. Usté está amarillo. Señores,
cuando eché mano al cinturón me relucieron las uñas, y pensó que era
jierro.

---¡Zorongo! Camará, usté ha escondido la lezna para que no haya
compromiso.

---Tú te la habrás metío en el garguero.

---Yo no la traigo, por humaniá---repuso Vejarruco,---porque como tengo
esta mano tan pesá, se necesita mucha prudencia pa no matar caa momento.

---Vaya, déjenlo para después---dijo Poenco,---y a beber.

---Lo que hace por mí, no tengo prisa\ldots{} Si Vejarruco se quiere
confesar antes que le endiñe\ldots{}

---Lo que es por mí\ldots{} cuando Lombrijón quiera el pasaporte para la
\emph{secula culorum}, se lo daré.

---Pelillos a la mar---dijo Poenco;---y pos que los dos han de morir,
mueran amigos.

---No hay por qué ofenderse, comparito. ¿Usté se ha ofendío?---preguntó
Lombrijón a su antagonista.

---¡Cachirulo! Yo no, ¿y usté?

---Tampoco.

---Pues vengan esos cinco mandamientos.

---Allá van, y vivan las Cortes y viva miloro.

---Para cortar la cuestión---dijo lord Gray,---yo pagaré a todo el
mundo. Poenco, sírvenos.

Las majas que allí había obsequiaron a lord Gray con sonrisas y dichos
graciosos; pero el inglés no tenía humor de bromas.

---¿Ha venido María de las Nieves?---preguntó a una.

---Pesaíto está con María de las Nieves. ¿Nosotras somos aljofifas?

---Si miloro va esta noche a mi casa---dijo en voz baja otra, que era,
si no me engaño, Pepa Higadillos,---verá lo bueno. Mi marío ha ido a
comprar burros, y me divierto pa matar la soleá.

---A donde irá miloro esta noche es a mi casa---indicó otra que era ya
matrona.---A mi casa va toda la sal del mundo, y si miloro quiere poner
un par de pesetas a un caballo, no tengo comeniente\ldots{} Mi casa es
muy principal\ldots{}

Lord Gray se apartó con hastío de aquella gente, y entramos en un
cuarto, donde el tabernero recibía tan sólo a cierta clase de personas,
y la mesa junto a la cual nos sentamos viose al punto cubierta del rico
tributo de aquellas viñas costaneras, que no tuvieron ni tienen igual en
el mundo.

\hypertarget{xv}{%
\chapter{XV}\label{xv}}

---Hoy voy a beber mucho---me dijo el inglés.---Si Dios no hubiese hecho
a Jerez, ¡cuán imperfecta sería su obra! ¿En qué día lo hizo? Yo creo
que debió de ser en el sétimo, antes del descanso, pues ¿cómo había de
descansar tranquilo si antes no rematara su obra?

---Así debió de ser.

---No; me parece que fue en el célebre día, cuando dijo: «Hágase la
luz;» porque esto es luz, amigo mío, y quien dice la luz, dice el
entendimiento.

---Señó miloro---dijo Poenco acercándose a mi amigo para hablarle con
oficioso sigilo;---María de las Nieves está ya loquita por vucencia. Se
hizo todo, y ya tiene su pañolón, sus zarcillos y su basquiña. Si no hay
nada que resista a ese jociquito rubio; y como vucencia siga aquí, nos
vamos a quedar sin donceyas.

---Poenco---dijo lord Gray,---déjame en paz con tus doncellas, y lárgate
de aquí, si no quieres que te rompa una botella en la cara.

---Pues najencia, me voy. No se enfade mi niño. Yo soy hombre discreto.
Pero sabe vucencia que ofrecí dos duros a la tía Higadillos que llevó el
pañolón\ldots{} cétera; cétera.

Lord Gray sacó dos duros y los tiró al suelo sin mirar al tabernero,
quien tomándolos, tuvo a bien dejarnos solos.

---Amigo---me dijo el inglés,---ya no me queda nada por ver en las
negras profundidades del vicio. Todo lo que se ve allá abajo es
repugnante. Lo único que vale algo es este vivífico licor, que no engaña
jamás, como proceda de buenas cepas. Su generoso fuego, encendiendo
llamas de inteligencia en nuestra mente, nos sutiliza, elevándonos sobre
la vulgar superficie en que vivimos.

Lord Gray bebía con arte y elegancia, idealizando el vicio como
Anacreonte. Yo bebía también, inducido por él, y por primera vez en la
vida, sentía aquel afán de adormecimiento, de olvido, de modificación en
las ideas, que impulsa en sus incontinencias a los buenos bebedores
ingleses.

Resonó un cañonazo en el fondo de la bahía.

---Los franceses arrecian el bombardeo---dije asomándome al ventanillo.

---Y al son de esta música los clérigos y los abogados de las Cortes se
ocupan en demoler a España para levantar otra nueva. Están borrachos.

---Me parece que los borrachos son otros, milord.

---Quieren que haya igualdad. Muy bien. Lombrijón y Vejarruco serán
ministros.

---Si viene la igualdad y se acaba la religión, ¿quién le impedirá a
usted casarse con una española?---dije regresando junto a la mesa.

---Yo quiero que me lo impidan.

---¿Para qué?

---Para arrancarla de las garras que la sujetan; para romper las
barreras que la religión y la nacionalidad ponen entre ella y yo; para
reírme en las barbas de doce obispos y de cien nobles finchados, y
derribar a puntapiés ocho conventos, y hacer burla de la gloriosa
historia de diez y siete siglos, y restablecer el estado primitivo.

Decía esto en plena efervescencia, y no pude menos de reírme de él.

---Hermoso país es España---continuó.---Esa canalla de las Cortes lo va
a echar a perder. Huí de Inglaterra para que mis paisanos no me
rompieran los oídos con sus chillidos en el Parlamento, con sus pregones
del precio del algodón y de la harina, y aquí encontré las mayores
delicias, porque no hay fábricas, ni fabricantes panzudos, sino
graciosos majos; ni polizontes estirados, sino chusquísimos ladrones y
contrabandistas; porque no había boxeadores, sino toreros; porque no hay
generales de academia, sino guerrilleros; porque no hay fondas, sino
conventos llenos de poesía; y en vez de lores secos y amojamados por la
etiqueta, estos nobles que van a las tabernas a emborracharse con las
majas; y en vez de filósofos pedantes, frailes pacíficos que no hacen
nada; y en vez de amarga cerveza, vino que es fuego y luz, y
sobrenatural espíritu\ldots{} ¡Oh, amigo! Yo debí nacer en España. Si yo
hubiese nacido bajo este sol, habría sido guerrillero hoy y mendigo
mañana, y fraile al amanecer y torero por la tarde, y majo y sacristán
de conventos de monjas, abate y petimetre contrabandista y salteador de
caminos\ldots{} España es el país de la naturaleza desnuda, de las
pasiones exageradas, de los sentimientos enérgicos, del bien y el mal
sueltos y libres, de los privilegios que traen las luchas, de la guerra
continua, del nunca descansar\ldots{} Amo todas esas fortalezas que ha
ido levantando la historia, para tener yo el placer de escalarlas; amo
los caracteres tenaces y testarudos para contrariarlos; amo los peligros
para acometerlos; amo lo imposible para reírme de la lógica,
facilitándolo; amo todo lo que es inaccesible y abrupto en el orden
moral, para vencerlo; amo las tempestades todas para lanzarme en ellas,
impelido por la curiosidad de ver si salgo sano y salvo de sus
mortíferos remolinos; gusto de que me digan «de aquí no pasarás,» para
contestar «pasaré.»

Yo sentía inusitado ardor en mi cabeza, y la sangre se me inflamaba
dentro de las venas. Oyendo a lord Gray, sentime inclinado a abatir su
estupendo orgullo, y con altanería le dije:

---Pues no, no pasará usted.

---¡Pues pasaré!---me contestó.

---Yo amo lo recto, lo justo, lo verdadero, y detesto los locos absurdos
y las intenciones soberbias. Allí donde veo un orgulloso, le humillo;
allí donde veo un ladrón, le mato; allí donde veo un intruso, le arrojo
fuera.

---Amigo---me dijo el inglés,---me parece que a usted se le van los
humos de la manzanilla a la cabeza. Yo le digo como Lombrijón a
Vejarruco: «Camaraíta, ¿eso que ha dicho es conmigo?»

---Con usted.

---¿No somos amigos?

---No: no somos ni podemos ser amigos---exclamé con la exaltación de la
embriaguez.---¡Lord Gray, le odio a usted!

---Otro traguito---dijo el inglés con socarronería.---Hoy está usted
bravo. Antes de beber, habló de matar a un hombre.

---Sí, sí\ldots{} Y ese hombre es usted.

---¿Por qué he de morir, amigo?

---Porque quiero, lord Gray; ahora mismo. Elija usted sitio y armas.

---¿Armas? Un vaso de Pero Jiménez.

Me levanté fuera de mí, y así una silla con resolución hostil; pero lord
Gray permaneció tan impasible, tan indiferente a mi cólera, y al mismo
tiempo tan sereno y risueño, que sentime sin bríos para descargarle el
golpe.

---Despacio. Nos batiremos luego---dijo rompiendo a reír con expansiva
jovialidad.---Ahora voy a declarar la causa de ese repentino enfado y
anhelo de matarme. ¡Pobrecito de mí!

---¿Cuál es?

---Cuestión de faldas. Una supuesta rivalidad, Sr.~D. Gabriel.

---Dígalo usted todo de una vez---exclamé sintiendo que se redoblaba mi
coraje.

---Usted está celoso y ofendido, porque supone que le he quitado su
dama.

No le contesté.

---Pues no hay nada de eso, amigo mío.---añadió.---Respire usted
tranquilo las auras del amor. Me parece haberle oído decir a Poenco que
usted anda a caza de esa Mariquilla, que no de las Nieves, sino de los
Fuegos debería llamarse. A usted le han dicho que yo\ldots{} pues, diré
como Poenco\ldots{} «cétera, cétera.» Amigo mío, cierto es que me
gustaba esa muchacha; pero basta que un camaraíyahaya puesto los ojos en
ella para que yo no intente seguir adelante. Esto se llama generosidad;
no es el primer caso que se encuentra en mi vida. En celebración de paz,
acabemos esta botella.

Al frenesí que antes había yo sentido sucedió un entorpecimiento y
oscuridad tal de mis facultades intelectuales, que no supe qué responder
a lord Gray, ni realmente le respondí nada.

---Pero, amigo mío---prosiguió él, menos afectado que yo por la
bebida,---hemos sabido que a Mariquilla de las Nieves la corteja\ldots{}
¡cortejar!, hermosa palabra que no tiene igual en ningún idioma\ldots{}
pues decía que la corteja un guapo de Jerez que se me figura es más
afortunado que nosotros. Sin duda a ese es a quien usted quiere matar.

---¡A ese, a ese!---dije sintiendo que se me despejaban un tanto los
aposentos altos.

---Cuente usted conmigo. Currito Báez, que así se llama el jerezano, es
un necio presumido y matasiete, que con todo el mundo arma camorra.
Deseo tener cuestión con él. Le provocaremos.

---¡Le provocaremos, sí, señor; le provocaremos!

---Le mataremos delante de toda la gente del bronce, para que vean cómo
sucumbe un tonto a manos de un caballero\ldots{} Pero no sabía que
estuviera usted enamorado. ¿Desde cuándo?

---Desde hace mucho, mucho tiempo---respondí viendo cómo daba vueltas la
habitación delante de mis ojos.---Éramos niños; ella y yo estábamos
abandonados y solos en el mundo. La desgracia nos impelió a
compadecernos, y compadeciéndonos, sin saber cómo, nos amamos. Padecimos
juntos grandes desventuras, y fiando en Dios y en nuestro amor vencimos
inmensos peligros. Llegué a considerarla como indisolublemente unida a
mí por superior destino, y mi corazón fortalecido por una fe sin
límites, no padeció en mucho tiempo los martirios de celos,
desconfianzas, temores ni amorosos sobresaltos.

---Hombre: eso es extraordinario. ¡Y todo por María de las
Nieves!\ldots{}

---Pero todo se acabó, amigo mío. El mundo se me ha caído encima. ¿No lo
ve usted, no lo ve usted caer a pedazos sobre mi cabeza? ¿No ve usted
estas montañas que me machacan los sesos? Mi cerebro hecho trizas salta
en piltrafas mil y salpicando se esparce por las paredes\ldots{}
aquí\ldots{} allí\ldots{} más allá. ¿No lo ve usted?

---Ya lo veo\ldots---repuso lord Gray, rematando una botella.

---El mundo se me cayó encima. Se apagó el sol\ldots{} ¿No lo ve usted,
hombre; no advierte las horribles tinieblas que nos rodean? Todo se
oscureció, cielo y tierra, y el sol y la luna cayeron, como ascuas de un
cigarro\ldots{} Ella y yo nos separamos: leguas y más leguas, días y
días y más días se pusieron entre nosotros; yo alargaba los brazos
ansiando tocarla con mis manos; pero mis manos no tocaban sino el vacío.
Ella subió y yo me quedé donde estaba. Yo miraba y no veía nada\ldots{}
estaba escondida: ¿dónde?, dirá usted\ldots{} dentro de mi cerebro. Yo
me metía las manos en la cabeza y escarbaba allí dentro; pero no la
podía coger. Era una burbuja, una partícula, un átomo bullicioso y
movible que me atormentaba en sueños y despierto. Quise olvidarla y no
pude. De noche cruzaba los brazos y decía: «aquí la tengo; nadie me la
quitará\ldots» Cuando me dijeron que me había olvidado, no lo quería
creer. Salí a la calle y todo el mundo se reía de mí. ¡Espantosa noche!
Escupí al cielo y lo dejé negro\ldots{} Me metí la mano en el pecho,
saqué el corazón, lo estrujé como una naranja y se lo arrojé a los
perros.

---¡Qué inmenso e ideal amor!---exclamó lord Gray.---Y todo eso por
Mariquilla de las Nieves\ldots{} Beba usted esa copa.

---Supe que amaba a otro---añadí sintiendo que mi cerebro despedía una
lumbre vagorosa y desparramada, llama de alcohol que trazaba mil figuras
en el espacio con sus lenguas azules.---Amaba a otro. Una noche se me
apareció. Iba de brazo con su nuevo amante. Pasaron por delante de mí y
no me miraron. Yo me levanté y tomando la espada, herí en el vacío, y en
el vacío surgió un manantial de sangre. La vi que se llegaba hacia mí
pidiéndome perdón. La manga de su vestido tocó mi rostro, y me quemó.
¿Ve usted la quemadura, la ve usted?

---Sí, la veo, la veo. ¡Y todo por María de las Nieves!\ldots{} Hombre
es gracioso. A ver a qué sabe este Montilla.

---Yo quiero matar a ese hombre, o que él me mate a mí.

---No, a él, a él. ¡Pobre Currito Báez!

---Le mataré, le mataré, sí---exclamaba yo con furor, poniendo mi puño
cerrado en el pecho de lord Gray.---¿No siente usted cómo baila el mundo
bajo nuestros pies? El mar entra por esa ventana. Ahoguémonos juntos y
todo se concluirá.

---¿Ahogarme? No---dijo el inglés.---Yo también amo.

A pesar de mi lastimoso estado intelectual presté atención vivísima a
sus palabras.

---Yo también amo---prosiguió.---Mi amor es secreto, misterioso y
oculto, como las perlas, que además de estar dentro de una concha están
en el fondo del mar. No tengo celos de nadie, porque su corazón es todo
mío. No tengo celos más que de la publicidad; odio de muerte a todo el
que descubra y propale mi secreto. Antes me arrancaré la lengua que
pronunciar su nombre delante de otra persona. Su nombre, su casa, su
familia, todo es misterioso. Yo me deslizo en la oscuridad, en oscuridad
profunda que no proyecte sobra alguna, y abro mis brazos para recibirla,
y los oscuros cuerpos se confunden en el negro espacio. Bullen átomos de
luz, como estos que ahora nos rodean, y en las puntas de nuestros
cabellos palpita con galvánica fuerza, embriagadora sensibilidad. ¿No
percibe usted estas ondas que vienen del cielo, no siente usted cómo se
abre la tierra y despide cien mil vidas nuevas, creadas en esta corola
donde estamos, y en cuyos bordes nos movemos a impulso de la suave y
embalsamada brisa?

---¡Sí, lo veo, lo veo!---respondí llevando el vaso a mis labios.

---Amigo mío, Dios hizo perfectamente al amasar este barro del mundo.
Habría sido lástima que no lo hiciera. La materia vivificada por el amor
es sin duda lo mejor que existe después del espíritu. Yo adoro el
universo lleno de luz, pintado con lindos colores, sombreado por
amorosas opacidades que cubren el discreto amor; yo adoro la naturaleza
que todo lo hizo hermoso, y detesto a los hombres corruptores del
elemento donde habitan, como ensucian los sapos la laguna. Mi alma se
arroja fuera de este lodazal y busca los aires puros; huye de las
infectas madrigueras de la civilización, abiertas en fango pestilente y
se baña en los rayos de oro que cruzan los espacios. Olvidaba decir a
usted que para hacer más encantadora mi aventura, la historia, es decir,
diez y siete siglos de guerras, de tratados de privilegios, de tiranía,
de fanatismo religioso, se oponen a que sea mía. Necesito demoler las
torres del orgullo, abatir los alcázares del fanatismo, burlarme de la
fatuidad de cien familias que cifran su orgullo en descender de un rey
asesino, D. Enrique II, y de una reina liviana, doña Urraca de Castilla;
apalear cien frailes, azotar cien dueñas, profanar la casa llena de
pintarreados blasones, y hasta el mismo templo lleno de sepulcros, si la
refugian en él.

---¿La va usted a robar, milord?---pregunté en un instante de rápida
lucidez.

---Sí; la robaré y me la llevaré a Malta, donde tengo un palacio. He
pedido un barco a Inglaterra.

Sentí súbito estremecimiento, como si mi conturbada naturaleza hiciera
un esfuerzo colosal para recobrar su perdido aliento.

---Lord Gray---dije,---somos amigos. Soy discreto. Yo le ayudaré a usted
en esa empresa, que no será fácil por desgracia.

---No lo será\ldots{} veremos---repuso exaltado después de beber con
ardiente anhelo.---Yo le ayudaré a usted a matar a Currito Báez.

---Sí, le mataré; así tuviera mil vidas. Pero permítame usted que le
pague su auxilio, ofreciéndole el mío para robar a esa mujer, y
burlarnos de diez y siete siglos de guerras, de tratados, de
privilegios, de fanatismo, de religión, de tiranía.

---Bien, amigo Gabriel; venga esa mano. ¡Viva lo imposible! El placer de
acometerlo es el único placer real.

---Yo quisiera estar en los secretos de usted, milord.

---Lo estará usted.

---Yo mataré a mi hombre.

---Y pronto. Venga esa mano.

---Ahí va.

---Ahora bajemos---dijo lord Gray en el apogeo de su delirio.

---¿A dónde?

---Al mundo.

---El mundo se ha hecho pedazos, no existe---dije yo.

---Lo compondremos. Una vez se me rompió en mil pedazos un vaso etrusco
que compré en Nápoles. Yo recogí los trozos uno a uno y los pegué
perfectamente\ldots{} ¡Oh, amada mía! ¿Dónde estás que no te veo? Este
perfume de flores, esta música me anuncian que no estás lejos. Sr.~de
Araceli, ¿no la oye usted?

---Sí, una música encantadora---respondí, y era verdad que creí oírla.

---Ella viene envuelta en la nube que la rodea. ¿No advierte usted la
deslumbradora claridad que entra en la pieza?

---Sí, la veo.

---Mi amada viene, Sr.~de Araceli; ya entra; aquí está.

Miré a la puerta y la vi; era ella misma, rodeada de una luz dorada y
pálida como la manzanilla y el Jerez que habíamos bebido. Quise
levantarme; pero mi cuerpo se hizo de plomo, mi cabeza pesó más que una
montaña y cayó entre mis brazos sobre la mesa, perdiendo de súbito toda
noción de existencia.

\hypertarget{xvi}{%
\chapter{XVI}\label{xvi}}

Al recobrarla lenta y oscura, la voz del señor Poenco fue el accidente
que me dio a conocer que había mundo. Lord Gray había desaparecido.
Reconocime y me encontré estúpido; pero la vergüenza, motivada por el
recuerdo de mi envilecimiento, vino más tarde. ¡Y qué vergüenza aquella,
señores! Mucho tiempo tardé en perdonarme.

Pero echemos un velo, como dicen los historiadores, sobre el infausto
suceso de mi embriaguez, y sigamos el cuento.

Desde tal día, el servicio en la Cortadura y en Matagorda me entretuvo
algún tiempo, y no me fueron posibles aquellas visitas, ya tristísimas,
ya alegres, que hacía a Cádiz; pero al fin, como el asedio no era
penoso, disfruté de algún vagar, y un día púseme en camino de la calle
Ancha, con intento de resolver allí qué dirección tomar.

En tiempos normales era la calle Ancha el sitio donde se reunía la
caterva de mentirosos, desocupados, noveleros y toda la gente curiosa,
alegre y holgazana. Allí iban también de paseo a la hora de medio día en
invierno y por las tardes en verano las damas a la moda y los
petimetres, abates y enamorados, ocurriendo con estos mil lances y
escenas de que nos ha dejado retrato muy vivo D. Juan del Castillo en
sus sainetes urbanos, no menos graciosos y verdaderos que los populares
y consagrados a la majeza.

Pero en 1811, y después que las Cortes se trasladaron a Cádiz, la calle
Ancha, además de un paseo público, era, si se me permite el símil, el
corazón de España. Allí se conocían, antes que en ninguna parte, los
sucesos de la guerra, las batallas ganadas o perdidas, los proyectos
legislativos, los decretos del gobierno legítimo y las disposiciones del
intruso, la política toda, desde la más grande a la más menuda, y lo que
después se ha llamado chismes políticos, marejada política, mar de fondo
y cabildeos. Conocíanse asimismo los cambios de empleados y el
movimiento de aquella administración que, con su enorme balumba de
consejos, secretarías, contadurías, real sello, juntas superiores,
superintendencias, real giro, real estampilla, renovación de vales,
medios, arbitrios, etc., se refugió en Cádiz después de la invasión de
las Andalucías. Cádiz reventaba de oficinas y estaba atestada de
legajos.

Además, la calle Ancha obtenía la primacía en la edición y propaganda de
los diferentes impresos y manuscritos con que entonces se apacentaba la
opinión pública; y lo mismo las rencillas de los literatos que las
discordias de los políticos, lo mismo los epigramas que las diatribas,
que los vejámenes, que las caricaturas, allí salieron por primera vez a
la copiosa luz de la publicidad. En la calle Ancha se recitaban, pasando
de boca en boca, los malignos versos de Arriaza, y las biliosas
diatribas de Capmany contra Quintana.

Allí aparecieron, arrebatados de una mano a otra mano, los primeros
números de aquellos periodiquitos tan inocentes, mariposillas nacidas al
tibio calor de la libertad de la imprenta, en su crepúsculo matutino;
aquellos periodiquitos que se llamaron \emph{El Revisor Político, El
Telégrafo Americano, El Conciso, La Gaceta de la Regencia, El
Robespierre Español, El Amigo de las Leyes, El Censor General, El Diario
de la Tarde, La Abeja Española, El Duende de los Cafés y El Procurador
general de la Nación y del Rey}, algunos, absolutistas y enemigos de las
reformas; los más, liberales y defensores de las nuevas leyes.

Allí se trabaron las primeras disputas de las cuales hicieron luego
escandalosa síntesis los autores respectivamente de los dos célebres
libros \emph{Diccionario manual} y \emph{Diccionario crítico burlesco},
ambos signo claro de la gran reyerta y cachetina que en el resto de
siglo se había de armar entre los dos fanatismos que ha tiempo vienen
luchando y lucharán por largo espacio todavía.

En la calle Ancha, en suma, se congregaba todo el patriotismo con todo
el fanatismo de los tiempos; allí, la inocencia de aquella edad; allí,
su bullicioso deseo de novedades; allí, la voluble petulancia española
con el heroico espíritu, la franqueza, el donaire, la fanfarronada, y
también la virtud modesta y callada. Tenía la calle Ancha mucho de lo
que llamamos Salón de conferencias, de lo que hoy es Bolsa, Bolsín,
Ateneo, Círculo, Tertulia, y era también un club.

Cualquiera que entonces entrase en ella por las calles de la Verónica o
Novena y la atravesase en dirección a la plaza de San Antonio, habríase
creído transportado a la capital de un pueblo en pleno goce del más
acabado bienestar y aun de la paz más completa, si no mostrara otra cosa
la multitud de uniformes militares, tan varios como alegres, que
abundantemente se veían. Gastaban las damas gaditanas ostentoso lujo, no
sólo por hacer alarde de tranquilidad ante las amenazas de los
franceses, sino porque era Cádiz entonces ciudad de gran riqueza,
guardadora de los tesoros de ambas Indias. Casi todos los petimetres y
la juventud florida en masa, lo mismo de la aristocracia que del alto
comercio, se habían instalado en los diferentes cuerpos de voluntarios
que en Febrero de 1810 se formaron; y como en tales cuerpos ha dominado
siempre, por lo común, la vanidad de lucir uniformes y arreos de gran
golpe de vista, aquello fue una bendición de Dios para el lucimiento de
sastres y costureras, y los milicianos de Cádiz estaban que ni pintados.

Debo advertir que se portaron bien y con verdadero espíritu militar en
todo lo muy difícil y arriesgado que durante el sitio se les confió;
pero su principal triunfo estaba en la calle Ancha entre muchachas
solteras, casadas y viuditas.

Llamábanse unos los \emph{guacamayos}, por haber elegido el color grana
para su uniforme, y estos formaban cuatro batallones de línea. Menos
vistoso y deslumbrador era el vestido de los dos batallones de ligeros,
a quienes llamaron cananeos, por usar cananas en vez de cartucheras.
Otros, por haber aplicado profusamente a sus personas el color verde,
fueron designados con el nombre de \emph{lechuguinos}, si bien hay quien
atribuye este apodo a la circunstancia de pertenecer los tales
lechuguinos a los barrios de Puerta de Tierra y extramuros, donde se
crían lechugas. Con los mozos de cuerda y trabajadores formose un
regimiento de artillería, y como eligieran para decorarse el morado, el
rojo y el verde, en episcopal combinación, fueron llamados los
\emph{obispos}, y no hubo quien les quitara el nombre durante todo el
transcurso de la guerra. Otros, que militaron en la infantería, y eran
modestísimos en estatura y traje, fueron designados con el mote de
\emph{perejiles}, y a las personas graves que habían formado una milicia
urbana y exornádose con un levitón negro y cuello encarnado, se les
tituló los \emph{pavos}. Todos llevaban nombre contrahecho, y hasta el
cuerpo que se formó con los desertores polacos, no pudo llamarse nunca
de los \emph{polacos}, sino de las \emph{polacras}.

Todo este inmenso, variado y pintoresco personal de \emph{guacamayos},
\emph{cananeos}, \emph{obispos}, \emph{perejiles} y \emph{pavos}
discurría por la calle Ancha y plaza de San Antonio, llamada entonces
\emph{Golfo de las damas}, en las horas que dejaba libres el servicio,
menos penoso y arriesgado allí que en Zaragoza. Formaban los variados
uniformes, a los cuales se añadían los nuestros y los de los ingleses,
la más animada y alegre mescolanza que puede ofrecerse a la vista; y
como las señoras no llevaban sus guardapiés y faldellinas de luto, sino
por el contrario, de los más brillantes rasos blancos, amarillos o rosa,
con mantillas quier blancas, quier negras, y cintas emblemáticas, y
cucardas patrióticas a falta de flores, júzguese de cuán bonita sería
aquella calle Ancha, la cual, como calle, y aun desierta y abandonada
por el alegre gentío, es, con sólo el adorno de sus lindas casas, de sus
balcones siempre pintados y de sus mil vidrios, lo más bonito que existe
en ciudades del Mediodía.

Desde que llegué hube de encontrar muchos amigos, y comenzó el preguntar
y el responder, de esta manera:

---¿Qué dice hoy El \emph{Diario Mercantil}?

---Llama ladrones a todos los amigos de las reformas, y dice que llegará
día en que el obispo de Orense ponga un grillete al pie a los pícaros
que le encausaron por no querer jurar.

---Pues para ser enemigo de la libertad de la imprenta, El \emph{Diario
Mercantil} no se muerde la lengua.

---¡Pero qué bien le contesta hoy \emph{El Conciso!} Le dice que los
\emph{matacandelas de toda luz de la razón, no quisieran que alumbrase
al mundo más luz que la de las hogueras inquisitoriales.}

---Peor les trata \emph{El Robespierre Español}, que dice: \emph{«El
antiguo edificio romanesco-gótico-moruno de las preocupaciones caerá, y
quedaranse a la luna de Valencia tanto vampiro, cárabo y lechuzo
como\ldots{}}

\small
\newlength\mlenh
\settowidth\mlenh{lámparas mata y el aceite chupa.»}
\begin{center}
\parbox{\mlenh}{lámparas mata y el aceite chupa.»}                  \\
\end{center}
\normalsize

---Pero veamos qué dice \emph{El Concisín}.

Y sacaron un diminuto papel, húmedo aún como recién salido de la prensa,
el cual era una especie de suplemento, hijuela y lugarteniente de
\emph{El Conciso} grande, y en su lenguaje figuraba un niño que venía a
contarle a su papá lo que ocurría por las Cortes.

\emph{---El Concisín} dice: «Después del Sr.~Argüelles, que habló con
tanta elocuencia como de costumbre, antojósele a Ostolaza dar al viento
el repiqueteo de su voz clueca y becerril, y entre las risas de las
tribunas y el alborozo del paraíso, defendió a los uñilargos y
pancirrellenos que viven del arca-boba de la Iglesia.»

---Hombre, los trata con demasiada benevolencia.

---Ellos nos llaman a nosotros \emph{herejotes} y \emph{calabazones}.

---Si no se puede sufrir a esa canalla. Hay que poner una horca en el
Golfo de las Damas para colgar serviles, empezando por los de capilla y
acabando por los de faldón.

---Deje usted que nos sacudamos a Soult, y los cananeos dejaremos a
España como una balsa de aceite. ¿Y qué se sabe del lord?

---Va sobre Badajoz.

---Massena viene en retirada desde Portugal.

---Los franceses han abandonado a Campomayor.

---Pronto se unirá Castaños a Wellington.

---Señora doña Flora de Cisniega, tenga usted felices días.

---Felices, señores guacamayos. Lord Gray, felices, y usted, Sr.~de
Araceli, téngalos muy buenos, aunque no sea sino por lo caro que se
vende.

Al mismo tiempo que doña Flora, se presentó ante mí lord Gray. Hablome
la dama con cierto sonsonete reprensivo que me hizo mucha gracia.
Recibía al mismo tiempo plácemes y finezas de todos los del corrillo, y
cortesía va, cortesía viene, la rodeamos llevándola calle adelante como
en procesión, con cola de cortesanos.

---Señores---dijo doña Flora,---la libertad de la imprenta es cosa que
ha de darnos muchas jaquecas. ¿No han visto ustedes cómo se atreve
\emph{El Revisor Político} a ocuparse de mis tertulias, y de si van o no
van a ellas filósofos y jacobinos? ¿Pues acaso entra en mi casa persona
que no sea digna del mayor respeto? No se han atrevido esos pícaros
diaristas a nombrarme, pero harto se conoce a quién va dirigido el
dardo.

---Señora---dijo un \emph{guacamayo},---la libertad de la imprenta,
según dijo Argüelles en las Cortes, allí donde tiene el veneno tiene
también la triaca. Pues ellos andan con alusioncitas, devolvámoselas, y
no pequeñas como nueces, sino gordas como calabazas, y no rellenas de
plomo frío cual las bombas de Villantroys, sino de fuego y metralla cual
las nuestras.

---¿Qué quiere decir eso, amiguito?

---Que a nuestra disposición tenemos \emph{El Robespierre Español},
\emph{El Duende de los Cafés} y al pícaro \emph{Concisín} que se
encargarán de poner cual no digan dueñas a los apaga-candelas.

---La alusión, señora doña Flora---dijo un \emph{obispo},---ha salido
sin duda de la tertulia de Paquita Larrea, la esposa del Sr.~Böhl de
Faber.

---¿Qué más que escribir una sátira de la tal tertulia con mucha sal y
pimienta, retratando a todos los que van a ella, y mandarla al
\emph{Robespierre} para que la estampe?---añadió un \emph{pavo}.

---No quiero que se diga que la sátira se ha fraguado en mi casa---dijo
doña Flora.---En paz con todo el mudo es mi mote, y si a mis tertulias
van tantas personas honradas y discretas es por pasar el tiempo
cultamente, y no para enredos e intriguillas.

---Es preciso defender la libertad hasta en las tertulias---dijo un
\emph{obispo}, o un lechuguino, que esto no lo recuerdo bien.

---En las trincheras es mejor---repuso doña Flora.---No quiero reñir con
Paquita Larrea, que si ella recibe a los Valientes, Ostolazas, Teneyros,
a los Morros y Borrulles, yo tengo el gusto de que vayan a mi casa los
Argüelles, Torenos y Quintanas, y no porque los haya escogido en el haz
de los que llaman liberales, sino porque casualmente concordaron en
ideas.

---No nos prive usted del placer de hacer una letrilla al menos en honor
de los tertulios de la Larrea---dijo un \emph{perejil}.

---No, señor \emph{perejil}---repuso ella,---reprima usted sus bríos
liberales, que ya voy viendo que la dichosa libertad de la imprenta es
un azote de Dios, y un castigo de nuestros pecados, como dice el Sr.~D.
Pedro del Congosto.

Debo indicar, que doña Francisca Larrea, esposa del entendido y digno
alemán Böhl de Faber, era mujer de mucho entendimiento, escritora, lo
mismo que su marido a quien eran muy familiares los primores de la
lengua castellana. De este matrimonio, nació Eliseo Böhl, a quien
debemos las mejores y más bellas pinturas de las costumbres de
Andalucía, novelista sin igual y de fama tan grande como merecida dentro
y fuera de España\footnote{Fernán Caballero.}.

Luego que la nube de \emph{guacamayos}, \emph{cananeos} y demás tropa
voluntaria descargó el nublado de sus adulaciones y cortesías, doña
Flora, aprovechando un claro de la conversación, me dijo:

---¡Muy bien, Sr.~D. Gabriel! Días y más días sin pasar por casa.
Después de aquella tremenda y borrascosa escena con D. Pedro, pocas
veces has ido por allá. Y no quedó poco comprometido mi honor\ldots{}

---Señora, francamente, temo que el señor D. Pedro me ensarte con su
gran espadón, porque de que está celoso como un turco no me queda duda
alguna. Su señoría el gran cruzado, va a tomar una venganza terrible por
el grandísimo agravio que le he hecho.

Conté a lord Gray en breves palabras lo ocurrido.

---No temas nada---dijo doña Flora.---Ahora te agradeceré que vayas a
casa a llevar a la señora condesa un recadito que me importa mucho.

---Con mil amores. ¿Pero está allí D. Pedro?

---¡Qué ha de estar!

---Respiro.

---Pues bien. Vas a casa al momento, y dices a Amaranta, que si quiere
ver a Inés y aun hablarla, vaya a las Cortes. Ella tiene cédula para la
tribuna.

---¿Qué dice usted?---exclamé con asombro.---¿Que Inés está en las
Cortes?

---Sí, se han plantado en San Felipe las tres niñas beatas. ¿Qué te
parece? Hace un rato volvía yo de la secretaría de Consolidación y
Contaduría general, en la plazuela de San Agustín, y me las encontré con
D. Paco. Díjome el buen preceptor, que las pobrecitas hacía dos semanas
que estaban suplicando a la señora doña María que las dejase salir a dar
un paseíllo por la muralla; y por último parece que los muchos ruegos y
continuas lamentaciones ablandaron la roca de las terquedades de la
condesa, que permitió a sus tres cautivas esparcirse un poco en el día
de hoy, durante hora y media. Bajo la tutela de D. Paco, en quien tiene
confianza sin límites la señora, dejolas esta salir, después de
vestirlas a lo monjil en tales modos, que parece van pidiendo para la
\emph{Archicofradía de los Clavos y Sagradas Espinas de Hermanas
Siervitas con voto de pobreza}. Dioles orden expresa de pasearse desde
la Aduana hasta el baluarte de la Candelaria, yendo y viniendo tres
veces, sin que por causa alguna infringiesen esta premática paseantil,
ni traspasasen la línea indicada, ni menos se internasen en las calles
de Cádiz, por donde después que están aquí las Cortes, discurren, como
dice el Sr.~Teneyro, todos los pecados y vicios en endemoniada
procesión\ldots{} Pero, ¿qué hacen mis niñas? Verás. En cuanto llegaron
a la calle del Baluarte amotináronse, empeñándose en que D. Paco las
había de llevar a las Cortes, porque tenían gran curiosidad, sed
devoradora de ver tan bonito espectáculo; gruñó el pobre preceptor,
chillaron ellas, se aferró él al programa que le trazara su ama,
rebeláronse las chicas, negándose a ir a la muralla, y luego le
acribillaron a pellizcos y alfilerazos. Presentación propuso a las otras
dos arrojar a D. Paco al mar, y después le quitaron el sombrero para
guardarlo en rehenes y privarle de tan útil prenda, si no las llevaba al
Congreso Nacional. Una de ellas tenía una papeleta de tribuna, que sin
duda algún galán travieso le dio con el fin que puede suponerse. Antes
los galanes, cuando no podían comunicarse con sus amadas, las citaban en
las iglesias, donde la religiosa oscuridad protegía el trasiego de las
cartitas, el apretón de manos u otro desahogo de peor especie, mientras
los padres embobados contemplaban las llamaradas del cuadro de Ánimas
del Purgatorio. Hoy cuando no puede haber reja ni correo, los amantes se
suelen citar en la tribuna de las Cortes. Es esta una invención
donosísima, ¿no es verdad, lord Gray? Sin duda está muy en boga en los
parlamentos de Inglaterra, y ahora nos la introducen en España para
mejoramiento de las costumbres.

Lord Gray, que había puesto atención a lo que doña Flora nos contaba,
repuso con malicia:

---Señora mía, deme usted licencia para retirarme, porque tengo una
ocupación, un quehacer imprescindible no lejos de aquí.

---Sí, vaya usted, vaya usted. Ahora deben estar en la discusión de los
señoríos jurisdiccionales. Mucho ruido, mucho barullo en las tribunas.
Usted entrará en la de los diplomáticos, que está mano a mano con la de
señoras. Corra usted, adiós.

Dejome lord Gray en las garras de doña Flora, la cual continuó así:

---El pobre D. Paco se defendió hasta que no pudo más. ¡Pobre señor! No
tuvo más remedio que bajar la cabeza ante el número y llevarlas a las
Cortes. Cuando le encontré y me contó el lance, iba el pobre tan
cari-entristecido, cual si lo llevaran a ajusticiar, y me dijo: «Ay de
mí, si doña María llega a saber esto\ldots{} ¡Malditas sean las Cortes y
el perro que las inventó!»

---¿Estarán todavía allá?

---Sí: corre a avisárselo a la condesa. La pobrecita hace tiempo que
está arando la tierra por ver a Inés dentro o fuera de su cárcel, y no
puede conseguirlo, pues a ella no la admiten allá, y se pasan meses y
meses sin que se les permita dar un paseo con el ayo. Conque ve a
decírselo y tú mismo la acompañarás a San Felipe. No tardes, hijo, y en
seguida a casa derechito que tengo que hablarte. ¿Comerás hoy con
nosotros?

Me despedí con gran precipitación de doña Flora, dejándola en poder de
los guacamayos, y me alejé de allí; pero en vez de correr hacia la calle
de la Verónica, mi curiosidad, mi pasión y un afán invencible me
impulsaron hacia la plaza de San Felipe, olvidando a Amaranta y a doña
Flora, fija el alma y la vida toda en las tres muchachas, en D. Paco, en
lord Gray, en las Cortes, en los diputados y en la discusión sobre
señoríos jurisdiccionales.

\hypertarget{xvii}{%
\chapter{XVII}\label{xvii}}

Llegué, y en la pequeña plazoleta que hay a la entrada de la iglesia,
entonces convertida en Congreso, había, como de costumbre, gran gentío.
Extendí con avidez la vista por la multitud de caras que allí se
confundían, y no vi ninguna de las que buscaba. Pensando que estarían
todos arriba, traspasé la puertecilla que conducía a la escalera de las
tribunas, pero en el vestíbulo, o más bien pasadizo, la gente que
bajaba, tropezando con la que quería subir, formaba remolinos y
marejada. Pugnaba yo por entrar cuando vi cerca de mí a Presentación,
que estrujada por espaldas y hombros muy robustos, mostraba gran
aflicción y pesadumbre de haberse metido en tal fregado. Las otras dos y
D. Paco no estaban allí.

Al punto acudí a sacarla de apreturas, y al reconocerme se alegró mucho
y me dio las gracias.

---¿Dónde están las otras dos y D. Paco?---le pregunté.

---¡Ay!, no sé\ldots---exclamó con zozobra.---Entre el gentío, Inés y
Asunción se separaron de mí. Después las vimos con lord Gray en el fondo
de este pasadizo. D. Paco fue tras ellas y a ninguno veo.

---Pues avancemos---dije resguardándola con mis brazos.---Ya parecerán.

Despejose algo el local con la salida de una fuerte masa de gente,
cansada ya de oír discursos, y entonces vi venir a D. Paco, como que
bajaba de la escalera de las tribunas reservadas.

---No están---decía el pobre viejo con la mayor ansiedad.---Asuncioncita
e Inesita han desaparecido. Deben de haber salido otra vez a la calle.
Lord Gray se juntó a ellas. ¡Dios mío! ¿Qué nueva tribulación es esta?
Señor de Araceli, ¿las ha visto usted?

---Subamos, que arriba han de estar.

---Que no están. ¡En buena nos han metido!\ldots{} El santo Ángel de la
Guarda me acompañe. Estas niñas me harán condenar, señor de
Araceli\ldots{} ¿Se habrán metido abajo en el salón de sesiones?

---Yo no he traído papeleta para las tribunas reservadas; pero subamos a
la pública y desde allí veremos si están.

---Yo me muero de pena---exclamó el buen profesor con lastimosos
aspavientos.---¿Dónde estarán esas dos niñas? El gentío las separó de
nosotros por casualidad\ldots{} ¿qué digo casualidad? El demonio ha
andado aquí.

---Yo subiré con esta madamita a la tribuna pública, y veremos si están
o no están aquí.

---Yo saldré a la calle\ldots{} Yo buscaré por todo el edificio; yo
volveré patas arriba Cortes y procuradores, y han de parecer, aunque se
hayan metido dentro de la campanilla del presidente o en la urna donde
se vota. ¡Qué aprieto, qué compromiso, qué situación!

Y el pobre viejo se echó a llorar como un chiquillo.

---Subamos, Sr.~de Araceli---dijo resueltamente Presentación,---que
tengo mucho deseo de ver eso.

La muchacha, en su anhelo de ver las Cortes, no se cuidaba de la pérdida
de sus compañeras.

---Suban ustedes a la tribuna pública---dijo D. Paco,---y aguárdenme
allí, que voy a preguntar a los porteros.

Presentación se aferró a mi brazo, y lejos de hacer peso en él, parecía
que me impulsaba y aligeraba, según era su impaciencia y afán de subir
pronto. Cuando llegamos arriba y entramos, no sin trabajo, en la
tribuna, la pobre muchacha mostraba en sus asombrados ojos y en el
encendido color de sus mejillas, la viva emoción que espectáculo tan
nuevo para ella le produjera. Al abarcar con la vista la iglesia-salón,
observé la tribuna de señoras, la de diplomáticos, y no vi a las dos
muchachas ni a lord Gray. Asombrado de esto, pensé retirarme para buscar
fuera; pero Presentación, arrobada y suspensa con la gravedad del
Congreso y el hablar de los diputados, me dijo deteniéndome:

---D. Paco las buscará. Yo he venido aquí para ver esto, Sr.~de Araceli.
Acompáñeme usted un momento. Mi hermana e Inés pueden parecer cuando
quieran. ¿Quién les mandó separarse?

---¿Pero no vio usted hacia qué parte fueron con lord Gray?

---No sé---repuso sin poder apartar su atención de lo que estaba
viendo.---¿Sabe usted, Sr.~de Araceli, que esto es muy bonito? Me gusta
tanto como los toros.

Traté de acomodarla en un asiento, y para esto me fue forzoso molestar a
algunas personas de las que se habían instalado allí desde el principio
de la sesión y asistían con devotísimo recogimiento a los debates.
Gruñeron unos, murmuraron otros; pero al fin Presentación obtuvo un
puesto y yo otro a su lado; pero mi inquietud y ansiedad eran tales, que
me levantaba con frecuencia para alargar el cuerpo fuera de las
barandillas con objeto de examinar todo el ámbito del salón y las
pobladas tribunas. Fáltame decir que el gentío que nos acompañaba en la
pública, era compuesto, en parte, de gente de baja esfera; y en parte,
de personas graves del comercio menudo, de tenderos, periodistas y
también muchos vagos de la calle Ancha y algunas mozas de diferente
estofa.

La iglesia, convertida en salón, no era grande. Ocupaban los diputados
el pavimento, la presidencia el presbiterio y los altares estaban
cubiertos con cortinones de damasco, que los escondían, lo mismo que a
las imágenes, de la vista del público, como objetos que no habían de
tener aplicación por el momento. El arquitecto Prast, reformador del
edificio, discurrió también sin duda que a los santos no les haría mucha
gracia aquello. Algunos han creído que los diputados subían al púlpito
para hablar; pero no es cierto. Los diputados hablaban, como hoy, desde
sus asientos; y los púlpitos no servían para nada más que para
apolillarse. Tenía la iglesia sus tribunas laterales, que fueron
destinadas a los diplomáticos, a las señoras y al público distinguido; y
en los pies del edificio abriéronse dos nuevas con barandal de madera,
que se dedicaron al pueblo en general, y que éste invadió desde las
primeras sesiones, alborotando más de lo que parecía conveniente al
decoro de su recién lograda soberanía.

Presentación no tenía ojos más que para observar la presidencia, los
diputados, y muy principalmente al que hablaba; las tribunas, los
ujieres, el dosel, el retrato del rey; ni tenía alma más que para
atender a aquellos indefinibles bullicios, propios de todo cuerpo
deliberante, y que son como el aliento de la pasión que allí por tan
diferentes órganos habla, del noble entusiasmo, del vil egoísmo; el
sordo mugir de las mil ideas, siempre desacordes, que hierven dentro de
ese cerebro calenturiento que se llama salón de sesiones. Yo observé la
estupefacción de la muchacha, y le dije:

---¿Le gusta a usted este espectáculo?

---Muchísimo. Nos habían dicho que era muy feo, pero es bonito. ¿Quién
es aquel señor que está en medio del redondel?

---Es el presidente. Es el que dirige esto.

---Ya, ya\ldots{} Y cuando quiera mandar una cosa, sacará el pañuelo y
lo agitará en el aire.

---No, señora doña Presentacioncita. Así pasa en los toros; pero aquí el
presidente se vale de una campanilla.

---Y el diputado que va a hablar, ¿por dónde sale? ¿Por detrás de
aquella cortina o por esa puertecilla?

---El diputado no sale por ninguna parte, que aquí no hay toril ni
telones. El diputado está en su asiento, y cuando quiere hablar se
levanta. Vea usted: todos esos que ahí están son diputados.

A cada nueva conquista hecha por su inteligencia en el conocimiento de
las cosas parlamentarias, más sorpresa mostraba la niña, y no distraía
su atención del Congreso sino para hacerme preguntas tan originales a
veces, y a veces tan inocentes, que me era muy difícil contestarle.
Carecía en absoluto de toda idea exacta respecto de lo que estaba
presenciando; y aquel espectáculo la conmovía hondamente, sin que las
ideas políticas tuviesen ni aun parte mínima en tal emoción, hija sólo
de la fuerte impresionabilidad de una criatura educada en estrechos
encierros y con ligaduras y cadenas, mas con poderosas alas para volar,
si algunavez rompía su esclavitud.

Era tierna, sensible, voluble, traviesa, y por efecto de la educación,
disimuladora y comedianta como pocas; pero en ocasiones tan ingenua, que
no había pliegue de su corazón que ocultase, ni escondrijo de su alma
que no descubriese. Por esto, que era sin duda efecto de un anhelo
irresistible de libertad, aparecía a veces descomedida y desenvuelta con
exceso. Poseía en alto grado el don de la fantasía; la falta de
instrucción profana unida a aquella cualidad, la hacía incurrir en
desatinos encantadores. No sólo en aquella ocasión, sino en otras
varias, observé que al separarse de doña María y al sentirse libre del
peso de aquella gran losa de la autoridad materna, desbordábanse en ella
con desenfrenada impetuosidad, fantasía, sentimiento, ideas y deseos.
Presenciando la sesión, no cabía en sí misma; tan inquieta estaba, y tan
sublevados sus nervios y tan impresionados sus sentidos.

---Señor de Araceli---me dijo después que por un instante meditó---¿y
esto para qué es?

---Las Cortes.

---Sí, eso es; quiero decir que para qué sirven las Cortes.

---Sirven para gobernar a los pueblos, juntamente con el rey.

---Comprendido, comprendido---repuso vivamente agitando su
abaniquillo.---Quiere decir que todos estos caballeros vienen aquí a
predicar, y así como los curas de las iglesias predican diciendo que
seamos buenos, los procuradores de la nación predican otras cosas; viene
la gente, los oye y nada más. Sólo que, según dicen los que van de noche
a casa, los diputados predican que seamos malos, y esto es lo que no
entiendo.

---Esos discursos---le contesté risueño,---no son sermones, son debates.

---Efectivamente; me ha parecido que no son sermones, sino que uno dice
una cosa, otro otra, y parece como que disputan.

---Justamente. Disputan; cada uno dice lo que cree más conveniente, y
después\ldots{}

---El disputar me gusta mucho. ¿Sabe usted que me estaría aquí las horas
muertas oyendo esto? Pero me agradaría que hablaran fuerte y se
insultaran, tirándose los bancos a la cabeza.

---Alguna vez\ldots{}

---Pues yo quiero venir ese día. ¿Se anunciará por carteles en las
esquinas?

---Nada de eso. La política no es una función de teatro.

---¿Y qué es la política?

---Esto.

---Ahora me parece que lo entiendo menos. Pero ¿quién es ese hombre
alto, moreno y de aspecto temeroso, que está hablando ahora? Le aseguro
a usted que ese modo de charlar me gusta.

---Es el Sr.~García Herreros, diputado por Soria.

La atención del Congreso estaba fija en el orador, uno de los más
severos y elocuentes de aquella primera fecunda hornada. Profundo
silencio reinaba en el salón lo mismo que en las tribunas. Callamos
Presentación y yo, y atendimos también, ambos absortos y suspensos,
porque la palabra de García Herreros, enérgica y sonora, era de las que
imperiosamente se hacen oír y acallan todos los rumores de una Asamblea.

Combatiendo las servidumbres, exclamaba:---«¿Qué diría de su
representante aquel pueblo numantino, que por no sufrir la servidumbre
quiso ser pábulo de la hoguera? Los padres y tiernas madres que
arrojaban a ellas a sus hijos, me juzgarían digno del honor de
representarles, si no lo sacrificase todo al ídolo de la libertad? Aún
conservo en mi pecho el calor de aquellas llamas, y él me inflama para
asegurar que el pueblo numantino no reconocerá ya más señorío que el de
la nación. Quiere ser libre y sabe el camino de serlo.»

\hypertarget{xviii}{%
\chapter{XVIII}\label{xviii}}

Ruidosos aplausos de abajo, y aplausos, patadas y gritos de arriba,
ahogaron las últimas palabras del orador. Presentación me miró, y sus
mejillas estaban inundadas de lágrimas.

---¡Oh, Sr.~de Araceli!---me dijo.---Ese hombre me ha hecho llorar. ¡Qué
hermoso es lo que ha dicho!

---Señora doña Presentacioncita, ¿no repara usted que ni su hermana, ni
Inés, ni lord Gray parecen por ningún lado?

---Ya parecerán. D. Paco ha ido a buscarlas y dará con ellas\ldots{}
Ahora está hablando otro, y dice que aquel no tiene razón. ¿Cómo
entendemos esto?

Otro orador usó de la palabra, pero por poco tiempo.

---Parece que ahora tratan de otro asunto---dijo la muchacha, observando
siempre.---Y allí se ha levantado uno que saca un papel y lo lee.

---Se me figura que ese es D. Joaquín Lorenzo Villanueva, el diputado
por Valencia.

---Es clérigo. Parece que lee un papel impreso.

---Sin duda un periódico de los que ponen como chupa de dómine a las
Cortes. Aquí acostumbran leer las picardías que los papeles públicos
dicen de los diputados, y las contestaciones que estos se sirven
dirigirles.

En efecto: Villanueva, furioso porque \emph{El Conciso} se reía de sus
proyectos de ley, lo denunciaba al Congreso Nacional, y luego nos
regalaba la contestación. Era esta una de las anomalías y rarezas de
aquella nuestra primera Asamblea, bastante inocente para detenerse en
disputar con los periódicos, dictando luego severas penas que
contradecían la libertad de la imprenta.

---Parece que va a haber tumulto---me dijo Presentación.---¡Cielos
divinos! Se levanta a hablar otro predicador\ldots{} Pero si es
Ostolaza\ldots{} ¿no le ve usted?, el mismo Ostolaza. ¿No ve usted su
cara redonda y encarnada?\ldots{} Si su voz parece una matraca\ldots{} y
¡qué gestos, qué miradas!\ldots{}

Rompió a hablar Ostolaza, y con su discurso las risas y burlas, arriba y
abajo, sin que el presidente pudiera acallarlas, ni el orador hacerse
oír con claridad. Volviose a las tribunas y con el gesto desenfadado las
despreció, y crecieron tumultos y voces, sobre todo en nuestro balcón,
donde varios individuos de sombrero gacho y marsellés no podían
convencerse de que estaban en lugar muy distinto de la plaza de Toros.

---Dice que nos desprecia---exclamó Presentación en voz muy baja.---Se
ha puesto rojo como un tomate. Amenaza a las tribunas porque nos reímos
de su facha. Sí, Sr.~Ostolaza, nos reímos de usted\ldots{} Miren el
mamarracho, espantajo. ¿Por qué no le retiran las licencias? Si es un
predicador de aldea\ldots{} Insulta a los demás. ¿Usted qué sabe, so
bruto? ¿Porque en casa le oímos con la boca abierta cuando nos sermonea,
cree que le van a tolerar aquí?\ldots{}

Un individuo de las tribunas gritó:

---¡Afuera el apagacandelas!

Y el barullo y vocerío tomaron proporciones tales que los porteros nos
amenazaron con echarnos a todos a la calle.

---Sr.~de Araceli---me dijo Presentación, encendida y agitada por el
entusiasmo,---tendría un grandísimo placer\ldots{} ¿en qué creerá usted?
Me regocijaría muchísimo\ldots{} ¿de qué pensará usted? De que ahora se
levantara de su asiento el señor presidente y le diera dos palos a
Ostolaza.

---Aquí no es costumbre que el presidente apalee a los diputados.

---¿No?---exclamó con extrañeza.---Pues debiera hacerlo. Me estaría
riendo hasta mañana: dos palos, sí señor, o mejor cuatro. Los merece.
Aborrezco a ese hombre con todo mi corazón. Él es quien aconseja a mamá
que no nos deje salir, ni hablar, ni reír, ni pestañear. Asunción dice
que es un zopenco. ¿No cree usted lo mismo?

---¡Que le den morcilla!---gritó una voz becerril en el fondo de la
galería.

---Comparito---dijo otra voz dirigiéndose al orador---¿todo ese enfao es
verdá o conversasión?

---Señores---exclamó volviéndose a todos lados, un diarista almibarado,
peli-crecido y amarillento,---estos escándalos no son propios de un
pueblo culto. Aquí se viene a oír y no a gritar.

---Camaraíta---preguntole con sorna un viejo chusco que allí cerca
había,---eso que osté ha dicho ¿es jabla o rebuzno?

---Sóplenme ese ojo---gritó otro.

---Señores, que el presidente nos va a echar a la calle y perderemos lo
mejor de la sesión.

---Señora doña Presentacioncita---dije yo a la muchacha,---bueno será
que nos marchemos. La tribuna se alborota y no es prudente seguir aquí.
Además los extraviados no parecen y debemos buscarlos fuera.

---Esperemos aún\ldots{} En suma, Sr.~D. Gabriel---me dijo con
encantadora inocencia---¿todos esos hombres para qué están aquí, para
qué hablan, para qué gritan?

Le contesté lo que me parecía y no me entendió.

---Ostolaza sigue hablando. Sus brazos parecen aspas de molino\ldots{}
Todos se ríen de él. Veo que las Cortes, como los teatros, tienen su
gracioso.

---Así es en efecto.

---Y el gracioso es Ostolaza\ldots{} Pues me parece que junto a él está
el Sr. Teneyro\ldots{} ¡Qué par! Si querrá también hablar\ldots{} Dígame
usted otra cosa, ¿quién es ese señor \emph{Preopinante} de quien todos
hablan tan mal?

---El \emph{Preopinante} es el que ha hablado antes.

---Dígame usted. Y cuando tengamos rey, ¿Su Majestad vendrá también a
predicar aquí?

---No lo creo.

---¿Y en qué consiste eso que dicen de que con Cortes hay libertad?

---Es una cosa difícil de explicar en pocas palabras.

---Pues yo lo entiendo de este modo\ldots{} Pongo por caso\ldots{} las
Cortes dirán: ordeno y mando, que todos los españoles salgan a paseo por
las tardes, y vayan una vez al mes al teatro, y se asomen al balcón
después de haber hecho sus obligaciones\ldots{} Prohíbo que las familias
recen más de un rosario completo al día\ldots{} Prohíbo que se case a
nadie contra su voluntad y que se descase a quien quiere hacerlo\ldots{}
Todo el mundo puede estar alegre siempre que no ofenda al decoro\ldots{}

---Las Cortes harán eso y mucho más.

---¡Oh, Sr.~Araceli, yo estoy muy alegre!

---¿Por qué?

---No sé por qué. Siento deseos de reír a carcajadas. Siempre que salgo
de casa, y voy a alguna parte donde puedo estar con alguna libertad, me
parece que el alma quiere salírseme del cuerpo y volar bailando y
saltando por el mundo; me embriaga la atmósfera y la luz me embelesa.
Todo cuanto veo me parece hermoso, cuanto oigo elocuente (menos lo de
Ostolaza), todos los hombres justos y buenos, todas las mujeres guapas,
y me parece que las casas, la calle, el cielo, las Cortes con su
presidente y su preopinante me saludan sonriendo. ¡Oh, qué bien estoy
aquí! Inés y Asunción no parecen, D. Paco tampoco. Cuanto más tarde
vengan mejor. Otra cosa\ldots, ¿por qué no ha seguido usted yendo a casa
por las noches? Nosotras nos hemos reído de usted.

---¿De mí?---pregunté con turbación.

---Sí, porque se la echaba usted de devoto para agradar a mamá. ¡Qué
bien hacía usted su papel! Lo mismo, lo mismito hacemos nosotras.

Me asombré de la frescura con que la infeliz niña decía claramente que
engañaba a su mamá.

---Vaya usted a casa. A nosotras no nos dejaban hablar con usted, pero
nos entretuvimos mirándole.

---¡Mirándome!

---Sí, sí: a todo el que va a casa le examinamos y le medimos las
facciones línea por línea. Después, cuando nos quedamos solas, decimos
cómo tiene el pelo, los ojos, la boca, los dientes, las orejas, y
disputamos sobre cuál de las tres se acuerda mejor.

---Bonita ocupación.

---Las tres estamos siempre juntas. La señora marquesa de Leiva está muy
enferma, y como mamá dice que quiere tener a Inés bajo su vigilancia, ha
mandado que viva en casa. Las tres dormimos en una misma alcoba y
charlamos bajito por las noches. ¡Ah! ¿Sabe usted lo que me ha dicho
Inés? Que usted está enamorado.

---¡Qué bromazo! Tal cosa no es verdad.

---Sí, nos lo dijo, y aunque no me lo dijera\ldots{} Eso se conoce.

---¿Lo conoce usted?

---Al instante. En cuanto veo a una persona.

---¿Dónde ha aprendido usted eso? ¿Lee usted novelas?

---Jamás. No las leo; pero las invento.

---Eso es peor.

---Todas las noches saco de mi cabeza una distinta.

---Las novelas inventadas son peores que las leídas, señora doña
Presentacioncita.

---Vuelva usted a casa por las noches.

---Volveré. Lord Gray las entretiene a ustedes bastante.

---Lord Gray no va tampoco---dijo con pena.

---¿Y si supiera doña María que usted ha venido aquí?

---Creo que nos mataría. Pero no lo sabrá. Inventaremos algo muy gordo.
Diremos que venimos del Carmen, donde fray Pedro Advíncula nos entretuvo
contándonos vidas de santos. Otras veces le hemos dicho esto, y luego
fray Pedro Advíncula no nos ha desmentido. Es un santo varón y yo le
quiero mucho. Tiene las manos blancas y finas, los ojos dulces, la voz
suave, el habla graciosa; sabe tocar el ole en un organito muy mono, y
cuando no está mamá delante, habla de cosas mundanas con tanta gracia
como decencia.

---¿Y fray Pedro Advíncula, va a casa de usted?

---Sí\ldots{} es amigo de lord Gray. Es el que hace la preparación
espiritual de Inés para el matrimonio, y de Asunción para el
monjío\ldots{} Se me figura (y esto es reservado) que él llevó la
papeleta de la tribuna.

---Y a usted ¿no la prepara para algo?

---A mí---contestó la muchacha con profundo desconsuelo,---a mí, para
nada.

Yo estaba absorto, pasmado y lelo, contemplando la seductora ignorancia,
la infantil malicia, la franqueza sin freno de aquella alma, a quien la
falta de toda educación mundana presentaba en la desnudez de su
inocencia. Como era linda de rostro, y había tal viveza en su hablar
espontáneo y armonioso, me encantaba verla y oírla, y como vulgarmente
se dice con respecto a los niños, me la hubiera comido. No hallo otra
frase mejor para expresar la admiración que aquel raudal de gracia y
travesura, de sentimiento y de dulce ingenuidad me producía. Nombré
antes a los niños, y aquí repito, aunque Presentacioncita había dejado
de serlo, a mí me hacía el efecto de uno de esos chiquillos
sentenciosos, que con sus verdades como puños nos causan asombro y risa.
Verdad es que la de Rumblar, aun haciéndome reír, me causaba al mismo
tiempo tristeza.

\hypertarget{xix}{%
\chapter{XIX}\label{xix}}

De pronto miré a la tribuna de señoras, que estaba al lado de la
Epístola, en lo que podemos llamar el proscenio de la iglesia, y creí
distinguir a las dos muchachas.

---¡Allí están, allí están!\ldots---dije a mi acompañante.

---Sí, y en la tribuna inmediata, que es la de los diplomáticos, está
lord Gray. ¿No le ve usted?\ldots{} Está con la cabeza entre las manos,
pensativo y meditabundo.

---No habla con ellas, ni puede hablar, porque una tabla les separa.
Acaban de entrar en este momento.

Llegó a la sazón D. Paco, rojo como un pimiento, y abriéndose paso por
entre la apiñada muchedumbre de \emph{galerios} (así llamaban a los
devotos de aquella religión, y así les nombraron después en son de
remoquete en el tiempo de las persecuciones), acercósenos y nos dijo:

---¡Gracias a Dios que han parecido!\ldots{} Lord Gray las llevó
engañadas al campanario de la iglesia\ldots{} después adentro\ldots{}
después a la calle\ldots{} ¿Hase visto infamia semejante?\ldots{} ¡Estoy
bramando de furor!\ldots{} ¿Qué habrán hecho, señor de Araceli, qué
habrán hecho?\ldots{} La señora doña Inesita estaba más pálida que una
muerta, y la señora doña Asuncioncita más roja que una amapola\ldots{}
Vámonos, niña, vámonos de aquí.

---Sí, vámonos---repetí yo.

---Yo no me muevo de aquí, Paquito. Esto me gusta mucho. Ya han acabado
de leer periódicos y papeles y vuelven los discursos\ldots{} ¿Quién
habla?

---Es el Sr.~de Argüelles. ¡Buen pájaro está! ¡Pues bonitas cosas está
oyendo la niña!---dijo D. Paco en voz más alta que la que a la
respetabilidad del sitio correspondía.---Tratar de abolir las
jurisdicciones, los señoríos, los fueros, el tormento y el derecho de
poner la horca a la entrada del pueblo, y de nombrar jueces; quieren
quitar las prestaciones y demás sabias prácticas en que consiste la
grandeza de estos reinos.

---Pues que lo supriman todo---dijo Presentación con enfado.---De aquí
no me muevo hasta que lo supriman todo.

---La niña no sabe lo que habla---exclamó D. Paco, suscitando los
murmullos de los circunstantes con lo destemplado de su voz.---Ahora la
señora doña María no podrá nombrar el alcalde de Peña-Horadada, ni
cobrará tanto de fanega en el molino de Herrumblar, ni las doce gallinas
de Baeza, ni podrá prohibir la pesca en el arroyo, ni los asnos de casa
podrán meterse en las heredades del vecino a comerse lo que se les
antoje.

---Señó abate---gritó una voz, mientras una mano pesaba con formidable
empuje sobre los hombros del preceptor;---siéntese y calle.

---Caballero---dijo otro---¿se podría saber quién es usted?

---Soy D. Francisco Xavier de Jindama---repuso con timidez y urbanidad
el viejo.

---Lo digo porque en cuanto le vi a usted y le oí, diome olor a
lechucería.

---Quiere decir que es usted de la hermandad de los bobos---añadió una
moza que frontera a D. Paco estaba.---Con su voz de matraca no nos deja
oír los escursos.

---Haya paz, señores---exclamó un tercero,---y silencio. Aquí no se
viene a lamentarse de que los asnos no puedan entrar en la heredad
ajena.

---El asno será él.

---¡Orden y conveniencia!---gritó el portero.---Si no, en nombre de Su
Majestad les echo a todos a la calle.

---Aquí no hay ninguna Majestad---dijo D. Paco.

---La Majestad son las Cortes, señor esparaván---afirmó con enfado un
galerio.

---Es de los que vienen a aplaudir cuando rebuzna Ostolaza---dijo otro
señalando a don Paco.

Viendo que la cuestión se agriaba, empeñeme en romper por medio del
gentío, y esto causó nueva confusión y reconvenciones. Al mismo tiempo
entre los diputados sonó rumor de disgusto por lo que pasaba en la
tribuna, habló el presidente imponiendo silencio a los galerios, y
acallados estos un tanto, el diputado Teneyro tomó la palabra. Como si
la primera pronunciada por el buen cura de Algeciras fuera señal
convenida, desatose una tempestad de risas y demostraciones, y cuanto
más el orador alzaba la voz, más la ahogaban entre su murmullo los de
arriba. Repetir el sinnúmero de dichos, agudezas y apodos que salieron
como avalancha de la tribuna pública, fuera imposible. Jamás actor
aborrecido o antipático recibió tan atroz silba en corrales de Madrid.
Lo extraño es que siempre pasaba lo mismo. Ya se sabía: hablar Teneyro y
alborotarse el pueblo soberano, eran una misma cosa. ¡Y qué ceceo el
suyo, qué ademanes tan graciosos, qué ira olímpica para apostrofar a las
tribunas, qué lastimoso gesto, qué cruzar de brazos, qué arrugada cara,
qué singular donaire para decir disparates, ya abogando por la
Inquisición, ya por una soberanía popular a la moda, representada por
una especie de concilio de párrocos y guerrilleros! Vamos, francamente,
era cosa de morir de risa.

El presidente sabía que sesión en la cual Teneyro hablase, era sesión
perdida, por no ser posible contener a las tribunas; trabábanse disputas
inevitables entre ciertos procuradores y el público, y el escándalo
obligaba a despejar los altos de la iglesia.

Esto ocurrió en aquel día, cuando el Cicerón de Algeciras, volviéndose
hacia arriba con ademanes descompuestos y lengua balbuciente, gritó:

---Ya sabemos que esa es gente pagada.

Al oír esto, los denuestos, los improperios que lanzó el pueblo llenaron
el ámbito de la iglesia en términos que aquello parecía una jaula de
locos. Agitábanse los diputados, echándose unos a otros la culpa del
alboroto; nos apostrofaban también desde abajo llamándonos canalla soez,
y los porteros dieron principio a la expulsión. Aquí de los apuros.
Presentación y yo queríamos salir sin poder lograrlo, por tener delante
una muralla de carne humana que resistía la orden del presidente.
Algunos se echaron fuera; mas no por eso se acalló el tumulto, y lo peor
fue que aparecieron de súbito dos o tres personas que tomaron el partido
del orador silbado contra el silbante pueblo.

---¡Que ustedes son unos servilones, mata candelas!

---¡Que ustedes son unos afrancesados!

---Que ustedes son\ldots---imagínese el lector lo peor que haya oído en
plazas, presenciado en tabernas y aprendido en garitos.

Y no paró aquí el desastre, sino que don Paco, viendo que alguien tomaba
a pechos la defensa del pobre Teneyro, arriesgose, como leal amigo y
contertulio, a ponerse de su parte.

---Envidia, no es más que envidia y rabia por las verdades como puños
que dice---exclamó.

En mal hora lo dijera. Vimos desaparecer su enjuta figura entre una masa
uniforme de brazos y manos. Presentación gritó con angustia:

---¡Que matan al pobre D. Paco!

Salió el infeliz, o lo sacaron, es decir, allá se fue todo junto,
víctima y verdugos, por la puerta afuera. Con esto se despejó un tanto
la tribuna y pudimos salir de los últimos tras la oleada de gente que
mal de su grado abandonaba la sesión. Quisimos auxiliar al maestro, pero
no nos era posible por hallarse distante; y aunque el infeliz no recibió
golpe de arma alguna, las herramientas de puños y codos le hacían mucho
daño. Al fin, acosado por todos, huyó, corriendo velozmente por la
escalera abajo, dando no pocos tumbos y costaladas.

Nuestra gran contrariedad consistía en que nos separaba de él una masa
enorme de gente que nunca acababa de salir; así es que, cuando llegamos
abajo, en vano mirábamos a todos lados. D. Paco no estaba. Hacíamos
preguntas a todos, pero nadie nos daba razón satisfactoria. Quién decía;
«le han llevado adentro;» quién «le han llevado afuera.»

---¡Qué situación, qué compromiso!---decía la muchacha.---¿Pero dónde
está el pobre don Paco? Ahora tendré que ir a casa sola o con usted.

En la calle había también apiñado gentío, entre el cual vi a uno de esos
individuos que se aparecen como llovidos en toda escena de agitación
popular, dispuestos a echar el peso, no de su autoridad, sino de sus
garrotes, en la balanza de las contiendas políticas. ¡Desgraciado
Teneyro, desgraciado Ostolaza! ¡Qué ovación les esperaba!

La hermandad de la porra no es tan antigua como el mundo, no; pero
entradilla en años es.

---Busquemos, busquemos a ese infeliz---me decía mi linda pareja.---De
modo que tengo que ir sola a casa\ldots{} ¿Y qué voy a decir?\ldots{} Y
mi hermana e Inés ¿dónde están?\ldots{} ¡Oh, señor de Araceli, más vale
que se abra la tierra y me trague!

Al fin nos dio razón del desgraciado preceptor un soldado, diciéndonos:

---Se lo llevaron entre cuatro.

---¿Pero a dónde, no se sabe a dónde?

El soldado, encogiéndose de hombros, fijó su vista en la puerta de San
Felipe, por donde salían bastantes diputados. Felizmente y gracias a la
intervención de D. Juan María Villavicencio, los que se disponían a
obsequiar a Teneyro y Ostolaza no pasaron a vías de hecho; mas con la
agudeza de sus silbidos y el mugir de sus insultos fueron dando música a
ambos personajes por largo trecho de la calle.

Fue aquel lance uno de los muchos que afearon la primera época
constitucional; pero no llegó a ser tan escandaloso como el ocurrido
poco después con motivo del famoso incidente Lardizábal, y que puso en
gran peligro la vida de D. José Pablo Valiente, diputado absolutista, el
cual hubiera sido despedazado por el pueblo si Villavicencio no le
librara heroicamente de las garras de aquel, embarcándole al instante.

---¡Virgen Santísima!---repetía Presentación.---¡Y esas niñas no
parecen!\ldots{} Vámonos al punto de aquí. Allí sale el
Sr.~Ostolaza\ldots{} Me va a conocer.

Marchamos por la calle de San José para tomar la del Jardinillo: pero no
nos fue posible esquivar las miradas y la persecución del Sr.~Ostolaza,
que llamándonos desde lejos nos obligó a detenernos.

---Señora mía---dijo el taimado clérigo,---eso está muy bien\ldots{} En
la calle con un mozalbete\ldots{} Por fuerza ha muerto la señora
condesa.

---Por Dios y la Virgen---exclamó la muchacha llorando.---Sr.~de
Ostolaza\ldots{} no diga usted nada a mamá\ldots{} Yo le explicaré a
usted\ldots{} Salimos a paseo y como nos perdiéramos, pues\ldots{} No
diga usted nada a mamá. ¡Ay! Sr.~de Ostolaza; usted es un buen sujeto y
tendrá lástima de mí.

---En efecto; siento lástima de la señorita.

---Quiero decir\ldots{} Lléveme usted a casa\ldots{} Amigo---añadió
esforzándose en aparecer jovial,---oí su discurso y me pareció muy
bonito. ¡Qué bien habla usted, qué bien!\ldots{} Da gusto\ldots{}

---Basta de lisonjas---dijo el clérigo; y luego mirándome añadió:---y
usted, señor militar-teólogo, ¿de qué arterías se ha valido para sacar
de su casa a esta señorita?

---Yo no he sacado de su casa a esta señorita---repuse;---la acompaño
porque la he encontrado sola.

---A causa del gentío nos perdimos D. Paco y yo\ldots{} quiero decir: se
perdieron ellas.

---Comprendido, comprendido.

---¿Sabe usted, señor oficial-teólogo---me dijo con aviesa mirada,---que
antes de poner esto en conocimiento de doña María voy a dar parte a la
justicia?

---¿Sabe usted---respondí,---señor clerigón entrometido, que si no se me
quita de delante ahora mismo, le enseñaré a ser comedido y a no meterse
en camisa de once varas?

---Comprendido, comprendido---repuso poniéndose como de almagre su
abominable rostro, y echándome de lleno su insolente mirada.---Sigan los
pimpollitos su camino. Adiós\ldots{}

Marchose a toda prisa y cuando le perdimos de vista, Presentación me
dijo dando un suspiro.

---Nos llamó pimpollitos y cree que somos novios, y que nos hemos
escapado\ldots{} Ahora ¿qué diré a mamá cuando me vea entrar con usted?
Necesito inventar algo muy ingenioso y bien urdido.

---Lo mejor es decir la verdad clara y desnuda. Esto ofenderá menos a la
señora que las invenciones con que usted pretenda engañarla.

---¡La verdad!\ldots{} ¿está usted loco? Yo no digo la verdad aunque me
maten\ldots{} Corramos\ldots{} ¿Habrán llegado ya las otras dos? ¡Jesús
divino! Si ellas dicen una mentira distinta de la mía\ldots{}

---Por eso lo mejor es decir la verdad.

---Eso ni pensarlo. Mamá nos mataría\ldots{} A ver qué le parece a usted
mi proyecto. Yo entraré llorando, llorando mucho.

---Malo\ldots{}

---Pues me desmayaré, diciendo que usted es un traidor que quiso
robarme.

---Peor. Diga usted que se perdieron, que encontraron a lord
Gray\ldots{}

---No nombraré al inglés; eso jamás.

---¿Por qué?

---Porque ahora, nombrar en casa a lord Gray y nombrar al demonio es lo
mismo.

---Yo sé la causa, lord Gray es amado por una de ustedes.

---¡Oh, qué cosas dice usted!---exclamó muy turbada.---Nosotras\ldots{}

---Usted.

---No; ni mi hermana tampoco.

---Sé que la señora Inesita está loca por él.

---¡Oh! Sí\ldots{} ¡loca\ldots{} loca!\ldots{} Dios mío ya
llegamos\ldots{} Estoy medio muerta.

Al entrar en la calle y acercarnos a la casa, alcé la vista y detrás del
vidrio de uno de los miradores, distinguí un bulto siniestro, después
dos ojos terribles separados por el curvo filo de una nariz aguileña,
después un rayo de indignación que partía de aquellos ojos. Presentación
vio también la fatídica imagen y estuvo a punto de desmayarse en mis
brazos.

---Mi mamá nos ha visto---dijo.---Sr.~de Araceli. Escápese usted,
sálvese usted, pues todavía es tiempo.

---Subamos, y diciendo la verdad nos salvaremos los dos.

\hypertarget{xx}{%
\chapter{XX}\label{xx}}

En el corredor Presentación cayó de rodillas ante su madre que al
encuentro nos salía, y exclamó con ahogada voz:

---Señora madre ¡perdón!, yo no he hecho nada.

---¡Qué horas son estas de venir a casa!\ldots{} ¿Y D. Paco, y las otras
dos niñas?\ldots{}

---Señora madre\ldots---continuó con aturdimiento la muchacha,---íbamos
por la muralla\ldots{} cayó una bomba, que partió en dos pedazos a D.
Paco\ldots{} no, no fue tanto\ldots{} pero corrimos, nos separamos, nos
perdimos, yo me desmayé\ldots{}

---¿Cómo es eso?---dijo la madre con furor.---Si el Sr.~de Ostolaza que
acaba de llegar, dice que te vio en la tribuna de las Cortes\ldots{}

---Eso es\ldots{} me desmayé\ldots{} me llevaron a las Cortes\ldots{}
Después mataron a D. Paco\ldots{}

---Esto debe de ser obra de alguna infame maquinación---exclamó la
condesa llevándonos a la sala.---¡Señores\ldots{} ya no hay nada
seguro\ldots{} no pueden las personas decentes salir a la calle!

En la sala estaban Ostolaza, D. Pedro del Congosto y un joven como de
treinta y cuatro años y de buena presencia, a quien yo no conocía.
Mirome el primero con penetrante encono, el segundo con altanero desdén
y el tercero con curiosidad.

---Señora---dije a la condesa,---usted se ha exaltado sin razón,
interpretando mal un hecho que en sí no tiene malicia alguna.

Y le conté lo ocurrido, disfrazando de un modo discreto los accidentes
que pudieran ser desfavorables a las pobres niñas.

---Caballero---me contestó con acrimonia,---dispénseme usted, pero no
puedo darle crédito. Yo me entenderé después con estas inconsideradas y
locas niñas; y en tanto no puedo menos de creer que usted y lord Gray
han urdido un abominable complot para turbar la paz de mi casa. Señores,
¿no hablo con razón? Estamos en una sociedad donde se hallan indefensos
y desamparados el honor de las familias y el decoro de las personas
mayores. ¡No se puede vivir! Me quejaré al gobierno, a la
Regencia\ldots{} ¡pero a qué, si todo esto proviene de las altas
regiones, donde no se alberga más que alevosía, desvergüenza, escándalo
y despreocupación!

Los tres personajes, que cual tres estatuas exornaban con simétrica
colocación el testero de la sala, movieron sus venerables cabezas con
ademán afirmativo, y alguno de ellos golpeó con la maciza mano el brazo
del sillón.

---Señor de Araceli, siento decir a usted que ya reconozco la lamentable
equivocación en que incurrí respecto al carácter de usted.

---Señora, usted puede juzgarme como guste, pero en el suceso de hoy, no
ha habido malicia por mi parte.

---Yo me vuelvo loca---repuso la señora.---Por todas partes asechanzas,
celadas, inicuos planes. No hay defensa posible; son inútiles las
precauciones; de nada sirve el aislamiento; de nada sirve el apartarse
de ese corruptor bullicio. En nuestro secreto asilo viene a buscarnos la
traidora maldad que todo lo invade y hasta en lo más recóndito penetra.

Los tres personajes dieron nuevas señales de su unánime asentimiento.

---Basta de farsas---dijo Ostolaza.---La señora doña María no necesita
que usted se disculpe ante ella, porque le conoce. ¿Cómo va de teología?

---Con la poca que sé---repuse,---cualquier sacristán podía pronunciar
en las Cortes discursos dignos de ser oídos.

---El señor es de los que van todos los días a alborotar a la tribuna.
Es un oficio con el cual viven muchos.

---¡Qué aberración! ¿Y desde tal sitio y desde tales tribunas se piensa
gobernar el reino?

---No quiero hacer aquí apologías de mi conducta---repuse con
calma,---ni las injurias de ese hombre me harán olvidar el hábito que
viste y el respeto que debo a la casa en que estoy. Aquí está una
persona que, si puede haber formado de mí juicio desfavorable en ciertas
cuestiones, conoce muy bien mis antecedentes y mi reputación como hombre
honrado. El Sr.~D. Pedro del Congosto me oye, y yo apelo a su lealtad,
para que doña María sepa si ha admitido en su casa a una persona
indigna.

Oyendo esto D. Pedro, que indolentemente se apoyaba en el respaldo del
sillón, irguiose, atusó los largos bigotes y gravemente habló de esta
manera:

---Señora, señorita y caballeros: puesto que este joven apela a mi
lealtad, probada en cien ocasiones, declaro que no una, sino muchísimas
veces he oído elogiar su buen comportamiento, su caballerosidad, su
valor como militar, con otras distinguidas prendas de paisano que le han
creado abundante número de amigos en el ejército y fuera de él.

---¡Pues qué duda tiene!---exclamó Presentación, descuidándose en
manifestar sus sentimientos.

---Calla tú, necia---dijo la madre.---Tu cuenta se ajustará después.

---Nunca---continuó el estafermo,---ha llegado a mis oídos noticia
alguna de este joven que no le sea favorable. Bien quisto de todos, ha
hecho su carrera por el mérito, no por la intriga; por el valor, no por
la astucia; y como esto es verdad, y yo lo sé, y me consta, y lo afirmo
y lo sostengo, y soy hombre que sabe sostener lo que dice, estoy
dispuesto a defenderle contra todo agravio que en este terreno se le
haga. Señora, señorita y caballeros: como hombre que ama a ese don del
cielo, esa inmaculada virgen de la verdad, que es norte de los buenos,
he dicho todo lo que puede favorecer a este joven; ahora voy a decir lo
que le desfavorece\ldots{}

Mientras D. Pedro tosía y sacaba el infinito pañuelo encarnado y azul
para limpiarse boca y narices, reinó solemne silencio en la sala y todos
me miraban con afanosa curiosidad.

---Es, pues, el caso---continuó el cruzado,---que este joven, si bajo un
aspecto es la misma virtud, bajo otro es un monstruo, señores, un
monstruo; el mayor enemigo del sosiego doméstico, el corruptor de las
familias, el terror de la pudorosa amistad\ldots{}

Nueva pausa y asombro de todos. Presentación me miraba con la mitad de
su alma en cada ojo.

---Sí; ¿qué otro nombre merece quien posee un arte infernal para romper
lazos de muy antiguo trabados entre dos personas, y que resistieran
durante veinticinco años a las asechanzas del mundo y a la persecución
de los más diestros cortejos?\ldots{} Permítanme los presentes que no
nombre personas. Básteles saber que este joven, poniendo en juego sus
malas artes amorosas, embaucó y engañó y arrastró tras sí a quien había
sido la misma firmeza, el pudor mismo y la mismísima lealtad, dejando
burlada la ideal adoración de un hombre que había sido el dechado de la
constancia y delicadeza. El desairado llora en silencio su desaire, y el
victorioso mozalbete goza sin reparo de las incomparables delicias que
puede ofrecer aquel tesoro de hermosura. Pero ¡guay!, que no es bueno
confiar en las delicias de un día; ¡guay!, que en la hora menos pensada
encontrarán uno y otro criminales amantes delante de sí la aterradora
imagen del hombre ofendido, que está dispuesto a vengar su
afrenta\ldots{} Conque díganme si el que tal ha hecho, si el que en la
difícil conquista de esa humana fortaleza, jamás antes rendida, ha
probado su travesura, ¿qué no hará dirigiéndola contra inexpertas
jovenzuelas? Abrirle las puertas de una casa es abrirlas a la liviandad,
a la seducción, a la imprudencia. Esto es todo lo que sé acerca del
Sr.~de Araceli, sin quitar ni poner cosa alguna.

Presentación estaba absorta y doña María aterrada.

---Señora, señorita y caballeros---repuse yo, no disimulando la
risa.---Al Sr.~D. Pedro del Congosto han informado mal respecto al
suceso que últimamente ha contado. Ese portento de hermosura habrá caído
en las redes de otra persona, que no en las mías.

---Yo sé lo que me digo---exclamó D. Pedro con atronadora voz,---y
basta. Denme licencia para retirarme, que avanza la hora y esta tarde he
de embarcarme con la expedición que va al Condado de Niebla a operar
contra los franceses. La ociosidad me enfada y deseo hacer algo en bien
de la patria oprimida. No tenemos gobierno, no tenemos generales; las
Cortes entregarán maniatado el reino al pícaro francés\ldots{} Sr.~de
Araceli, ¿va usted al Condado?

---No señor; guarneceré a Matagorda en todo el mes que viene\ldots{}
Pero yo también me retiro, porque la señora doña María no ve con buenos
ojos que entre en su casa.

---La verdad, Sr.~de Araceli, si hubiese sabido\ldots{} Aprecio sus
buenas prendas de militar y de caballero; pero\ldots{} Presentación,
retírate. ¿No te da vergüenza oír estas cosas?\ldots{} Pues, como decía,
deseo aclarar el punto oscurísimo del encuentro de usted en la calle con
mi hija. Aún creo que hay tribunales en España, ¿no es verdad, Sr.~D.
Tadeo Calomarde?

Esto lo dijo dirigiéndose al joven que antes he mencionado.

---Señora---repuso este desplegando para sonreír toda su boca, que era
grandísima;---a fe de jurisconsulto diré a usted que aún puede
arreglarse. Hablemos con franqueza. Estoy acostumbrado a presenciar
lances muy chuscos en mi carrera y nada me asusta. ¿Ha habido noviazgo?

---¡Jesús!, qué abominación---exclamó con indecible trastorno doña
María.---¡Noviazgo!\ldots{} Presentación, retírate al instante.

La damisela no obedeció.

---Pues si ha habido noviazgo, y los dos se quieren, y han dado un
paseíto juntos, y el señor es un buen militar, a qué andar con
farándulas y mojigatería, lo mejor es casarlos y en paz.

Doña María, de roja que estaba volviose pálida y cerró los ojos, y
respiró con fuerza, y el torbellino de su dignidad se le subió a la
cabeza, y se mareó, y estuvo a punto de caer desmayada.

---No esperaba yo tales irreverencias del Sr.~D. Tadeo Calomarde---dijo
con voz entrecortada por la ira.---El Sr.~D. Tadeo Calomarde no sabe
quién soy; el Sr. D. Tadeo Calomarde recuerda los planes casamenteros
que servían para hacer fortuna en los tiempos de Godoy. Mi dignidad no
me permite seguir este asunto. Ruego al Sr.~D. Tadeo Calomarde y al
Sr.~D. Gabriel de Araceli que se sirvan abandonar mi casa.

Calomarde y yo nos levantamos. Presentación me miró, y con toda su alma
en los ojos, me dijo en mudo lenguaje:

---Lléveme usted consigo.

Cuando nos retirábamos, entraron en la sala Inés y Asunción, conducidas
por un fraile.

---Fray Pedro Advíncula, ¿qué es esto?---dijo doña María.---¿Me
explicará usted al fin el singular suceso de la desaparición de las
niñas?

---Señora\ldots{} nada más natural---repuso jovialmente el fraile, que
era joven por más señas.---Una bomba\ldots{} ¡Pobre D. Paco!, no se ha
sabido más de él\ldots{} ¡Iban por la muralla!\ldots{} Las dos niñas
corrieron, corrieron\ldots{} pobrecitas\ldots{} Las recogimos en
casa\ldots{} se les dio agua y vino\ldots{} ¡qué susto!,
pobrecillas\ldots{} a la señora doña Presentacioncita no se la pudo
encontrar\ldots{}

---La pícara se fue a las Cortes con\ldots{} ¡Justicia, cielos divinos,
justicia!

No oí más porque salí de la casa. Desde aquel momento fui amigo de
Calomarde. ¿Hablaré de él algún día? Creo que sí.

\hypertarget{xxi}{%
\chapter{XXI}\label{xxi}}

Pasaron días y San Lorenzo de Puntales me vio ocupado en su defensa
durante un mes, en compañía de los valientes canarios de Alburquerque.
Allí ni un instante de reposo, allí ni siquiera noticias de Cádiz, allí
ni la compañía de lord Gray, ni cartas de Amaranta, ni mimos de doña
Flora, ni amenazas de D. Pedro del Congosto.

Dentro de Cádiz, el sitio era una broma y los gaditanos se reían de las
bombas. La alegre ciudad, cuyo aspecto es el de una perpetua sonrisa,
miraba desde sus murallas el vuelo de aquellos mosquitos, y aunque
picaran, los recibía con coplas donosas, como los bilbaínos de la
presente época. Cuando el bombardeo hizo verdaderos estragos, los
llantos y lágrimas perdiéronse en el bullicioso rumor de aquel hervidero
de chistes. Pero eran contadas las desgracias. Una bomba mató a un
inglés, y estuvo a punto de ser víctima de otra en los mismos brazos de
su nodriza D. Dionisio Alcalá Galiano, hijo de D. Antonio. Fuera de
estos casos y otros que no recuerdo, los efectos de la artillería
enemiga eran risibles. Un proyectil penetró en cierta iglesia,
arrancando las narices a un ángel de madera que sostenía la lámpara;
otro destrozó el lecho de un fraile de San Juan de Dios que
afortunadamente se hallaba fuera en el instante crítico.

Cuando, después de ausencia tan larga, fui a visitar a Amaranta, la
encontré desesperada, porque el aislamiento de Inés en la casa de la
calle de la Amargura, había tomado el carácter de una esclavitud
horrorosa. Cerrada la puerta a los extraños con rigor inquisitorial, era
locura aspirar ya a burlar vigilancias, y engañar suspicacias y menos a
romper la fatal clausura. La desgraciada condesa me expresó con estas
palabras sus pensamientos:

---Gabriel, no puedo vivir más tiempo en esta triste soledad. La
ausencia de lo que más amo en el mundo, y más que su ausencia, la
consideración de su desgracia, me causan un dolor inmenso. Estoy
decidida a intentar, por cualquier medio, una entrevista con mi hija, en
la cual, revelándole lo que ignora, espero conseguir que ella misma
rompa espontáneamente los hierros de su esclavitud y se decida a vivir,
a huir conmigo. No me queda ya más recurso que el de la violencia. Yo
esperé que tú me sirvieras en este negocio; pero con la necedad de tus
celos no has hecho nada. ¿No sabes cuál es mi proyecto ahora? Confiarme
a lord Gray, revelarle todo, suplicándole que me facilite lo que tanto
deseo. Ese inglés tiene una audacia sin límites, en nada repara y será
capaz de traerme aquí la casa entera con doña María dentro, cual una
cotorra en su jaula. ¿No le crees tú capaz de eso?

---De eso y de mucho más.

---Pero lord Gray no parece. Nadie sabe su paradero. Fue a la expedición
del Condado, y aunque se cree que regresó a Cádiz, no se le ve por
ninguna parte. Búscamele por Dios, Gabriel, tráemele aquí o dile de mi
parte que me interesa hablar con él de un asunto que es de vida o muerte
para mí.

Efectivamente, nadie sabía el paradero del noble inglés, aunque se
suponía que estuviese en Cádiz. Había tomado parte en la expedición que
fue al condado de Niebla con objeto de hostilizar a los franceses por su
ala derecha, y que, si menos célebre, no fue menos lastimosa que la de
Chiclana, con su célebre batallón del \emph{Cerro de la cabeza del
Puerco}. Acaeció en la jornada del Condado un suceso digno de pasar a la
historia, y fue que en ella descalabraron del modo más lamentable a
nuestro heroico y por tantos títulos famoso D. Pedro del Congosto, quien
en lo más recio de un combate que cerca de San Juan del Puerto trabaron
con los nuestros los franceses, metiose denodadamente, llevando en pos a
sus cruzados de rojo y amarillo, con lo cual dicen hubo gran risa en el
campo francés. Trajéronlo todo molido y quebrantado a Cádiz, donde decía
que por haber perdido una herradura su caballo no se ganó la batalla,
pues cuando el maldito jaco tropezó, ya empezaban a huir cual bandadas
de conejos los batallones franceses; y fija esta idea en su acalorada
mente, no cesaba de repetir: «¡Si no me hubiese faltado la
herradura!\ldots»

Lord Gray también fue al Condado, y se contaban de él maravillas; pero a
su regreso desapareció su persona de todos los sitios públicos, y aun
hubo quien le creyese muerto. Fui a su casa y el criado me dijo:

---Milord está vivo y sano, aunque no del juicio. Estuvo encerrado
quince días sin querer ver a nadie. Después me mandó que reuniese a
todos los mendigos de Cádiz, y cuando lo hice, juntolos en el comedor, y
allí les obsequió con un banquete como para reyes. Dioles a beber los
mejores vinos; los pobres, se reían unos y lloraban otros; pero todos se
emborracharon. Luego fue preciso echarles a puntapiés de la casa, y
trabajamos tres días para limpiarla, porque dejaron por fanegas las
pulgas y otra cosa peor.

---Pero ¿dónde está en este momento milord?

---Debe andar ahora allá por el Carmen.

Dirigime hacia el Carmen Calzado, cuyo gran pórtico frontero a la
Alameda, llama la atención del forastero. No es una obra maestra de los
buenos tiempos de nuestra arquitectura aquella fachada, pero los mil
accidentes con que lujosamente la adornó la imaginación del artista, le
dan cierta belleza que el mar allí cercano parece que fantasea a su
antojo. No sé por qué se me ha parecido siempre dicho frontispicio a las
popas de los grandes navíos antiguos; hasta parece que se mece
gallardamente impulsado por el viento y las olas. Los santos que lo
adornan semejan farolones gigantescos; las hornacinas troneras, los
barandajes, los nichos, las mórbidas roscas de las columnas salomónicas,
todo se me antoja como perteneciente al dominio de la antigua
arquitectura naval.

Caía la tarde. Entraban mansamente los buenos frailes, como ovejas que
vuelven al aprisco; los pobres árboles de la Alameda apenas sombreaban
el espacio que media entre el edificio y la muralla, y el sol iluminaba
el frontis, dorándolo completamente. En línea recta se extendía la
pequeña pared del convento; y en su extremo una puertecilla estrecha,
que servía de ingreso al claustro, estaba completamente obstruida por un
regular gentío que hormigueaba allí en formas oscuras y movedizas,
acompañadas de un rumor sordo o gruñido chillón, como de plebe menuda
que se impacienta. Eran los pobres que esperaban la sopa boba.

En Cádiz no han abundado tanto como en otros lugares los mendigos
haraposos y medio desnudos, esos escuadrones de gente llagada, sarnosa e
inválida que aún hoy nos sale al encuentro en ciudades de Aragón y
Castilla. Pueblo comercial de gran riqueza y cultura, Cádiz carecía de
esa lastimosa hez; pero en aquellos tiempos de guerra muchos pedigüeños
que pululaban en los caminos de Andalucía, refugiáronse en la
improvisada corte. Para que nada faltase y fuese Cádiz en tales días
compendio de la nacionalidad española, puso allí sus reales hasta la
hermandad de pan y piojos, que tanto ha figurado en nuestra historia
social, y tanto, tantísimo ha dado que hablar a propios y extranjeros.

Acerqueme a los infelices y los vi de todas clases; unos mutilados,
otros entecos, demacrados y andrajosos los más, y todos chillones,
desenfadados, resueltos, como si la mendicidad, más que la desgracia,
fuese en ellos un oficio y gozasen a falta de rentas, del fuero
inalienable y sagrado de pedir al resto del humano linaje. Salió el lego
con el calderón de bazofia, y allí era de ver cómo se empujaban y
revolvían unos contra otros, disputándose la vez, y con qué bríos y con
qué altivo lenguaje alargaban el cazuelillo. Repartía el cogulla a
diestro y siniestro golpes de cuchara, y ellos se aporreaban para
quitarse la ración, y entre manotadas y coces iban logrando la parte
correspondiente, para retirarse después a un rincón, donde pacíficamente
se lo comían.

Yo les miraba con lástima, cuando divisé en el hueco de una puerta una
figura que me hizo quedar perplejo y aturdido. No creyendo a mis ojos la
miré y remiré, sin convencerme de que era realidad lo que ante mí tenía.
El mendigo que así llamaba mi atención (pues mendigo era) vestía con los
andrajos más desgarrados, más rotos, más desordenados y extravagantes
que puede darse. Aquel vestido no era vestido, sino una informe hilacha
que se deshacía al compás de los movimientos del individuo. La capa no
era capa sino un mosaico de diversas y descoloridas telas; pero tan mal
hilvanadas que el aire se entraba por las mil puertas, ventanas y rejas,
obra de la tosca aguja. Su sombrero no era sombrero, sino un mueble
indefinido, una cosa entre plato y fuelle, entre forro y cojín vacío; y
por este estilo las demás prendas de su cuerpo anunciaban el último
grado de la miseria y abandono, cual si todas hubiesen sido recogidas
entre aquello que la misma mendicidad arroja de sí, materias que se
devuelven a la masa general de lo inorgánico, para que de nuevo tomen
forma en las revoluciones del universo.

También me causó sorpresa ver el garbo con que el hi de mala mujer se
terciaba la capita y echaba sobre la ceja el sombrerete y guiñaba el ojo
a los compañeros, y decía donaires al buen lego. Pero ¡ay!, lo que más
que traje y sombrero me asombró, dejándome lelo delante de tan
esclarecido concurso, fue la cara del mendigo, sí señores, su cara;
porque sepan ustedes que era la del mismísimo lord Gray.

\hypertarget{xxii}{%
\chapter{XXII}\label{xxii}}

Creí soñar, le miré mejor, y hasta que no me llamó saludándome, no me
atreví a hablarle, temiendo padecer una equivocación.

---No sé, milord---le dije,---si debo reírme o enfadarme de ver a un
hombre como usted, con ese traje, y llenando su escudilla en la puerta
de un convento.

---El mundo es así---me respondió.---Un día arriba y otro abajo. El
hombre debe recorrer toda la escala. Muchas veces paseando por estos
sitios, me detenía a contemplar con envidia la pobre gente que me rodea.
Su tranquilidad de espíritu, su carencia absoluta de cuidados, de
necesidades, de relaciones, de compromisos; despertaron en mí el deseo
de cambiar de estado, probando por algún tiempo la inefable satisfacción
que proporciona este eclipse de la personalidad, este verdadero sueño
social.

---Es verdad, milord, que tan descomunal extravagancia no la he visto
jamás en ningún inglés, ni en hombre nacido.

---Parece esto una aberración---me dijo.---La aberración está en usted y
en los que de ese modo piensan. Amigo, aunque parezca contradictorio, es
cierto que para ponerse encima de todo lo creado, lo mejor es bajar aquí
donde yo estoy\ldots{} Lo explicaré mejor. Yo tenía la cabeza loca del
ruido de los martillos de Londres, y venía maldiciendo la ingrata tierra
en que el hombre para poder vivir necesita hacer clavos, bisagras y
cacerolas. ¡Bendita tierra esta, donde el sol alimenta y donde lleva la
atmósfera en su inmensa masa ignoradas sustancias!\ldots{} Mi cuerpo se
rebela hace tiempo contra los repugnantes bodrios de nuestros cocineros,
inmundos envenenadores del humano linaje. Yo sentía ha tiempo profundo
rencor hacia los sastres, que serían capaces de ponerle casaquín, chupa
y corbata al Apolo de Fidias si se lo permitieran. Yo experimentaba
profunda aversión hacia las casas y ciudades, que, según vamos viendo en
nuestra graciosa época, sólo sirven para que se luzcan y diviertan los
artilleros destruyéndolas. Yo detestaba cordialmente la sociedad de los
hombres de hoy compuesta de multitud de casacas que hacen cortesías, y
dentro de las cuales suele haber la persona de un hombre. Me horrorizaba
al oír hablar de naciones, de políticas, de diferencias religiosas, de
guerras, de congresos; invenciones todas de la necedad humana que al
mismo tiempo que ha establecido leyes, estados, privilegios, dogmas, ha
inventado cañones y fusiles para destruirlo todo. Yo detestaba los
libros que se han creado para muestra de que no hay en todo el mundo dos
hombres que piensen de la misma manera, y que nacieron en manos de un
artesano, como en manos de un fraile la pólvora, otra especie de libro
que habla más alto, pero que tampoco dice nada que no sea confusión.

Lord Gray se expresaba con exaltado acento. Tomé su mano y advertí que
quemaba.

---Vi luego este país bendito, y mi pensamiento agitado descansó
contemplando esta suprema estabilidad, este profundo reposo, este sueño
benéfico de la sociedad española. Mis ojos se deleitaron contemplando en
la inmensidad de la tierra las siluetas de los grandes conventos, a cuyo
amparo protector un pueblo, a quien todo se lo dan hecho, puede esparcir
su gran fantasía por los espacios de lo soñado y buscar lo ideal en la
única región donde existe; sin cuidarse de desempeñar papeles más o
menos difíciles en la sociedad, sin cuidarse de su persona, ni de los
molestos accidentes del escenario humano, que se llaman posición,
representación, nombre, fortuna, gloria\ldots{} Quise saciar mi ardiente
anhelo de conocer este beatífico estado, y aquí me tiene usted en él.
Amigo mío, durante dos días he vivido tan lejos de la sociedad, cual si
me hubiera transportado a otro planeta; he podido apreciar la rara
hermosura de un día de sol, la pureza del ambiente, la profunda
melancolía de la noche, mar donde el pensamiento navega a su antojo sin
llegar jamás a ninguna orilla; he experimentado la indecible
satisfacción de que centenares de hombres con casaca, entorchados y
sombreros de distintas formas, pero todos más feos que los que en Egipto
ponen al buey Apis, pasen junto a mí sin saludarme; he conocido el
purísimo deleite de ver pasar los minutos, las horas, los días, cual
cortejo de dulces sombras que llevan en sus suaves manos la vida, a la
manera de aquellas deidades hermosísimas que pintaron los antiguos,
transportando en sus brazos las almas de los justos al cielo; he
saboreado las delicias de no ir a ninguna parte deliberadamente, de
sentir mis hombros libres de toda obligación, de no sentir en mi
pensamiento ese hierro candente cuya quemadura significamos en el
lenguaje con la palabra después, y que encierra un mundo de deberes, de
ocupaciones, de molestias sin fin.

Después de una breve pausa, prosiguió así:

---Esta gente que me rodea tiene las mismas pasiones que las de allá
arriba; pero no disimula nada. Es una ventaja. Prendas diversas les
caracterizan, pero aquí todo es abrupto y primitivo como las rocas,
donde no ha golpeado aún el martillo del hombre para labrar un camino.
Los hay más crueles que Glocester, más mentirosos que Walpole, más
orgullosos que Cromwell, más poetas que Shakespeare, y casi todos son
ladrones. Yo me deleito con la salvaje manifestación de sus pasiones y
me finjo ignorante de sus truhanerías. Aquel viejo que allí se ve
haciendo cruces encima de la escudilla, me ha robado todos los doblones
de oro que yo llevaba en mi bolsillo. Juntos pasábamos largas horas por
las noches en la muralla. Él me contaba vidas de santos españoles; yo
fingía dormitar, embelesado por los místicos encantos de su relato, y
entonces metía bonitamente sus manos en mi bolsillo para sacarme el
dinero. Yo lo observaba y callaba, gozándome en su avariciosa
concupiscencia, como se goza viendo un abismo, una tempestad, un
incendio o cualquier aparente desorden de la naturaleza. Aquellos
gitanos que están allí rezando el rosario, me han entretenido dulcemente
contándome sus ingeniosas maneras de robar. »Amigo mío; aquí también hay
una especie de alta sociedad, y se pasa el rato alegremente en
conciertos, fiestas y representaciones. Los romances moriscos que recita
aquella vieja que parece exacto traslado de la tía Fingida, y en efecto
lo es, han producido en mí mayor sensación que las fanfarronadas de
todos los cómicos modernos. Hay allí una muchacha ciega, a quien llaman
la Tiñosa, la cual canta el jaleo y el ole con tanto primor, que
oyéndola he sentido emociones dulcísimas y me he trasportado a las
últimas, a las más remotas regiones de lo ideal. Aquellos niños cojos y
mancos, en cuyos grandes ojos negros parece centellear el genio del gran
pueblo que guerreó durante siete siglos con los moros y descubrió,
conquistó y dominó regiones y continentes hasta que ya no había más
mundo para saciar su ambición, aquellos niños, digo, son la más graciosa
pareja de pilletes que he visto en mi vida, y cuanta sal, ingenio y
travesura ha derramado la Naturaleza en granujas de Madrid, léperos de
Méjico, lazzaronis de Nápoles, lipendes de Andalucía, pilluelos de
París, \emph{pic-pockets} de Londres, es nada en comparación de su gran
ciencia. Si les educaran, es decir, si les corrompieran torciendo el
natural curso de sus instintos, yo quisiera ver dónde se quedaban Pitt,
Talleyrand, Bonaparte, y todos los grandes políticos de la época.

---Amigo---le dije sin poder reprimir mi enfado,---me da compasión verle
a usted entre esta desgraciada gente, y más aún oírle encomiar su triste
estado.

---No parece sino que nosotros somos mejores que ellos. ¡Ah! Desde que
hay en España filósofos y políticos charlatanes y escritores con pujos
de estadista, se ha empezado a declarar ominosa guerra a estos mis
buenos amigos, lo mismo que a los salteadores de caminos, que no son
otra cosa que una protesta viva contra los privilegios de los
cosecheros; a los buenos frailes que son la piedra fundamental de esta
armonía envidiable, de este sistema benéfico, en que todos viven
modestamente sin molestarse unos a otros.

Esto decía cuando una vieja que acababa de llenar la escudilla, llegose
a nosotros y después de pedirme una limosna, que le di, puso la
descarnada mano sobre el hombro del par de Inglaterra y cariñosamente le
dijo:

---Niñito querido, ¡qué buenas nuevas te traigo esta tarde! Alégrate,
picarón, y escupe otra moneda amarilla, otro pedazo de sol como el que
ayer me diste en premio de mis desinteresados servicios.

---¿Qué me cuentas, tía Alacrana, espejo de las busconas?

---A mí no se me han de decir esos feos vocablos. ¿Pues qué? ¿Acaso en
mi vida he hecho algo que tenga olor de alcahuetería? Aquí donde me ven,
yo, doña Eufrasia de Hinestrosa y Membrilleja soy muy principal y mi
difunto fue empleado en la renta del noveno y el excusado. Pero vamos a
lo que importa.

---¿Fuiste allá, brujita mía?

---Por sétima vez. ¡Y qué buena que es mi doña María! Hemos brindado
juntas muchos \emph{paternoster}, a modo de copas de vino, en esta
iglesia del Carmen y en obsequio de nuestros respectivos difuntos.
Señora más enseñorada no la hay en todo Cádiz. En generosidad no, pero
en principalidad se monta por encima de cuanta gente conozco, que es
medio mundo. Me da algunos ochavos y lo que sobra de la olla que es
(dicho sea sin incurrir en el feo vicio de la murmuración) bien poco
sustanciosa. Me ha comprado algunas crucecitas de los padres
mendicantes, y huesecillos benditos para hacer rosarios. Hoy le llevé mi
comercio y la noble señora hizo que le contara mi historia; y como esta
es de las más patéticas y conmovedoras, lloró un tantico. Después, como
ella saliera de la sala para ir a sus quehaceres, quedeme sola con las
tres niñas, y allí de las mías. En cuarenta años de piadoso ejercicio en
este ajetreo de ablandar muchachas, avivar inclinaciones, y hacer el
recado, ¿qué no habré aprendido, niñito mío, qué trazas no tendré, qué
maquinaciones no inventaré, y qué sutilezas no me serán tan familiares
como los dedos de la mano? Así es que si me hallo con bríos para
pegársela al mismo Satanás, de quien estos pícaros dicen que soy sobrina
carnal, ¿cómo no he de poder pegársela a doña María, que aunque
principalota, se deja embobar por un credo bien rezado y por una parla
sobre la gente antigua, siempre que cuide uno de adornar el rostro con
dos lagrimones, de cruzar las manos y mirar al techo, diciendo: «¡Señor,
líbranos de las maldades y vicios de estos modernos tiempos!»

---Tu charlatanería me enfada, Alacrana. ¿Qué recado me traes?

---¿Qué recado? Tres días de santa conferencia he empleado, mi niño.
¿Qué ha de hacer la pobrecita? Creo que está dispuesta a echarse fuera y
huir contigo a donde quieras llevarla. Para entrar en la casa y en el
sagrado tabernáculo de su alcoba, ya tienes las llavecitas que has
forjado, gracias al molde de cera que te traje. ¡Oh, dichoso, mil veces
dichoso niño! Ya sabes que la doña María duerme en aquella alcobaza de
la derecha y las tres niñas en un cuarto interior. La sala y dos piezas
más separan un dormitorio de otro: no hay peligro ninguno.

---¿Pero no te ha dado recado escrito o de palabra?

---Me lo ha dado, sí señor; a fe que es la niña poco cortés para no
contestarte. En esta hoja de libro que aquí traigo, marca, apunta y
especifica el día, hora y punto en que caerá en los brazos de este
haraposo la más\ldots{}

---Calla y dame.

---Paciencia. Hoy me ha dicho doña María que tiene un dormir tan
profundo como el de los muertos. Eso prueba una conciencia tranquila.
¡Dios la bendiga!\ldots{} Ahora, para darte el documento, deja caer
sobre mí el rocío de esas monedas de oro que me fueron prometidas.

Lord Gray dio algunas monedas a la vieja, recogiendo luego un papel que
guardó en el seno. Después se levantó, dispuesto a partir conmigo.

---Vámonos---le dije,---o estrangulo a esa maldita bruja.

---Es una respetable señora esta doña Eufrasia---me contestó con
ironía.---Admirable tipo que hace revivir a mi lado la incomparable
tragicomedia de Rodrigo Cota y Fernando de Rojas.

Y luego, volviéndose hacia la miserable turba, con voz entre grave y
burlona, le dijo:

---Adiós España; adiós soldados de Flandes, conquistadores de Europa y
América, cenizas animadas de una gente que tenía el fuego por alma y se
ha quemado en su propio calor; adiós, poetas, héroes y autores del
Romancero; adiós, pícaros redomados que ilustrasteis, Almadrabas de
Tarifa, Triana de Sevilla, Potro de Córdoba, Vistillas de Madrid,
Azoguejo de Segovia, Mantería de Valladolid, Perchel de Málaga,
Zocodover de Toledo, Coso de Zaragoza, Zacatín de Granada y los demás
que no recuerdo del mapa de la picaresca. Adiós, holgazanes que en un
siglo habéis cansado a la historia. Adiós, mendigos, aventureros,
devotos, que vestís con harapos el cuerpo y con púrpura y oro la
fantasía. Vosotros habéis dado al mundo más poesía y más ideas que
Inglaterra clavos, calderos, medias de lana y gorros de algodón. Adiós,
gente grave y orgullosa, traviesa y jovial, fecunda en artificios y
trazas, tan pronto sublime como vil, llena de imaginación, de dignidad,
y con más chispa en la mollera que lumbre tiene en su masa el sol. De
vuestra pasta se han hecho santos, guerreros, poetas y mil hombres
eminentes. ¿Es esta una masa podrida que no sirve ya para nada? ¿Debéis
desaparecer para siempre, dejando el puesto a otra cosa mejor, o sois
capaces de echar fuera la levadura picaresca, oh nobles descendientes de
Guzmán de Alfarache?\ldots{} Adiós, Sr.~Monipodio, Celestina, Garduña,
Justina, Estebanillo, Lázaro, adiós.

Indudablemente lord Gray estaba loco. Yo no pude menos de reír oyéndole,
en lo cual me imitaron los pilletes a quienes se dirigía, y pensé que
las ideas expresadas por él eran frecuentes entre los extranjeros que
venían a España. Si eran exactas o no, mis lectores lo sabrán.

---Amigo---me dijo el lord,---uno de los placeres más halagüeños de mi
vida es pasar largas horas entre las ruinas.

Marchábamos despacio por la muralla adelante hacia las Barquillas de
Lope, cuando encontramos a dos padres del Carmen que volvían
apresuradamente a su casa.

---Adiós, Sr.~Advíncula---dijo lord Gray.

---¡San Simeón bendito!---exclamó perplejo uno de los frailes.---¡Es
milord! ¡Quién le había de conocer en semejante traje!

Uno y otro carmelita rieron a carcajada tendida.

---Voy a soltar el manto real.

---Creíamos que milord se había marchado a Inglaterra.

---Y me alegré, sí señor me alegré---dijo el más joven,---porque no
quiero compromisos, y milord me está comprometiendo. Acabáronse las
condescendencias peligrosas.

---Bueno---dijo Gray con desdén.

El más anciano preguntó:

---¿Entró al fin milord en el seno de la iglesia católica?

---¿Para qué?

---Ese traje---dijo fray Pedro Advíncula con sorna,---indica que milord
se prepara a ello con dolorosas penitencias\ldots{} Veo que ahora usted
se las arregla usted por sí mismo, y que no necesita amigos.

---Sr.~Advíncula, ya no los necesito. ¿Sabe usted que mañana me marcho?

---¿Sí? ¿Para dónde?

---Para Malta. Nada tengo que hacer en Cádiz. Vayan al diablo los
gaditanos.

---Me alegro. La señora se defiende bien. Su casa es una fortaleza a
prueba de galanes. ¿Sabe usted que lo ha hecho por consejo mío?

---¡Picarón!\ldots{}

---¿De veras que ya no hay nada?

---Nada.

---Es una determinación acertada. Hágase usted católico y le prometo
arreglarlo todo.

---Ya es tarde.

Advíncula rió de muy buena gana, y apretando las manos al lord, ambos
frailes se despidieron de él con cariñosas demostraciones,

\hypertarget{xxiii}{%
\chapter{XXIII}\label{xxiii}}

Dos horas después, lord Gray estaba en el salón de su casa, vestido como
de costumbre, después de haber borrado con abundantes abluciones la
huella de sus barrabasadas picarescas.

Vestido al fin con la elegancia y el lujo que le eran comunes, mandó que
pusiesen la cena, y en tanto que venían dos personas a quienes dirigió
verbal invitación por conducto de sus criados, paseábase muy agitado en
la larga estancia. A ratos me dirigía algunas palabras, preguntas
incongruentes y sin sentido; a ratos se sentaba junto a mí como
intentando hablarme, pero sin decir nada.

Como el oro improvisa maravillas en la casa del rico, la mesa (sólo
había en ella cuatro cubiertos) ofrecía esplendidez portentosa.
Centenares de luces brillaban en dorados candelabros, reflejándose en
mil chispas de varios colores sobre los vasos tallados y los vistosos
jarros llenos de flores y frutas. El mismo desorden que allí había, como
en todo lo perteneciente a lord Gray, hacía más deslumbradora la extraña
perspectiva del preparado festín.

Al fin, mostrando impaciencia, dijo el inglés:

---Ya no pueden tardar.

---¿Los amigos?

---Son amigas. Dos muchachas.

---¿Las que dan quehacer a la señora Alacrana?

---Araceli---dijo con inquietud---¿usted oyó el coloquio que conmigo
tuvo aquella mujer?\ldots{} Es una indiscreción. Los buenos amigos
cierran los oídos al susurro de lo que no les importa.

---Yo estaba tan cerca, y la señora Alacrana se cuidaba tan poco de la
presencia de un extraño, que no pude cerrar los oídos. Milord, lo oí
todo.

---Pues muy mal, muy mal---exclamó con acritud.---Todo aquel que se
jacte de conocer lo que yo quiero ocultar hasta de Dios, es mi enemigo.
¿No he dicho lo mismo otra vez?

---Entonces reñiremos, lord Gray.

---Reñiremos.

---¿Por tan poca cosa?---dije afectando buen humor, pues no me convenía
chocar con él en ocasión tan inoportuna.---Yo soy el más discreto y
prudente de los hombres. Usted mismo me ha puesto al corriente de sus
aventuras. Vamos, amigo mío, seamos francos. ¿No me dijo usted mismo que
pensaba llevársela a Malta?

Lord Gray sonrió.

---Yo no he dicho eso---exclamó vacilando.

---Usted\ldots{} usted mismo. Y yo prometí ayudarle en la empresa, a
cambio de su auxilio para matar a mi aborrecido rival Currito Báez.

---Es verdad---dijo riendo.---Bien, amigo mío. Mataremos a Currito y
robaremos a la muchacha. En caso de que necesite ayuda ¿puedo contar con
usted?

---Sin duda. Sólo me falta saber para cuándo se dispone el gran golpe.

---¿Qué golpe?

---El del rapto.

Lord Gray meditó largo rato. Sin duda vacilaba en fiarse de mí.

---Para el rapto no necesito de nadie---dijo al fin.---Necesitaré sí
para huir de Cádiz, lo cual no es cosa fácil.

---Yo sacaré a usted del apuro. Sepamos cuándo\ldots{}

---¿Cuándo?

---Para ayudar a usted necesito pedir licencia con anticipación.

---Es verdad. Pues bien. Antes me arrancaré la lengua que revelarle a
usted todavía el lugar y la persona\ldots{}

---Ni yo quiero saberlo: lo que me importa es la hora\ldots{}

---Es cierto\ldots{} Bien; repito que ni lugar ni persona los sabrá
usted. Diré únicamente\ldots{}

Sacó un papel que reconocí como el mismo que le entregara la Alacrana, y
añadió:

---Este papel fija día y hora. Será mañana por la noche.

---Basta. Es todo lo que necesito saber. Mañana por la noche.

---Lo demás no lo diré ni a mi sombra. Temo traiciones y emboscadas y
desconfío hasta de mis mejores amigos.

---Ni yo quiero ser indiscreto preguntando\ldots{} No me importa. Me
basta saber que mañana a la noche tengo que venir a Cádiz para ponerme a
disposición de un amigo a quien estimo mucho.

Yo pensé que lord Gray escondería de mis ojos el papel que tan extraños
avisos traía para él, pero con gran sorpresa mía, me lo mostró. Era una
hoja de un libro, en cuyo margen había algunas rayas con lápiz.

---¿Esta es la carta? A fe que no puedo entender lo que dice, ni es
fácil conocer el carácter de la escritura.

---Yo lo entiendo bien\ldots{} Estas rayas se refieren a determinadas
letras de los renglones impresos y con un poco de paciencia se descifra.
Pero me parece que sabe usted bastante. Silencio, pues, y no se nombre
más este asunto. Me mortifica, me pone nervioso y colérico el ver que
hay alguien que posee una parte de mi secreto. Ahora no pensemos más que
en Currito Báez. Amigo, siento deseo irresistible, anhelo profundo de
matar a un hombre.

---Yo también.

---¿Cuándo le despachamos?

---Mañana por la noche se lo diré a usted.

---¿Quiere usted que le ejercite un poco en la esgrima?

---Nada más oportuno. Vengan los floretes. Espero adquirir de aquí a
mañana tanta destreza como mi maestro.

Empezamos a tirar.

---¡Oh, qué fuerte está usted, amigo!---dijo al recibir una estocada
medianilla.

---No estoy mal, no.

---¡Pobre Currito Báez!

---Sí. ¡Pobre Currito Báez! Mañana veremos.

Sonó en la escalera gran estrépito, suspendimos al punto el juego,
permaneciendo con los floretes en la mano en actitud observadora, y he
aquí que entran metiendo ruido y cual brazos de mar que todo lo arrollan
e inundan delante de sí, dos mozas de lo mejor que puede criar
Andalucía. ¿Las conocéis? Eran María Encarnación llamada la Churriana y
Pepilla la Poenca, a quien nombraban así por ser sobrina del Sr.~Poenco.

---¡Endinote!---exclamó una corriendo ligerísima hacia mi amigo.---¿Cómo
tanto tiempo sin verte? ¿No sabías que esta probe se estaba muriendo?

---Miloro está encalabrinao por aquí dentro, y ya no quiere nada con la
gente de la Viña.

---Amable canalla---dijo el inglés,---sentaos. Sentaos y cenemos.

Los cuatro tomamos asiento y no pasó después nada digno de contarse, por
lo cual me abstengo de quitar espacio y atención a asuntos de mayor
importancia.

\hypertarget{xxiv}{%
\chapter{XXIV}\label{xxiv}}

D. Diego de Rumblar fue a despertarme a mi alojamiento en la tarde del
siguiente día. No habiendo podido dormir en la noche, había pasado en
calenturientos sueños parte del día, y me hallaba al despertar afectado
de gran postración. Mi alma llena de tristeza se abatía, incapaz del
menor vuelo, y encontrándose inferior a sí misma, hasta parecía perder
aquella antigua pena que le producían sus propias faltas, y se adormecía
en torpe indiferencia. Tolerante con los errores, con los extravíos, con
el mismo vicio, iba degradándose de hora en hora. D. Diego me dijo:

---Te participo que el sábado de esta semana tendrán lugar en casa dos
acontecimientos. Yo me caso y mi hermana entrará de novicia en las
Capuchinas de Cádiz.

---Lo celebro.

---Ya he perdido aquellos escrúpulos, hijos de una delicadeza excesiva y
ridícula. Mi mamá me dice que soy un asno si al punto no me decido.

---Tiene razón.

---Además, chico, has de saber que mi mamá me ha sitiado por hambre.

---¡Por hambre!

---Sí, hombre. Asegura que nuestra fortuna está por los suelos a causa
de la guerra, y luego añade: «Como no te cases, hijo, ¡no sé cómo
podremos vivir!» A todas estas ni un real para mis gastos. Eminente
joven, gloria de la patria, si le prestaras cuatro duros al señor conde
de Rumblar, Europa entera te lo agradecería.

Le di los cuatro duros.

---Gracias, gracias, benemérito soldado. Te los pagaré cuando me case.
Dime, ¿no te parece que hago bien en desechar vanos escrúpulos?

---¿Eso qué duda tiene?

---Lord Gray no ha vuelto por casa; nadie sabe dónde está, y es
probable, que haya marchado a Inglaterra.

---Creo que en efecto se ha marchado a su país.

---Te advierto que mi novia no me puede ver ni pintado; pero eso no hace
al caso. Mi madre me ha bloqueado por mar y tierra, y yo me rindo,
chico, me rindo a discreción. Con mi señora mamá no hay burlas,
amiguito. Si vieras qué coscorrones me da\ldots{} He tenido que hacer
llaves nuevas para poder salir de noche. Pues ¿y mis hermanitas y mi
novia? Hace lo menos dos meses que no saben de qué color es la calle. Ni
siquiera salen a misa; en paseos no hay que pensar. Han sido clavados
por dentro los cristales de los balcones, y no se les permite que tengan
a la mano papel, tinta ni plumas. Las tres infelices están que da
lástima verlas de marchitas y acongojadas, y de seguro preferirían la
peor vida del mundo a la que ahora llevan, aguantando con gusto palos de
marido o rigores de abadesa, con tal de abandonar las sombrías mazmorras
de mi casa. No ven a otros hombres que a mí y a D. Paco. ¿Te parece que
estarán divertidas?

---¿Usted sale por las noches de su casa?

---Sí: ¿no sabes que ahora voy todas las noches a una reunión de hombres
solos donde se trata de política? ¡Encantadora, deliciosa es la
política! Pues te diré: nos juntamos en una casa de la calle de la
Santísima Trinidad y allí estamos horas y más horas hablando de la
democracia y del servilismo, diciendo perrerías de los frailes
escribiendo a trozos el graciosísimo papel satírico que se llama el
\emph{Duende de los Cafés}. Nos ocupamos de la vida y milagros de todo
\emph{quisque}, y criticamos sin piedad. Pero lo más salado es aquella
parte en la cual con mucho donaire nos burlamos de los clérigos, de la
Inquisición, del Papa, de la santa Iglesia y del Concilio de Trento.
Átame esa mosca\ldots{}

---Por fuerza anda en ese lío el gran Gallardo.

---Si mi madre supiera esto, me colgaría del techo de la sala, ya que no
tenemos almenas en que hacer conmigo un escarmiento. Vamos ahora a la
tertulia. También nos reunimos de día. Hoy van a leer un folleto que ha
escrito uno en contestación al \emph{Diccionario manual para
inteligencia de ciertos escritores que por equivocación han nacido en
España}. ¿Conoces ese librito? Es una sarta de necedades. Ostolaza lo ha
llevado a casa, y por las noches él, el Sr.~Teneyro y mamá lo leen y
celebran mucho sus sandios chistes y groserías. Verás el que va a salir
en contestación.

---Por pasar el rato iremos allá---dije disponiéndome a salir.

---Esta noche---añadió,---iremos a casa de Poenco. Te convido a echar
unas copas\ldots{}

---Magnífica idea. Cuando la señora doña María duerma sale usted, se
mete la llave en el bolsillo, y a casa de Poenco\ldots{} Pasaremos una
buena noche. Sé que estarán allí María Encarnación y Pepilla y la
Poenca.

---Me chupo los dedos, amigo Araceli, con la noticia. Allá voy de
cabeza. Mi señora madre duerme como una piedra, y no advierte mis
escapatorias.

---Pero lo advertirán las hermanitas.

---Ellas lo saben, y me impulsan a salir para que les cuente lo que
ocurre por ahí durante la noche. También voy al teatro. Las pobrecitas
llevan una vida\ldots{} Como duermen juntas las tres en una misma
alcoba, se entretienen de noche contándose historias en voz baja.

Llegamos a la calle de la Santísima Trinidad y en un cuarto bajo, oscuro
y humildísimo, había hasta dos docenas de personas de diferentes edades,
aunque abundaban más que los viejos los jóvenes, todos alegres y
bulliciosos, como grey estudiantil, vestidos de voluntarios los unos y
con sotana un par de ellos, si no estoy trascordado. Describir la
confusión y bulla que allí reinaba fuera imposible; pintar la variedad
de sus fachas, la movilidad de sus gestos y la comezón de hablar y reír
que les poseía, fuera prolijo. Unos se sentaban en desvencijadas sillas,
otros de pie sobre las mesas haciendo de estas tribuna, se adiestraban
en el ejercicio parlamentario; algunos disputaban furiosamente en los
rincones, y no faltaba quien en las rodillas o sobre el breve espacio de
mesa que dejaban libre los pies de los oradores, emborronaba cuartillas.
Era aquello un nido, una hechura de políticos, de periodistas, de
tribunos, de agitadores, de ministros, y daba gusto ver con cuánto
donaire rompían el cascarón los traviesos polluelos.

Aquello era club incipiente, redacción de periódico, academia
parlamentaria, todo esto, y algo más. ¡Qué hervidero! ¡Cuántas pasiones,
cuántas crisis, cuántas revoluciones, cuánta historia, en fin, bullían
dentro daaquel pastel que acababa de ponerse al fuego! Los huevecillos
que deposita la mariposa para dar vida al gusano no se abren, no echan
fuera la diminuta criatura, ni esta se desarrolla con más presteza al
calor de la primavera que aquellos inocentes embriones de gente
política. Su precocidad asombraba, y oyéndoles hablar, se les creía
capaces de dar guerra al universo entero.

Al punto D. Diego y yo fuimos tratados como antiguos amigos.

---Ahora va a venir ese insigne bibliotecario de las Cortes---dijo
uno,---y nos acabará de leer su obra.

---Ya veo cómo tiemblan los frailes panzudos y los rollizos canónigos.
Yo he dicho que debe grabarse letra por letra con oro y plata en las
esquinas de las calles.

---¡Aquí está, aquí está el insigne Gallardo!

Era altísimo, flaco, desgarbado, amarillento, siendo de notar en su
rostro la viveza de los ojos así como la regular longitud de las
abanicadas orejas. ¡Singular hombre! Cincuenta años después le habéis
visto en las calles de Madrid desfigurado por el medio siglo; pero
siempre distinguiéndose muy bien por la prolongación longitudinal de su
persona; le habréis visto siempre flaco, siempre amarillo, pero antes
atrabiliario que jovial, marchando aprisa con los bolsillos de un como
redingotgris llenos de libros viejos, con su sombrero de hule hecho a
las injurias de aguas y soles; y si por acaso dirigisteis vuestros pasos
a la Alberquilla, dehesa próxima a Toledo, le veríais allí sepultado en
una biblioteca, donde le devoraba, como a D. Quijote la caballería, la
estupenda locura de los apuntes; le veríais encerrado semanas enteras,
sin tomar otro alimento que el modestísimo de una diaria ración de sopas
de leche. Algo había en aquella cabeza, para ofrecer el fenómeno de que
sabiendo cuanto había que saber en materia de libros, y siendo el
almacén de apuntes y datos y noticias más colosal que ha existido en el
mundo, jamás hiciese cosa de provecho.

Pero ustedes no conocieron a Gallardo como yo le conocí, en la plenitud
de su frenesí clerofóbico; ustedes no le oyeron leer como yo las
célebres páginas del \emph{Diccionario burlesco}, el libro más atroz y
más insolente que contra la religión y los religiosos se había escrito
en España. Estaba poseído de un estro impío, y fue la primera musa de
esa gárrula poesía progresista que durante muchos años atontó a la
juventud, persuadiéndola de que la libertad consiste en matar curas.

---¡A leer, a leer!---gritaron seis o siete voces.

---¿Has acabado el párrafo del \emph{Cristianismo}?

---Calma y no me vuelvan loco---dijo Gallardo sacando unos
papelotes.---No se puede ir tan aprisa.

---Si estás a la mitad, insigne bibliotecario, habrás llegado al
parrafillo de la \emph{Inquisición} que caerá en la I.

---No, porque pongo la Inquisición en la \emph{y griega}.

Grandes y estrepitosas y retumbantes risas.

---Atended un poco. A ver qué os parece esto de la Constitución---dijo
sentándose, mientras se formaba corrillo en torno suyo.---Ya sabéis que
el asno hilvanador del \emph{Diccionario manual} decía que la
Constitución será una \emph{taracea de párrafos de Condillac cosidos con
hilo gordo}\ldots{} Pero mirad antes cómo defino el Cristianismo. Digo
así: «Amor ardiente a las rentas, honores y mandos de la Iglesia de
Cristo. Los que poseen este amor saben unir todos los extremos y atar
todos los cabos, y son tan diestros que a fuerza de amor a la esposa de
Jesucristo, han logrado tener a su disposición dos tesorerías, que son
las del \emph{arca boba} de la corte de España y la de los tesoreros de
las gracias de la corte de Roma.» Ya veis que he parafraseado lo que
dijo el Manual en el párrafo del \emph{Patriotismo}.

---Bartolillo---preguntó uno,---¿y no le has contestado nada a aquello
de que el alma es un \emph{huesecillo o ternilla que hay en el celebro,
o según otros en el diafragma, colocado así como el palitroquillo que se
pone dentro de los violines? }

---Paciencia. Allá va lo que pongo a la voz \emph{Fanatismo}\ldots{}
«Enfermedad físico-moral, cruel y desesperada, porque los que la padecen
aborrecen más la medicina que la enfermedad. Es una como rabia canina
que abrasa las entrañas, especialmente a los que arrastran holapandas.
Los síntomas son bascas, convulsión, delirio, frenesí; en su último
período degenera en licantropía y misantropía, en cuyo estado el enfermo
se siente con arranques de hacer una gran hoguera para quemar a medio
linaje humano.»

---Eso está bien dicho; pero algo frío, Bartolo.

---Duro, más duro en ellos. Veamos cómo te desenvuelves en la voz
\emph{Fraile}.

---\emph{Frailes}\ldots{} Atención---continuó el lector.---Una especie
de animales viles y despreciables que viven en la sociedad a costa de
los sudores del vecino en una especie de café-fonda, donde se entregan a
todo género de placeres y deleites, sin más que hacer que rascarse la
barriga.

Aquí no pudieron contener los mozalbetes su entusiasmo, y fue tal la
algazara y el jaleo de pies y manos, que los transeúntes se detenían en
la calle sorprendidos por el estentóreo ruido.

---Vaya, señores, que no leo más---dijo Gallardo guardando sus papeles
con orgullo.---Esto va a perder la novedad cuando se publique.

---Bartolo, echa el \emph{Obispo}.

---Bartolo, léenos el \emph{Papa}.

---Eso se quedará para mañana.

---Ya andan por ahí los Zampatortas con la cabeza inclinada como higo
maduro desde que saben va a salir tu \emph{Diccionario}.

---Bartolo, ¿escribes hoy algo contra Lardizábal?

Lardizábal, individuo de la Regencia que había dejado de funcionar el
año anterior, publicó en aquellos días un tremendo folleto contra las
Cortes.

---¿Yo? Jamás le he echado paja ni cebada al señor Lardizábal.

---Hombre, defendamos la soberanía de la nación.

---Si no tiene más enemigos que Lardizábal\ldots{} Sopla, y vivo te lo
doy\ldots{}

---Mañana saldrá bueno nuestro \emph{Duende}.

---Cuando sea diputado---dijo uno que por lo enteco parecía
sietemesino---pediré que todos los frailes que hay en España sean
destinados a dar vueltas a las norias para sacar agua.

---De ese modo se regará muy bien la Mancha.

---Señores, no olvidarse de que mañana habla Ostolaza y quizás D. José
Pablo Valiente.

---Hay que ir a la tribuna.

---Yo esperaré en la calle para ver la función de salida.

---Eh\ldots{} Antonio, échanos un discurso.

---Un discurso como el de anoche, y sobre el mismo tema de la
democracia.

---Pero no digas, como el \emph{Diccionario manual}, que la democracia
«es una especie de guarda-ropa en donde se amontonan confusamente
medias, polainas, botas, zapatos, calzones y chupas, con fraques,
levitas y chaquetas, casacas, sortúes y capotes ridículos, sombreros
redondos y tricornios, manteos y unos \emph{monstruos de la naturaleza
que se llaman abates.»}

---De ese modo ha querido pintar a las Cortes.

---La democracia---dijo otro mozalbete con voz elocuente, aunque
ceceosa---es aquella forma de \emph{gobierno en que el pueblo, en uso de
su soberanía, se rige por sí mismo, siendo todos los ciudadanos tan
iguales ante la ley que ellos se imponen, como lo somos los desterrados
hijos de Eva a los ojos de Dios}.

---Hombre, repíteme eso que es muy bonito, y quiero aprenderlo de
memoria para decírselo a mi papá esta noche al tiempo de cenar. A mi
papá, que es muy liberal, le gustan estas cosas.

Yo me aburría entre aquella gente, sin poder sacar sustancia de tan
inaguantable confusión de voces diversas, ni de aquel laberinto de
opiniones, de insensateces, de puerilidades, manifestadas en coro
inarmónico, cuyo susurro hubiera enloquecido la cabeza más fuerte. Dije
a D. Diego, que me marchaba, y él se empeñó en que le acompañase hasta
el fin.

---Yo oigo atentamente todo lo que hablan---me dijo,---para aprendérmelo
de memoria y soltarlo después en los cafés y en los ventorrillos. De
este modo voy adquiriendo fama de gran político, y cuando me acerco a la
mesa del café, todos me dicen: «a ver, D. Diego, qué piensa usted de la
sesión de hoy.»

Nos detuvimos un poco más; pero al fin pude sacarle con grandes
esfuerzos de allí, y nos marchamos a tomar el fresco a la muralla.

---¿Qué diría doña María---le pregunté,---si ahora me presentase yo en
la casa?

---Hombre, se me figura que mi señora madre no te juzga del todo mal.
Ostolaza dice de ti mil herejías; pero mamá se opone a que hablen mal de
nadie delante de ella\ldots{} Sin embargo, tienes en casa fama de ser un
terrible conquistador de hermosuras. Más vale que no vayas allá. ¡Ah,
pícaro!, ya sé que te gusta mi hermanita Presentación. Todos los días me
pregunta por ti\ldots{} Por mi parte si la quieres\ldots{} yo sé que
eres un hombre honrado.

---En efecto, me agrada.

---Como que te la llevaste a las Cortes una tarde\ldots{} Sí, cuando
salieron y cayó la bomba, y les dio auxilio el padre Pedro de
Advíncula\ldots{} El pobre D. Paco estuvo enfermo cinco días\ldots{}
volvió a casa lleno de bizmas, porque el estallido de la bomba,
¡asómbrate, chico!, le molió como si le hubieran dado una paliza.

---¡Desgraciado preceptor!\ldots{} No olvide usted, amiguito, que esta
noche hemos de ir a casa de Poenco.

---Sí; a olvidarme iba. Las carnes me tiemblan ya del gusto. ¿Dices que
va Pepilla la Poenca?

---Y toda la flor de la majeza.

---Me parece que no ha de llegar el momento en que mi señora mamá cierre
los ojos.

---Aguardo en Puerta de Tierra.

---Puerta del Cielo debía llamarse. ¿Irá también la Churriana?

---También.

---Pues aunque supiera que mi mamá estaba en vela toda la noche\ldots{}
adiós\ldots{} me voy a cenar y a rezar el rosario. Dentro de hora y
media estaré allá\ldots{} Tunante, diré a Presentación que te he visto.
¡Qué contenta se va a poner!

Cuando nos separamos visité de nuevo a lord Gray, y como le encontrara
dispuesto a salir a la calle, le dije:

---Milord, la señora condesa (Amaranta) me encargó ayer que rogase a
usted pasase a verla.

---Ahora mismo marcharé allá\ldots{} ¿Está usted libre esta noche?

---Libre, y a la orden de usted.

---Será algo tarde cuando yo necesite de su auxilio. ¿Dónde nos
encontraremos?

---No es preciso fijar sitio---repuse.---Yo tengo la seguridad de que
nos encontraremos. Una súplica tengo que hacer a usted. Mi espada no es
buena. ¿Quiere usted prestarme esa magnífica hoja toledana que está en
la panoplia?

---Con mil amores: ahí va.

Diómela, y cambié su arma por la mía.

---¡Pobre Currito Báez!---dijo riendo.---Han fijado ustedes el duelo
para esta noche. Pero, amigo mío, yo no puedo estar en todas partes.
Esta noche no podré asistir a la muerte de ese hombre.

---¿Pues no ha de poder? Hay tiempo para todo.

---Fijemos horas.

---No es preciso. Ya nos encontraremos. Adiós.

---Pues adiós.

Era de noche y corrí al ventorrillo. Don Diego tardó mucho; pasó una
hora, pasaron dos y yo no cabía en mí de ansiedad y afán. Por fin le vi
aparecer y calmose mi febril impaciencia con su llegada.

---Poenco---gritó dando manotadas sobre la mesa,---trae manzanilla. ¿Hay
algo de pescado para hacer sed?\ldots{} Querido Gabriel, hombre benévolo
y caritativo, pongo en tu conocimiento que ahora al pasar por la calle
del Burro me dieron ganas de entrar en casa de Pepe Caifás, y allí perdí
los cuatro duros que me diste esta tarde. ¿Llevarías tu longanimidad
hasta el extremo de darme otros cuatro? Ya sabes que me caso pronto.

Le di lo que me pedía.

---Señor Poenco, ¿dónde está Pepilla?

---Ha ido a confesar y está haciendo penitencia.

---¡A confesar! ¿Tu hija se confiesa? No la dejes acercarse a ningún
fraile. Ya sabes que los frailes son \emph{unos animales viles y
despreciables que viven en la ociosidad y holganza en una especie de
café-fondas donde se entregan a todo género de placeres}\ldots{}

---Todo lo que gastemos lo pago yo, tío Poenco---dije.---Venga Jerez.

---Gracias, gracias, valiente soldado. Siempre has sido generoso. De
modo que podré emborracharme\ldots{} Poenquillo, ¿me sabrás decir dónde
se puede ver esta noche a María Encarnación?

---Señorito D. Diego---dijo el pícaro,---no me comprometeré yo a decirle
dónde está, manque me diera esos cuatro soles de plata mejicana, porque
María Encarnación salió de aquí con Currito Báez, y tomando hacia la
calle del Torno de Santa María\ldots{} cétera, cétera.

Entraron varios majos ya de nosotros conocidos, y D. Diego les convidó a
beber, lo cual lejos de molestarles les causó muchísimo agrado.

---¿Vienes de las Cortes, Vejarruco?---preguntó D. Diego a uno de ellos.

---Sí\ldots{} y qué borrasca han armado allí con el papé de Lardizábal.

---Toos, toos son unos pillos---exclamó Lombrijón.---¡Qué gomitaeras
tenía aquel diputao alto, berrendo, querencioso, y qué cosas les dijo
cuando le dio aquel súpito, engrimpolándose too!\ldots{}

---¿Qué entiendes tú de eso, Lombrijón?\ldots{} Si lo que dijo fue que
el puebro\ldots{}

---En las orejas tengo el voquible, Vejarruco. Fue lo de la
mococrasia\ldots{}

---Apostad a cuál es más bruto---dijo don Diego con pedantería.---La
democracia, y no la mococrasia \emph{es aquella forma de gobierno en que
el pueblo, en uso de su soberanía se rige por sí mismo, siendo todos los
ciudadanos iguales ante la ley}\ldots{}

---Justo y cabal. ¡Qué bien parla este angelito! Si en mi poder
estuviera, mañana sería diputado.

---Algún día me votaréis, amigos Vejarruco y Lombrijón---dijo mi amigo
sintiendo ya en su cabeza con los vapores del generoso licor el humo de
la vana ambición.

---¡Viva el puebro soberano!---gritó Vejarruco.

---¡Vivan las Cortes!---gruñó Lombrijón batiendo palmas con el ritmo de
la malagueña.---Lo que igo es que un ruedo de muchachas bailando, con un
par de guitarras y otros tantos mozos güenos y un tonel de lo de
Trebujena que dé güelta a la reonda, me gustan más que las Cortes, donde
no hay otra música que la del cencerro que toca el presiente y el romrom
de los escursos.

---Que vengan las muchachas, que vengan las guitarras---gritó el de
Rumblar, dueño ya tan sólo de la mitad de su corto entendimiento.

---Poenco, si las traes te hacemos\ldots{}

---Te hacemos diputao\ldots{}

---¿Qué es eso? ¡Menistro! ¡Viva la libertad de la imprenta y el
menistro señó Poenco!

Mientras de este modo se enardecía el espíritu y se exaltaban los
sentidos de aquellos bárbaros, iba pasando mucho tiempo, más tiempo del
que yo quería que pasase sin poner en ejecución mi pensamiento. Habían
sonado las nueve, las diez, casi las once.

Más fuerte que si tuviera algo dentro, la cabeza de mi amigo D. Diego
resistía a frecuentes trasiegos del ardiente líquido; pero cuando
vinieron las mozas y comenzó la música, el noble vástago perdió los
estribos y dio con su alma y su cuerpo en el torbellino de la más
grosera orgía que ventorrillo andaluz puede ofrecer al sibaritismo.
Bailó, cantó, pronunció discursos políticos sobre una mesa, imitó el
pavo y el cerdo, y por último, ya muy tarde, cuando el afán me devoraba
y la impaciencia me tenía nervioso y aturdido, dio con su noble cuerpo
en tierra, cayendo inerte, como un pellejo de vino. Las mozas formaban
elegantes parejas con Vejarruco y Lombrijón; los guitarristas se
divertían por su cuenta en otro extremo de la taberna, roncaba como una
bestia enferma el gran Poenco y la ocasión era propicia para mí. Tomé
las dos llaves que el durmiente D. Diego llevaba en su bolsillo, y corrí
como un insensato fuera de la taberna.

La repugnante zambra habíase alargado bastante, porque eran ya casi las
doce.

\hypertarget{xxv}{%
\chapter{XXV}\label{xxv}}

Yo no corría, volaba, y en poco tiempo llegué a la calle de la Amargura,
mortificado por el recelo de acudir tarde. Un hombre que se lanza
desesperado al crimen no experimenta en el instante de perpetrar su
primer robo, su primer asesinato, emoción tan viva como la que yo
experimenté cuando introduje la llave, cuando le di vueltas poco a poco
para evitar todo ruido, cuando empujando la puerta ya abierta, esta
cedió ante mí sin rechinar, merced a las precauciones que con este fin
había tomado D. Diego. Entré, y por un rato halleme desorientado en la
profunda oscuridad del zaguán; pero a tientas y cuidadosamente pude
llegar al patio, donde la claridad del cielo que por la cubierta de
vidrios entraba, me permitió marchar con pie más seguro. Abriendo la
segunda puerta que daba paso a la escalera, subí muy despacio asido al
barandal.

El corazón me latía con loca presteza, pareciéndome tan desmesuradamente
ensanchado, que experimenté la sensación de llevar dentro del pecho un
objeto mayor que la casa en que estaba. Me tenté la espada, por ver si
estaba en mi cintura, y probé si salía con holgura de la vaina. En las
sombras que me rodeaban, creía ver a cada instante la imagen de lord
Gray y otra imagen, corriendo ambas fuera de la casa profanada.
Verdaderamente, señores, discurriendo con serenidad, no podía darme
cuenta del objeto de mi arriesgada expedición allí dentro. ¿Iba a
satisfacer en la persona de lord Gray mi anhelo de venganza, iba a
gozarme en mi propio desaire o a impedir la violenta determinación de
los locos amantes? Yo no lo sabía. En mi pecho bullían ardientes
furores, y se quemaba mi frente circundada por anillo de candente
hierro. Los celos me llevaban en sus alas negras llenas de agudas uñas
que desgarran el pecho, y dejándome arrastrar, no podía prever cuál
sería el término de mi viaje.

Al llegar al corredor de cristales que daba vuelta a todo el patio,
percibí con claridad los objetos, por la mucha luz de la luna que allí
penetraba. Entonces medité, y formulando vagamente un plan, dije:

---Aquí buscaré un sitio donde ocultarme. Lord Gray no puede haber
llegado todavía. Le espero, y cuando venga le saldré al paso.

Puse atento el oído, y creí sentir un rumor vago. Parecíame ruido de
faldas y pasos muy tenues. Aguardando un rato, al cabo distinguí una
forma de mujer que salía al corredor por la puerta menos próxima al
sitio donde yo me encontraba. Había allí un alto, pesado y negro ropero
que proyectaba sombra muy oscura sobre sus costados, y junto a él me
guarecí. Atisbé la figura que se acercaba, y al punto la reconocí. Era
Inés. Acercábase más, y al fin pasó por delante de mí. Yo me aplasté
contra la pared: hubiera querido ser de papel para ocupar el menor
espacio posible.

A la escasa luz pude advertir en ella una gran confusión. Inés iba hacia
la escalera, volvía, tornaba a adelantar, retrocediendo después. Sus
ademanes indicaban zozobra vivísima, más que zozobra, desesperación.
Exhalaba hondos suspiros, miraba al cielo como implorando misericordia,
reflexionaba después con la barba apoyada en la mano, y al fin volvía a
sus anteriores inquietudes.

---Es que le espera---dije para mí.---Lord Gray no ha venido.

Inés entró de repente en las habitaciones y salió al poco rato con un
largo mantón negro sobre la cabeza. Andaba con gran cautela, y sus
delicados pies parecía que apenas esfloraban los ladrillos del piso.
Volvió a pasar junto a mí, dirigiéndose a la escalera, pero retrocedió
otra vez.

---Está loca---pensé,---se dispone a salir sola. Sin duda él le espera
en la calle.

La muchacha descendió dos o tres peldaños, y tornó a subir. Entonces
observé claramente su rostro; estaba muy inmutada. Balbucía o ceceaba, y
su soliloquio, en que se le escapaban voces articuladas, era de los que
indican una gran agitación del alma. Algunas voces tenues y confusas que
salían de sus labios, llegaron a mi oído y percibí con toda claridad
estas dos palabras: \emph{«Tengo miedo.»}

Al pasar cerca de mí, no sé si sintió mi respiración o el roce de mi
cuerpo contra la pared, porque me era imposible permanecer en absoluta
quietud. Estremeciose toda, miró al rincón, y de seguro me vio, es
decir, vio un bulto, un fantasma, un ladrón, cualquiera de esos
vestigios o imaginarios duendes de la noche, que asustan a los niños y a
las muchachas tímidas. En el paroxismo de su miedo, tuvo, sin embargo,
bastante presencia de ánimo para no gritar; quiso correr, mas le
faltaron las fuerzas. Maquinalmente salí de mi escondite, dando algunos
pasos hacia ella, la vi temblorosa con los ojos desencajados y las manos
abiertas, acerqueme más, y le dije en voz muy baja:

---Soy yo; ¿no me conoces?

---Gabriel---dijo como quien despierta de un mal sueño.---¿Cómo has
entrado aquí? ¿Qué buscas?

---No me esperabas sin duda.

Su acento de profunda sorpresa no indicaba pesadumbre ni contrariedad.
Después añadió:

---No parece sino que te ha enviado Dios en socorro mío. Acompáñame:
tengo que salir a la calle.

---¡A la calle!---exclamé más desconcertado aún.

---Sí---dijo recobrandola zozobra que al principio había advertido en
ella;---quiero traerla aunque sea arrastrada por los cabellos\ldots{}
¡Ay! Gabriel, estoy tan angustiada que no sé cómo contarte lo que me
pasa. Pero vamos, acompáñame. No me atrevía a salir sola a estas horas.

Diciendo esto tomaba mi brazo, y con impulso convulsivo me empujaba
hacia la escalera.

---Esta casa está deshonrada\ldots{} ¡Qué vergüenza! Si mañana despierta
doña María y no la encuentra aquí\ldots{} Vamos, vamos. Yo espero que me
obedecerá.

---¿Quién?

---Asunción. Voy a buscarla.

---¿En dónde está?

---Se ha marchado\ldots{} Ha huido\ldots{} Vino lord Gray\ldots{} En la
calle te contaré\ldots{}

Hablábamos tan bajo que nos decíamos las palabras en el oído. En un
instante y andando con toda la prisa que permitía la oscuridad de la
casa, bajamos, abrimos las puertas y nos encontramos en la calle.

---¡Ay!---exclamó al ver cerrar por fuera la puerta.---En mi
atolondramiento se me olvidaba, al querer salir, que no tenía llaves
para abrir la puerta.

---Pero ¿a dónde vas tú, a dónde vamos?

---Corramos---dijo aferrándose a mi brazo.

---¿A dónde?

---A la casa de lord Gray.

Aquel nombre encendió de nuevo mi sangre, y pregunté con desabrimiento:

---¿Y a qué?

---A buscar a Asunción. Tal vez lleguemos a tiempo para impedir su fuga
de Cádiz\ldots{} Está loca esa muchacha, loca, loca, loca\ldots{}
Gabriel, ¿con qué objeto entrabas esta noche en la casa? ¿Ibas a
buscarme?\ldots{} ¿Ibas de parte de mi prima?

---Pero lord Gray\ldots{} Explícame eso.

---Lord Gray entró esta noche. Asunción le esperaba\ldots{} levantose
callandito de su cama y se vistió. Yo desperté también\ldots{} Asunción
se llega a mi cama cuando iba a partir, y besándome, en voz muy bajita
me dijo: «Inés de mi corazón, adiós, me voy de esta casa.» Yo salté de
mi cama, quise detenerla, pero la pícara lo tenía todo muy bien
dispuesto y salió con gran ligereza. Quise gritar, pero tuve
miedo\ldots{} La idea de que despertase doña María en aquel instante me
hacía temblar\ldots{} Se fueron muy despacito, y cuando me quedé
sola\ldots{} ¡Ay! La insensatez de esa muchacha, a quien todos tienen
por santa, me enardecía la sangre. Lord Gray la ha engañado; lord Gray
la abandonará\ldots{} Vamos, vamos pronto.

---¡Me parece que estoy soñando! De modo que Asunción\ldots{} ¿Pero qué
vamos a hacer, qué vamos a decir a Asunción y a lord Gray?

---¿Y eso dice un hombre, un caballero, un militar que lleva una espada?
Cuando les vi salir sentí un impulso de cólera\ldots{} quise correr tras
ellos\ldots{} luego me ocurrió llamar a los de la casa\ldots{} pero
después, pensando que lo mejor sería impedir la fuga de Asunción,
discurrí si podría traerla de nuevo a casa, con lo cual la condesa no se
enterará de nada\ldots{} Yo pedí auxilio al cielo y dije: «Dios mío,
¿qué puede hacer una mujer, una pobre y desvalida mujer, contra la
perfidia, la astucia y la fuerza de ese maldito inglés? Dios poderoso,
ayúdame en esta empresa.» Cuando yo decía esto te me presentaste tú.

---¿Y cuál es tu intención?

---Yo dudaba si salir o no. Era una locura salir\ldots{} ¿Qué hubiera
podido lograr sola? Nada. Ahora es distinto. Me presentaré en casa de
ese bandido; procuraré convencer a esa desgraciada de la miserable
suerte que le espera. ¡Oh!, nunca la creí capaz de acto tan
abominable\ldots{} Haré lo posible por traérmela conmigo. Un hombre me
acompaña, no temo a lord Gray, y veremos si persiste en sus viles
proyectos delante de mí.

---No persistirá. Lo que está pasando es un plan admirable de la
Providencia.

---La pobre Asunción es una tonta. Su fondo es bueno, pero con la
santidad, con el encierro y con lord Gray se le ha convertido la
imaginación en un hervidero. Nos queremos mucho. Varias veces he
conseguido de ella con mis cariñosas amonestaciones más que su madre con
el rigor y toda la Iglesia católica con sus santidades\ldots{} Volverá,
volverá con nosotros\ldots{} ¡Qué peligroso paso!\ldots{} ¡Ella y yo
fuera de casa!\ldots{} Corramos, corramos. La casa de ese hombre está en
el fin del mundo.

---Lord Gray abandonará su presa. Ya pronto llegamos. Lord Gray tendrá
el castigo que merece.

---¡Así te oyera Dios! ¡Pobre Asunción! ¡Pobre amiga! ¡Tan buena y tan
loca! Se me parte el corazón al considerarla deshonrada y perdida para
siempre. La arrancaremos de manos de su seductor\ldots{} No, no huirá de
Cádiz\ldots{} Aún faltan muchas horas para el día\ldots{} Vamos,
corramos pronto.

\hypertarget{xxvi}{%
\chapter{XXVI}\label{xxvi}}

Por fin llegamos a casa de lord Gray. Toqué fuertemente a la puerta y un
criado soñoliento y malhumorado bajó a abrirnos.

---El señor no está---nos dijo.

Creyendo que nos engañaba, empujé puerta y portero para abrir paso, y
entramos diciendo:

---Sí está. Me consta que está.

Como la casa de lord Gray era centro de aventuras, y allí entraban con
frecuencia hombres y mujeres a distintas horas del día y de la noche, el
criado no puso obstáculo a que invadiéramos imperiosamente la casa, y
guiándonos a la sala, encendió luces, sin cesar de repetir:

---El señor no está, el señor no ha venido esta noche.

Inés, desfallecida, dejose caer en un sillón. Yo recorrí la casa toda, y
en efecto, lord Gray no estaba. Después de mis pesquisas Inés y yo nos
miramos con angustiosa perplejidad, confundidos ante la inutilidad del
arriesgado paso que habíamos dado.

---No están, Inés. Lord Gray ha tomado sus precauciones y es inútil
pensar en impedir la fuga.

---¡Inútil!---exclamó con dolor.---No sé qué pensar. Llévame otra vez a
mi casa. ¡Dios mío santísimo, si me sienten llegar contigo!\ldots{} ¡Si
doña María se levanta y ve que Asunción y yo no estamos allí!\ldots{}
¡Esto ha sido una locura! ¡Desgraciada Asunción! ¡Tan buena y tan loca!

Inés lloraba con vivo dolor la pérdida de su amiga.

---Para mí es como si hubiera muerto---añadió.---¡Que Dios la perdone!

---Engañado por su aparente santidad, jamás creí que tuviera tan ciega
pasión por un hombre.

---Su hipocresía es superior a todo lo que puede concebirse. Ha
aprendido a disimular con tal arte sus sentimientos, que todos se
engañan respecto a ella.

---Para decírtelo todo de una vez, Inés, yo creí que la que amaba a lord
Gray eras tú. Todos, incluso Amaranta, creían lo mismo.

---Ya lo sé. Yo misma tengo la culpa de esto, porque deseando evitar a
mi amiga las crueles reprensiones y castigos de su madre, callaba y
sufría siempre, y las sospechas caían sobre mí. Conmigo tenían cierta
tolerancia, y como sólo se trataba de cartitas y tonterías, dejé correr
el engaño, pasando por casquivana\ldots{} Algunas veces me apropiaba
deliberadamente las faltas de Asunción, por el beneficio que me
traían\ldots{} ¿no entiendes? Mi mayor gusto era ver rabiar a D. Diego,
diciendo que no se casaría nunca conmigo.

---Él espera que pronto le darás tu mano.

Por primera vez en aquella noche la vi reír.

---Yo sabía---añadió después,---que todas las sospechas caían sobre mí,
y callaba. Jamás hubiera delatado a la pobre Asunción. Esperaba
arrancarle de la cabeza esa locura, y en una ocasión creí conseguirlo.
Lord Gray ponía en juego mil ingeniosas estratagemas\ldots{} ¿Tú sabes
todo lo que pasó el día que fuimos a las Cortes?\ldots{} ¡Hombre más
original!\ldots{} Yo esperaba que siguieras yendo a casa por la
noche\ldots{} te hubiera informado de todo\ldots{} Pasaron días y meses,
y entretanto, sola y abandonada de todos, necesitaba valerme de mis
propios esfuerzos para ir prolongando, prolongando mi situación, con la
esperanza de verme libre algún día\ldots{} Pero marchemos al punto de
aquí. ¡Dios mío, qué tarde!

---Inés, te he recobrado, te he reconquistado después de creerte perdida
para siempre---afirmé olvidando la situación en que nos
encontrábamos.---Has resucitado para mí. ¡Querida mía, imitemos la
conducta de Asunción y lord Gray, y vámonos por esos mundos!

Me miró con severidad.

---¿Deseas volver a aquella horrible prisión, más cerrada y más sombría
que la casa de los Requejos?---le dije con exaltación, estrujando sus
manecitas entre las mías.

---Más vale esperar---me contestó.---Llévame a mi casa.

---¡Otra vez allá!---exclamé deteniéndola en su marcha con la barrera de
mis brazos, que hubieran querido ser muralla indestructible para
separarla del resto del mundo.---¡Otra vez allá! Ya no te volveré a ver
más. Se cerrarán las puertas de ese purgatorio presidido por doña María,
y adiós para siempre. Querida mía, vamos a casa de la condesa; allí te
convenceremos. Sabrás lo que importa más que nada en el mundo.

Inés demostraba gran impaciencia.

---¡Pero un momento más, un momento! Pasan meses sin verte. Sabe Dios
hasta cuándo no nos veremos. ¿No sabes lo que me pasa? El gobierno ha
dispuesto que salga una expedición para desembarcar en Cartagena y
socorrer a las partidas de Castilla. Me han designado para formar parte
de ella. Pobre soldado, tengo que obedecer. ¿Cuándo nos volveremos a
ver? Nunca. No te separes de mí esta noche. Salgamos de aquí, y te
llevaré al lado de la condesa, tu prima.

---¡No, a casa, a casa!

---La puerta de aquella mansión me parece que es la losa de tu sepulcro.
Cuando se cierre, dejándote dentro, todo se acabó.

---No, yo no quiero salir como Asunción, acechando el sueño de su madre
para escapar. Yo no quiero salir así de mi encierro, sino en pleno día,
con las puertas abiertas y a la vista de todos. Vámonos. ¡Qué locura he
hecho esta noche, Dios mío! Asunción, ¿dónde estás? ¿Has muerto ya para
mí y para los demás?\ldots{} No puedo estar aquí ni un instante más. Me
parece que siento la voz de doña María llamándome, y los cabellos se me
erizan de espanto.

Inés se dirigió a la salida. En el mismo instante oímos ruido de un
coche en la calle. Aguardamos, sintiendo que alguien subía, y por fin
abriose la puerta de la sala, y apareció lord Gray. Estaba sombrío,
fosco, agitado, nervioso.

~

Nos miró con asombro, quiso reír, pero su colérico semblante no echaba
de sí más que rayos. Temblaba de ira, iba de un lado para otro de la
sala, como un tigre en su jaula, nos miraba, nos decía algo inconexo,
risible, estúpido, y luego hablaba consigo mismo en monosílabos
incomprensibles, mezclando la lengua inglesa con la española.

---Sr.~de Araceli, buenas noches\ldots{} Y usted, niña, ¿qué hace aquí?
¡Ah!, ya\ldots{} Mi casa sirve de refugio a los amantes\ldots{} Son
ustedes más afortunados que yo\ldots{} ¡Condenación eterna para las
niñas mojigatas!\ldots{} Un hombre como yo\ldots{} No debí
acceder\ldots{} ¡Por San Jorge y San Patricio!\ldots{}

---Lord Gray---dije,---hemos venido a esta casa con móvil muy distinto
del que usted supone.

---¿En dónde está Asunción?---exclamó Inés con vehemencia.---No, no
saldrán ustedes de Cádiz. Voy a alborotar toda la ciudad.

---¿Asunción?---repuso el inglés pateando con cólera y elevando el
puño.---He sido un necio\ldots{} pero mañana veremos\ldots{} El demonio
me lleve si cedo\ldots{} ¿Qué decía usted? Asunción\ldots{} es una niña
honradita y formalita\ldots{} ¡Maldito bigotism!\ldots{} Mucho lloro,
mucho hipo, mucho suspirito\ldots{} ¡Mala peste!\ldots{} ¿Qué decía
usted?\ldots{} Perdone usted\ldots{} Estoy nervioso\ldots{} despido
fuego y electricidad\ldots{} Pues como decía, Asunción\ldots{}

---¡Sí!, ¿dónde está? Es usted un malvado.

---La pobrecita niña está ya de vuelta en casa rezando el
\emph{Confiteor} con las manecitas cruzadas delante del altarejo\ldots{}
¡Malditas sean las niñas piadosas!\ldots{} Parece que su voluntad ha de
ser de roca, y es cera de iglesia. Están buenas para sacristanes\ldots{}
Pues sí. En su casa está ya de vuelta. El seráfico arcangelillo se
asustó al verse solo conmigo en lugar extraño\ldots{} ¡No les gusta más
que la sacristía!\ldots{} Lloró, rabió, quiso matarse, escandalizó la
casa de aquella ilustre doña Mónica a donde la llevé\ldots{} Jamás me ha
pasado otra como esta\ldots{} ¡Pobre gatita, cómo mayaba! ¡Qué
lastimeros ayes! ¡Qué gritos para clamar por su honor!\ldots{} Nada; es
preciso ser fraile o sacristán\ldots{} En fin, ya está otra vez en su
casa, a donde acabo de llevarla sigilosamente, lo mismo que la
saqué\ldots{} Señora doña Inesita, veo que es usted mujer
resuelta\ldots{} Usted se ha echado a la calle con este insigne
mancebo\ldots{} No hay que hacer aspavientos de honor y demás
bambolla\ldots{} La señora condesa me lo ha contado todo esta tarde
desde la cruz a la fecha\ldots{} Ella quería que yo me comprometiese a
librarla a usted de su cautiverio, y convine en ello\ldots{} Pero
ustedes lo han sabido arreglar. Así se hace\ldots{} Esta noche las
contrariedades y las desdichas son para mí\ldots{} Pero mañana\ldots{}
tomaré precauciones\ldots{} O hizo Lucifer a las mojigatas para reírse
de los enamorados, o las hizo Dios para castigarlos\ldots{}
Recapacitemos; ¡las hizo Dios, Dios, Dios!\ldots{}

---Salgamos al instante de aquí---dijo Inés.---

Este hombre está loco. Si es cierto que la infeliz ha vuelto a casa,
pronto lo sabremos.

Impulsado por una determinación súbita, dije al inglés:

---Milord, ¿me presta usted su coche?

---Está a la puerta.

---Pues vamos.

Bajamos. Cogí a Inés en mis brazos, y subiéndola en la alta carroza (una
de las singularidades del Cádiz de entonces, introducida por lord Gray)
dije al cochero:

---A casa de la señora de Cisniega, en la calle de la Verónica.

\hypertarget{xxvii}{%
\chapter{XXVII}\label{xxvii}}

---¿A dónde me llevas?---exclamó Inés con espanto cuando me senté junto
a ella dentro del coche que empezó a rodar pesadamente.

---Ya lo has oído. No me preguntes por qué. Allá lo sabrás. He tomado
esta resolución y no hay fuerza humana que me aparte de ella. No es una
calaverada; es un deber.

---¡Qué dices! Yo salí para salvar a mi amiga de la deshonra, y la
deshonrada soy yo.

---Inés, oye lo que te digo. ¿Estás decidida a casarte con D. Diego?

---Déjate de simplezas.

---Pues entonces calla y resígnate a ir a donde yo te lleve. Una serie
de acontecimientos providenciales te ha puesto en mi poder y creería
cometer un crimen si te llevara de nuevo a aquel aborrecido encierro,
donde al fin serías víctima del egoísmo fanático y de la insoportable
autoridad de quien no tiene ningún derecho a martirizarte\ldots{}
Pobrecilla, graba en tu memoria lo que te estoy diciendo y más tarde
bendecirás esta locura mía. No, no volverás allá. No pienses más en doña
María. Confía en mí. Dime: ¿te he engañado alguna vez? Desde que nos
conocimos ¿no has sido para mí una criatura venerada a quien de ningún
modo se puede ofender? ¿No has visto siempre en mí, junto con el cariño
más vivo que jamas se tuvo hacia persona alguna, un respeto, un culto
superior a todas las debilidades humanas? Inés, tú eres víctima de un
gran error. ¿Temes a doña María, temes a la de Leiva, temes a esas
siniestras y medrosas figuras que constantemente te están vigilando con
sus ojos terribles? Pues bien; esas dos personas no son para ti otra
cosa que dos figurones como los que asustan a los chicos. Acércate,
tócalos y verás cómo son cartón puro.

---No sé qué quieres decir.

---Quiero decir---continué hablando con tanta vehemencia como
rapidez,---que te has forjado respetos de familia, consideraciones e
ideas que son hijas de un error. Te han engañado, están abusando de tu
bondad, de tu dulzura para fines execrables, y no pudiendo amoldar tu
hermosa condición a la suya, te corrompen por grados, falsificándote,
querida mía, con la escuela del disimulo. No hagas caso, no pienses en
ellas, considérate libre. Vivirás al amparo de la única persona que
tiene derecho a mandar en ti; serás libre, disfrutarás de los goces
inocentes, de los nobles placeres de la Naturaleza; podrás mirar al
cielo, admirar las obras de Dios, podrás ser buena sin hipocresía,
alegre sin desenfado, vivir rodeada de personas que te adoren, y con la
conciencia en paz y tranquila. No interrumpirá tu sueño la cavilación de
los fingimientos que tendrás que hacer al día siguiente para que no te
castiguen. No te verás en el doloroso caso de mentir; no te aterrará la
idea de desposarte con un hombre aborrecido; no estarás expuesta a la
alternativa de que peligre tu virtud o seas desgraciada, desgraciadísima
y digna de lástima en esta breve vida y luego condenada en la eternidad
de la otra.

---Gabriel---me dijo ella bañado el rostro en lágrimas,---no entiendo lo
que me dices. No puedo creer que tú seas capaz de engañarme. ¿Lo que
dices es una locura o qué es\ldots? ¿A dónde me llevas\ldots? Por Dios,
no hagas una locura. Cochero, cochero, a la calle de la Amargura.

---El cochero irá donde yo le mande---exclamé alzando la voz, porque el
ruido del carruaje nos obligaba a hablar a gritos.---Regocíjate, Inés,
alégrate, amiguita. El aspecto de tu existencia va a cambiar desde esta
noche. ¡Cuántas penas, pobrecita, cuántas alternativas y vaivenes en tan
pocos años! Por un lado tú, por otro yo. Ambos sujetos a mil fatigas,
mecidos y arrastrados por este oleaje terrible que ya nos sube, ya nos
baja, ya nos junta, ya nos separa\ldots{}

---Es verdad, es verdad.

---¡Pobre amiga mía! ¡Quién había de decirte que en tu grandeza serías
tan desgraciada como en tu miseria!

---Sí, es verdad, es verdad\ldots{} Pero me dejo arrastrar por tu
demencia. ¡Llévame a mi casa, por Dios! Después concertaremos\ldots{}

---Ya está concertado\ldots{}

---Pero mi familia\ldots{} Yo tengo nombre y familia\ldots{}

---A eso voy.

---No, no puedo consentirlo. Es imposible que me engañes\ldots{} ¡A
casa, a casa!

¡Qué dirán de mí! ¡Virgen Santísima!

---No dirán nada.

---Yo tengo imaginado un gran plan\ldots{}

---Este plan es el mejor\ldots{} Tu prima acabará de dártelo a conocer.
Al diablo doña María y la de Leiva.

---Es el jefe de la familia. Ella manda.

---Ahora mando yo, Inés. Obedece y calla. ¿No recuerdas que en todos los
instantes supremos de tu vida has necesitado de mi ayuda? Ahora es lo
mismo. Hace tiempo que buscaba esta ocasión\ldots{} te atisbaba con
vigilante mirada\ldots{} quería robarte, como te robé en casa de los
Requejos, y al fin lo he conseguido\ldots{} Que venga acá doña María a
arrancarte de mi poder. Lo demás te lo dirá tu prima. Ya llegamos.

Fuera que confiaba en mí entonces como en otras ocasiones de su vida,
abandonándose a aquel destino suyo, de que yo había sido tantas veces
celoso ejecutor; fuera que un vago presentimiento la inclinaba a aprobar
mi conducta, lo cierto es que no hizo esfuerzo para resistir cuando
entré con ella en la casa y la conduje arriba, despertando con el
estruendo de mi llegada a todos los habitantes de la casa. Gran susto
tuvo Amaranta al sentir tan a deshora los golpes y voces con que yo me
anuncié. Al salir a mi encuentro, doña Flora y la condesa estaban
aturdidas de puro asombradas.

---¿Qué es esto? ¿Cómo has salido de la casa?---exclamó la condesa,
besándola con ternura.---A Gabriel debemos sin duda esta buena obra.

---Qué placer es estar junto a usted, querida primita---dijo Inés
sentándose en el sofá de la sala tan cerca de Amaranta, que casi estaba
sobre sus rodillas.---Me olvido de la falta que he cometido huyendo de
mi casa, y los gritos de mi conciencia son ahogados por la gran
felicidad que ahora siento. Estaré un ratito, un ratito nada más.

---Gabriel---dijo Amaranta con el rostro inundado de lágrimas---¿cuándo
sale la expedición? Yo pediré permiso para marchar en ella y nos
llevaremos a Inés.

---¡Huir!---exclamó la muchacha con terror.---Yo apareceré a los ojos de
todos como una criatura sin pudor que deshonra y envilece a su
familia\ldots{} Volveré a casa de doña María.

---¡Fuera engañosas apariencias!---grité yo.---Por más que vuelvas a
todos lados la vista, no encontrarás más familia que la que en estos
momentos te rodea.

La condesa con su mirada penetrante quiso imponerme silencio; pero yo no
podía callar, y los pensamientos que se agitaban con febril empuje en mi
cerebro, afluían precipitadamente a mis labios, dándome una locuacidad
que no podía contener.

---El entrañable amor que te ha manifestado siempre la persona en cuyos
brazos estás, ¿no te dice nada, Inés? Cuando pasaste de la humildad de
tu niñez a la grandeza de tu juventud, ¿qué brazos te estrecharon con
cariño? ¿Qué voz te consoló? ¿Qué corazón respondió al tuyo? ¿Quién te
hizo llevadera la soledad de tu nobleza? Seguramente has comprendido que
entre ella y tú existían lazos de parentesco más estrechos que los que
reconoce el mundo. Tú lo conoces, tú lo sabes, tu corazón no puede
haberse engañado en esto. ¿Necesito decírtelo más claro? La voz de la
Naturaleza antes de ahora, en todas ocasiones, y más que nunca ahora
mismo clamará dentro de ti para declarártelo. Señora condesa, abrácela
usted, porque nadie vendrá a arrancarla de manos de su verdadero dueño.
Inés, descansa tranquila en ese seno, que no encierra egoísmo ni
intrigas contra ti, sino sólo amor. Ella es para ti lo más santo, lo más
noble, lo más querido, porque es tu madre.

Diciendo esto callé; descansé como Dios después de haber hecho el mundo.
Estaba tan satisfecho de haber hablado, que las lágrimas, la turbación,
la emoción silenciosa y profunda de las dos mujeres, abrazadas y
oprimidas una contra otra como queriendo formar una sola persona, me
halagaban más que al orador elocuente los aplausos de la multitud y el
delirio del triunfo. Las últimas palabras las solté como se echa fuera
algo que nos ahoga.

\hypertarget{xxviii}{%
\chapter{XXVIII}\label{xxviii}}

Mientras madre e hija espaciaban a sus anchas y a solas los sentimientos
y ternezas de su corazón, yo me encontraba (seis horas después de lo
contado, y ya muy entrado el día) frente a frente de mi señora doña
Flora, separada su persona de la mía tan sólo por la breve superficie de
una mesa, donde dos regulares tazones de chocolate nos servían de
almuerzo. Hablamos un rato del acontecimiento que mis lectores conocen,
y después, arrimando con arte la conversación hacia asunto más de su
gusto, me dijo:

---Amaranta me asegura que no miras con malos ojos a esa jovenzuela que
nos trajiste anoche. ¡Bonita formalidad es la tuya! ¿Y qué dirán de un
chiquillo que en vez de inclinarse a buscar apoyo para sus
inexperiencias en la compañía de personas mayores, se enloquece con las
niñas de su misma edad?\ldots{} Vuelve en ti, hombre\ldots{} oye la voz
de la razón\ldots{} penétrate bien de\ldots{}

---Vuelvo, oigo y penetro, señora doña Flora. Estoy arrepentido de mi
locura\ldots{} Tentome el demonio, y\ldots{} Pero siento pasos, que se
me figura son los del Sr.~D. Pedro del Congosto.

---Jesús, María y José\ldots{} ¡Y tú ahí tan serio tomando chocolate
conmigo!\ldots{} Pero hombre, ¿y el pudor y la decencia?

No pudo continuar porque entró D. Pedro, todo lleno de bizmas y parches,
fruto amarguísimo de la brillante campaña del Condado. Levantose azorada
doña Flora, y dijo:

---Sr.~D. Pedro\ldots{} es una casualidad, créalo usted, que se
encuentre aquí este mozuelo\ldots{} Nunca está una libre de
calumnias\ldots{} Este chico es tan loco, tan imprudente\ldots{}

Congosto me miró con ira, y tomando asiento, habló así:

---Dejemos a un lado esa cuestión. A su tiempo será tratada\ldots{}
Ahora vengo a decir a usted que se prepare a recibir a la señora condesa
de Rumblar, que viene seguida de respetables personas para que le sirvan
de testigos.

---¡Dios mío! ¡La justicia en mi casa!

---Parece que lord Gray robó anoche a la señora doña Inesita,
depositándola aquí.

---¡Es un error! ¿Pero de veras viene doña María? Yo estoy
temblando\ldots{} Alguien ha entrado en la casa.

No había acabado de decirlo cuando sintiose gran ruido abajo y arriba
gran conmoción. Apareció Amaranta, apareció Inés, emitiéronse distintos
pareceres, pero prevaleció el de que se recibiese decorosamente a la de
Rumblar, contestando a sus cargos en el terreno legal, si ella en el
mismo los hacía.

Todos menos Inés nos reunimos en la sala, y a poco entró el lúgubre
cortejo, presidido por doña María, con una pompa y severa majestad que
le habrían envidiado reinas y emperatrices. Profundo silencio reinó en
la sala por un instante, mas rompiolo al fin, sin gastar tiempo en
saludos, doña María, no pudiendo contener el volcán que bramaba dentro
de las cavidades de su pecho.

---Señora condesa---dijo,---venimos a casa de usted en busca de una
doncella puesta a mi cuidado, la cual ha sido robada esta noche de mi
casa por un hombre que se supone sea lord Gray.

---Aquí está, sí, señora---repuso Amaranta.---Es Inés. Si estaba puesta
al cuidado de personas extrañas, yo la reclamo porque es mi hija.

---Señora---dijo doña María temblando de cólera,---ciertas supercherías
no producen efecto ante la declaración categórica de la ley. La ley no
la reconoce a usted por madre de esa joven.

---Pues yo me reconozco y declaro aquí delante de los que me escuchan,
para que conste con arreglo a derecho. Si usted alega una ley, yo alego
otra, y entretanto mi hija no saldrá de mi casa, porque a ella ha venido
espontáneamente y por su propia voluntad, no seducida por un cortejo,
sino con deliberado propósito de vivir a mi lado, como hija obediente y
cariñosa.

---No me sorprende la conducta de lord Gray---dijo doña María.---Los
nobles de Inglaterra suelen corresponder de este modo a la hospitalidad
que se les da en las casas honradas\ldots{} Pero no debo culpar tan sólo
a él, hombre de mundo, privado de ideas religiosas y ciego ante la luz
de la verdadera y única Iglesia, no. ¿Qué ha de hacer el ciego sino
tropezar? A quien principalmente acuso es a ella; lo que más que nada me
asombra es la liviandad de esa muchacha casquivana\ldots{}
Verdaderamente, señora condesa, voy creyendo que tiene usted razón en
llamarla su hija. Árbol y fruto con iguales propiedades se distinguen.

---Señora doña María---replicó Amaranta con la voz tan temblorosa, a
causa de la cólera, que apenas se entendían sus palabras,---no vino mi
hija seducida por lord Gray. Vino acompañada por él o por otro, que esto
no hace al caso, y movida de propia inspiración y deseo. Me congratulo
de ello, porque así la persona que más amo en el mundo estará libre de
corromperse con el mal ejemplo de dos conocidas niñas mojigatas, que
esconden a sus novios bajo las faldas de brocado de los santos que
tienen en los altares de su casa.

Doña María se levantó como si el sillón en que estaba sentada se
sacudiera repelido por subterránea explosión. Sus ojos fulminaban rayos,
su curva nariz, afilándose y tiñéndose de un verde lívido, parecía el
cortante pico del águila majestuosa: moviose convulsivamente su barba
picuda, reliquia de la antigua casta celtíbera a que pertenecía, hizo
ademán de querer hablar; mas con gesto majestuoso semejante al de las
reinas de la dinastía goda cuando mandaban hacer alguna gran justicia,
señaló a la otra condesa, y desdeñosamente dijo:

---Vámonos de aquí. No es este mi lugar. Me he equivocado. Señora
condesa, quise que no se agriara esta cuestión; quise evitar a usted la
visita de los emisarios de la ley. Pero usted no merece otra cosa, y no
seré yo quien desempeñe en esta casa el papel que corresponde a
alguaciles y polizontes.

---Como experta en pleitos---repuso Amaranta,---y conocedora de tal laya
de gente, puede usted buscar en la familia de estos una esposa para su
digno hijo el señor conde, varón insigne en las tabernas y garitos de
Madrid. Jugando al monte podrá restablecer el mermado patrimonio, sin
verse en el caso de solicitar un enlace violento con una joven
mayorazga.

---Salgamos de aquí, señores; son ustedes testigos de lo que aquí ha
pasado---dijo doña María dirigiéndose a la puerta.

Y sin esperar a más, resueltamente y bramando de ira, que expresaba con
olímpico fruncimiento de cejas, salió de la sala y de la casa, seguida
de los mismos que le habían acompañado, a cuya cola iba D. Paco.

Por largo rato reinó profundo silencio en la sala. Amaranta, después de
desahogar las antiguas cóleras de su pecho, estaba meditabunda y aun
diré que arrepentida de todo lo que había dicho, doña Flora preocupada,
y Congosto, con los ojos fijos en el suelo, revolvía sin duda en su
cabeza altos y caballerescos pensamientos. Sacó a todos de su
perplejidad una visita que nadie esperaba, y que causara general
asombro. En la sala se presentó de improviso lord Gray.

Advertí en su fisonomía las huellas de la agitación de la pasada noche,
y lo turbado de su hablar indicaba que aquel singular espíritu no había
recobrado su asiento.

---En mal hora viene milord---le dijo secamente D. Pedro.---Ahora acaba
de salir de aquí doña María, cuyo enojo por las picardías de usted es
tan fuerte como justo.

---La he visto salir---repuso el inglés.---Por eso he entrado. Deseo
saber\ldots{} ¿Se sospecha de mí, señora condesa, se me acusa?\ldots{}

---¡Pues no se le ha de acusar, hombre de Dios!\ldots---dijo D.
Pedro.---Pues a fe que echó requiebros la señora doña María\ldots{} y
con mucha razón por cierto. Pues qué, robar a la señora doña Inesita,
aun con consentimiento de la que se llama su madre\ldots{}

---Vamos, estoy tranquilo---dijo lord Gray.---Veo que me imputan las
hazañas de este pícaro Araceli, dejando en el olvido las mías propias.
Desvaneceré el engaño, aunque en realidad, yo acepto todas las glorias
de esta clase que me quieran adjudicar\ldots{} La señora condesa estará
ya contenta.

Amaranta no contestó.

---Disimule usted---dijo D. Pedro.---Eche usted sobre el prójimo sus
abominables culpas.

---Veo con dolor---repuso lord Gray jovialmente,---que en el rostro de
usted, Sr.~de Congosto, están escritas con parches y ungüentos las
gloriosas páginas de la expedición al Condado.

---Milord---exclamó el héroe con ira,---no es propio de un caballero
zaherir desgracias motivadas por la casualidad. Antes que hacer tal cosa
examinaría yo mi conciencia por ver si está libre de faltas. La mía no
me acusa de haber cometido en ningún tiempo bellaquerías como la de
anoche.

---¿Cuál?

---Ya lo sabe usted. Acabamos de oír a la señora de Rumblar---añadió la
estantigua enfureciéndose gradualmente.---Digo y repito que es una gran
bellaquería.

---Eso va con usted, Araceli.

---No, con usted, con usted, lord Gray. Usted es quien ha sacado a esa
joven de aquella honesta casa, morada augusta de los buenos principios;
usted quien la ha quitado de la protección y amparo de doña María, cuya
santidad y nobleza engrandecen cuanto a su alcance se halla.

---¿Con que es una gran bellaquería?---repitió lord Gray
burlonamente.---Eso quiere decir que soy un gran bellaco.

---¡Sí señor, un grandísimo bellaco!---repitió don Pedro, poniéndose tan
encendido que las arrugas de su rostro semejaban los pliegues y
abolladuras de un pimiento riojano.---Y aquí está D. Pedro del Congosto,
para sostener lo que ha dicho, aquí y fuera de aquí en la forma y manera
que usted lo crea conveniente.

---¡Oh, Sr.~D. Pedro!---exclamó lord Gray con júbilo.---¡Qué gran placer
me proporciona usted! Desde que por primera vez visité esta noble
tierra, he buscado ansiosamente al gran D. Quijote de la Mancha; yo
quería verle, yo quería hablarle, yo quería medir la fuerza de mi brazo
con la del suyo, pero ¡ay!, hasta ahora lo he buscado en vano. He
revuelto media península buscando a D. Quijote, y D. Quijote no parecía
por ninguna parte. Yo creí que tan noble tipo se había extinguido,
disipándose en la corruptora sociedad de los modernos tiempos; pero no,
aquí está, al fin le encuentro con idéntico traje y rostro, un Quijote
algo degenerado en verdad, pero Quijote al fin, que no se encuentra ni
puede encontrarse más que en España.

---Si usted bromea, señor lord, yo soy hombre serio---repuso D.
Pedro.---Yo tomo a mi cargo la defensa de esa ultrajada señora que acaba
de salir; yo desharé su agravio y me tomo a pechos el castigar esta gran
injuria que ha recibido limpiando con la sangre del traidor la infame
mancha. Esto digo sin nada de quijotería. Ya se ve\ldots{} en esta casa
no me entienden. Es indudable que han entrado aquí las ideas
filosóficas, ateas y masónicas, según las cuales ya se acabó el honor y
la grandeza, lo noble y lo justo, para que no haya más que pillería,
liberalismo, libertad de la imprenta, igualdad y demás
corruptelas\ldots{} Lo dicho, dicho. Este traje que visto prueba que he
tomado a mi cargo la defensa de los principios en cuyo nombre se ha
levantado la nación contra Bonaparte. ¡Oh, si todos me imitaran!\ldots{}
¡Si todos empezando por el traje acabaran por las obras!\ldots{} Pero
basta de palabras. Elija usted hora y sitio. Acción tan aleve no puede
quedar sin castigo.

---D. Quijote, sí, es él mismo---dijo el inglés.---D. Quijote degenerado
y nacido de cruzamientos, pero que algo conserva de la generosa sangre
del padre, como el mulo lleva en sí un poco de la dignidad y nobleza del
caballo.

---¡Cómo! ¿Llama usted mulo a un hombre como yo?---exclamó Congosto
requiriendo coléricamente la espada.

---No, caballero insigne; decía que el quijotismo español de hoy se
parece al antiguo, como se parece el mulo al caballo. Por lo demás
acepto el reto de usted y nos batiremos a la jineta, a pie, con sable,
espada, lanza, honda, ballesta, arcabuz, o como usted quiera. Pronto
partiré de Cádiz, quizás mañana mismo. Disponga usted de mí cuando
guste.

---¿De verás se marcha usted?---dijo Amaranta saliendo de su atonía.

---Sí, señora, estoy decidido\ldots{} Vendré a despedirme de
usted\ldots{} Con que Sr.~D. Pedro\ldots{}

---Lo dicho, dicho. Enviaré mi padrino.

---Lo dicho, dicho. Enviaré el mío.

D. Pedro salió mirándonos con altanera soberbia, que nos hizo sonreír a
todos menos a doña Flora, la que reprendió al inglés su deseo de sujetar
a nuevas pruebas la quebrantada osamenta del héroe del Condado. Después
la condesa, que no participaba de nuestro humor festivo por la escena
cómica que había seguido a la trágica, cual ordinariamente ocurre en el
mundo, llevome aparte, y con aflicción me dijo:

---Temo haberme dejado arrastrar demasiado lejos por la ira que me
produjo la presencia de aquella mujer. Le dije cosas demasiado duras, y
cada palabra me pesa sobre la conciencia. Exasperada por lo que le dije,
tomará venganza de mí, y si acude a la ley, no creo que la ley me sea
favorable. Yo no tomé precaución alguna cuando se verificó el
reconocimiento de Inés.

---Venceremos esas y otras dificultades, señora.

---Yo transigiría con ella y con mi tía, con tal que me dejaran a Inés.
Creo que cediendo a doña María parte de mis derechos mayorazguiles,
sería fácil aplacar esa furia. La de Leiva no es ni con mucho tan
inconquistable.

---¿Quiere usted que lo proponga a la señora doña María?\ldots{} Nada se
pierde\ldots{} No sé si me recibirá; pero intentaré hablarla. Me
favorece el que no sospecha nada de mí en el suceso de anoche.

---Es una buena idea. Sí\ldots{} tampoco sería malo que yo me mostrase
arrepentida de las atrocidades que le dije\ldots{} no\ldots{} ¡Oh, qué
confusión, Dios mío! No sé qué hacer\ldots{}

---Cualquiera de esos actos me parece aceptable.

---¿Te parece que debo ir allá?

---Hoy no es conveniente. Se reanudaría al punto la reyerta, porque
aquel volcán en erupción estará echando fuego, humo y lava por algún
tiempo. Será prudente que yo me anticipe e indique a doña María esa idea
de transacción que usted le propone, con tal que no la priven de su
hija.

---Sí, hazlo tú primero. Yo me arriesgaré a tratar con mi tía, que es el
jefe de la familia, pero antes conviene tantear a la de Rumblar, a ver
qué tal se presenta.

---Ante todo debo indicar prudentemente a doña María que usted reconoce
haber estado algo dura en la entrevista.

---Sí\ldots{} lo encomiendo a tu habilidad, y me quedo tranquila\ldots{}
Si te recibe mal, no te importe. Con tal que te deje hablar, aguanta
desprecios y desaires.

Hago mención de este diálogo que tuvimos la condesa y yo, para que
comprenda el lector la razón de la extraña visita que hice a doña María
un día después de aquel de tanto ruido en que ocurrió lo que acabo de
contar.

\hypertarget{xxix}{%
\chapter{XXIX}\label{xxix}}

En efecto, traslademe a hora que me pareció oportuna a casa de doña
María, recelando no ser recibido, pero con el firme propósito de no
salir de allí sin intentar por todos los medios ver y hablar a la
orgullosa dama. Encontré a D. Diego, quien, contra mi creencia,
recibiome muy bien y me dijo:

---Ya sabrás los escándalos de esta casa. Lord Gray es un canalla.
Cuando yo dormía en casa de Poenco, fue allá y me sacó las llaves del
bolsillo\ldots{} No podía haber sido otro. ¿Le viste tú entrar?

---Sr.~D. Diego, quiero ver a la señora condesa para hablarle de un
asunto que a esta familia, lo mismo que a la de Leiva, importa mucho.
¿Tendrá la señora la bondad de recibirme?

Madre e hijo conferenciaron a solas un rato allá dentro, y por fin la
señora se dignó ordenar que me llevaran a su presencia. Estaba la de
Rumblar en la sala acompañada de sus dos hijas. La madre tenía en el
altanero semblante la huella de la gran pesadumbre y borrasca del día
anterior, y la penosa impresión se traslucía en una especie de repentino
envejecimiento. De las dos muchachas, Presentación revelaba al verme
cierta alegría infantil, que ni aun la proximidad de su madre podía
domar, y Asunción una tristeza, una decadencia, una languidez taciturna
y sombría, señal propia de los muy místicos o muy apasionados.

La señora de Rumblar, después de ordenar a Presentación que se alejase,
me recibió con un exordio severísimo, y luego añadió:

---No debía ocuparme de nada que se refiera a aquella casa donde ayer
por mi desgracia estuve; pero la cortesía me obliga a oírle a usted,
nada más que a oírle por breve tiempo.

---Señora---dije,---yo me marcharé pronto. Recuerdo que usted me rogó
que no volviese más a su casa. Hoy me trae un deber, un deseo vehemente
de restablecer la paz y armonía entre personas de una misma familia,
y\ldots{}

---¿Y a usted quién le mete en tales asuntos?

---Señora, aunque extraño a la casa, me ha afectado tan profundamente el
agravio recibido por esta augusta familia, a quien respeto y admiro
(aunque mis enemigos calumniadores hayan hecho creer a usted lo
contrario) que me sentí vivamente inclinado a terciar de parte de usted.
Señora doña María, vengo a decir a usted que la condesa se muestra hoy
arrepentida de las duras palabras\ldots{}

---¿Arrepentimientos?\ldots{} Yo no lo creo, caballero. Suplico a usted
que no me hable de esa señora. Si es eso lo que usted quería
decirme\ldots{} La justicia está ya encargada de esto y de devolver a
Inés al jefe de la familia.

Asunción alzó la vista y miró a su madre. Parecía deseosa de hablarle,
pero con tanto miedo como deseo. Al fin, cobrando valor, se expresó de
este modo con voz quejosa y tristísima, que producía en mí extraña
sensación.

---Señora madre, ¿me permite usted que hable una palabra?

---Hija mía, ¿qué vas a decir? Tú no entiendes de esto.

---Señora madre, déjeme usted decirle una cosa que pienso.

---Está delante una persona extraña y no puedo negártelo. Habla.

---Pues yo pienso, señora, que Inés es inocente.

---He aquí, Sr.~D. Gabriel, lo que es la limpieza de corazón. Esta
tierna y piadosa criatura, a quien una celestial ignorancia de las
maldades de la tierra eleva sobre el vulgo de los mortales, es incapaz
de comprender que haya ruines pasiones en la sociedad. Hija mía, bendita
sea tu ignorancia.

---Inés es inocente, lo repito---afirmó Asunción.---Lord Gray no puede
haberla sacado de esta casa, porque lord Gray no la quiere.

---No la quiere porque no te lo ha dicho\ldots{} ¿Qué sabes tú de eso,
hija mía? ¿Tienes acaso idea de los ardides, de la perfidia, de los
disimulos y malignas artes que usa la seducción?

---Inés es inocente---repitió cruzando las manos.---Algún otro motivo la
habrá impulsado a abandonarnos, pero no el amor de lord Gray. No, lord
Gray no la ama. ¿Cree usted en los Evangelios? Pues tan verdad como los
Evangelios es esto que estoy diciendo.

---En otra ocasión me enfadaría---dijo la madre,---al ver la exageración
de tu benevolencia. Hoy mi espíritu está quebrantado: anhelo la
tranquilidad y te perdono.

---¿No me deja usted decir otra cosita que me falta?

---Acaba de una vez.

---Yo quiero ver a Inés.

---¡Verla!---exclamó con enfado doña María.---Mis hijas no estiman sin
duda su dignidad.

---Señora, yo quiero verla y hablarla---prosiguió Asunción con
suplicante acento.---Si hay en ella pecado, estoy segura de que me lo
confesará. Si no le hay, como creo, tendré la dicha de descubrir la
verdadera causa de su fuga, y reconciliarla con la familia.

---No pienses en eso. Que cada cual se entienda con su conciencia. Si tú
a fuerza de devoción y reconcentración, y gracias también al rigor de mi
prudente autoridad has logrado elevar tu alma a cierto grado de
beatitud, concedido a pocos, no te achiques empeñándote en disculpar a
los demás. La perfecta virtud anda muy escasa por el mundo. Si en
algunas honestas moradas, inaccesibles a las profanidades de hoy, se
conserva encerrada como el más precioso tesoro, no debe contaminarse con
el roce de la desenvoltura. En infausta hora vino Inés a mi casa.
Renuncia a verla y a hablar con ella, mientras esté fuera de aquí. Tu
sublimada virtud debe quedar satisfecha con perdonarla.

---No: yo quiero verla, yo quiero ir allá---exclamó la joven derramando
de súbito un torrente de lágrimas.---Yo quiero verla. Inés es una buena
alma. Estamos engañados. Ella no puede haber cometido ninguna mala
acción. Señora, lord Gray no la ama ni puede amarla. Quien lo dijese es
un infame que merece arder en el infierno por toda la eternidad,
traspasada la lengua con un hierro candente.

---As:nción, sosiégate---dijo la madre con menos severidad, al notar que
la infeliz muchacha padecía una febril excitación, semejante a los
primeros síntomas de una enfermedad grave.---¿A qué tanto empeño?
Siempre eres lo mismo\ldots{} Tus manos arden\ldots{} los ojos se te
quieren saltar de la cara; estás lívida\ldots{} Hija, tu piedad exaltada
de algún tiempo a esta parte te hace mucho daño, y es preciso no olvidar
la salud del cuerpo. Tus largos insomnios cavilando en las cosas santas,
tus meditaciones sin fin, la viva pasión que te consume por lo
religioso, te han marchitado en pocos días.

Y luego, dirigiéndose a mí, añadió:

---Yo:no quisiera que se extremara tanto en sus devociones; pero no se
la puede contener. Su alma es muy vehemente, y una vez que logré
dirigirla al santo fin que me proponía, hase inflamado en una piedad
estupenda. Es un fuego abrasador su espíritu, no un vano soplo, y la
creo capaz de grandes cosas en la esfera de la vida mística que tan
celosamente ha abrazado.

---Por Dios y todos los santos, ruego a usted, señora, que me permita
ver a Inés. Es mi amiga, mi hermana. Yo tengo orgullo en su virtud, yo
me siento ofendida y lastimada por la mala opinión que hoy se tiene de
ella en esta casa. Quiero hacer una buena obra y volverle su honor. ¿Por
qué ha de intervenir en esto la justicia, si yo confío en que la traeré
a casa? La justicia es el escándalo\ldots{} Yo quiero ver a Inés, y
conseguiré de ella con una palabra más que toda la curia con una montaña
de papeles. Señora madre, esto que digo es inspiración de Dios, me salen
estas palabras del fondo del alma, siento dentro de mí un blando
susurro, como si la voz de un ángel me las dictara. No se oponga usted a
esta divina voluntad, pues voluntad divina es en este momento la mía.

La señora de Rumblar reflexionó, miró al techo, después a mí, luego a su
hija, y al fin exhalando un hondo suspiro, dijo:

---La dignidad y entereza tienen su límite, y la razón no puede a veces
resistir a las súplicas del sentimiento y la piedad reunidos. Asunción,
puedes ir a ver a Inés. Te llevará D. Paco.

La muchacha corrió ligera a vestirse.

---Pues como indiqué a usted, señora condesa\ldots---dije, reanudando mi
interrumpida conferencia diplomática.

---Haga usted cuenta de que no ha indicado nada, caballero. Todo es
inútil. Si el objeto de su visita es traerme recados o proposiciones de
la condesa, puede usted retirarse.

---La señora condesa se apresura a conceder a usted\ldots{}

---No quiero que me conceda nada. El jefe de la familia es la señora
marquesa de Leiva, y a estas horas ha tomado todas las providencias
necesarias para que todo vuelva a su lugar. Nada me corresponde hacer.

---¡La señora condesa está tan arrepentida de aquellas palabras!

---Que Dios la perdone\ldots{} Mi responsabilidad está a
cubierto\ldots{} ¿Pero a qué estos artificios, Sr.~de Araceli? ¿Cree
usted que no le comprendo?

---Señora, no hay artificio en lo que digo.

---Vamos, que a mí no se me engaña fácilmente. ¿Me faltará entendimiento
para comprender que todos esos supuestos recados de la condesa, son
pretexto que usted toma para entrar aquí y ver a mi hija Presentación,
de quien está tan enamorado?

---Señora, la verdad, no había pensado\ldots{}

---Un ardid amoroso\ldots{} en efecto, no es ningún crimen. Pero ha de
saber usted que he destinado a mi hija al celibato. Ella no quiere
casarse\ldots{} Además, aunque de mis repetidos informes resulta que no
es usted mala persona, no basta\ldots{} porque, veamos, ¿quién es
usted?\ldots{} ¿de dónde ha salido usted?

---Creo que del vientre de mi madre.

---Bueno será, pues, que renuncie a sus locas esperanzas.

---Señora, usted padece una equivocación.

---Yo sé lo que digo. Ruego a usted que se retire.

---Pero\ldots{} si me permitiera usted que acabara de exponerle\ldots{}

---Ruego a usted que se retire---repitió con grave acento.

Me retiré, pues, y en el corredor, una puerta se entreabrió para dejarme
ver el lindo rostro de Presentación y una blanca manecita que me
saludaba.

\hypertarget{xxx}{%
\chapter{XXX}\label{xxx}}

Poco después entraba en casa de doña Flora. Después de enterar a la
condesa del resultado de mi visita, dije a Inés:

---Asunción vendrá aquí. Ahora salía con D. Paco.

Un momento después, Asunción entró y las dos amigas se abrazaban
llorando. Salimos del gabinete Amaranta y yo, dejándolas solas para que
hablaran a su gusto; pero la condesa apostándose tras de la puerta, me
dijo con malicioso acento:

---Yo me quedo aquí para oírlo todo. Será curioso lo que hablen. Ya
sabes que en palacio he realizado grandes cosas escuchando detrás de las
cortinas.

---No es ningún negocio de Estado lo que van a tratar. Yo me voy.

---Quédate, necio, y oye\ldots{} Por no querer oír rompimos las
amistades en el Escorial\ldots{} Considera que han de hablar algo de
ti\ldots{}

Verdad es que si la delicadeza me ordenaba cerrar los oídos, la
curiosidad me impulsaba a abrirlos. Venció la curiosidad, mejor dicho,
venció la pícara Amaranta, que no podía dejar de ser cortesana. Las
muchachas hablaban en alto y lo oímos todo, y aun veíamos algo.

---No quería mamá que te viera, Inés---exclamó Asunción.---¡Qué raro
acontecimiento! Yo me despedí creyendo no verte más\ldots{} y ahora yo
estoy en casa y tú fuera. Hipócrita, tan preparado lo tenías, y no me
habías dicho nada.

---Te equivocas---repuso Inés,---yo no he salido como tú\ldots{} Pero no
quiero acusarte ahora, puesto que arrepentida de tu gran falta, volviste
a casa de tu madre. ¿Has conocido tu error, has abierto los ojos
comprendiendo el abismo de perdición en que ibas a caer, en que quizás
has caído ya?

---No sé lo que me pasa---exclamó Asunción apretando las manos de su
amiga.---Estoy horrorizada de lo que hice. Me volví loca, se me
encendieron en la imaginación unas llamas que no me dejaban vivir, y
conociendo el mal me era imposible evitarlo. Lord Gray ha tiempo que
quería sacarme de la casa; yo me resistía; mas al fin tanto pensé en
ello, tanto discurrí sobre aquel gran pecado a que él me quería inducir,
que se me clavó dentro de la cabeza la idea de cometerle, y sin saber
cómo lo cometí. ¿Por qué no te echaste en mis brazos para impedirme
salir? Ahora vengo a que me fortalezcas. Yo no puedo vivir lejos de ti;
y si desde mucho antes no caí en el lazo, lo debo a tu buena amistad.
¿Nos separaremos ahora? Entonces voy a ser muy desgraciada, querida mía.
Vuelve a casa, por Dios, y yo te juro que lucharé con todas las fuerzas
de mi alma para olvidar a lord Gray, como tú deseas.

---Yo no podré lograr ahora lo que antes no logré---repuso
Inés.---Asunción, entra en el convento mañana mismo. Cuando traspases la
puerta de la santa casa, deja fuera todos los pensamientos de este
mundo, pide a Dios que te libre de la gran enfermedad que padece tu
alma, procura formarte de nuevo y ser otra mujer diferente de la que hoy
eres.

---¡Ay!---exclamó la otra con dolor, arrodillándose delante de su
amiga.---Todo eso lo he intentado; pero cuanto más he querido no pensar
en él, más he pensado. ¿De qué me vale rezar, si no puedo representarme
imagen ninguna de Dios ni de santo que sea distinta de la suya?\ldots{}
¡Ay, Inés! Tú sabes muy bien la vida que llevamos en casa de mi madre;
tú sabes muy bien la espantosa soledad, tristeza y fastidio de nuestra
vida. Tú sabes muy bien que allí quiere una rezar y no puede, quiere una
trabajar y no puede, quiere una ser buena y no puede. Obligadas por el
rigor de mi madre, trabajan las manos, pero no el entendimiento; reza la
boca, pero no el alma; se ciegan y abaten los ojos, pero no el
espíritu\ldots{} Las mil prohibiciones que por todas partes nos
entorpecen, despiertan en nuestro pecho ardientes curiosidades. Ya sabes
que todo lo queremos saber, todo lo averiguamos y de todo hacemos un
objeto de afanes e inquietudes. Como sabemos disimular, vivimos en
realidad con dos vidas, una para mamá y otra para nosotras mismas; una
vida, acá para una sola, y que tiene sus pesares y sus delicias\ldots{}
Como nos apartan del mundo, nosotras nos hacemos un mundito a nuestro
modo, y echando fuego, mucho fuego al horno de la imaginación, allí
forjamos todo lo que nos hace falta. Ya lo ves, amiga. ¿Tengo yo la
culpa? Si no lo podemos remediar, si se nos ha metido dentro un demonio,
un demonio grandísimo, Inés, al cual no es posible echar fuera.

---Tú y tu hermana seréis muy desgraciadas.

---Sí: desde que éramos chiquitas, mamá nos asignó a cada una el puesto
que habíamos de tener en la sociedad: yo monja, mi hermana nada. A mí me
educaron para el claustro; a mi hermana la criaron para no ser nada.
Nuestro entendimiento, nuestra voluntad, no podía apartarse ni tanto así
del camino que se les había trazado; a mí el camino del monjío, a
Presentación el camino de no ser nada. ¡Ay, qué niñez tan triste! No nos
atrevíamos a decir, ni a desear, ni siquiera a pensar cosa alguna que
antes no estuviera previsto e indicado por mamá. No respirábamos en su
presencia, y nos infundían tanto, tanto pavor sus mandatos y
reprimendas, que nos era imposible vivir. ¡Ay, para poder vivir nos fue
preciso engañarla, y la engañamos!\ldots{} Dios, o no sé quien, nos
inspiraba un día y otro mil ingeniosidades, y se desarrolló en las dos
un talento superior para el engaño. Yo me esforzaba, sin embargo, en
tener devoción, y pedía a Dios que me diera fuerzas para no mentir y que
me hiciera santa; yo se lo pedía todas las noches cuando me quedaba sola
y podía rezar con el corazón. Delante de mamá no rezaba sino con los
labios\ldots{} Pues bien; en cierta época de mi vida llegué a conseguir
lo que a Dios pedía; llegué a aficionarme a las cosas santas; llegué a
sentir un entusiasmo, una exaltación religiosa semejante a la que ahora
siento por muy distinto objeto. Me consideraba feliz y pedía a la Virgen
que conservara en mí tan agradable estado. Entonces me perfeccioné por
algún tiempo, se acabaron los disimulos y tuve la gran satisfacción de
hablar repetidas veces con mi madre sin decir cosa alguna que no saliese
de mi corazón. Raudales de verdad, de fe, de amor apacible y místico a
los santos y santas brotaban de él. Yo dije: «¡Qué fortuna he tenido en
que me destinaran al claustro!» Mis insomnios eran dulces y placenteros,
y mi imaginación era como un celaje poblado de angelitos. Cerraba los
ojos y veía a Dios\ldots{} sí, a Dios, no te rías; a Dios mismo, con su
barba blanca y su capa\ldots{} pues, como le pintan\ldots{}

---Todo eso duró hasta que viste a lord Gray con su pelo rubio y su capa
negra\ldots{} pues, como es---dijo Inés.

---Me lo has quitado de la boca---prosiguió Asunción, siempre de
rodillas y con los brazos apoyados en los de su amiga.---Lord Gray fue a
casa; yo le miré y dije para mí que se parecía a un San Miguel que está
pintado en mi devocionario. Le dijeron que yo era muy piadosa y él hizo
demostraciones de gran admiración. Después, en las noches sucesivas,
empezó a contar las maravillosas aventuras de sus viajes, y yo le oía
con más religiosidad que si fuera el primer predicador del mundo
narrando las hermosuras del cielo. En aquellas noches yo no veía
alrededor de mí más que tigres del África, cataratas de América,
pirámides de Egipto y lagunas de Venecia. Estaba encantada y bendecía a
Dios por haber creado tantas cosas bellas, incluso a lord Gray. ¡Oh!
Lord Gray no se apartaba de mi imaginación. Al sentir sus pasos me era
difícil disimular la alegría; si tardaba me ponía triste; si hablaba con
vosotras, y no conmigo, me moría de rabia\ldots{} Le decían siempre que
yo era muy piadosa; ya recordarás que él me alababa mucho por esto. Mamá
nos permitía a las tres que habláramos con él. Con el pretexto de la
piedad, me decía mil cosas sobre asuntos de religión delante de
vosotras. Una noche que pudo hablarme a solas me dijo que me
amaba\ldots{} Yo sentí un sacudimiento; me pareció que el mundo se había
abierto en dos pedazos debajo de nosotras. Le miré y él clavaba los ojos
en mí. Estaba fascinada y no acertaba a contestarle\ldots{} Todas las
noches hablaba, como sabes, de cosas santas; con dificultad me decía
algunas palabras a solas; me preguntó durante tres noches seguidas si le
amaba, y a la tercera noche le contesté que sí\ldots{} Tú sabes muy bien
cómo nos entendíamos. Lord Gray me dijo: «Yo hablaré con Inés cerca de
ti. Pon atención a lo que le diga y haz cuenta de que te lo digo a ti.
Habla tú con tu hermano y procura contestarme con palabras dirigidas a
él\ldots» Teníamos además mil señales. Tú eras tan buena que te
conformaste con tu papel. Ojalá no hubieras sido tan condescendiente.
Cuando lord Gray me arrojaba cartas por la ventana y tú te apropiabas la
culpa para librarme de las crueles reprensiones, lejos de detenerme en
la pendiente me hacías precipitar más por ella. Nada conoció ni ha
conocido mamá; ¡ojalá lo conociera, aunque me hubiese matado!\ldots{}
¿Te acuerdas del día en que fui con ella al convento del Carmen,
convidadas por fray Pedro Advíncula para ver desde una tribuna la
función de la Virgen? ¡Ay! Después de la función, un lego nos llevó a
ver la sala de capítulo. No sé cómo, ni por qué causa me encontré
separada de los demás en una celdita sombría. Tuve miedo\ldots{} de
repente se me presentó lord Gray, quien me estrechó en sus brazos
repitiéndome con ardientes palabras que me quería mucho. Fue un segundo
y nada más, pero en aquel segundo lord Gray me dijo que me era forzoso
partir con él, porque si no moriría de desesperación\ldots{}

---Nada de eso me habías dicho.

---Te tenía miedo. Verás lo demás. Me reuní al instante con mi madre y
con el lego. Aquella súplica, o más bien que súplica mandato de huir con
él, se me clavó en el pensamiento como una espina. No dormía, no vivía,
no pensaba más que en aquello. Me parecía un delito horroroso: echaba de
mí esta idea y cuando me encontraba sin ella salía volando a buscarla,
porque sin ella no podía vivir\ldots{} No creas que aborrecí la
devoción, al contrario. La meditación era mi delicia y meditando era
feliz\ldots{} ¡Ay! Lord Gray en todas partes; lord Gray en los altares
de la iglesia, en el de mi casa; lord Gray en el breve espacio de calle
y de mundo que se nos permitía ver desde nuestro cuarto; lord Gray en
mis rezos, en mi libro de oraciones, en la oscuridad, en la luz, en el
bullicio y en el silencio. Las campanas tocando a misa me hablaban de
él. La noche se llenaba toda con él. ¡Oh, Inés de mi corazón! ¡Cuán
desgraciada soy! ¡Tener esta enfermedad en el espíritu y no poderla
desechar, tener esta fragua de pensamientos en el cerebro y no poder
echarle agua para que se apague\ldots!

Breve rato permanecieron las dos amigas en silencio y después Asunción
prosiguió de este modo:

---Nos comunicábamos al fin por un medio que tú no conociste ni llegaste
a sospechar. Parece imposible que por tanto tiempo pueda guardarse
secreto tan peligroso sin que por nadie sea descubierto. Yo le había
dicho que si por indiscreción o vanidad suya alguna persona, cualquiera
que fuese, llegaba a conocer nuestro secreto, le aborrecería\ldots{}
Después del día en que hablé con él en las Cortes, cuando se empeñó en
que le habíamos de seguir a bordo de no sé qué barco, y al fin nos envió
a casa con fray Pedro Advíncula; después de aquel día, digo, no le había
vuelto a ver\ldots{} Mi madre sospechaba de ti y le había prohibido
entrar en casa. ¿Recuerdas aquella anciana pordiosera que iba a casa a
vender rosarios? Pues ella me traía sus recados y le llevaba los míos.
Yo le escribía poniendo ciertos signos con lápiz en una hoja arrancada
de la \emph{Guía de Pecadores o del Tratado de la tribulación}; de modo
que el gran fray Luis de Granada y el padre Rivadeneyrahan sido nuestras
estafetas. Él me decía cosas hermosísimas y apasionadas que más me
arrebataban y confundían. Me pintaba su infelicidad lejos de mí y las
grandes dichas que Dios nos tenía reservadas. Por algún tiempo dudé. Yo
creo que viéndole, hablándole, o distrayendo con el trato de diversas
gentes mi espíritu, se habría aplacado la efervescencia, el bullicio, la
borrasca que yo sentía dentro de mí; pero ¡ay!, el largo encierro, la
soledad, la idea de sepultarme para siempre en el claustro me
perdieron\ldots{} Inés, figúrate que el corazón se destroza y se oprime,
que con la opresión de la naturaleza toda, alma y cuerpo estallan;
figúrate que se siente por dentro una iluminación, una inquietud no
comparable a las demás inquietudes, porque es la sed del espíritu que
quiere saciarse, una quemazón que crece por grados, un mareo que
desfigura todo cuando nos rodea, un impulso, un frenesí, una necesidad,
porque necesidad es la de romper el cerco de hierro que nos estrecha;
figúrate esto, y me comprenderás y me disculparás\ldots{} Yo decía: «Sí,
Dios mío, me marcharé con él, me marcharé.» Momentos de alegría loca
sucedían a otros de tristeza más negra que el purgatorio. Glorias e
infiernos se sucedían rápidamente unos tras otros dentro de mi pecho.
Dudaba, deseaba y temía, hasta que un día dije: «Sé que me condenaré,
pero no me importa condenarme\ldots,» y después me ponía a llorar
pensando en la deshonra de mi familia. Por último, pudo más mi amor que
todas las consideraciones y me decidí. Lord Gray por unos moldes de cera
que le envié, falsificó las llaves de la casa, le escribí fijando hora,
fue\ldots{} salí\ldots{} Pero ¡ay!, al verme fuera de casa, parece que
se me cayó el cielo encima con todas sus estrellas\ldots{} lord Gray me
llevó a una casa que está muy cerca de la nuestra, en la calle de la
Novena\ldots{} No era aquella su vivienda. Salió una señora de edad a
recibirnos. Yo me sentí acongojada y aturdida, empecé a llorar y pedí
ardientemente a lord Gray que me llevase otra vez a mi casa. Quiso
consolarme; el sentimiento del honor se encendió en mí con inusitada
fuerza, y la vergüenza me inflamaba el alma como momentos antes la
pasión. Deseé la muerte y busqué un arma para extinguir mi vida; lord
Gray fingió enojarse o se enojó realmente. Díjome algunas palabras
duras. Prometí amarle con más vivo cariño si me volvía a mi casa. Viendo
que no accedía a mis súplicas, grité, acudió la señora anciana, diciendo
que la vecindad se había alarmado y que nos fuéramos a otra parte.
Irritose lord Gray y amenazó a aquella señora con ahorcarla. Después
pareció conformarse con mi deseo, y dándome mil quejas llevome sin
dilación a mi casa. Por el camino me aseguró que partiría pronto para
Inglaterra y que le concediera otra entrevista fuera de casa. Yo se lo
prometí, porque al paso que me aterraba la idea de mi deshonor, me hacía
muchísimo daño su determinación de partir para Inglaterra\ldots{} ¡Ay,
Inés qué noche! Entré en casa llena de miedo. Me parecía ver a mi madre
esperándome en la escalera con una espada de fuego\ldots{} subí
temblando\ldots{} Tardé más de una hora en volver a mi cuarto, porque no
andaba, sino que me arrastraba lentamente para no hacer ruido. Al fin,
llegando a la alcoba, corrí a tu cama para confesártelo todo y no
estabas allí. Figúrate cuál sería mi confusión.

---Yo desperté---dijo la otra.---Creí sentir pasos dentro de la casa. Te
vi salir, y por un instante el temor no me permitió hacer ningún
movimiento ni tomar resolución alguna. Quise después correr tras de ti;
yo sabía que tenía poder bastante para destruir tu alucinación, y fiaba
en el cariño que nos profesábamos, en lo que me debes, en la deuda que
tienes conmigo por haberte librado de las sospechas de tu madre. La idea
de tu deshonor me volvía loca\ldots{} Salí en busca tuya. Lo demás no
necesitas saberlo. Yo no soy esclava de la autoridad de doña María como
lo eres tú; aquella casa no es la mía; mi casa es esta. Asunción,
querida amiga y hermana mía, nos separamos hoy quizás para siempre.

---No te separes de mí---exclamó Asunción abrazando a su amiga y
besándola con ardiente cariño.---Si te separas, no sé qué será de mí.
Recuerda lo que hice anoche\ldots{} Inés, no me dejes. Vuelve a mi casa,
y prometo no hacer cosa alguna sin tu permiso, esclavizando mi
pensamiento al tuyo, y lograré adquirir una parte al menos de la santa
serenidad que te distingue. He venido sólo a rogarte que vuelvas a mi
casa. Prométeme que volverás.

---Por distintos caminos nos lleva Dios a ti y a mí, Asunción. Por de
pronto no admitas cartas, ni avisos, ni recados de lord Gray. Levántate
a la altura de tu dignidad, abraza con resignación la vida del claustro,
y dentro de algún tiempo te verás libre de ese gran peso.

---No, no puedo. La vida del claustro me aterra. ¿Sabes por qué? Porque
tengo la seguridad de que en el convento he de amarle más, mucho más. Lo
sé por experiencia, sí: la soledad, el mucho rezar, las penitencias, las
meditaciones, las vueltas y revueltas y dolorosos giros del pensamiento,
más y más avivan en mí la pasión que me quema. Lo sé muy bien, lo veo,
lo toco. Yo he amado a lord Gray porque en mis solitarias devociones se
ha apoderado de mi espíritu como el demonio tentador\ldots{} No, no iré
al claustro, porque sé que lo tendré siempre delante, mezclado con
aquella dulce poesía del coro y el altar. ¡Ay, amiga mía! ¿Creerás esto
que te digo? ¿Creerás esta profanación horrible? Pues sí, es verdad. En
la iglesia ha tomado cuerpo esta insensata inclinación. Tal efecto hace
en mi espíritu turbado todo lo que se refiere a devociones y piedades,
que siempre que escucho el son de un órgano, tiemblo de emoción; las
campanas de la iglesia hacen palpitar mi pecho con ardiente viveza; la
oscuridad de los templos me marea, y Jesucristo crucificado no puede
serme amable si no me lo presento con el mismo rostro que veo en todas
partes\ldots{} Esto espanta, ¿no es verdad? Pero no puedo remediarlo. Yo
creo que esto es una enfermedad. ¿Tendré yo un mal incurable? Ojalá me
muera mañana de él. Así descansaría\ldots{} No, no quiero claustro.
Quiero distraerme con el trato de multitud de gentes, ver diversidad de
espectáculos, visitar el mundo, la sociedad, asistir a tertulias donde
se hable de muchas cosas que no sean lord Gray: quiero que mi
pensamiento se enrede aquí y allí, se desparrame pasando y repasando por
distintos caminos, para dejarse un vellón de lana en cada flor, en cada
espina. Lo que me ha de curar es el mundo, amiga querida, es el mundo
con todo lo bueno que encierra, la sociedad, la amistad, las artes, el
viajar, el mucho ver y el mucho oír; que verdaderamente, aunque mi madre
crea lo contrario, la mayor parte de lo que se ve y oye en el mundo es
honrado, lícito y provechoso\ldots{} Apártenme de la soledad, que es
causa de mi perdición; apártenme de las meditaciones, del cavilar, de
este perenne volteo y constante rodar sobre el eje de una sola idea. Si
he de curarme, no me curarán los conventos. Querida amiga, segura estoy
de que si entro en él, amaré más locamente a lord Gray, porque no habrá
cosa alguna que lo aparte de los vigilantes y calenturientos ojos de mi
espíritu; y si ese hombre se empeña en perseguirme aun en la casa de
Dios, como sabe hacerlo, no podré guardar la santidad de mis juramentos,
y rompiendo rejas y votos, me asiré a la primera cuerda que ponga en la
ventana de mi celda para arrojarme a la calle. Yo me conozco, querida
mía; sé leer claramente en este oscuro libro de mi alma, y no me
equivoco, no.

Oyendo estas palabras en boca de la infeliz joven, al paso que
compadecía su desventurada pasión, admiraba la gran perspicacia de su
entendimiento.

---Pues ten valor. Di a tu madre que no quieres ser monja---indicó Inés.

---Ayudada por tu amistad, podría hacerlo. Sola no me atrevo. Ella
considerará esto como una deshonra, y entonces tendré el claustro en
casa, porque me encerrará para siempre.

---Todo eso puede vencerse. Principia por rechazar a lord Gray.

---Lo haré si no le veo, si no me persigue\ldots{}

Asunción pronunciaba estas palabras, cuando sentimos los pasos de lord
Gray.

---¡Es él!---dijo con terror.

---Ocúltate y sal de la casa.

Amaranta hizo pasar a lord Gray a una estancia inmediata y al instante
me llamó a su lado. El inglés afectaba tranquilidad; mas la condesa
adivinando sus propósitos, le desconcertó al momento.

---Ya sé a que viene usted---le dijo.---Sabe que Asunción ha entrado en
mi casa\ldots{} Por Dios, lord Gray, retírese usted. No quiero tener
nuevas ocasiones de disgusto con doña María.

---Discreta amiga mía---repuso él con vehemencia.---Usted me juzgue mal.
¿Impedirá usted que me despida de ella? Dos palabras nada más. ¿Saben
que me voy esta noche?

---¿Es de veras?

---Tan cierto como que nos alumbra el sol\ldots{} ¡Pobrecita
Asunción!\ldots{} También ella se alegrará de verme\ldots{} Vamos, no
salgo de aquí sin decirle adiós\ldots{}

---Francamente, milord---indicó Amaranta.---No creo en su partida.

---Señora, aseguro a usted que partiré de madrugada. Me ha detenido tan
sólo la broma que pensamos dar a Congosto\ldots{} Sea testigo Araceli de
lo que digo.

La condesa sin aguardar más, abrió la mampara, y las dos muchachas
aparecieron ante nosotros.

Asunción no podía ocultar la angustia que la dominaba y quiso retirarse.

---¿Se marcha usted porque estoy aquí?---dijo secamente lord
Gray.---Pronto saldré de Cádiz y de España, para no pisar más esta
tierra de la ingratitud. Los desengaños que aquí he padecido me impelen
con fuerza a huir, aunque mi corazón no ha de encontrar ya reposo en
ninguna parte.

---Asunción no puede detenerse para oírle a usted---dijo Inés.---Tiene
que marcharse a su casa.

---¿No merezco ya ni dos minutos de atención?---afirmó con amargura el
noble lord.---¿Ya no se me concede ni el favor de una palabra?\ldots{}
Está bien, no me quejo.

---Ahora parece indudable que parte---dijo Amaranta.

---Señora, adiós---exclamó lord Gray con emoción profunda, verdadera o
fingida.---Araceli, adiós; Inés, amigos míos, procuren olvidar a este
miserable. Y usted, Asunción, a quien sin duda debo haber ofendido,
según el encono con que me mira, adiós también.

La infeliz se deshacía en lágrimas.

---Había solicitado de usted el último favor, una entrevista para
despedirme de la que tanto he amado, pero no espero conseguirlo. He sido
un insensato\ldots{} Ha hecho usted bien en cobrarme de pronto ese
aborrecimiento que me están revelando sus bellos ojos\ldots{} ¡Miserable
de mí, he aspirado a lo que me era tan superior! En mi demencia juzgué
posible apartar esta noble alma de la piedad a que desde el nacer se
inclina; aspiré a lo imposible, a luchar con Dios, único amante que cabe
en la inconmensurable grandeza de ese corazón\ldots{} Adiós, vuelva
usted a sus santidades, remóntese usted a aquellas celestiales alturas,
de donde este infame quiso hacerla descender. Entre usted en el
claustro\ldots{} entre usted\ldots{} Perdóneme Dios mis arrebatados
pensamientos\ldots{} cada cual a su puesto. Ángeles al cielo, miseria y
debilidad a la tierra\ldots{} Antes amor, locura, ardientes arrebatos;
ahora respeto, culto. Mañana, como ayer, vivirá usted en mi corazón;
pero ahora, santa mujer, está usted dentro de él canonizada\ldots{}
Adiós, adiós.

Y apretando calurosamente las manos de la joven, partió con tales modos,
que todos le creíamos con el corazón despedazado y tuvimos lástima de
él.

Poco después Asunción, acompañada de su ayo, salió a la calle, y la
santa imagen, entrando en la casa materna, volvió a su altar.

~

Mis lectores creerán, juzgando a lord Gray por las palabras arriba
reproducidas, que el astuto seductor partía realmente renunciando a la
empresa frustrada en la célebre noche. ¡Qué error! Sigan leyendo un poco
más, y verán que aquella despedida, admirable y hábil recurso
estratégico empleado contra la alucinada muchacha, sirviole de
preparación para el hecho (catástrofe podemos llamarlo) consumado
aquella misma noche, y con el cual da fin la curiosa aventura que estoy
contando.

\hypertarget{xxxi}{%
\chapter{XXXI}\label{xxxi}}

Narraré punto por punto. Aconteció, pues, que cerca ya del oscurecer en
el siguiente día entraba yo con toda tranquilidad en casa de doña Flora,
cuando esta, Amaranta y su hija saliéronme al encuentro con gran
sobresalto y alarma.

---¿No sabes lo que ocurre?---dijo doña Flora.---El bribón de lord Gray
ha cargado con la santa y la limosna. La Asuncioncita ha desaparecido
anoche de la casa.

---Pero ha sido violentamente---dijo Inés,---porque D. Paco apareció
atado al barandal de la escalera. Ella debió de resistir\ldots{} A sus
gritos despertose doña María, pero cuando salieron ya estaban fuera.
Esta mañana, Presentación, hostigada por su madre, hizo confesión de los
amores de su hermana.

---No me digan a mí que ha resistido---objetó doña Flora;---lord Gray es
muy galán y muy lindo mozo\ldots{} ¿A qué vienen con
hipocresías?\ldots{} La niña se marchó con él porque le dio la gana.

---Doña María estará satisfecha de la formalidad de las
niñas\ldots---dijo Amaranta riendo.---Ahora repetirá su muletilla: «Yo
educo a mis hijas como me educaron a mí.»

---¿Pero se ha marchado lord Gray con ella?---pregunté.

---Se dispone a partir.

---Ahora acaba de estar aquí un capitán de navío, el cual me ha dicho
que milord ha fletado el bergantín inglés \emph{Deucalión}, que sale
mañana.

---¿Pero no corremos a impedirlo?---dijo Inés con gran zozobra.---Aún es
tiempo.

---Eso será de cuenta de doña María.

---Pero será forzoso avisarle que el \emph{Deucalión} sale esta noche y
que lo ha fletado lord Gray.

---Sí, es preciso avisárselo---repitió Inés con energía.---Iré yo misma.

---Gabriel irá al momento.

---¿Por qué no? Aunque doña María me arrojó ayer de su casa, no tengo
inconveniente en prestarle este servicio.

---Pero no pierdas tiempo\ldots{} Yo me muero de impaciencia---indicó
Inés.

---Ve pronto, que la niña se impacienta.

---Allá voy\ldots{} De veras no creí volver a poner los pies en aquella
casa\ldots{} ¿Conque el \emph{Deucalión?\ldots{}} Un bergantín
inglés\ldots{} Me parece que no les atraparán.

Corrí a la casa de Rumblar, y desde que entré todo me indicó que reinaba
allí la consternación más profunda. D. Diego y D. Paco estaban sentados
en el corredor, el uno frente al otro, mirándose como dos esfinges de la
tristeza, y en las manos del último los verdes cardenales indicaban el
suplicio de que había sido víctima. El infeliz anciano a ratos hendía
los aires con la ráfaga de sus fuertes suspiros, que habrían hecho
navegar de largo a un navío de línea. Cuando entré, levantáronse los
dos, y el ayo dijo:

---Vamos a ver si la encontramos ahora. Es el sétimo viaje\ldots{}

La condesa de Rumblar y su hija menor estaban escondiendo su dolor y
vergüenza en un gabinete inmediato a la sala, y en ésta la marquesa de
Leiva, atada por el reuma a un sillón portátil; Ostolaza, Calomarde y
Valiente sostenían viva polémica sobre el gran suceso. Cuando oí la voz
de la de Leiva, lleno de recelo, aunque sin arredrarme, dije para mí:

---Ahora va a ser la tuya, Gabriel. La marquesa te conocerá, con lo
cual, hijo, has hecho tu suerte.

Entré, sin embargo, resueltamente.

---De modo---decía la marquesa,---que un inglés se puede burlar
impunemente de toda España\ldots{}

---En la embajada---indicó Valiente,---rieron mucho cuando les conté lo
ocurrido, y dijeron: «Cosas de lord Gray.»

---Yo he afirmado siempre---dijo Ostolaza con petulancia,---que la
alianza con los ingleses sería a España muy funesta.

Yo corté de súbito el coloquio, diciendo:

---Traigo noticias de lord Gray.

La marquesa examinome de pies a cabeza, y luego, señalándome
impertinentemente con la muleta que sus doloridas piernas le obligaban a
usar, preguntó:

---¿Usted?\ldots{} ¿Y usted quién es?

---Es el Sr.~de Araceli---dijo Ostolaza con sonsonete desdeñoso.

---Ya\ldots{} ya conozco a este caballero---dijo la de Leiva con
malicia.---¿Sigue usted al servicio de mi sobrina?

---Me honro en ello.

---¿Viene usted de allá? ¿Inés está ya dispuesta a volver a su casa? Ya
sabrá que el gobernador de Cádiz va esta noche misma por ella\ldots{}

---No saben nada---repuse tan desconcertado como sorprendido.

---Creo que bajo el punto legal, la cosa no ofrecerá dificultad alguna,
¿no es verdad, señor de Calomarde?

---Absolutamente ninguna. La niña volverá a casa de usted, que es el
jefe de la familia, y cuantas sutilezas se aleguen en contrario no
tienen fuerza de derecho.

---Tal vez la señora condesa---dije,---alegue algún motivo que no esté
previsto.

---Todo está previsto; Sr.~Calomarde, ¿no es verdad? Y agradézcame mi
sobrina que no he solicitado se dicte auto de prisión contra
ella\ldots{} Pero a esta fecha no nos ha dicho usted lo que anunciaba
con respecto a lord Gray. ¿En qué piensa usted, señor de\ldots{} de qué?

---De Araceli---repitió Ostolaza con el mismo sonsonete.

Muy brevemente les dije lo que sabía.

---Pues hay que avisar a la Comandancia de Marina---replicó la de Leiva
con viveza.---Plumas, papel\ldots{}

En aquel instante entró en la sala un personaje grave, al cual saludaron
todos con el mayor respeto. Era D. Juan María Villavicencio, gobernador
de la ciudad, varón estimabilísimo, buen patriota, instruido, algo
filósofo y hábil por demás en el conocimiento y trato de las gentes.

---Ya tenemos datos, Sr.~Villavicencio---dijo la marquesa, contándole lo
del \emph{Deucalión}.

---En este negocio, señora---respondió el funcionario bajando la
voz,---hay que andar con prudencia\ldots{} Antes de ocuparme de lord
Gray voy a cumplir el acto legal, en cuya virtud la Inesita volverá esta
noche a su casa.

El alma se me partió al oír esto.

---Pronto, pronto, amigo mío---dijo la reumática.---También temo que se
me escapen. La gente de esta casa se marcha por el escotillón, y esto
parece escenario de un teatro\ldots{} Y creímos que había sido robada
por lord Gray. La pícara se marchó sola\ldots{}

---En cuanto a lord Gray---dijo Villavicencio en tono dubitativo y con
cierto embarazo,---me parece que no podemos hacer nada contra él\ldots{}
La Asuncioncita volverá al lado de su madre o a donde la quieran llevar;
pero eso de prender y castigar a milord\ldots{}

---Pero\ldots{}

---Señora, no podemos chocar con la embajada\ldots{} Ya conoce usted las
circunstancias; Wellesley es quisquilloso\ldots{} la alianza\ldots{}

---¡Maldita sea la alianza!

---¡Y esto lo dice una dama española---exclamó Villavicencio con
entusiasmo,---el día en que nos llega la noticia de una gloriosa
batalla, de esa gran victoria, señores, ganada por españoles, ingleses y
portugueses en los campos de Albuera!

---¡Otra batalla!---exclamó la marquesa con hastío.---Siempre batallas,
y la guerra no se acaba nunca.

---Creo que ha sido muy sangrienta---dijo Calomarde.

---Como todas las que damos---repuso con orgullo Villavicencio.---Hemos
perdido cinco mil hombres y matado a los franceses más de diez
mil\ldots{} ¡Precioso resultado!\ldots{} Han muerto dos generales
franceses, dos ingleses, y de los nuestros han quedado heridos D. Carlos
España y el insigne Blake.

---De todo eso se deduce que no podemos hacer nada contra Gray---dijo
con disgusto la de Leiva.

---Nada, señora\ldots{} Se va a erigir un monumento a Jorge III\ldots{}
La embajada inglesa\ldots{} Wellesley\ldots{} ¡Oh!, esta batalla de la
Albuera estrechará más aún las relaciones entre ambos países.

---¡Gran victoria!---dijo Valiente.---En Extremadura nos envalentonamos
un poco.

---Pero está muy mal de la parte del Ebro. Tortosa ha caído ya en poder
del enemigo\ldots{}

---Traición, pura traición del conde de Alacha.

---También se han apoderado los franceses del fuerte de San Felipe en el
Coll de Balaguer.

---Pero aún resiste Tarragona.

---Y resistirá más todavía.

---Y de Manresa, ¿qué se ha dicho hoy?

---Ya es seguro que ha sido incendiada.

---Nada de eso nos importa por ahora---dijo la marquesa, interrumpiendo
la chispeante conversación patriótica.---En suma, Sr.~Villavicencio, si
milord se escapa\ldots{}

---¡Qué le hemos de hacer! Nadie sabe dónde está.

---Creo que esta noche se le podrá ver---dijo Valiente,---porque a las
diez se verificará, según he oído, entre lord Gray y D. Pedro del
Congosto una especie de desafío quijotesco con que espera reírse mucho
la gente.

---Bobadas\ldots{} En fin, señora marquesa, Wellesley me ha prometido
que la muchacha volverá, pero hay que dejar en paz a lord Gray\ldots{}
Señora marquesa, me llama mucho la atención este extraño caso. Soy
experto en ciertos asuntos, y creo que en el lance de que nos ocupamos
juega alguna persona que no es lord Gray.

---¿Lo cree usted? Yo opino que Inés se ha marchado sola.

---Pues yo creo que no.

---O con lord Gray. Ese señor inglés se propone desocupar mi casa.

---Algún otro pájaro, señora, algún otro pájaro ha enredado aquí, y no
pararé hasta averiguar quién es\ldots{} Los dos raptos tienen entre sí
íntima conexión.

---Busque usted, pues---dijo la marquesa,---a ese cómplice desconocido,
y haga caer sobre él todo el peso de la ley, si es que nada puede
hacerse contra lord Gray.

---Espero sacar mucho partido de mis averiguaciones esta noche.

---Verdaderamente---dijo Calomarde,---si ha de haber un choque con la
embajada inglesa, lo mejor es dar fuerte sobre el pobre cómplice si se
descubre, y decir: «aquí que no peco.»

---Así anda la justicia en España---objetó la de Leiva.

---Veremos lo que saco en limpio---dijo Villavicencio.---Vaya, señora
mía, me voy a hacer una visita de cumplido a la calle de la Verónica.
Creo que bastará mi autoridad\ldots{}

~

De pronto presentose D. Paco en la sala sofocado y jadeante, y exclamó:

---¡Ahí está, ahí está ya!\ldots{} al fin la encontramos.

---¿Quién?

---La señora doña Asuncioncita\ldots{} ¡Pobre niña de mi alma!\ldots{}
Está en la escalera\ldots{} No quiere subir\ldots{} ¡parece medio muerta
la pobrecita!\ldots{}

\hypertarget{xxxii}{%
\chapter{XXXII}\label{xxxii}}

Reinó sepulcral silencio, y miramos todos a la puerta del fondo por
donde apareció doña María. Con decoroso silencio, que no con lágrimas,
mostraba esta señora su honda pena. El color blanco de su cara habíase
convertido en una palidez pergaminosa; su frente estaba surcada de
repentinas arrugas, y los secos ojos tan pronto irradiaban el fulgor de
la ira como se abatían amortiguados. Pero otro incidente llamó la
atención más que el grave silencio y la amarillez y las arrugas, y fue
que sus cabellos, entrecanos algunos días antes, estaban enteramente
blancos.

---¡Está ahí!---repitió un sordo murmullo.

---¿Te negarás a recibirla?---dijo con emoción la marquesa, adivinando
los pensamientos de doña María.

---No\ldots{} que venga aquí---repuso la madre con energía.---Veré a la
que ha sido mi hija\ldots{} ¿La encontró usted? ¿Estaba sola?

---Sola, señora---exclamó llorando D. Paco.---¡Y en qué triste y
lastimoso estado! Los vestidos están rotos, en su preciosa cabecita
tiene varias heridas, y en su voz y ademanes demuestra el más grande
arrepentimiento. No ha querido subir, y yace exánime y sin fuerzas en la
escalera.

---Que entre---dijo la de Leiva.---La infeliz empieza a expiar su culpa.
María, pasó la ocasión del rigor y ha llegado el momento de la
benevolencia. Recibe a tu hija, y si acabó para el mundo, no acabe para
ti.

---Retirémonos para evitarle la vergüenza de verse delante de
nosotros---dijo Valiente.

---No: queden todos aquí.

---Sr.~D. Francisco---dijo doña María al ayo,---traiga usted a Asunción.

El ayo salió determinando fuertes corrientes atmosféricas con la
violencia de sus suspiros.

Bien pronto oímos la voz de Asunción que gritaba:

---Mátenme, que me maten: no quiero que mi madre me vea.

Por D. Diego y el ayo conducida, a intervalos suavemente arrastrada,
casi traída a cuestas, entró la infeliz muchacha en la sala. En la
puerta arrojose al suelo, y sus cabellos en desorden sueltos, le cubrían
la cara. Todos acudimos a ella, la levantamos, la consolamos con
palabras cariñosas; pero ella clamaba sin cesar:

---Mátenme de una vez. No quiero vivir.

---La señora doña María la perdonará a usted---le dijimos.

---No, mi madre no me perdonará. Estoy condenada para siempre.

Doña María, por largo tiempo llena de entereza y superioridad, comenzó a
declinar y su grande ánimo se abatió ante espectáculo tan lamentable.
Después de mucho luchar con la sensibilidad y el cariño materno, pugnó
por sobreponerse a este, y resueltamente exclamó:

---¿He dicho que la traigan aquí? No, me equivoqué. No quiero verla, no
es mi hija. Váyase a los lugares de donde ha venido. Mi hija ha muerto.

---Señora---exclamó D. Paco poniéndose de rodillas,---si la señora doña
Asuncioncita no se queda en la casa, usted se condenará. ¿Pues qué ha
hecho? Salir a dar un paseo. ¿Verdad, niña mía?

---No: ¡mi madre no me perdona!---gritó con desesperación la
muchacha.---Llévenme fuera de aquí. No merezco pisar esta casa\ldots{}
Mi madre no me perdona. Vale más que me maten de una vez.

---Sosiégate, hija mía---dijo la de Leiva.---Grande es tu culpa; pero si
no pued:s reconquistar el cariño de tu madre y la estimación de todos,
no serás abandonada a tu dolor. Levántate. ¿Dónde está lord Gray?

---No sé.

---¿Vino a buscarte con conocimiento y consentimiento tuyo?

La desgraciada se cubría el rostro con las manos.

---Habla, hija mía, es preciso saber la verdad---dijo la de Leiva.---Tal
vez tu culpa no sea tan grande como parece. ¿Saliste de buen grado?

La presencia de doña María se conocía por su respiración que era como un
sordo mugido. Luego oímos distintamente estas palabras que parecían
salir de la cavernosa garganta de una leona:

---Sí\ldots{} de grado\ldots{} de grado.

---Lord Gray---dijo Asunción,---me juró que al día siguiente abrazaría
el catolicismo.

---Y que se casaría contigo, ¡pobrecita!---dijo con benevolencia la
marquesa.

---Lo de siempre\ldots{} historia vieja---balbuceó Calomarde a mi oído.

---Señores---dijo Villavicencio,---retirémonos. Estamos aumentando con
nuestra presencia la confusión de esta desgraciada niña.

---Repito que se queden todos---dijo la de Rumblar con fúnebre
acento.---Quiero que asistan a los funerales del honor de mi casa.
Asunción, si quieres, no que te perdone, sino que tolere tu presencia
aquí, confiesa todo.

---Me prometió abrazar el catolicismo\ldots{} me dijo que marcharía de
Cádiz para siempre, si no\ldots{} Yo creí\ldots{}

---Basta---exclamó Villavicencio.---Que se retire a buscar algún reposo
esta criatura.

---Pero ese infame hombre la ha abandonado\ldots{}

---La ha arrojado de su casa---dijo D. Paco.

Múltiple exclamación de horror resonó en la sala.

---Esta mañana---añadió Asunción sacando difícilmente de su pecho el
aliento necesario para hablar,---lord Gray salió dejándome sola en la
casa. Yo temblaba de zozobra\ldots{} Entraron luego unas mujeres, unas
mujerzuelas\ldots{} ¡qué horrible gente!\ldots{} Con sus gritos me
desvanecieron y con sus manos me maltrataron. Todas se reían de mí y me
desgarraron los vestidos, diciéndome palabras ignominiosas\ldots{}
Bebían y comían en una mesa que el criado de milord les dispuso\ldots{}
disputaban unas con otras sobre cuál de ellas era más amada por
él\ldots{} Entonces comprendí el abismo en que había caído\ldots{} Lord
Gray volvió\ldots{} Le increpé por su vil conducta\ldots{} Estaba
taciturno y sombrío\ldots{} Tomó una chinela y con ella les azotó la
cara a aquellas viles mujeres\ldots{} Me colmó de cuidados. Me dijo que
me iba a llevar a Malta\ldots{} Yo me negué a ello y empecé a llorar
amargamente invocando el nombre de Jesús\ldots{} Volvieron las mujeres
acompañadas de hombres soeces; uno de ellos quiso ultrajarme. Lord Gray
le rompió la cabeza con una silla\ldots{} Corrió la sangre\ldots{} ¡Dios
mío, qué horror!\ldots{}

Deteníase a cada rato, y luego con gran esfuerzo seguía:

---Lord Gray me dijo después que él no podía hacerse católico, y que se
alegraba de que yo entrase en el convento para robarme. Quise salir y el
criado anunció la llegada de una señora\ldots{} ¡Oh! Entró una señora
principal que le llamó ingrato\ldots{} La señora se reía de mí\ldots{}
¡Qué hora, Dios mío, qué hora!\ldots{} La señora dijo que yo era la más
piadosa y devota señorita de todo Cádiz, y luego me rogó que encomendase
a lord Gray a Dios en mis oraciones\ldots{} La vergüenza me inflamaba, y
busqué un cuchillo para matarme\ldots{} Después\ldots{}

Estábamos todos conmovidos y aterrados con la patética relación de la
desgraciada niña, digna de mejor suerte.

---Después\ldots{} entraron unos hombres; ¡qué hombres! Vestían de
cruzados como don Pedro del Congosto, y venían a recordar a lord Gray
que este le había desafiado\ldots{} Entraron los amigos de lord Gray y
todos se rieron mucho del desafío con D. Pedro. Luego\ldots{} milord me
rogó de nuevo que partiese con él a Malta\ldots{} Yo le decía que me
hiciese el favor de matarme\ldots{} Reíase a carcajadas y jugando con un
puñal hacía como que me quería matar\ldots{} Me inspiraba tal horror que
huí de su lado\ldots{} Yo corrí por la casa dando gritos\ldots{} él se
reía\ldots{} un criado me dijo: «milord me ha mandado que la acompañe a
usted a su casa.» Salimos a la calle y en la puerta añadió: «No tengo
ganas de ir tan lejos: vaya usted sola,» y cerró la puerta\ldots{} Di
algunos pasos\ldots{} una mujer frenética que dijo haber perdido por mí
los favores de lord Gray, quiso castigarme\ldots{} ¡Ay!, yo estaba medio
muerta y me dejé castigar\ldots{} Libre al fin recorrí varias
calles\ldots{} me perdí\ldots{} yo buscaba la muralla para arrojarme al
mar\ldots{} al fin después de dar mil vueltas volví junto a la casa de
lord Gray\ldots{} Encontráronme D. Paco y mi hermano\ldots{} yo no
quería venir aquí\ldots{} pero me trajeron al fin a mi casa de donde
salí culpable, y a donde vuelvo castigada, pues las penas todas del
purgatorio y el infierno no son superiores a las que yo he padecido
hoy\ldots{} Aun así no merezco perdón. Mi falta es grande\ldots{} No
merezco más que la muerte, y pido a Dios que me la conceda esta noche
misma, para que ni un día más soporte la vergüenza y el deshonor que han
caído sobre mí. ¡Señora madre mía, adiós! ¡Hermana mía, adiós! ¡No
quiero vivir!

No dijo más y cayó desmayada en el pavimento.

Conmovidos y aterrados, contemplamos el semblante de doña María, que
reclinada en el sillón, con la barba apoyada en la mano, silenciosa,
ceñuda primero como una sibila de Miguel Ángel, y conmovida después,
pues también las montañas se quebrantan al sacudimiento del rayo,
derramó lágrimas abundantes. Parecía que su rostro se quemaba. Su llanto
era metal derretido.

---Hija mía---dijo la marquesa,---retírate a descansar\ldots{} Sr.~D.
Francisco, o tú, Diego, llévala a su cuarto.

El conmovedor espectáculo de la infeliz Asunción desapareció de nuestra
vista.

---Señoras---dijo Villavicencio,---tengo el alma despedazada, y me
retiro.

---Siento mucho\ldots{} pues\ldots---murmuró Ostolaza, y se retiró
también.

---He tenido un verdadero sentimiento\ldots---dijo Valiente, marchándose
tras el anterior.

---Por mi parte\ldots---indicó Calomarde saludando.---Si es preciso
entablar recurso\ldots{}

Se fueron todos. Yo me quedé, porque una fuerza irresistible me clavaba
en aquella sala, y no podía apartar el pensamiento del desolado cuadro
que había visto. Delante de mí estaba la de Rumblar en la misma actitud
en que antes la he descrito. El fenómeno de su llanto me llenaba de
asombro. A mi lado la marquesa de Leiva lloraba también.

Pero no estábamos solos los tres. Acababa de entrar una figura
estrambótica, un mamarracho de los antiguos tiempos, una caricatura de
la caballería, de la nobleza, de la dignidad, del valor español de otras
edades. Mirando aquella figura de sainete que se presentaba tan
inoportunamente, dije para mí:

---¿Qué vendrá a hacer aquí D. Pedro del Congosto? ¿Si creerá que sus
caballerías ridículas sirven de alguna cosa en estas circunstancias?

La de Leiva abrió los ojos, vio al estafermo, y como si no diera
importancia alguna a su persona, volviose a mí y me dijo:

---¿Qué piensa usted de lord Gray?

---Que es un infame, señora.

---¿Quedará sin castigo?

---No quedará---exclamé arrebatado por la ira.

D. Pedro del Congosto dio algunos pasos, púsose delante de doña María, y
alzando el brazo, con voz y gesto que al mismo tiempo parecían trágicos
y cómicos, habló así:

---Señora doña María\ldots{} ¡esta noche!\ldots{} ¡a las once!\ldots{}
¡en la Caleta!

---¡Oh! ¡Gracias a Dios!---exclamó la noble señora levantándose con
ímpetu.---Gracias a Dios que hay en España un caballero\ldots{} Cuatro
personas han presenciado el lastimoso cuadro de la deshonra de mi hija,
y a ninguno se le ha ocurrido tomar por su cuenta el castigo de ese
miserable.

---Señora---dijo Congosto con voz hueca, que antes que risa, como otras
veces, me produjo un espanto indefinible.---Señora, lord Gray morirá.

Aquellas palabras retumbaron en mi cerebro. Miré a D. Pedro y me pareció
trasfigurado. Aquel espantajo, recuerdo de los heroicos tiempos, dejó de
ser a mis ojos una caricatura desde el momento en que me lo representé
como providencial brazo de la justicia.

---No es usted, D. Pedro---dijo con incredulidad la de Leiva,---quien ha
de arreglar esto.

---Señora doña María---repitió el estafermo sublimado por una alta idea
de su propio papel, por la idea de la hidalguía, del honor, de la
justicia---¡esta noche!\ldots{} ¡a las once!\ldots{} ¡en la Caleta! Todo
está dispuesto.

---¡Oh! Bendita sea mil veces la única voz que ha sonado en mi defensa
en esta sociedad indiferente. Abominables tiempos, aún hay dentro de
vosotros algo noble y sublime.

Esto que en otras circunstancias hubiera sido ridículo, tratándose de D.
Pedro, en aquellas me hacía estremecer.

---Bendito sea mil veces---continuó doña María,---el único brazo que se
ha alzado para vengar mi ultraje en esta generación corrompida, incapaz
de un sentimiento elevado.

---Señora---dijo D. Pedro,---adiós\ldots{} voy a prepararme.

Y partió rápidamente de la sala.

---María---dijo la de Leiva a su parienta,---sosiégate; debes procurar
dormir\ldots{}

---No puedo sosegar---repuso la dama.---No puedo dormir\ldots{} ¡Oh Dios
mío! Si permites que el miserable quede sin castigo\ldots{} Si vieras,
mujer\ldots{} siento una salvaje complacencia al recordar aquellas
palabras «esta noche\ldots{} a las once\ldots{} en la Caleta.»

---No esperes de D. Pedro más que ridiculeces\ldots{} Sosiégate\ldots{}
Han dicho aquí que el desafío de D. Pedro con lord Gray era una función
quijotesca. ¿No es verdad, caballero?

---Sí, señora---repuse.---Son ya las diez\ldots{} Soy amigo de lord Gray
y no puedo faltar.

Respetuosamente me despedí de ellas y salí. Detúvome en la escalera D.
Diego, que a toda prisa y muy sofocado subía, y me dijo:

---Gabriel, ahí me traen otra vez a la buena alhaja de doña Inesita.

---¿Quién?

---El gobernador. Esta noche todas las ovejas descarriadas vuelven al
redil\ldots{} Vengo de allá\ldots{} si vieras. La condesa ha llorado
mucho y se ha puesto de rodillas delante de Villavicencio; pero no pudo
conseguir nada. La ley y siempre la ley. Si es lo que yo digo: la
ley\ldots{} Por supuesto, chico, no puedo negarte que me dio lástima de
la pobre condesa. Lloraba tanto\ldots{} Inés estaba más serena y se
conformaba. Aguárdate y la verás llegar. Sin embargo, más vale que no
parezcas en tu vida por aquí. Villavicencio quiso averiguar el cómo y
cuándo de la fuga de Inés, y allá le dijeron que la sacaste tú de la
casa. Te anda buscando porque no te conoce. Dice que eres cómplice de
lord Gray y el verdadero criminal. Calumnia, pura calumnia; pero no te
metas en vindicar tu honra mancillada y echa a correr, que Villavicencio
tiene malas pulgas, y aunque te escuda el fuero militar\ldots{} Conque
en marcha y no vuelvas a Cádiz en tres meses.

---Pues sí; yo fui quien la sacó de casa.

---¡Tú!---exclamó con tanto asombro como cólera.---Ya no me acordaba que
eres servidor de mi famosa parienta la condesa. ¿Conque la sacaste tú?

---Y la volveré a sacar.

---Tú bromeas\ldots{} no pienses que me apuro mucho\ldots{} ¿Crees que
insisto en casarme con ella?\ldots{} Pues ahora de mejores veras debes
poner los pies en polvorosa, porque voy a contarle a mamá tu
hazaña\ldots{} Francamente, yo creí que era una calumnia. Ahora me
explico el furor de Villavicencio contra ti. ¿Pues no dice que tú eres
el autor de todo y que es preciso sentarte la mano?

---¿A mí?

---Y disculpaba a lord Gray\ldots{} Se me figura que quieren hacer
justicia en tu persona sin molestar para nada al señor milord. Ándate
con cuidado, pues se le ha puesto en la cabeza que tú eres cómplice del
maldito inglés y le ayudaste en esta gran bribonada que nos ha hecho.

---¿Ha visto usted a lord Gray?---le pregunté.---¿Dónde se le podrá
encontrar?

---Ahora mismo me han dicho que le acaban de ver paseando solo por la
muralla. ¡Maldito inglés! Las pagará todas juntas\ldots{} Hace poco la
Inesita me llamó vil y cobarde por dejar sin castigo esto de anoche, y
aseguraba que si ella fuera hombre\ldots{} estaba furiosa la niña. Por
supuesto, yo pienso buscar a lord Gray, y cuando le vea le he de decir
«so tunante\ldots» pues\ldots{} conque márchate\ldots{} tú también eres
buena pieza. Adiós.

No me podía detener a contestar sus majaderías, porque un pensamiento
fijo me atormentaba, y dirigida mi voluntad a un punto invariable con
arrebatadora fuerza; nada podía apartarme de aquella corriente por donde
se precipitaba impetuosamente todo mi ser.

\hypertarget{xxxiii}{%
\chapter{XXXIII}\label{xxxiii}}

Un cuarto de hora después tropezaba en la muralla, frente al Carmen, con
lord Gray, el cual, deteniendo la velocidad de su paso, me habló así:

---¡Oh, Sr.~de Araceli\ldots{} gracias a Dios que viene alguien a
hacerme compañía!\ldots{} He dado siete vueltas a Cádiz corriendo todo
lo largo de la muralla\ldots{} ¡Aburrimiento y desesperación!\ldots{} Mi
destino es dar vueltas\ldots{} dar vueltas a la noria.

---¿Está usted triste?

---Mi alma está negra\ldots{} más negra que la noche---repuso con
alucinación.---Camino sin cesar buscando la claridad, y no hago más que
dar vueltas recorriendo un círculo fatal. Cádiz es una cárcel redonda,
cuya pared circular gira alrededor de nuestro cerebro\ldots{} Me muero
aquí.

---¡Tan feliz ayer y tan desgraciado hoy!---le dije.---

---¿Yo feliz?---respondió mirándome con asombro.---¡Cuán limitada es la
creación que está a nuestro alcance! ¡Cuán pobre es el universo!\ldots{}
El Omnipotente se ha reservado para sí lo mejor, dejándonos la
escoria\ldots{} No podemos salir de este maldito círculo\ldots{} no hay
escape por la tangente\ldots{} El ansia de lo infinito quema nuestra
alma, y no es posible dar un paso en busca de alivio\ldots{} Vueltas y
más vueltas\ldots{} ¡Mula de noria\ldots{} arre!\ldots{} Otro circulito
y otro y otro\ldots{}

---Lord Gray, Dios le ha dado a usted todo y usted malgasta y arroja las
riquezas de su alma haciéndose infortunado sin deber serlo.

---Amigo---me dijo apretándome la mano tan fuertemente que creí me la
deshacía,---soy muy desgraciado. Tenga usted lástima de mí.

---Si eso es desgracia, ¿qué nombre daremos a la horrenda agonía de una
criatura, a quien usted acaba de precipitar en la mayor deshonra y
vergüenza?

---¿Usted la ha visto?\ldots{} ¡Infeliz muchacha!\ldots{} Le he rogado
que vaya conmigo a Malta y no quiere.

---Y hace bien.

---¡Pobre santita! Cuando la vi, más que su hermosura que es mucha, más
que su talento que es grande, me cautivó su piedad\ldots{} Todos decían
que era perfecta, todos decían que merecía ser venerada en los
altares\ldots{} Esto me inflamaba más. Penetrar los misterios de aquella
arca santa; ver lo que existía dentro de aquel venerable estuche de
recogimiento, de piedad, de silencio, de modestia, de santa unción;
acercarme y coger con mis manos aquella imagen celestial de mujer
canonizable; alzarle el velo y mirar si había algo de humano tras los
celajes místicos que la envolvían; coger para mí lo que no estaba
destinado a ningún hombre y apropiarme lo que todos habían convenido en
que fuese para Dios\ldots{} ¡Qué inefable delicia, qué sublime
encanto!\ldots{} ¡Ay!, fingí, engañé, burlé\ldots{} Maldita
familia\ldots{} Luchar con ella es luchar con toda una nación\ldots{}
Para atacarla toda la inteligencia y la astucia toda no bastan\ldots{}
Mil veces sea condenada la historia que crea estas fortalezas
inexpugnables.

---La audacia y la despreocupación de un hombre son más fuertes que la
historia.

---Pero cómo se desvanece todo\ldots{} Aquello que ayer aún valía, hoy
no vale nada y su encanto desaparece como el humo, como la nave, como la
sombra\ldots{} El hermoso misterio se disipó\ldots{} La realidad todo lo
mata\ldots{} ¡Ay! Yo buscaba algo extraordinario, profundamente
grandioso y sublime en aquella encarnación del principio religioso que
caía en mis brazos; yo esperaba un tesoro de ideales delicias para mi
alma, abrasada en sed inextinguible; yo esperaba recibir una impresión
celeste que transportara mi alma a la esfera de las más altas
concepciones; pero ¡maldita Naturaleza!, la criatura seráfica que yo
soñaba rodeada de nubes y de angelitos en sobrenatural beatitud, se
deshizo, se disipó, se descompuso, como una imagen de máquina óptica
cuya luz sopla el bárbaro titiritero diciendo: «buenas noches\ldots»
Todo desapareció\ldots{} Las alas de ángel agitándose zumbaban en mi
oído, pero yo me desencajaba los ojos mirando y no veía nada,
absolutamente nada más que una mujer\ldots{} una mujer como otra
cualquiera, como la de ayer, como la de anteayer\ldots{}

---Hay que conformarse con lo que Dios nos ha dado y no aspirar a más.
En resumen: usted sacó a Asunción de su casa, jurándole que abrazaría el
catolicismo y se casaría con ella.

---Es verdad.

---Y lo cumplirá usted.

---No pienso casarme.

---Entonces\ldots{}

---Ya le he dicho que venga conmigo a Malta.

---Ella no irá.

---Pues yo sí.

---Milord---dije dando a mis palabras toda la serenidad posible,---usted
debajo de ese humor melancólico, debajo de los oropeles de su
imaginación tan brillante como loca, guarda sin duda un profundo sentido
y un corazón de legítimo oro, no de vil metal sobredorado como sus
acciones.

---¿Qué quiere usted decirme?

---Que una persona honrada como usted sabrá reparar la más reciente y la
más grave de sus faltas.

---Araceli---me dijo con mucha sequedad,---es usted impertinente. ¿Acaso
es usted hermano, esposo o cortejo de la persona ofendida?

---Lo mismo que si lo fuera---repuse, obligándole a detenerse en su
marcha febril.

---¿Qué sentimiento le impulsa a usted a meterse en lo que no le
importa? Quijotismo, puro quijotismo.

---Un sentimiento que no sé definir y que me mueve a dar este paso con
fuerza extraordinaria---repuse.---Un sentimiento que creo encierra algo
de amor a la sociedad en que vivo y amor a la justicia que adoro\ldots{}
No le puedo contener ni sofocar. Quizás me equivoque; pero creo que
usted es una peligrosa, aunque hermosa bestia, a quien es preciso
perseguir y castigar.

---¿Es usted doña María?---me dijo con los ojos extraviados y la faz
descompuesta---¿es usted doña María que toma forma varonil para
ponérseme delante? Sólo a ella debo dar cuentas de mis acciones.

---Yo soy quien soy. Por lo demás, si parte de la responsabilidad
corresponde a la madre de la víctima, eso no aminora la culpa de
usted\ldots{} Pero no es una sola víctima; las víctimas somos varias. La
salvaje pasión de una furia loca y desenfrenada para quien no hay en el
mundo ni ley, ni sentimiento, ni costumbre respetables, alcanza en sus
estragos a cuanto la rodea. Por la acción de usted personas inocentes
están expuestas a ser mortificadas y perseguidas, y yo mismo aparezco
responsable de faltas que no he cometido.

---En fin, Araceli, ¿en qué viene a parar toda esa música?---dijo con
tono y modales que me recordaban el día de la borrachera en casa de
Poenco.

---Esto viene a parar---repuse con vehemencia,---en que usted se me ha
hecho profundamente aborrecible, en que me mortifica verle a usted
delante de mí, en que le odio a usted, lord Gray, y no necesito decir
más.

Yo sentía inusitado fuego circulando por mis venas. No me explicaba
aquello. Deseaba sofocar aquel sentimiento exterminador y sanguinario;
pero el recuerdo de la infeliz muchacha a quien poco antes había visto,
me hacía crispar los nervios, apretar los puños, y el corazón se me
quería saltar del pecho. No había cálculo en mí. Todo lo que determinaba
mi existencia en aquel momento era pasión pura.

---Araceli---añadió respirando con fuerza,---esta noche no estoy para
bromas. ¿Crees que soy Currito Báez?

---Lord Gray---repuse,---tampoco yo estoy para bromas.

---Todavía---dijo con amargo desdén,---no he gustado el placer de matar
a un deshacedor de agravios propios y amparador de doncellas ajenas.

---Maldito sea yo, si no es noble y nuevo lo que inflama mi espíritu en
este instante.

---¡Araceli!---exclamó con súbita furia---¿quieres que te mate? Deseo
acabar con alguien.

---Estoy dispuesto a darle a usted ese gusto.

---¿Cuándo?

---Ahora mismo.

---¡Ah!---dijo riendo a carcajadas.---Tiene la preferencia el Sr.~D.
Quijote de la Mancha. España, me despido de ti luchando con tu héroe.

---No importa. Después de las burlas pueden venir las veras.

---Nos batiremos\ldots{} ¿Quiere usted antes recibir las últimas
lecciones de esgrima?

---Gracias, ya sé lo bastante.

---¡Pobre niño!\ldots{} ¡Le mataré a usted!\ldots{} Pero son las diez y
media\ldots{} mis amigos me esperan\ldots{}

---A la Caleta.

---¿Nombramos padrinos?

---No nos faltarán amigos para elegir.

---Vamos pronto.

---Ahora mismo.

---Creí---dijo con espontánea fruición,---que no había en Cádiz más
Quijote que D. Pedro del Congosto\ldots{} ¡Oh, España! ¡Delicioso país!

\hypertarget{xxxiv}{%
\chapter{XXXIV}\label{xxxiv}}

La noche era oscura y serena. Al acercarnos a la puerta de la Caleta
vimos de lejos la iluminación que había en la plazuela de las
Barquillas, junto al teatro y en las barracas. Inmensa multitud se
apiñaba en aquellos improvisados sitios de recreo, y oíanse los gritos y
vivas con que se celebraba el gran suceso de la Albuera.

Aguardamos largo rato. Los amigos de lord Gray y D. Pedro esperaban en
la muralla en dos grupos distintos.

---¿Se han traído los garrotes?---preguntó sigilosamente uno de los de
lord Gray.

---Sí\ldots{} son vergajos de cuero para que pueda ser vapuleado sin
recibir golpes mortales\ldots{}

---¿Y las hachas de viento?

---¿Y los cohetes?

---Todo está---dijo uno sin poder disimular su gozo.---El figurón
vestido de todas armas a la antigua que ha de presentarse en lugar de
lord Gray aguarda en aquella casa. Mamarracho igual no le ha visto
Cádiz.

---Pero D. Pedro no parece\ldots{}

---Allá viene\ldots{} sus amigos los cruzados le rodean.

---Todo ha de hacerse como lo he dispuesto yo\ldots---indicó lord
Gray,---quiero despedirme de Cádiz con buen bromazo.

---Lástima que esto no pudiera hacerse en el escenario del teatro.

---Señores, se acerca la hora. ¿Baja usted\ldots{} Araceli?

---Al instante voy.

Bajaron todos, y me detuve deseando aislarme por breve rato para recoger
mi espíritu y dar alas a mi pensamiento. Habíame paseado un poco entre
la puerta y la plataforma de Capuchinos, cuando vi en la muralla una
persona, un bulto negro, cuya forma y figura no podía distinguirse bien,
y que se volvía hacia la playa, siguiendo con la vista a los
espectadores y héroes del burlesco desafío. Picábame la curiosidad por
saber quién era; mas teniendo prisa, no me detuve y bajé al instante.

Dos grandes grupos se formaron en la playa, y los de uno y otro bando,
excepto algunos bobalicones que vestían el traje de cruzados, estaban en
el ajo. Entre los de lord Gray, vi un figurón armado de pies a cabeza,
con peto y espaldar de latón, celada de encaje, rodela y con tantas
plumas en la cabeza que más que guerrero parecía salvaje de América.
Dábanle instrucciones los demás y él decía:

---Ya sé lo que tengo que hacer. Triste cosa es dejarse matar, manque
sea de mentirijiyas\ldots Yo le diré que me pongo en guardia, luego
hablaré inglés así: «Pliquis miquis\ldots,» y después daré un berrido,
cétera, cétera\ldots{}

---Haz todo lo posible por imitar mis modales y mi voz---le dijo lord
Gray.

---Descuide miloro.

Uno de los presentes acercose al otro grupo y dijo en voz alta:

---Su excelencia lord Gray, duque de Gray, está dispuesto. Vamos a
partir el sol; pero como no hay sol, se partirán las estrellas\ldots{}
Hagamos una raya en la arena.

---Por mi parte, pronto estoy---dijo D. Pedro, viendo avanzar hacia el
ruedo la espantable figura del caballero armado.---Me parece que tiembla
usted, lord Gray.

Y en efecto, el supuesto lord temblaba.

---Dios venga en mi ayuda---exclamó huecamente Congosto,---y que este
brazo, pronto a defender la justicia y a vengar un vergonzoso ultraje,
sea más fuerte que el del Cid\ldots{} ¿Lord Gray, reconoce usted su
error y se dispone a reparar la afrenta que ha causado?

El Sr.~Poenco (pues no era otro) creyó prudente contestar en inglés de
esta manera:

---Pliquis miquis\ldots{} ¡ay!, ¡ooo!\ldots{} Esperpentis
Congosto\ldots{} ¡Nooo!

---¡Pues sea!---dijo D Pedro sacando la espada,---y a quien Dios se la
dé\ldots{}

Cruzáronse los terribles aceros; daba don Pedro unos mandobles que
habrían hendido en dos mitades al Sr.~Poenco, si este con prudencia suma
no se retirara dando saltos hacia atrás. Los presentes aguantaban con
gran trabajo la risa, porque el desafío era una especie de baile, en el
cual veíase a don Pedro saltando de aquí para allí para atrapar bajo el
filo de su espada al supuesto lord Gray. Por fin, después de repetidas
vueltas y revueltas, este exhaló un rugido y cayó en tierra, diciendo:

---Muerto soy.

Al punto D. Pedro viose rodeado por un lado y otro. Multitud de vergajos
cayeron sobre sus lomos, y con loco estrépito repetían los
circunstantes:

---¡Viva el gran D. Pedro del Congosto, el más valiente caballero de
España!

Las hachas de viento se encendieron y comenzó una especie de escena
infernal. Este le empujaba de un lado, aquel del otro, querían llevarle
en vilo; pero fue preciso arrastrarle, y en tanto llovían los palos
sobre el infeliz caballero y los dos o tres cruzados que salieron en su
defensa.

---¡Viva el valiente, el invencible D. Pedro del Congosto, que ha matado
a lord Gray!

---¡Atrás canalla!---gritaba defendiéndose el estafermo.---Si le maté a
él, haré lo mismo con vosotros, gentuza vengativa y desvergonzada.

Y apaleado, pinchado, empujado, arrastrado, fue conducido hacia la
puerta como en grotesco triunfo, hasta que condolidos de tanta crueldad,
le cargaron a cuestas, llevándole procesionalmente a la ciudad. Unos
tocaban cuernos, otros golpeaban sartenes y cacharros, otros sonaban
cencerros y esquilas, y con el ruido de tales instrumentos y el fulgor
de las hachas, aquel cuadro parecía escena de brujas o fantástica
asonada del tiempo en que había encantadores en el mundo. Ya en lo alto
de la muralla, dejaron de mortificar al héroe, y llevado en hombros, su
paseo por delante de las barracas fue un verdadero triunfo. La espada de
D. Pedro quedó abandonada en el suelo. Era según antes he dicho, la
espada de Francisco Pizarro. A tal estado habían venido a parar las
grandezas heroicas de España.

~

Lord Gray y yo con otros dos, nos habíamos quedado en la playa.

---¿Una segunda broma?---preguntó Figueroa, que era uno de los padrinos,
sobre el terreno nombrados.

---Acabemos de una vez---dijo lord Gray con impaciencia.---Tengo que
arreglar mi viaje.

---Dense explicaciones---dijo el otro,---y se evitará un lance
desagradable.

---Araceli es quien tiene que darlas, no yo---afirmó el inglés.

---A lord Gray corresponde hablar, sincerándose de su vil conducta.

---En guardia---exclamó él con frenesí.---Me despido de Cádiz matando a
un amigo.

---En guardia---exclamé yo sacando la espada.

Los preliminares duraron poco y los dos aceros culebrearon con luz de
plata en la oscuridad de la noche.

De pronto oímos gritar a uno de los padrinos:

---Alto, alguien nos ve\ldots{} Por allí avanza una persona.

---Un bulto negro\ldots{} Maldito sea el curioso.

---Si será Villavicencio, que ha tenido noticia de la broma y creyendo
venir a impedirla, sorprende las veras\ldots{}

---Parece una mujer.

---Más bien parece un hombre. Se detiene allí\ldots{} nos observa.

---Adelante---dijo lord Gray.---Que venga el mundo entero a observarnos.

---Adelante.

Volvieron a cruzarse los aceros. Yo me sentía fuerte en la segunda
embestida; lord Gray era tirador habilísimo; pero estaba agitado,
mientras que yo conservaba bastante serenidad. De pronto mi mano avanzó
con rápido empuje; sintiose el chirrido de un acero al resbalar contra
el otro, y lord Gray articulando una exclamación, cayó en tierra.

---Muero---dijo, llevándose la mano al pecho.---Araceli\ldots{} buen
discípulo\ldots{} honra a su maestro.

\hypertarget{xxxv}{%
\chapter{XXXV}\label{xxxv}}

Arrojando la espada, mi primer impulso fue correr hacia el herido y
auxiliarle; pero Figueroa lleno de turbación, me dijo:

---Esto es hecho\ldots{} Araceli, huye\ldots{} no pierdas tiempo. El
gobernador\ldots{} la embajada\ldots{} Wellesley.

Comprendiendo lo arriesgado de mi situación, corrí hacia la muralla.
Turbado y hondamente impresionado y conmovido andaba hacia la puerta,
cuando me detuvo una persona que avanzaba resueltamente hacia el lugar
de la catástrofe.

---¡El gobernador Villavicencio!---dije en el primer momento antes de
distinguir con claridad el bulto de aquel extraño espectador del duelo.

Mas reconociendo a la persona al acercarme a ella, exclamé con asombro:

---Señora doña María\ldots{} ¡Usted aquí a esta hora!

---Ha caído---dijo mirando con viva atención hacia donde estaba lord
Gray.---Acertó la marquesa al asegurar que no era D. Pedro hombre a
propósito para llevar adelante esta grande empresa. Usted\ldots{}

---Señora---dije bruscamente,---no alabe usted mi hazaña\ldots{} Quiero
olvidarla, quiera olvidar que esta mano\ldots{}

---Ha castigado usted la infamia de un malvado, y el alto principio del
honor ha quedado triunfante.

---Lo dudo mucho, señora. El orgullo de mi hazaña es una llama que me
quema el corazón.

---Quiero verlo---dijo bruscamente la señora.

---¿A quién?

---A lord Gray.

---Yo no---exclamé con espanto, deseando alejarme de allí.

Doña María se acercó al cuerpo y lo examinó.

---Una venda---dijo uno.

Doña María arrojó un pañuelo sobre el cuerpo, y quitándose luego un chal
negro que bajo el manto traía, hízolo jirones y lo tiró sobre la arena.

Lord Gray abriendo los ojos, con voz débil habló así:

---¡Doña María! ¿Por qué tomaste la figura de este amigo?\ldots{} Si tu
hija entra en el convento, la sacaré.

La condesa de Rumblar se alejó con presteza de allí.

Movido de un sentimiento compasivo, acerqueme a lord Gray. Aquella
hermosa figura, arrojada en tierra, aquel semblante descolorido y
cadavérico me inspiraba profundo dolor. El herido se incorporó al verme,
y alzando su mano me dijo algunas palabras que resonaron en mi cerebro
con eco que no pude nunca olvidar; ¡extrañas palabras!

Aparteme rápidamente de allí y entraba por la puerta de la Caleta,
cuando la de Rumblar, andando a buen paso tras de mí, me detuvo.

---Lléveme usted a mi casa. Si es preciso ocultarle a usted, yo me
encargo. Villavicencio quiere prenderle a usted; pero no permito que tan
buen caballero caiga en manos de la justicia.

Ofrecile el brazo y anduvimos despacio. Yo no decía nada.

---Caballero---prosiguió.---¡Oh, cuánto me complazco en dar a usted este
nombre! La hermosa palabra rarísima vez tiene aplicación en esta
corrompida sociedad.

No le contesté. Seguimos andando, y por dos o tres veces me prodigó los
mismos elogios. Yo principiaba a cobrar aborrecimiento a mi estupenda
caballerosidad. La sangre de lord Gray corría en surtidor espantoso
delante de mis ojos.

---Desde hoy, valeroso joven, ha adquirido usted el último grado en mi
estimación, y le daré una prueba de ello.

Tampoco dije nada.

---Cuando mi hija se presentó en casa en el lastimoso estado en que
usted pudo verla, invoqué a Dios, pidiéndole el castigo de ese verdugo
de nuestra honra. Me indignaba ver que de tantos hombres como en casa se
reunieron, ni uno solo comprendió los deberes que el honor impone a un
caballero\ldots{} Cuando vi al buen Congosto dispuesto a vengar mi
ultraje, creí firmemente que Dios le había hecho ejecutor de su
justicia. Dicen que D. Pedro es ridículo; pero ¡ay!, como la hidalguía,
la nobleza y la elevación de sentimientos son una excepción en esta
sociedad, las gentes llaman ridículo al que discrepa de su nauseabunda
vulgaridad\ldots{} Yo, no sé por qué confiaba en el éxito del valor de
Congosto\ldots{} Anhelaba ser hombre, y me consumía en mi profundo
dolor. Yo creía que la armonía del mundo no podía existir mientras lord
Gray viviera, y una curiosidad intensa devoraba mi alma\ldots{} No podía
dormir, el velar me hacía daño\ldots{} no se apartaba de mi pensamiento
la escena que después he presenciado aquí, y cada minuto que pasaba sin
saber el resultado de una contienda que yo creí seria, me parecía un
siglo\ldots{}

---Señora doña María---dije procurando echar fuera el gran peso que
tenía sobre mi alma,---el varonil espíritu de usted me asombra. Pero si
vuelve usted a nacer y vuelve a tener hijas\ldots{}

---Ya sé lo que me quiere usted decir, sí\ldots{} que las tenga más
sujetas, que no les permita ni siquiera mirar a un hombre. He sido
demasiado tolerante\ldots{} Pero apartémonos de aquí\ldots{} el ruido de
esa canalla me hace daño.

---Son los patriotas que celebran la victoria de Albuera y la
Constitución que se ha leído hoy a las Cortes.

Detúvose un instante ante las barracas y al andar de nuevo, habló así
lúgubremente:

---Yo he muerto, he muerto ya. El mundo acabó para mí. Le dejo entregado
a los charlatanes. Al dirigirle la última mirada, mi espíritu se recoge
en sí mismo, se alimenta de sí mismo, y no necesita más\ldots{} Siento
haber nacido en esta infame época. Yo no soy de esta época, no\ldots{}
Desde esta noche mi casa se cerrará como un sepulcro\ldots{} Valeroso
joven, al despedirme de usted para siempre, quiero darle una prueba de
mi gratitud.

Tampoco dije nada\ldots{} Lord Gray continuaba delante de mí.

---Usted---prosiguió,---se presenta desde este instante a mis ojos
rodeado de una aureola. Usted ha respondido a mis ideas como responde el
brazo al pensamiento.

---Maldita aureola---exclamé para mí,---maldito brazo y maldito
pensamiento.

---Le premiaré a usted del modo siguiente. Ya sé que usted ama a la
estudianta\ldots{} me lo ha dicho la de Leiva.

---¿Quién es la estudianta, señora?

---La estudianta es Inés, hija como usted sabe\ldots{} dejémonos de
misterios\ldots{} hija de la buena pieza de mi parienta la condesa y de
un estudiantillo llamado D. Luis. He querido sacar algún partido de esa
infeliz; pero no es posible. Su liviana condición la hace incapaz de
toda enmienda. Vale bien poco. ¿Es cierto que la sacó usted de casa?

---Sí, señora. La saqué para llevarla al lado de su madre. Me vanaglorio
de esta acción más que de la que usted acaba de presenciar.

---¿Y la ama usted?

---Sí, señora.

---Es una lástima. La estudianta es indigna de usted. Yo se la regalo.
Puede usted divertirse con ella\ldots{} Será como su madre\ldots{} le
han dado una educación lamentable, y criada entre gente humildísima,
tuvo tiempo de aprender toda clase de malicias.

Oí tales palabras con indignación, pero callé.

---Me asombro de mi necedad. ¡Oh! Mi hijo no puede casarse con tal
chiquilla\ldots{} La condesa la reclama, la llama su hija, desbarata la
admirable trama de la familia para asegurar el porvenir de la hija y
poner un velo al deshonor de la madre. La condesa la reclama\ldots{}
¿Qué nombre llevará? Desde este momento Inés es una desgraciada criatura
espúrea, a quien ningún caballero podrá ofrecer dignamente su mano.

Continué en silencio. Mi entendimiento estaba como paralizado y
entumecido por el estupor.

---Sí---prosiguió.---Todo ha concluido. Pleitearé\ldots{} porque el
mayorazgo me corresponde. La casa de Leiva no tiene sucesión\ldots{}
Supongo que usted no será capaz de dar su nombre a una\ldots{} Llévesela
usted, llévesela pronto. No quiero tener en casa esa deshonra\ldots{}
Una muchacha sin nombre\ldots{} una infeliz espúrea. ¡Qué horrible
espectáculo para mi pobrecita Presentación, para mi única hija!\ldots{}

Doña María exhaló un suspiro en que parecía haberse desprendido de la
mitad de su alma, y no dijo más por el camino. Yo tampoco hablé una
palabra.

Llegamos a la casa, donde con impaciencia y zozobra esperaba a su ama D.
Paco. Subimos en silencio, aguardé un instante en la sala, y doña María
después de pequeña ausencia apareció trayendo a Inés de la mano, y me
dijo:

---Ahí la tiene usted\ldots{} Puede usted llevársela, huir de
Cádiz\ldots{} divertirse, sí, divertirse con ella. Le aseguro a usted
que vale poco\ldots{} Después de la declaración de su madre, yo aseguro
que ni la marquesa de Leiva ni yo haremos nada por recobrarla.

---Vamos, Inés---exclamé,---huyamos de aquí, huyamos para siempre de
esta casa y de Cádiz.

---¿Van ustedes a Malta?---me preguntó doña María con una sonrisa, de
cuya expresión espantosa no puedo dar idea con las palabras de nuestra
lengua.

---¿No me deja usted---dijo Inés llorando,---entrar en el cuarto donde
está encerrada Asunción, para despedirme de ella?

Doña María por única contestación nos señaló la puerta. Salimos y
bajamos. Cuando la condesa de Rumblar se apartó de nuestra vista; cuando
la claridad de la lámpara que ella misma sostenía en alto, dejó de
iluminar su rostro, me pareció que aquella figura se había borrado de un
lienzo, que había desaparecido, como desaparece la viñeta pintada en la
hoja, al cerrarse bruscamente el libro que la contiene.

~

---Huyamos, querida mía, huyamos de esta maldita casa y de Cádiz y de la
Caleta---dije estrechando con mi brazo la mano de Inés.

---¿Y lord Gray?---me preguntó.

---Calla\ldots{} no me preguntes nada---exclamé con zozobra.---Apártate
de mí. Mis manos están manchadas de sangre.

---Ya entiendo---dijo ella con viva emoción.---La infame conducta de ese
hombre ha sido castigada\ldots{} Ha muerto lord Gray.

---No me preguntes nada---repetí avivando el paso.---Lord Gray\ldots{}
Yo tuve más suerte que él en el duelo. Mañana dirán que el honor\ldots{}
pues\ldots{} me pondrán por las nubes\ldots{} ¡Infeliz de mí!\ldots{} El
desgraciado cayó bañado en sangre; acerqueme a él y me dijo: «¿Crees que
he muerto? ¡Ilusión!\ldots{} yo no muero\ldots{} yo no puedo
morir\ldots{} yo soy inmortal\ldots»

---¿De modo que no ha muerto?

---Huyamos\ldots{} no te detengas\ldots{} yo estoy loco. ¿Esa figura que
ha pasado delante de nosotros no es la de lord Gray?

Inés estrechándose más contra mí, añadió:

---Huyamos, sí\ldots{} quizás te persigan\ldots{} Mi madre y yo te
esconderemos y huiremos contigo.

\flushright{Septiembre-Octubre de 1874.}

~

\bigskip
\bigskip
\begin{center}
\textsc{Fin de Cádiz}
\end{center}

\end{document}
