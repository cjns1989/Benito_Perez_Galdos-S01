\PassOptionsToPackage{unicode=true}{hyperref} % options for packages loaded elsewhere
\PassOptionsToPackage{hyphens}{url}
%
\documentclass[oneside,14pt,spanish,]{extbook} % cjns1989 - 27112019 - added the oneside option: so that the text jumps left & right when reading on a tablet/ereader
\usepackage{lmodern}
\usepackage{amssymb,amsmath}
\usepackage{ifxetex,ifluatex}
\usepackage{fixltx2e} % provides \textsubscript
\ifnum 0\ifxetex 1\fi\ifluatex 1\fi=0 % if pdftex
  \usepackage[T1]{fontenc}
  \usepackage[utf8]{inputenc}
  \usepackage{textcomp} % provides euro and other symbols
\else % if luatex or xelatex
  \usepackage{unicode-math}
  \defaultfontfeatures{Ligatures=TeX,Scale=MatchLowercase}
%   \setmainfont[]{EBGaramond-Regular}
    \setmainfont[Numbers={OldStyle,Proportional}]{EBGaramond-Regular}      % cjns1989 - 20191129 - old style numbers 
\fi
% use upquote if available, for straight quotes in verbatim environments
\IfFileExists{upquote.sty}{\usepackage{upquote}}{}
% use microtype if available
\IfFileExists{microtype.sty}{%
\usepackage[]{microtype}
\UseMicrotypeSet[protrusion]{basicmath} % disable protrusion for tt fonts
}{}
\usepackage{hyperref}
\hypersetup{
            pdftitle={LA BATALLA DE LOS ARAPILES},
            pdfauthor={Benito Pérez Galdós},
            pdfborder={0 0 0},
            breaklinks=true}
\urlstyle{same}  % don't use monospace font for urls
\usepackage[papersize={4.80 in, 6.40  in},left=.5 in,right=.5 in]{geometry}
\setlength{\emergencystretch}{3em}  % prevent overfull lines
\providecommand{\tightlist}{%
  \setlength{\itemsep}{0pt}\setlength{\parskip}{0pt}}
\setcounter{secnumdepth}{0}

% set default figure placement to htbp
\makeatletter
\def\fps@figure{htbp}
\makeatother

\usepackage{ragged2e}
\usepackage{epigraph}
\renewcommand{\textflush}{flushepinormal}

\usepackage{indentfirst}

\usepackage{fancyhdr}
\pagestyle{fancy}
\fancyhf{}
\fancyhead[R]{\thepage}
\renewcommand{\headrulewidth}{0pt}
\usepackage{quoting}
\usepackage{ragged2e}

\newlength\mylen
\settowidth\mylen{...................}

\usepackage{stackengine}
\usepackage{graphicx}
\def\asterism{\par\vspace{1em}{\centering\scalebox{.9}{%
  \stackon[-0.6pt]{\bfseries*~*}{\bfseries*}}\par}\vspace{.8em}\par}

 \usepackage{titlesec}
 \titleformat{\chapter}[display]
  {\normalfont\bfseries\filcenter}{}{0pt}{\Large}
 \titleformat{\section}[display]
  {\normalfont\bfseries\filcenter}{}{0pt}{\Large}
 \titleformat{\subsection}[display]
  {\normalfont\bfseries\filcenter}{}{0pt}{\Large}

\setcounter{secnumdepth}{1}
\ifnum 0\ifxetex 1\fi\ifluatex 1\fi=0 % if pdftex
  \usepackage[shorthands=off,main=spanish]{babel}
\else
  % load polyglossia as late as possible as it *could* call bidi if RTL lang (e.g. Hebrew or Arabic)
%   \usepackage{polyglossia}
%   \setmainlanguage[]{spanish}
%   \usepackage[french]{babel} % cjns1989 - 1.43 version of polyglossia on this system does not allow disabling the autospacing feature
\fi

\title{LA BATALLA DE LOS ARAPILES}
\author{Benito Pérez Galdós}
\date{}

\begin{document}
\maketitle

\hypertarget{i}{%
\chapter{I}\label{i}}

Las siguientes cartas, supliendo ventajosamente mi narración, me
permitirán descansar un poco.

\begin{flushright}\small\textit{Madrid, 14 de Marzo.}\normalsize\end{flushright}

Querido Gabriel: Si no has sido más afortunado que yo, lucidos estamos.
De mis averiguaciones no resulta hasta ahora otra cosa que la triste
certidumbre de que el comisario de policía no está ya en esta corte, ni
presta servicio a los franceses, ni a nadie como no sea al demonio.
Después de su excursión a Guadalajara, pidió licencia, abandonó luego su
destino, y al presente nadie sabe de él. Quién le supone en Salamanca,
su tierra natal, quién en Burgos o en Vitoria, y algunos aseguran que ha
pasado a Francia, antiguo teatro de sus criminales aventuras. ¡Ay, hijo
mío, para qué habrá hecho Dios el mundo tan grande, tan sumamente
grande, que en él no es posible encontrar el bien que se pierde! Esta
inmensidad de la creación sólo favorece a los pillos, que siempre
encuentran donde ocultar el fruto de sus rapiñas.

Mi situación aquí ha mejorado un poco. He capitulado, amigo mío; he
escrito a mi tía contándole lo ocurrido en Cifuentes, y el jefe de mi
ilustre familia me demuestra en su última carta que tiene lástima de mí.
El administrador ha recibido orden de no dejarme morir de hambre.
Gracias a esto y al buen surtido de mi antiguo guarda-ropas, la pobre
condesa no pedirá limosna por ahora. He tratado de vender las alhajas,
los encajes, los tapices y otras prendas no vinculadas; pero nadie las
quiere comprar. En Madrid no hay una peseta, y cuando el pan está a
catorce y diez y seis reales, figúrate quién tendrá humor para comprar
joyas. Si esto sigue, llegará día en que tenga que cambiar todos mis
diamantes por una gallina.

Para que comprendas cuán glorioso porvenir aguarda a mi histórica casa,
uno de los astros más brillantes del cielo de esta gran monarquía, me
bastará decirte que el pleito entre nuestra familia y la de Rumblar se
ha entablado ya, y la cancillería de Granada ha dado a luz con este
motivo una montaña de papel sellado, que, si Dios no lo remedia, crecerá
hasta lo sumo y nuestros nietos veranla con cimas más altas que las de
la misma Sierra Nevada. La de Rumblar se engolfa con delicia en este mar
de jurisprudencia. Me parece que la veo. Convertiría el linaje humano en
jueces, escribas, alguaciles y roe-pandectas para que todo cuanto
respira pudiese entender en su cuita.

El licenciado Lobo, que frecuentemente me visita con el doble objeto de
ilustrarme en mi asunto y de pedirme una limosna (hoy en Madrid la piden
los altos servidores del Estado), me ha dicho que en el tal pleito hay
materia para un ratito, es decir, que no pasará un par de siglos mal
contados sin que la sala de su sentencia o un auto para mejor proveer,
que es el colmo de las delicias. Me asegura también el susodicho Lobo,
que si nos obstinamos en transmitir a Inés los derechos mayorazguiles,
es fácil que perdamos el litigio dentro de algunos meses, pues para
perder no es preciso esperar siglos. Las informalidades que hubo en el
reconocimiento y la indiscreción de mi pobre tío, que ya bajó al
sepulcro, ponen a nuestra heredera en muy mala situación para reclamar
su mayorazgo. Nuestro papel se reduce hoy, según Lobo, a reclamar la no
transmisión del mayorazgo a la casa de Rumblar, fundándonos en varias
razones de \emph{posesión civilísima, agnación rigurosa, masculinidad
nuda, emineidad, saltuario}, con otras lindas palabras que voy
aprendiendo para recreo de mi triste soledad y entretenimiento de mis
últimos días.

Mi tía dice que yo tengo la culpa de este desastre y cataclismo en que
va a hundirse la más gloriosa casa que ha desafiado siglos y afrontado
el desgaste del tiempo, sin criar hasta ahora ni una sola carcoma, y
funda su anatema en mi oposición al proyectado himeneo de nuestro
derecho con el derecho de los Rumblar. Verdaderamente no carece de razón
mi tía, y sin duda se me preparan en el purgatorio acerbos tormentos por
haber ocasionado con mi tenacidad este conflicto.

Esta carta te la envío a Sepúlveda. Creo que serán infructuosas tus
pesquisas en todo el camino de Francia hasta Aranda. Procura ir a
Zamora. Yo sigo aquí mis averiguaciones con ardor infatigable; y
demostrando gran celo por la causa francesa, he adquirido conocimiento
con empleados de alta y baja estofa, principalmente de policía pública y
secreta.

Si te unes a la división de Carlos España, avísamelo. Creo que conviene
a tu carrera militar el abandonar a esos feroces guerrilleros; más por
Dios no pases al ejército de Extremadura. Creo que de ese lado no vendrá
la luz que deseamos; sigue en Castilla mientras puedas, hijo mío, y no
abandones mi santa empresa. Escríbeme con frecuencia. Tus cartas y el
placer que me causa contestarlas son mi único consuelo. Me moriría si no
llorara y si no te escribiera.

\begin{flushright}\small\textit{22 de Marzo.}\normalsize\end{flushright}

No puedes figurarte la miseria espantosa que reina en Madrid. Me han
dicho que hoy está la fanega de trigo a 540 reales. Los ricos pueden
vivir, aunque mal; pero los pobres se mueren por esas calles a
centenares sin que sea posible aliviar su hambre. Todos los arbitrios de
la caridad son inútiles, y el dinero busca alimentos sin encontrarlos.
Las gentes desvalidas se disputan con ferocidad un troncho de col, y las
sobras de aquellos pocos que tienen todavía en su casa mesa con
manteles. Es imposible salir a la calle, porque los espectáculos que se
ofrecen a cada momento a la vista causan horror y desconfianza de la
Providencia infinita. Vense a cada paso los mendigos hambrientos,
arrojados en el arroyo, y en tal estado de demacración que parecen
cadáveres en que ha quedado olvidado un resto de inútil y miserable
vida. El lodo y la inmundicia de las calles y plazuelas les sirven de
lecho, y no tienen voz sino para pedir un pan que nadie puede darles.

Si la policía se lo permitiera, maldecirían a los franceses, que tienen
en sus almacenes copioso repuesto de galleta, mientras la nación se
muere de hambre. Dicen que de Agosto acá se han enterrado veinte mil
cuerpos, y lo creo. Aquí se respira muerte; el silencio de los sepulcros
reina en Platerías, en San Felipe y en la Puerta del Sol. Como han
derribado tantos edificios, entre ellos Santiago, San Juan, San Miguel,
San Martín, los Mostenses, Santa Ana, Santa Catalina, Santa Clara y
bastantes casas de las inmediatas a palacio, las muchas ruinas dan a
Madrid el aspecto de una ciudad bombardeada. ¡Qué desolación, qué
tristeza!

Los franceses se pasean, alegres rollizos por este cementerio, y su
policía mortifica de un modo cruel a los vecinos pacíficos. No se
permiten grupos en las calles, ni pararse a hablar, ni mirar a las
tiendas. A los tenderos se les aplica una multa de 200 ducados si
permiten que los curiosos se detengan en las puertas o vidrieras, de
modo que a cada rato los pobres horteras tienen que salir a apalear a
sus parroquianos con la vara de medir.

Ayer dispuso el rey que hubiese corrida de toros para divertir al
pueblo: ¡qué sarcasmo! Me han dicho que la plaza estaba desierta.
Figúrome ver en el redondel a media docena de esqueletos vestidos con el
traje bordado de plata y oro, y más deseosos de comerse al toro que de
trastearlo. Asistió José, que de este modo piensa ganar la voluntad del
pueblo de Madrid.

Dícese que se trata de reunir Cortes en Madrid, no sé si también para
divertir al pueblo. Azanza, ministro de Su Majestad Bonaparciana, me
dijo que así levantarían \emph{un altar frente a otro altar}. Creo que
el retablo de aquí no tendrá tantos devotos como el que dejamos en
Cádiz.

Ahora dicen que Napoleón va a emprender una guerra contra el emperador
de todas las Rusias. Esto será favorable a España, porque sacarán tropas
de la península, o al menos no podrán reparar las bajas que
continuamente sufren. Veo la causa francesa bastante malparada, y he
observado que los más discretos de entre ellos no se hacen ya ilusiones
respecto al resultado final de esta guerra.

De nuestro asunto ¿qué puedo decir que no sea triste y desconsolador?
Nada, hijo mío, absolutamente nada. Mis indagaciones no dan resultado
alguno, no he podido adquirir ni la más pequeña luz, ni el más ligero
indicio. Sin embargo, confío en Dios y espero. Dirijo esta carta a Santa
María de Nieva, que es lo más seguro.

\begin{flushright}\small\textit{1º de Abril.}\normalsize\end{flushright}

Poco o nada tengo que añadir a mi carta de 22 de Marzo. Continúo en la
oscuridad; pero con fe. ¡Cuánta se necesita para permanecer en Madrid!
Esto es un purgatorio por la miseria, la soledad, la tristeza, y un
infierno por la corrupción, las violencias e inmoralidades de todo
género que han introducido aquí los franceses.

Yo no creo, como la mayoría de las gentes, que nuestras costumbres
fueran perfectas antes de la invasión; pero entre aquel recatado y
compungido modo de vivir y esta desvergonzada licencia de hoy, es
preferible a todas luces lo primero. La policía francesa es un instituto
de cuya perversidad no se puede tener idea, sino viviendo aquí y viendo
la execrable acción de esta máquina, puesta en las más viles manos.

Multitud de comisarios y agentes, escogidos entre la hez de la sociedad,
se encargan de atrapar a los individuos que se les antoja y almacenarlos
en la cárcel de villa, sin forma de juicio, ni más guía que la
arbitrariedad y la delación. El motivo aparente de estas tropelías es la
\emph{complicidad con los insurgentes}; pero los malvados de uno y otro
bando se dan buena maña para utilizar esta nueva Inquisición que hará
olvidar con sus gracias las lindezas de la pasada. Todo aquel que quiere
deshacerse de una persona que le estorba, encuentra fácil medio para
ello, y aun ha habido quien, no contentándose con ver emparedado a su
enemigo, le ha hecho subir al cadalso. Se cuentan cosas horribles, que
me resisto a darles crédito, entre ellas la maldad de una señora de esta
corte, que, mal avenida con su esposo le delató como insurgente y
despacharon la causa en cosa de tres días, lo necesario para ir de la
callejuela del Verdugo a la plaza de la Cebada. También se habla de un
tal Vázquez, que delató a su hermano mayor, y de un tal Escalera que
subió la del patíbulo por intrigas de su manceba.

Hay una Junta criminal que inspira más horror que los jueces del
infierno. Los hombres bajos que la forman condenan a muerte a los que
leen los papeles de los insurgentes, a los \emph{empecinados}, que aquí
llaman \emph{madripáparos}, y a todo ser sospechoso de relaciones con
los \emph{espías}, \emph{ladrones}, \emph{asesinos}, \emph{bandoleros},
\emph{cuatreros y\ldots{}} \emph{tahures}, a quienes llamáis vosotros
guerrilleros o soldados de la patria.

Una de las cosas más criticadas a los franceses, además de su infame
policía, es la introducción de los bailes de máscaras. En esto hay
exageración, porque antes que tales escandalosas reuniones fuesen
instituidas en nuestro morigerado país, había intrigas y gran burla de
vigilancia de padres y maridos. Yo creo que las caretas no han traído
acá todos los pecados grandes y chicos que se les atribuyen. Pero la
gente honesta y timorata brama contra tal novedad, y no se oye otra cosa
sino que con los tapujos de las caras ya no hay tálamo nupcial seguro,
ni casa honrada, ni padre que pueda responder del honor de sus hijas, ni
doncella que conserve su espíritu libre y limpio de deshonestos
pensamientos. Creo que no es justa esta enemiga contra las caretas, más
cómodas aunque no más disimuladoras que los antiguos mantos, y tengo
para mí que muchas personas hablan mal de las reuniones de máscaras
porque no las encuentran tan divertidas ni tan oscuritas como las
verbenas de San Juan y San Pedro.

Pero la novedad que más indignada y fuera de sus casillas trae a esta
buena gente, es un juego de azar llamado la \emph{roleta}, donde parece
baila el dinero que es un gusto. Los franceses son Barrabás para
inventar cosas malas y pecaminosas. No respetan nada, ni aun las
venerandas prácticas de la antigüedad, ni aun aquello que forma parte
desde remotísimas edades, de la ejemplar existencia nacional. Lo justo
habría sido dejar que los padres y los hijos de familia se arruinaran
con la baraja, siguiendo en esto sus patriarcales y jamás alteradas
costumbres, y no introducir \emph{roletas} ni otros aparatos infernales.
Pero los franceses dicen que la \emph{roleta} es un adelanto con
respecto a los naipes, así como la guillotina es mejor que la horca, y
la policía mucho mejor que la Inquisición.

Lo peor de esto es que, según dicen, la tal endemoniada \emph{roleta},
no sólo es consentida por el gobierno francés, sino de su propiedad, y
para él son las pingües ganancias que deja. De este modo los franceses
piensan embolsarse el poco dinero que han dejado en nuestras arcas.

No concluiré sin ponerte al corriente de un proyecto que tengo, y que,
realizado, me parece ha de ser más eficaz para nuestro objeto que todas
las averiguaciones y búsquedas hechas hasta ahora. El plan, hijo mío,
consiste en interesar al mismo José en favor mío. Pienso ir a palacio,
donde seré recibida por el señor Botellas, el cual no desea otra cosa y
ve el cielo abierto cuando le anuncian que un grande de España quiere
visitarle. Hasta ahora he resistido todas las sugestiones de varios
personajes amigos míos que se han empeñado en presentarme al Rey; pero
pensándolo mejor, estoy decidida a ir a la corte. En Diciembre del 8
traté a los dos Bonaparte, y las bondades que encontré en José me hacen
esperar que no será inútil este paso que doy, aun a riesgo de
comprometerme con una causa que considero perdida. Adiós: te informaré
de todo.

\begin{flushright}\small\textit{22 de Abril.}\normalsize\end{flushright}

He estado en palacio, hijo mío, y me he prosternado ante esa católica
majestad de oropel, a quien sirven unos pocos españoles, moviéndose
bulliciosamente para parecer muchos. Si yo dijera a cualquier habitante
de Madrid que José I, conocido aquí por \emph{el tuerto}, o por
\emph{Pepe Botellas}, es una persona amable, discreta, tolerante, de
buenas costumbres, y que no desea más que el bien, me tendrían por loca
o quizás por vendida a los franceses.

Recibiome \emph{Copas} con gozo. El buen señor no puede ocultarlo cuando
alguna persona de categoría da, al visitarle, una especie de tácito
asentimiento a su usurpación. Sin duda cree posible ser dueño de España
conquistando uno a uno los corazones. Habrías de ver su diligencia y
extremado empeño de hacer cumplidos.

Cierto es que su etiqueta es menos severa y finchada que la de nuestros
reyes, sin perder por eso la dignidad, antes bien aumentándola. Habla
hasta con familiaridad, se ríe, también se permite algunas gentilezas
galantes con las damas, y a veces bromea con cierta causticidad muy
fina, propia de los italianos. El acento extranjero es el único que afea
su palabra. Confunde a menudo su lengua natal con la nuestra y hay
ocasiones en que son necesarios grandes esfuerzos para no reír.

Su figura no puede ser mejor. José vale mucho más que el barrilete de su
hermano. Poco falta a su rostro grave y expresivo para ser perfecto.
Viste comúnmente de negro, y el conjunto de su persona es muy agradable.
No necesito decirte que cuanto hablan las gentes por ahí sobre sus
turcas, es un arma inventada por el patriotismo para ayudar a la defensa
nacional. José no es borracho. También se cuentan de él mil
abominaciones referentes a vicios distintos del de la embriaguez; pero
sin negarlos rotundamente, me resisto a darles crédito. En resumen,
Botellas (nos hemos acostumbrado de tal manera a darle este nombre, que
cuesta trabajo llamarle de otra manera) es un rey bastante bueno, y al
verle y tratarle, no se puede menos de deplorar que lo hayan traído, en
vez del nacimiento y el derecho, la usurpación y la guerra.

Sus partidarios aquí son pocos, tan pocos, que se pueden contar. Esta
dinastía no tiene más súbditos leales que los ministros y dos o tres
personas colocadas por ellos en altos puestos. Estos españoles que le
sirven parecen víctimas humilladas y no tienen aquel aire triunfador y
vanaglorioso que suelen tomar aquí los que por méritos propios o ajeno
favor se elevan dos dedos sobre los demás. Viven o avergonzados o
medrosos, sin duda porque prevén que el \emph{Lord} ha de dar al traste
con todo esto. Algunos, sin embargo, se hacen ilusiones y dicen que
tendremos Botellas, Azumbres y Copas por los siglos de los siglos.

No pertenece a estos Moratín, el cual está más triste y más pusilánime
que nunca. Ya no es secretario de la interpretación de lenguas, sino
bibliotecario mayor, cargo que debe de desempeñar a maravilla. Pero él
no está contento; tiene miedo a todo, y más que a nada a los peligros de
una segunda evacuación de la Corte por los franceses. Me ha dicho que el
día en que cayese el poder intruso no daría dos cuartos por su pellejo;
pero creo que su hipocondría y pésimo humor, entenebreciendo su alma, le
hacen ver enemigos en todas partes. Está enfermo y arruinado; mas
trabaja algo, y ahora nos ha dado \emph{La escuela de los maridos},
traducción del francés. Ni la he visto representar ni he podido leerla,
porque mi espíritu no puede fijarse en nada de esto.

Moratín viene a verme a menudo con su amigo Estala, el cual es
afrancesado rabioso y ardiente, como aquel lo es tímido y melancólico.
Aquí no pueden ver a Estala, que publica artículos furibundos en
\emph{El Imparcial}, y hace poco escribió, aludiendo a España, que
\emph{los que nacen en un país de esclavitud no tienen patria sino en el
sentido en que la tienen los rebaños destinados para nuestro consumo.}
Por esto y otros atroces partos de su ingenio que publica la Gaceta, es
aborrecido aún más que los franceses.

Máiquez sigue en el Príncipe, y como José ha señalado a su teatro 20.000
reales mensuales para ayuda de costa, le tachan también de afrancesado.
Ahora, según veo en el diario, dan alternativamente el \emph{Orestes},
\emph{La mayor piedad de Leopoldo el Grande} y una mala comedia
arreglada del alemán, y cuyo título es \emph{Ocultar, de honor movido,
al agresor el herido}.

El teatro está, según me dicen, vacío. La pobre Pepilla González, de
quien no te habrás olvidado, se muere de miseria, porque no pudiendo
representar, a causa de una enfermedad que ha contraído, está sin
sueldo, abandonada de sus compañeros. Lo estaría de todo el mundo, si yo
no cuidase de enviarle todos los días lo muy preciso para que no expire.
Pepilla, el venerable padre Salmón y mi confesor, Castillo, son las
únicas personas a quienes puedo favorecer, porque el estado de mi
hacienda y la carestía de las subsistencias no me permiten más. Te
asombrará saber que los opulentos padres de la Merced necesiten de
limosnas para vivir: pero a tal situación ha llegado la indigencia
pública en la corte de España, que los más gordos se han puesto como
alambres.

De intento he dejado para el fin de mi carta nuestro querido asunto,
porque quiero sorprenderte. ¿No has adivinado en el tono de mi epístola
que estoy menos triste que de ordinario? Pero nada te diré hasta que no
tenga seguridad de no engañarte. Refrena tu impaciencia, hijo
mío\ldots{} Gracias a José, se me han suministrado algunos datos
preciosos, y muy pronto, según acaba de decirme Azanza, este resplandor
de la verdad será luz clara y completa. Adiós.

\begin{flushright}\small \textit{21 de mayo.} \normalsize\end{flushright}

Albricias, querido amigo, hijo y servidor mío. Ya está descubierto el
paradero de nuestro verdugo. ¡Benditos sean mil veces José y esa
desconocida reina Julia, cuyo nombre invoqué para inclinarle en mi
favor! Santorcaz no ha pasado todavía a Francia. Desde aquí, querido
mío, considerándote en camino hacia Occidente, puedo decirte como a los
niños cuando juegan a la gallina ciega: «Que te quemas.» Sí, chiquillo,
alarga la mano y cogerás al traidor. ¡Cuántas veces buscamos el sombrero
y lo llevamos puesto! Aquello que consideramos más perdido está
comúnmente más cerca. La idea de que esta carta no te encuentre ya en
Piedrahíta me espanta. Pero Dios no puede sernos tan desfavorable y tú
recibirás este papel; inmediatamente marcharás hacia Plasencia, y valido
de tu astucia, de tu valor, de tu ingenio o de todas estas cualidades
juntas, penetrarás en la vivienda del pícaro para arrancarle la joya
robada que lleva siempre consigo.

¡Cuánto trabajo ha costado averiguarlo! Ha tiempo que Santorcaz dejó el
servicio. Su carácter, su orgullo, su extravagancia, le hacían
insoportable a los mismos que le colocaron. Por algún tiempo fue
tolerado en gracia de los buenos servicios que presta, mas se descubrió
que pertenecía a la sociedad de los \emph{filadelfos}, nacida en el
ejército de Soult, y cuyo objeto era destronar al Emperador, proclamando
la república. Quitáronle el destino poco después de habernos robado a
Inés, y desde entonces ha vagado por la Península fundando logias.
Estuvo en Valladolid, en Burgos, en Salamanca, en Oviedo; mas luego se
perdió su rastro, y por algún tiempo se creyó que había entrado en
Francia. Finalmente, la policía francesa (la peor cosa del mundo produce
algo bueno) ha descubierto que está ahora en Plasencia, bastante enfermo
y un tanto imposibilitado de trastornar a los pueblos con sus logias y
cónclaves revolucionarios. ¡Qué indignidad! ¡Los perdidos, los tunantes,
los mentirosos y falsarios quieren reformar el mundo!\ldots{} Estoy
colérica, amigo mío, estoy furiosa.

El que ha completado mis noticias sobre Santorcaz es un afrancesado no
menos loco y trapisondista que él, José Marchena, ¿le conoces? uno que
pasa aquí por clérigo relajado, una especie de abate que habla más
francés que español, y más latín que francés, poeta, orador, hombre de
facundia y de chiste, que se dice amigo de madama Staël, y parece lo fue
realmente de Marat, Robespierre, Legendre, Tallien y demás gentuza.
Santorcaz y él vivieron juntos en París. Son hoy muy amigos, se escriben
a menudo. Pero este Marchena es hombre de poca reserva y contesta a todo
lo que le preguntan. Por él sé que nuestro enemigo no goza de buena
salud, que no vive sino en las poblaciones ocupadas por los franceses, y
que cuando pasa de un punto a otro, se disfraza hábilmente para no ser
conocido. ¡Y nosotros le creíamos en Francia! ¡Y yo te decía que no
fueras al ejército de Extremadura! Ve, corre, no tardes un solo día. El
ejército del \emph{Lord} debe de andar por allí. Te escribiré al cuartel
general de D. Carlos España. Contéstame pronto. ¿Irás donde te mando?
¿Encontrarás lo que buscamos? ¿Podrás devolvérmelo? Estoy sin alma.

\hypertarget{ii}{%
\chapter{II}\label{ii}}

Cuando recibí esta carta, marchaba a unirme al ejército llamado de
Extremadura, pero que no estaba ya en Extremadura, sino en Fuente
Aguinaldo, territorio de Salamanca.

En Abril había yo dejado definitivamente la compañía de los guerrilleros
para volver al ejército. Tocome servir a las órdenes de un mariscal de
campo llamado Carlos Espagne, el que después fue conde de España, de
fúnebre memoria en Cataluña. Hasta entonces aquel joven francés,
alistado en nuestros ejércitos desde 1792, no tenía celebridad, a pesar
de haberse distinguido en las acciones de Barca del Puerto, de Tamames,
del Fresno y de Medina del Campo. Era un excelente militar, muy bravo y
fuerte, pero de carácter variable y díscolo. Digno de admiración en los
combates, movían a risa o a cólera sus rarezas cuando no había enemigos
delante. Tenía una figura poco simpática, y su fisonomía, compuesta casi
exclusivamente de una nariz de cotorra y de unos ojazos pardos bajo
cejas angulosas, revueltas, movibles y en las cuales cada pelo tenía la
dirección que le parecía, revelaba un espíritu desconfiado y pasiones
ardientes, ante las cuales el amigo y el subalterno debían ponerse en
guardia.

Muchas de sus acciones revelaban lamentable vaciedad en los aposentos
cerebrales, y si no peleamos algunas veces contra molinos de viento, fue
porque Dios nos tuvo de su mano; pero era frecuente tocar llamada en el
silencio y soledad de la alta noche, salir precipitadamente de los
alojamientos, buscar al enemigo que tan a deshora nos hacía romper el
dulce sueño, y no encontrar más que al lunático España vociferando en
medio del campo contra sus invisibles compatriotas.

Mandaba este hombre una división pertenecienteal ejército de que era
comandante general D. Carlos O'Donnell. Habíasele unido por aquel tiempo
la partida de D. Julián Sánchez, guerrillero muy afortunado en Castilla
la Vieja, y se disponía a formar en las filas de Wellington, establecido
en Fuente Aguinaldo, después de haber ganado a Badajoz a fines de Marzo.
Los franceses de Castilla la Vieja mandados por Marmont andaban muy
desconcertados. Soult, operaba en Andalucía sin atreverse a atacar al
\emph{Lord} y este decidió avanzar resueltamente hacia Castilla. En
resumen, la guerra no tomaba mal aspecto para nosotros; por el
contrario, parecía en evidente declinación la estrella imperial, después
de los golpes sufridos en Ciudad-Rodrigo, Arroyomolinos y Badajoz.

Yo había recibido el empleoLde comandante en Febrero de aquel mismo año.
Por mi ventura mandé durante algún tiempo (pues también fui jefe de
guerrillas) una partida que corrió el país de Aranda y luego las sierras
de Covarrubias y la Demanda. A principios de Marzo tenía la seguridad de
que Santorcaz no estaba en aquel país. Alargué atrevidamente mis
excursiones hasta Burgos, ocupada por los franceses, entré disfrazado en
la plaza, y pude saber que el antiguo comisario de policía había
residido allí meses antes. Bajando luego a Segovia, continué mis
pesquisas; pero una orden superior me obligó a unirme a la división de
D. Carlos España.

Obedecí, y como en los mismos días recibiese la última carta de las que
puntualmentehe copiado, juzgué favor especial del cielo aquella
disposición militar que me enviaba a Extremadura. Pero, como he dicho,
Wellington, a quien debiera unirse España, había dejado ya las orillas
del Tiétar. Nosotros debíamos salir de Piedrahíta para unirnos a él en
Fuente Aguinaldo o en Ciudad-Rodrigo. De aquí se podía ir fácilmente a
Plasencia.

Mientras con zozobra y desesperación revolvía en mi mente distintos
proyectos, ocurrieron sucesos que no debo pasar en silencio.

\hypertarget{iii}{%
\chapter{III}\label{iii}}

Después de larguísima jornada durante la tarde y gran parte de una
hermosísima noche de Junio, España ordenó que descansásemos en
Santibáñez de Valvaneda, pueblo que está sobre el camino de Béjar a
Salamanca. Teníamos provisiones relativamente abundantes, dada la gran
escasez de la época, y como reinaba en el ejército muy buena disposición
a divertirse, allí era de ver la algazara y alegría del pueblo a media
noche cuando tomamos posesión de las casas, y con las casas, de los
jergones y baterías de cocina.

Tocome habitar en el mejor aposento de una casa con resabios de palacio
y honores de mesón. Acomodó mi asistente para mí una hermosa cama, y no
tengo inconveniente en decir que me acosté, sí, señores, sin que nada
extraordinario ni con asomos de poesía me ocurriese en aquel acto vulgar
de la vida. Y también es cierto, aunque igualmente prosaico, que me
dormí, sin que el crepúsculo de mis sentidos me impresionase otra cosa
que la histórica canción cantada a media voz por mi asistente en la
estancia contigua:

\small
\newlength\mlena
\settowidth\mlena{Como va el Tormes por medio,}
\begin{center}
\parbox{\mlena}{\quad En el Carpio está Bernardo           \\
                y el Moro en el Arapil.                    \\
                Como va el Tormes por medio,               \\
                non se pueden combatir.}                   \\
\end{center}
\normalsize

Me dormí, y no se crea que ahora van a salir fantasmas, ni que los rotos
artesonados o vetustas paredes de la histórica casa, ogaño palacio y hoy
venta, se moverán para dar entrada a un deforme vestiglo, ni mucho menos
a una alta doncella de acabada hermosura que venga a suplicar me tome el
trabajo de desencantarla o prestarle cualquier otro servicio, ora del
dominio de la fábula, ora del de las bajas realidades. Ni esperen que
dueña barbuda, ni enano enteco, ni gigante fiero vengan súbito a hacerme
reverencias y mandarme les siga por luengos y oscuros corredores que
conducen a maravillosos subterráneos llenos de sepulturas o tesoros.
Nada de esto hallarán en mi relato los que lo escuchan. Sepan tan sólo
que me dormí. Por largo tiempo, a pesar de la profundidad del sueño, no
me abandonó la sensación del ruido que sonaba en la parte baja de la
casa. Las pisadas de los caballos retumbaban en mi cerebro con eco
lejano, produciendo vibración semejante a las de un hondo temblor de
tierra. Pero estos rumores cesaron poco a poco, y al fin todo quedó en
silencio. Mi espíritu se sumergió en esa esfera sin nombre, en que
desaparece todo lo externo, absolutamente todo, y se queda él solo,
recreándose en sí propio o jugando consigo mismo.

Pero de repente, no sé a qué hora, ni después de cuántas horas de sueño,
despertome una sensación singularísima, que no puedo descifrar, porque
sin que fuese afectado ninguno de mis sentidos, me incorporé rápidamente
diciendo: «¿quién está aquí?.

Ya despierto, grité a mi asistente:

---Tribaldos, levántate y enciende luz.

Casi en el mismo instante en que esto decía, comprendí mi engaño. Estaba
enteramente solo. No había ocurrido otra cosa sino que mi espíritu, en
una de sus caprichosas travesuras (pues esto son indudablemente las
fantasmagorías del sueño) había hecho el más común de todos, que
consiste en fingirse dos, con ilusoria y mentida división, alterando por
un instante su eternal unidad. Este misterioso \emph{yo y tú} suele
presentarse también cuando estamos despiertos.

Pero si en mi alcoba nada ocurría de extraño fuera de mí, como lo
demostró al entrar en ella Tribaldos alumbrando y registrando, algo
ocurría en los bajos del edificio, donde el grave silencio de la noche
fue interrumpido por fuerte algazara de gentes, coches y caballos.

---Mi comandante---dijo Tribaldos sacando el sable para dar tajos en el
aire a un lado y otro---esos pillos no quieren dejarnos dormir esta
noche. ¡Afuera, tunantes! ¿Pensáis que os tengo miedo?

---¿Con quién hablas?

---Con los duendes, señor---repuso.---Han venido a divertirse con usía,
después que jugaron conmigo. Uno me cogía por el pie derecho, otro por
el izquierdo, y otro más feo que Barrabás atome una cuerda al cuello,
con cuyo tren y el tirar por aquí y por allí me llevaron volando a mi
pueblo para que viese a Dorotea hablando con el sargento Moscardón.

---¿Pero crees tú en duendes?

---¡Pues no he de creer, si los he visto! Más paseos he dado con ellos
que pelos tengo en la cabeza---repuso con acento de convicción
profunda.---Esta casa está llena de sus señorías.

---Tribaldos, hazme el favor de no matar más mosquitos con tu sable.
Deja los duendes y baja a ver de qué proviene ese infernal ruido que se
siente en el patio. Parece que han llegado viajeros; pero según lo que
alborotan, ni el mismo sir Arturo Wellesley con todo su séquito traería
más gente.

Salió el mozo dejándome solo, y al poco rato le vi aparecer de nuevo,
murmurando entre dientes frases amenazadoras, y con desapacible mohín en
la fisonomía.

---¿Creerá mi comandante que son ingleses o príncipes viajantes los que
de tal modo atruenan la casa? Pues son cómicos, señor, unos comiquillos
que van a Salamanca para representar en las fiestas de San Juan. Lo
menos conté ocho entre damas y galanes, y traen dos carros con lienzos
pintados, trajes, coronas doradas, armaduras de cartón y mojigangas.
Buena gente\ldots{} El ventero les quiso echar a la calle; pero han
sacado dinero y su majestad el Sr.~Chiporro, al ver lo amarillo, les
tratará como a duques.

---¡Malditos sean los cómicos! Es la peor raza de bergantes que
hormiguea en el mundo.

---Si yo fuera D. Carlos España---dijo mi asistente demostrándome los
sentimientos benévolos de su corazón---cogería a todos los de la
compañía, y llevándoles al corral, uno tras otro, a toditos les
arcabuceaba.

---Tanto, no.

---Así dejarían de hacer picardías. Pedrezuela y su endemoniada mujer la
María Pepa del Valle, cómicos eran. Había que ver con qué talento hacía
él su papel de comisionado regio y ella el de la señora comisionada
regia. De tal modo engañaron a la gente, que en todos los pueblos por
donde corrían les creyeron, y en el Tomelloso, que es el mío, y no es
tierra de bobos, también.

---Ese Pedrezuela---dije, sintiendo que el sueño se apoderaba nuevamente
de mí---fue el que en varios pueblos de la margen del Tajo condenó a
muerte a más de sesenta personas.

---El mismo que viste y calza---repuso---pero ya las pagó todas juntas,
porque cuando el general Castaños y yo fuimos a ayudar al \emph{Lord} en
el bloqueo de Ciudad- Rodrigo, cogimos a Pedrezuela y a su mujercita y
los fusilamos contra una tapia. Desde entonces, cuando veo un cómico,
muevo el dedo buscando el gatillo.

Tribaldos salió para volver un momento después.

---Me parece que se marchan ya---dije advirtiendo cierto acrecentamiento
de ruido que anunciaba la partida.

---No, mi comandante---repuso riendo;---es que el sargento Panduro y el
cabo Rocacha han pegado fuego al carro donde llevan los trebejos de
representar. Oiga mi comandante chillar a los reyes, príncipes y
senescales al ver cómo arden sus tronos, sus coronas y mantos de armiño.
¡Cáspita; cómo graznan las princesas y archipámpanas! Voy abajo a ver si
esa canalla llora aquí tan bien como en el teatro\ldots{} El jefe de la
compañía da unos gritos\ldots{} ¿Oye, mi comandante?\ldots{} Vuelvo
abajo a verlos partir.

Claramente oí aquella entre las demás voces irritadas, y lo más extraño
es que su timbre, aunque lejano y desfigurado por la ira, me hizo
estremecer. Yo conocía aquella voz.

Levanteme precipitadamente y vestime a toda prisa; pero los ruidos
extinguiéronse poco a poco, indicando que las pobres víctimas de una
cruel burla de soldados, salían a toda prisa de la venta. Cuando yo
salía, entró Tribaldos y me dijo:

---Mi comandante, ya se ha ido esa flor y nata de la pillería. Todo el
patio está lleno con pedazos encendidos de los palacios de Varsovia y
con los yelmos de cartón y la sotana encarnada del Dux de Venecia.

---¿Y por qué lado se han ido esos infelices?

---Hacia Grijuelo.

---Es que van a Salamanca. Coge tu fusil y sígueme al momento.

---Mi comandante, el general España quiere ver a usía ahora mismo. El
ayudante de su excelencia ha traído el recado.

---El demonio cargue contigo, con el recado, con el ayudante y con el
general\ldots{} Pero me he puesto el corbatín del revés\ldots{} dame acá
esa casaca, bruto\ldots{} pues no me iba sin ella.

---El general le espera a usía. De abajo se sienten las patadas y voces
que da en su alojamiento.

Al bajar a la plaza, ya los incómodos viajeros habían desaparecido. D.
Carlos España me salió al encuentro diciéndome:

---Acabo de recibir un despacho del \emph{Lord}, mandándome marchar
hacia Santi Spíritus\ldots{} Arriba todo el mundo; tocar llamada.

Y así concluyó un incidente que no debiera ser contado, si no se
relacionara con otros curiosísimos que se verán a continuación.

\hypertarget{iv}{%
\chapter{IV}\label{iv}}

Dejando el camino real a la derecha, nos dirigimos por una senda áspera
y tortuosa para atravesar la sierra. Vino la aurora y el día sin que en
todo él ocurriese ningún suceso digno de ser marcado con piedra blanca,
negra ni amarilla, mas en el siguiente tuve un encuentro que desde luego
señalo como de los más felices de mi vida.

Marchábamos perezosamente al medio día sin cuidado ni precauciones, por
la seguridad de que no encontraríamos franceses en tan agrestes parajes.
Iban cantando los soldados, y los oficiales disertando en amena
conversación sobre la campaña emprendida, dejábamos a los caballos
seguir en su natural y pacífica andadura, sin espolearles ni
reprimirles. El día era hermoso, y a más de hermoso algo caliente, por
lo cual caía la llama del sol sobre nuestras espaldas, calentándolas más
de lo necesario.

Yo iba de vanguardia. Al llegar a la vista de San Esteban de la Sierra,
pueblo pequeño, rodeado de frondosa verdura y grata sombra de árboles, a
cuyo amparo habíamos resuelto sestear, sentí algazara en los primeros
grupos de soldados, que marchaban delante, rotas las filas y haciendo de
las suyas con los aldeanos que se parecían en el camino.

---No es nada, mi comandante---me contestó Tribaldos, a quien pregunté
la causa de tan escandalosa gritería.---Son Panduro y Rocacha que han
topado con un fraile agustino, y más que agustino pedigüeño, y más que
pedigüeño tunante, el cual no se apartó del camino cuando la tropa
pasaba.

---¿Y qué le han hecho?

---Nada más que jugar a la pelota---respondió riendo.---Su paternidad
llora y calla.

---Veo que Rocacha monta un asno y corre en él hacia el lugar.

---Es el asno de su paternidad, pues su paternidad trae un asno consigo
cargado de nabos podridos.

---Que dejen en paz a ese pobre hombre, ¡por vida de!\ldots---exclamé
con ira---y que siga su camino.

Adelanteme y distinguí entre soldados, que de mil modos le mortificaban,
a un bendito cogulla, vestido con el hábito agustino, y azorado y
lloroso.

---¡Señor---decía mirando piadosamente al cielo y con las manos
cruzadas---que esto sea en descargo de mis culpas!

Su hábito descolorido y lleno de agujeros cuadraba muy bien a la
miserable catadura de un flaquísimo y amarillo rostro, donde el polvo
con lágrimas o sudores amasado formaba costras parduscas. Lejos de
revelar aquella miserable persona la holgura y saciedad de los conventos
urbanos, los mejores criaderos de gente que se han conocido, parecía
anacoreta de los desiertos o mendigo de los caminos. Cuando se vio menos
hostigado, volvió a todos lados los ojos buscando su desgraciado
compañero de infortunio, y como le viese volver a escape y jadeando,
oprimidos los ijares por el poderoso Rocacha, se apresuró a acudir a su
encuentro. En tanto yo miraba al buen fraile, y cuando le vi volver,
tirando ya del cordel de su asno reconquistado, no pude reprimir una
exclamación de sorpresa. Aquella cara, que al pronto despertó vagos
recuerdos en mi mente, reveló al fin su enemiga, y a pesar de la edad
transcurrida y de lo injuriada que estaba por años y penas, la reconocí
como perteneciente a una persona con quien tuve amistad en otro tiempo.

---Sr.~Juan de Dios---exclamé deteniendo mi caballo a punto que el
fraile pasaba junto a mí.---¿Es usted o no el que veo dentro de esos
hábitos y detrás de esa capa de polvo?

El agustino me miró sobresaltado, y luego que por buen rato me
contemplara, díjome así con melifluo acento:

---¿De dónde me conoce el señor general? Juan de Dios soy, en efecto.
Doy las gracias a su eminencia por haber mandado que me devolvieran el
burro.

---¿Eminencia me llama usted\ldots?---repuse.---Todavía no me han hecho
cardenal.

---En mi turbación no sé lo que me digo. Si su alteza me da licencia, me
retiraré.

---Antes pruebe a ver si me conoce. ¿Mi cara ha variado tanto desde
aquel tiempo en que estábamos juntos en casa de D. Mauro Requejo?

Este nombre hizo estremecer al buen agustino, que fijó en mí sus ojos
calenturientos, y más bien espantado que sorprendido dijo:

---¿Será posible que el que tengo delante sea Gabriel? ¡Jesús mío! Señor
general, ¿es usted Gabriel, el que en Abril de 1808\ldots? Lo recuerdo
bien\ldots{} Deme usted a besar sus pies\ldots{} ¿Conque es Gabriel en
persona?

---El mismo soy. ¡Cuánto me alegro de que nos hayamos encontrado! Usted
hecho un frailito\ldots{}

---Para servir a Dios y salvar mi alma. Hace tiempo que abracé esta vida
tan trabajosa para el cuerpo como saludable para el alma. ¿Y tú,
Gabriel?\ldots{} ¿Y usted Sr.~D. Gabriel, se dedicó a la milicia?
También es honrosa vida la de las armas, y Dios premia a los buenos
soldados, algunos de los cuales santos han sido.

---A eso voy, padre, y usted parece que ya lo ha conseguido, porque su
pobreza no miente y su cara de mortificación me dice que ayuna los siete
reviernes.

---Yo soy un humildísimo siervo de Dios---dijo bajando los ojos---y hago
lo poco que está en mi miserable poder. Ahora, señor general,
experimento mucho gozo en ver a usted\ldots{} y en reconocer al generoso
mancebo que fue mi amigo, y con esto y su venia, me retiro, pues este
ejército va sierra adentro, y yo busco el camino real.

---No permito que nos separemos tan pronto, amigo mío, usted está
fatigado y además no tiene cara de haber cumplido aquel precepto que
manda empiece la caridad por uno mismo. En ese pueblo descansará el
regimiento. Vamos a comer lo que haya, y usted me acompañará para que
hablemos un poco, refrescando viejas memorias.

---Si el señor general me lo manda, obedeceré, porque mi destino es
obedecer---dijo marchando junto a mí en dirección al pueblo.

---Veo que el asno tiene mejor pelaje que su dueño y no se mortifica
tanto con ayunos y vigilias. Le llevará a usted como una pluma, porque
parece una pieza de buena andadura.

---Yo no monto en él---me respondió sin alzar los ojos del suelo.---Voy
siempre a pie.

---Eso es demasiado.

---Llevo conmigo este bondadoso animal para que me ayude a cargar las
limosnas y los enfermos que recojo en los pueblos para llevarlos al
hospital.

---¿Al hospital?

---Sí, señor. Yo pertenezco a la Orden Hospitalaria que fundó en Granada
nuestro santo padre y patrono mío el gran San Juan de Dios, hace
doscientos y setenta años poco más o menos. Seguimos en nuestros
estatutos la regla del gran San Agustín, y tenemos hospitales en varios
pueblos de España. Recogemos los mendigos de los caminos, visitamos las
casas de los pobres para cuidar a los enfermos que no quieren ir a la
nuestra y vivimos de limosnas.

---¡Admirable vida, hermano!---dije bajando del caballo y encaminándome
con otros oficiales y el hermano Juan a un bosquecillo que a la vera del
pueblo estaba, donde a la grata sombra de algunos corpulentos y frescos
árboles nos prepararon nuestros asistentes una frugal comida.

---Ate usted su burro en el tronco de un árbol---dije a mi antiguo
amigo---y acomódese sobre este césped junto a mí, para que demos al
cuerpo alguna cosa, que todo no ha de ser para el alma.

---Haré compañía al Sr.~D. Gabriel---dijo Juan de Dios humildemente
luego que ató la cabalgadura.---Yo no como.

---¿Qué no come? ¿Por ventura manda Dios que no se coma? ¿Y cómo ha de
estar dispuesto a servir al prójimo un cuerpo vacío? Vamos, Sr.~Juan de
Dios, deje a un lado esa cortedad.

---Yo no como viandas aderezadas en cocina, ni nada caliente y compuesto
que tenga olor a gastronomía.

---¿Llama gastronomía a este carnero fiambre y seco y a este pan más
duro que la roca?

---Yo no puedo probar eso---repuso sonriendo.---Me alimento tan sólo con
yerbas del campo y raíces silvestres.

---Hombre, lo admiro; pero francamente\ldots{} Al menos beberá usted un
trago. Es de Rueda.

---No bebo más que agua.

---¡Hombre\ldots{} agua y yerbecitas del campo! Lindo comistrajo es ese.
En fin, si de tal modo se salva uno\ldots{}

---Ya hace tiempo que hice voto firmísimo de vivir de esa manera, y
hasta hoy, D. Gabriel mío, aunque no limpio de pecados, tengo la
satisfacción de no haber cometido el de faltar a mi voto una sola vez.

---Pues no insisto, amigo. No se vaya usted a condenar por culpa mía. La
verdad es que tengo un hambre\ldots{} Pobre Sr.~Juan de Dios\ldots{}
¡Quién había de decir que nos encontraríamos después de tantos
años\ldots! ¿No es verdad?

---Sí señor.

---Yo creí que usted había pasado a mejor vida. Como desapareció\ldots{}

---Entré en la Orden en Enero del año 9. Acabé mis primeros ejercicios
en Marzo y recibí las primeras órdenes el año último. Todavía no soy
fraile profeso.

---¡Cuántas cosas han pasado desde que no nos vemos!

---¡Sí señor, cuántas!

---Usted, retirado del mundo, vive de un modo beatífico sin penas ni
alegrías, contento de su estado\ldots{}

Juan de Dios exhaló un suspiro profundísimo y después bajó los ojos.
Observándole bien, advertí las señales que en su extenuado rostro
patentizaban no ser jactancia de beato aquello de las campestres
yerbecitas y agua de los arroyos cristalinos. Bordeaba sus ojos un cerco
violáceo muy intenso que hacía más vivo el brillo de sus pupilas, y
marcándosele los huesos de la cara bajo la estirada y amarillenta piel.
Su expresión era la de las almas exaltadas por una piedad que igualmente
hace sus efectos en el espíritu y en el sistema nervioso. Misticismo y
enfermedad al mismo tiempo es una devoción singular que ha llevado
hermosísimas figuras al cielo de las grandezas humanas. Si en un
principio creí ver en Juan de Dios un poco de artificio e hipocresía,
muy luego convencime de lo contrario, y aquel santo varón arrojado por
las tempestades mundanas a la vida contemplativa y austera, estaba
inflamado por un fervor tan ardiente y verdadero. Se le veía quemarse,
se observaba la combustión de aquel cuerpo, que poco a poco se convertía
en ceniza, calcinado por la llama de la espiritual calentura; se veía
que aquel hombre apenas tocaba a la tierra, apenas al mundo de los
vivos, y que la miserable arcilla que aún mantenía el noble espíritu con
endeble atadura, se iba descomponiendo y desmenuzando grano a grano.

---Es admirable, amigo mío---le dije---que haya llegado a tan lisonjero
estado de santidad un hombre que no se vio libre ciertamente de las
pasiones mundanas.

La fisonomía de fray Juan de Dios contrájose con ligero temblor. Pero
serenándose al punto su rostro, me dijo:

---¿No sabe usted qué ha sido de aquellos benditos señores de Requejo?
Sentiría que les hubiese pasado alguna desgracia.

---No he vuelto a saber de ellos. Estarán cada vez más ricos, porque los
pícaros hacen fortuna.

El fraile no hizo gesto alguno de asentimiento.

---Pero Dios les habrá castigado al fin---continué---por los martirios
que hicieron padecer a aquella infeliz muchacha\ldots{} Al decir esto
advertí que en las venas de aquel miserable cuerpo humano, que la tumba
pedía para sí, quedaba todavía un resto de sangre. Bajo la piel de la
cara se traslucieron por un instante las hinchadas venas azules, y un
ligero tinte amoratado encendió la austera frente. No me hubiera
sorprendido más ver una imagen de madera sonrojándose al contacto del
beso de las devotas.

---Dios sabrá lo que tiene que hacer con los señores de Requejo por esa
conducta ---me contestó.

---Creo que no le será indiferente a usted saber el fin que ha tenido
aquella desgraciada joven.

---¿Indiferente? no---repuso poniéndose como un cadáver.

---¡Oh! Las personas destinadas a padecer\ldots---dije observando
atentamente la impresión que en el santo producían mis
palabras.---Aquella pobre joven tan buena, tan bonita, tan
modesta\ldots{}

---¿Qué?

---Ha muerto.

Yo creí que Juan de Dios se conmovería al oír esto; pero con gran
sorpresa vi su rostro resplandeciente de serenidad y beatitud. Mi
asombro llegó a su colmo cuando en tono de convicción profundísima,
dijo:

---Ya lo sabía. Murió en el convento de Córdoba, donde la encerró su
familia en Junio de 1808.

---¿Y cómo sabe usted eso?---pregunté respetando el engaño del pobre
agustino.

---Nosotros tenemos visiones singulares. Dios permite que por un estado
especial de nuestro espíritu, sepamos algunos hechos ocurridos en país
lejano, sin que nadie nos los cuente. Inés murió. Yo la he visto
repetidas veces en mis éxtasis, y es indudable que sólo se nos presenta
la imagen de las personas que han tenido la suerte de abandonar para
siempre este ruin y miserable mundo.

---Así debe de ser.

---Así es, aunque los torpes ojos del cuerpo crean otra cosa. ¡Ay! Los
del alma son los que no se engañan nunca, porque hay siempre en ellos un
rayo de eterna luz. La corporal vista es un órgano de quien dispone a su
antojo el demonio para atormentarnos. Lo que vemos en ella es muchas
veces ilusorio y fantástico. Yo, Sr.~D. Gabriel, padezco tormentos muy
horrorosos por las continuas pruebas a que sujeta mi espíritu el Señor
de cielo y tierra, y por los pérfidos amaños del ángel de las tinieblas,
que anhelando perderme, juega con mis débiles sentidos y se burla de
esta desgraciada criatura.

---Querido amigo, cuénteme usted lo que pasa. Yo también sirvo a veces
de juguete y mofa a ese señor demonio, y puedo dar a usted algún buen
consejo sobre el modo de vencerle y burlarse de él en vez de ser
burlado.

\hypertarget{v}{%
\chapter{V}\label{v}}

---Puesto que usted ha nombrado a una persona que tanta parte ha tenido
en que yo abandonase el perverso siglo, y puesto que usted conoció
entonces mis secretos, nada debo ocultarle. Cuando Dios me crió dispuso
que padeciese, y he padecido como ningún otro mortal sobre la tierra.
Antes de sentir en mi alma el rayo divino de la eterna gracia, que me
alumbró el sendero de esta nueva vida, una pasión mundana me hizo
desgraciado. Después que me abracé a la santa cruz para salvarme, las
turbaciones, debilidades y agonías de mi espíritu han sido tales, que
pienso es esto disposición de Dios para que conozca en vida infierno y
purgatorio antes de subir a la morada de los justos\ldots{} Amé a una
mujer, mas con tanta exaltación, que mi naturaleza quedó en aquel trance
trastornada. Cuando comprendí que todo había concluido, yo no tenía ya
entendimiento, memoria ni voluntad. Era una máquina, señor oficial, una
máquina estúpida: mis sentidos estaban muertos. Vivía en las tinieblas,
pues nada veía, y en una especie de letargoso asombro. Varias veces he
pensado después si como aquel estupor mío será el limbo a donde van los
que apenas han nacido.

---Justo. Así debe de ser.

---Cuando volví en mí, querido señor, formé el proyecto de hacerme
fraile. Yo había concluido para el mundo. Me confesé con grandísimo
fervor. El padre Busto aprobó con entusiasmo mi propósito de consagrar a
la religión el resto de mis tristes días, y como yo manifestara deseo de
entrar en la Orden más pobre y donde más trabajase el cuerpo y más
apartada de mundanales atractivos estuviese el ánima, señalome esta
regla de hermanos hospitalarios. ¡Ay! mi alma recibió un consuelo
inexplicable. Buscaba los sitios solitarios para meditar, y meditando
sentía rodeada mi cabeza de celestial atmósfera. ¡Qué luz tan pura! ¡Qué
dulzura y suave silencio en el aire!

---¿Y después?

---¡Ay! después empezaron nuevamente mis infortunios bajo otra forma.
Dios decretó que yo padeciese, y padeciendo estoy\ldots{} Oígame usted
un momento más. Comencé mis estudios y las prácticas religiosas para
ingresar en la Orden. Recibiéronme una mañana en el convento, donde
vestí el traje de lego. Di aquel día mis lecciones más contento que
nunca; asistí como fámulo a los pobres de la enfermería, y por la tarde,
tomando el segundo tomo de \emph{Los nombres de Cristo,} por el maestro
fray Luis de León, libro que me agradaba en extremo, fuime a la huerta y
en el sitio más secreto y callado de ella entregué mi espíritu a las
delicias de la lectura. No había acabado el capítulo hermosísimo que se
titula, \emph{Descripción de la miseria humana y origen de su
fragilidad}, cuando sentí un calofrío muy intenso en todo mi cuerpo, una
gran turbación, una zozobra muy viva, pues toda la sangre agolpose en mi
pecho, y experimenté una sensación que no puedo decir si era gozo
profundísimo o agudo dolor. Una extraña figura, bulto o sombra
impresionó mi vista, miré, y la vi; era ella misma, sentada en el banco
de piedra junto a mí.

---¿Quién?

---¿Necesito decir su nombre?

---Ya.

---El libro se me cayó de las manos, observé la asombrosa visión, pues
visión era, y el mundano amor renació violentamente en mi pecho como la
explosión de una mina. Quedé absorto, señor, mudo y entre suspendido y
aterrado. Era ella misma, y me miraba con sus dulces ojos,
trastornándome. Separábala de mí una distancia como de media vara; mas
no hice movimiento alguno para acercarme a ella, porque el mismo
estupor, la admiración que tal prodigio de belleza me producía, el mismo
fuego amoroso que quemaba mi ser, teníanme arrobado y sin movimiento.
Estaba vestida con riquísima túnica de una blanca y sutil tela, la cual,
así como las nubes ocultan el sol sin esconderlo, ocultaba su hermoso
cuerpo, antes empañándolo que cubriéndolo. Bajo la falda asomaba desnudo
uno de sus delicados pies; sus cabellos, ensortijados con arte
incomparable le caían en hermosas guedejas a un lado y otro de la cara
entre sartas de orientales perlas, y en la mano derecha sostenía un
pequeño ramillete de olorosas flores, cuya esencia llegaba hasta mí
embriagándome el sentido.

---En verdad, Sr.~Juan de Dios, que nunca he visto a la señorita Inés en
semejante traje, no muy propio por cierto para pasear en jardines.

---¿Qué había usted de verla, si aquella imagen no era forma corporal y
tangible, sino una fábrica engañosa del demonio, que desde aquel día me
escogió para víctima de sus abominables experimentos?

---¿Y la joven del pie desnudo y el ramo de flores, no dijo alguna
palabrilla?

---Ni media, hermano.

---¿Y usted no le dijo nada, ni traspasó el espacio de media vara que
había entre los dos?

---No podía hablar. Acerqueme, sí, a ella, y en el mismo momento
desapareció.

---¡Qué picardía! Pero el demonio es así; amigo mío: ofrece y no da.

---Mucho tardé en reponerme de la horrible sensación que aquello dejó en
mi alma. Al fin recogí el libro, y dirigí mis pensamientos a Dios. ¡Ay,
qué extraña sensación! Tan extraña es que no puedo explicarla. Figuraos,
querido señor, que mis pensamientos al remontarse al cielo tomando forma
material, fueran detenidos y rechazados por una mano poderosa. Esto ni
más ni menos era lo que yo sentía. Quería pensar y no tenía espíritu más
que para sentir. Por mi cuerpo corrían a modo de relámpagos del
movimiento, unas convulsiones ardientes\ldots{} ¡Ay! no, no puedo de
modo alguno explicar esto\ldots{} En mi cuerpo chisporroteaba algo, como
mechas que se van apagando, y cuyas pavesas mitad fuego mitad ceniza
caen al suelo\ldots{} Levanteme; quise entrar en la iglesia;
pero\ldots{} ¿creerá usted que no podía? No, no podía. Alguien me tiraba
de la cola del hábito hacia afuera. Corrí a la celda que me habían
destinado, y arrojándome en el suelo, puse la frente sobre mis manos y
mis manos sobre los ladrillos. Así estuve toda la noche orando y
pidiendo a Dios que me librara de aquellas horribles tentaciones,
diciéndole que yo no quería pecar sino servirle; que yo quería ser bueno
y puro y santo.

---¿Por qué no contó usted el caso a otros frailes experimentados en
cosas de visiones y tentaciones?

---Así lo hice al punto. Consulté aquella misma tarde con el padre
Rafael de los ángeles, varón muy pío y que me mostraba gran cariño, el
cual me dijo que no tuviese cuidado, pues para desnudar el entendimiento
(así mismo lo dijo), de tales aprensiones imaginarias y naturales,
bastaba una piedad constante, una mortificación infatigable y una
humildad sin límites. Añadiome que él en los primeros años de vida
monástica había experimentado iguales aprietos y compromisos, mas que al
fin con las rudas penitencias y lecturas místicas había convencido al
demonio de la inutilidad de sus esfuerzos para pervertirlo, con lo cual
le dejó tranquilo. Aconsejome que entrase en la vida activa de la Orden,
que marchase en pos de las miserias y lástimas del mundo, recogiendo
enfermos por los pueblos para traerlos a los hospitales; que vagase por
los campos, haciendo corporal ejercicio y alimentándome con yerbas y
raíces, para que el miserable y torpe cuerpo privado de todo regalo,
adquiriese la sequedad y rigidez que ahuyentan la concupiscencia.
Encargome además, que durmiese poco, y jamás sobre blanduras, sino más
bien encima de duras rocas o picudas zarzas, siempre que pudiere; que
asimismo me apartase de toda sociedad de amigos, esquivando coloquios
sobre negocios mundanos, no mostrando afición a persona alguna, sino
huyendo de todos para no pensar más que en la perfección de mi alma.

---Y haciéndolo así, ha conseguido usted\ldots{}

---Así lo he hecho, hermano, mas poco o nada he conseguido. Cerca de
tres años de mortificaciones, de ejercicios, de penitencias, de
vigilias, de rigores, de dormir en campo raso y comer berraza y
jaramagos crudos, si han fortalecido mi espíritu, librándome de aquellas
vaguedades voluptuosas que al principio ponían al borde del precipicio
mi santidad, no me han librado de los continuos asaltos del ángel
infernal, que un día y otro, señor, en el campo y bajo techo, en la
dulce oscuridad de la alta y triste noche, lo mismo que a la luz
deslumbradora del sol, me pone ante los ojos la imagen de la persona que
adoré en el siglo. ¡Ay! en aquel tiempo, cuando estábamos en la tienda,
yo blasfemé, sí\ldots{} me acuerdo que un día entré en la iglesia y
arrodillándome delante del Santísimo Sacramento, dije: «Señor, te
aborreceré, te negaré si no me la das, para que nuestras almas y
nuestros cuerpos estén siempre unidos en la vida, en la sepultura y en
la eternidad.» Dios me castiga por haberle amenazado.

---De modo que siempre\ldots{}

---Sí, siempre, siempre lo veo, unas veces en esta, otras en la otra
forma, aunque por temporadas el demonio me permite descansar y no veo
nada. Esta funesta desgracia mía me ha impedido hasta ahora recibir los
últimos y más sublimes grados del sacramento del Orden, pues me creo
indigno de que Dios baje a mis manos. ¡Es terrible sentirse uno con el
corazón y el espíritu todo dispuesto a la santidad, y no poder conseguir
el perfecto estado! Yo me desespero y lloro en silencio, al ver cuán
felices son otros frailes de mi Orden, los cuales disfrutan con la paz
más pura, las delicias de visiones santas, que son el más regalado
manjar del espíritu. Unos en sus meditaciones ven ante sí la imagen de
Cristo crucificado, mirándolos con ojos amorosísimos; otros se deleitan
contemplando la celestial figura del Niño Dios; a otros les embelesa la
presencia de Santa Catalina de Siena o Santa Rosa de Viterbo, cuya
castísima imagen y compuestos ademanes incitan a la oración y a la
austeridad; pero yo ¡desgraciado de mí! yo, pecador abominable, que
sentí quemadas mis entrañas por el mundano amor, y me alimenté con aquel
rocío divino de la pasión, y empapé el alma en mil liviandades
inspiradas por la fantasía, me he enfermado para siempre de impureza, me
he derretido y moldeado en un desconocido crisol que me dejó para
siempre en aquella ruin forma primera. No puedo ser santo, no puedo
arrojar de mí esta segunda persona que me acompaña sin cesar. ¡Oh
maldita lengua mía! Yo había dicho: «Quiero unirme a ella en la vida, en
la sepultura y en la eternidad,» y así está sucediendo.

Fray Juan de Dios bajó la cabeza y permaneció largo rato meditando.

\hypertarget{vi}{%
\chapter{VI}\label{vi}}

---¿En qué nuevas formas se ha presentado?---le pregunté.

---Una mañana iba yo por el campo, y abrasado por la sed busqué un
arroyo en que apagarla. Al fin bajo unos frondosos álamos que entre
peñas negruzcas erguían sus viejos troncos, vi una corriente cristalina
que convidaba a beber. Después que bebí senteme en una peña, y en el
mismo instante cogiome la singular zozobra que me anunciaba siempre la
influencia del ángel del mal. A corta distancia de mí estaba una
pastora; ella misma, señor, hermosa como los querubines.

---¿Y guardaba algún rebaño de vacas o carneros?

---No señor, estaba sola, sentada como yo sobre una peña, y con los
nevados pies dentro del agua, que movía ruidosamente haciendo saltar
frías gotas las cuales salpicando me mojaron el rostro. Había desatado
los negros cabellos y se los peinaba. No puedo recordar bien todas las
partes de su vestido; pero sí que no era un vestido que la vestía mucho.
Mirábame sonriendo. Quise hablar y no pude. Di un paso hacia ella y
desapareció.

---¿Y después?

---La volví a ver en distintos puntos. Yo me encontraba dentro de
Ciudad-Rodrigo cuando la asaltó el \emph{Lord} en Enero de este mismo
año. Hallábame sirviendo en el hospital, cuando comenzó el cerco, y
entonces otros buenos padres y yo salimos a asistir a los muchos heridos
franceses que caían en la muralla. Yo estaba aterrado, pues nunca había
visto mortandad semejante, e invocaba sin cesar a la divina Madre de
Nuestro Señor para que por su intercesión se amansase la furia de los
anglo-portugueses. El día 18 el arrabal, donde yo estaba, diome idea de
cómo es el infierno. Deshacíase en mil pedazos el convento de San
Francisco, donde íbamos colocando los heridos\ldots{} Los franceses
burlábanse de mí, y como a los frailes nos tenían mucha ojeriza por
creernos autores de la resistencia que se les hace, me maltrataron de
palabra y obra\ldots{} ¡Ay! cuando entraron los aliados en la plaza, yo
estaba herido, no por las balas de los sitiadores, sino por los golpes
de los sitiados. Los ingleses, españoles y portugueses entraron por la
brecha. Al oír aquel laberinto de imprecaciones victoriosas,
pronunciadas en tres idiomas distintos, sentí gran espanto. Unos y otros
se destrozaban como fieras\ldots{} yo exánime y moribundo, yacía en
tierra en un charco de sangre y fango y rodeado de cuerpos humanos.
Abrasábame una sed rabiosa, una sed, querido señor mío, tan ardiente
como si mis venas estuviesen llenas de fuego, y la boca, lengua y
paladar fuesen en vez de carne viva y húmeda, estopa inerte y seca. ¡Qué
tormento! Yo dije para mí: «Gracias a ti, Señor, que te has dignado
llevarme a tu seno. Ha llegado la hora de mi muerte.» No había acabado
de decirlo, mejor dicho, de pensarlo, cuando sentí en mis labios el
celeste contacto del agua fresca. Suspiré y mi espíritu sacudió su
fúnebre sopor. Abrí los ojos y vi pegada a mis ardientes labios una
blanca mano, en cuya palma ahuecada brillaba el cristalino licor tan
fresco y puro como el manar de la rústica fuente.

---¿Y en qué traza venía entonces la señorita Inés?

---Venía de monja.

---¿Y las monjas daban de beber en el hueco de la mano?

---Aquélla sí. Pintar a usted cuán hermosa estaba su cara entre las
blancas tocas y cuán bien le sentaba la austeridad de la pobre estameña
del traje, me sería imposible. Apenas la miré cuando voló de súbito,
dejándome más sediento que antes.

---Una cosa me ocurre, Sr.~Juan de Dios---dije condolido en extremo de
la extraña enfermedad del desgraciado hospitalario---y es que siendo esa
persona un artificio del más malo, del más pícaro y desvergonzado
espíritu creado por Dios, y habiendo ocasionado a usted tantos
disgustos, congojas, mortales ansias y acalorados paroxismos, parecía
natural que la tomase usted en aborrecimiento y que viese en ella más
bien una espantable y horrenda fealdad que ese portento de hermosura que
con tanto deleite encarece.

Fray Juan de Dios suspiró tristemente y me dijo:

---El Malo no presenta jamás a nuestros ojos cosas aborrecibles ni
repugnantes, sino antes bien hermosas, odoríferas, o gratas al paladar,
al olfato, al oído y al tacto. Bien sabe él lo que se hace. Si ha leído
usted la vida de la madre Santa Teresa de Jesús, habrá visto que alguna
vez el demonio le pintó delante la imagen de Nuestro Señor Jesucristo
para engañarla. Ella misma dice que el Malo es gran pintor y añade que
cuando vemos una imagen muy buena, aunque supiésemos la ha pintado un
mal hombre, no dejaríamos de estimarla.

---Eso está muy bien dicho\ldots{} Se me ocurre otra cosa. Si yo hubiera
sido atormentado de esa ruin manera por el espíritu maligno, el cual
según voy viendo es un redomado tunante, habría tratado de perseguir la
imagen, de tocarla, de hablarle, para ver si efectivamente era vana
ilusión o materia corpórea.

---Yo lo he hecho, querido señor y amigo mio---repuso el hospitalario
con acento ya debilitado por el mucho hablar---y nunca he podido poner
mis manos sobre ella, habiendo conseguido tan sólo una vez tocar el
halda de su vestido. Puedo asegurar a usted que a la vista su figura se
me ha representado siempre como una criatura humana con su natural
espesor, corpulencia y el brillo y la dulzura de los ojos, el dulce
aliento de la boca, y la añadidura del vestido flotando al viento, en
fin, todo en tal manera fabricado que es imposible no creerla persona
viva y como las demás de nuestra especie.

---¿Y siempre se presenta sola?

---No señor, que algunas veces la he visto en compañía de otras
muchachas, como por ejemplo en Sevilla el año pasado. Todas eran obra
vana de la infernal industria, pues desaparecieron con ella, como
multitud de luces que se apagan de un solo soplo.

---¿Y siempre desaparecen así como luz que se apaga?

---No señor, que a veces corre delante de mí, y la sigo, y o se pierde
entre la multitud, o avanza tanto en su camino que no puedo alcanzarla.
Un día la vi en una soberbia cabalgadura que corría más que el viento, y
ayer la vi en un carro.

---¿Que corría también como el viento?

---No señor, pues apenas corría como un mal carro. La visión de ayer
ofrece para mí una particularidad aterradora, y que me prueba cierta
recrudescencia y gravedad del mal que padezco.

---¿Por qué?

---Porque ayer me habló.

---¿Cómo?---exclamé sonriendo, mas no asombrado del extremo a que
llegaban las locuras de mi amigo.

---¿Habló al fin la señorita del pie desnudo, la pastora, la monja de
Ciudad-Rodrigo?

---Sí señor. Iba en un carro en compañía de unos cómicos que venían al
parecer de Extremadura.

---¡En un carro!\ldots{} ¡Con unos cómicos!\ldots{} ¡De Extremadura!

---Sí señor: veo que se asombra usted y lo comprendo, porque el caso no
es para menos. Delante iban algunos hombres a caballo; luego seguía un
carro con dos mujeres, y después otro carro con decoraciones y trebejos
de teatro, todos quemados y hechos pedazos.

---Hermano, usted se burla de mí---dije levantándome de súbito y
volviéndome a sentar, impulsado por ardiente desasosiego.

---Cuando la vi, señor mío, experimenté aquel calofrío, aquella
sensación entre placentera y dolorosa que acompaña a mis terribles
crisis.

---¿Y cómo iba?

---Triste, arropada en un manto negro.

---¿Y la otra mujer?

---Engañosa imaginación también, sin duda, la acompañaba en silencio.

---¿Y los hombres que iban a caballo?

---Eran cinco, y uno de ellos vestía de juglar con calzón de tres
colores y montera de picos. Disputaban, y otro de ellos, que parecía
mandar a todos, era una persona de buena apostura y presencia, con barba
picuda como la del demonio.

---¿No sintió usted olor de azufre?

---Nada de eso, señor. Aquellos hombres hablaban con animación y
nombraron a unos soldados que les habían quemado sus infernales
cachivaches.

---Sospecho, querido hermano Juan---dije con turbación---que ya no es
usted solo el endemoniado, sino que yo lo estoy también, pues esos
cómicos, y esas mujeres, y esos carros, y esos trastos escénicos son
reales y efectivos, y aunque no los vi, sé que estuvieron en Santibáñez
de Valvaneda. ¿Sería que alguna de las cómicas se le antojó a usted ser
la misma persona de marras, sin que en esto hubiese la más ligera
picardía por parte de la majestad infernal?

---Bien he dicho yo---continuó el fraile con candor---que esta aparición
de hoy es la más extraordinaria y asombrosa que he tenido en mi vida,
pues en ella la demoniaca hechura ha presentado tales síntomas, señales
y vislumbres de realidad, que al más licurgo y despreocupado engañaría.
Esta es también la primera vez que la imagen querida, además de tomar
cuerpo macizo de mujer, ha remedado la humana voz.

---¿Ha hablado?

---Sí señor; ha hablado---dijo el hospitalario con terror.---Su voz no
es la misma que aún resuena en mis oídos, desde que la oí en casa de
Requejo, así como su figura en el día de hoy me ha parecido más hermosa,
más robusta, más completa y más formada. Tal como la vi en el convento,
en el bosque, en la iglesia y en Ciudad-Rodrigo era casi una niña, y
hoy\ldots{}

---Pero si habló, ¿qué dijo?

---Yo me acerqué al carro, la miré, mirome ella también\ldots{} Sus ojos
eran rayos que me quemaban cuerpo y alma. Luego apareció asombrada, muy
asombrada\ldots{} ¡Ay! sus labios se movieron y pronunciaron mi propio
nombre. «Sr.~Juan de Dios, dijo, ¿se ha hecho usted fraile?\ldots» Me
pareció que iba yo a morir en aquel mismo momento. Quise hablar y no
pude. Ella hizo ademán de darme una limosna, y de pronto el hombre que
parecía mandar a todos, como advirtiera mi presencia junto al carro de
las cómicas, detuvo el caballo, y volviéndose me dijo con voz fiera:
«Largo de aquí, holgazán pancista.» Ella dijo entonces: «Es un pobre
mendicante que pide limosna.» El hombre alzó el palo para pegarme y ella
dijo: «Padre, no le hagas daño.»

---¿Está usted seguro de que dijo eso?

---Sí, seguro estoy; mas el infame, como criatura infernal que era,
enemigo natural de las personas consagradas al servicio de Dios, llamome
de nuevo holgazán, y recibí al mismo tiempo tal porrazo en la cabeza,
que caí sin sentido.

---Sr.~Juan de Dios---le dije después de reflexionar un poco sobre lo
extraño de aquella aventura---júreme usted que es verdad cuanto ha dicho
y que no es su ánimo burlarse de mí.

---¡Yo burlarme, señor oficial de mi alma!---exclamó el hospitalario,
que estuvo a punto de llorar viendo que se ponía en duda su
veracidad.---Cierto es lo que he dicho, y tan evidente es que hay
demonio en el infierno, como que hay Dios en el cielo, pues infinito es
en el mundo el número de casos de obsesión, y todos los días oímos
contar nuevas tropelías y estupendas gatadas del mortificador del linaje
humano.

---¿Y no puede usted precisar el sitio en que ocurrió eso del carro de
comediantes?

---Pasado Santibáñez de Valvaneda, como a tres leguas. Iban a buen paso
camino de Salamanca.

El infeliz hospitalario no podía mentir, y en cuanto a la endemoniada
composición de las cosas y personas referidas, yo tenía mis razones para
creer que entre los primeros y el último encuentro del fraile había
alguna diferencia.

De nuevo le insté para que tomase alguna cosa, y segunda vez se resistió
a dar a su cuerpo regalo alguno. Ya nos disponíamos a marchar, cuando le
vi palidecer, si es que cabía mayor grado de amarillez en su amojamada
carne; le vi aterrado, con los ojos medio salidos del casco, el labio
inferior trémulo y toda su persona desasosegada. Miraba a un punto fijo
detrás de mí, y como yo rápidamente me volviese y nada hallase que
pudiera motivar aquel espanto, le pregunté la causa de sus terrores y si
allí entre tantos soldados se atrevía Satanás a hacer de las suyas.

---Ya se ha desvanecido---dijo con voz débil y dejando caer
desmayadamente los brazos.

---¿Pues qué, otra vez ha estado aquí?

---Sí en aquel grupo donde bailan los soldados\ldots{} ¿Ve usted que hay
allí unas mozas de San Esteban?

---Es cierto; pero o yo he olvidado la cara de la señora Inés, o no está
entre ellas---repuse sin poder contener la risa.---Si estuviera, bien se
le podían decir cuatro frescas por ponerse a bailar con los soldados.

---Pues dude usted de que ahora es de día, señor mío---afirmó no
repuesto aún de la emoción---pero no dude usted de que estaba allí. Veo
que el demonio recrudece sus tentaciones y aumenta el rigor de sus
ataques contra los reductos de mi fortaleza, y esto lo hace porque estoy
pecando\ldots{}

---¿Pecando ahora, pecando por hablar con un antiguo amigo?

---Sí señor, pues pecar es entregar sin freno el espíritu a los deleites
de la conversación con gente seglar. Además he estado aquí descansando
más de hora y media, cosa que en tres años no he hecho, y he gustado de
la fresca sombra de estos árboles. Alma mía---añadió con exaltado
fervor---arriba, no duermas, vigila sin cesar al enemigo que te acecha,
no te entregues al corruptor deleite de la amistad, ni desmayes un solo
momento, ni pruebes las dulzuras del reposo. Alerta, alerta siempre.

---¿Se marcha usted ya?---dije, al ver que desataba al buen
pollino.---Vamos, no rechazará usted este pedazo de pan para el camino.

Tomolo y poniéndoselo en la boca al pacífico asno, que no estaba sin
duda por cenobíticas abstinencias, cogió él para sí un puñado de yerba y
la guardó en el seno.

---O es un farsante---dije para mí---o el más puro y candoroso beato que
ciñe el cíngulo monacal.

---Buenas tardes, Sr.~D. Gabriel---dijo con humilde acento.---Me voy a
Béjar para seguir mañana a Candelario, donde tenemos un hospital. ¿Y
usted, a dónde marcha?

---¿Yo? a donde me lleven; tal vez a conquistar a Salamanca, que está en
poder de Marmont.

---Adiós, hermano y querido señor mío---repuso.---Gracias, mil gracias
por tantas bondades.

Y tirando del ronzal, partió con el burro tras sí. Cuando su enjuta
figura negruzca se alejó al bajar un cerro, pareciome ver en él un
cuerpo que melancólicamente buscaba su perdida sepultura sin poder
encontrarla.

\hypertarget{vii}{%
\chapter{VII}\label{vii}}

Dos días después, más allá de Dios le guarde, un gran acontecimiento
turbó la monotonía de nuestra marcha. Y fue que a eso de la madrugada
nuestras tropas avanzadas prorrumpieron en exclamaciones de júbilo;
mandose formar, dando a las compañías el marcial concierto y la buena
apariencia que han menester para presentarse ante un militar
inteligente, y algunos acudieron por orden del general a cortar ramos a
los vecinos carrascales para tejer no sé si coronas, cenefas o
triunfales arcos. Al llegar al camino de Ciudad-Rodrigo vimos que
apareció falange numerosa de hombres vestidos de encarnado y caballeros
en ligerísimos corceles; verlos y exclamar todos en alegre concierto:
«¡Viva el \emph{Lord!»} fue todo uno.

---Es la caballería de Cotton de la división del general Graham---dijo
D. Carlos España.---Señores, cuidado no hagamos alguna gansada. Los
ingleses son muy ceremoniosos y se paran mucho en las formas. Si se coge
bastante carrasca haremos un arquito de triunfo para que pase por él el
vencedor de Ciudad-Rodrigo, y yo le echaré un discurso que traigo
preparado elogiando su pericia en el arte de la guerra y la Constitución
de Cádiz, cosas ambas bonísimas, y a las cuales deberemos el triunfo al
fin y a la postre.

---No es el señor \emph{Lord} muy amigo de la Constitución de
Cádiz---dijo D. Julián Sánchez, que a derecha mano de D. Carlos
estaba;---pero a nosotros ¿qué nos va ni qué nos viene en esto?
Derrotemos a Marmont y vivan todos los milores.

Los jinetes rojos llegaron hasta nosotros, y su jefe, que hablaba
español como Dios quería, cumplimentó a nuestro brigadier, diciéndole
que su excelencia el señor duque de Ciudad-Rodrigo no tardaría en llegar
a Santi Spíritus.

Al punto comenzamos a levantar el arco con ramajes y palitroques a la
entrada de dicho pueblo, y vierais allí que un dómine del país apareció
trayendo unos al modo de tarjetones de lienzo con sendos letreros y
versos que él mismo había sacado de su cabeza, y en las cuales piezas
poéticas se encomiaban hasta más allá de los cuernos de la luna las
virtudes del moderno Fabio, o sea el Sr.~D. Arturo Wellesley,
\emph{Lord} vizconde de Wellington de Talavera, duque de Ciudad-Rodrigo,
grande de España y par de Inglaterra.

Iban llegando unos tras otros numerosos cuerpos de ejército, que se
desparramaban por aquellos contornos ocupando los pueblos inmediatos, y
al fin entre los más brillantes soldados escoceses, ingleses y
españoles, apareció una silla de postas, recibida con aclamaciones y
vítores por las tropas situadas a un lado y otro del camino. Dentro de
ella vi una nariz larga y roja, bajo la cual lucieron unos dientes
blanquísimos. Con la rapidez de la marcha apenas pude distinguir otra
cosa que lo indicado y una sonrisa de benevolencia y cortesía que desde
el fondo del carruaje saludó a las tropas.

No debo pasar en silencio, aunque esto concuerde mal con la gravedad de
la historia, que al pasar el coche bajo el arco triunfal, como este no
lo habían construido ingenieros ni artífices romanos, con la sacudida y
golpe que recibiera de una de las ruedas, hizo como si quisiera venirse
abajo, y al fin se vino, cayendo no pocas ramas y lienzos sobre la
cabeza del dómine que tuviera parte tan importante en su malhadada
fábrica. Como no hubo que lamentar desgracia alguna, celebrose con risas
la extraña ruina. Los chicos apoderáronse al punto de los tarjetones,
que eran como de tres cuartas de diámetro, y abriéndoles en el centro un
agujero y metiendo por él la cabeza se pasearon delante de Wellington
con aquella valona o flamenca golilla.

Entre tanto D. Carlos España desembuchaba su discurso delante del
\emph{Lord}, y luego que concluyera, presentose el dómine con el
amenazador proyecto de hablar también. Consintiolo el general, que como
persona finísima disimulaba su cansancio, y oyendo las pedanterías del
orador, movía la cabeza, acompañando sus gestos de la especial sonrisa
inglesa, que hace creer en la existencia de algún cordón
intermandibular, del cual tiran para plegar la boca como si fuera una
cortina.

---Mi comandante---me dijo con cara de júbilo mi asistente cuando me
aparté de los generales para ocuparme del alojamiento,---¿no ha visto
usía el otro ejército que viene detrás?

---Serán los portugueses.

---¡Qué portugueses ni qué garambainas! Son mujeres, un ejército de
mujeres. Esto se llama darse buena vida. Los ingleses, en vez de
impedimenta llevan la faldamenta. Así da gusto de hacer la guerra.

Miré y vi veinte, ¿qué digo, veinte? cuarenta y aun cincuenta carros,
coches y vehículos de distintas formas, llenos todos de mujeres, unas al
parecer de alta, otras de baja calidad, y de distinta belleza y edad,
aunque por lo general, dicho sea esto imparcialmente, predominaba el
género feo. Al punto que pararon los vehículos entre nubes de polvo,
vierais descender con presteza a las señoras viajeras y resonar una de
las más discordes algarabías que pueden oírse. Por un lado chillaban
ellas llamando a sus consortes, y ellos por otro penetraban en la
femenil multitud gritando: \emph{Anna, Fanny, Mathilda, Elisabeth}. En
un instante formáronse alegres parejas, y un tumultuoso concierto de
voces guturales y de inflexiones agudas y de articulaciones líquidas
llenó los aires.

Pero como la división aliada que acababa de llegar no podía pernoctar
entera en aquel pueblo, una parte de ella siguió el camino adelante
hacia Aldehuela de Yeltes. Tornaron a montar en sus carricoches muchas
de las hembras formando parte del convoy de víveres y municiones, y
otras quedaron en Santi Spíritus. El día pasó, ocupándonos todos en
buscar el mejor alojamiento posible; pero como éramos tantos, al caer de
la tarde no habíamos resuelto la cuestión. En cuanto a mí, me creía
obligado a dormir en campo raso. Tribaldos me notificó que el dómine del
lugar tenía sumo placer en cederme su habitación. Después de visitar a
mi honrado patrono, salí a desempeñar varias obligaciones militares, y
ya me retiraba a casa, cuando junto al camino sentí gritos y voces de
alarma. Corrí a donde sonaban, y no era más sino que por el camino
adelante venía un cochecillo cuyo caballo le arrastraba dando tan
terribles tumbos y saltos, que cada instante parecía iba a deshacerse en
pedazos mil. Cuando con rapidez inmensa pasó por delante de nosotros, un
grito de mujer hirió mis oídos.

---En ese coche va una mujer, Tribaldos---grité a mi asistente que se
había unido a mí.

---Es una inglesa, señor, que se quedó rezagada y detrás de las demás.

---¡Pobre mujer!\ldots{} ¿Y no hay entre tantos hombres uno solo que se
atreva a detener el caballo y salvar a esa desgraciada?\ldots{} Parece
que no va desbocado\ldots{} Detiene el paso\ldots{} Corramos allá.

---El coche se ha salido del camino---dijo Tribaldos con espanto---y ha
parado en un sitio muy peligroso.

Al instante vi que el carricoche estaba a punto de despeñarse.
Habiéndose enredado el caballo entre unas jaras, se había ido al suelo,
quedando como reventado a consecuencia del fuerte choque que recibiera.
Pero como la pendiente era grande, la gravedad lo atraía hacia lo hondo
del barranco.

Me era imposible ver la situación terrible de la infeliz viajera sin
acudir pronto a su socorro. Había caído el coche sin romperse; mas lo
peligroso estaba en el sitio. Corrí allá solo, bajé tropezando a cada
paso y despegando con mi planta piedrecillas que rodaban con ruido
siniestro, y llegué al fin adonde se había detenido el vehículo. Una
mujer lanzaba desde el interior lastimeras voces.

---Señora---grité---allá voy. No tenga usted cuidado. No caerá al
barranco.

El caballo pataleaba en el suelo, pugnando por levantarse y con sus
movimientos de dolor y desesperación arrastraba el coche hacia el
abismo. Un momento más y todo se perdía.

Apoyeme en una enorme piedra fija, y con ambas manos detuve el coche que
se inclinaba.

---Señora---grité con afán---procure usted salir. Agárrese usted a mi
cuello\ldots{} sin miedo. Si salta usted en tierra no hay qué temer.

---No puedo, no puedo, caballero---exclamó con dolor.

---¿Se ha roto usted alguna pierna?

---No, caballero\ldots{} veré si puedo salir.

---Un esfuerzo\ldots{} Si tardamos un instante los dos caeremos abajo.

No puedo describir los prodigios de mecánica que ambos hicimos. Ello es
que en casos tan apurados, el cuerpo humano, por maravilloso instinto,
imprime a sus miembros una fuerza que no tiene en instantes ordinarios,
y realiza una serie de admirables movimientos que después no pueden
recordarse ni repetirse. Lo que sé es que como Dios me dio a entender, y
no sin algún riesgo mío, saqué a la desconocida de aquel grave
compromiso en que se encontraba, y logré al fin verla en tierra. Asido a
las piedras la sostuve y me fue forzoso llevarla en brazos al camino.

---Eh, Tribaldos, cobarde, holgazán---grité a mi asistente que había
acudido en mi auxilio,---ayúdame a salir de aquí.

Tribaldos y otros soldados, que no me habían prestado socorro hasta
entonces, me ayudaron a salir; porque es condición de ciertas gentes no
arrimarse al peligro que amenaza sino al peligro vencido, lo cual es
cómodo y de gran provecho en la vida.

Una vez arriba, la desconocida dio algunos pasos.

---Caballero, os debo la vida---dijo recobrando el perdido color y el
brillo de sus ojos.

Era como de veinte y tres años, alta y esbelta. Su airosa figura, su
acento dulce, su hermoso rostro, aquel tratamiento de vos que
ceremoniosa me daba, sin duda por poseer a medias el castellano, me
hicieron honda y duradera impresión.

\hypertarget{viii}{%
\chapter{VIII}\label{viii}}

Apoyose en mí, quiso dar algunos pasos; mas al punto sus piernas
desmayadas se negaron a sostenerla. Sin decir nada la tomé en brazos y
dije a Tribaldos:

---Ayúdame; vamos a llevarla a nuestro alojamiento.

Por fortuna este no estaba lejos, y bien pronto llegamos a él. En la
puerta la inglesa movió la cabeza, abrió los ojos y me dijo:

---No quiero molestaros más, caballero. Podré subir sola. Dadme el
brazo.

En el mismo momento apareció presuroso y sofocado un oficial inglés,
llamado sir Tomás Parr, a quien yo había conocido en Cádiz, y enterado
brevemente de la lamentable ocurrencia, habló con su compatriota en
inglés.

---¿Pero habrá aquí una habitación \emph{confortable} para la
señora?---me dijo después.

---Puede descansar en mi propia habitación---dijo el dómine que había
bajado oficiosamente al sentir el ruido.

---Bien---dijo el inglés.---Esta señorita se detuvo en Ciudad-Rodrigo
más de lo necesario y ha querido alcanzarnos. Su temeridad nos ha dado
ya muchos disgustos. Subámosla. Haré venir al médico mayor del ejército.

---No quiero médicos---dijo la desconocida.---No tengo herida grave: una
ligera contusión en la frente y otra en el brazo izquierdo.

Esto lo decía subiendo apoyada en mi brazo. Al llegar arriba dejose caer
en un sillón que en la primera estancia había y respiró con desahogo
expansivo.

---A este caballero debo la vida---dijo señalándome.---Parece un
milagro.

---Mucho gusto tengo en ver a usted, mi querido Sr.~Araceli---me dijo el
inglés.---Desde el año pasado no nos habíamos visto. ¿Se acuerda usted
de mí\ldots{} en Cádiz?

---Me acuerdo perfectamente.

---Usted se embarcó con la expedición de Blake. No pudimos vernos porque
usted se ocultó después del duelo en que dio la muerte a lord Gray.

La inglesa me miró con profundo interés y curiosidad.

---Este caballero\ldots---dijo.

---Es el mismo de quien os he hablado hace días---contestó Parr.

---Si el libertino que ha hecho desgraciadas a tantas familias de
Inglaterra y España hubiese tropezado siempre con hombres como
vos\ldots{} Según me han dicho, lord Gray se atrevió a mirar a una
persona que os amaba\ldots{} La energía, la severidad y la nobleza de
vuestra conducta son superiores a estos tiempos.

---Para conocer bien aquel suceso---dije yo, no ciertamente orgulloso de
mi acción,---sería preciso que yo explicase algunos
antecedentes\ldots---Puedo aseguraros que antes de conoceros, antes de
que me prestaseis el servicio que acabo de recibir, sentía hacia vos una
grande admiración.

Dije entonces todo lo que la modestia y el buen parecer exigían.

---¿De modo que esta señora se alojará aquí?,---me dijo Parr.---Donde yo
estoy, es imposible. Dormimos siete en una sola habitación.

---He dicho que le cederé la mía, la cual es digna del mismo sir
Arturo,---dijo Forfolleda, pues este era el nombre del dómine.

---Entonces estará bien aquí.

Sir Tomás Parr habló largamente en inglés con la bella desconocida y
después se despidió. No dejaba de causarme sorpresa que sus compatriotas
abandonasen a aquella hermosa mujer que sin duda debía de tener esposo o
hermanos en el ejército; pero dije para mí: «será que las costumbres
inglesas lo ordenan de este modo.»

En tanto la señora de Forfolleda (pues Forfolleda tenía señora) bizmó el
brazo de la desconocida, y restañó la sangre de la rozadura que
recibiera en la cabeza, con cuya operación dimos por concluidos los
cuidados quirúrgicos y pensamos en arreglar a la señora cuarto y cama en
que pasar la noche.

Un momento después el precioso cuerpo de la dama inglesa descansaba
sobre un lecho algo más blando que una roca, al cual tuve que conducirla
en mis brazos, porque la acometió nuevamente aquel desmayo primero que
la imposibilitaba toda acción corporal. Ella me dio las gracias en
silencio volviendo hacia mí sus hermosos ojos azules, que dulcemente y
con la encantadora vaguedad y extravío que sigue a los desmayos se
fijaron, primero en mi persona y después en las paredes de la
habitación. Más la miraba yo y más hermosa me parecía a cada momento. No
puedo dar idea de la extremada belleza de sus ojos azules. Todas las
facciones de su rostro distinguíanse por la más pura corrección y
finura. Los cabellos rubios hacían verosímil la imagen de las trenzas de
oro tan usada por los poetas, y acompañaban la boca los más lindos y
blancos dientes que pueden verse. Su cuerpo atormentado bajo las
ballenas de un apretado jubón, del cual pendían faldas de amazona, era
delgadísimo, mas no carecía de las redondeces y elegantes contornos y
desigualdades que distinguen a una mujer de un palo torneado.

---Gracias, caballero---me dijo con acento melancólico y usando siempre
el \emph{vos.---} Si no temiera molestaros, os suplicaría que me dieseis
algún alimento.

---¿Quiere la señora un pedazo de pierna de carnero---dijo Forfolleda,
que arreglaba los trastos de la habitación,---unas sopas de ajo,
chocolate o quizás un poco de salmorejo con guindilla? También tengo
abadejo. Dicen que al Sr.~D. Arturo le gusta mucho el abadejo.

---Gracias---repuso la inglesa con mal humor,---no puedo comer eso. Que
me hagan un poco de té.

Fui a la cocina, donde la señora de Forfolleda me dijo que allí no había
té ni cosa que lo pareciese, añadiendo que si ella probara tan sólo un
buche de tal enjuagadero de tripas, arrojaría por la boca juntamente con
los hígados la primer leche que mamó. Luego se puso a reprender a su
esposo por admitir en la casa a herejes luteranos y calvinistas, cuales
eran los ingleses; mas el dómine refutó victoriosamente el ataque
afirmando que merced a la ayuda de los herejes luteranos y calvinistas,
la católica España triunfaría de Napoleón, lo cual no significaba más
sino que Dios se vale del mal para producir el bien.

---Vete a cualquier casa donde haya ingleses---dije a Tribaldos,---y
trae té. ¿Sabes lo que es?

---Unas hojas arrugaditas y negras. Ya sé\ldots{} todas las noches lo
tomaba la mujer del capitán.

Volví al lado de la inglesa que me dijo no podía comer cosa alguna de
nuestra cocina, y habiéndome pedido pan, se lo di mientras llegaba el
anhelado té.

Al poco rato entró Tribaldos trayendo una ancha taza que despedía un
olor extraño.

---¿Qué es esto?---dijo la dama con espanto, cuando los vapores del
condenado licor llegaron a su nariz.

---¿Qué menjurgue has puesto aquí, maldito?---exclamé amenazando al
aturdido mozo.

---Señor, no he puesto nada, nada más que las hojas arrugaditas, con un
poco de canela y de clavo. La señora de Forfolleda dijo que así se
hacía, y que lo había compuesto muchas veces para unos ingleses que
fueron a Salamanca a ver la catedral vieja.

La inglesa prorrumpió en risas.

---Señora, perdone usted a este animal que no sabe lo que hace. Voy yo
mismo a la cocina y beberá usted té.

Poco después volví con mi obra, que debió de satisfacer a la interesada,
pues la aceptó con gozo.

---Ahora, señora mía, me retiraré, para que usted descanse---le
dije.---Deme usted órdenes para mañana o para esta noche misma. Si
quiere usted que avise a su esposo\ldots{} o es que se halla en la
división de Picton que no está en este pueblo\ldots{}

---Señor oficial---dijo solemnemente bebiendo su té,---yo no tengo
esposo; yo soy soltera.

Esto puso el límite a mi asombro, y vacilante al principio en mis ideas
no supe contestarle sino con medias palabras.

---¡Buena pieza será ésta que se ha colgado de mi brazo!---dije para
mí.---Los franceses traen consigo mujeres de mala vida, pero de los
ingleses, no sabía que\ldots{}

---Soltera, sí---añadió con aplomo y apartando la taza de sus
labios.---Os asombráis de ver una señorita como yo en un campo de
batalla, en tierra extranjera y lejos, muy lejos de su familia y de su
patria. Sabed que vine a España con mi hermano, oficial de ingenieros de
la división de Hill, el cual hermano mío pereció en la sangrienta
batalla de Albuera. El dolor y la desesperación tuviéronme por algunos
días enferma y en peligro de muerte; pero me reanimó la conciencia de
los deberes que en aquel trance tenía que cumplir, y consagreme a buscar
el cuerpo del pobre soldado para enviarle a Inglaterra, al panteón de
nuestra familia. En poco tiempo cumplí esta triste misión, y hallándome
sola traté de volver a mi país. Pero al mismo tiempo me cautivaban de
tal modo la historia, las tradiciones, las costumbres, la literatura,
las artes, las ruinas, la música popular, los bailes, los trajes de esta
nación tan grande en otro tiempo y otra vez grandísima en la época
presente, que formé el proyecto de quedarme aquí para estudiarlo todo, y
previa licencia de mis padres, así lo he hecho.

---Sabe Dios qué casta de pájaro serás tú---dije para mi capote; y luego
en voz alta añadí sosteniendo fijamente la dulce mirada de sus ojos de
cielo:---¡Y los padres de usted consintieron, sin reparar en los
continuos y graves peligros a que está expuesta una tierna doncella sola
y sin amparo en país extranjero, en medio de un ejército! Señora, por
amor de Dios\ldots---¡Ah! no conocéis sin duda que nosotras, las hijas
de Inglaterra estamos protegidas por las leyes de tal manera y con tanto
rigor, que ningún hombre se atreve a faltarnos al respeto.

---Sí, así dicen que pasa en Inglaterra. Y parece que allá salen las
señoritas solas a paseo y viajan solas o acompañadas de cualquier
galancete.

---Aunque fuera su novio, no importa,---dijo la inglesa.

---¡Pero estamos en España, señora, en España! Usted no sabe bien en qué
país se ha metido.

---Pero sigo al ejército aliado y estoy al amparo de las leyes
inglesas---dijo sonriendo.---Caballero, faltad al pudor si os parece,
intentad galantearme de una manera menos decorosa que la que empleáis
para amar a esa Dulcinea que fue causa de la muerte de Gray, y lord
Wellington os mandará fusilar, si no os casáis conmigo.

---Me casaría, señora.

---Caballero, veo que quizás sin malicia principiáis a faltar al
comedimiento.

---Pues no me casaría\ldots{} Permítame usted que me retire.

---Podéis hacerlo---me dijo levantándose penosamente para cerrar por
dentro la puerta.---Os agradeceré que mañana hagáis traer mi maleta.
Felizmente no la traía conmigo. Está en el convoy.

---Se traerá la maleta. Buenas noches, señora.

\hypertarget{ix}{%
\chapter{IX}\label{ix}}

Fuera de la estancia sentí el ruido de los cerrojos que corría por
dentro la hermosa inglesa y me retiré a mi aposento que era el rincón de
un oscuro pasillo, donde Tribaldos me había arreglado un lecho con
mantas y capotes. Tendime sobre aquellas durezas y en buena parte de la
noche no pude conciliar el sueño; de tal modo se había encajado dentro
de mi cerebro la extraña señora inglesa, con su caída, sus desmayos, su
té y su acabada hermosura. Pero al fin, rendido por el gran cansancio,
me dormí sosegadamente. Por la mañana, díjome la señora de Forfolleda
que la señorita rubia estaba mejor, que había pedido agua y té y pan,
ofreciendo dinero abundante por cualquier servicio que se le prestara.
Como manifestase deseos de entrar a saludarla, añadió la Forfolleda que
no era conveniente, por estar la señorita arreglándose y componiéndose,
a pesar de las heridas leves de su brazo.

Al salir a mis quehaceres, que fueron muchísimos y me ocuparon casi todo
el día, encontré a sir Tomás Parr, a quien encargué lo de la maleta.

Por la tarde, después del gran trabajo de aquel día que me hizo poner un
tanto en olvido a la interesante dama, regresé a casa de Forfolleda, y
vi a gran número de ingleses que entraban y salían, como diligentes
amigos que iban a informarse de la salud de su compatriota. Entré a
saludarla, y la pequeña estancia estaba llena de casacas rojas
pertenecientes a otros tantos hombres rubios que hablaban con animación.
La joven inglesa reía y bromeaba, y habíase puesto tan linda, sin
cambiar de traje, que no parecía la misma persona demacrada, melancólica
y nerviosa de la noche anterior. La contusión del brazo entorpecía algo
sus graciosos movimientos.

Después que nos saludamos y cambié con aquellos señores algunos fríos
cumplidos, uno de ellos invitó a la señorita a dar un paseo; otro
ponderó la hermosura de la apacible tarde, y no hubo quien no dijese una
palabra para decidirla a dejar la triste alcoba. Ella, sin embargo,
afirmó que no saldría hasta la siguiente mañana y con estos diálogos y
otros en que la graciosa joven no hacía maldito caso de su libertador,
vino la noche y con la noche luces dentro del cuarto y tras las luces un
par de teteras que trajeron los criados de los ingleses. Entonces se
alegraron todos los semblantes y empezó el trasiego con tanto ahínco que
el que menos se echó dentro un río de licor de la China, sin que ni un
momento cesase la charla. Trajeron después botellas de vino de Jerez,
que en un santiamén dejaron como cuerpos sin alma, porque toda ella pasó
a fortificar las de aquellos claros varones; mas ninguno perdió su
gravedad. Brindamos a la salud de Inglaterra, de España, y a eso de las
nueve nos retiramos todos, despidiéndonos la hermosa ninfa con
afabilidad, pero sin que ni con frase, ni gesto, ni mirada me
distinguiese de los demás.

Me retiraba a mi escondite cuando sentí que la desconocida echaba el
cerrojo. Aquella noche me mortificó como en la anterior un tenaz
desvelo; mas ya estaba a punto de vencerlo cuando hízome saltar en el
lecho el chirrido del cerrojo con que aseguraba su cuarto la consabida.
Miré hacia la puerta, pues desde mi alcoba-rincón se distinguía esta muy
bien, y vi a la inglesa que salía, encaminándose a una galería o solana
situada al otro confín del pasillo y de la casa. Como había dejado
abierta la puerta, la luz de su cuarto iluminaba la casa lo suficiente
para ver cuanto pasaba en ella.

Llegó la inglesa a la destartalada galería y abriendo una ventana que
daba al campo se asomó. Como estaba vestido, fácil me fue levantarme en
un momento y dirigirme hacia ella con paso quedo para no asustarla.
Cuando estuve cerca, volvió la cara y con gran sorpresa mía, no se
inmutó al verme. Antes bien con imperturbable tranquilidad, me dijo:

---¿Andáis rondando por aquí?\ldots{} Hace en aquel cuarto un calor
insoportable.

---Lo mismo sucede en el mío, señora---dije;---cuando la he visto a
usted pensaba salir al campo a respirar el aire fresco de la noche.

---Eso mismo pensaba yo también\ldots{} La noche está hermosa\ldots{} ¿y
pensabais salir?\ldots{}

---Sí señora, pero si usted lo permite tendré el honor de acompañarla y
juntos disfrutaremos de este suave ambiente, del grato aroma de esos
pinares\ldots{}

---No\ldots{} salid, bajad, iré yo también,---dijo con viva resolución y
mucha naturalidad.

Entrando rápidamente en su cuarto de donde sacara una capa de forma
extraña y echándosela sobre los hombros, me suplicó que cuidadosamente
la embozara por no tener ella aún agilidad en su brazo herido; y una vez
que la envolví bien, salimos ambos, sin tomar ella mi brazo, y como dos
amigos que van a paseo. Por todas partes se oía rumor de soldados, y la
claridad de la luna permitía ver todos los objetos y conocer a las
personas.

Súbitamente y sin contestar a no sé qué vulgar frase pronunciada por mí,
la inglesa me dijo:

---Ya sé que sois noble, caballero. ¿A qué familia pertenecéis? ¿A los
Pachecos, a los Vargas, a los Enríquez, a los Acuñas, a los Toledos o a
los Dávilas?

---A ninguna de esas, señora---le respondí ocultando con mi embozo la
sonrisa que no pude contener---sino a los Aracelis de Andalucía, que
descienden, como usted no ignora, del mismo Hércules.

---¿De Hércules? No lo sabía ciertamente---repuso con
naturalidad.---¿Hace mucho que estáis en campaña?

---Desde que empezó, señora.

---Sois valiente y generoso, sin duda---dijo mirándome fijamente al
rostro.---Bien se conoce en vuestro semblante que lleváis en las venas
la sangre de aquellos insignes caballeros que han sido asombro y envidia
de Europa por espacio de muchos siglos.

---Señora, usted me favorece demasiado.

---Decidme: ¿sabéis tirar las armas, domar un potro, derribar un toro,
tañer la guitarra y componer versos?

---No puedo negar que un poco entendido soy en alguna, sino en todas
esas habilidades.

Después de pequeña pausa y deteniendo el paso, me preguntó bruscamente:

---¿Y estáis enamorado?

Durante un rato no supe qué responder; tan extrañas me parecían aquellas
palabras.

---¿Cómo no, siendo español, siendo joven y militar?---contesté decidido
a llevar la conversación a donde la fantasía de mi incógnita amiga
quisiera llevarla.

---Veo que os sorprende mi modo de hablaros---añadió
ella.---Acostumbrado a no oír en boca de vuestras mojigatas compatriotas
sino medias palabras, vulgaridades, y frases de hipocresía, os sorprende
esta libertad con que me expreso, estas extrañas preguntas que os
dirijo\ldots{} Quizás me juzguéis mal\ldots{}

---Oh, no señora.

---Pero mi honor no depende de vuestros pensamientos. Seríais un necio
si creyerais que esto es otra cosa que una curiosidad de inglesa, casi
diré de artista y de viajera. Las costumbres y los caracteres de este
país son dignos de profundo estudio.

---De modo que lo que quiere es estudiarme---dije entre
dientes.---Resignémonos a ser libro de texto.

---El hombre que ha dado muerte a lord Gray, que ha realizado esa gran
obra de justicia, que ha sido brazo de Dios y vengador de la moral
ultrajada, excita mi curiosidad de un modo pasmoso\ldots{} Me han
hablado de vos con admiración y contádome algunos hechos vuestros dignos
de gran estima\ldots{} Dispensad mi curiosidad, que escandalizaría a una
española y que sin duda os escandaliza a vos\ldots{} Habiendo matado a
Gray por celos, claro que estabais enamorado. Y vuestra dama (esto de
\emph{vuestra dama} me hizo reír de nuevo), ¿habita en algún castillo de
estas cercanías o en algún palacio de Andalucía? ¿Es noble como
vos?\ldots{}

Al oír esto comprendí que tenía que habérmelas con una imaginación
exaltada y novelesca, y al punto apoderose de mí cierto espíritu de
socarronería. No me inclinaba a burlarme de la inglesa, que a pesar de
su sentimentalismo fuera de ocasión no era ridícula; pero mi carácter me
inducía a seguir la broma, como si dijéramos, prestándome a los
caprichos de aquella idealidad tan falsa como encantadora. Todos somos
algo poetas, y es muy dulce embellecer la propia vida, y muy natural
regocijarnos con este embellecimiento aun sabiendo que la transformación
es obra nuestra. Así es que con cierta exaltación novelesca también, mas
no con completa seriedad, contesté a la damisela:

---Noble es, señora, y hermosísima y principal; pero ¿de qué me vale
tener en ella un dechado de perfecciones, si un funesto destino la aleja
constantemente de mí? ¿Qué pensará usted, señora, si le digo que hace
tiempo cierto maligno encantador la tiene transfigurada en la persona de
una vulgar comiquilla que recorre los pueblos formando parte de una
compañía de histriones de la legua?

Esto era, sin duda, demasiado fuerte.

---Caballero---dijo la inglesa con estupor;---¿pues qué, todavía hay
encantamientos en España?

---Encantamientos, precisamente no---dije tratando de abatir el
vuelo;---pero hay artes del demonio, y si no artes del demonio, malicias
y ardides de hombres perversos.

---Veo que leéis libros de caballería.

---Pues ¿quién duda que son los más hermosos entre todos los que se han
escrito? Ellos suspenden el ánimo, despiertan la sensibilidad, avivan el
valor, infunden entusiasmo por las grandes acciones, engrandecen la
gloria y achican el peligro en todos los momentos de la vida.

---¡Engrandecen la gloria y achican el peligro!---exclamó
deteniéndose.---Si esto que habéis dicho es verdad, sois digno de haber
nacido en otros tiempos\ldots{} pero no he entendido bien eso de que
vuestra dama está transformada en una comiquilla\ldots{}

---Así es, señora. Si pudiera contar a usted todo lo que ha precedido a
esta transformación, no dudo que usted me compadecería.

---¿Y dónde están la encantada y el encantador? Les doy estos nombres
porque veo que creéis en encantamentos.

---Están en Salamanca.

---Como si estuvieran en el otro mundo. Salamanca está en poder de los
franceses.

---Pero la tomaremos.

---Decís eso como si fuera lo más natural del mundo.

---Y lo es. No se ría usted de mi petulancia; pero si todo el ejército
aliado desapareciera y me quedase solo\ldots{}

---Iríais solo a la conquista de la ciudad, queréis decir.

---¡Ah, señora!---exclamé con énfasis.---Un hombre que ama no sabe lo
que dice. Veo que es un desatino.

---Un desatino relativo---repuso.---Pero ahora comprendo que os estáis
burlando de mí. Os habéis enamorado de una cómica y queréis hacerla
pasar por gran señora.

---Cuando entremos en Salamanca podré convencer a usted de que no me
burlo.

---No dudo que haya cómicos en el país, ni menos cómicas guapas---dijo
riendo .---Hace dos días pasó por delante de mí una compañía que me
recordó el carro de las Cortes de la Muerte. Iban allí siete u ocho
histriones, y, en efecto dijeron que iban a Salamanca.

---Llevaban dos o tres carros. En uno de ellos iban dos mujeres, una de
ellas hermosísima. Venían de Plasencia.

---Me parece que sí.

---Y en otro carro llevaban lienzos pintados.

---Los habéis visto; pero no sabéis lo que yo sé. Cuando pasaron por
delante de mí, sorprendiéndome por su extraño aspecto que me recordaba
una de las más graciosas aventuras del \emph{Libro}, un vecino de Puerto
de Baños me dijo: «Esos no son cómicos sino pícaros masones que se
disfrazan así para pasar por entre los españoles, que les
descuartizarían si les conocieran.»

---No me dice usted nada que yo no sepa---contesté.---Señora, ¿ha oído
usted decir a lord Wellington cuándo lanzará nuestros regimientos sobre
Salamanca?

---Impaciente estáis\ldots{} Quiero saber otra cosa. ¿Amáis a vuestra
Dulcinea de una manera ideal y sublime, embelleciéndola con vuestro
pensamiento aún más de lo que ella es en sí, atribuyéndole cuantas
perfecciones pueden idearse y consagrándole todos los dulces transportes
de un corazón siempre inflamado?

---Así, así mismo, señora---dije con entusiasmo que no era enteramente
falso, y deseando ver a dónde iba a parar aquella misteriosa mujer, cuyo
carácter comenzaba a penetrar.---Parece que lee usted en mi alma como en
un libro.

Después que oyó esto, permaneció largo rato en silencio, y luego reanudó
el diálogo con una brusca variación de ideas, que era la tercera en
aquel extraño coloquio.

---Caballero, ¿tenéis madre?---me dijo.

---No señora.

---¿Ni hermanas?

---Tampoco. Ni madre, ni padre, ni hermanos, ni pariente alguno.

---Veo que está muy malparado el linaje de Hércules. De modo que estáis
solo en el mundo---añadió con acento compasivo.---¡Desgraciado
caballero! ¿Y esa gran señora, cómica, o mujer masónica, os ama?

---Creo que sí.

---¿Habéis hecho por ella sacrificios, arrostrado peligros y vencido
obstáculos?

---Muchísimos; pero son nada en comparación con lo que aún me resta por
hacer.

---¿Qué?

---Una acción peligrosa, una locura; el último grado del atrevimiento.
Espero morir o lograr mi objeto.

---¿Tenéis miedo a los peligros que os aguardan?

---Jamás lo he conocido---respondí con una fatuidad, cuyo recuerdo me ha
hecho reír muchas veces.

---Estad tranquilo, pues los aliados entrarán en Salamanca, y entonces
fácilmente\ldots{}

---Cuando entren los aliados, mi enemigo y su víctima habrán huido
corriendo hacia Francia. Él no es tonto\ldots{} Es preciso ir a
Salamanca antes\ldots{}

---¡Antes de tomarla!---exclamó con asombro.

---¿Por qué no?

---Caballero---dijo súbitamente deteniendo el paso.---Veo que os estáis
burlando de mí.

---¡Yo, señora!---contesté algo turbado.

---Sí: me ponéis ante los ojos una aventura caballeresca, que es pura
invención y fábula; os pintáis a vos mismo como un carácter superior,
como un alma de esas que se engrandecen con los peligros, y habéis
adornado la ficción con hermosas figuras de Dulcinea y encantadores, que
no existen sino en vuestra imaginación.

---Señora mía, usted\ldots{}

---Tened la bondad de acompañarme a mi alojamiento. El olor de esos
pinares me marea.

---Como usted guste.

Confieso ¿por qué no confesarlo? que me quedé algo corrido.

La elegante inglesa no me dijo una palabra más en todo el camino, y
cuando subimos a casa de Forfolleda y la conduje a su cuarto, que ya
empezaba a figurárseme regio camarín tapizado de rasos y organdíes,
metiose en su tugurio como un hada en su cueva, y dándome desabridamente
las buenas noches, corrió los cerrojos de oro\ldots{} o de hierro, y me
quedé solo.

\hypertarget{x}{%
\chapter{X}\label{x}}

Acomodándome en mi lecho, hablé conmigo de esta manera:

---¿La tal inglesa será una de esas mujeres de equívoca honradez que
suelen seguir a los ejércitos? Las hay de diferentes especies; pero en
realidad, jamás vi en pos de los soldados de la patria ninguna tan
hermosa ni de porte tan noble y aristocrático. He oído que tras el
ejército francés van pájaros de diverso plumaje. ¡Bah!\ldots{} ¿pues no
dicen que Massena ha tenido tan mala suerte en Portugal por la
corrupción de sus oficiales y soldados, y aun por sus propios descuidos
con ciertas amazonas muy emperifolladas que andaban en los campamentos
tan a sus anchas como en París?\ldots{}

Después dando otra dirección a mis ideas, dije a punto que empezaba a
embargarme el dulce entorpecimiento que precede al sueño:

---Tal vez me equivoque. Después de haber conocido a lord Gray, no debo
poner en duda que las extravagancias y rarezas de la gente inglesa
carecen de límite conocido. Tal vez mi compañera de alojamiento sea tan
cabal que la misma virginidad parezca a su lado una moza de partido, y
yo estoy injuriándola. Mañana preguntaré a los oficiales ingleses que
conozco\ldots{} Como no sea una de esas naturalezas impresionables y
acaloradas que nacen al acaso en el Norte, y que buscan como las
golondrinas los climas templados, bajan llenas de ansiedad al Mediodía,
pidiendo luz, sol, pasiones, poesía, alimento del corazón y de la
fantasía, que no siempre encuentran o encuentran a medias; y van con
febril deseo tras de la originalidad, tras las costumbres raras y adoran
los caracteres apasionados aunque sean casi salvajes, la vida
aventurera, la galantería caballeresca, las ruinas, las leyendas, la
música popular y hasta las groserías de la plebe siempre que sean
graciosas.

Diciendo o pensando así y enlazando con éstos otros pensamientos que más
hondamente me preocupaban, caí en profundísimo sueño reparador.
Levanteme muy temprano a la mañana siguiente, y sin acordarme para nada
de la hermosa inglesa, cual si la noche limpiara todas las telas de
araña fabricadas y tendidas el día anterior dentro de mi cerebro, salí
de mi alojamiento.

---Marchamos hacia San Muñoz---me dijo Figueroa, oficial portugués amigo
mío que servía con el general Picton.

---¿Y el \emph{Lord}?

---Va a partir no sé a dónde. La división de Graham está sobre Tamames.
Nosotros vamos a formar el ala izquierda de la división de D. Carlos
España y la partida de D. Julián Sánchez.

Cuando nos dirigíamos juntos al alojamiento del general, pedile informes
de la dama inglesa cuya figura y extraños modos he dado a conocer, y me
contestó:

---Es miss Fly, o lo que es lo mismo, miss Mosquita, Mariposa, Pajarita
o cosa así. Su nombre es Athenais. Tiene por padre a lord Fly, uno de
los señores más principales de la Gran Bretaña. Nos ha seguido desde la
Albuera, pintando iglesias, castillos y ruinas en cierto librote que
trae consigo, y escribiendo todo lo que pasa. El \emph{Lord} y los demás
generales ingleses la consideran mucho, y si quieres saber lo que es
bueno, atrévete a faltar al respeto a la señorita Fly, que en inglés se
dice \emph{Flai}, pues ya sabes que en esa lengua se escriben las
palabras de una manera y se pronuncian de otra, lo cual es un encanto
para el que quiere aprenderla.

Acto continuo referí a mi amigo las escenas de la noche anterior y el
paseo que en la soledad de la noche dimos miss Fly y yo por aquellos
contornos, lo que oído por Figueroa, causó a este muchísima sorpresa.

---Es la primera vez---dijo---que la rubita tiene tales familiaridades
con un oficial español o portugués, pues hasta ahora a todos les miró
con altanería\ldots{}

---Yo la tuve por persona de costumbres un tanto libres.

---Así parece, porque anda sola, monta a caballo, entra y sale por medio
del ejército, habla con todos, visita las posiciones de vanguardia antes
de una batalla y los hospitales de sangre después\ldots{} A veces se
aleja del ejército para recorrer sola los pueblos inmediatos, mayormente
si hay en estos abadías, catedrales o castillos, y en sus ratos de ocio
no hace más que leer romances.

Hablando de este y de otros asuntos, empleamos la mañana, y cerca del
medio día fuimos al alojamiento de Carlos España, el cual no estaba
allí.

---España---nos dijo el guerrillero Sánchez---está en el alojamiento del
cuartel general.

---¿No marcha lord Wellington?

---Parece que se queda aquí, y nosotros salimos para San Muñoz dentro de
una hora.

---Vamos al alojamiento del duque---dijo Figueroa;---allí sabremos
noticias ciertas.

Estaba lord Wellington en la casa-ayuntamiento, la única capaz y
decorosa para tan insigne persona. Llenaban la plazoleta, el soportal,
el vestíbulo y la escalera multitud de oficiales de todas las
graduaciones, españoles, ingleses y lusitanos que entraban, y salían,
formaban corrillos, disputando y bromeando unos con otros en amistosa
intimidad cual si todos perteneciesen a una misma familia. Subimos
Figueroa y yo, y después de aguardar más de una hora y media en la
antesala, salió España y nos dijo:

---El general en jefe pregunta si hay un oficial español que se atreva a
entrar disfrazado en Salamanca para examinar los fuertes y las obras
provisionales que ha hecho el enemigo en la muralla, ver la artillería y
enterarse de si es grande o pequeña la guarnición, y abundantes o
escasas las provisiones.

---Yo voy---repuse resueltamente sin aguardar a que el general
concluyese.

---Tú---dijo España con la desdeñosa familiaridad que usaba hablando con
sus oficiales,---¿tú te atreves a emprender viaje tan arriesgado? Ten
presente que es preciso ir y volver.

---Lo supongo.

---Es necesario atravesar las líneas enemigas, pues los franceses ocupan
todas las aldeas del lado acá del Tormes.

---Se entra por donde se puede, mi general.

---Luego has de atravesar la muralla, los fuertes, has de penetrar en la
ciudad, visitar los acantonamientos, sacar planos\ldots{}

---Todo eso es para mí un juego, mi general. Entrar, salir, ver\ldots{}
una diversión. Hágame vuecencia la merced de presentarme al señor duque,
diciéndole que estoy a sus órdenes para lo que desea.

---Tú eres un atolondrado y no sirves para el caso---repuso D.
Carlos.---Buscaremos otro. No sabes una palabra de geometría ni de
fortificación.

---Eso lo veremos---contesté sofocado.

---Y es preciso, es preciso ir---añadió mi jefe.---Aún no ha formado el
lord su plan de batalla. No sabe si asaltará a Salamanca o la bloqueará;
no sabe si pasará el Tormes para perseguir a Marmont, dejando atrás a
Salamanca o si\ldots{} ¿Dices que te atreves tú?\ldots{}

---¿Pues no he de atreverme? Me vestiré de charro, entraré en Salamanca
vendiendo hortalizas o carbón. Veré los fuertes, la guarnición, las
vituallas; sacaré un croquis y me volveré al campamento\ldots{} Mi
general---añadí con calor,---o me presenta vuecencia al duque, o me
presento yo solo.

---Vamos, vamos al momento---dijo España entrando conmigo en la sala.

\hypertarget{xi}{%
\chapter{XI}\label{xi}}

Junto a una gran mesa colocada en el centro estaba el duque de
Ciudad-Rodrigo con otros tres generales examinando una carta del país, y
tan profundamente atendían a las rayas, puntos y letras con que el
geógrafo designara los accidentes del terreno, que no alzaron la cabeza
para mirarnos. Hízome seña don Carlos España de que debíamos esperar, y
en tanto dirigí la vista a distintos puntos de la sala para examinar,
siguiendo mi costumbre, el sitio en que me encontraba. Otros oficiales
hablaban en voz baja retirados del centro, y entre ellos ¡oh sorpresa!
vi a miss Fly, que sostenía conversación animada con un coronel de
artillería llamado Simpson.

Por fin, lord Wellington levantó los ojos del mapa y nos miró. Hice una
amabilísima reverencia: entonces el inglés me miró más, observándome de
pies a cabeza. También yo le observé a él a mis anchas, gozoso de tener
ante mi vista a una persona tan amada entonces por todos los españoles,
y que tanta admiración me inspiraba a mí. Era Wellesley bastante alto,
de cabellos rubios y rostro encendido, aunque no por las causas a que el
vulgo atribuye las inflamaciones epidérmicas de la gente inglesa. Ya se
sabe que es proverbial en Inglaterra la afirmación de que el único
grande hombre que no ha perdido jamás su dignidad después de los
postres, es el vencedor de Tipoo Sayd y de Bonaparte.

Representaba Wellington cuarenta y cinco años, y esta era su edad, la
misma exactamente que Napoleón, pues ambos nacieron en 1769, el uno en
Mayo y el otro en Agosto. El sol de la India y el de España habían
alterado la blancura de su color sajón. Era la nariz, como antes he
dicho, larga y un poco bermellonada; la frente, resguardada de los rayos
del sol por el sombrero, conservaba su blancura y era hermosa y serena
como la de una estatua griega, revelando un pensamiento sin agitación y
sin fiebre, una imaginación encadenada y gran facultad de ponderación y
cálculo. Adornaba su cabeza un mechón de pelo o tupé que no usaban
ciertamente las estatuas griegas; pero que no caía mal, sirviendo de
vértice a una mollera inglesa. Los grandes ojos azules del general
miraban con frialdad, posándose vagamente sobre el objeto observado, y
observaban sin aparente interés. Era la voz sonora, acompasada, medida,
sin cambiar de tono, sin exacerbaciones ni acentos duros, y el conjunto
de su modo de expresarse, reunidos el gesto, la voz y los ojos, producía
grata impresión de respeto y cariño.

Su excelencia me miró, como he dicho, y entonces D. Carlos España, dijo:

---Mi general, este joven desea desempeñar la comisión de que vuecencia
me ha hablado hace poco. Yo respondo de su valor y de su lealtad; pero
he intentado disuadirle de su empeño, porque no posee conocimientos
facultativos.

Aquello me avergonzó, mayormente porque estaba delante de miss Fly, y
porque, en efecto, yo no había estado en ninguna academia.

---Para esta comisión---dijo Wellington en castellano bastante
correcto,---se necesitan ciertos conocimientos\ldots{}

Y fijó los ojos en el mapa. Yo miré a España y España me miró a mí. Pero
la vergüenza no me impidió tomar una resolución, y sin encomendarme a
Dios ni al diablo, dije:

---Mi general. Es cierto que no he estado en ninguna academia; pero una
larga práctica de la guerra en batallas y sobre todo en sitios, me ha
dado tal vez los conocimientos que vuecencia exige para esta comisión.
Sé levantar un plano.

El duque de Ciudad-Rodrigo alzando de nuevo los ojos, habló así:

---En mi cuartel general hay multitud de oficiales facultativos; pero
ningún inglés podría entrar en Salamanca, porque sería al instante
descubierto por su rostro y por su lenguaje. Es preciso que vaya un
español.

---Mi general---dijo con fatuidad España,---en mi división no faltan
oficiales facultativos. He traído a este porque se empeñó en hacer
alarde de su arrojo delante de vuecencia.

Miré con indignación a D. Carlos, y luego exclamé con la mayor
vehemencia:

---Mi general, aunque en esta empresa existan todos los peligros, todas
las dificultades imaginables, yo entraré en Salamanca y volveré con las
noticias que vuecencia desea.

Tranquila y sosegadamente lord Wellington me preguntó:

---Señor oficial, ¿dónde empezó usted su vida militar?

---En Trafalgar---contesté.

Cuando esta histórica y grandiosa palabra resonó en la sala en medio del
general silencio, todas las cabezas de las personas allí presentes se
movieron como si perteneciesen a un solo cuerpo, y todos los ojos
fijáronse en mí con vivísimo interés.

---¿Entonces ha sido usted marino?---interrogó el duque.

---Asistí al combate teniendo catorce años de edad. Yo era amigo de un
oficial que iba en el \emph{Trinidad}. La pérdida de la tripulación me
obligó a tomar parte en la batalla.

---¿Y cuándo empezó usted a servir en la campaña contra los franceses?

---El 2 de Mayo de 1808, mi general. Los franceses me fusilaron en la
Moncloa. Salveme milagrosamente; pero en mi cuerpo han quedado escritos
los horrores de aquel día.

---¿Y desde entonces se alistó usted?

---Alisteme en los regimientos de voluntarios de Andalucía, y estuve en
la batalla de Bailén.

---¡También en la batalla de Bailén!---dijo Wellington con asombro.

---Sí, mi general, el 19 de Julio de 1808. ¿Quiere vuecencia ver mi hoja
de servicios, que comienza en dicha fecha?

---No, me basta---repuso Wellington.---¿Y después?

---Volví a Madrid, y tomé parte en la jornada del 3 de Diciembre. Caí
prisionero y me quisieron llevar a Francia.

---¿Le llevaron a usted a Francia?

---No, mi general, porque me escapé en Lerma, y fui a parar a Zaragoza
en tan buena ocasión, que alcancé el segundo sitio de aquella inmortal
ciudad.

---¿Todo el sitio?---dijo Wellington con creciente interés hacia mi
persona.

---Todo, desde el 19 de Diciembre hasta el 12 de Febrero de 1809. Puedo
dar a vuecencia noticia circunstanciada de todas las peripecias de aquel
grande hecho de armas, gloria y orgullo de cuantos nos encontramos en
él.

---¿Y a qué ejército pasó usted luego?

---Al del Centro, y serví bastante tiempo a las órdenes del duque del
Parque. Estuve en la batalla de Tamames y en Extremadura.

---¿No se encontró usted en un nuevo asedio?

---En el de Cádiz, mi general. Defendí durante tres días el castillo de
San Lorenzo de Puntales.

---¿Y luego formó usted parte de la expedición del general Blake a
Valencia?

---Sí, mi general; pero me destinaron al segundo cuerpo que mandaba
O'Donnell, y durante cuatro meses serví a las órdenes del Empecinado en
esa singular guerra de partidas en que tanto se aprende.

---¿También ha sido usted guerrillero?---dijo Wellington
sonriendo.---Veo que ha ganado usted bien sus grados. Irá usted a
Salamanca, si así lo desea.

---Señor, lo deseo ardientemente.

Todos los presentes seguían observándome, y miss Fly con más atención
que ninguno.

---Bien---añadió el héroe de Talavera, fijando alternativamente la vista
en mí y en el mapa.---Tiene usted que hacer lo siguiente: Se dirigirá
usted hoy mismo disfrazado a Salamanca, dando un rodeo para entrar por
Cabrerizos. Forzosamente ha de pasar usted por entre las tropas de
Marmont que vigilan los caminos de Ledesma y Toro. Hay muchas
probabilidades de que sea usted arcabuceado por espía; pero Dios protege
a los valientes, y quizás\ldots{} quizás logre usted penetrar en la
plaza. Una vez dentro sacará usted un croquis de las fortificaciones,
examinando con la mayor atención los conventos que han sido convertidos
en fuertes, los edificios que han sido demolidos, la artillería que
defiende los aproches de la ciudad, el estado de la muralla, las obras
de tierra y fajina, todo absolutamente, sin olvidar las provisiones que
tiene el enemigo en los almacenes.

---Mi general---repuse---comprendo bien lo que se desea, y espero
contentar a vuecencia. ¿Cuándo debo partir?

---Ahora mismo. Estamos a doce leguas de Salamanca. Con la marcha que
emprenderemos hoy, espero que pernoctemos en Castroverde, cerca ya de
Valmuza. Pero adelántese usted a caballo y pasado mañana martes podrá
entrar en la ciudad. En todo el martes ha de desempeñar por completo
esta comisión, saliendo el miércoles por la mañana para venir al cuartel
general, que en dicho día estará seguramente en Bernuy. En Bernuy, pues,
le aguardo a usted el miércoles a las doce en punto de la mañana. No
acostumbro esperar.

---Corriente mi general. El miércoles a las doce estaré en Bernuy de
vuelta de mi expedición.

---Tome usted precauciones. Diríjase usted a la calzada de Ledesma, pero
cuidando de marchar siempre fuera del arrecife. Disfrácese usted bien,
pues los franceses dejan entrar a los aldeanos que llevan víveres a la
plaza; y al levantar el croquis evite en lo posible las miradas de la
gente. Lleve usted armas, ocultándolas bien: no provoque a los enemigos;
fínjase amigo de ellos, en una palabra, ponga usted en juego su ingenio,
su valor, y todo el conocimiento de los hombres y de la guerra que ha
adquirido en tantos años de activa vida militar. El \emph{Mayor} general
del ejército entregará a usted la suma que necesite para la expedición.

---Mi general---dije---¿tiene vuecencia algo más que mandarme?

---Nada más---repuso sonriendo con benevolencia---sino que adoro la
puntualidad y considero como origen del éxito en la guerra la exacta
apreciación y distribución del tiempo.

---Eso quiere decir que si no estoy de vuelta el miércoles a las doce,
desagradaré a vuecencia.

---Y mucho. En el tiempo marcado puede hacerse lo que encargo. Dos horas
para sacar el croquis, dos para visitar los fuertes, ofreciendo en venta
a los soldados algún artículo que necesiten, cuatro para recorrer toda
la población y sacar nota de los edificios demolidos, dos para vencer
obstáculos imprevistos, media para descansar. Son diez horas y media del
martes por el día. La primera mitad de la noche para estudiar el
espíritu de la ciudad, lo que piensan de esta campaña la guarnición y el
vecindario, una hora para dormir y lo restante para salir y ponerse
fuera del alcance y de la vista del enemigo. No deteniéndose en ninguna
parte puede usted presentárseme en Bernuy a la hora convenida.

---A la orden de mi general---dije disponiéndome a salir.

Lord Wellington, el hombre más grande de la Gran Bretaña, el rival de
Bonaparte, la esperanza de Europa, el vencedor de Talavera, de la
Albuera, de Arroyo Molinos y de Ciudad-Rodrigo, levantose de su asiento,
y con una grave cortesanía y cordialidad, que inundó mi alma de orgullo
y alegría, diome la mano, que estreché con gratitud entre las mías.

Salí a disponer mi viaje.

\hypertarget{xii}{%
\chapter{XII}\label{xii}}

Hallábame una hora después en una casa de labradores ajustando el precio
del vestido que había de ponerme, cuando sentí en el hombro un golpecito
producido al parecer por un látigo que movían manos delicadas. Volvime y
miss Fly, pues no era otra la que me azotaba, dijo:

---Caballero, hace una hora que os busco.

---Señora, los preparativos de mi viaje me han impedido ir a ponerme a
las órdenes de usted.

Miss Fly no oyó mis últimas palabras, porque toda su atención estaba
fija en una aldeana que teníamos delante, la cual, por su parte,
amamantando un tierno chiquillo, no quitaba los ojos de la inglesa.

---Señora---dijo esta---¿me podréis proporcionar un vestido como el que
tenéis puesto?

La aldeana no entendía el castellano corrompido de la inglesa, y
mirábala absorta sin contestarle.

---Señorita Fly---dije---¿va usted a vestirse de aldeana?

---Sí---me respondió sonriendo con malicia.---Quiero ir con vos.

---¡Conmigo!---exclamé con la mayor sorpresa.

---Con vos, sí; quiero ir disfrazada con vos a Salamanca---añadió
tranquilamente, sacando de su bolsillo algunas monedas para que la
aldeana la entendiese mejor.

---Señora, no puedo creer sino que usted se ha vuelto loca---dije.---¿Ir
conmigo a Salamanca, ir conmigo en esta expedición arriesgada y de la
cual ignoro si saldré con vida?

---¿Y qué? ¿No puedo ir porque hay peligro? Caballero, ¿en qué os
fundáis para creer que yo conozco el miedo?

---Es imposible, señora, es imposible que usted me acompañe---afirmé con
resolución.

---Ciertamente no os creía grosero. Sois de los que rechazan todo
aquello que sale de los límites ordinarios de la vida. ¿No comprendéis
que una mujer tenga arrojo suficiente para afrontar el peligro, para
prestar servicios difíciles a una causa santa?

---Al contrario, señora, comprendo que una mujer como usted es capaz de
eminentes acciones, y en este momento miss Fly me inspira la más sincera
admiración; pero la comisión que llevo a Salamanca es muy delicada,
exige que nadie vaya al lado mío, y menos una señora que no puede
disfrazarse, ocultando su lengua extranjera y noble porte.

---¿Que no puedo disfrazarme?

---Bueno, señora---dije sin poder contener la risa.---Principie usted
por dejar su guardapiés de amazona, y póngase el manteo, es decir, una
larga pieza de tela que se arrolla en el cuerpo, como la faja que ponen
a los niños.

Miss Fly miraba con estupor el extraño y pintoresco vestido de la
aldeana.

---Luego---añadí---desciña usted esas hermosas trenzas de oro,
construyéndose en lo alto un moño del cual penderán cintas, y en las
sienes dos rizos de rueda de carro con horquillas de plata. Cíñase usted
después la jubona de terciopelo, y cubra en seguida sus hermosos hombros
con la prenda más graciosa y difícil de llevar, cual es el dengue o
rebociño.

Athenais se ponía de mal humor, y contemplaba las singulares prendas que
la charra iba sacando de un arcón.

---Y después de calzarse los zapatitos sobre media de seda calada, y
ceñirse el picote negro bordado de lentejuelas, ponga usted la última
piedra a tan bello edificio, con la mantilla de rocador prendida en los
hombros.

La señorita Mariposa me miró con indignación comprendiendo la
imposibilidad de disfrazarse de aldeana.

---Bien---afirmó mirándome con desdén.---Iré sin disfrazarme. En
realidad no lo necesito, porque conozco al coronel Desmarets, que me
dejará entrar. Le salvé la vida en la Albuera\ldots{} Y no creáis, mi
conocimiento con el coronel Desmarets puede seros útil\ldots{}

---Señora---le dije, poniéndome serio,---el honor que recibo y el placer
que experimento al verme acompañado por usted son tan grandes, que no sé
cómo expresarlos. Pero no voy a una fiesta, señora, voy al peligro.
Además, si este no asusta a una persona como usted ¿nada significa el
menoscabo que pueda recibir la opinión de una dama ilustre que viaja con
hombre desconocido por vericuetos y andurriales?

---Menguada idea tenéis del honor, caballero---declaró con nobleza y
altanería.---O vuestros hechos son mentira, o vuestros pensamientos
están muy por bajo de ellos. Por Dios, no os arrastréis al nivel de la
muchedumbre, porque conseguiréis que os aborrezca. Iré con vos a
Salamanca.

Y tomando el partido de no contestar a mis razonables observaciones, se
dirigió al cuartel general, mientras yo tomaba el camino de mi
alojamiento para trocarme de oficial del ejército en el más rústico
charro que ha parecido en campos salmantinos. Con mi calzón estrecho de
paño pardo, mis medias negras y zapatos de vaca; con mi chaleco
cuadrado, mi jubón de aldetas en la cintura y cuchillada en la sangría,
y el sombrero de alas anchas y cintas colgantes que encajé en mi cabeza,
estaba que ni pintado. Completaron mi equipo por el momento una cartera
que cosí dentro del jubón con lo necesario para trazar algunas líneas, y
el alma de la expedición, o sea el dinero que puse en la bolsa interna
del cinto.

\hypertarget{xiii}{%
\chapter{XIII}\label{xiii}}

---Ya está mi Sr.~Araceli en campaña---me dije.---El miércoles a las
doce de vuelta en Bernuy\ldots{} ¡En buena me he metido!\ldots{} Si la
inglesa da en el hito de acompañarme, soy hombre perdido\ldots{} Pero me
opondré con toda energía, y como no entre en razón, denunciaré al
general en jefe el capricho de su audaz paisana para que acorte los
vuelos de esta sílfide andariega y voluntariosa.

No era tanta mi inmodestia que supusiese a Athenais movida
exclusivamente de un antojo y afición a mi persona; pero aún creyéndome
indigno de la solícita persecución de la hermosa dama, resolví poner en
práctica un medio eficaz para librarme de aquel enojoso, aunque adorable
y tentador estorbo, y fue que bonitamente y sin decir nada a nadie, como
D. Quijote en su primera salida, eché a correr fuera de Santi Spíritus y
delante de la vanguardia del ejército, que en aquel momento comenzaba a
salir para San Muñoz.

Pero juzgad, ¡oh señores míos!, ¡cuál sería mi sorpresa cuando a poco de
haber salido espoleando mi cabalgadura, que en el andar allá se iba con
Rocinante, sentí detrás un chirrido de ásperas ruedas y un galope de
rocín y un crujir de látigo y unas voces extrañas de las que en todos
los idiomas se emplean para animar a un bruto perezoso! ¡Juzgad de mi
sorpresa cuando me volví y vi a la misma miss Fly dentro de un
cochecillo indescriptible, no menos destartalado y viejo que aquel de la
célebre catástrofe, guiando ella misma y acompañada de un rapazuelo de
Santi Spíritus!

Al llegar junto a mí, la inglesa profería exclamaciones de triunfo. Su
rostro estaba enardecido y risueño, como el de quien ha ganado un premio
en la carrera, sus ojos despedían la viva luz de un gozo sin límites;
algunas mechas de sus cabellos de oro flotaban al viento, dándole el
fantástico aspecto de no sé qué deidad voladora de esas que corren por
los frisos de la arquitectura clásica, y su mano agitaba el látigo con
tanta gallardía como un centauro su dardo mortífero. Si me fuera lícito
emplear las palabras que no entiendo bien aplicadas a la figura humana,
pero que son de uso común en las descripciones, diría que estaba
\emph{radiante}.

---Os he alcanzado---dijo con acento verdaderamente triunfal.---Si
\emph{Mistress} Mitchell no me hubiera prestado su carricoche, habría
venido sobre una cureña, señor Araceli.

Y como nuevamente le expusiera yo los inconvenientes de su
determinación, me dijo:

---¡Qué placer tan grande experimento! Esta es la vida para mí;
libertad, independencia, iniciativa, arrojo. Iremos a Salamanca\ldots{}
Sospecho que allí tendréis que hacer además de la comisión de lord
Wellington\ldots{} Pero no me importan vuestros asuntos. Caballero,
sabed que os desprecio.

---¿Y qué he hecho para merecerlo?---dije poniendo mi cabalgadura al
paso del caballo de tiro y aflojando la marcha, lo cual ambas bestias
agradecieron mucho.

---¿Qué? Llamar locura a este designio mío. No tienen otra palabra para
expresar nuestra inclinación a las impresiones desconocidas, a los
grandes objetos que entrevé el alma sin poderlos precisar, a las
caprichosas formas con que nos seduce el acaso, a las dulces emociones
producidas por el peligro previsto y el éxito deseado.

---Comprendo toda la grandeza del varonil espíritu de usted; pero ¿qué
puede encontrar en Salamanca digno del empleo de tan insignes
facultades? Voy como espía, y el espionaje no tiene nada de sublime.

---¿Querréis hacerme creer---dijo con malicia,---que vais a Salamanca a
la comisión de lord Wellington?

---Seguramente.

---Un servicio a la patria no se solicita con tanto afán. Recordad lo
que me dijisteis acerca de la persona a quien amáis, la cual está presa,
encantada o endemoniada (así lo habéis dicho) en la ciudad adonde vamos.

Una risa franca vino a mis labios, mas la contuve diciendo:

---Es verdad; pero quizás no tenga tiempo para ocuparme de mis propios
asuntos.

---Al contrario---dijo con gracia suma.---No os ocuparéis de otra cosa.
¿Se podrá saber, caballero Araceli, quién es cierta condesa que os
escribe desde Madrid?

---¿Cómo sabe usted\ldots?---pregunté con asombro.

---Porque poco antes de salir yo de casa de Forfolleda, llegó un oficial
con una carta que había recibido para vos. La miré por fuera, y vi unas
armas con corona. Vuestro asistente dijo: «Ya tenemos otra cartita de mi
señora la condesa.»

---¡Y yo salí sin recoger esa carta!---exclamé contrariado.---Vuelvo al
instante a Santi Spíritus.

Pero miss Fly me detuvo con un gesto encantador, diciendo con gracejo
sin igual:

---No seáis impetuoso, joven soldado; tomad la carta.

Y me la dio, y al punto la abrí y la leí. En ella me decía simplemente,
a más de algunas cosas dulces y lisonjeras, que por Marchena acababa de
saber que nuestro enemigo se disponía a salir de Plasencia para
Salamanca.

---Parece que os dan alguna noticia importante, según lo mucho que
reflexionáis sobre ella---me dijo Athenais.

---No me dice nada que yo no sepa. La infeliz madre, agobiada por el
dolor y la impaciencia, me apremia sin cesar para que le devuelva el
bien que le han quitado.

---Esa carta es de la mamá de la encantada---dijo la señorita Mariposa
con incredulidad.---Forjáis historias muy lindas, caballero pero que no
engañarán a personas discretas como yo.

Recorrí la carta con la vista, y seguro de que no contenía cosa alguna
que a los extraños debiera ocultarse, pues la misma condesa había hecho
público el secreto de su desgraciada maternidad, la di a miss Fly para
que la leyese. Ella, con intensa curiosidad, la leyó en un momento, y
repetidas veces alzó los ojos del papel para clavarlos en mí,
acompañando su mirada de expresivas exclamaciones y preguntas.

---Yo conozco esta firma---dijo primero.---La condesa de ***. La vi y la
traté en el Puerto de Santa María.

---En Enero del año 10, señora.

---Justamente\ldots{} Y dice que sois su ángel tutelar, que espera de
vos su felicidad\ldots{} que os deberá la vida\ldots{} que cambiaría
todos los timbres de su casa por vuestro valor, por la nobleza de
vuestro corazón y la rectitud de vuestros altos sentimientos.

---¿Eso dice?\ldots{} pasé la vista sin fijarme más que en lo esencial.

---Y también que tiene completa confianza en vos, porque os cree capaz
de salir bien en la gran empresa que traéis entre manos\ldots{} Que Inés
(¿con que se llama Inés?), a pesar de lo mucho que vale por su hermosura
y por sus prendas, le parece poco galardón para vuestra
constancia\ldots{} Miss Fly me devolvió la carta. Estaba inflamada por
una dulce confusión, casi diré arrebatador entusiasmo, y su brillante
fantasía, despertándose de súbito con briosa fuerza, agrandaba sin duda
hasta límites fabulosos la aventura que delante tenía.

---¡Caballero!---exclamó sin ocultar el expansivo y grandioso
arrobamiento de su alma poética---esto es hermosísimo, tan hermoso que
no parece real. Lo que yo sospechaba y ahora se me revela por completo
tiene tanta belleza como las mentiras de las novelas y romances. De modo
que vos al ir a Salamanca vais a intentar\ldots{}

---Lo imposible.

---Decid mejor dos imposibles---afirmó Athenais con exaltado
acento---porque la comisión de Wellington\ldots{} ¡Qué sublime paso, qué
incomparable atrevimiento, señor Araceli! El coronel Simpson decía hace
poco que hay noventa y nueve probabilidades contra una de que seréis
fusilado.

---Dios me protegerá, señora.

---Seguramente. Si no hubieran existido en el mundo hombres como vos, no
habría historia o sería muy fastidiosa. Dios os protegerá. Hacéis muy
bien\ldots{} apruebo vuestra conducta. Os ayudaré.

---¿Pero todavía insiste usted?

---¡Extraño suceso!---dijo sin hacer caso de mi pregunta---¡y cómo me
seduce y cautiva! En España, sólo en España podría encontrarse esto que
enciende el corazón, despierta la fantasía y da a la vida el aliciente
de vivas pasiones que necesita. Una joven robada, un caballero leal que,
despreciando toda clase de peligros, va en su busca y penetra con ánimo
fuerte en una plaza enemiga, y aspira sólo con el valor de su corazón y
los ardides de su ingenio a arrancar el objeto amado de las bárbaras
manos que la aprisionan\ldots{} ¡Oh, qué aventura tan hermosa! ¡Qué
romance tan lindo!

---¿Gustan a usted, señora, las aventuras y los romances?

---¿Que si me gustan? ¡Me encantan, me enamoran, me cautivan más que
ninguna lectura de cuantas han inventado los ingenios de la
tierra!---repuso con entusiasmo.---¡Los romances! ¿Hay nada más hermoso,
ni que con elocuencia más dulce y majestuosa hable a nuestra alma? Los
he leído y los conozco todos, los moriscos, los históricos, los
caballerescos, los amorosos, los devotos, los vulgares, los de cautivos
y forzados y los satíricos. Los leo con pasión, he traducido muchos al
inglés en verso o prosa.

---¡Oh señora mía e insigne maestra!---dije, afirmando para mí que la
enfermedad moral de miss Fly era una monomanía literaria.---¡Cuánto
deben a usted las letras españolas!

---Los leo con pasión---añadió sin hacerme caso---pero ¡ay! los busco
ansiosamente en la vida real y no puedo, no puedo encontrarlos.

---Justo, porque esos tiempos pasaron, y ya no hay Lindarajas, ni
Tarfes, ni Bravoneles, ni Melisendras---afirmé, reconociendo que me
había equivocado en mi juicio anterior respecto a la enfermedad de la
Pajarita.---¿Pero de veras se ha empeñado usted en encontrar en la vida
real los romances? por ejemplo, aquellas moritas vestidas de verde que
se asomaban a las rejas de plata para despedir a sus galanes cuando iban
a la guerra, aquellos mancebos que salían al redondel con listón
amarillo o morado, aquellos barbudos reyes de Jaén o Antequera
que\ldots?

---Caballero---dijo con gravedad interrumpiéndome---¿habéis leído los
romances de Bernardo del Carpio?

---Señora---respondí turbado---confieso mi ignorancia. No los conozco.
Me parece que los he oído pregonar a los ciegos; pero nunca los compré.
He descuidado mucho mi instrucción, miss Fly.

---Pues yo los sé todos de memoria, desde

\small
\newlength\mlenc
\settowidth\mlenc{El quinto Alfonso reinaba;}
\begin{center}
\parbox{\mlenc}{\quad En los reinos de León                   \\
                El quinto Alfonso reinaba;                    \\
                Hermosa hermana tenía,                        \\
                Doña Jimena se llama,}                        \\ 
\end{center}
\normalsize

\justifying{ \noindent hasta la muerte del héroe, donde hay aquello de }

\small
\newlength\mlend
\settowidth\mlend{\quad Al pie de un túmulo negro}
\begin{center}
\parbox{\mlend}{\quad Al pie de un túmulo negro               \\
                Está Bernardo del Carpio.}                    \\
\end{center}
\normalsize

\justifying{\noindent ¡Incomparable poesía! Después de la Ilíada no se ha compuesto nada mejor. Pues
bien. ¿No conocéis ni siquiera de oídas el romance en que \textit{Bernardo liberta de los moros 
a su amada Estela, y al Carpio que tenían cercado?}}

---Eso ha de ser bonito.

---Parece que resucitan los tiempos---dijo miss Fly con cierta vaguedad
inexplicable, al modo de expresión profética en el semblante---parece
que salen de su sepultura los hombres, revistiendo forma antigua, o que
el tiempo y el mundo dan un paso atrás para aliviar su tristeza,
renovando por un momento las maravillas pasadas\ldots{} La Naturaleza,
aburrida de la vulgaridad presente, se viste con las galas de su
juventud, como una vieja que no quiere serlo\ldots{} Retrocede la
Historia, cansada de hacer tonterías, y con pueril entusiasmo hojea las
páginas de su propio diario y luego busca la espada en el cajón de los
olvidados y sublimes juguetes\ldots{} ¿pero no veis esto, Araceli, no lo
veis?

---Señora, ¿qué quiere usted que vea?

---El romance de Bernardo y de la hermosa Estela, que por segunda
vez\ldots{}

Al decir esto, el caballo que arrastraba no sin trabajo el carricoche de
la poética Athenais, empezó a cojear, sin duda porque no podía
reverdecer, como la Historia, las lozanas robusteces y agilidades de su
juventud. Pero la inglesa no paró mientes en esto, y con gravedad suma
continuó así:

---También tiene ahora aplicación el romance de D. Galván, que no está
escrito; pero que puede recogerse de boca del pueblo como lo he hecho
yo. En él, sin embargo, D. Galván no hubiera podido sacar de la torre a
la infanta, sin el auxilio de una hada o dama desconocida que se le
apareció\ldots{}

El caballo entonces, que ya no podía con su alma, tropezó cayendo de
rodillas.

---Mi estimable hada, aquí tiene usted la realidad de la vida---le
dije.---Este caballo no puede seguir.

---¡Cómo!---exclamó con ira la inglesa.---Andará. Si no enganchad el
vuestro al carricoche, e iremos juntos aquí.

---Imposible, señora, imposible.

---¡Qué desolación! Bien decía \emph{Mistress} Mitchell, que este animal
no sirve para nada. A mí, sin embargo, me pareció digno del carro de
Faetonte.

Levantamos al animal, que dio algunos pasos y volvió a caer al poco
trecho.

---Imposible, imposible---exclamé.---Señora me veo obligado muy a pesar
mío a abandonar a usted.

---¡Abandonarme!---dijo la inglesa.

En sus hermosos ojos brilló un rayo de aquella cólera augusta que los
poetas atribuyen a las diosas de la antigüedad.

---Sí, señora; lo siento mucho. Va a anochecer. De aquí a Salamanca hay
diez leguas, el miércoles a las doce tengo que estar de vuelta en
Bernuy. No necesito decir más.

---Bien, caballero---dijo con temblor en los labios y acerba
reconvención en la mirada.---Marchaos. No os necesito para nada.

---El deber no me permite detenerme ni una hora más---dije volviendo a
montar en mi caballo, después que, ayudado por el aldeanillo, puse sobre
sus cuatro patas al de miss Fly.---El ejército aliado no tardará\ldots{}
¡Ah! ya están aquí. En aquella loma aparecen las avanzadas\ldots{} Las
manda Simpson su amigo de usted, el coronel Simpson\ldots{} Conque deme
usted su licencia\ldots{} No dirá usted, señora mía, que la dejo
sola\ldots{} Allí viene un jinete. Es Simpson en persona.

Miss Fly miró hacia atrás con despecho y tristeza.

---Adiós, hermosa señora mía---grité picando espuelas.---No puedo
detenerme. Si vivo contaré a usted lo que me ocurra.

Apresurado por mi deber, me alejé a todo escape.

\hypertarget{xiv}{%
\chapter{XIV}\label{xiv}}

Marché aquella tarde y parte de la noche, y después de dormir unas
cuantas horas en Castrejón, dejé allí el caballo, y habiendo adquirido
gran cantidad de hortalizas, con más un asno flaquísimo y tristón, hice
mi repuesto y emprendí la marcha por una senda que conducía
directamente, según me indicaron, al camino de Vitigudino. Halleme en
este al medio día del lunes: mas una vez que lo reconocí, aparteme de
él, tomando por atajos y vericuetos hasta llegar al Tormes, que pasé
para coger el camino de Ledesma y lugar de Villamayor. Por varios
aldeanos que encontré en un mesón jugando a la calva y a la rayuela,
supe que los franceses no dejaban entrar a quien no llevase carta de
seguridad dada por ellos mismos, y que aun así detenían a los vendedores
en la plaza sin dejarlos pasar adelante para que no pudiesen ver los
fuertes.

---No me han quedado ganas de volver a Salamanca, muchacho---me dijo el
charro fornido y obeso, que me dio tan lisonjeros informes después de
convidarme a beber en la puerta del mesón.---Por milagro de Dios y de
María Santísima está vivo el señor Baltasar Cipérez, o sea yo mismo.

---¿Y por qué?

---Porque\ldots{} verás. Ya sabes que han mandado vayan a trabajar a las
fortificaciones todos los habitantes de estos pueblos. El lugar que no
envía a su gente es castigado con saqueo y a veces con degüello\ldots{}
Bien dicen que el diablo es sutil. La costumbre es que mientras los
aldeanos trabajan, los soldados estén quietos, hablando y fumando, y de
trecho en trecho hay sargentos que con látigo en mano que están allí con
mucho ojo abierto para ver el que se distrae o mira al cielo o habla a
su compañero\ldots{} Bien dijo el otro, que el diablo no duerme y todo
lo añasca\ldots{} En cuanto se descuida uno tanto así\ldots{}
¡plas!\ldots{}

---Le toman la medida de las espaldas.

---Yo tengo mala sangre---añadió Cipérez---y no creo haber nacido para
esclavo. Soy aldeano rico, estoy acostumbrado a mandar y no a que me den
latigazos. A perro viejo no hay tus tus\ldots{} Así es que cuando aquel
Lucifer me\ldots---Si soy yo el azotado, allí mismo lo tiendo.

---Yo cerré los ojos; yo no vi más que sangre, yo me metí entre todos
porque\ldots{} ¡Baltasar Cipérez azotado por un francés!\ldots{} Yo daba
mojicones\ldots{} quien no puede dar en el asno da en la albarda. En
fin, allí nos machacamos las liendres durante un cuarto de hora\ldots{}
Mira las resultas.

El rico aldeano, apartando la anguarina puesta del revés, según uso del
país, mostrome su brazo vendado y sostenido en un pañuelo al modo de
cabestrillo.

---¿Y nada más? ¡Pues yo creí que le habían ahorcado a usted!

---No, tonto, no me ahorcaron. ¿De veras lo creías tú? Habríanlo hecho
si no se hubiera puesto de parte mía un soldado francés, llamado
Molichard, que es buen hombre y un tanto borracho. Como éramos amigos y
habíamos bebido tantas copas juntos, se dio sus mañas, y sacándome del
calabozo me puso salvo, aunque no sano, en la puerta de Zamora. ¡Pobre
Molichard, tan borracho y tan bueno! Cipérez el rico no olvidará su
generosa conducta.

---Señor Cipérez---dije al leal salamanquino,---yo voy a Salamanca y no
tengo carta de seguridad. Si su merced me proporcionara una\ldots{}

---¿Y a qué vas a allá?

---A vender estas verduras---repuse mostrando mi pollino.

---Buen comercio llevas. Te lo pagarán a peso de oro. ¿Llevas lo que
ellos llaman jericó?

---¿Habichuelas? Sí. Son de Castrejón.

El aldeano me miró con atención algo suspicaz.

---¿Sabes por dónde anda el ejército inglés?---me preguntó clavando en
mí los ojos.---Por la uña se saca al león\ldots---Cerca está, señor
Cipérez. ¿Conque me da su merced la carta de seguridad?\ldots---Tú no
eres lo que pareces---dijo con malicia el aldeano.---¡Vivan los buenos
patriotas y mueran los franceses, todos los franceses, menos Molichard,
a quien pondré sobre las niñas de mis ojos!

---Sea lo que quiera\ldots{} ¿me da su merced la carta de seguridad?

---Baltasarillo---gritó Cipérez---llégate aquí.

Del grupo de los jugadores salió un joven como de veinte años, vivaracho
y alegre.

---Es mi hijo---dijo el charro.---Es un acero\ldots{} Baltasarillo, dame
tu carta de seguridad.

---Entonces\ldots---No, no vayas mañana a Salamanca. Vuelve conmigo a
Escuernavacas. ¿No dices que tu madre quedó muy triste?

---Madre tiene miedo a las moscas; pero yo no.

---¿Tú no?

---Por miedo de gorriones no se dejan de sembrar cañamones---replicó el
mancebo.---Quiero ir a Salamanca.

---A casa, a casa. Te mandaré mañana con un regalito para el señor
Molichard\ldots{} Dame tu carta.

El joven sacó su documento y entregómelo el padre diciendo:

---Con este papel te llamarás Baltasarillo Cipérez, natural de
Escuernavacas, partido de Vitigudino. Las señas de los dos mancebos allá
se van. El papel está en regla y lo saqué yo mismo hace dos meses, la
última vez que mi hijo estuvo en Salamanca con su hermana María, cuando
la fiesta del rey Copas.

---Pagaré a su merced el servicio que me ha hecho---dije echando mano a
la bolsa, cuando Baltasarito se apartó de mí.

---Cipérez el rico no toma dinero por un favor---dijo con
nobleza.---Creo que sirves a la patria, ¿eh? Porque a pesar de ese
pelaje\ldots{} Tan bueno es como el rey y el Papa el que no tiene
capa\ldots{} Todos somos unos. Yo también\ldots---¿Cómo recibirán estos
pueblos al \emph{Lord} cuando se presente?

---¿Cómo le han de recibir\ldots? ¿Le has visto? ¿Está cerca?---preguntó
con entusiasmo.

---Si su merced quiere verle, pásese el miércoles por Bernuy.

---¡Bernuy! Estar en Bernuy es estar en Salamanca---exclamó con exaltado
gozo .---El refrán dice: «Aquí caerá Sansón;» pero yo digo: «Aquí caerá
Marmont y cuantos con él son.» ¿Has visto los estudiantes y los mozos de
Villamayor?

---No he visto nada, señor.

---Tenemos armas---dijo con misterio.---Ténganos el pie al herrar y verá
del que cojeamos\ldots{} Cuando el \emph{Lord} nos vea\ldots{} Y luego,
llevándome aparte con toda reserva, añadió:

---Tú vas a Salamanca mandado por el \emph{Lord}, ¿eh?\ldots{} como si
lo viera\ldots{} No haya miedo. El que tiene padre alcalde, seguro va a
juicio. Bien, amigo\ldots{} has de saber que en todos estos pueblos
estamos preparados, aunque no lo parece. Hasta las mujeres saldrán a
pelear\ldots{} Los franceses quieren que les ayudemos, pero lo que has
de dar al mur dalo al gato, y sacarte ha de cuidado. Yo serví algún
tiempo con Julián Sánchez, y muchas veces entré en la ciudad como
espía\ldots{} Mal oficio\ldots{} pero en manos está el pandero que lo
saben bien tañer.

---Señor Cipérez---dije.---¡Vivan los buenos patriotas!

---No esperamos más que ver al inglés para echarnos todos al campo con
escopetas, hoces, picos, espadas y cuanto tenemos recogido y guardado.

---Y yo me voy a Salamanca. ¿Me dejarán trabajar en las fortificaciones?

---Peligrosillo es. ¿Y el látigo? Quien a mí me trasquiló, las tijeras
le quedaron en la mano\ldots{} Pero si ahora no trabajan los aldeanos en
los fuertes.

---¿Pues quién?

---Los vecinos de la ciudad.

---¿Y los aldeanos?

---Los ahorcan si sospechan que son espías. Que ahorquen. Al freír de
los huevos lo verán, y a cada puerco le llega su San Martín\ldots{} Por
mí nada temo ahora, porque en salvo está el que repica.

---Pero yo\ldots{}

---Ánimo, joven\ldots{} Dios está en el cielo\ldots{} y con esto me voy
hacia Valverdón, donde me esperan doscientos estudiantes y más de
cuatrocientos aldeanos. ¡Viva la patria y Fernando VII! ¡Ah! por si te
sirvo de algo, puedes decir en Salamanca que vas a buscar hierro viejo
para tu señor padre Cipérez el rico\ldots{} adiós\ldots{}

---Adiós, generoso caballero.

---¿Caballero yo? Poco va de Pedro a Pedro\ldots{} Aunque las calzo no
las ensucio\ldots{} Adiós, muchacho, buena suerte. ¿Sabes bien el
camino? Por aquí adelante, siempre adelante. Encontrarás pronto a los
franceses; pero siempre adelante, adelante siempre. Aunque mucho sabe la
zorra, más sabe el que la toma.

Nos despedimos el bravo Cipérez y yo dándonos fuertes apretones de
manos, y seguí a buen paso mi camino.

\hypertarget{xv}{%
\chapter{XV}\label{xv}}

Detúveme a descansar en Cabrerizos ya muy alta la noche del lunes al
martes, y al amanecer del día siguiente, cuando me disponía a hacer mi
entrada en la ciudad, insigne maestra de España y de la civilización del
mundo, los franceses, que hasta entonces no me habían incomodado,
aparecieron en el camino. Era un destacamento de dragones que custodiaba
cierto convoy enviado por Marmont desde Fuentesaúco. A pesar de que no
había motivo para creer que aquellos señores se metieran conmigo, yo
temía una desgracia; mas disimulé mi zozobra y recelo, arreando el
pollino, y afectando divertir la tristeza del camino con cantares
alegres.

No me engañó el corazón, pues los invasores de la patria ¡que comidos de
los lobos sean antes, ahora y después! sin intentar hacerme manifiesto
daño, antes bien un beneficio aparente, contrariaron mi plan de un modo
lastimoso.

---Hermosas hortalizas---dijo en francés un cabo llevando su caballo al
mismo paso que mi pollino.

No dije nada, y ni siquiera le miré.

---¡Eh, imbécil!---gritó en lengua híbrida, dándome con su sable en la
espalda---¿llevas esas verduras a Salamanca?

---Sí, señor---respondí afectando toda la estupidez que me era posible.

Un oficial detuvo el paso y ordenó al cabo que comprase toda mi
mercancía.

---Todo, lo compramos todo---dijo el cabo sacando un bolsillo de trapo
mugriento.---\emph{¿Combien?}

Hice señas negativas con la cabeza.

---¿No llevas eso a Salamanca para venderlo?

---No, señor, es para un regalo.

---¡Al diablo con los regalos! Nosotros compramos todo, y así, gran
imbécil, podrás volverte a tu pueblo.

Comprendí que resistir a la venta era infundir sospechas, y les pedí un
sentido por las verduras, cuya escasez era muy grande en aquella época y
en aquel país. Mas enfurecido el soldado, amenazome con abrirme
bonitamente en dos: subió luego el precio más de lo ofrecido, bajé yo un
tantico, y nos ajustamos. Recibí el dinero, mi pollino se quedó sin
carga, y yo sin motivo aparente para justificar mi entrada en la ciudad,
porque a los que no iban con víveres les daban con la puerta en los
hocicos. Seguí, sin embargo, hacia adelante, y el cabo me dijo:

---¡Eh, buen hombre! ¿No os volvéis a vuestro pueblo? No he visto mayor
estúpido.

---Señor---repuse---voy a cargar mi burro de hierro viejo.

---¿Tienes carta de seguridad?

---¿Pues no la he de tener? Cuando estuve en Salamanca hace dos meses,
para ver las fiestas del rey, me la dieron\ldots{} Pero como ahora no
llevo carga puede que no me dejen entrar a recoger el hierro viejo. Si
el señor cabo quiere que vaya con su merced para que diga cómo me compró
las verduras\ldots{} pues, y que voy por hierro viejo.

---Bueno, \emph{saco de papel}: pon tu burro al paso de mi caballo y
sígueme; mas no sé si te dejarán entrar, porque hay órdenes muy
rigurosas para evitar el espionaje.

Llegamos a la puerta de Zamora y allí me detuvo con muy malos modos el
centinela.

---Déjalo pasar---dijo mi cabo;---le he comprado las verduras y va a
cargar de hierro su jumento.

Mirome el cabo de guardia con recelo, y al ver retratada en mi semblante
aquella beatífica estupidez propia de los aldeanos que han vivido largo
tiempo en lo más intrincado de selvas y dehesas, dijo así:

---Estos palurdos son muy astutos. ¡Eh! \emph{monsieur le badaud}. En
esta semana hemos ahorcado a tres espías.

Yo fingí no comprender, y él añadió:

---Puedes entrar si tienes carta de seguridad.

Mostré el documento y entonces me dejaron pasar.

Atravesé una calle larga, que era la de Zamora, y me condujo en
derechura a una grande y hermosa plaza de soportales, ocupada a la sazón
por gran gentío de vendedores. Busqué en las inmediaciones posada donde
dejar mi burro para poder dedicarme con libertad al objeto de mi viaje,
y cuando hube encontrado un mesón, que era el mejor de la ciudad, y
acomodado en él con buen pienso de paja y cebada a mi pacífico
compañero, salí a la calle. Era la de la Rúa, según me dijo una muchacha
a quien pregunté. Mi afán era trasladarme al recinto amurallado para
recorrerlo todo. De pronto vi multitud de personas de diversas clases
que marchaban en tropel llevando cada cual al hombro azadón o pico.
Escoltábanles soldados franceses, y no iban ciertamente muy a gusto
aquellos señores.

---Son los habitantes de la ciudad que van a trabajar a las
fortificaciones---dije para mí.---Los franceses les llevan a la fuerza.

Aparteme a un lado por temor a que mi curiosidad infundiese sospechas, y
andando sin rumbo ni conocimiento de las calles, llegué a un convento,
por cuyas puertas entraban a la sazón algunas piezas de artillería. De
repente sentí una pesada mano sobre mi hombro, y una voz que en mal
castellano me decía:

---¿No tomáis una azada, holgazán? Venid conmigo a casa del comisario de
policía.

---Yo soy forastero---repuse;---he venido con mi borriquito\ldots{}

---Venid y se sabrá quién sois---continuó mirándome atentamente.---Si
\emph{par exemple}, fueseis \emph{espion\ldots{}}

Mi primer intento fue resistirme a seguirle; pero hubiérame vendido la
resistencia, y parecía más prudente ceder. Afectando la mayor humildad
seguí a mi extraño aprehensor, el cual era un soldado pequeño y
vivaracho, ojinegro, morenito y oficioso, cuyo empaque y modos me hacían
poquísima gracia. En el recodo que hacía una calle tortuosa y oscura,
traté de burlarle, quedándome un instante atrás para poner los pies en
polvorosa con la ligereza que me era propia; mas adivinando el menguado
mis intenciones, asiome del brazo y socarronamente me dijo:

---¿Creéis que soy menos listo que vos? Adelante y no deis coces, porque
os levanto la tapa de los sesos, señor patán. Ya no me queda duda que
sois \emph{espion}. Estabais observando la artillería de las monjas
Bernardas. Estabais midiendo la muralla. Sabed que aquí hay unos
funcionarios muy astutos que espían a los espías, y yo soy uno de ellos.
¿No habéis bailado nunca al extremo de una cuerda?

Nuevamente sentí impulsos de librarme de aquel hombre por la violencia;
mas por fortuna tuve tiempo de reflexionar, sofocando mi cólera, y
fiando mi salvación a la astucia y al disimulo. Llevome el endemoniado
francesillo a un vasto edificio, en cuyo patio vi mucha tropa, y
deteniéndose conmigo ante un grupo formado de cuatro robustos y
poderosos militares de brillantes uniformes, bigotazos retorcidos e
imponente apostura, me señaló con expresión de triunfo.

---¿Qué traes, Tourlourou?---preguntó con fastidio el más viejo de
todos.

---Un \emph{crapaud} pescado ahora mismo.

Quiteme el sombrero, y con aire contrito y humildísimo hice varias
reverencias a aquellos apreciables sujetos.

\emph{---¡Un crapaud!}---repitió el viejo oficial, dirigiéndose a mí con
fieros ojos.---¿Quién sois?

---Señor---dije cruzando las manos.---Ese señor soldado me ha tomado por
un espía. Yo vengo de Escuernavacas a buscar hierro viejo, tengo mi
burro en el mesón de una tal tía Fabiana, y me llamo Baltasar Cipérez
para lo que vuecencia guste mandar. Si quieren ahorcarme,
ahórquenme\ldots---y luego sollozando del modo más lastimero y exhalando
gritos de dolor que hubieran conmovido al mismísimo bronce,
exclamé:---¡Adiós, madre querida; adiós, padre de mi corazón; ya no
veréis más a vuestro hijito; adiós, Escuernavacas de mi alma, adiós,
adiós! Pero yo, ¿qué he hecho, qué he hecho yo, señores?

El oficial anciano dijo con calma imperturbable:

Quitadme de delante este canalla. Molichard, sargento Molichard, mandad
que le encierren en el calabozo. Después le interrogaremos. Ahora estoy
muy ocupado. Voy a ver al \emph{Maréchal de Logis}, porque se dice que
esta tarde saldremos de Salamanca.

Presentose otro francés alto como un poste, derecho como un huso, flaco
y duro y flexible cual caña de Indias, de fisonomía curtida y burlona,
ojos vivos, lacios y negros bigotes, y manos y pies de descomunal
magnitud. Cuando vi a aquel pedazo de militar, de cuya osamenta pendía
el uniforme como de una percha; cuando oí su nombre, una idea salvadora
iluminó súbito mi cerebro, y pasando del pensamiento a la ejecución con
la rapidez de la voluntad humana en casos de apuro, lancé una
exclamación en que al mismo tiempo puse afectadamente sorpresa y júbilo;
corrí hacia él, me abracé con vehemente ardor a sus rodillas, y llorando
dije:

---¡Oh, Sr.~Molichard de mi alma, Sr.~Molichard, queridísimo y
reverenciadísimo! Al fin le encuentro. Y ¡cuánto le he buscado sin que
estos pícaros me dieran razón de su merced! Déjeme que le abrace, que
bese sus rodillas y que le reverencie y acate y venere\ldots{} ¡Oh,
Santa Virgen María: qué gozo tan grande!

---Creo que estáis loco, buen hombre---dijo el francés sacudiendo sus
piernas.

---Pero, ¿no me conoce usía?---añadí.---Pero, ¿cómo me ha de conocer, si
no me ha visto nunca? Deme esa mano que la bese y viva mil años el buen
Sr.~Molichard que salvó a mi padre de la muerte. Soy Baltasar Cipérez,
mire la carta de seguridad, soy hijo del tío Baltasar a quien llaman
Cipérez el rico, natural de Escuernavacas. Bendito sea el Sr.~Molichard.
Estoy en Salamanca porque hame mandado mi padre con un obsequio para su
merced.

---¡Un obsequio!---exclamó el sargento con alborozado semblante.

---Sí señor, un obsequio miserable, pues lo que usía ha hecho no lo
pagará mi padre con los pobres frutos de su huerta.

---¡Verduras! ¿Y dónde están?---dijo Molichard volviendo en derredor los
ojos.

---Me las quitó en el camino un cabo de dragones, cuyo nombre no sé;
pero que debe de andar por aquí y podrá dar testimonio de lo que digo.
Pues poco le gustaron a fe. Regostose la vieja a los bledos, no dejó
verdes ni secos.

---¡Oh, peste de dragones!---exclamó con furia el protector de mi
padre.---Yo se las sacaré de las tripas.

---Me obligó a que se las vendiera---continué;---pero puedo dar a usía
el dinero que me entregó; además, de que en el primer viaje que haga a
Salamanca traeré, no una, sino dos cargas para el Sr.~Molichard. Mas no
es el único obsequio que traigo a su merced. Mi padre no sabía qué
hacer, porque quien da luego da dos veces; mi madre, que no ha venido en
persona a ponerse a los pies de usía, porque le están echando cintas
nuevas a la mantilla, quería que padre echase la casa por la ventana
para obsequiar a su protector, y cuando me puse en camino pensaron los
dos que la verdura era regalo indigno de su agradecido corazón,
liberalidad y mucha hacienda; por cuya razón diéronme tres doblones de
oro para que en Salamanca comprase para usía un tercio de vino de la
Nava, que aquí lo hay bueno, y el del pueblo revuelve los hígados.

---El Sr.~Cipérez es hombre generoso---dijo el francés pavoneándose ante
sus amigos, que no estaban menos absortos y gozosos que él.

---Lo primero que hice en Salamanca esta mañana fue contratar el tercio
en el mesón de la tía Fabiana. Conque vamos por él\ldots---El vino de la
tía Fabiana no puede ser mejor que el que hay en la taberna de la
Zángana. Puedes comprarlo allí.

---Daré aína el dinero a su merced para que lo compre a su gusto. Bien
dicen que al que Dios quiere bien, en casa le traen de comer. ¡Cuánto
trabajo para encontrar al Sr.~Molichard! Preguntaba a todo el mundo sin
que nadie me diera razón, hasta que este buen amigo me tomó por espía y
trájome aquí\ldots{} no hay mal que por bien no venga\ldots{} ¡Al fin he
tenido el gusto de abrazar al amigo de mi padre! ¡Qué casualidad! Ojos
que se quieren bien, desde lejos se ven\ldots{} Sr. Molichard, cuando me
deje su merced en el calabozo, donde el oficial mandó que me pusieran,
puede ir a escoger el vino que más le acomode. ¡Bendito sea Dios que
hizo rico a mi buen padre para poder pagar con largueza los beneficios!
Mi padre quiere mucho al Sr.~Molichard. Quien te da el hueso no quiere
verte muerto.

---En lo de ensartar refranes---dijo Molichard,---se conoce la sangre
del Sr. Cipérez.

---Si bien canta el cura, no le va en zaga el monaguillo.

Molichard pareció indeciso y después de consultar a sus compañeros con
la vista y algún monosílabo que no entendí, me dijo:

---Yo bien quisiera no encerraros en el calabozo, porque, en verdad,
cuando le obsequian a uno de parte del Sr.~Cipérez\ldots{} pero\ldots{}

---No\ldots{} no se apure por mí el Sr.~Molichard---dije con la mayor
naturalidad del mundo.---Ni quiero que por mí le riña el señor oficial.
Al calabozo. Como estoy seguro de que el señor oficial y todos los
oficiales del mundo se convencerán de que no soy malo\ldots{}

---En el calabozo lo pasaréis mal, joven\ldots---dijo el
francés.---Veremos. Se le dirá al oficial que\ldots{}

---El oficial no se acuerda ya de lo que mandó---afirmó Tourlourou,
quien, por encantamiento, había olvidado sus rencores contra mí.

---¡Eh! Jean-Jean---gritó Molichard llamando a un compañero que cercano
al lugar de la escena pasaba, y en cuya pomposa figura conocí al cabo de
dragones que comprara mis verduras en el camino.

Acercose Jean-Jean, por quien fui al punto reconocido.

---Buen amigo---le dije,---me parece que fue su merced quien me compró
las verduras que traje para el señor.---¿Para Molichard?\ldots---¿No
dije que eran para un regalo?

---A saber que eran para este \emph{chauve souris}---dijo
Jean-Jean,---no os hubiera dado un céntimo por ellas.

---Jean-Jean---dijo Molichard en francés,---¿te gusta el vino de la
Nava?

---Verlo no. ¿Dónde lo hay?

---Mira, Jean-Jean. Este joven me ha regalado un trago. Pero tenemos que
ponerle a él en el calabozo\ldots---¡En el calabozo!

---Sí, \emph{mon vieux}, le han tomado por espía sin serlo.

---Vámonos a la taberna los cuatro---dijo Tourlourou---y luego el señor
se quedará en su calabozo.

---Yo no quiero que por mí se indispongan sus mercedes con los
jefes---dije con humildad y apocamiento.---Llévenme a la prisión,
enciérrenme\ldots{} Cada lobo en su senda y cada gallo en su muladar.

---¿Qué es eso de encerrar?---gritó Molichard en tono campechano y
tocando las castañuelas con los dedos.---A casa de la Zángana,
\emph{messieurs}. Cipérez, nosotros respondemos de ti.

\hypertarget{xvi}{%
\chapter{XVI}\label{xvi}}

---¿Y si se enfada el oficial? Yo no me muevo de aquí.

---Un francés, un soldado de Napoleón---dijo Tourlourou con un gesto
parecido al de Bonaparte señalando las pirámides,---no bebe tranquilo
mientras que su amigo español se muere de sed en una mazmorra. Bravo,
Cipérez---añadió abrazándome,---sois el primero entre mis camaradas.
Abracémonos\ldots{} Bien, así\ldots{} amigos hasta la muerte. Señores,
ved juntos aquí \emph{l'aigle de l'Empire et le lion de l'Espagne}.

Francamente, a mí, león de España, me hacían poquísima gracia, como
aquella, los brazos del águila del imperio.

Y con esto y otros excesos verbales de los tres servidores del gran
imperio, me sacaron fuera del cuartel y en procesión lleváronme a un
ventorrillo cercano a las fortificaciones de San Vicente.

Señor Molichard, aparte del tercio de lo de la Nava, que es regalo de mi
señor padre, yo pago todo el gasto---dije al entrar.

En poco tiempo, Tourlourou, Molichard y Jean-Jean, regalaron sus
venerandos cuerpos con lo mejor que había en la bodega, y helos aquí que
por grados perdían la serenidad, si bien el cabo de dragones parecía
tener más resistencia alcohólica que sus ilustres compañeros de armas y
de vino.

---¿Tiene mucha hacienda vuestro padre?---me preguntó Molichard.

---Bastante para pasar---respondí con modestia.

---Llámanle Cipérez el rico.

---Cierto, y lo es\ldots{} Veo que mi obsequio parece poco\ldots{} Por
ahí se empieza. Ya sabemos que sobre un huevo pone la gallina.

---No digo eso. ¡A la salud de \emph{monsieurrrr} Cipérez!

---Esto que hoy he traído, es porque como venía a mercar hierro
viejo\ldots{} Pero mi padre y mi madre y toda la familia, vendrán en
procesión \emph{solene} con algo mejor. Sr.~Molichard, mi hermana quiere
conocer al Sr.~Molichard\ldots{}

---Es una linda muchacha, según decía Cipérez. ¡A la salud de María
Cipérez!

---Muy guapa. Parece un sol, y cuantos la ven la tienen por princesa.

---Y una buena dote\ldots{} Si al fin irá uno a dejar su pellejo en
España. Digamos como Luis XIV: «Ya no hay \emph{Pirrineos.»} Bebed,
Baltasarico.

---Yo tengo muy floja la cabeza. Con tres medias copas que he bebido, ya
estoy como si me hubieran metido a toda Salamanca entre sien y
sien---dije fingiendo el desvanecimiento de la embriaguez.

Jean-Jean cantaba:

\small
\newlength\mlene
\settowidth\mlene{Le crocodile en partant pour la guerre}
\begin{center}
\parbox{\mlene}{\textit{Le crocodile en partant pour la guerre  \\
                Disait adieux a ses petits enfants.}}           \\
\end{center}
\normalsize

\small
\newlength\mlenf
\settowidth\mlenf{dans la poussière…}
\begin{center}
\parbox{\mlenf}{\textit{Le malheureux                           \\
                traînait sa queue                               \\
                dans la poussière…}}                            \\
\end{center}
\normalsize

Tourlourou, después de remedar el gato y el perro, púsose de pie y con
gesto majestuoso exclamó:

---Camaradas, desde lo alto de esta botella \emph{quarrrrente siècles
vous contemplent}.

Yo dije a Molichard:

---Señor sargento, como no acostumbro a beber, me he mareado de tal
modo\ldots{} Voy a salir un momento a tomar el aire. ¿Ha escogido usted
su vino de la Nava?

Y sin esperar contestación, pagué a la Zángana.

---Bien; vamos un momento afuera---repuso Molichard tomándome del brazo.

Al salir encontreme en un sitio que no era plaza, ni patio, ni calle;
sino más bien las tres cosas juntas. A un lado y otro veíanse altas
paredes, unas a medio derribar, otras en pie todavía, sosteniendo los
techos destrozados. Al través de estos se distinguía el interior abierto
de los que fueron templos, cuyos altares habían quedado al aire libre; y
la luz del día, iluminando de lleno las pinturas y dorados, daba a estos
el aspecto de viejos objetos de prendería cuando los anticuarios de
feria los amontonan en la calle. Soldados y paisanos trabajaban llevando
escombros, abriendo zanjas, arrastrando cañones, amontonando tierra,
acabando de demoler lo demolido a medias, o reparando lo demolido con
exceso. Vi todo esto, y acordándome de lord Wellington, puse mi alma
toda en los ojos. Yo hubiera querido abarcar de un solo golpe de vista
lo que ante mí tenía y guardarlo en mi memoria, piedra por piedra, arma
por arma, hombre por hombre.

---¿Qué es esto que hacen aquí, señor Molichard?---pregunté
cándidamente.

---¡Fortificaciones, animal!---dijo el sargento, que después que se
llenó el cuerpo con mi vino, había empezado a perderme el respeto.

---Ya, ya comprendo---repuse afectando penetración.---Para la guerra. ¿Y
cómo llaman este sitio?

---Esto en que estamos es el fuerte de San Vicente, y aquí había un
convento de benedictinos, que se derribó. Una guarida de mochuelos, mi
amiguito.

---¿Y qué van a hacer aquí con tanto cañón?---pregunté estupefacto.

---Pues no eres poco bestia. ¿Qué se ha de hacer? Fuego.

---¡Fuego!---dije medrosamente.---¿Y todos a la vez?

---Te pones pálido, cobarde.

---Uno, dos, tres, cuatro\ldots{} allí traen otro. Son cinco. Y esa
tierra, mi sargento, ¿para qué es?

---No he visto un animal semejante. ¿No ves que se están haciendo
escarpa y contra escarpa?

---¿Y aquel otro caserón hecho pedazos que se ve más allá?

---Es el castillo árabe-romano. \emph{¡Foudre et tonnerre!} Eres un
ignorante\ldots{} Dame la mano, que San Cayetano me baila delante.

---¿San Cayetano?

---¿No lo ves, zopenco? Aquel convento grande que está a la derecha.
También lo estamos fortificando.

---Esto es muy bonito, señor Molichard. Será gracioso ver esto cuando
empiece el fuego. ¿Y aquellos paredones que están derribando?

---El colegio Trilingüe\ldots{} \emph{triquis lingüis} en latín, esto
es, \emph{de tres lenguas}. Todavía no han acabado el camino cubierto
que baja a la Alberca.

---Pero aquí han derribado calles enteras, señor Molichard---dije
avanzando más y dándole el brazo para que no se cayese.

---Pues no parece sino que viene del Limbo, \emph{¡Ventre de bœuf!} ¿No
ves que hemos echado al suelo la calle larga para poder esparcir los
fuegos de San Vicente?\ldots{}

---Y allí hay una plaza\ldots{}

---Un baluarte.

---Dos, cuatro, seis, ocho cañones nada menos. Esto da miedo.

---Juguetes\ldots{} Los buenos son aquellos cuatro, los del rebellín.

---Y por aquí va un foso\ldots{}

---Desde la puerta hasta los Milagros, bruto.

¿Y detrás?\ldots{} Jesús, María y José ¡qué miedo!

---Detrás el parapeto donde están los morteros.

---Vamos ahora por aquel lado.

---¿Por San Cayetano?\ldots{} ¡Oh!\ldots{} Veo que eres curioso,
curiosito\ldots{} \emph{Saperlotte}. Te advierto que si sigues haciendo
tales preguntas y mirando con esos ojos de buey\ldots{} me harás creer
que ciertamente eres espía\ldots{} y a la verdad, amiguito,
sospecho\ldots{}

El sargento me miró con descaro y altanería. Llegó a la sazón Tourlourou
en lastimoso estado, y mal sostenido por su amigo Jean-Jean, que
entonaba una canción guerrera.

\emph{---¡Espion,} sí, \emph{espion!}---dijo Tourlourou
señalándome.---Sostengo que eres espion. ¡Al calabozo!

---Francamente, caballero Cipérez---dijo Molichard---yo no quisiera
faltar a la disciplina, ni que el jefe me pusiera en el nicho por ti.

---Tiene este mancebo---afirmó Jean-Jean sentándome la mano en el hombro
con tanta fuerza, que casi me aplastó---cara de tunante.

---Desde que le vi sospeché algo malo---dijo Molichard.---No está uno
seguro de nadie en esta maldita tierra de España. Salen espías de debajo
de las piedras\ldots{}

Yo me encogí de hombros, fingiendo no entender nada.

---¿Pero no os dije que estaba observando el convento de Bernardas, cuya
muralla se está aspillerando?---dijo Tourlourou.

Comprendí que estaba perdido; pero esforceme en conservar la serenidad.
De pronto entró en mi alma un rayo de esperanza al oír pronunciar a
Jean-Jean las siguientes palabras en mal castellano:

---Sois unos bestias. Dejadme a mí al Sr.~Cipérez, que es mi amigo.

Pasó un brazo por encima de mi hombro con familiaridad cariñosa aunque
harto pesada.

---Volvámonos al cuartel---dijo Molichard.---Yo entro de guardia a las
diez.

Y asiéndome por el brazo añadió:

---\emph{¡Peste, mille pestes!\ldots{}} ¿Queríais escapar?

---En el cuartel se le registrará---exclamó Tourlourou.

---Fuera de aquí \emph{goguenards}---dijo con energía Jean-Jean.---El
Sr.~Cipérez es mi amigo y le tomo bajo mi protección. Andad con mil
demonios y dejádmelo aquí.

Tourlourou reía; pero Molichard mirome con ojos fieros, e insistió en
llevarme consigo; mas aplicole mi improvisado protector tan fuerte
porrazo en el hombro que al fin resolvió marcharse con su compañero,
ambos describiendo eses y otros signos ortográficos con sus desmayados
cuerpos. He referido con alguna minuciosidad los hechos y dichos de
aquellos bárbaros, cuya abominable figura no se borró en mucho tiempo de
mi memoria. Al reproducir los primeros no me he separado de la verdad lo
más mínimo. En cuanto a las palabras, imposible sería a la retentiva más
prodigiosa conservarlas tal y como de aquellas embriagadas bocas
salieron, en jerga horrible que no era español ni francés. Pongo en
castellano la mayor parte, no omitiendo aquellas voces extranjeras que
más impresas han quedado en mi memoria, y conservo el tratamiento de
vos, que comúnmente nos daban los franceses poco conocedores de nuestro
modo de hablar.

¿La protección de Jean-Jean era desinteresada o significaba un nuevo
peligro mayor que los anteriores? Ahora se verá si tienen mis amigos
paciencia para seguir oyendo el puntual relato de mis aventuras en
Salamanca el día 16 de Junio de 1812, las cuales, a no ser yo mismo
protagonista y actor principal de todas ellas, las diputara por hechuras
engañosas de la fantasía o invenciones de novelador para entretener al
vulgo.

\hypertarget{xvii}{%
\chapter{XVII}\label{xvii}}

El señor Jean-Jean me tomó el brazo y llevándome adelante por entre
aquellas tristes ruinas, díjome:

---Amigo Cipérez, he simpatizado con vos; nos pasearemos juntos\ldots{}
¿Cuándo pensáis dejar a Salamanca? Os juro que lo sentiré.

Tan relamidas expresiones fueron funestísimo augurio para mí, y
encomendé mi alma a Dios. En mi turbación, ni siquiera reparé en el
aparato de guerra que a mi lado había, y olvideme ¡oh Jesús divino! de
lord Wellington, de Inglaterra y de España.

---Mucho me agrada su compañía---dije afectando valor.---Vamos a donde
usted quiera.

Sentí que el brazo del francés, cual máquina de hierro, apretaba
fuertemente el mío. Aquel apretón quería decir: «No te me escaparás,
no.» A medida que avanzábamos, noté que era más escasa la gente y que
los sitios por donde lentamente discurríamos, estaban cada vez más
solitarios. Yo no llevaba más armas que una navaja. Jean-Jean, que era
hombre robustísimo y de buena estatura, iba acompañado de un poderoso
sable. Con rápida mirada observé hombre y arma para medirlos y
compararlos con la fuerza que yo podía desplegar en caso de lucha.

---¿A dónde me lleva usted?---pregunté deteniéndome al fin, resuelto a
todo.

---Seguid, mi buen amigo---dijo con burlesco semblante.---Nos pasearemos
por la orilla del Tormes.

---Estoy algo cansado.

Parose, y clavando sus pequeños ojos en mí, me dijo:

---¿No queréis seguir al que os ha librado de la horca?

Con esa llama de intuición que súbitamente nos ilumina en momentos de
peligro, con la perspicacia que adquirimos en la ocasión crítica en que
la voluntad y el pensamiento tratan de sobreponerse con angustioso
esfuerzo a obstáculos terribles, leí en la mirada de aquel hombre la
idea que ocupaba su alma. Indudablemente Jean-Jean había conocido que yo
llevaba conmigo mayor cantidad de dinero que la que mostré en la
taberna, y ya me creyese espía, ya el verdadero Baltasar Cipérez, tentó
mi caudal su codicia, y el fiero dragón ideó fáciles medios para
apropiárselo. Aquel equívoco aspecto suyo, aquel solitario paraje por
donde me conducía, indicaban su criminal proyecto, bien fuese este
matarme para dar luego con mi cuerpo en el río, bien fuese expoliarme,
denunciándome después como espía.

Por un instante sentí cobarde y vencida el alma, trémulo y frío el
cuerpo: la sangre toda se agolpó a mi corazón, y vi la muerte, un fin
horrible y oscuro, cuyo aspecto afligió mi alma más que mil muertes en
el terrible y glorioso campo de batalla\ldots{} Miré en derredor y todo
estaba desierto y solo. Mi verdugo y yo éramos los únicos habitantes de
aquel lugar triste, abandonado y desnudo. A nuestro lado ruinas deformes
iluminadas por la claridad de un sol que me parecía espantoso; delante
el triste río, donde el agua remansada y quieta no producía, al parecer,
ni corriente ni ruido; más allá la verde orilla opuesta. No se oía
ninguna voz humana, ni paso de hombre ni de bruto, ni más rumor que el
canto de los pájaros que alegremente cruzaban el Tormes para huir de
aquel sitio de desolación en busca de la frescura y verdor de la otra
ribera. No podía pedir auxilio a nadie más que a Dios.

Pero sentí de pronto la iluminación de una idea divina, divina, sí, que
penetró en mi mente, lanzada como rayo invisible de la inmortal y alta
fuente del pensamiento; sentí no sé qué dulces voces en mi oído, no sé
qué halagüeñas palpitaciones en mi corazón, un brío inexplicable, una
esperanza que me llenaba todo, y sentir esto, y pensarlo, y formar un
plan, fue todo uno. He aquí cómo.

Bruscamente y disimulando tanto mi recelo cual si fuera yo el criminal y
él la víctima, detuve a Jean-Jean, tomé una actitud severa, resuelta y
grave; le miré como se mira a cualquier miserable que va a prestarnos un
servicio, y en tono muy altanero le dije:

---Sr.~Jean-Jean: este sitio me parece muy a propósito para hablar a
solas.

El hombre se quedó lelo.

---Desde que le vi a usted, desde que le hablé, le tuve por hombre de
entendimiento, de actividad, y esto precisamente, esto, es lo que yo
necesito ahora.

Vaciló un momento, y al fin estúpidamente me dijo:

---De modo que\ldots{}

---No, no soy lo que parezco. Se puede engañar a esos imbéciles
Tourlourou y Molichard; pero no a usted.

---Ya me lo figuraba---afirmó,---sois espía.

---No.~Extraño que un entendimiento como el tuyo haya incurrido en esa
vulgaridad---dije tuteándole con desenfado.---Ya sabes que los espías
son siempre rústicos labriegos que por dinero exponen su vida. Mírame
bien. A pesar del vestido, ¿tengo cara de labriego?

---No, a fe mía. Sois un caballero.

---Sí, un caballero, un caballero, y tú también lo eres, pues la
caballerosidad no está reñida con la pobreza.

---Ciertamente que no.

---¿Y has oído nombrar al marqués de Rioponce?

---No\ldots{} sí\ldots{} sí me parece que le he oído nombrar.

---Pues ese soy yo. ¿Podré vanagloriarme de haber encontrado en este día
aciago para mí, un hombre de buenos sentimientos que me sirva, y al cual
demostraré mi gratitud recompensándole con lo que él mismo nunca ha
podido soñar?\ldots{} Porque tú como soldado eres pobre, ¿no es cierto?

---Pobre soy---dijo, no disimulando la avaricia que por las claras
ventanas de sus ojos asomaba.

---Escasa es la cantidad que llevo sobre mí; pero para la empresa que
hoy traigo entre manos he traído suma muy respetable, hábilmente
encerrada dentro del pelote que rellena el aparejo de mi cabalgadura.

---¿Dónde dejasteis vuestro pollino?---preguntó.

Me quería comer con los ojos.

---Eso se queda para después.

---Si sois espía, no contéis conmigo para nada, señor marqués---dijo con
cierta confusión.---No haré nunca traición a mis banderas.

---Ya he dicho que no soy espía.

---\emph{C'est drôle}. ¿Pues qué demonios os trae a Salamanca en ese
traje, vendiendo verduras y haciéndoos pasar por un campesino de
Escuernavacas?

---¿Qué me trae? Una aventura amorosa.

Dije esto y lo anterior con tal acento de seguridad, tanto aplomo y
dominio de mí mismo, que en los ojos del que había querido ser mi
asesino observé, juntamente con la avaricia, la convicción.

---¡Una aventura amorosa!---dijo asaltado nuevamente por la duda,
después de breve meditación.---¿Y por qué no habéis venido tal y como
sois? ¿Para qué ocultaros así de toda Salamanca?

---¡Qué pregunta!\ldots{} A fe que en ciertos momentos pareces un niño
inocente. Si la aventura amorosa fuera de esas que se vienen a la mano
por fáciles y comunes, tendrías razón; pero esta de que me ocupo es
peligrosa y tan difícil, que es indispensable ocultar por completo mi
persona.

---¿Es que algún francés os ha quitado vuestra novia?---preguntó el
dragón sonriendo por primera vez en aquel diálogo.

---Casi, casi\ldots{} parece que vas acertando. Hay en Salamanca una
persona que amo y a quien me llevaré conmigo, si puedo; ¡otra que
aborrezco y a quien mataré si puedo!

---¿Y esa segunda persona es quizás alguno de nuestros queridos
generales?---dijo con sequedad.---Señor marqués, no contéis conmigo para
nada.

---No, esa persona no es ningún general, ni siquiera es francés. Es un
español.

---Pues si es un español, le \emph{diable m'emporte\ldots{}} podéis
tratarle todo lo mal que os agrade. Ningún francés os dirá una palabra.

---No, porque ese hombre es poderoso, y aunque español ha tiempo que
sirve la causa francesa. Es travieso como ninguno, y si me hubiera
presentado aquí dando a conocer mi nombre, habríame sido imposible
evitar una persecución rápida y terrible, o quizás la muerte.

---En una palabra, señor mío---dijo con impaciencia,---¿qué es lo que
queréis que yo haga para serviros?

---Primero que no me denuncies, estúpido---exclamé tratándole
despóticamente para establecer mejor aún mi superioridad;---después que
me ayudes a buscar el domicilio de mi enemigo.

---¿No lo sabéis?

---No.~Es esta la primera vez que vengo a Salamanca. Como vuestros
groseros camaradas quisieron prenderme, no he tenido tiempo de nada.

---Ahora que nombráis a mis camaradas\ldots---dijo Jean-Jean con mucho
recelo---me ocurre\ldots{} Cuidado que hicisteis bien el papel de
aldeano. No me he olvidado de los refranes. Si ahora también\ldots{}

---¿Sospechas de mí?---grité con altanería.

---Nada de soberbia, señor marquesito---repuso con insolencia.---Ved que
puedo denunciaros.

---Si me denuncias, sólo experimento la contrariedad de no poder llevar
adelante mi proyecto; pero tú perderás lo que yo pudiera darte.

---No hay que reñir---dijo en tono benévolo.---Referidme en qué consiste
esa aventura amorosa, pues hasta ahora no me habéis dicho más que
vaguedades.

---Un miserable hijo de Salamanca, un perdido, un \emph{sans culotte} ha
robado de la casa paterna a cierta gentil doncella, de la más alta
nobleza de España, un ángel de belleza y de virtud\ldots---¡La ha
robado!\ldots{} Pues qué, ¿así se roban doncellas?

---La ha robado por satisfacer una venganza, que la venganza es el único
goce de su alma perversa; por retener en su poder una prenda que le
permita amenazar a la más honrada y preclara casa de Andalucía, como
retienen los ladrones secuestradores la persona del rico, pidiendo a la
familia la suma del rescate. Por largo tiempo ha sido inútil toda mi
diligencia y la de los parientes de esa desgraciada joven para averiguar
el lugar donde la esconde su fementido secuestrador; pero una
casualidad, un suceso insignificante al parecer, pero que ha sido aviso
de Dios, sin duda, me ha dado a conocer que ambos están en Salamanca. Él
no habita sino en las ciudades ocupadas por los franceses, porque teme
la ira de sus paisanos, porque es un hombre maldito, traidor a su
patria, irreligioso, cruel, un mal español y un mal hijo, Jean-Jean,
que, devorado por impío rencor hacia la tierra en que nació, le hace
todo el daño que puede. Su vida tenebrosa, como la de los topos,
empléase en fundar y en propagar sociedades de masonería, en sembrar
discordias, en levantar del fondo de la sociedad la hez corrompida que
duerme en ella, en arrojar la simiente de las turbaciones de los
pueblos. Favorécenle ustedes porque favorecen todo lo que divida,
aniquile y desarme a los españoles. Él corre de pueblo en pueblo,
ocultando en sus viajes nombre, calidad y ocupación para no provocar la
ira de los naturales, y cuando no puede viajar acompañado por las tropas
francesas, se oculta con los más indignos disfraces. Últimamente ha
venido de Plasencia a Salamanca fingiéndose cómico, y su cuadrilla
imitaba tan perfectamente una compañías de la legua, que pocos en el
tránsito sospecharon el engaño\ldots{}

---Ya sé quién es---dijo súbitamente y sonriendo Jean-Jean.---Es
Santorcaz.

---El mismo, D. Luis de Santorcaz.

---A quien algunos españoles tienen por brujo, encantador y nigromante.
Y para entenderos con ese mal sujeto---añadió el francés---¿os
disfrazáis de ese modo? ¿Quién os ha dicho que Santorcaz es poderoso
entre nosotros? Lo sería en Madrid; pero no aquí. Las autoridades le
consienten, pero no le protegen. Hace tiempo que ha caído en desgracia.

---¿Le conoces bien?

---Pues ya; en Madrid éramos amigos. Le escolté cuando salió a Toledo a
conferenciar con la junta, y nos hemos reconocido después en Salamanca.
Estuvo aquí hace tres meses, y después de una ausencia corta, ha
vuelto\ldots{} Caballero marqués, o lo que seáis, para luchar contra
semejante hombre no necesitáis llevar ese vestido burdo ni disimular
vuestra nobleza; podéis hacer con él lo que mejor os convenga, incluso
matarle, sin que el gobierno francés os estorbe. Oscuro, olvidado y no
muy bien quisto, Santorcaz se consuela con la masonería, y en la logia
de la calle de Tentenecios unos cuantos perdidos españoles y franceses,
lo peor sin duda de ambas naciones, se entretienen en exterminar al
género humano, volviendo al mundo patas arriba, suprimiendo la
aristocracia y poniendo a los reyes una escoba en la mano, para que
barran las calles. Ya veis que esto es ridículo. Yo he ido varias veces
allí en vez de ir al teatro, y en verdad que no debieran disfrazarse de
cómicos porque realmente lo son.

---Veo que eres un hombre de grandísimo talento.

---Lo que soy---dijo el soldado en tono de alarmante sospecha---es un
hombre que no se mama el dedo. ¿Cómo es posible que siendo vuestro único
enemigo un hombre tan poco estimado y siendo vos marqués de tantas
campanillas, necesitéis venir aquí vendiendo verdura y engañando a todo
el pueblo, cual si no hubierais de luchar con un intrigante de baja
estofa, sino con todos nosotros, con nuestro poder, nuestra policía, y
el mismo gobernador de la plaza, el general Thiebaut-Tibo?

Jean-Jean razonaba lógicamente, y por breve rato no supe qué
contestarle.

---\emph{Connu, connu\ldots{}} Basta de farsas. Sois espía---exclamó con
acento brutal.---Si después de venir aquí como enemigo de la Francia os
burláis de mí, juro\ldots{}

---Calma, calma, amigo Jean-Jean---dije procurando esquivar el gran
peligro que me amenazaba, después que lo creí conjurado.---Ya te dije
que una aventura amorosa\ldots{} ¿No has reparado que Santorcaz lleva
consigo una joven\ldots?

---Sí, ¿y qué? Dicen que es su hija\ldots{}

---¡Su hija!---exclamé afectando una cólera frenética;---¿ese miserable
se atreve a decir que es su hija? No puede ser.

---Así lo dicen, y en verdad que se le parece bastante---repuso con
calma mi interlocutor.

---¡Oh! por Dios, amigo mío, por todos los santos, por lo que más ames
en el mundo, llévame a casa de ese hombre, y si delante de mí se atreve
a decir que Inés es su hija le arrancaré la lengua.

---Lo que puedo aseguraros es que la he visto paseando por la ciudad y
sus alrededores, dando el brazo a Santorcaz, que está muy enfermo, y la
muchacha, muy linda por cierto, no tenía modos de estar descontenta al
lado del masón, pues cariñosamente le conduce por las calles y le hace
mimos y monerías\ldots{} Y ahora, \emph{mon petit}, salís con que es
vuestra novia, y una señora encantada o \emph{princesse d' Araucaine},
según habéis dado a entender\ldots{} Bueno, ¿y qué?

---Que he venido a Salamanca para apoderarme de ella y restituirla a su
familia, empresa en la cual espero que me ayudarás.

---Si ha sido robada, ¿por qué esa familia, que es tan poderosa, no se
ha quejado al rey José?

---Porque esa familia no quiere pedir nada al rey José. Eres más
preguntón que un fiscal, y yo no puedo sufrirte más---grité sin poder
contener mi impaciencia y enojo.---¿Me sirves, sí o no?

Jean-Jean, viendo mi actitud resuelta, vaciló un momento y después me
dijo:

---¿Qué tengo que hacer? ¿Llevaros a la calle del Cáliz, donde está la
casa de Santorcaz, entrar, acogotarle y coger en brazos a la princesa
encantada?

---Eso sería muy peligroso. Yo no puedo hacer eso sin ponerme antes de
acuerdo con ella, para que prepare su evasión con prudencia y sin
escándalo. ¿Puedes tú entrar en la casa?

---No muy fácilmente, porque el señor Santorcaz tiene costumbres de
anacoreta y no gusta de visitas; pero conozco a Ramoncilla, una de las
dos criadas que le sirven, y podría introducirme en caso de gran
interés.

---Pues bien; yo escribo dos palabras, haces que lleguen a manos de la
señorita Inés, y una vez que esté prevenida\ldots{}

---Ya os entiendo, tunante---dijo con malicia de zorro y burlándose de
mí.---Queréis que me quite de vuestra presencia para escaparos.

---¿Todavía dudas de mi sinceridad? Atiende a lo que escribo con lápiz
en este papel.

Apoyando un pedazo de papel en la pared escribí lo siguiente que por
encima de mi hombro leía Jean-Jean:

«Confía en el portador de este escrito, que es un amigo mío y de tu mamá
la condesa de ***, y al cual señalarás el sitio y hora en que puedo
verte, pues habiendo venido a Salamanca decidido a salvarte, no saldré
de aquí sin ti.---\emph{Gabriel.»}

---¿Nada más que esto?---dijo tomando el papel y observándolo con la
atención profunda del anticuario que quiere descifrar una inscripción
oscura.

---Concluyamos. Tú llevas ese papel; procuras entregarlo a la señorita
Inés; y si me traes en el dorso del mismo una sola letra suya, aunque
sea trazada con la uña, te entregaré los seis doblones que llevo aquí,
dejando para recompensar servicios de más importancia, lo que guardé en
el mesón.

---¡Sí, bonito negocio!---dijo el francés con desdén.---Yo voy a la
calle del Cáliz, y en cuanto me aleje, vos que no deseáis sino perderme
de vista, echáis a correr, y\ldots{}

---Iremos juntos y te esperaré en la puerta\ldots{}

---Es lo mismo, porque si subo y os dejo fuera\ldots{}

---¡Desconfías de mí, miserable!---exclamé inflamado por la indignación,
que se mostró de un modo terrible en mi voz y en mi gesto.

---Sí, desconfío\ldots{} En fin, voy a proponeros una cosa, que me dará
garantía contra vos. Mientras voy a la calle del Cáliz, os dejaré
encerrado en paraje muy seguro, del cual es imposible escapar. Cuando
vuelva de mi comisión os sacaré y me daréis el dinero.

La ira se desbordaba en mí, mas viendo que era imposible escapar del
poder de tan vil enemigo, acepté lo que me proponía, reconociendo que
entre morir y ser encerrado durante un espacio de tiempo que no podía
ser largo; entre la denuncia como espía y una retención pasajera, la
elección no era dudosa.

---Vamos---le dije con desprecio---llévame a donde quieras.

Sin hablar más, Jean-Jean marchó a mi lado y volvimos a penetrar en
aquel laberinto de ruinas, de edificios medio demolidos y revueltos
escombros donde empezaban las fortificaciones. Vimos primero alguna
gente en nuestro camino, y después la multitud que iba y venía, y
trabajaba en los parapetos, amontonando tierra y piedras, es decir,
fabricando la guerra con los festos de la religión. Ambos silenciosos
llegamos a un pórtico vasto, que parecía ser de convento o colegio, y
nos dirigimos a un claustro, donde vi hasta dos docenas de soldados, que
tendidos por el suelo jugaban y reían con bullicio, gente feliz en medio
de aquella nacionalidad destruida, pobres jóvenes sencillos e ignorantes
de las causas que les habían movido a convertir en polvo la obra de los
siglos.

---Este es el convento de la Merced Calzada---me dijo Jean-Jean.---No se
ha podido acabar de demoler, porque había mucha faena por otro lado. En
lo que queda nos acuartelamos doscientos hombres. ¡Buen alojamiento!
Benditos sean los frailes. \emph{¡Charles le téméraire!---}gritó después
llamando a uno de los soldados que estaban en el corro.

---¿Qué hay?---dijo adelantándose un soldado pequeño y gordinflón.---¿A
quién traes contigo?

---¿Dónde está mi primo?

---Por ahí anda. \emph{¡Pied-de-mouton!}

Presentose al poco rato un sargento bastante parecido a mi acompañante
maldito, y este le dijo:

\emph{---Pied-de-mouton,} dame la llave de la torre.

\hypertarget{xviii}{%
\chapter{XVIII}\label{xviii}}

Un instante después, Jean-Jean entraba conmigo en un aposento que no era
ni oscuro ni húmedo, como suelen ser los destinados a encerrar
prisioneros.

---Permitidme, \emph{señor pequeño Marqués}---me dijo con burlona
cortesía---que os encierre aquí mientras voy a la calle del Cáliz. Si me
dais antes de partir los doblones prometidos, os dejaré libre.

Roma la chica---No---repuse con desprecio.---Para tener la recompensa
sin el servicio, necesitas matarme, vil. Inténtalo y me defenderé como
pueda.

---Pues quedaos aquí. No tardaré en volver.

Marchose, cerrando por fuera la puerta que era gruesísima. Al verme
solo, toqué los muros, cuyo espesor de dos varas anunciaba una solidez
de construcción a prueba de terremotos\ldots{} ¡Triste situación la mía!
Cerca del medio día, y antes de que pudiera adquirir todos los datos que
mi general deseaba, encontrábame prisionero, imposibilitado de recorrer
solo y a mis anchas la población. Hablando en plata, Dios no me había
favorecido gran cosa, y a tales horas, poco sabía yo, y nada había
hecho.

Senteme fatigado, alcé la cabeza para explorar lo que había encima, y vi
una escalera que, arrancando del suelo, seguía doblándose en los ángulos
y arrollándose hasta perderse en alturas que no distinguía claramente mi
vista. Los negros tramos de madera subían por el prisma interior,
articulándose en las esquinas como una culebra con coyunturas, y las
últimas vueltas perdíanse arriba en la alta región de las campanas. Una
luz vivísima, entrando por las rasgadas ventanas sin vidrios, iluminaba
aquel largo tubo vertical, en cuya parte inferior me encontraba.
Atracción poderosa llamábame hacia arriba, y subí corriendo. Más que
subir, aquella veloz carrera mía fue como si me arrojara en un pozo
vuelto del revés.

Saltando los escalones de dos en dos, llegué a un piso donde varios
aparatos destruidos me indicaron que allí había existido un reloj. Por
fuera una flecha negra que estuvo dando vueltas durante tres siglos,
señalaba con irónica inmovilidad una hora que no había de correr más.
Por todas partes pendían cuerdas; pero no había campanas. Era aquello el
cadáver de una cristiana torre, mudo e inerte como todos los cadáveres.
El reloj había cesado de latir marcando la oscilación de la vida, y las
lenguas de bronce habían sido arrancadas de aquellas gargantas de piedra
que por tanto tiempo clamaran en los espacios, saludando el alba
naciente, ensalzando al Señor en sus grandes días y pidiendo una oración
para los muertos. Seguí subiendo, y en lo más alto dos ventanas, dos
enormes ojos miraban atónitos el vasto cielo y la ciudad y el país, como
miran los espantados ojos de los muertos, sin brillo y sin luz. Al
asomarme a aquellas cavidades, lancé un grito de júbilo.

Debajo de mi vista se desarrollaba un mapa de gran parte de la ciudad y
sus contornos, su río y su campiña.

Un viento suave mugía en la bóveda de la torre solitaria, articulando en
aquel cráneo vacío sílabas misteriosas. Figurábaseme que la mole se
tambaleaba como una palmera, amenazando caer antes que las piquetas de
los franceses la destruyeran piedra a piedra. A veces me parecía que se
elevaba más, más todavía, y que la ciudad ilustre, la insigne \emph{Roma
la chica}, se desvanecía allá abajo perdiéndose entre las brumas de la
tierra. Vi otras torres, los tejados, las calles, la majestuosa masa de
las dos catedrales, multitud de iglesias de diferentes formas que habían
tenido el privilegio de sobrevivir; innumerables ruinas, donde
centenares de hombres, parecidos a hormigas que arrastran granos de
trigo, corrían y se mezclaban; vi el Tormes, que se perdía en anchas
curvas hacia Poniente, dejando a su derecha la ciudad y faldeando los
verdes campos del Zurguen por la otra orilla; vi las plataformas, las
escarpas y contra-escarpas, los rebellines, las cortinas, las troneras,
los cañones, los muros aspillerados, los parapetos hechos con columnatas
de los templos, los espaldones amasados con el polvo y la tierra que
fueron huesos y carne de venerables monjas y frailes; vi los cañones
enfilados hacia afuera, los morteros, el foso, las zanjas, los sacos de
tierra, los montones de balas, los parques al aire libre\ldots{} ¡Oh,
Dios poderoso, me diste más de lo que yo pedía! Vagaba por la ciudad
imposibilitado de cumplir con mi deber, amenazado de muerte, expuesto a
mil peligros, vendido, perdido, condenado, sin poder ver, sin poder
mirar, sin poder escuchar, sin poder adquirir idea exacta ni aun confusa
de lo que me rodeaba, hasta que un brazo de piedra, recogiéndome de
entre las ruinas del suelo, alzome en los aires para que todo lo viese.

---Bendito sea el Señor omnipotente y
misericordioso---exclamé.---Después de esto no necesito más que ojos, y
afortunadamente los tengo.

La torre de la Merced tenía suficiente elevación para observar todo
desde ella. Casi a sus pies estaba el colegio del Rey; seguía San
Cayetano; después, en dirección al ocaso, el colegio mayor de Cuenca, y
por último, los Benitos; en la elevación de enfrente vi una masa de
edificios arruinados, cuyos nombres no conocía, pero cuyas murallas se
podían determinar perfectamente, con las piezas de artillería que las
guarnecían. Volviéndome al lado opuesto, vi lo que llamaban Teso de San
Nicolás, los Mostenses, el Monte Olivete, y entre estas posiciones y
aquellas, el foso y los caminos cubiertos que bajaban al puente.

Desde la puerta de San Vicente, donde estaba el rebellín con los cuatro
cañones giratorios de que habló Molichard, partía un foso que se
enlazaba con los Milagros. En la parte anterior y superior del foso
había una línea de aspilleras sostenida por fuerte estacada. Todo el
edificio de San Vicente estaba aspillerado, y sus fuegos podían
dirigirse al interior de la ciudad y al campo. San Cayetano era
imponente. Demolido casi por completo, habían formado espacioso
terraplén con baterías de todos calibres, y sus fuegos podían barrer la
plazuela del Rey, el puente y la explanada del Hospicio.

Aunque el recelo de que mi carcelero volviese pronto me obligó a trazar
con mucha precipitación el dibujo que deseaba, este no salió mal, y en
él representé imperfectamente, pero con mucha claridad, lo mucho y bueno
que veía. Hícelo ocultándome tras el antepecho de la torre, y aunque la
proyección geométrica dejaba algo que desear como obra de ciencia, no
olvidé detalle alguno, indicando el número de cañones con precisión
escrupulosa. Terminado mi trabajo, guardelo muy cuidadosamente y bajé
hasta la entrada de la torre. Echándome sobre el primer escalón, aguardé
al r. Jean-Jean, con intento de fingir que dormía cuando él llegase.

Tardó bastante tiempo, poniéndome en cuidado y zozobra; mas al fin
apareció, y le recibí haciendo como que me despertaba de largo y sabroso
sueño. La expresión de su rostro pareciome de feliz augurio. Dios había
empezado a protegerme, y hubiera sido crueldad divina torcer mi camino
en aquella hora cuando tan fácil y transitable se presentaba delante de
mí, llevándome derechamente a la buena fortuna.

---Podéis seguirme---dijo Jean-Jean.---He visto a vuestra adorada.

---¿Y qué?---pregunté con la mayor ansiedad.

---Me parece que os ama, señor Marqués---dijo en tono de lisonja y
sonriendo con el servilismo propio de quien todo lo hace por
dinero.---Cuando le di vuestro billete, se quedó más blanca que el papel
en que lo escribisteis\ldots{} El Sr. Santorcaz, que está muy enfermo,
dormía. Yo llamé a Ramoncilla, le prometí un doblón si hacía venir a la
niña delante de mí para darle el billete; pero ¡cosa imposible! La niña
está encerrada y el amo cuando duerme, guarda la llave debajo de la
almohada\ldots{} Insistí, prometiendo dos doblones\ldots{} Entró la
muchacha, hizo señas, apareció por un ventanillo una hermosísima figura,
que alargó la mano\ldots{} Subime a un tonel\ldots{} no era bastante y
puse sobre el tonel una silla\ldots{} ¡Oh, señor Marqués! Después de
leer el papel me dijo que fueseis al momento y luego como le indicase
que necesitabais ver dos letras suyas para creerme, trazó con un pedazo
de carbón esto que aquí veis\ldots{} si he ganado bien mis seis
doblones---añadió lisonjeándome con una de esas cortesías que sólo saben
hacer los franceses,---vuecencia lo dirá.

El pícaro había cambiado por completo en gesto y modales para conmigo.
Tomé el papel y decía: \emph{«Ven al instante,»} trazado en caracteres
que reconocí al momento. Los garabatos con que los ángeles deben de
escribir en el libro de ingresos del cielo el nombre de los elegidos, no
me hubieran alegrado más.

Sin hacerme repetir la súplica indirecta, pagué a Jean-Jean.

Salimos a toda prisa de la torre, atalaya de mi espionaje, y luego del
claustro y convento arruinado; enderezando nuestros pasos por calles o
callejuelas, pasamos por delante de la catedral, y luego nos internamos
de nuevo por varias angostas vías, hasta que al fin parose Jean-Jean y
dijo:

---Aquí es. Entremos despacito, aunque sin miedo, porque nadie nos
estorba llegar hasta el patio. Ramoncilla nos dejará pasar. Después Dios
dirá.

Atravesamos el portal oscuro, y empujando una puerta divisamos un patio
estrecho y húmedo, donde se nos apareció Ramoncilla, la cual gravemente
hizo señas de que no metiésemos ruido, y luego inclinó su cabeza sobre
la palma de la mano, para indicar sin duda que el señor seguía
durmiendo. Avanzamos paso a paso, y Jean-Jean, sin abandonar su sonrisa
de lisonja, señalome una estrecha ventana que se abría en uno de los
muros del patio. Miré, pero nadie asomó por ella. Mi emoción era tan
grande que me faltaba el aliento, y dirigía con extravío los ojos a
todos lados como quien ve fantasmas.

Sentí un ruido extraño, rumor como el de las alas de un insecto cuando
surca el aire junto a nuestra cabeza, o el roce de una sutil tela con
otra. Alcé la vista y la vi, vi a Inés en la ventana, sosteniendo la
cortina con la mano izquierda y fijo en la boca el índice de la derecha
para imponerme silencio. Su semblante expresaba un temor semejante al
que nos sobrecoge cuando nos vemos al borde de un hondo precipicio sin
poder detener ya la gravitación que nos empuja hacia él. Estaba pálida
como la muerte, y el mirar de sus espantados ojos me volvía loco.

Vi una escalera a mi derecha y me precipité por ella, pero la criada y
el francés dijéronme más con signos que con palabras que subiendo por
allí no podía entrar. Moví los brazos ordenando a Inés que bajase; pero
hizo ella signos negativos que me desesperaron más.

---¿Por dónde subo?---pregunté.

La infeliz llevose ambas manos a la cabeza, lloró, y repitió su
negativa. Luego parecía quererme decir que esperase.

---Subiré---dije al francés, buscando algún objeto que disminuyese la
distancia.

Pero Jean-Jean, oficioso y solícito, como quien ha recibido seis
doblones, había ya rodado el tonel que en un ángulo del patio estaba y
puéstolo bajo la ventana. Aquel auxilio era pequeño, pues aún faltaba
gran trecho sin apoyo ni asidero alguno. Yo devoraba con los ojos la
pared, o más que pared, inaccesible montaña, cuando Jean-Jean, rápido,
diligente y risueño, subió al tonel señalándome sus robustos hombros.
Comprender su idea y utilizarla fue obra del mismo momento, y trepando
por aquella escalera de carne francesa, así con mis trémulas manos el
antepecho de la ventana. Estaba arriba.

\hypertarget{xix}{%
\chapter{XIX}\label{xix}}

Encontreme frente a Inés que me miraba, confundiendo en sus ojos la
expresión de dos sentimientos muy distintos: la alegría y el terror. No
se atrevía a hablarme; puso violentamente su mano en mi boca cuando
quise articular la primera palabra; inundó de lágrimas ardientes mi
pecho, y luego, indicándome con movimientos de inquietud que yo no podía
estar allí, me dijo:

---¿Y mi madre?

---Buena\ldots{} ¿qué digo buena?\ldots{} medio muerta por tu
ausencia\ldots{} ven al instante\ldots{} estás en mi poder\ldots{}
¿Lloras de alegría?

La estreché con vehemente cariño en mis brazos y repetí:

---¡Sígueme al momento\ldots{} pobrecita!\ldots{} Te ahogas aquí\ldots{}
tanto tiempo buscándote\ldots{} ¡Huyamos, vida y corazón mío!

La noticia de mi próxima muerte no me hubiera producido tanto dolor como
las palabras de Inés cuando, temblando en mis brazos, me dijo:

---Márchate tú. Yo no.

Separeme de ella y la miré como se mira un misterio que espanta.

---¿Y mi madre?---repitió ella.

Su voz débil y quejumbrosa apenas se oía. Resonaba tan sólo en mi alma.

---Tu madre te aguarda. ¿Ves esta carta? Es suya.

Arrebatándome la carta de las manos, la cubrió de besos y lágrimas y se
la guardó en el seno. Luego con rapidez suma se apartó de mí,
señalándome con insistencia el patio.

El espíritu que va consentido al cielo y encuentra en la puerta a San
Pedro que le dice: «Buen amigo, no es este vuestro destino; tomad por
aquella senda de la izquierda;» ese espíritu que equivoca el camino,
porque ha equivocado su suerte, no se quedará tan absorto como me quedé
yo.

En mi alma se confundían y luchaban también sentimientos diversos;
primero una inmensa alegría, después la zozobra, mas sobre todos
dominaron la rabia y el despecho, cuando vi que aquella criatura tan
amada, a quien yo quería devolver la libertad, me despedía sin que se
pudiera traslucir el motivo. ¡Era para volverse loco! ¡Encontrarla
después de tantos afanes, entrever la posibilidad de sacarla de allí
para devolverla a su angustiada madre, a la sociedad, a la vida;
recobrar el perdido tesoro del corazón, tomarlo en la mano y sentir
rechazada esta mano!\ldots{}

---¡Ahora mismo vas a salir de aquí conmigo!---dije sin bajar la voz y
estrechando tan fuertemente su brazo que, a causa del dolor, no pudo
reprimir un ligero grito.

Arrojose a mis plantas y tres veces, tres veces, señores, con acento que
heló la sangre en mis venas, repitió:

---No puedo.

---¿No me mandaste que viniera?---dije recordando el papel escrito con
carbón.

Tomó de una mesa un largo pliego escrito recientemente, y dándomelo, me
dijo:

---Toma esa carta, vete y haz lo que te digo en ella. Te veré otro día
por esta ventana.

---No quiero---grité haciendo pedazos el papel.---No me voy sin ti.

Me asomé por la ventana y vi que Jean-Jean y Ramoncilla habían
desaparecido. Inés se arrodilló de nuevo ante mí.

---¡La llave, trae pronto la llave!---dije bruscamente.---Levántate del
suelo\ldots{} ¿oyes?\ldots{}

---No puedo salir---murmuró.---Vete al momento.

Sus grandes ojos abiertos con espanto, me expulsaban de la casa.

---¡Estás loca!---exclamé.---Dime «muere,» pero no digas «vete\ldots»
Ese hombre te impide salir conmigo; tiene tanto poder sobre ti que te
hace olvidar a tu madre y a mí que soy tu hermano, tu esposo, ¡a mí que
he recorrido media España buscándote, y cien veces he pedido a Dios que
tomara mi vida en cambio de tu libertad!\ldots{} ¿Te niegas a
seguirme?\ldots{} Dime dónde está ese verdugo, porque quiero matarle; no
he venido más que a eso.

Su turbación hizo expirar las palabras en mi garganta. Estrechó
amorosamente mi mano, y con voz angustiosa que apenas se oía, me dijo:

---Si me quieres todavía, márchate.

Mi furor iba a estallar de nuevo con mayor violencia, cuando un acento
lejano, un eco que llegaba hasta nosotros debilitado por la distancia,
clamó repetidas veces:

---Inés, Inés.

Una campanilla sonó al mismo tiempo con discorde vibración.

Levantose ella despavorida, trató de componer su rostro y cabello
secando las lágrimas de sus ojos, vino hacia mí poniendo en la mirada
toda su alma para decirme que callase, que estuviese quieto, que la
obedeciese retirándome, y partió velozmente por un largo pasadizo que se
abría en el fondo de la habitación.

Sin vacilar un instante la seguí. En la oscuridad, servíanme de guía su
forma blanca que se deslizaba entre las dos negras paredes, y el ruido
de su vestido al rozar contra una y otra en la precipitada marcha. Entró
en una habitación espaciosa y bien iluminada, en donde entré también.
Era su dormitorio, y al primer golpe de vista advertí la agradable
decencia y pulcritud de aquella estancia, amueblada con arte y esmero.
El lecho, las sillas, la cómoda, las láminas, la fina estera de colores,
los jarros de flores, el tocador, todo era bonito y escogido.

Cuando puse mis pies en la alcoba, ella que iba mucho más a prisa que
yo, había pasado a otra pieza contigua por una puerta vidriera, cuya luz
cubrían cortinas blancas de indiana con ramos azules. Allí me detuve y
la vi avanzar hacia el fondo de una vasta estancia medio oscura, en cuyo
recinto resonaba la voz de Santorcaz. El rencor me hizo reconocerle en
la penumbra de la ancha cuadra, y distinguí la persona del miserable,
doloridamente recostada en un sillón con las piernas extendidas sobre un
taburete y rodeado de almohadas y cojines.

También pude ver que la forma blanca de Inés se acercaba al sillón:
durante corto rato ambos bultos estuvieron confundidos y enlazados, y
sentí el estallido de amorosos besos que imprimían los labios del hombre
sobre las mejillas de la mujer.

---Abre, abre esas maderas, que está muy oscuro el cuarto---dijo
Santorcaz---y no puedo verte bien.

Inés lo hizo así, y la copiosa y rica luz del Mediodía iluminó la
estancia. Mis ojos la escudriñaron en un segundo, observando todo,
personajes y escena. A Santorcaz con la barba crecida y casi enteramente
blanca, el rostro amarillo, hundidos los ojos de fuego, surcada de
arrugas la hermosa y vasta frente, huesosas las manos, fatigado el
aliento, no le hubiera conocido otro que yo, porque tenía grabadas en la
mente sus facciones con la claridad del rostro aborrecido. Estaba viejo,
muy viejo. La pieza contenía armas puestas en bellas panoplias, algunos
muebles antiguos de gastado entalle, muchos libros, diversos armarios,
arcones, un lecho cuyo dosel sostenían torneadas columnas, y un ancho
velador lleno de papeles en confusión revueltos.

Inés se juntó al hombre a quien por su vejez prematura puedo llamar
anciano.

---¿Por qué has tardado en venir?---dijo Santorcaz con acento dulce y
cariñoso, que me causó gran sorpresa.

---Estaba leyendo aquel libro\ldots{} aquel libro\ldots{} ya
sabes---dijo la muchacha con turbación.

El anciano tomando la mano de Inés la llevó a sus labios con inefable
amor.

---Cuando mis dolores---prosiguió---me permiten algún reposo y duermo,
hija mía, en el sueño me atormenta una pena angustiosa; me parece que te
vas y me dejas solo, que te vas huyendo de mí. Quiero llamarte y no
puedo proferir voz alguna, quiero levantarme para seguirte y mi cuerpo
convertido en estatua de hierro no me obedece\ldots{}

Callando un momento para reposar su habla fatigosa, prosiguió luego así:

---Hace un instante dormía con sueño indeciso. Me parecía que estaba
despierto. Sentí voces en la habitación que da al patio; te vi dispuesta
a huir, quise gritar; un peso horroroso, una montaña, oprimía mi
pecho\ldots{} todavía moja mi frente el sudor frío de aquella
angustia\ldots{} Al despertar eché de ver que todo era una nueva
repetición del mismo sueño que me atormenta todas las noches\ldots{} Di,
¿me abandonarás?, ¿abandonarás a este pobre enfermo, a este hombre ayer
joven, hoy anciano y casi moribundo, que te ha hecho algún daño, lo
confieso, pero que te ama, te adora como no suelen amar los hombres a
sus semejantes, sino como se adora a Dios o a los ángeles? ¿Me
abandonarás, me dejarás solo?\ldots{}

---No---dijo Inés.

Aquel monosílabo apenas llegó hasta mí.

---¿Y me perdonas el mal que te he hecho, la libertad que te he quitado?
¿Olvidas las grandezas vanas y falaces que has perdido por mí\ldots?

---Sí---contestó la muchacha.

---Pero no me amarás nunca como yo te amo. La prevención, el horror que
te inspiré en los primeros días no podrá borrarse de tu corazón, y esto
me desespera. Todos mis esfuerzos para complacerte, mi empeño en hacerte
agradable esta vida, el bienestar tranquilo que te he proporcionado,
todo es inútil\ldots{} La odiosa imagen del ladrón no te dejará ver en
mí la venerable faz del padre. ¿No estás aún convencida de que soy un
hombre bueno, honrado, leal, cariñoso, y no un monstruo abominable, como
creen algunos necios?

Inés no contestó. La observé dirigiendo inquietas miradas a los vidrios,
tras los cuales yo me ocultaba.

---Si por algo temo la muerte, es por ti---continuó el anciano.---¡Oh!
si pudiera llevarte conmigo sin quitarte la vida\ldots{} Pero ¿quién
asegura que moriré\ldots? No; mi enfermedad no es mortal. Viviré muchos
años a tu lado, mirándote y bendiciéndote, porque has llenado el vacío
de mi existencia. ¡Bendito sea el \emph{Ser Supremo!} Viviré, viviremos,
hija mía; yo te prometo que serás feliz\ldots{} ¿Pero no lo eres ahora?
¿Qué te falta\ldots? ¿No me respondes\ldots? Estás aterrada, te causo
miedo\ldots{}

El anciano calló un momento, y durante breve rato no se oyó en la
habitación más que el batir de las tenues alas de una mosca que se
sacudía contra los cristales, engañada por la transparencia de estos.

---¡Dios mío!---exclamó él con amargura.---¿Seré yo tan criminal como
dicen? ¿Lo crees tú así? Dímelo con franqueza\ldots{} ¿Me juzgas un
malvado? Hay en mi vida hechos extraños, hija mía, ya lo sabes; pero
todo se explica y se justifica en este mundo\ldots{} ¿Qué razón hay para
que te posea tu madre que durante tanto tiempo te tuvo abandonada
pudiendo recogerte, y no te posea yo, que te amo por lo menos tanto como
ella? no, que te amo más, muchísimo más, porque en la condesa pudo
siempre el orgullo más que la maternidad, y jamás te llamó hija. Te
tenía a su lado como un juguete precioso o fútil pasatiempo. Hija mía,
la holgazanería, la corrupción y la vanidad de esos grandes, tan
despreciables por su carácter, no tiene límites. Aborrece a esa gente,
convéncete de la superioridad que tienes sobre ellos por la nobleza de
tu alma; no les hagas el honor de ocupar tu entendimiento con una idea
relativa a su vil orgullo. Haz tus alegrías con sus tormentos, y espera
con deleite el día en que todos ellos caigan en el lodo. Apacienta tu
fantasía con el espectáculo de reparación y justicia de esa gran caída
que les espera, y acostúmbrate a no tener lástima de los explotadores
del linaje humano, que han hecho todo lo posible para que el pueblo
baile sobre sus cuerpos, después de muertos\ldots{} ¿Pero estás
llorando, Inés\ldots? Siempre dices que no entiendes esto. No puedo
borrar de tu alma el recuerdo de otros días\ldots{}

Inés no contestó nada.

---Ya\ldots---dijo Santorcaz con amarga ironía, después de breve
pausa.---La señorita no puede vivir sin carroza, sin palacio, sin
lacayos, sin fiestas y sin pavonearse como las cortesanas corrompidas en
los palacios de los reyes\ldots{} Un hombre del \emph{estado llano} no
puede dar esto a una señorita, y la señorita desprecia a su padre.

La voz de Santorcaz tomó un acento duro y reprensivo.

---Quizás esperes volver allá\ldots---añadió.---Quizás trames algún plan
contra mí\ldots{} ¡Ah! ingrata; si me abandonas, si tu corazón se deja
sobornar por otros amores, si menosprecias el cariño inmenso, infinito,
de este desgraciado\ldots{} Inés, dame la mano, ¿por qué lloras\ldots?
vamos, vamos, basta de gazmoñerías\ldots{} Las mujeres son mimosas y
antojadizas\ldots{} Vamos, hijita, ya sabes que no quiero lágrimas.
Inés, quiero un rostro alegre, una conformidad tranquila, un ademán
satisfecho\ldots{}

El anciano besó a su hija en la frente, y después dijo:

---Acerca una mesa, que quiero escribir.

No pudiendo contenerme más, empujé las vidrieras para penetrar en la
habitación.

\hypertarget{xx}{%
\chapter{XX}\label{xx}}

---¡Un hombre, un ladrón!---gritó Santorcaz.

---El ladrón eres tú---afirmé adelantando con resolución.

---¡Oh! Te conozco, te conozco\ldots---exclamó el anciano levantándose
no sin trabajo de su asiento y arrojando a un lado almohadas y cojines.

Inés al verme lanzó un grito agudísimo, y abrazando a su padre:

---No le hagas daño---dijo---se marchará.

---Necio---gritó él.---¿Qué buscas aquí? ¿Cómo has entrado?

---¿Qué busco? ¿Me lo preguntas, malvado?---exclamé poniendo todo mi
rencor en mis palabras.---Vengo a quitarte lo que no es tuyo. No temas
por tu miserable vida, porque no me ensañaré en ese infeliz cuerpo a
quien Dios ha dado el merecido infierno con anticipación; pero no me
provoques, ni detengas un momento más lo que no te pertenece, reptil,
porque te aplasto.

Al mirarme, los ojos de Santorcaz envenenaban y quemaban. ¡Tanta ponzoña
y tanto fuego había en ellos!

---Te esperaba\ldots---gritó.---Sirves a mis enemigos. Hijo del pueblo
que comes las sobras de la mesa de los grandes, sabe que te desprecio.
Enfermo e inválido estoy; mas no te temo. Tu vil condición y el
embrutecimiento que da la servidumbre te impulsarán a descargar sobre mí
la infame mano con que cargas la litera de los nobles. Desprecio tus
palabras. Tu lengua, que adula a los poderosos e insulta a los débiles,
sólo sirve para barrer el polvo de los palacios. Insúltame o mátame;
pero mi adorada hija, mi hija que lleva en sus venas la sangre de un
mártir del despotismo, no te seguirá fuera de aquí.

---Vamos---grité a Inés ordenándole imperiosamente que me siguiera, y
despreciando aquel gárrulo estilo revolucionario que tan en boga estaba
entonces entre afrancesados y masones.---Vamos fuera de aquí.

Inés no se movía. Parecía la estatua de la indecisión. Santorcaz, gozoso
de su triunfo, exclamó:

---¡Lacayo, lacayo! Di a tus indignos amos que no sirves para el caso.

Al oír esto, una nube de sangre cubrió mis ojos; sentí llamas ardientes
dentro de mi pecho, y abalancéme hacia aquel hombre. El rayo, al caer,
debe de sentir lo que yo sentí. Alargó su brazo para coger una pistola
que en la cercana mesa había, y al dirigirla contra mi pecho, Inés se
interpuso tan violentamente, que si dispara, hubiérala muerto sin
remedio.

---¡No le mates, padre!---gritó.

Aquel grito, el aspecto del anciano enfermo, que arrojó el arma lejos de
sí, renunciando a defenderse, me sobrecogieron de tal modo, que quedé
mudo, helado y sin movimiento.

---Dile que nos deje en paz---murmuró el enfermo abrazando a su
hija.---Sé que conoces hace tiempo a ese desgraciado.

La muchacha ocultó en el pecho del padre su rostro lleno de lágrimas.

---Joven sin corazón---me dijo Santorcaz con voz trémula.---Márchate; no
me inspiras ni odio ni afecto. Si mi hija quiere abandonarme y seguirte,
llévatela.

Clavó en su hija los ojos ardientes, apretando con su mano huesosa, no
menos dura y fuerte que una garra, el brazo de la infeliz joven:

---¿Quieres huir de mi lado y marcharte con ese mancebo?---añadió
soltándola y empujándola suavemente lejos de sí.

Di algunos pasos hacia adelante para tomar la mano de Inés.

---Vamos---le dije.---Tu madre te espera. Estás libre, querida mía, y se
acabaron para ti el encierro y los martirios de esta casa, que es un
sepulcro habitado por un loco.

---No, no puedo salir---me dijo Inés corriendo al lado del anciano, que
le echó los brazos al cuello y la besó con ternura.

---Bien, señora---dije con un despecho tal, que me sentí impulsado a no
sé qué execrables violencias.---Saldré. Nunca más me verá usted; nunca
más verá usted a su madre.

---Bien sabía yo que no eras capaz de la infamia de
abandonarme---exclamó el anciano llorando de júbilo.

Inés me lanzó una mirada encendida y profunda, en la cual sus negras
pupilas, al través de las lágrimas, dijéronme no sé qué misterios,
manifestáronme no sé qué enigmáticos pensamientos que en la turbación de
aquel instante no pude entender. Ella quiso sin duda decirme mucho; pero
yo no comprendí nada. El despecho me ahogaba.

---Gabriel---dijo el anciano recobrando la serenidad.---Aquí no haces
falta. Ya has oído que te marches. Supongo que habrás traído escala de
cuerda; mas para que bajes más seguro, toma la llave que hay sobre esa
mesa, abre la puerta que hay en el pasillo, y por la escalera que veas
baja al patio. Te ruego que dejes la llave en la puerta.

Viendo mi indecisión y perplejidad, añadió con punzante y cruel ironía:

---Si puedo serte útil en Salamanca, dímelo con franqueza. ¿Necesitas
algo? Parece que no has comido hoy, pobrecillo. Tu rostro indica
vigilias, privaciones, trabajos, hambre\ldots{} En la casa del hombre
del \emph{estado llano} no falta un pedazo de pan para los pobres que
vienen a la puerta. ¿Sucede lo mismo en casa de los nobles?

Inés me miró con tanta compasión, que yo la sentí por ella, pues no se
me ocultaba que padecía horriblemente.

---Gracias---respondí con sequedad;---no necesito nada. El pedazo de pan
que he venido a buscar no ha caído en mi mano; pero volveré por
él\ldots{} Adiós.

Y tomando la llave, salí bruscamente de la estancia, de la escalera, del
patio, de la horrible casa; pero padre, hija, estancia, patio y casa,
todo lo llevaba dentro de mí.

\hypertarget{xxi}{%
\chapter{XXI}\label{xxi}}

Cuando me encontré en la calle traté de reflexionar, para que la razón,
enfriando mi sofocante ira, iluminara un poco mi entendimiento sobre
aquel inesperado suceso; pero en mí no había más que pasión, una
irritación salvaje que me hacía estúpido. Fuera ya de la escena, lejos
ya de los personajes, traté de recordar palabra por palabra todo lo
dicho allí; traté de recordar también la expresión de las fisonomías,
para escudriñar antecedentes, indagar causas y secretos. Estos no pueden
salir desde el fondo de las almas a la superficie de los apasionados
discursos en un diálogo vivo entre personas que con ardor se aman o se
odian.

A veces sentía no haber estrangulado a aquel hombre envejecido por las
pasiones; a veces sentía hacia él inexplicable compasión. La conducta de
Inés, tan desfavorable para mi amor propio, infundíame a ratos una ira
violenta, ira de amante despreciado, y a ratos un estupor secreto con
algo de la instintiva admiración que producen las grandezas de la
Naturaleza cuando está uno cerca de ellas, cuando sabe uno que las va a
ver, pero no las ha visto todavía.

Mi cerebro estaba lleno con la anterior entrevista. Pasaba el tiempo,
pasaba yo maquinalmente de un sitio a otro, y aún los tenía a los dos
ante la vista, a ella afligida y espantada, queriendo ser buena conmigo
y con su padre; a Santorcaz furioso, irónico, díscolo e insultante
conmigo, tierno y amoroso con ella. Observando bien a Inés, ahondando en
aquel dolor suyo y en aquella su patética simpatía por la miseria
humana, no había realmente nada de nuevo. En él sí, mucho.

Yo traía el pasado y lo ponía delante; registraba toda aquella parte de
mi vida en que tuviera relación con ambos personajes. Finalmente, hice
respecto a mi propio pensar y sentir en aquella ocasión un raciocinio
que iluminó un poco mi espíritu.

---Largo tiempo, y hoy mismo al encontrarme frente a él---dije---he
considerado a ese hombre como un malvado, y no he considerado que es un
padre.

Sin duda me había acostumbrado a ver aquel asunto desde un punto de
vista que no era el más conveniente.

Así pensando y sintiendo, con el cerebro lleno, el corazón lleno,
proyectando en redor mío mi agitado interior, lo cual me hacía ver de un
modo extraño lo que me rodeaba, sin vivir más que para mí mismo,
olvidado en absoluto lo que me llevara a Salamanca, discurrí por varias
calles que no conocía.

De improviso ante mi cara apareció una cara. La vi con la indiferencia
que inspira un figurón pintado, y tardé mucho tiempo en llegar al
convencimiento de que yo conocía aquel rostro. En las grandes
abstracciones del alma, el despertar es lento y va precedido de una
serie de raciocinios en que aquella disputa con los sentidos sobre si
reconoce o no lo que tiene delante. Yo razoné al fin, y dije para mí:

---Conozco estos ojuelos de ratón que delante tengo.

Recobrando poco a poco mi facultad de percepción, hablé conmigo de este
modo:

---Yo he visto en alguna parte esta nariz insolente y esta boca infernal
que se abre hasta las orejas para reír con desvergüenza y descaro.

Dos manos pesadas cayeron sobre mis hombros.

---Déjame seguir, borracho---exclamé, empujando al importuno, que no era
otro que Tourlourou.

\emph{---¡Satané farceur!---} gritó Molichard, que acompañaba por mi
desgracia al otro.---Venid al cuartel.

\emph{---Drôle de pistolet\ldots{}} venid---dijo Tourlourou riendo
diabólicamente.---Caballero Cipérez, el coronel Desmarets os
aguarda\ldots{}

\emph{---¡Ventre de biche!\ldots{}} os escapasteis cuando ibais a ser
encerrado.

---Y sacasteis la navaja para asesinarnos.

\emph{---Monseigneur} Cipérez, \emph{vous serez coffré et niché.}

Intenté defenderme de aquellos salvajes; pero me fue imposible, pues
aunque borrachos, juntos tenían más fuerza que yo. Al mismo tiempo, como
la escena en la casa de Santorcaz embargaba de un modo lastimoso mis
facultades intelectuales, no me ocurría ardid ni artificio alguno que me
sacase de aquel nuevo conflicto, más grave sin duda que los vencidos
anteriormente.

Lleváronme, mejor dicho, arrastráronme hasta el cuartel, donde por la
mañana tuve el honor de conocer a Molichard, y en la puerta detúvose
Tourlourou, mirando al extremo de la calle.

\emph{---Dame\ldots{}}---chilló---allí viene el coronel Desmarets.

Cuando mis verdugos anunciaron la proximidad del coronel encargado de la
policía de la ciudad, encomendé mi alma a Dios, seguro de que si por
casualidad me registraban y hallaban sobre mí el plano de las
fortificaciones, no tardaría un cuarto de hora en bailar al extremo de
una cuerda, como ellos decían. Volví angustiado los ojos a todas partes,
y pregunté:

---¿No está por ahí el Sr.~Jean-Jean?

Aunque el dragón no era un santo, le consideré como la única persona
capaz de salvarme.

El coronel Desmarets se acercaba por detrás de mí. Al volverme\ldots{}
¡oh asombro de los asombros!\ldots{} le vi dando el brazo a una dama,
señores míos, a una dama que no era otra que la mismísima miss Fly, la
mismísima Athenais, la mismísima Pajarita.

Quedeme absorto, y ella al punto saludome con una sonrisa vanagloriosa
que indicaba su gran placer por la sorpresa que me causaba.

Molichard y su vil compañero adelantáronse hacia el coronel, hombre
grave y de más que mediana edad, y con todo el respeto que su
embrutecedora embriaguez les permitiera, dijéronle que yo era espía de
los ingleses.

---¡Insolentes!---exclamó con indignación y en francés miss Fly.---¿Os
atrevéis a decir que mi criado es espía? Señor coronel, no hagáis caso
de esos miserables a quienes rebosa el vino por los ojos. Este muchacho
es el que ha traído mi equipaje, y el que con vuestra ayuda he buscado
inútilmente hasta ahora por la ciudad\ldots{} Di, tonto, ¿dónde has
puesto mi maleta?

---En el mesón de la Fabiana, señora---respondí con humildad.

---Acabáramos. Buen paseo he hecho dar al señor coronel que me ha
ayudado a buscarte\ldots{} Dos horas recorriendo calles y plazas\ldots{}

---No se ha perdido nada, señora---le dijo Desmarets con
galantería.---Así habéis podido ver lo más notable de esta
interesantísima ciudad.

---Sí; pero necesitaba sacar algunos objetos de mi maleta, y este
idiota\ldots{} Es idiota, señor coronel\ldots{}

---Señora---dije señalando a mis dos crueles enemigos.---Cuando iba en
busca de su excelencia, estos borrachos me llevaron engañado a una
taberna, bebieron a mi costa, y luego que me quedé sin un real, dijeron
que yo era espía y querían ahorcarme.

Miss Fly miró al coronel con enfado y soberbia, y Desmarets, que sin
duda deseaba complacer a la bella amazona, recogió todo aquel femenino
enojo para lanzarlo militarmente sobre los dos bravos franchutes, los
cuales al verse convertidos de acusadores en acusados, parecían más
beodos que antes y más incapaces de sostenerse sobre sus vacilantes
piernas.

---¡Al cuartel, canalla!---gritó el jefe con ira.---Yo os arreglaré
dentro de un rato.

Molichard y Tourlourou, asidos del brazo, confusos y tan lastimosamente
turbados en lo moral como en lo físico, entraron en el edificio dando
traspiés, y recriminándose el uno al otro.

---Os juro que castigaré a esos pícaros---dijo el bravo
oficial.---Ahora, puesto que habéis encontrado vuestra maleta, os
conduciré a vuestro alojamiento.

---Sí, lo agradeceré---dijo miss Fly poniéndose en marcha, ordenándome
que la siguiera.

---Y luego---añadió Desmarets---daré una orden para que se os permita
visitar el hospital. Tengo idea de que no ha quedado en él ningún
oficial inglés. Los que había hace poco, sanaron y fueron canjeados por
los franceses que estaban en Fuente Aguinaldo.

---¡Oh, Dios mío! ¡Entonces habrá muerto!---exclamó con afectada pena
miss Fly.---¡Desgraciado joven! Era pariente de mi tío el vizconde de
Marley\ldots{} ¿Pero no me acompañáis al hospital?

---Señora, me es imposible. Ya sabéis que Marmont ha dado orden para que
salgamos hoy mismo de Salamanca.

---¿Evacuáis la ciudad?

---Así lo ha dispuesto el general. Estamos amenazados de un sitio
riguroso. Carecemos de víveres, y como las fortificaciones que se han
hecho son excelentes, dejamos aquí ochocientos hombres escogidos que
bastarán para defenderlas. Salimos hacia Toro para esperar a que nos
envíen refuerzos del Norte o de Madrid.

---¿Y marcháis pronto?

---Dentro de una hora. Sólo de una hora puedo disponer para serviros.

---Gracias\ldots{} Siento que no podáis ayudarme a buscar a ese valiente
joven, paisano mío, cuyo paradero se ignora y es causa de este mi
intempestivo y molesto viaje a Salamanca. Fue herido y cayó prisionero
en Arroyomolinos. Desde entonces no he sabido de él\ldots{} Dijéronme
que tal vez estaría en los hospitales franceses de esta ciudad.

---Os proporcionaré un salvo-conducto para que visitéis el hospital, y
con esto no necesitáis de mí.

---Mil gracias; creo que llegamos a mi alojamiento.

---En efecto, este es.

Estábamos en la puerta del mesón de la Lechuga, distante no más de
veinte pasos de aquel donde yo había dejado mi asno. Desmarets
despidiose de miss Fly, repitiendo sus cumplidos y caballerescos
ofrecimientos.

---Ya veis---me dijo Athenais cuando subíamos a su aposento---que
hicisteis mal en no permitir que os acompañase. Sin duda habéis pasado
mil contrariedades y conflictos. Yo, que conozco de antiguo al bravo
Desmarets, os los hubiera evitado.

---Señora de Fly, todavía no he vuelto de mi asombro, y creo que lo que
tengo delante no es la verídica y real imagen de la hermosa dama
inglesa, sino una sombra engañosa que viene a aumentar las confusiones
de este día. ¿Cómo ha venido usted a Salamanca, cómo ha podido entrar en
la ciudad, cómo se las ha compuesto para que ese viejo relamido, ese
Desmarets?\ldots{}

---Todo eso que os parece raro, es lo más natural del mundo. ¡Venir a
Salamanca! Existiendo el camino, ¿os causa sorpresa? Cuando con tanta
grosería y vulgares sentimientos me abandonasteis, resolví venir sola.
Yo soy así. Quería ver cómo os conducíais en la difícil comisión, y
esperaba poder prestaros algún servicio, aunque por vuestra ingratitud
no merecíais que me ocupara de vos.

---¡Oh! Mil gracias, señora. Al dejar a usted lo hice por evitarle los
peligros de esta expedición. Dios sabe cuánta pena me causaba sacrificar
el placer y el honor de ser acompañado por usted.

---Pues bien, señor aldeano, al llegar a las puertas de la ciudad,
acordeme del coronel Desmarets, a quien recogí del campo de batalla
después de la Albuera, curando sus heridas y salvándole la vida:
pregunté por él, salió a mi encuentro, y desde entonces no tuve
dificultad alguna ni para entrar aquí ni para buscar alojamiento. Le
dije que me traía el afán de saber el paradero de un oficial inglés,
pariente mío, perdido en Arroyomolinos y como deseaba encontraros, fingí
que uno de los criados que traía conmigo, portador de mi maleta, había
desaparecido en las puertas de la ciudad. Deseando complacerme,
Desmarets me llevó a distintos puntos. ¡Dos horas paseando!\ldots{}
Estaba desesperada\ldots{} Yo miraba a un lado y otro diciendo: «¿Dónde
estará ese bestia?\ldots{} Se habrá quedado lelo mirando los
fuertes\ldots{} Es tan bobo\ldots»

---¿Y el mozuelo que acompañaba a usted?

---Entró conmigo. ¿Os burlabais del carricoche de mistress Mitchell? Es
un gran vehículo, y tirado por el caballo que me dio Simpson, parecía el
carro de Apolo\ldots{} Veamos ahora, señor oficial, cómo habéis empleado
el tiempo, y si se ha hecho algo que justifique la confianza del señor
duque.

---Señora, llevo sobre mí un plano de las fortificaciones muy
oculto\ldots{} Además poseo innumerables noticias que han de ser muy
útiles al general en jefe. He experimentado mil contratiempos; pero al
fin, en lo relativo a mi comisión militar, todo me ha salido bien.

---¡Y lo habéis hecho sin mí!---dijo la Mariposa con despecho.

---Si tuviera tiempo de referir a usted las tragedias y comedias de que
he sido actor en pocas horas\ldots{} pero estoy tan fatigado que hasta
el habla me va faltando. Los sustos, las alegrías, las emociones, las
cóleras de este día abatirían el ánimo más esforzado y el cuerpo más
vigoroso, cuanto más el ánimo y cuerpo míos, que están el uno aturdido y
apesadumbrado, el otro, tan vacío de toda sólida sustancia, como quien
no ha comido en diez y seis horas.

---En efecto, parecéis un muerto---dijo entrando en su habitación.---Os
daré algo de comer.

---Es una felicísima idea---respondí---y pues tan milagrosamente nos
hemos juntado aquí, lo cual prueba la conformidad de nuestro destino,
conviene que nos establezcamos bajo un mismo techo. Voy a traer mi
burro, en cuyas alforjas dejé algo digno de comerse. Al instante vuelvo.
Pida usted en tanto a la mesonera lo que haya\ldots{} pero pronto,
prontito\ldots{}

Fui al mesón donde había dejado mi asno, y al entrar en la cuadra sentí
la voz del mesonero muy enfrascada en disputas con otra que reconocí por
la del venerable señor Jean-Jean.

---Muchacho---me dijo el mesonero al entrar---este señor francés se
quería llevar tu burro.

---¡Excelencia!---afirmó cortésmente aunque muy turbado Jean-Jean---no
me quería llevar la bestia\ldots{} preguntaba por vos.

Acordeme de la promesa hecha al dragón, y del ánima de la albarda,
invención mía para salir del paso.

---Jean-Jean---dije al francés---todavía necesito de ti. Hoy salen los
franceses, ¿no es verdad?

---Sí señor, pero yo me quedo. Quedamos veinte dragones para escoltar al
gobernador.

---Me alegro---dije disponiéndome a llevar el burro conmigo.---Ahora,
amigo Jean-Jean, necesito saber si el tal jefe de los masones se dispone
a salir hoy también de Salamanca. Es lo más probable.

---Lo averiguaré, señor.

---Estoy en el mesón de al lado, ¿sabes?

---La \emph{Lechuga}, sí.

---Allí te espero. Tenemos mucho que hacer hoy, amigo Jean-Jean.

---No deseo más que servir a su excelencia.

---Y yo pago bien a los que me sirven.

\hypertarget{xxii}{%
\chapter{XXII}\label{xxii}}

Miss Fly, pretextando que la criada del mesón no debía enterarse de lo
que hablábamos, me sirvió la frugal comida ella misma, lo cual, si no
era conforme a los cánones de la etiqueta inglesa, concordaba
perfectamente con las circunstancias.

---Vuestra tristeza---dijo la inglesa---me prueba que si en la comisión
militar salisteis bien, no sucede lo mismo en lo demás que habéis
emprendido.

---Así es en efecto señora---repuse---y juro a usted que mi pesadumbre y
descorazonamiento son tales que nunca he sentido cosa igual en ninguna
ocasión de mi vida.

---¿No está vuestra princesa en Salamanca?

---Está, señora---repliqué---pero de tal manera, que más valdría no
estuviese aquí ni en cien leguas a la redonda. Porque ¿de qué vale
hallarla si la encuentro\ldots?

---Encantada---dijo la inglesa, interrumpiéndome con picante
jovialidad---y convertida, como Dulcinea, en rústica y fea labradora la
que era señora finísima.

---Allá se va una cosa con otra---dije---porque si mi princesa no ha
perdido nada de la gallardía de su presencia, ni de la sin igual belleza
de su rostro, en cambio ha sufrido en su alma transformación muy grande,
porque no ha querido aceptar la libertad que yo le ofrecí, y prefiriendo
la compañía de su bárbaro carcelero, me ha puesto bonitamente en la
puerta de la calle.

---Eso tiene una explicación muy sencilla---me dijo la dama riendo con
verdadero regocijo---y es que vuestra archiduquesa prisionera ya no os
ama. ¿No habéis pensado en el inconveniente de presentaros ante ella con
ese vestido? El largo trato con su raptor le habrá inspirado amor hacia
este. No os riáis, caballero. Hay muchos casos de damas robadas por los
bandidos de Italia y Bohemia, que han concluido por enamorarse locamente
de sus secuestradores. Yo misma he conocido a una señorita inglesa que
fue robada en las inmediaciones de Roma, y al poco tiempo era esposa del
jefe de la partida. En España, donde hay ladrones tan poéticos, tan
caballerescos, que casi son los únicos caballeros del país, ha de
suceder lo mismo. Lo que me contáis, señor mío, no tiene nada de absurdo
y cuadra perfectamente con las ideas que he formado de este país.

---La grande imaginación de usted---le dije,---tal vez se equivoque al
querer encontrar ciertas cosas fuera de los libros; pero de cualquier
modo que sea, señora, lo que me pasa es bien triste\ldots{}
porque\ldots{}

---Porque amáis más a vuestra niña, desde que ella adora a ese pachá de
tres colas, a ese Fra-Diávolo, en quien me figuro ver un grandísimo
ladrón, pero hermoso como los más hermosos tipos de Calabria y
Andalucía, más valiente que el Cid, gran jinete, espadachín sublime,
algo brujo, generoso con los pobres, cruel con los ricos y malvados,
rico como el gran turco, y dueño de inmensas pedrerías que siempre le
parecen pocas para su amada. También me lo figuro como Carlos Moor, el
más poético e interesante de los salteadores de caminos.

---¡Oh! miss Fly, veo que usted ha leído mucho. Mi enemigo no es tal
como usted le pinta, es un viejo enfermo.

---Pues entonces, Sr.~Araceli---dijo Athenais con disgusto,---no tratéis
de engañarme pintando a esa joven como una persona principal, porque si
se ha aficionado al trato de un viejo enfermo, habrá sido por avaricia,
cualidad propia de costureras, doncellas de labor, cómicas u otra gente
menuda, a cuyas respetables clases creo desde ahora que pertenecerá esa
tan decantada señora que adoráis.

---No he engañado a usted respecto a la elevación de su clase. Respecto
a la afición que ha podido sentir hacia su secuestrador, no tiene nada
de vituperable, porque es su padre.

---¡Su padre!---exclamó con asombro.---Eso sí que no estaba escrito en
mis libros. ¿Y a un padre que retiene consigo a su hija le llamáis
ladrón? Eso sí que es extraño. No hay país como España para los sucesos
raros y que en todo difieren de lo que es natural y corriente en los
demás países. Explicadme eso, caballero.

---Usted cree que todos los lances de amor y de aventura han de pasar en
el mundo conforme a lo que ha leído en las novelas, en los romances, en
las obras de los grandes poetas y escritores, y no advierte que las
cosas extrañas y dramáticas suelen verse antes en la vida real que en
los libros, llenos de ficciones convencionales y que se reproducen unas
a otras. Los poetas copian de sus predecesores, los cuales copiaron de
otros más antiguos, y mientras fabrican este mundo vano, no advierten
que la naturaleza y la sociedad van creando a escondidas del público y
recatándose de la imprenta mil novedades que espantan o enamoran.

Yo hacía esfuerzos de ingenio por sostener de algún modo un coloquio en
que miss Fly con su ardoroso sentimiento poético me llevaba ventaja, y a
cada palabra mía su atrevida imaginación se inflamaba más volando en pos
de sucesos raros, desconocidos, novelescos, fuente de pasión y de
idealismo. No puedo negar que Athenais me causaba sorpresa, porque yo,
en mi ignorancia, no conocía el sentimentalismo que entonces estaba en
moda entre la gente del Norte, invadiendo literatura y sociedad de un
modo extraordinario.

---Referidme eso---me dijo con impaciencia.

Sin temor de cometer una indiscreción, conté punto por punto a mi
hermosa acompañante, todo lo que el lector sabe. Oíame tan atentamente y
con tales apariencias de agrado, que no omití ningún detalle. Algunas
veces creí distinguir en ella señales más bien de entusiasmo varonil,
que de emoción femenina, y cuando puse punto final en mi relato,
levantose y con ademán resuelto y voz animosa, hablome así:

---¿Y vivís con esa calma, caballero, y referís esos dramas de vuestra
vida como si fueran páginas de un libro que habéis leído la noche
anterior? No sois español, no tenéis en las venas ese fuego sublime que
impulsa al hombre a luchar con las imposibilidades. Os estáis ahí mano
sobre mano contemplando a una inglesa y no se os ocurre nada, no se os
ocurre entrar en esa casa, arrancar a esa infeliz mujer del poder que la
aprisiona; echar una cuerda al cuello de ese hombre para llevarle a una
casa de locos; no se os ocurre comprar una espada vieja y batiros con
medio mundo, si medio mundo se opone a vuestro deseo; romper las puertas
de la casa, pegarle fuego si es preciso; coger a la muchacha sin tratar
de persuadirla a que os siga, y llevarla donde os parezca conveniente;
matar a todos los alguaciles que os salgan al paso, y abriros camino por
entre el ejército francés si el ejército francés en masa se opone a que
salgáis de Salamanca. Confieso que os creí capaz de esto.

---Señora---repliqué con ardor---dígame usted en qué libro ha leído eso
tan bonito que acaba de decirme. Quiero leerlo también, y después
probaré si tales hazañas son posibles.

---¿En qué libro, menguado?---repuso con exaltación admirable.---En el
libro de mi corazón, en el de mi fantasía, en el de mi alma. ¿Queréis
que os enseñe algo más?

---Señora---afirmé confundido,---el alma de usted es superior a la mía.

Vamos al instante a esa casa---dijo tomando un látigo, y disponiéndose a
salir.

Miré a miss Fly con admiración; pero con una admiración que no era
enteramente seria, quiero decir que algo se reía dentro de mí.

---¿A dónde, señora, a dónde quiere usted que vayamos?

---¡Y lo pregunta!---exclamó Athenais.---Caballero, si os hubiera creído
capaz de hacerme esa pregunta que indica las indecisiones de vuestra
alma, no hubiera venido a Salamanca.

---No, si comprendo perfectamente---respondí, no queriendo aparecer
inferior a mi interlocutora.---Comprendo\ldots{} vamos\ldots{}
pues\ldots{} a hacer una barbaridad, una que sea sonada\ldots{} yo me
atrevo a ello, y aun a cosas mayores.

---Entonces\ldots{}

---Precisamente pensaba en eso. Yo no conozco el miedo.

---Ni los obstáculos, ni el peligro, ni nada. Así, así, caballero, así
se responde---gritó con acalorado y sonoro acento.

Su inflamado semblante, sus brillantes ojos, el timbre de su patética
voz, ejercían extraño poder sobre mí, y despertaban no sé qué vagas
sensaciones de grandeza, dormidas en el fondo de mi corazón, tan
dormidas que yo no creía que existiesen. Sin saber lo que hacía,
levanteme de mi asiento, gritando con ella:

---¡Vamos, vamos allá!

---¿Estáis preparado?

---Ahora recuerdo que necesito una espada\ldots{} vieja.

---O nueva\ldots{} No será malo ver a Desmarets.

---Yo no necesito de nadie, me basto y me sobro---exclamé con brío y
orgullo.

---Caballero---dijo ella con entusiasmo---eso debiera decirlo yo para
parecerme a Medea.

---Decía que no podemos entrar con Desmarets---indiqué pensando un poco
en lo positivo---porque sale hoy de Salamanca.

En aquel momento sentimos ruido en el exterior. Era el ejército francés
que salía. Los tambores atronaban la calle. Apagaba luego sus
retumbantes clamores el paso de los escuadrones de caballería, y por
último, el estrépito de las cureñas hacía retemblar las paredes cual si
las conmoviera un terremoto. Durante largo tiempo estuvieron pasando
tropas.

---Espero ser yo quien primero lleve a lord Wellington la noticia de que
los franceses han salido de Salamanca---dije en voz baja a miss Fly,
mirando el desfile desde nuestra ventana.

---Allí va Desmarets---repuso la inglesa fijando su vista en las tropas.

En efecto, pasaba a caballo Desmarets al frente de su regimiento, y
saludó a miss Fly con galantería.

---Hemos perdido un protector en la ciudad---me dijo;---pero no importa;
no lo necesitaremos.

En este momento sonaron algunos golpecitos en la puerta; abrí, y se nos
presentó el Sr.~Jean-Jean, que sombrero en mano, hizo varios arqueos y
cortesías\ldots{}

---Excelencia, la mesonera me dijo que estabais aquí, y he venido a
deciros\ldots{} ---¿Qué?

Jean-Jean miró con recelo a miss Fly; pero al punto le tranquilicé,
diciéndole:

---Puedes hablar, amigo Jean-Jean.

---Pues venía a deciros---prosiguió el soldado---que ese señor Santorcaz
saldrá de la ciudad. Como Salamanca va a ser sitiada, huyen esta noche
muchas familias, y el masón no será de los últimos, según me ha dicho
Ramoncilla. Ha salido hace un momento de su casa, sin duda para buscar
carros y caballerías.

---Entonces se nos va a escapar---dijo miss Fly con viveza.

---No saldrán---repuso---hasta después de media noche.

---Amigo Jean-Jean, quiero que me proporciones un sable y dos pistolas.

---Nada más fácil, excelencia---contestó.

---Y además una capa\ldots{} Luego que sea de noche, prepararás el
coche\ldots---No se encuentra ninguno en la ciudad.

---Abajo tenemos uno. Enganchas el caballo, que también está abajo, y lo
llevas a la puerta más próxima a la calle del Cáliz.

---Que es la de Santi-Spíritus\ldots{} Os advierto que Santorcaz ha
vuelto a su casa; le he visto acompañado de sus cinco amigotes, cinco
hombres terribles, que son capaces de cualquier cosa\ldots{}

---¡Cinco hombres!\ldots{}

---Que no permiten se juegue con ellos. Todas las noches se reúnen allí
y están bien armados.

---¿Tienes algún amigo que quiera ganarse unos cuantos doblones y que
además sea valiente, sereno y discreto?

---Mi primo \emph{Pied-de-mouton} es bueno para el caso, pero está algo
enfermo. No sé si \emph{Charles le Téméraire} querrá meterse en tales
fregados; se lo diré.

---No necesitamos de vuestros amigos---dijo miss Fly.---No queremos a
nuestro lado gente soez. Iremos enteramente solos.

---Dentro de un momento tendréis las armas---afirmó Jean-Jean.---¿Y no
me decís nada de vuestro asno?

---Te lo regalaré con albarda y todo\ldots{} mas no busques ya nada en
ella. Lo que merezcas te lo daré cuando nos hallemos sin peligro fuera
de las puertas de la ciudad.

Jean-Jean me miró con expresión sospechosa; pero, o renació pronto en su
pecho la confianza, o supo disimular su recelo, y se marchó. Cuando de
nuevo se me puso delante al anochecer y me trajo las armas, ordenele que
me esperase en la calle del Cáliz, con lo cual dimos la inglesa y yo por
terminados los preparativos de aquel estupendo y nunca visto suceso, que
verá el lector en los capítulos siguientes.

\hypertarget{xxiii}{%
\chapter{XXIII}\label{xxiii}}

Al llegar a esta parte de mi historia, oblígame a detenerme cierta duda
penosa que no puedo arrojar lejos de mí, aunque de mil maneras lo
intento. Es el caso que, a pesar de la fidelidad y veracidad de mi
memoria, que tan puntualmente conserva los hechos más remotos, dudo si
fui yo mismo quien acometió la temeridad en cuestión, apretado a ello
por el poético y voluntarioso ascendiente de una hermosa mujer inglesa,
o si habiéndolo yo soñado, creí que lo hice, como muchas veces sucede en
la vida, por no ser fácil deslindar lo soñado de lo real; o si en vez de
ser mi propia persona la que a tales empeños se lanzara, fue otro yo
quien supo interpretar los fogosos sentimientos y caballerescas ideas de
la hechicera Athenais. Ello es que, teniéndome por cuerdo hoy, como
entonces, me cuesta trabajo determinarme a afirmar que fui yo propio el
autor de tal locura, aunque todos los datos, todas las noticias y las
tradiciones todas concuerden en que no pudo ser otro. Ante la evidencia
inclino la frente y sigo contando.

Vino, pues, la noche, envolviendo en sus sombras todo el ámbito de
\emph{Roma la chica}. Salimos miss Fly y yo, y atravesando la Rúa, nos
internamos por las oscuras y torcidas calles que nos debían llevar al
lugar de nuestra misteriosa aventura. Bien pronto, ignorantes ambos de
la topografía de la ciudad, nos perdimos y marchamos al acaso,
procurando brujulearnos por los edificios que habíamos visto durante el
día; mas con la oscuridad no distinguíamos bien la forma de aquellas
moles que nos salían al paso. A lo mejor nos hallábamos detenidos por
una pared gigantesca, cuya eminencia se perdía allá en los cielos; luego
creeríase que la enorme masa se apartaba a un lado para dejarnos libre
el paso de una calleja alumbrada a lo lejos por las lamparillas de la
devoción, encendidas ante una imagen.

Seguíamos adelante creyendo encontrar el camino buscado, y tropezábamos
con un pórtico y una torre que en las sombras de la noche venían cada
cual de distinto punto y se juntaban para ponérsenos delante. Al fin
conocimos la catedral entre aquellas montañas de oscuridad que nos
cercaban. Dintinguimos perfectamente su vasta forma irregular, sus
torres, que empiezan en una edad del arte y acaban en otra, sus ojivas,
sus cresterías, su cúpula redonda, y detrás del nuevo edificio, la
catedral vieja, acurrucada junto a él como buscando abrigo. Quisimos
orientarnos allí, y tomando la dirección que creímos más conveniente,
bien pronto tropezamos con los pórticos gemelos de la Universidad, en
cuyo frontispicio las grandes cabezas de los Reyes Católicos nos
contemplaron con sus absortos ojos de piedra. Deslizándonos por un
costado del vasto edificio, nos hallamos cercados de murallas por todas
partes, sin encontrar salida.

---Esto es un laberinto, miss Fly---dije no sin mal humor;---busquemos
hacia la espalda de la catedral esa dichosa calle. Si no, pasaremos la
noche andando y desandando calles.

---¿Os apuráis por eso? Cuanto más tarde mejor.

---Señora, lord Wellington me espera mañana a las doce en Bernuy. Me
parece que he dicho bastante\ldots{} Veremos si aparece algún transeúnte
que nos indique el camino.

Pero ningún alma viviente se veía por aquellos solitarios lugares.

---¡Qué hermosa ciudad!---dijo miss Fly con arrobamiento
contemplativo.---Todo aquí respira la grandeza de una edad ilustre y
gloriosa. ¡Cuán excelsos, cuán poderosos no fueron los sentimientos que
han necesitado tanta, tantísima piedra para manifestarse! ¿Para vos no
dicen nada esas altas torres, esas largas ojivas; esos techos, esos
gigantes que alzan sus manos hacia el cielo, esas dos catedrales, la una
anciana y de rodillas, arrugada, inválida, agazapada contra el suelo y
al arrimo de su hija, la otra flamante y en pie, hermosa, inmensa,
lozana, respirando vida en su robusta mole? ¿Para vos no dicen nada esos
cien colegios y conventos, obra de la ciencia y la piedra reunidas? ¿Y
esos palacios de los grandes señores, esas paredes llenas de escudos y
rejas, indicio de soberbia y precaución? ¡Dichosa edad aquella en que el
alma ha encontrado siempre de qué alimentar su insaciable hambre! Para
las almas religiosas el monasterio, para las heroicas la guerra, para
las apasionadas el amor, más hermoso cuanto más contrariado, para todas
la galantería, los grandes afectos, los sacrificios sublimes, las
muertes gloriosas\ldots{} La sociedad vive impulsada por una sola
fuerza, la pasión\ldots{} El cálculo no se ha inventado todavía. La
pasión gobierna el mundo y en él pone su sello de fuego. El hombre lo
atropella todo por la posesión del objeto amado, o muere luchando ante
las puertas del hogar que se le cierran\ldots{} Por una mujer se
encienden guerras y dos naciones se destrozan por un beso\ldots{} La
fuerza que aparentemente impera no es el empuje brutal de los modernos,
sino un aliento poderoso, el resoplido de los dos pulmones de la
sociedad, que son el honor y el amor.

---No vendría mal el discursito---murmuré---si al fin
encontráramos\ldots{}

Cuando esto decía habíamos perdido de vista la catedral, y nos
internábamos por calles angostas y oscuras, buscando en vano la del
Cáliz. Vimos una anciana que apoyándose en un palo marchaba lentamente
arrimada a la pared, y le pregunté:

---Señora, ¿puede usted decirme dónde está la calle del Cáliz?

---¿Buscan la calle del Cáliz y están en ella?---repuso la vieja con
desabrimiento .---¿Van a la casa de los masones o a la logia de la calle
de Tentenecios? Pues sigan adelante y no mortifiquen a una pobre vieja
que no quiere nada con el demonio.

---¿Y la casa de los masones, cuál es, señora?

---Tiénela en la mano y pregunta\ldots---contestó la anciana.---Ese
portalón que está detrás de usted es la entrada de la vivienda de esos
bribones; ahí es donde cometen sus feas herejías contra la religión, ahí
donde hablan pestes de nuestros queridos reyes\ldots{} ¡Malvados! ¡Ay,
con cuánto gusto iría a la Plaza Mayor para veros quemar! Dios querrá
quitarnos de en medio a los franceses que tales suciedades
consienten\ldots{} Masones y franceses todos son unos, la pata derecha y
la izquierda de Satanás.

Marchose la vieja hablando consigo misma, y al quedarnos solos reconocí
en el portalón que cerca teníamos la casa de Santorcaz.

---¡Cuántas veces habremos pasado por aquí sin conocer la casa!---dijo
miss Fly.---Si yo la hubiese visto una sola vez\ldots{} Pero parece que
sois torpe, Araceli.

La puerta era un antiquísimo arco bizantino, compuesto por seis u ocho
curvas concéntricas, por donde corrían misteriosas formas vegetales,
gastadas por el tiempo, cascabeles y entrelazadas cintas; y en la
imposta unos diablillos, monos o no sé qué desvergonzados animales que
hacían cabriolas confundiendo sus piernecillas enjutas con los tallos de
la hojarasca de piedra. Letras ininteligibles y que sin duda expresaban
la época de la construcción, dejaban ver sus trazos grotescos y
torcidos, como si un dedo vacilante las trazara al modo de conjuro.
Estaba reforzada la puerta con garabatos de hierro tan mohosos como
apolilladas y rotas las mal juntas tablas, y un grueso llamador en
figura de culebrón enroscado pendía en el centro, aguardando una
impaciente mano que lo moviese.

Yo interrogué a miss Fly con la mirada, vi que acercaba su mano al
aldabón.

---¿Ya, señora?---dije deteniendo su movimiento.

---¿Pues a qué esperáis?

---Conviene explorar primero al enemigo\ldots{} La casa es
sólida\ldots{} Jean-Jean dijo que había dentro\ldots{} ¿cuántos hombres?

---Cincuenta, si no recuerdo mal\ldots{} pero aunque sean mil\ldots---Es
verdad, aunque sea un millón.

Vimos que se acercaba un hombre, y al punto reconocí a Jean-Jean.

---Vienen refuerzos, señora---dije.---Verá usted qué pronto despacho.

Miss Fly, asiendo el aldabón, dio un golpe.

Yo toqué mis armas, y al ver que no se me habían olvidado, no pude
evitar un sentimiento que no sé si era burla o admiración de mí mismo,
porque a la verdad, señores, lo que yo iba a hacer, lo que yo intentaba
en aquel momento, o era una tontería o una acción semejante a aquellas
perpetuadas en romances y libros de caballería. Yo recordaba haber leído
en alguna parte que un desvalido amante llega bonitamente y sin más
ayuda que el valor de su brazo, o la protección de tal o cual potencia
nigromántica, a las puertas de un castillo donde el más barbudo y zafio
moro o gigante de aquellos agrestes confines, tiene encerrada a la más
delicada doncella, princesa o emperatriz que ha peinado hebras de oro y
llorado líquidos diamantes, y el tal desvalido amante grita desde abajo:
«Fiero arráez, o bárbaro sultán, vengo a arrancarte esa real persona que
aprisionada guardas, y te conjuro que me la des al instante si no
quieres que tu cuerpo sea partido en dos pedazos por esta mi espada; y
no te rías ni me amenaces, porque aunque tuvieras más ejércitos que
llevó el partho a la conquista de la Grecia, ni uno solo de los tuyos
quedará vivo.»

Así, señores, así, ni más o menos, era lo que yo iba a emprender. Cuando
toqué las pistolas del cinto, y el tahalí de que pendía la tajante
espada y me eché el embozo a la capa, y el ala del ancho sombrero sobre
la ceja, confieso que entre los sentimientos que luchaban en mi corazón
predominó la burla, y me reí en la oscuridad. Tenía yo un aire de
personaje de valentías, guapezas y gatuperios, que habría puesto miedo
en el ánimo más valeroso, cuando no mofa y risa; pero miss Fly había
leído sin duda las hazañas de D. Rodulfo de Pedrajas, de Pedro Cadenas,
Lampuga, Gardoncha y Perotudo, y mi catadura le había de parecer más
propia para enamorar que para reír.

Viendo que no respondían, cogí el aldabón y repetí los golpes.

Yo no medía la extensión del peligro que iba a afrontar, ni era posible
reflexionar en ello, aunque habría bastado un destello de luz de mi
razón para esclarecerme el horrible jaleo en que me iba a meter\ldots{}
Yo no pensaba en esto, porque sentía el inexplicable deleite que tiene
para la juventud enamorada todo lo que es misterioso y desconocido, más
bello y atractivo cuanto más peligroso; porque sentía dentro de mí un
deseo de acometer cualquier brutalidad sin nombre, que pusiese mi fuerza
y mi valor al servicio de la persona a quien más amaba en el mundo.

No se olvide que aún me duraba el despecho y la sofocación de la mañana.
El recuerdo de las escenas que antes he descrito completaba mi ceguera;
y realizar por la violencia lo que no pude conseguir por otro medio, era
sin duda gran atractivo para mi excitado espíritu. En la calle me
aguijoneaba la fantasía, y desde dentro me llamaba el corazón, toda mi
vida pasada y cuanto pudiese soñar para el porvenir\ldots{} ¿Quién no
rompe una pared, aunque sea con la cabeza, cuando le impulsan a ello dos
mujeres, una desde dentro y otra desde fuera?

No debo negar que la hermosa inglesa había adquirido gran ascendiente
sobre mí. No puedo expresar aquel dominio suyo y aquella esclavitud mía,
sino empleando una palabra muy usada en las novelas, y que ignoro si
indicará de un modo claro mi idea; pero no teniendo a mano otro vocablo,
la emplearé. Miss Fly me fascinaba. Aquella grandeza de espíritu, aquel
sentimiento alambicado y sin mezcla de egoísmo que había en sus
palabras; aquel carácter que atesoraba, tras una extravagancia sin
ejemplo, todo el material, digámoslo así, de las grandes acciones,
hallaban secreta simpatía en un rincón de mi ser. Me reía de ella y la
admiraba; parecíanme disparates sus consejos y los obedecía. Aquella
inmensidad de su pensamiento tan distante de la realidad me seducía, y
antes que confesarme cobarde para seguir el vuelo de su voluntad
poderosa, hubiérame muerto de vergüenza.

Repetí con más fuerza los golpes, y nada se oía en el interior de la
casa. Oscuridad y silencio como el de los sepulcros reinaban en ella. El
animalejo, lagarto, o culebrón que figuraba la aldaba, alzó (al menos
así parecía) su cabeza llena de herrumbre y clavando en mí los verdes
ojuelos, abrió la horrible boca para reírse.

---No quieren abrir---me dijo Jean-Jean.---Sin embargo, dentro están:
los he visto entrar\ldots{} Son los principales afrancesados que hay en
la ciudad, más masones que el gran Copto, y más ateos que Judas\ldots{}
Mala gente. Mi opinión, señor marqués, es que os marchéis. El coche os
aguarda en la puerta de Santi-Spíritus.

---¿Tienes miedo, Jean-Jean?

---Además, señor marqués---continuó este,---debo advertiros que pronto
ha de pasar por aquí la ronda\ldots{} Vos y la señora tenéis todo el
aspecto de gente sospechosa\ldots{} Todavía hay quien cree que sois
espía y la señora también.

---¿Yo espía?---dijo miss Fly con desprecio.---Soy una dama inglesa.

---Márchate tú, Jean-Jean, si tienes miedo.

---Hacéis una locura, caballero---repuso el dragón.---Esos hombres van a
salir y a todos nos molerán a palos.

Creí sentir el ruido de las maderas de una ventanilla que se abría en lo
alto, y grité:

---¡Ah de la casa! Abrid pronto.

---Es una locura, señor marqués---dijo el dragón bruscamente.---Vámonos
de aquí\ldots{} Entonces noté en el semblante hosco y sombrío de
Jean-Jean una alteración muy visible que no era ciertamente la que
produce el miedo.

---Repito que os dejo solo, señor marqués\ldots{} La ronda va a
venir\ldots{} Vamos hacia Santi-Spíritus, o no respondo de vos.

Su insistencia y el empeño de llevarnos hacia las afueras de la ciudad,
infundió en mí terrible sospecha.

Miss Fly redobló los martillazos, diciendo:

---Será preciso echar la puerta abajo, si no abren.

Los garabatos de hierro que reforzaban la puerta, se contrajeron,
haciendo muecas horribles, signos burlescos, figurando no sé si extrañas
sonrisas o mohínes o visajes de misteriosos rostros.

Yo empezaba a perder la paciencia y la serenidad. Jean-Jean me causaba
inquietud y temí una alevosía, no por la sospecha de espionaje, como él
había dicho, sino por la tentación de robarnos. El caso no era nuevo, y
los soldados que guarnecían las poblaciones del pobre país conquistado,
cometían impunemente todo linaje de excesos. Además, la aventura iba
tomando carácter grotesco, pues nadie respondía a nuestros golpes ni
asomaba rostro humano en la alta reja.

---Sin duda no hay aquí rastro de gente. Los masones se han marchado y
ese tunante nos ha traído aquí para expoliarnos a sus anchas.

De pronto vi que alguien aparecía en el recodo que hace la calle. Eran
dos personas que se fijaron allí como en acecho. Dirigime hacia el
dragón; pero este sin esperar a que le hablase, nos abandonó súbitamente
para unirse a los otros.

---Ese miserable nos ha vendido---exclamé rugiendo de cólera.---¡Señora,
estamos perdidos! No contábamos con la traición.

---¡La traición!---dijo confusa miss Fly.---No puede ser.

No tuvimos tiempo de razonar, porque los dos que nos observaban y
Jean-Jean se nos vinieron encima.

---¿Qué hacéis aquí?---me preguntó uno de ellos, que era soldado de
artillería sin distintivo alguno.

---No tengo que darte cuenta---respondí.---Deja libre la calle.

---¿Es ésta la tarasca inglesa?---dijo el otro dirigiéndose a miss Fly
con insolencia.

---¡Tunante!---grité desenvainando.---Voy a enseñarte cómo se habla con
las señoras.

---El marquesito ha sacado el asador---dijo el primero.---Jóvenes, venid
al cuerpo de guardia con nosotros, y vos, \emph{milady sauterelle}, dad
el brazo a \emph{Charles le Téméraire} para que os conduzca al palacio
del cepo.

---Araceli---me dijo miss Fly,---toma mi látigo y échalos de aquí.

\emph{---Pied-de-mouton}, atraviésalo---vociferó el artillero.

\emph{Pied-de-mouton} como sargento de dragones, iba armado de sable.
\emph{Carlos el Temerario} era artillero y llevaba un machete corto,
arma de escaso valor en aquella ocasión. En un momento rapidísimo,
mientras Jean-Jean vacilaba entre dirigirse a la inglesa o a mí,
acuchillé a \emph{Pied-de-mouton} con tan buena suerte, con tanto ímpetu
y tanta seguridad, que le tendí en el suelo. Lanzando un ronco aullido
cayó bañado en sangre\ldots{} Me arrimé a la pared para tener guardadas
las espaldas y esperé a Jean-Jean que, al ver la caída de su compañero,
se apartó de miss Fly, mientras Carlos el Temerario se inclinaba a
reconocer el herido. Rápida como el pensamiento, Athenais se bajó a
recoger el sable de este. Sin esperar a que Jean-Jean me atacase y
viéndole algo desconcertado, fuime sobre él; mas sobrecogido dio algunos
pasos hacia atrás, bramando así:

\emph{---¡Corne du Diable!} \emph{¡Mille millions de bombardes!\ldots{}}
¿Creéis que os tengo miedo?

Diciéndolo apretó a correr a lo largo de la calle, y más ligero que el
viento le siguió Carlos. Ambos gritaban:

---¡A la guardia, a la guardia!

---Cerca hay un grupo de guardia, señora. Huyamos. Aquí dio fin el
romance.

Corrimos en dirección contraria a la que ellos tomaron, mas no habíamos
andado siete pasos, cuando sentimos a lo lejos pisadas de gente y
distinguimos un pelotón de soldados que a toda prisa venía hacia
nosotros.

---Nos cortan la retirada, señora---dije retrocediendo.---Vamos por otro
lado.

Buscamos una boca-calle que nos permitiera tomar otra dirección y no la
encontramos. La patrulla se acercaba. Corrimos al otro extremo, y sentí
la voz de nuestros dos enemigos, gritando siempre:

---¡A la guardia!\ldots{}

---Nos cogerán---dijo miss Fly con serenidad incomparable, que me
inspiró aliento.---No importa. Entreguémonos.

En aquel instante, como pasáramos junto al pórtico en cuyo aldabón
habíamos martillado inútilmente, vi que la puerta se abría y asomaba por
ella la cabeza de un curioso, que sin duda no había podido dominar su
anhelo de saber lo que resultaba de la pendencia\ldots{} El cielo se
abría delante de nosotros. La patrulla estaba cerca, pero como la calle
describía un ángulo muy pronunciado, los soldados que la formaban no
podían vernos. Empujé aquella puerta y al hombre, que curiosamente y con
irónica sonrisa en el rostro se asomaba; y aunque ni una ni otro
quisieron ceder al principio, hice tanta fuerza, que bien pronto miss
Fly y yo nos encontramos dentro, y con presteza increíble corrí los
pesados cerrojos.

\hypertarget{xiv-1}{%
\chapter{XIV}\label{xiv-1}}

---¿Qué hace usted?---preguntó con estupor un hombre a quien vi delante
de mí, y que alumbraba el angosto portal con su linterna.

---Salvarme y salvar a esta señora---respondí atendiendo a los pasos que
un rato después de nuestra entrada sonaban en la calle, fuera de la
puerta.---La patrulla se detiene\ldots{}

---Ahora examina el cuerpo\ldots{}

---No nos han visto entrar\ldots{}

---Pero, o yo estoy tonto, o es Araceli el que tengo delante---dijo
aquel hombre, el cual no era otro que Santorcaz.

---El mismo, Sr.~D. Luis. Si su intento es denunciarme, puede hacerlo
entregándome a la patrulla; pero ponga usted en lugar seguro a esta
señora hasta que pueda salir libremente de Salamanca\ldots{} Todavía
están ahí---añadí con la mayor agitación.---¡Cómo gruñen!\ldots{} parece
que recogen el cuerpo\ldots{} ¿Estará muerto o tan sólo herido?\ldots{}

---Se marchan---dijo Athenais.---No nos han visto entrar\ldots{} Creerán
que ha sido una pendencia entre soldados, y mientras aquellos pícaros no
expliquen\ldots{}

---Adelante, señores---dijo Santorcaz con petulancia.---El primer deber
del hijo del pueblo es la hospitalidad, y su hogar recibe a cuantos han
menester el amparo de sus semejantes. Señora, nada tema usted.

---¿Y quién os ha dicho que yo temo algo?---dijo con arrogancia miss
Fly.

---Araceli, ¿eres tú quien me echaba la puerta abajo hace un momento?

Vacilé un instante en contestar, y ya tenía la palabra en la boca,
cuando miss Fly se anticipó diciendo:

---Era yo.

Santorcaz después de hacer una cortesía a la dama inglesa, permaneció
mudo y quieto, esperando oír los motivos que había tenido la señora para
llamar tan reciamente.

---¿Por qué me miráis con la boca abierta?---dijo bruscamente miss
Fly.---Seguid y alumbrad.

Santorcaz me miró con asombro. ¿Quién le causaría más sorpresa, yo o
ella? A mi vez yo no podía menos de sentirla también, y grande, al ver
que el jefe de los masones nos recibía con urbanidad. Subimos lentamente
la escalera. Desde esta oíanse ruidosas voces de hombres en lo interior
de la casa. Cuando llegamos a una habitación desnuda y oscura, que
alumbró débilmente la linterna de Santorcaz, este nos dijo:

---¿Ahora podré saber qué buscan ustedes en mi casa?

---Hemos entrado aquí buscando refugio contra unos malvados que querían
asesinarnos. Mi deseo es que oculte usted a esta señora si por acaso
insistieran en perseguirla dentro de la casa.

---¿Y a ti?---me preguntó con sorna.

---Yo estimo mi vida---repuse---y no quisiera caer en manos de
Jean-Jean; pero nada pido a usted, y ahora mismo saldré a la calle, si
me promete poner en seguridad a esta señora.

---Yo no abandono a los amigos---dijo Santorcaz con aquella sandunga y
marrullería que le eran habituales.---La dama y su galán pueden respirar
tranquilos. Nadie les molestará.

Miss Fly se había sentado en un incómodo sillón de vaqueta, único mueble
que en la destartalada estancia había, y sin atender a nuestro diálogo,
miraba los dos o tres cuadros apolillados que pendían de las paredes,
cuando entró la criada trayendo una luz.

---¿Es esta vuestra hija?---preguntó vivamente la inglesa clavando los
ojos en la moza.

---Es Ramoncilla, mi criada---repuso Santorcaz.

---Deseo ardientemente ver a vuestra hija, caballero---dijo la
inglesa.---Tiene fama de muy hermosa.

---Después de lo presente---dijo el masón con galantería---no creo que
haya otra más hermosa\ldots{} Pero volviendo a nuestro asunto, señora,
si usted y su esposo desean\ldots{}

---Este caballero no es mi esposo---afirmó miss Fly sin mirar a
Santorcaz.

---Bien: quise decir su amigo.

---No es tampoco mi amigo, es mi criado---dijo la dama con enojo.---Sois
en verdad impertinente.

Santorcaz me miró, y en su mirada conocí que no daba fe a la afirmación
de la dama.

---Bien\ldots{} ¿Usted y su criado piensan permanecer en
Salamanca?\ldots{}

---No, precisamente lo que queremos es salir sin que nadie nos moleste.
No puedo realizar el objeto que me trajo a Salamanca y me marcho\ldots{}

---Pues a entrambos sacaré de la ciudad antes del día---dijo
Santorcaz---porque estoy preparándolo todo para salir a la madrugada.

---¿Y lleváis a vuestra hija?---preguntó con gran interés miss Fly.

---Mi hija me ama tanto---respondió el masón con orgullo---que nunca se
separa de mí.

---¿Y a dónde vais ahora?

---A Francia. No pienso volver a poner los pies en España.

---Mal patriota sois\ldots{}

---Señora\ldots{} dígame usted su tratamiento para designarle con él.
Aunque hijo del pueblo y defensor de la igualdad, sé respetar las
jerarquías que establecieran la monarquía y la historia.

---Decidme simplemente \emph{señora}, y basta.

---Bien, puesto que la señora quiere conocer a mi hija, se la voy a
mostrar---dijo Santorcaz.---Dígnese la señora seguirme.

Seguímosle, y nos llevó a una sala, compuesta con más decoro que la que
dejábamos e iluminada por un velón de cuatro mecheros. Ofreció el
anciano un asiento a la inglesa, y luego desapareció volviendo al poco
rato con su hija de la mano. Cuando la infeliz me vio, quedose pálida
como la muerte, y no pudo reprimir un grito de asombro que por su
intensidad, parecía de miedo.

---Hija mía, esta es la señora que acaba de llegar a casa pidiéndome
hospitalidad para ella y para el mancebo que la acompaña.

Inés estaba como quien ve fantasmas. Tan pronto miraba a miss Fly como a
mí, sin convencerse de que eran reales y tangibles las personas que
tenía delante. Yo sonreía tratando de disipar su confusión con el
lenguaje de los ojos y las facciones; pero la pobre muchacha estaba cada
vez más absorta.

---Sí que es hermosa---dijo miss Fly con gravedad.---Pero no quitáis los
ojos de este joven que me acompaña. Sin duda le encontráis parecido a
otro que conocéis. Hija mía, es el mismo que pensáis, el mismo.

---Sólo que este perillán---dijo Santorcaz sacudiéndome el brazo con
familiaridad impertinente---ha cambiado tanto\ldots{} Cuando era oficial
se le podía mirar; pero después que ha sido expulsado del ejército por
su cobardía y mal comportamiento y puéstose a servir\ldots{} Tan grosera
burla no merecía que la contestase, y callé, dejando que Inés se
confundiese más.

---Caballero---dijo miss Fly con enojo volviéndose hacia Santorcaz---si
hubiera sabido que pensabais insultar a la persona que me acompaña,
habría preferido quedarme en la calle. Dije que era mi criado; pero no
es cierto. Este caballero es mi amigo.

---Su amigo---añadió D. Luis.---Justo, eso decía yo.

---Amigo leal y caballero intachable, a quien agradeceré toda la vida el
servicio que me ha prestado esta noche exponiendo su vida por mí.

Nueva confusión de Inés. Mudaba de color su alterado semblante a cada
segundo, y todo se le volvía mirar a la inglesa y a mí, como si
mirándonos, leyéndonos, devorándonos con la vista, pudiera aclarar el
misteriosísimo enigma que tenía delante.

La venganza es un placer criminal, pero tan deleitoso que en ciertas
ocasiones es preciso ser santo o arcángel para sofocar esta partícula,
para extinguir esta pavesa de infierno que existe en nuestro corazón.
Así es que sintiendo yo en mí la quemadura de aquel diabólico fuego del
alma que nos induce a mortificar alguna vez a las personas que más
amamos, dije con gravedad:

---Señora mía, no merecen agradecimiento acciones comunes que son un
deber para todas las personas de honor. Además, si se trata de
agradecer, ¿qué podría decir yo, al recordar las atenciones que de usted
he merecido en el cuartel general aliado, y antes de que viniésemos
ambos a Salamanca?

Miss Fly pareció muy regocijada de estas palabras mías, y en su mirada
resplandeció una satisfacción que no se cuidaba de disimular. Inés
observaba a la inglesa, queriendo leer en su rostro lo que no había
dicho.

---Señor Santorcaz---dijo la Mosquita después de una pausa---¿no pensáis
en casar a vuestra hija?

---Señora, mi hija parece hasta hoy muy contenta de su estado y de la
compañía de su padre. Sin embargo, con el tiempo\ldots{} No se casará
con un noble; ni con un militar, porque ella y yo aborrecemos a esos
verdugos y carniceros del pueblo.

---Podemos darnos por ofendidos con lo que decís contra dos clases tan
respetables---repuso con benevolencia miss Fly.---Yo soy noble y el
señor es militar. Con que\ldots{}

---He hablado en términos generales, señora. Por lo demás, mi hija no
quiere casarse.

---Es imposible que siendo tan linda no tenga los pretendientes a
millares---dijo miss Fly mirándola.---¿Será posible que esta hermosa
niña no ame a nadie?

Inés en aquel instante no podía disimular su enojo.

---Ni ama ni ha amado jamás a nadie---contestó oficiosamente su padre.

---Eso no, Sr.~Santorcaz---dijo la inglesa.---No tratéis de engañarme,
porque conozco de la cruz a la fecha la historia de vuestra adorada
niña, hasta que os apoderasteis de ella en Cifuentes.

Inés se puso roja como una cereza, y me miró no sé si con desprecio o
con terror. Yo callaba, y midiendo por mi propia emoción la suya, decía
para mí con la mayor inocencia: «La pobrecita será capaz de enfadarse.»

---Tonterías y mimos de la infancia---dijo Santorcaz, a quien había
sabido muy mal lo que acababa de oír.

---Eso es---añadió la inglesa señalando sucesivamente a Inés y a
mí.---Ambos son ya personas formales, y sus ideas así como sus
sentimientos han tomando camino más derecho. No conozco el carácter y
los pensamientos de vuestra encantadora hija; pero conozco el grande
espíritu, el noble entendimiento del joven que nos escucha, y puedo
aseguraros que leo en su alma como en un libro.

Inés no cabía en sí misma. El alma se le salía por los ojos en forma de
aflicción, de despecho, de no sé qué sentimiento poderoso, hasta
entonces desconocido para ella.

---Hace algún tiempo---añadió la inglesa---que nos une una noble, franca
y pura amistad. Este caballero posee un espíritu elevado. Su corazón,
superior a los sentimientos mezquinos de la vida ordinaria, arde en el
deseo fogoso de una vida grandiosa, de lucha, de peligro, y no quiere
asociar su existencia a la menguada medianía de un hogar pacífico, sino
lanzarla a los tumultos de la guerra, de la sociedad, donde hallará
pareja digna de su alma inmensa.

No pude reprimir una sonrisa; pero nadie, felizmente, a no ser Inés que
me observaba, advirtió mi indiscreción.

---¿Qué decís a esto?---preguntó Athenais a mi novia.

---Que me parece muy bien---contestó allá como Dios le dio a entender,
entre atrevida y balbuciente.---Cuando se tiene un alma de tal
inmensidad, parece propio afrontar los peligros de una patrulla, en vez
de llamar a la primera puerta que se presenta.

---Ya comprenderá usted, señora---dijo don Luis---que mi hija no es
tonta.

---Sí; pero lo sois vos---contestó desabridamente miss Fly.

Y diciéndolo, en la casa retumbaron aldabonazos tan fuertes como los que
nosotros habíamos dado poco antes.

---¡La patrulla!---exclamé.

---Sin duda---dijo Santorcaz.---Pero no haya temor. He prometido ocultar
a ustedes. Si manda la patrulla Cerizy, que es amigo mío, no hay nada
que temer. Inés, esconde a la señora en el cuarto de los libros, que yo
archivaré a este sujeto en otro lado.

Mientras Inés y miss Fly desaparecieron por una puerta excusada, dejeme
conducir por mi antiguo amigo, el cual me llevó a la habitación donde
por la mañana le había visto, y en la cual estaban aquella noche y en
aquella ocasión cinco hombres sentados alrededor de la ancha mesa. Vi
sobre esta libros, botellas y papeles en desorden, y bien podía decirse
que las tres clases de objetos ocupaban igualmente a todos. Leían,
escribían y echaban buenos tragos, sin dejar de charlar y reír. Observé
además que en la estancia había armas de todas clases.

---Otra vez te atruenan la casa a aldabonazos, papá Santorcaz---dijo, al
vernos entrar, el más joven, animado y vivaracho de los presentes.

---Es la ronda---respondió el masón.---A ver dónde escondemos a este
joven. Monsalud, ¿sabes quién manda la ronda esta noche?

---Cerizy---contestó el interpelado, que era un joven alto, flaco y
moreno, bastante parecido a una araña.

---Entonces no hay cuidado---me dijo.---Puedes entrar en esta habitación
y esconderte allí, por si acaso quiere subir a beber una copa.

Escondido, mas no encerrado, en la habitación que me designara,
permanecí algún tiempo, el necesario para que Santorcaz bajase a la
puerta, y por breves momentos conferenciase con los de la ronda, y para
que el jefe de esta subiese a honrar las botellas que galantemente le
ofrecían.

---Señores---exclamó el oficial francés entrando con Santorcaz---buenas
noches\ldots{} ¿Se trabaja? Buena vida es esta.

---Cerizy---replicó el llamado Monsalud llenando una copa,---a la salud
de Francia y España reunidas.

---A la salud del gran imperio galo-hispano---dijo Cerizy alzando la
copa.---A la salud de los buenos españoles.

---¿Qué noticias, amigo Cerizy?---preguntó otro de los presentes, viejo,
ceñudo y feo.

---Que el lord está cerca\ldots{} pero nos defenderemos bien. ¿Han visto
ustedes las fortifícaciones?\ldots{} Ellos no tienen artillería de
sitio\ldots{} El ejército aliado es un ejército \emph{pour rire\ldots{}}

---¡Pobrecitos!---exclamó el viejo, cuyo nombre era Bartolomé
Canencia.---Cuando uno piensa que van a morir tantos hombres\ldots{} que
se va a derramar tanta sangre\ldots{}

---Señor filósofo---indicó el francés---porque ellos lo quieren\ldots{}
Convenced a los españoles de que deben someterse\ldots{}

---Descanse usted un momento, amigo Cerizy.

---No puedo detenerme\ldots{} Han herido a un sargento de dragones en
esta calle\ldots{}

---Alguna disputa\ldots{}

---No se sabe\ldots{} los asesinos han huido\ldots{} Dicen que son
espías.

---¡Espías de los ingleses!\ldots{} Si Salamanca está llena de espías.

---Han dicho que un español y una inglesa\ldots{} o no sé si un inglés
acompañado de una española\ldots{} Pero no puedo detenerme. Se me mandó
registrar las casas\ldots{} Decidme: ¿no hay logia esta noche?

---¿Logia? Si nos marchamos\ldots{}

---¿Se marchan?---dijo el francés.---Y yo que estaba concluyendo a toda
prisa mi \emph{Memoria sobre las distintas formas de la tiranía.}

---Léasela usted a sí propio---indicó el filósofo Canencia.---Lo mismo
me pasará a mí con mi \emph{Tratado de la libertad individual} y mi
traducción de Diderot.

---¿Y por qué es esa marcha?

---Porque los ingleses entrarán en Salamanca---dijo Santorcaz---y no
queremos que nos cojan aquí.

---Yo no daría dos cuartos por lo que me quedara de pescuezo después de
entrar los aliados---advirtió el más joven y más vivaracho de todos.

---Los ingleses no entrarán en Salamanca, señores---afirmó con
petulancia el oficial.

Santorcaz movió la cabeza con triste expresión dubitativa.

---Y pues así echan ustedes a correr, desde que nos hallamos
comprometidos, Sr. Santorcaz---añadió Cerizy con la misma petulancia y
cierto tonillo reprensivo,---sepan que en el cuartel general de Marmont
no estarán los masones tan seguros como aquí.

---¿Que no?

---No: porque no son del agrado del general en jefe que nunca fue
aficionado a sociedades secretas. Las ha tolerado porque era preciso
alentar a los españoles que no seguían la causa insurgente; pero ya sabe
usted que Marmont es algo \emph{bigot}.

---Sí\ldots{}

---Pero lo que no sabe usted es que han venido órdenes apremiantes de
Madrid para separar la causa francesa de todo lo que trascienda a
masonería, ateísmo, irreligiosidad y filosofía.

---Lo esperaba, porque José es también algo\ldots{}

\emph{---Bigot\ldots{}} Con que buen viaje y no fiar mucho del general
en jefe.

---Como no pienso parar hasta Francia, mi querido señor
Cerizy\ldots---dijo Santorcaz---estoy sin cuidado.

---No se puede vivir en esta abominable nación---afirmó el viejo
filósofo.---En París o en Burdeos publicaré mi \emph{Tratado de la
libertad individual} y mi traducción de Diderot.

---Buenas noches, señor Santorcaz, señores todos.

---Buenas noches y buena suerte contra el lord, señor Cerizy.

---Nos veremos en Francia---dijo el francés al retirarse.---Qué lástima
de logia\ldots{} Marchaba tan bien\ldots{} Sr.~Canencia, siento que no
conozca usted mi \emph{Memoria sobre las tiranías}.

Cuando el jefe de la ronda bajaba la escalera, sacome de mi escondite
Santorcaz, y presentándome a sus amigos, dijo con sorna:

---Señores, presento a ustedes un espía de los ingleses.

No le contesté una palabra.

---Bien se conoce, amiguito\ldots{} pero no reñiremos---añadió el masón
ofreciéndome una silla y poniéndome delante una copa que llenó.---Bebe.

---Yo no bebo.

---Amigo Ciruelo---dijo D. Luis al más joven de los presentes---te
quedarás en Salamanca hasta mañana, porque en lugar tuyo va a salir este
joven.

---Sí, eso es---objetó Ciruelo mirándome con enojo.---Y si vienen los
aliados y me ahorcan\ldots{} Yo no soy espía de los ingleses.

---¡Ingleses, franceses!\ldots---exclamó el filósofo Canencia en tono
sibilítico\ldots---hombres que se disputan el terreno, no las
ideas\ldots{} ¿Qué me importa cambiar de tiranos? A los que como yo
combaten por la filosofía, por los grandes principios de Voltaire y
Rousseau, lo mismo les importa que reinen en España las casacas rojas o
los capotes azules.

---¿Y usted qué piensa?---me dijo Monsalud, observándome con
curiosidad.---¿Entrarán los aliados en Salamanca?

---Sí señor, entraremos---contesté con aplomo.

---Entraremos\ldots{} luego usted pertenece al ejército aliado.

---Al ejército aliado pertenezco.

---¿Y cómo está usted aquí?---me preguntó con ademán y tono de la mayor
fiereza otro de los presentes, que era hombre más fuerte y robusto que
un toro.

---Estoy aquí, porque he venido.

Necesitaba hacer grandes esfuerzos para sofocar mi indignación.

---Este joven se burla de nosotros---dijo Ciruelo.

---Pues yo sostengo que los aliados no entrarán en Salamanca---añadió
Monsalud .---No traen artillería de sitio.

---La traerán\ldots{}

---Ignoran con qué clase de fortificaciones tienen que habérselas.

---El duque de Ciudad-Rodrigo no ignora nada.

---Bueno, que entren---dijo Santorcaz.---Puesto que Marmont nos
abandona\ldots{}

---Lo que yo digo---indicó el filósofo;---casacas rojas o casacas
azules\ldots{} ¿qué más da?

---Pero es indigno que favorezcamos a los espías de Wellington---exclamó
con ira el bárbaro Monsalud, levantándose de su asiento.

Yo decía para mí:

---No habrá en esta maldita casa un agujero por donde escapar solo con
ella.

---Siéntate y calla, Monsalud---dijo Santorcaz.---A mí me importa poco
que \emph{Narices} entre o no en Salamanca. Ponga yo el pie en mi
querida Francia\ldots{} Aquí no se puede vivir.

---Si siguieran los franceses mi parecer---dijo el joven Ciruelo con la
expresión propia de quien está seguro de manifestar una gran
idea,---antes de entregar esta ciudad histórica a los aliados, la
volarían. Basta poner seis quintales de pólvora en la catedral, otros
seis en la Universidad, igual dosis en los Estudios Menores, en la
Compañía, en San Esteban, en Santo Tomás y en todos los grandes
edificios\ldots{} Vienen los aliados, ¿quieren entrar?, ¡fuego! ¡Qué
hermoso montón de ruinas! Así se consiguen dos objetos; acabar con
ellos, y destruir uno de los más terribles testimonios de la tiranía,
barbarie y fanatismo de esos ominosos tiempos, señores\ldots{}

---Orador Ciruelo, tú harás revoluciones---dijo Canencia con majestuosa
petulancia.

---Lo que yo afirmo---gruñó Monsalud---es que venzan o no los aliados,
no me marcharé de España.

---Ni yo---mugió el toro.

---Prefiero volverme con los insurgentes---dijo el quinto personaje, que
hasta entonces no había desplegado los bozales labios.

---Yo me voy para siempre de España---afirmó Santorcaz.---Veo malparada
aquí la causa francesa. Antes de dos años Fernando VII volverá a Madrid.

---¡Locura, necedad!

---Si esta campaña termina mal para los franceses, como creo\ldots{}

---¿Mal? ¿Por qué?

---Marmont no tiene fuerzas.

---Se las enviarán. Viene en su auxilio el rey José con tropas de
Castilla la Nueva.

---Y la división Esteve, que está en Segovia.

---Y el ejército de Bonnet viene cerca ya.

---Y también Cafarelli con el ejército del Norte.

---Todavía no ha venido---dijo Santorcaz con tristeza.---Bien, si vienen
esas tropas y ponen los franceses toda la carne en el asador\ldots{}

---Vencerán.

---¿Qué crees tú, Araceli?

---Que Marmont, Bonnet, Esteve, Cafarelli y el rey José no hallarán
tierra por donde correr si tropiezan con los aliados---dije con gran
aplomo.

---Lo veremos, caballero.

---Eso es, lo verán ustedes---repuse.---Lo veremos todos. ¿Saben ustedes
bien lo que es el ejército aliado que ha tomado a Ciudad-Rodrigo y
Badajoz? ¿Saben ustedes lo que son esos batallones portugueses y
españoles, esa caballería inglesa?\ldots{} Figúrense ustedes una fuerza
inmensa, una disciplina admirable, un entusiasmo loco, y tendrán idea de
esa ola que viene y que todo lo arrollará y destruirá a su paso.

Los seis hombres me miraban absortos.

---Supongamos que los franceses son derrotados; ¿qué hará entonces el
Emperador?

---Enviar más tropas.

---No puede ser. ¿Y la campaña de Rusia?

---Que va muy mal, según dicen---indiqué yo.

---No va sino muy bien, caballero---exclamó Monsalud, con gesto
amenazador.

---Las últimas noticias---dijo el quinto personaje, que tenía facha de
militar, y era hombre fuerte, membrudo, imponente, de mirar atravesado y
antipática catadura---son estas\ldots{} Acabo de leerlas en el papel que
nos han mandado de Madrid. El Emperador es esperado en Varsovia. El
primer cuerpo va sobre Piegel; el mariscal duque de Regio, que manda el
segundo, está en Wehlan; el mariscal duque de Elchingen, en Soldass; el
rey de Westphalia en Varsovia\ldots{}

---Eso está muy lejos y no nos importa nada---dijo Santorcaz con
disgusto.---Por bien que salga el Emperador de esa campaña temeraria, no
podrá en mucho tiempo mandar tropas a España\ldots{} y parece que Soult
anda muy apretado en Andalucía y Suchet en Valencia.

---Todo lo ves negro---gritó con enojo Monsalud.

---Veo la guerra del color que tiene ahora\ldots{} De modo que a Francia
me voy, y salga el sol por Antequera.

---Triste cosa es vivir de esta manera---dijo el filósofo.---Somos
ganado trashumante. Verdad es que no pasamos por punto alguno sin dejar
la semilla del \emph{Contrato social} que germinará pronto poblando el
suelo de verdaderos ciudadanos\ldots{} Y es además de triste vergonzoso
vernos obligados a pasar por cómicos de la legua.

---Yo no me vestiré más de payaso, aunque me aspen---declaró Monsalud.

---Y yo, antes de dejarme descuartizar por afrancesado, me volveré con
los insurgentes---indicó el que tenía figura y corpulencia de salvaje
toro.

---Nada perdemos con adoptar nuestro disfraz---dijo D. Luis.---Con que
se vista uno y nos siga el carro lleno de trebejos, bastará para que no
nos hagan daño en esos feroces pueblos\ldots{} Conque en marcha,
señores. Araceli, dame tus armas, porque nosotros no llevamos
ninguna\ldots{} En caso contrario, no me expondré a sacarte.

Se las di, disimulando la rabia que llenaba mi alma, y al punto
empezaron los preparativos de marcha. Unos corrían a cerrar sus breves
maletas, más llenas de papeles que de ropas. Arregló Ramoncilla el
equipaje de su amo, y no tardaron en atronar las casas los ruidos que
caballerías y carros hacían en el patio. Cuando pasé a la habitación
donde estaban Inés y miss Fly, sorprendiome hallarlas en conversación
tirada, aunque no cordial al parecer, y en el semblante de la primera
advertí un hechicero mohín irónico, mezclado de tristeza profunda. Yo
ocultaba y reprimía en el fondo de mi pecho una tempestad de
indignación, de zozobra. Aun allí, rodeado de tan diversa gente, miraba
con angustia a todos los rincones, ansiando descubrir alguna brecha,
algún resquicio, por donde escapar solo con ella. Creíame capaz de las
hazañas que soñaba el alto espíritu de miss Fly.

Pero no había medio humano de realizar mi pensamiento. Estaba en poder
de Santorcaz, como si dijéramos, en poder del demonio. Traté de
acercarme a Inés para hablarla a solas un momento, con esperanzas de
hallar en ella un amoroso cómplice de mi deseo; pero Santorcaz con claro
designio y miss Fly quizás sin intención, me lo impidieron. Inés misma
parecía tener empeño en no honrarme con una sola mirada de sus amantes
ojos.

Athenais, conservando su falda de amazona, se había transfigurado,
escondiendo graciosamente su busto y hermosa cabeza bajo los pliegues de
un manto español.

---¿Qué tal estoy así?---me dijo riendo en un instante que estuvimos
solos.

---Bien---contesté fríamente, preocupado con otra imagen que atraía los
ojos de mi alma.

---¿Nada más que bien?

---Admirablemente. Está usted hermosísima.

---Vuestra novia, Sr.~Araceli---dijo con expresión festiva y algo
impertinente,---es bastante sencilla.

---Un poco, señora.

---Está buena para un pobre hombre\ldots{} ¿Pero es cierto que
amáis\ldots{} a eso?

---¡Oh! Dios de los cielos---dije para mí sin hacer caso de miss
Fly,---¿no habrá un medio de que yo escape solo con ella?

Iba la inglesa a repetir su pregunta, cuando Santorcaz nos llamó
dándonos prisa para que bajásemos. Él y sus amigos habían forrado sus
personas en miserables vestidos.

---Las dos señoras en el coche que guiará Juan---dijo D. Luis.---Tres a
caballo y los otros en el carro. Araceli, entra en el carro con Monsalud
y Canencia.

---Padre, no vayas a caballo---dijo Inés.---Estás muy enfermo.

---¿Enfermo? Más fuerte que nunca\ldots{} Vamos: en marcha\ldots{} Es
muy tarde.

Distribuyéronse los viajeros conforme al programa, y pronto salimos en
burlesca procesión de la casa y de la calle y de Salamanca. ¡Oh, Dios
poderoso! Me parecía que había estado un siglo dentro de la ciudad.
Cuando sin hallar obstáculos en las calles ni en la muralla, me vi fuera
de las temibles puertas, me pareció que tornaba a la vida.

Según orden de Santorcaz, el cochecillo donde iban las dos damas
marchaba delante, seguían los jinetes, y luego los carros, en uno de los
cuales tocome subir con los dos interesantes personajes citados. Al
verme en el campo libre, si se calmó mi desasosiego por los peligros que
corrí dentro de \emph{Roma la chica}, sentí una aflicción vivísima por
causas que se comprenderán fácilmente. Me era forzoso correr hacia el
cuartel general, abandonando aquel extraño convoy donde iban los amores
de toda mi vida, el alma de mi existencia, el tesoro perdido, encontrado
y vuelto a perder, sin esperanza de nueva recuperación. Llevado,
arrastrado yo mismo por aquella cuadrilla de demonios, ni aun me era
posible seguirla, y el deber me obligaba a separarme en medio del
camino. La desesperación se apoderó de mí, cuando mis ojos dejaron de
ver en la oscuridad de la noche a las dos mujeres que marchaban delante.
Salté al suelo y corriendo con velocidad increíble, pues la hondísima
pena parecía darme alas, grité con toda la fuerza de mis pulmones:

---¡Inés, miss Fly!\ldots{} aquí estoy\ldots{} parad, parad\ldots{}

Santorcaz corrió al galope detrás de mí y me detuvo.

---Gabriel---gritó---ya te he sacado de la ciudad y ahora puedes
marcharte dejándonos en paz. A mano derecha tienes el camino de
Aldea-Tejada.

---¡Bandido!---exclamé con rabia.---¿Crees que si no me hubieras quitado
las armas me marcharía solo?

---¡Muy bravo estás!\ldots{} Buen modo de pagar el beneficio que acabo
de hacerte\ldots{} Márchate de una vez. Te juro que si vuelves a ponerte
delante de mí y te atreves a amenazarme, haré contigo lo que
mereces\ldots{}

---¡Malvado!\ldots---grité abalanzándome al arzón de su cabalgadura y
hundiendo mis dedos en sus flacos muslos.---¡Sin armas estoy y podré dar
cuenta de ti!

El caballo se encabritó, arrojándome a cierta distancia.

---¡Dame lo que es mío, ladrón!---exclamé tornando hacia mi
enemigo.---¿Crees que te temo? Baja de ese caballo\ldots{} devuélveme mi
espada y veremos.

Santorcaz hizo un gesto de desprecio, y en el silencio de la noche oí el
rumor de su irónica risa. El otro jinete, que era el semejante a un
toro, se le unió incontinenti.

---O te marchas ahora mismo---dijo D. Luis---o te tendemos en el camino.

---La señora inglesa ha de partir conmigo. Hazla detener---dije
sofocando la intensa cólera que a causa de mi evidente inferioridad me
sofocaba.

---Esa dama irá a donde quiera.

---¡Miss Fly, miss Fly!---grité ahuecando ambas manos junto a mi boca.

Nadie me respondía, ni aun llegaba a mis oídos el rumor de las ruedas
del coche. Corrí largo trecho al lado de los caballos, fatigado,
jadeante, cubierto de sudor y con profunda agonía en el alma\ldots{}
Volví a gritar luego diciendo:

---¡Inés, Inés! ¡Aguarda un instante\ldots{} allá voy!

Las fuerzas me faltaban. Los jinetes se dirigieron en disposición
amenazadora hacia mí; pero un resto de energía física que aún
conservaba, me permitió librarme de ellos, saltando fuera del camino.
Pasaron adelante los caballos, y las carcajadas de Santorcaz y del
hombre-toro resonaron en mis oídos como el graznar de pájaros carniceros
que revoloteaban junto a mí, describiendo pavorosos círculos en torno a
mi cabeza. Si mi cuerpo estaba desmayado y casi exánime, conservaba aún
voz poderosa, y vociferé mientras creí que podía ser oído:

---¡Miserables!\ldots{} ya caeréis en mi poder\ldots{} ¡Eh, Santorcaz,
no te descuides!\ldots{} ¡allá iré yo!\ldots{} ¡allá iré!

Bien pronto se extinguió a lo lejos el ruido de herraduras y ruedas. Me
quedé solo en el camino. Al considerar que Inés había estado en mi mano
y que no me había sido posible apoderarme de ella, sentía impulsos de
correr hacia adelante, creyendo que la rabia bastaría a hacer brotar de
mi cuerpo las potentes alas del cóndor\ldots{} En mi desesperada
impotencia me arrojaba al suelo, mordía la tierra y clamaba al cielo con
alaridos que habrían aterrado a los transeúntes, si por aquella desolada
llanura hubiese pasado en tal hora alma viviente\ldots{} ¡Se me escapaba
quizás para siempre! Registré el horizonte en derredor, y todo lo vi
negro; pero las imágenes de los dos ejércitos pertenecientes a las dos
naciones más poderosas del mundo se presentaron a mi agitada
imaginación. ¡Por allí los franceses\ldots{} por allí los ingleses! Un
paso más y el humo y los clamores de sangrienta batalla se elevarán
hasta el cielo; un paso más y temblará, con el peso de tanto cuerpo que
cae, este suelo en que me sostengo.---¡Oh, Dios de las batallas, guerra
y exterminio es lo que deseo!---exclamé.---Que no quede un solo hombre
de aquí hasta Francia\ldots{} Araceli, al cuartel real\ldots{}
Wellington te espera.

Esta idea calmó un tanto mi exaltación y me levanté del suelo en que
yacía. Cuando di los primeros pasos experimenté esa suspensión del
ánimo, ese asombro indefinible que sentimos en el momento de observar la
falta o pérdida de un objeto que poco antes llevábamos.

---¿Y miss Fly?---dije deteniéndome estupefacto.---No lo sé\ldots{}
adelante.

\hypertarget{xxv}{%
\chapter{XXV}\label{xxv}}

Seguro de que los franceses habían tomado la dirección de Toro, me
encaminé yo hacia el Mediodía buscando el Valmuza, riachuelo que corre a
cuatro o cinco leguas de la capital. Marchaba a pie con toda la prisa
que me permitían el mucho cansancio corporal y las fatigas del alma, y a
las ocho de la mañana entré en Aldea Tejada, después de vadear el Tormes
y recorrer un terreno áspero y desigual desde Tejares. Unos aldeanos
dijéronme antes de llegar allí que no había franceses en los alrededores
ni en el pueblo, y en este oí decir que por Siete Carreras y Tornadizos
se habían visto en la noche anterior muchísimos ingleses.

---Cerca están los míos---dije para mí, y tomando algo de lo necesario
para sustentarme seguí adelante.

Nada me aconteció digno de notarse hasta Tornadizos, donde encontré la
vanguardia inglesa y varias partidas de D. Julián Sánchez. Eran las diez
de la mañana.

---Un caballo, señores, préstenme un caballo---les dije.---Si no,
prepárense a oír al señor duque\ldots{} ¿Dónde está el cuartel general?
Creo que en Bernuy. Un caballo pronto.

Al fin me lo dieron, y lanzándolo a toda carrera primero por el camino y
después por trochas y veredas, a las doce menos cuarto estaba en el
cuartel general. Vestí a toda prisa mi uniforme, informándome al mismo
tiempo de la residencia de lord Wellington, para presentarme a él al
instante.

---El duque ha pasado por aquí hace un momento---me dijo
Tribaldos.---Recorre el pueblo a pie.

Un momento después encontré en la plaza al señor duque, que volvía de su
paseo; conociome al punto, y acercándome a él le dije:

---Tengo el honor de manifestar a vuecencia que he estado en Salamanca y
que traigo todos los datos y noticias que vuecencia desea.

---¿Todos?---dijo Wellington sin hacer demostración alguna de
benevolencia ni de desagrado.

---Todos, mi general.

---¿Están decididos a defenderse?

---El ejército francés ha evacuado ayer tarde la ciudad, dejando sólo
ochocientos hombres.

Wellington miró al general portugués Troncoso que a su lado venía. Sin
comprender las palabras inglesas que se cruzaron, me pareció que el
segundo afirmaba:

---Lo ha adivinado vuecencia.

---Este es el plano de las fortificaciones que defienden el paso del
puente---dije, alargando el croquis que había sacado.

Tomolo Wellington, después de examinarlo con profundísima atención,
preguntó:

---¿Está usted seguro de que hay piezas giratorias en el rebellín, y
ocho piezas comunes en el baluarte?

---Las he contado, mi general. El dibujo será imperfecto; pero no hay en
él una sola línea que no sea representación de una obra enemiga.

---¡Oh, oh! Un foso desde San Vicente al Milagro---exclamó con asombro.

---Y un parapeto en San Vicente.

---San Cayetano parece fortificación importante.

---Terrible, mi general.

---Y estas otras en la cabecera del puente\ldots---Que se unen a los
fuertes por medio de estacadas en zig-zag.

---Está bien---dijo con complacencia, guardando el croquis.---Ha
desempeñado usted su comisión satisfactoriamente a lo que parece.

---Estoy a las órdenes de mi general.

Y luego, volviendo en derredor la perspicaz mirada, añadió:

---Me dijeron que miss Fly cometió la temeridad de ir también a
Salamanca a ver los edificios. No la veo.

---No ha vuelto---dijo un inglés de los de la comitiva.

Interrogáronme todos con alarmantes miradas y sentí cierto embarazo.
Hubiera dado cualquier cosa porque la señorita Fly se presentase en
aquel momento.

---¿Que no ha vuelto?---dijo el duque con expresión de alarma y clavando
en mí sus ojos.---¿Dónde está?

---Mi general, no lo sé---respondí bastante contrariado.---Miss Fly no
fue conmigo a Salamanca. Allí la encontré y después\ldots{} Nos
separamos al salir de la ciudad, porque me era preciso estar en Bernuy
antes de las doce.

---Está bien---dijo lord Wellington como si creyese haber dado excesiva
importancia a un asunto que en sí no lo tenía.---Suba usted al instante
a mi alojamiento para completar los informes que necesito.

No había dado dos pasos, puesto humildemente a la cola de la comitiva
del señor duque, cuando detúvome un oficial inglés, algo viejo, pequeño
de rostro, no menos encarnado que su uniforme, y cuya carilla arrugada y
diminuta se distinguía por cierta vivacidad impertinente, de que eran
signos principales una nariz picuda y unos espejuelos de oro.
Acostumbrados los españoles a considerar ciertas formas personales como
inherentes al oficio militar, nos causaban sorpresa y aun risa aquellos
oficiales de artillería y estado mayor que parecían catedráticos,
escribanos, vistas de aduanas o procuradores.

Mirome el coronel Simpson, pues no era otro, con altanería; mirele yo a
él del mismo modo, y una vez que nos hubimos mirado a sabor de
entrambos, dijo él:

---Caballero, ¿dónde está miss Fly?

---Caballero, ¿lo sé yo acaso? ¿Me ha constituido el duque en custodio
de esa hermosa mujer?

---Se esperaba que miss Fly regresase con usted de su visita a los
monumentos arquitectónicos de Salamanca.

---Pues no ha regresado, caballero Simpson. Yo tenía entendido que miss
Fly podía ir y venir y partir y tornar cuando mejor le conviniese.

---Así debiera ser y así lo ha hecho siempre---dijo el inglés;---pero
estamos en una tierra donde los hombres no respetan a las señoras, y
pudiera suceder que Athenais, a pesar de su alcurnia, no tuviese
completa seguridad de ser respetada.

---Miss Fly es dueña de sus acciones---le contesté.---Respecto a su
tardanza o extravío, ella sola podrá informar a usted cuando parezca.

Era ciertamente grotesco exigirme la responsabilidad de los pasos malos
o buenos de la antojadiza y volandera inglesa, cuando ella no conocía
freno alguno a su libertad, ni tenía más salvaguardia de su honor que su
honor mismo.

---Esas explicaciones no me satisfacen, caballero Araceli---me dijo
Simpson, dignándose dirigir sobre mí una mirada de enojo, que adquiría
importancia al pasar por el cristal de sus espejuelos.---El insigne lord
Fly, conde de Chichester, me ha encargado que cuide de su hija\ldots{}

---¡Cuidar de su hija! ¿Y usted lo ha hecho?\ldots{} Cuando estuvo a
punto de perecer en Santi Spíritus, no le vi a su lado\ldots{} ¡Cuidar
de ella! ¿De qué modo se cuida a las señoritas en Inglaterra? ¿Dejando
que los españoles les ofrezcan alojamiento, que las acompañen a visitar
abadías y castillos?

---Siempre han acompañado a esa señorita dignos caballeros que no
abusaron de su confianza. No se temen debilidades de miss Fly, que tiene
el mejor de los guardianes en su propio decoro; se temen, caballero
Araceli, las violencias, los crímenes que son comunes en las naturalezas
apasionadas de esta tierra. En suma, no me satisfacen las explicaciones
que usted ha dado.

---No tengo que añadir, respecto al paradero de miss Fly, ni una palabra
más a lo que ya tuve honor de manifestar a lord Wellington.

---Basta, caballero---repuso Simpson poniéndome como un pimiento.---Ya
hablaremos de esto en ocasión más oportuna. He manifestado mis recelos a
D. Carlos España, el cual me ha dicho que no era usted de fiar\ldots{}
Hasta la vista.

Apartose de mí vivamente para unirse a la comitiva que estaba muy
distante, y dejome en verdad pensativo el venerable y estudioso oficial.
Poco después D. Carlos España me decía riendo con aquella expresión
franca y un tanto brutal que le era propia:

---Picarón redomado, ¿dónde demonios has metido a la amazona? ¿Qué has
hecho de ella? Ya te tenía yo por buena alhaja. Cuando el coronel
Simpson me dijo que estaba sobre ascuas, le contesté: «No tenga usted
duda, amigo mío; los españoles miran a todas las mujeres como cosa
propia.»

Traté de convencer al general de mi inocencia en aquel delicado asunto;
pero él reía, antes impulsado por móviles de alabanza que de vituperio,
porque los españoles somos así. Luego le conté cómo habiendo necesitado
del auxilio de los masones para salir de Salamanca, nos acompañamos de
ellos hasta salir a buen trecho de la ciudad; mas cuando indiqué que
miss Fly les había seguido, ni España ni ninguno de los que me
escuchaban quisieron creerme.

Cuando fui al alojamiento del general en jefe para informarle de mil
particularidades que él quería conocer relativas a los conventos
destruidos, a municiones, a víveres, al espíritu de la guarnición y del
vecindario, hallé al duque, con quien conferencié más de hora y media,
tan frío, tan severo conmigo, que se me llenó el alma de tristeza.
Recogía mis noticias, harto preciosas para el ejército aliado, sin darme
claras y vehementes señales, cual yo esperaba, de que mi servicio fuese
estimado, o como si estimando el hecho, menospreciara la persona. Hizo
elogios del croquis; pero me pareció advertir en él cierta desconfianza
y hasta la duda de que aquel minucioso dibujo fuese exacto.

Consternado yo, mas lleno de respeto hacia aquel grave personaje, a
quien todos los españoles considerábamos entonces poco menos que un
Dios, no osé desplegar los labios en materia alguna distinta de las
respuestas que tenía que dar: y cuando el héroe de Talavera me despidió
con una cortesía rígida y fría como el movimiento de una estatua que se
dobla por la cintura, salí lleno de confusiones y sobresaltos, mas
también de ira porque yo comprendía que alguna sospecha tan grave como
injusta deslustraba mi buen concepto. ¡Después de tantos trabajos y
fatigas por prestar servicio tan grande al ejército aliado, no se me
trataba con mayor estima que a un vulgar y mercenario espía! ¡Yo no
quería grados ni dinero en pago de mis servicios! Quería consideración,
aprecio, y que el lord me llamase su amigo, o que desde lo alto de su
celebridad y de su genio, dejase caer sobre mi pequeñez cualquier frase
afectuosa y conmovedora, como la caricia que se hace al perro leal; pero
nada de esto había logrado. Trayendo a mi memoria a un mismo tiempo y en
tropel confuso las sofocaciones del día anterior, mi croquis, mis
servicios, y mis apuros, los horrendos peligros, y después la fisonomía
severa y un tanto ceñuda de lord Wellington, el despecho me inspiraba
frases íntimas como la siguiente:

---Quisiera que hubieses estado en poder de Jean-Jean y de Tourlourou, a
ver si ponías esa cara\ldots{} Una cosa es mandar desde la tienda de
campaña, y otra obedecer en la muralla\ldots{} Una cosa es la orden y
otra el peligro\ldots{} Expóngase uno cien veces a morir por un\ldots{}

\hypertarget{xxvi}{%
\chapter{XXVI}\label{xxvi}}

Esta y otras cosas peores que callo decía yo aquella tarde cuando
partimos hacia Salamanca, a cuyas inmediaciones llegamos antes de
anochecido, alejándonos después de la ciudad para pasar el Tormes por
los vados del Canto y San Martín. Por todas partes oía decir:

---Mañana atacaremos los fuertes.

Yo que los había visto, que los había examinado, conocía que esto no
podía ser.

---¡Si creerán ustedes que esos fuertes son juguetes como los que se
hicieron en Madrid el 3 de Diciembre!---decía yo a mis amigos, dándome
cierta importancia.---¡Si creerán ustedes que la artillería que los
defiende es alguna batería de cocina!

Y aquí encajaba descripciones ampulosas, que concluían siempre así:

---Cuando se han visto las cosas, cuando se las ha medido palmo a palmo,
cuando se las ha puesto en dibujo con más o menos arte, es cuando puede
formarse idea acabada de ellas.

---Di, ¿y a miss Fly también la has visto, la has medido palmo a palmo y
la has puesto en dibujo con más o menos arte?---me preguntaban.

Esto me volvía a mis melancolías y \emph{saudades} (hablando en
portugués) ocasionadas por el disfavor de lord Wellington y el ningún
motivo e injusticia de su frialdad y desabrimiento con un servidor leal
y obediente soldado.

Lord Wellington mandó atacar los fuertes por mera conveniencia moral y
por infundir aliento a los soldados, que no habían combatido desde
Arroyo Molinos. Harto conocía el señor duque que aquellas obras formadas
sobre las robustísimas paredes de los conventos no caerían sino ante un
poderoso tren de batir, y al efecto hizo venir de Almeida piezas de gran
calibre. Esperando, pues, el socorro, y simulando ataques pasaron dos o
tres días, en los cuales nada histórico ni particular ocurrió digno de
ser contado, pues ni adquirió lord Wellington nuevos títulos
nobiliarios, ni pareció miss Fly, ni tuve noticias del rumbo que tomaron
los traviesos y mil veces malditos masones.

De lo ocurrido entonces únicamente merecen lugar, y por cierto muy
preferente, en estas verídicas relaciones, las miradas que me echaba de
vez en cuando el coronel Simpson y sus palabras agresivas, a que yo le
contestaba siempre con las peores disposiciones del mundo. Y
francamente, señores, yo estaba inquieto, casi tan inquieto como el
sabio coronel Simpson, porque pasaban días y continuaba el eclipse de
miss Fly. Creí entender que se hacían averiguaciones minuciosas; creí
entender ¡oh cielos! que me amenazaba un interrogatorio severo, al cual
seguirían rigurosas medidas penales contra mí; pero Dios, para salvarme
sin duda de castigos que no merecía, permitió que el día 20 muy de
mañana apareciese en los cerros del Norte\ldots{} no la romancesca e
interesante inglesa, sino el mariscal Marmont con 40.000 hombres.

El mismo día en que se nos presentó el francés por el mismo camino de
Toro, se suspendió el ataque de los fuertes e hicimos varios movimientos
para tomar posiciones si el enemigo nos provocaba a trabar batalla. Mas
pronto se conoció que Marmont no tenía ganas de lanzar su ejército
contra nosotros, siendo su intento al aproximarse, distraer las fuerzas
sitiadoras y tal vez introducir algún socorro en los fuertes. Pero
Wellington, aunque no había recibido la artillería de Almeida, persistía
con tenacidad sajona en apoderarse de San Vicente y de San Cayetano, los
dos formidables conventos arreglados para castillos por una irrisión de
la historia. ¡Me parecía estar viéndolos aún desde la torre de la
Merced!

La tenacidad, que a veces es en la guerra una virtud, también suele ser
una falta, y el asalto de los conventos lo fue manifiestamente, cosa
rara en Wellington, que no acostumbraba cometer faltas. La división
española se hallaba en Castellanos de los Moriscos, observando al
francés que ya se corría a la derecha, ya a la izquierda, cuando nos
dijeron que en el asalto infructuoso de San Cayetano habían perecido 120
ingleses y el general Rowes, distinguidísimo en el ejército aliado.

---Ahora se ve cómo también los grandes hombres cometen errores---dije a
mis amigos.---A cualquiera se le alcanzaba que San Vicente y San
Cayetano no eran corrales de gallinas; pero respetemos las
equivocaciones de los de arriba.

---¡Ya está!, ¡ya está ahí\ldots{} albricias!, ¡ya la tenemos
ahí!---exclamó D. Carlos España que a la sazón, de improviso, se había
presentado.

---¿Quién, miss Fly?---pregunté con vivo gozo.

---La artillería, señores, la artillería gruesa que se mandó traer de
Almeida. Ya ha llegado a Pericalbo, esta tarde estará en las paralelas,
se montará mañana y veremos lo que valen esos fuertes que fueron
conventos.

---¡Ah, bien venida sea!\ldots{} creí que hablaba usted de miss Fly, por
cuya aparición daría las dos manos que tengo\ldots{}

Vino efectivamente, no miss Fly, que acerca de esta ni alma viviente
sabía palabra, sino la artillería de sitio, y Marmont, que lo adivinó,
quiso pasar el río para distraer fuerzas a la izquierda del Tormes. Le
vimos correrse a nuestra derecha, hacia Huerta, y al punto recibimos
orden de ocupar a Aldealuenga. Como los franceses cruzaron el Tormes, lo
pasó también el general Graham, y en vista de este movimiento pusieron
los pies en polvorosa. Marmont, que no tenía bastantes fuerzas,
careciendo principalmente de caballería, no osaba empeñar ninguna acción
formal.

Por lo demás, ante la artillería de sitio, San Vicente y San Cayetano no
ofrecieron gran resistencia. Los ingleses (y esto lo digo de referencia,
pues nada vi) abrieron brecha el 27 e incendiaron con bala roja los
almacenes de San Vicente. Pidieron capitulación los sitiados; mas
Wellington, no queriendo admitir condiciones ventajosas para ellos,
mandó asaltar la Merced y San Cayetano, escalando el uno y penetrando en
el otro por las brechas. Quedó prisionera la guarnición.

Este suceso colmó de alegría a todo el ejército, mayormente cuando vimos
que Marmont se alejaba a buen paso hacia el Norte, ignorábamos si en
dirección a Toro o a Tordesillas, porque nuestras descubiertas no
pudieron determinarlo a causa de la oscuridad de la noche. Pero he aquí
que pronto debíamos saberlo, porque la división española y las
guerrillas de D. Julián Sánchez recibieron orden de dar caza a la
retaguardia francesa, mientras todo el ejército aliado, una vez
asegurada Salamanca, marchaba también hacia las líneas del Duero.

Era la mañana del 28 de Junio, cuando nos encontrábamos cerca de
Sanmorales, en el camino de Valladolid a Tordesillas. Según nos dijeron,
la retaguardia enemiga y su impedimenta habían salido de dicho lugar
pocas horas antes, llevándose, según la inveterada e infalible
costumbre, todo cuanto pudieron haber a la mano. Pusiéronse al frente de
la división el conde de España y D. Julián Sánchez con sus intrépidos
guerrilleros que conocían el país como la propia casa, y se mandó forzar
la marcha para poder pescar algo del pesado convoy de los franchutes.
Sin reparar las fuerzas después del largo caminar de la noche, corrió
nuestra vanguardia hacia Babilafuente, mientras los demás rebuscábamos
en Sanmorales lo que hubiese sobrado de la reciente limpia y rapiña del
enemigo. Provistos, al fin, de algo confortativo, seguimos también hacia
aquel punto, y al cabo de dos horas de penosa jornada, cuando
calculábamos que nos faltarían apenas otras dos para llegar a
Babilafuente, distinguimos este lugar en lontananza, mas no lo
determinaba la perspectiva de las lejanas casas, ni ninguna alta torre
ni castillete, ni menos colina o bosquecillo, sino una columna de negro
y espeso humo, que partiendo de un punto del horizonte, subía y se
enroscaba hasta confundirse con la blanca masa de las nubes.

---Los franceses han pegado fuego a Babilafuente---gritó un guerrillero.

---Apretar el paso\ldots{} en marcha\ldots{} ¡Pobre Babilafuente!

---Queman para detenernos\ldots{} creen que nos estorba la tizne\ldots{}
¡Adelante!

---Pero D. Carlos y Sánchez les deben de haber alcanzado---dijo
otro.---Parece que se oyen tiros.

---Adelante, amigos. ¿Cuánto podemos tardar en ponernos allá?

---Una hora y minutos.

Viose luego otra negra columna de humo que salía de paraje más lejano, y
que en las alturas del cielo parecía abrazarse con la primera.

---Es Villorio que arde también---dijeron.---Esos ladrones queman las
trojes después de llevarse el trigo.

Y más cerca, divisamos las rojas llamas oscilando sobre las techumbres,
y una multitud de mujeres despavoridas, ancianos y niños corrían por los
campos huyendo con espanto de aquella maldición de los hombres, más
terrible que las del cielo. Por lo que aquellos infelices nos pudieron
decir entre lágrimas y gritos de angustia, supimos que los de España y
Sánchez entraban a punto que salían los franceses después de incendiar
el pueblo; que se habían cruzado algunos tiros entre unos y otros; pero
sin consecuencias, porque los nuestros no se ocuparon más que de cortar
el fuego.

Estábamos como a doscientos pasos de las primeras casas de la
infortunada aldea, cuando una figura extraña, hermosa, una verdadera y
agraciada obra de la fantasía, una gentil persona, tan distinta de las
comunes imágenes terrestres como lo son de la vulgar vida las admirables
creaciones de la poesía del Norte; una mujer ideal llevada por arrogante
y veloz caballo, pasó allá lejos ante la vista, semejante a los
gallardos jinetes que cruzan por los rosados espacios de un sueño
artístico, sin tocar la tierra, dando al viento cabellera y crin, y
modificando según los cambiantes de la luz su majestuosa carrera. Era
una figura de amazona, vestida no sé si de negro o de blanco, pero igual
a aquellas mujeres galopantes con cuya apostura y arranque ligero, se
representa al aire, al fuego, lo que vuela y lo que quema, y que corrían
en verdad, animando al corcel con varoniles exclamaciones. Iba la gentil
persona fuera del camino, en dirección contraria a la nuestra, por un
extenso llano cruzado de zanjas y charcos, que el corcel saltaba con
airoso brincar, asociando de tal modo su empuje y brío a la voluntad del
jinete, que hembra y caballo parecían una sola persona. Tan pronto se
alejaba como volvía la fantástica figura; pero a pesar de su carrera y
de la distancia, al punto que la vi, diome un vuelco el corazón,
subióseme la sangre con violento golpe al cerebro, y temblé de sorpresa
y alegría. ¿Necesito decir quién era?

Lanzando mi caballo fuera del camino, grité:

---Miss Fly, señorita Mariposa\ldots{} señora Pajarita\ldots{} señora
Mosquita\ldots{} ¡Carísima Athenais\ldots{} Athenais!

Pero la Pajarita no me oía y seguía corriendo, mejor dicho,
revoloteando, yendo, viniendo, tornando a partir y a volver, y trazando
sobre el suelo y en la claridad del espacio caprichosos círculos,
ángulos, curvas y espirales.

---¡Miss Fly, miss Fly!

El viento impedía que mi voz llegase hasta ella. Avivé el paso, sin
apartar los ojos de la hermosa aparición, la cual creeríase iba a
desvanecerse cual caprichosa hechura de la luz o del viento\ldots{} Pero
no: era la misma miss Fly; y buscaba una senda en aquella engañosa
planicie, surcada por zanjas y charcos de inmóvil agua verdosa.

---¡Eh\ldots{} señora Mosquita!\ldots{} ¡que soy yo!\ldots{} Por
aquí\ldots{} por este lado.

\hypertarget{xxvii}{%
\chapter{XXVII}\label{xxvii}}

Por último, llegué cerca de ella y oyó mi voz, y vio mi propia persona,
lo cual hubo de causarle al parecer mucho gusto y sacarla de su
confusión y atolondramiento. Corrió hacia mí riendo y saludándome con
exclamaciones de triunfo, y cuando la vi de cerca, no pude menos de
advertir la diferencia que existe entre las imágenes transfiguradas y
embellecidas por el pensamiento y la triste realidad, pues el corcel que
montaba, por cierto a mujeriegas, la intrépida Athenais, distaba mucho
de parecerse a aquel volador Pegaso que se me representaba poco antes;
ni daba ella al viento la cabellera, cual llama de fuego simbolizando el
pensamiento, ni su vestido negro tenía aquella diafanidad ondulante que
creí distinguir primero, ni el cuartajo, pues cuartajo era, tenía más
cerneja que media docena de mustios y amarillentos pelos, ni la misma
miss Fly estaba tan interesante como de ordinario, aunque sí hermosa, y
por cierto bastante pálida, con las trenzas mal entretejidas por arte de
los dedos, sin aquel concertado desgaire del peinado de las Musas, y
finalmente, con el vestido en desorden anti-armónico a causa del polvo,
arrugas y jirones que en diversos puntos tenía.

---Gracias a Dios que os encuentro---exclamó alargándome la mano.---D.
Carlos España me dijo que estabais en la retaguardia.

Mi gozo por verla sana y libre; lo cual equivalía a un testimonio
precioso de mi honradez, me impulsó a intentar abrazarla en medio del
campo, de caballo a caballo, y habría puesto en ejecución mi atrevido
pensamiento si ella no lo impidiera un tanto suspensa y escandalizada.

---En buen compromiso me ha puesto usted---le dije.

---Me lo figuraba---respondió riendo.---Pero vos tenéis la culpa. ¿Por
qué me dejasteis en poder de aquella gente?

---Yo no dejé a usted en poder de aquella gente; ¡malditos sean ellos
mil veces!\ldots{} Desapareció usted de mi vista y el masón me impidió
seguir. ¿Y nuestros compañeros de viaje?

---¿Preguntáis por la Inesita? La encontraréis en Babilafuente---dijo
poniéndose seria.

---¿En ese pueblo? ¡Bondad divina!\ldots{} Corramos allí\ldots{} ¿Pero
han padecido ustedes algún contratiempo? ¿Hanse visto en algún peligro?
¿Las han mortificado esos bárbaros?

---No, me he aburrido y nada más. A la hora y media de salir de
Salamanca tropezamos con los franceses, que echaron el guante a los
masones diciendo que en Salamanca habían hecho el espionaje por cuenta
de los aliados. Marmont tiene orden del Rey para no hacer causa común
con esos pillos tan odiados en el país. Santorcaz se defendió; mas un
oficial llamole farsante y embustero, y dispuso que todos los de la
brillante comitiva quedásemos prisioneros. Gracias a Desmarets, me han
tratado a mí con mucha consideración.

---¡Prisioneros!

---Sí: nos han tenido desde entonces en ese horrible Babilafuente,
mientras el lord tomaba a Salamanca. ¡Y yo que no he visto nada de eso!
¿Se rindieron los fuertes? ¡Qué gran servicio prestasteis con vuestra
visita a Salamanca! ¿Qué os dijo milord?

---Sí, sí: hable usted a milord de mí\ldots{} Contento está su
excelencia de este leal servidor\ldots{} Sepa miss Fly que lejos de
agradar al duque, me ha tomado entre ojos y se dispone a formarme
consejo de guerra por delitos comunes.

¿Por qué, amigo mío? ¿Qué habéis hecho?

¿Qué he de hacer? Pues nada, señora Pajarita; nada más sino seducir a
una honesta hija de la Gran Bretaña, llevármela conmigo a Salamanca,
ultrajarla con no sé qué insigne desafuero, y después, para colmo de
fiesta, abandonarla pícaramente, o esconderla, o matarla, pues sobre
este punto, que es el lado negro de mi feroz delito, no se han puesto
aún de acuerdo lord Wellington y el coronel Simpson.

Miss Fly rompió en risas tan francas, tan espontáneas y regocijadas, que
yo también me reí. Ambos marchábamos a buen paso en dirección a
Babilafuente.

---Lo que me contáis, Sr.~Araceli---dijo, mientras se teñía su rostro de
rubor hechicero,---es una linda historia. Tiempo hacía que no se me
presentaba un acontecimiento tan dramático, ni tan bonito embrollo. Si
la vida no tuviera estas novelas, ¡cuán fastidiosa sería!

---Usted disipará las dudas del general devolviéndome mi honor, miss
Fly, pues de la pureza de sentimientos de usted no creo que duden milord
ni sir Abraham Simpson. Yo soy el acusado, yo el ladrón, yo el ogro de
cuentos infantiles, yo el gigantón de leyenda, yo el morazo de romance.

---¿Y no os ha desafiado Simpson?---preguntó demostrándome cuánta
complacencia producía en su alma aquel extraño asunto.

---Me ha mirado con altanería y díchome palabras que no le perdono.

---Le mataréis, o al menos le heriréis gravemente, como hicisteis con el
desvergonzado e insolente lord Gray---dijo con extraordinaria luz en la
mirada.---Quiero que os batáis con alguien por causa mía. Vos acometéis
las empresas más arriesgadas por la simpatía que tienen los grandes
corazones con los grandes peligros; habéis dado pruebas de aquel valor
profundo y sereno cuyo arranque parte de las raíces del alma. Un hombre
de tales condiciones no permitirá que se ponga en duda su dignidad, y a
los que duden de ella, les convencerá con la espada en un abrir y cerrar
de ojos.

---La prueba más convincente, Athenais, ha de ser usted\ldots{} Ahora
pensemos en socorrer a esos infelices de Babilafuente. ¿Corre Inés algún
peligro? ¡Loco de mí! ¡Y me estoy con esta calma! ¿Está buena? ¿Corre
algún peligro?

---No lo sé---repuso con indiferencia la inglesa.---La casa en que
estaban empezó a arder.

---¡Y lo dice con esa tranquilidad!

---En cuanto se anunció la entrada de los españoles y me vi libre, salí
en busca del jefe. D. Carlos España me recibió con agrado, y no tuvo
inconveniente en cederme un caballo para volver al cuartel general.

---¿Santorcaz, Monsalud, Inés y demás compañía masónica habrán huido
también?

---No todos. El gran capitán de esta masonería ambulante está postrado
en el lecho desde hace tres días y no puede moverse. ¿Cómo queréis que
huya?

---Eso es obra de Dios---dije con alegría y acelerando el paso.---Ahora
no se me escapará. De grado o por fuerza arrancaremos a Inés de su lado
y la enviaremos bien custodiada a Madrid.

---Falta que quiera separarse de su padre. Vuestra dama encantada es una
joven de miras poco elevadas, de corazón pequeño; carece de imaginación
y de\ldots{} de arranque. No ve más que lo que tiene delante. Es lo que
yo llamo un ave doméstica. No, señor Araceli, no pidáis a la gallina que
vuele como el águila. Le hablaréis el lenguaje de la pasión y os
contestará cacareando en su corral.

---Una gallina, señorita Athenais---le dije, entrando en el pueblo,---es
un animal útil, cariñoso, amable, sensible, que ha nacido y vive para el
sacrificio, pues da al hombre sus hijos, sus plumas y finalmente su
vida; mientras que un águila\ldots{} pero esto es horroroso, miss
Fly\ldots{} arde el pueblo por los cuatro costados\ldots{}

---Desde la llanura presenta Babilafuente un golpe de vista
incomparable\ldots{} Siento no haber traído mi álbum.

Las frágiles casas se venían al suelo con estrépito. Los atribulados
vecinos se lanzaban a la calle, arrastrando penosamente colchones,
muebles, ropas, cuanto podían salvar del fuego, y en diversos puntos la
multitud señalaba con espanto los escombros y maderos encendidos,
indicando que allí debajo habían sucumbido algunos infelices. Por todas
partes no se oían más que lamentos e imprecaciones, la voz de una madre
preguntando por su hijo, o de los tiernos niños desamparados y solos que
buscaban a sus padres. Muchos vecinos y algunos soldados y guerrilleros
se ocupaban en sacar de las habitaciones a los que estaban amenazados de
no poder salir, y era preciso romper rejas, derribar tabiques, deshacer
puertas y ventanas para penetrar desafiando las llamas, mientras otros
se dedicaban a apagar el incendio, tarea difícil porque el agua era
escasa. En medio de la plaza D. Carlos España daba órdenes para uno y
otro objeto, descuidando por completo la persecución de los franceses, a
quienes solamente se pudieron coger algunos carros. Gritaba el general
desaforadamente y su actitud y fisonomía eran de loco furioso.

Miss Fly y yo echamos pie a tierra en la plaza, y lo primero que se
ofreció a nuestra vista fue un infeliz a quien llevaban maniatado cuatro
guerrilleros empujándolo cruelmente a ratos o arrastrándole cuando se
resistía a seguir. Una vez que lo pusieron ante la espantosa presencia
de D. Carlos España, este cerrando los puños y arqueando las negras y
tempestuosas cejas, gritó de esta manera:

---¿Por qué me lo traen aquí?\ldots{} Fusilarle al momento. A estos
canallas afrancesados que sirven al enemigo se les aplasta cuando se les
coge, y nada más.

Observando las facciones de aquel hombre reconocí al Sr.~Monsalud. Antes
de referir lo que hice entonces, diré en dos palabras, por qué había
venido a tan triste estado y funesta desventura. Sucedió que los pobres
masones igualmente malquistos con los franceses que salían y los
españoles que entraban en Babilafuente, optaron, sin embargo, por
aquellos, tratando de seguirles. Excepto Santorcaz, que seguía en
deplorable estado, todos corrieron, pero tuvo tan mala suerte el
travieso Monsalud, que al saltar una tapia buscando el camino de
Villorio, le echaron el guante los guerrilleros, y como desgraciadamente
le conocían por ciertas fechorías, ni santas ni masónicas, que cometiera
en Béjar, al punto le destinaron al sacrificio en expiación de las
culpas de todos los masones y afrancesados de la Península.

---Mi general---dije al conde, abriéndome paso entre la muchedumbre de
soldados y guerrilleros.---Este desgraciado es bastante tuno y no dudo
que ha servido a nuestros enemigos; pero yo le debo un favor que estimo
tanto como la vida, porque sin su ayuda no hubiera podido salir de
Salamanca.

---¿A qué viene ese sermón?---dijo con feroz impaciencia España.

---A pedir a vuecencia que le perdone, conmutándole la pena de muerte
por otra.

El pobre Monsalud, que estaba ya medio muerto, se reanimó, y mirándome
con vehemente expresión de gratitud, puso toda su alma en sus ojos.

---Ya vienes con boberías, ¡rayo de Dios! Araceli, te mandaré
arrestar\ldots---exclamó el conde haciendo extrañas
gesticulaciones.---No se te puede resistir, joven entrometido\ldots{}
Quitadme de delante a ese sabandijo, fusiladle al momento\ldots{} ¡Es
preciso castigar a alguien!, ¡a alguien!

A pesar de esta viva crueldad, que a veces manifestaba de un modo
imponente, España no había llegado aún a aquel grado de exaltación que
años adelante hizo tan célebre como espantoso su nombre. Miró primero a
la víctima, después a mí y a miss Fly, y luego que hubo dado algún
desahogo a su cólera con palabrotas y recriminaciones dirigidas a todos,
dijo:

---Bueno, que no le fusilen. Que le den doscientos palos\ldots{} pero
doscientos palos bien dados\ldots{} Muchachos, os lo entrego\ldots{}
Allí detrás de la iglesia.

---¡Doscientos palos!---murmuró la víctima con dolor.---Prefiero que me
den cuatro tiros. Así moriré de una vez.

Entonces aumentó el barullo, y un guerrillero apareció diciendo:

---Arden todas las sementeras y las eras del lado de Villorio, y arde
también Villoruela y Riolobos y Huerta.

Desde la plaza, abierta al campo por un costado, se distinguía la
horrible perspectiva. Llamas vagas y erráticas surgían aquí y allí del
seco suelo, corriendo por sobre las mieses, cual cabellera movible,
cuyas últimas negras guedejas se perdían en el cielo. En los puntos
lejanos las columnas de humo eran en mayor número y cada una indicaba la
troj o panera que caía bajo la planta de fuego del ejército fugitivo.
Nunca había yo visto desolación semejante. Los enemigos al retirarse
quemaban, talaban, arrancando los tiernos árboles de las huertas,
haciendo luminarias con la paja de las eras. Cada paso suyo aplastaba
una cabaña, talaba una mies, y su rencoroso aliento de muerte destruía
como la cólera de Dios. El rayo, el pedrisco, el simoún, la lluvia y el
terremoto obrando de consuno no habrían hecho tantos estragos en poco
tiempo. Pero el rayo y el simoún, todas las iras del cielo juntas, ¿qué
significan comparadas con el despecho de un ejército que se retira?
Fiero animal herido, no tolera que nada viva detrás de sí.

D. Carlos España tomó una determinación rápida.

---A Villorio, a Villorio sin descansar---gritó montando a
caballo.---Sr.~D. Julián Sánchez, a ver si les cogemos. Además, hay que
auxiliar también a esos otros pueblos.

Las órdenes corrieron al momento, y parte de los guerrilleros con dos
regimientos de línea se aprestaron a seguir a D. Carlos.

---Araceli---me dijo este,---quédate aquí aguardando mis órdenes. En
caso de que lleguen hoy los ingleses, sigues hacia Villorio; pero entre
tanto aquí\ldots{} Apagar el fuego lo que se pueda; salvar la gente que
se pueda, y si se encuentran víveres\ldots{}

---Bien, mi general.

---Y a ese bribón que hemos cogido, cuidado como le perdones un solo
palo. Doscientos cabalitos y bien aplicados. Adiós. Mucho orden,
y\ldots{} ni uno menos de doscientos.

\hypertarget{xxviii}{%
\chapter{XXVIII}\label{xxviii}}

Cuando me vi dueño del pueblo y al frente de la tropa y guerrillas que
trabajaban en él, empecé a dictar órdenes con la mayor actividad. Excuso
decir que la primera fue para librar a Monsalud del horrible tormento y
descomunal castigo de los palos; mas cuando llegué al sitio de la
lamentable escena, ya le habían aplicado veintitrés cataplasmas de
fresno, con cuyos escozores estaba el infeliz a punto de entregar
rabiando su alma al Señor. Suspendí el tormento, y aunque más parecía
muerto que vivo, aseguráronme que no iría de aquella, por ser los
masones gente de siete vidas, como los gatos.

Miss Fly me indicó sin pérdida de tiempo la casa que servía de asilo a
Santorcaz, una de las pocas que apenas habían sido tocadas por las
llamas. Vociferaban a la puerta algunas mujeres y aldeanos, acompañados
de dos o tres soldados, esforzándose las primeras en demostrar con toda
la elocuencia de su sexo, que allí dentro se guarecía el mayor pillo que
desde muchos años se había visto en Babilafuente.

---El que llevaron a la plaza---decía una vieja---es un santo del cielo
comparado con este que aquí se esconde, el capitán general de todos esos
luciferes.

---Como que hasta los mismos franceses les dan de lado. Diga usted, señá
Frasquita, ¿por qué llaman masones a esta gente? A fe que no entiendo el
\emph{voquible}.

---Ni yo; pero basta saber que son muy malos, y que andan de compinche
con los franceses para quitar la religión y cerrar las iglesias.

---Y los tales, cuando entran en un pueblo, apandan todas las doncellas
que encuentran. Pues digo: también hay que tener cuidado con los niños,
que se los roban para criarlos a su antojo, que es en la fe de Majoma.

Los soldados habían empezado a derribar la puerta y las mujeres les
animaban, por la mucha inquina que había en el pueblo contra los
masones. Ya vimos lo que le pasó a Monsalud. Seguramente, Santorcaz con
ser el pontífice máximo de la secta trashumante, no habría salido mejor
librado si en aquella ocasión no hubiese llegado yo. Luego que la puerta
cediera a los recios golpes y hachazos, ordené que nadie entrase por
ella, dispuse que los soldados, custodiando la entrada, contuvieran y
alejasen de allí a las mujeres chillonas y procaces, y subí. Atravesé
dos o tres salas cuyos muebles en desorden anunciaban la confusión de la
huida. Todas las puertas estaban abiertas, y libremente pude avanzar de
estancia en estancia hasta llegar a una pequeña y oscura, donde vi a
Santorcaz y a Inés, él tendido en miserable lecho, ella al lado suyo,
tan estrechamente abrazados los dos que sus figuras se confundían en la
penumbra de sala. Padre e hija estaban aterrados, trémulos como quien de
un momento a otro espera la muerte, y se habían abrazado para aguardar
juntos el trance terrible. Al conocerme, Inés dio un grito de alegría.

---Padre---exclamó,---no moriremos. Mira quién está aquí.

Santorcaz fijó en mí los ojos que lucían como dos ascuas en el
cadavérico semblante, y con voz hueca, cuyo timbre heló mi sangre, dijo:

---¿Vienes por mí, Araceli? ¿Ese tigre carnicero que os manda te envía a
buscarme porque los oficiales del matadero están ya sin trabajo?\ldots{}
Ya despacharon a Monsalud, ahora a mí\ldots{}

---No matamos a nadie---respondí acercándome.

---No nos matarán---exclamó Inés derramando lágrimas de gozo.---Padre,
cuando esos bárbaros daban golpes a la puerta, cuando esperábamos verles
entrar armados de hachas, espadas, fusiles y guillotinas para cortarnos
la cabeza, como dices que hacían en París, ¿no te dije que había creído
escuchar la voz de Araceli? Le debemos la vida.

El masón clavaba en mí sus ojos, mirándome cual si no estuviera seguro
de que era yo. Su fisonomía estaba en extremo descompuesta, hundidos los
ojos dentro de las cárdenas órbitas, crecida la barba, lustrosa y
amarilla la frente. Parecía que habían pasado por él diez años desde las
escenas de Salamanca.

---Nos perdonan la vida---dijo con desdén.---Nos perdonan la vida cuando
me ven enfermo y achacoso, sin poder moverme de este lecho, donde me ha
clavado mi enfermedad. El conde de España ¿va a subir aquí?

---El conde de España se ha ido de Babilafuente.

Cuando dije esto, el anciano respiró como si le quitaran de encima
enorme peso. Incorporose ayudado por su hija, y sus facciones,
contraídas por el terror, se serenaron un poco.

---¿Se ha marchado ese verdugo\ldots{} hacia Villorio?\ldots{} Entonces
escaparemos por\ldots{} por\ldots{} y los ingleses, ¿dónde están?

---Si se trata de escapar, en todas partes hay quien lo impida. Se
acabaron las correrías por los pueblos.

---De modo que estoy preso---exclamó con estupor.---¡Soy prisionero
tuyo, prisionero de\ldots! ¡Me has cogido como se coge a un ratón en la
trampa, y tengo que obedecerte y seguirte tal vez!

---Sí, preso hasta que yo quiera.

---Y harás de mí lo que se te antoje, como un chiquillo sin piedad que
martiriza al león en su jaula porque sabe que este no puede hacerle
daño.

---Haré lo que debo, y ante todo\ldots{}

Santorcaz, al ver que fijé los ojos en su hija, estrechola de nuevo en
sus brazos, gritando:

---No la separarás de mí sino matándola, ruin y miserable
verdugo\ldots{} ¿Así pagas el beneficio que en Salamanca te
hice?\ldots{} Manda a tus bárbaros soldados que nos fusilen, pero no nos
separes.

Miré a Inés y vi en ella tanto cariño, tan franca adhesión al anciano,
tanta verdad en sus demostraciones de afecto filial, que no pude menos
de cortar el vuelo a mi violenta determinación.

---Aquí encuentro un sentimiento cuya existencia no sospechaba---dije
para mí ;---un sentimiento grande, inmenso, que se me revela de
improviso y que me espanta y me detiene y me hace retroceder. He creído
caminar por sendero continuado y seguro, y he llegado a un punto en que
el sendero acaba y empieza el mar. No puedo seguir\ldots{} ¿Qué
inmensidad es esta que ante mí tengo? Este hombre será un malvado, será
carcelero de la infeliz niña; será un enemigo de la sociedad, un
agitador, un loco que merece ser exterminado; pero aquí hay algo más.
Entre estos dos seres, entre estas dos criaturas tan distintas, la una
tan buena, la otra odiosa y odiada, existe un lazo que yo no debo ni
puedo romper, porque es obra de Dios. ¿Qué haré?

A estas reflexiones sucedieron otras de igual índole, mas no me llevaron
a ninguna afirmación categórica respecto a mi conducta, y me expresé de
este modo, que me pareció el más apropiado a las circunstancias.

---Si usted varía de conducta podrá tal vez vivir cerca, cuando no al
lado de su hija y verla y tratarla.

---¡Variar de conducta!\ldots{} ¿Y quién eres tú, mancebo ignorante,
para decirme que varíe de conducta, y dónde has aprendido a juzgar mis
acciones? Estás lleno de soberbia porque el despotismo te ha enmascarado
con esa librea y puesto esas charreteras que no sirven sino para marcar
la jerarquía de los distintos opresores del pueblo\ldots{} ¡Qué sabes tú
lo que es conducta, necio! Has oído hablar a los frailes y a D. Carlos
España, y crees poseer toda la ciencia del mundo.

---Yo no poseo ciencia alguna---respondí exasperado,---¿pero se puede
consentir que criaturas inocentes y honradas y dignas por todos
conceptos de mejor suerte, vivan con tales padres?

---Y a ti, extraño a ella, extraño a mí, ¿qué te importa ni qué te va en
esto?---exclamó agitando sus brazos y golpeando con ellos las ropas del
desordenado lecho.

---Sr.~Santorcaz, acabemos. Dejo a usted en libertad para ir a donde
mejor le plazca. Me comprometo a garantizarle la mayor seguridad hasta
que se halle fuera del país que ocupa el ejército aliado. Pero esta
joven es mi prisionera y no irá sino a Madrid al lado de su madre. Si
han nacido por fortuna en usted sentimientos tiernos que antes no
conocía, yo aseguro que podrá ver a su hija en Madrid siempre que lo
solicite.

Al decir esto, miré a Inés, que con extraordinario estupor dirigía los
ojos a mí y a su padre alternativamente.

---Eres un loco---dijo D. Luis.---Mi hija y yo no nos separaremos.
Háblale a ella de este asunto, y verás cómo se pone\ldots{} En fin,
Araceli, ¿nos dejas escapar, sí o no?

---No puedo detenerme en discusiones. Ya he dicho cuanto tenía que
decir. Entre tanto quedarán en la casa y nadie se atreverá a hacerles
daño.

---¡Preso, cogido, Dios mío!---clamó Santorcaz antes afligido que
colérico, y llorando de desesperación.---¡Preso, cogido por esta
soldadesca asalariada a quien detesto; preso antes de poder hacer nada
de provecho, antes de descargar un par de buenos y seguros
golpes!\ldots{} ¡Esto es espantoso! Soy un miserable\ldots{} no sirvo
para nada\ldots{} lo he dejado todo para lo último\ldots{} me he ocupado
en tonterías\ldots{} lo grave, lo formal es destruir todo lo que se
pueda, ya que seguramente nada existe aquí digno de conservarse.

---Tenga usted calma, que el estado de ese cuerpo no es a propósito para
reformar el linaje humano.

---¿Crees que estoy débil, que no puedo levantarme?---gritó intentando
incorporarse con esfuerzos dolorosos.---Todavía puedo hacer algo\ldots{}
esto pasará, no es nada\ldots{} aún tengo pulso\ldots{} ¡Ay! en lo
sucesivo no perdonaré a nadie. Todo aquél que caiga bajo mi mano
perecerá sin remedio.

Inés le ponía las manos en los hombros para obligarle a estarse quieto y
recogía la ropa de abrigo, que los movimientos del enfermo arrojaban a
un lado y otro.

---¡Preso, cogido como un ratón!---prosiguió este.---Es para volverse
loco\ldots{} ¡Cuando había fundado treinta y cuatro logias en que se
afiliaba lo más atrevido y lo más revoltoso, es decir, lo mejor y lo más
malo de todo el país!\ldots{} ¡Oh!, ¡esos indignos franceses me han
hecho traición! Les he servido, y este es el pago\ldots{} Araceli,
¿dices que estoy preso, que me llevarán a la cárcel de Madrid, a Ceuta
tal vez?\ldots{} ¡Maldigo la infame librea del despotismo que vistes!
¡Ceuta!\ldots{} Bueno; me escaparé como la otra vez\ldots{} mi hija y yo
nos escaparemos. Aún tengo agilidad, aliento, brío; todavía soy
joven\ldots{} ¡Caer en poder de estos verdugos con charreteras, cuando
me creía libre para siempre y tocaba los resultados de mi obra de tantos
años!\ldots{} porque sí, no sois más que verdugos con charreteras,
grados falsos y postizos honores. ¡Mujeres de la tierra, parid hijos
para que los nobles los azoten, para que los frailes los excomulguen y
para que estos sayones los maten!\ldots{} ¡Bien lo he dicho siempre! La
masonería no debe tener entrañas, debe ser cruel, fría, pesada,
abrumadora como el hacha del verdugo\ldots{} ¿Quién dice que yo estoy
enfermo, que yo estoy débil, que me voy a morir, que no puedo levantarme
más?\ldots{} Es mentira, cien veces mentira\ldots{} Me levantaré y ¡ay
del que se me ponga delante! Araceli, cuidado, cuidado, aprendiz de
verdugo\ldots{} todavía\ldots{}

Siguió hablando algún tiempo más; pero le faltaba gradualmente el
aliento, y las palabras se confundían y desfiguraban en sus labios. Al
fin no oíamos sino mugidos entrecortados y guturales, que nada
expresaban. Su respiración era fatigosa, había cerrado los ojos; pero
los abría de cuando en cuando con la súbita agitación de la fiebre.
Toqué sus manos y despedían fuego.

---Este hombre está muy malo---dije a Inés, que me miraba con
perplejidad.

---Lo sé; pero en esta casa no hay nada, ni tenemos remedios, ni comida;
en una palabra, nada.

Llamando a mi asistente que estaba en la calle, le di orden de que
proporcionase a Inés cuanto fuese preciso y existiera en el lugar.

---Mi asistente no se separará de aquí mientras lo necesites---dije a mi
amiga.---La puerta se cerrará. Puedes estar tranquila. En todo el día no
saldremos de aquí. Adiós, me voy a la plaza, pero volveré pronto, porque
tenemos que hablar, mucho que hablar.

\hypertarget{xxix}{%
\chapter{XXIX}\label{xxix}}

Cuando volví, estaba sentada junto al lecho del enfermo, a quien miraba
fijamente. Volviendo la cabeza, indicome con un signo que no debía hacer
ruido. Levantose luego, acercó su rostro al de Santorcaz y cerciorada de
que permanecía en completo y bienhechor reposo, se dispuso a salir del
cuarto. Juntos fuimos al inmediato, no cerrando sino a medias la puerta,
para poder vigilar al desgraciado durmiente, y nos sentamos el uno
frente al otro. Estábamos solos, casi solos.

---¿Has tenido nuevas noticias de mi madre?---me preguntó muy conmovida.

---No, pero pronto la veremos\ldots{}

---¡Aquí, Dios mío! Tanta felicidad no es para mí.

---Le escribiré hoy diciendo que te he encontrado y que no te me
escaparás. Le diré que venga al instante a Salamanca.

---¡Oh! Gabriel\ldots{} haces precisamente lo mismo que yo deseaba, lo
que deseaba hace tanto tiempo\ldots{} Si hubieras sido prudente en
Salamanca; y me hubieras oído antes de\ldots{}

---Querida mía, tienes que explicarme muchas cosas que no he
entendido---le dije con amor.

---¿Y tú a mí? Tú sí que tienes necesidad de explicarte bien. Mientras
no lo hagas, no esperes de mí una palabra, ni una sola.

---Hace seis meses que te busco, alma mía, seis meses de fatigas, de
penas, de ansiedad, de desesperación\ldots{} ¡Cuánto me hace trabajar
Dios antes de concederme lo que me tiene destinado! ¡Cuánto he padecido
por ti, cuánto he llorado por ti! Dios sabe que te he ganado bien.

---Y durante ese tiempo---preguntó con graciosa malicia,---¿te ha
acompañado esa señora inglesa, que te llama su caballero y que me ha
vuelto loca a preguntas?

---¿A preguntas?

---Sí: quiere saberlo todo, y para cerrarle el pico he necesitado
decirle cómo y cuándo nos conocimos. Lo que se refiere a mí le importa
poco; tu vida es lo que le interesa; me ha marcado tanto deseando saber
las locuras y sublimidades que has hecho por esta infeliz, que no he
podido menos de divertirme a costa suya\ldots{}

---Bien hecho, querida mía.

---¡Qué orgullosa es\ldots! Se ríe de cuanto hablo y, según ella, no
abro la boca más que para decir vulgaridades. Pero la he
castigado\ldots{} Como insistiese en conocer tus empresas amorosas, la
he dicho que después de Bailén quisieron robarme veinticinco hombres
armados, y que tú solo les mataste a todos.

Inés sonreía tristemente, y yo sofocaba la risa.

---También le dije que en el Pardo, para poder hablarme, te disfrazaste
de duque, siendo tal el poder de la falsa vestimenta, que engañaste a
toda la corte y te presentaron al emperador Napoleón, el cual se encerró
contigo en su gabinete, y te confió el plan de su campaña contra el
Austria.

---Así te vengas tú---dije encantado de la malicia de mi pobre
amiga.---Dame un abrazo, chiquilla, un abrazo o me muero.

---Así me vengo yo. También le dije que estando en Aranjuez pasabas el
Tajo a nado todas las noches para verme; que en Córdoba entraste en el
convento y maniataste a todas las monjas para robarme; que otra vez
anduviste ochenta leguas a caballo para traerme una flor; que te batiste
con seis generales franceses porque me habían mirado, con otras mil
heroicidades, acometimientos y amorosas proezas que se me vinieron a la
memoria a medida que ella me hacía preguntas. ¡Eh, caballerito, no dirá
usted que no cuido de su reputación!\ldots{} Te he puesto en los cuernos
de la luna\ldots{} Puedes creer que la inglesa estaba asombrada. Me oía
con toda su hermosa boca abierta\ldots{} ¿Qué crees? Te tiene por un
Cid, y ella cuando menos se figura ser la misma doña Jimena.

---¡Cómo te has burlado de ella!---exclamé acercando mi silla a la de
Inés.---¿Pero has tenido celos?\ldots{} Dime si has tenido celos para
estarme riendo tres días\ldots{}

---Caballero Araceli---dijo arrugando graciosamente el ceño,---sí, los
he tenido y los tengo\ldots{}

---¡Celos de esa loca!\ldots{} si es una loca---contesté riendo y el
alma inundada de regocijo.---Inés de mi vida, dame un abrazo.

Las lindas manecitas de la muchacha se sacudían delante de mí, y me
azotaban el rostro al acercar. Yo pillándolas al vuelo, se las besaba.

---Inesilla, querida mía, dame un abrazo\ldots{} o te como.

---Hambriento estás.

---Hambriento de quererte, esposa mía. ¿Te parece?\ldots{} seis meses
amando a una sombra. ¿Y tú?\ldots{}

Yo no sabía qué decir. Estaba hondamente conmovido. Mi desgraciada amiga
quiso disimular su emoción; pero no pudo atajar el torrente de lágrimas
que pugnaba por salir de sus ojos.

---No te acuerdes de esa mujer, si no quieres que me enfade. Es
imposible que tú, con la elevación de tu alma, con tu penetración
admirable, hayas podido\ldots{}

---No, no lloro por eso, querido amigo mío---me dijo mirándome con
profundo afecto.---Lloro\ldots{} no sé por qué. Creo que de alegría.

---¡Oh! Si miss Fly estuviera aquí, si nos viera juntos, si viera cómo
nos amamos por bendición especial de Dios, si viera este cariño nuestro,
superior a las contrariedades del mundo, comprendería cuánta diferencia
hay de sus chispazos poéticos a esta fuente inagotable del corazón, a
esta luz divina en que se gozan nuestras almas, y se gozarán por los
siglos de los siglos.

---No me nombres a miss Fly\ldots{} Si en un momento me afligió el
conocerla, ya no hago caso de ella\ldots---dijo secando sus
lágrimas.---Al principio, francamente\ldots{} tuve dudas, más que dudas,
celos; pero al tratarla de cerca se disiparon. Sin embargo, es muy
hermosa, más hermosa que yo.

---Ya quisiera parecerse a ti. Es un marimacho.

---Es además muy rica, según ella misma dice. Es noble\ldots{} Pero a
pesar de todos sus méritos, miss Fly me causaba risa, no sé por qué: yo
reflexionaba y decía: «Es imposible, Dios mío. No puede ser\ldots{}
Caerán sobre mí todas las desgracias menos esta\ldots» ¡Oh! esta sí que
no la hubiera soportado.

---¡Qué bien pensaste! Te reconozco Inés. Reconozco tu grande alma. Duda
de todo el mundo, duda de lo que ven tus ojos; pero no dudes de mí, que
te adoro.

---Mi corazón se desborda\ldots---exclamó oprimiéndose el seno con una
mano que se escapó de entre las mías.---Hace tiempo que deseaba llorar
así\ldots{} delante de ti\ldots{} ¡Bendito sea Dios que empieza a hacer
caso de lo que le he dicho!

---Inés, yo también he tenido celos, queridita; celos de otra clase,
pero más terribles que los tuyos.

---¿Por qué?---dijo mirándome con severidad.

---¡Pobre de mí!\ldots{} Yo me acordaba de tu buena madre y decía
mirándote: «Esta pícara ya no nos quiere.»

---¿Que no os quiero?

---Alma mía: ahora te pregunto como a los niños; ¿a quién quieres tú?

---A todos---contestó con resolución.

Esta respuesta, tan concisa como elocuente, me dejó confuso.

---A todos---repitió.---Si no te creyera capaz de comprenderlo así,
¡cuán poco valdrías a mis ojos!

---Inés, tú eres una criatura superior---afirmé con verdadero
entusiasmo.---Tú tienes en tu alma mayor porción de aliento divino que
los demás. Amas a tus enemigos, a tus más crueles enemigos.

---Amo ami padre---dijo con entereza.

---Sí; pero tu padre\ldots{}

---Vas a decir que es un malvado, y no es verdad. Tú no le conoces.

---Bien, amiga mía, creo lo que me dices; pero las circunstancias en que
has ido a poder de ese hombre no son las más a propósito para que le
tomaras gran cariño\ldots{}

---Hablas de lo que no entiendes. Si yo te dijera una cosa\ldots{}

---Espera\ldots{} déjame acabar\ldots{} Ya sé lo que vas a decir. Es que
has encontrado en él cuando menos lo esperabas un noble y profundo
cariño paternal.

---Sí, pero he encontrado algo más.

---¿Qué?

---La desgracia. Es el hombre más desdichado, más sin ventura que existe
en el mundo.

---Es verdad: la nobleza de tu alma no tiene fin\ldots{} pero dime:
seguramente no hallarán eco en ella los sentimientos de odio y el
frenesí de este desgraciado.

---Yo espero reconciliarle---dijo sencillamente---con los que odia o
aparenta odiar, pues su cólera ante ciertas personas no brota del
corazón.

---¡Reconciliarle!---repetí con verdadero asombro.---¡Oh! Inés, si tal
hicieras, si tan grande objeto lograras tú con la sola fuerza de tu
dulzura y de tu amor, te tendría por la más admirable persona de todo el
mundo\ldots{} Pero debe de haber ocurrido entre ti y él mucho que
ignoro, querida mía. Cuando te viste arrebatada por ese hombre de los
brazos de tu madre enferma, ¿no sentiste?\ldots{}

---Un horror, un espanto\ldots{} no me recuerdes eso, amiguito, porque
me estremezco toda\ldots{} ¡Qué noche, qué agonía! Yo creí morir, y en
verdad pedía la muerte\ldots{} Aquellos hombres\ldots{} todos me
parecían negros, con el pelo erizado y las manos como garfios\ldots{}
aquellos hombres me encerraron en un coche. Encarecerte mi miedo, mis
súplicas, aquel continuo llorar mío durante no sé cuántos días, sería
imposible. Unas veces desesperada y loca, les decía mil injurias, otras
pedíales de rodillas mi libertad. Durante mucho tiempo me resistí a
tomar alimento y también traté de escaparme\ldots{} Imposible, porque me
guardaban muy bien\ldots{} Después de algunos días de marcha, fuéronse
todos, y él quedó solo conmigo en un lugar que llaman Cuéllar.

---¿Y te maltrató?

---Jamás, al principio me trataba con aspereza; pero luego, mientras más
me ensoberbecía yo, mayor era su dulzura. En Cuéllar me dijo que nunca
volvería a ver a mi madre, lo cual me causó tal desesperación y
angustia, que aquella noche intenté arrojarme por la ventana al campo.
El suicidio, que es tan gran pecado, no me aterraba\ldots{} Trájome en
seguida a Salamanca, y allí le oí repetir que jamás vería a mi madre.
Entonces advertí que mis lágrimas le conmovían mucho\ldots{} Un día,
después que largo rato disputamos y vociferamos los dos, púsose de
rodillas delante de mí, y besándome las manos me dijo que él no era un
hombre malo.

---Y tú, ¿sospechabas algo de tu parentesco con él?

---Verás\ldots{} Yo respondí que le tenía por el más malo, el más
abominable ser de toda la tierra, y entonces fue cuando me dijo que era
mi padre\ldots{} Esta revelación me dejó tan suspensa, tan asombrada,
que por un instante perdí el sentido\ldots{} Tomome en sus brazos, y
durante largo rato me prodigó las más afectuosas caricias\ldots{} Yo no
lo quería creer\ldots{} En lo íntimo de mi alma acusé a Dios por haberme
hecho nacer de aquel monstruo\ldots{} Después como advirtiese mi duda,
mostrome un retrato de mi madre y algunas cartas que escogió entre
muchas que tenía\ldots{} Yo estaba medio muerta\ldots{} aquello me
parecía un sueño. En la angustia y turbación de tan dolorosa escena,
fijé la vista en su rostro y un grito se escapó de mis labios.

---¿No le habías observado bien?

---Sí: yo había notado cierto incomprensible misterio en su fisonomía,
pero hasta entonces no vi\ldots{} no vi que su frente era mi frente, que
sus ojos eran mis ojos. Aquella noche me fue imposible dormir: entrome
una fiebre terrible y me revolvía en el lecho, creyéndome rodeada de
sombras o demonios que me atormentaban. Cuando abría los ojos, le
hallaba sentado a mis pies, sin apartar de mí su mirada penetrante que
me hacía temblar. Me incorporé y le dije: «¿Por qué aborrece usted a mi
querida madre?» Besándome las manos, me contestó: «Yo no la aborrezco:
ella es la que me aborrece a mí. Por haberla amado soy el más infeliz de
los hombres; por haberla amado soy este oscuro y despreciado satélite de
los franceses que en mí ves; por haberla adorado te causo espanto hoy en
vez de amor.» Entonces yo le dije: «Grandes maldades habrá hecho usted
con mi madre, para que ella le aborrezca.» No me contestó\ldots{} Se
esforzaba en calmar mi agitación, y desde aquella noche hasta el fin de
la enfermedad que padecí no se apartó de mi lado ni un momento. Cuanto
puede inventarse para distraer a una criatura triste y enferma, él lo
inventó; contábame historias, unas alegres, otras terribles, todas de su
propia vida, y finalmente refiriome lo que más deseaba conocer de
esta\ldots{} Yo temblaba a cada palabra. Había empezado a inspirarme
tanta compasión, que a ratos le suplicaba que callase y no dijese más.
Poco a poco fui perdiéndole el miedo: me causaba cierto respeto; pero
amarle\ldots{} ¡eso imposible!\ldots{} Yo no cesaba de afirmar que no
podía vivir lejos de mi madre, y esto, si le enfurecía de pronto, era
motivo después para que redoblase sus cariños y consideraciones conmigo.
Su empeño era siempre convencerme de que nadie en el mundo me quería
como él. Un día, impaciente y acongojada por el largo encierro, le hablé
con mucha dureza; él se arrojó a mis pies, pidiome perdón del gran daño
que me había causado, y lloró tanto, tanto\ldots{}

---¿Ese hombre ha derramado una lágrima?---dije con sorpresa.---¿Estás
segura? Jamás lo hubiera creído.

---Tantas y tan amargas derramó, que me sentí no ya compasiva, sino
también enternecida. Mi corazón no nació para el odio, nació para
responder a todos los sentimientos generosos, para perdonar y
reconciliar. Tenía delante de mí a un hombre desgraciado, a mi propio
padre, solo, desvalido, olvidado; recordaba algunas palabras oscuras y
vagas de mi madre acerca de él, que me parecían un poco injustas.
Lástima profunda oprimía mi pecho: la adoración, la loca idolatría que
aquel infeliz sentía por mí, no podían serme indiferentes, no, de ningún
modo, a pesar del daño recibido. Le dije entonces cuantas palabras de
consuelo se me ocurrieron, y el pobrecito me las agradeció tanto,
tantísimo\ldots{} Por la primera vez en su vida era feliz.

---¡Ángel del cielo---exclamé con viva emoción,---no digas más! Te
comprendo y te admiro.

---Suplicome entonces que le tratase con la mayor confianza, que le
dijese padre y tú al uso de Francia, con lo cual experimentaría gran
consuelo, y así lo hice. Ese hombre terrible que espanta a cuantos le
oyen y no habla más que de exterminar y de destruir, temblaba como un
niño al escuchar mi voz; y olvidado de la guillotina, de los nobles y de
lo que él llamaba el \emph{estado llano}, estaba horas enteras en
éxtasis delante de mí. Entonces formé mi proyecto, aunque no le dije
nada, esperando que el dominio que ejercía sobre él llegase al último
grado.

---¿Qué proyecto?

---Volver aquel cadáver a la vida, volverle al mundo, a la familia,
desatar aquel corazón de la rueda en que sufría tormento, sacar del
infierno aquel infeliz réprobo y extirpar en su alma el odio que le
consumía. Durante algún tiempo no hablé de volver al lado de mi madre,
ni me quejé de la larga y triste soledad, antes bien aparecía sumisa y
aun contenta. Entonces emprendimos esos horribles viajes para fundar
logias; empezó la compañía de esos hombres aborrecidos, y no pude
disimular mi disgusto. Cuando hablábamos los dos a solas él se reía de
las prácticas masónicas, diciendo que eran simples y tontas, aunque
necesarias para subyugar a los pueblos. Su odio a los nobles, a los
frailes y a los reyes continuaba siempre muy vivo; pero al hablar de mi
madre, la nombraba siempre con reserva y también con emoción. Esto era
señal lisonjera y un principio de conformidad con mi ardiente deseo. Yo
se lo agradecí y se lo pagué mostrándome más cariñosa con él; pero
siempre reservada. Los repetidos viajes, las logias y los compañeros de
masonería, me inspiraban repugnancia, hastío y miedo. No se lo oculté, y
él me decía: «Esto acabará pronto. No conquistaré a los necios sino con
esta farsa; y como los franceses se establezcan en España, verás la que
armo\ldots» «Padre, le decía yo, no quiero que armes cosas malas ni que
mates a nadie, ni que te vengues. La venganza y la crueldad son propias
de almas bajas.» Él me ponderaba las injusticias y picardías que rigen a
la sociedad de hoy, asegurando que era preciso volver todo del revés,
para lo cual era necesario empezar por destruirlo todo. ¡Cuánto hemos
hablado de esto! Por último, tales horrores han dejado de asustarme.
Tengo la convicción de que mi pobre padre no es cruel ni sanguinario
como parece\ldots{}

---Así será, pues tú lo dices.

---Estábamos en Valladolid, cuando cayó enfermo, muy enfermo. Un afamado
médico de aquella ciudad me dijo que no viviría mucho tiempo. Él, sin
embargo, siempre que experimentaba algún alivio, se creía restablecido
por completo. En uno de sus más graves ataques, hallándonos en
Salamanca, me dijo: «Te robé, hija mía, para hacerte instrumento de la
horrible cólera que me devora. Pero Dios, que no consiente sin duda la
perdición de mi alma, me ha llenado de un profundo y celeste amor que
antes no conocía. Has sido para mí el ángel de la guarda, la imagen viva
de la bondad divina, y no sólo me has consolado, sino que me has
convertido. Bendita seas mil veces por esta savia nueva que has dado a
mi triste vida. Pero he cometido un crimen: tú no me perteneces; entré
como un ladrón en el huerto ajeno y robé esta flor\ldots{} No, no puedo
retenerte ni un momento más al lado mío contra tu gusto.» El infeliz me
decía esto con tanta sinceridad, que me sentí inclinada a amarle más.
Luego siguió diciéndome: «Si tienes compasión de mí, si tu alma generosa
se resiste a dejarme en esta soledad, enfermo y aborrecido, acompáñame y
asísteme, pero que sea por voluntad tuya y no por violencia mía. Déjame
que te bese mil veces, y márchate después si no quieres estar a mi
lado.» No le contesté de otro modo que abrazándole con todas mis fuerzas
y llorando con él. ¿Qué podía, qué debía hacer?

---Quedarte.

---Aquélla era la ocasión más propia para confiarle mis deseos. Después
de repetir que no le abandonaría, díjele que debía reconciliarse con mi
madre. Recibió al principio muy mal la advertencia, mas tanto rogué y
supliqué que al fin consintió en escribir una carta. Empecela yo, y como
en ella pusiera no recuerdo qué palabras pidiendo perdón, enfureciose
mucho y dijo:---«¡Pedir perdón, pedirle perdón! Antes morir.»---Por
último, quitando y poniendo frases, di fin a la epístola; mas al día
siguiente le vi bastante cambiado en sus disposiciones conciliadoras, y
¿qué creerás, amigo mío?\ldots{} Pues rompió la carta, diciéndome: «Más
adelante la escribiremos, más adelante. Aguardemos un poco.» Esperé con
santa resignación, y hallándonos en Plasencia, hice una nueva tentativa.
Él mismo escribió la carta, empleando en ella no menos de cuatro horas,
y ya la íbamos a enviar a su destino cuando uno de esos aborrecidos
hombres que le acompañan entró diciéndole que la policía francesa le
buscaba y le perseguía por gestiones de una alta señora de Madrid. ¡Ay,
Gabriel! Cuando tal supo, renovose en él la cólera y amenazó a todo el
género humano. No necesito decirte que ni enviamos la carta, ni habló
más del asunto en algunos días. Pero yo insistía en mi propósito. Al
volver a Salamanca le manifesté la necesidad de la reconciliación;
enfadose conmigo, díjele que me marcharía a Madrid, abrazome, lloró,
gimió, arrojose a mis pies como un insensato, y al fin, hijo, al fin,
escribimos la tercera carta, la escribí yo misma. Por último, mi adorada
madre iba a saber noticias de su pobre hija. ¡Ay! aquella noche mi padre
y yo charlamos alegremente, hicimos dulces proyectos; maldijimos juntos
a todos los masones de la tierra, a las revoluciones y a las guillotinas
habidas y por haber; nos regocijamos con supuestas felicidades que
habían de venir; nos contamos el uno al otro todas las penas de nuestra
pasada vida\ldots{} pero al siguiente día\ldots{}

---Me presenté yo\ldots{} ¿no es eso?

---Eso es\ldots{} ya conoces su carácter\ldots{} Cuando te vio y conoció
que ibas enviado por mi madre, cuando le injuriaste\ldots{} Su ira era
tan fuerte aquel día que me causó miedo.---«Ahí lo tienes, decía, yo me
dispongo a ser bueno con ella, y ella envía contra mí la policía
francesa para mortificarme, y un ladrón para privarme de tu compañía. Ya
lo ves, es implacable\ldots{} A Francia, nos iremos a Francia, vendrás
conmigo. Esa mujer acabó para mí y yo para ella\ldots» Lo demás lo sabes
tú y no necesito decírtelo. ¡Esta mañana creímos morir aquí! ¡Cuánto he
padecido en este horrible Babilafuente viéndole enfermo, tan enfermo que
no se restablecerá más, viéndonos amenazados por el populacho que quería
entrar para despedazarnos!\ldots{} Y todo ¿por qué? Por la masonería,
por esas simplezas que a nada conducen.

---A algo conducen, querida mía, y la semilla que tu padre y otros han
sembrado, dará algún día su fruto. Sabe Dios cuál será.

---Pero él no es ateo, como otros, ni se burla de Dios. Verdad que suele
nombrarle de un modo extraño, así como el Ser Supremo, o cosa parecida.

---Llámese Dios o Ser Supremo---exclamé volviendo a aprisionar entre mis
manos las de mi adorada amiga,---ello es que ha hecho obras acabadas y
perfectas, y una de ellas eres tú, que me confundes, que me empequeñeces
y anonadas más cuanto más te trato y te hablo y te miro.

---Eres tonto de veras, pues ¿qué he hecho que no sea
natural?---preguntome sonriendo.

---Para los ángeles es natural existir sin mancha, inspirar las buenas
acciones, ensalzar a Dios, llevar al cielo las criaturas, difundir el
bien por el mundo pecador. ¿Que qué has hecho? Has hecho lo que yo no
esperaba ni adivinaba, aunque siempre te tuve por la misma bondad; has
amado a ese infeliz, al más infeliz de los hombres, y este prodigio que
ahora, después de hecho, me parece tan natural, antes me parecía una
aberración y un imposible. Tú tienes el instinto de lo divino y yo no:
tú realizas con la sencillez propia de Dios las más grandes cosas y a mí
no me corresponde otro papel que el de admirarlas después de hechas,
asombrándome de mi estupidez por no haberlas comprendido\ldots{}
¡Inesilla, tú no me quieres, tú no puedes quererme!

---¿Por qué dices eso?---preguntó con candor.

---Porque es imposible que me quieras, porque yo no te merezco.

Al decir esto, estaba tan convencido de mi inferioridad, que ni siquiera
intenté abrazarla, cuando cruzando ella las defensoras manos, parecía
dejarme el campo libre para aquel exceso amoroso.

---De veras, parece que eres tonto.

---Pero si tu corazón no sabe sino amar, si no sabe otra cosa, aunque de
mil modos le enseñe el mundo lo contrario, algo habrá para mí en un
rinconcito.

---¿Un rinconcito\ldots? ¿De qué tamaño?

---¡Qué feliz soy! Pero te digo la verdad, quisiera ser desgraciado.

No me contestó sino riéndose, burlándose de mí con un
descaro\ldots---Quiero ser desgraciado para que me ames como has amado a
tu padre, para que te desvivas por mí, para que te vuelvas loca por mí,
para que\ldots{} ¿Pero te ríes, todavía te ríes? ¿Acaso estoy diciendo
tonterías?

---Más grandes que esta casa.

---Pero, hija, si estoy aturdido. Dime tú, que todo lo sabes, si hay
alguna manera extraordinaria de querer, una manera nueva,
inaudita\ldots{}

---Así, así siempre, basta\ldots{} Ni es preciso tampoco que seas
desgraciado. No, dejémonos de desgracias, que bastantes hemos tenido.
Pidamos a Dios que no haya más batallas en que puedas morir.

---¡Yo quiero morir!---exclamé sintiendo que el puro y extremado afecto
llevaba mi mente a mil raras sutilezas y tiquis miquis, y mi corazón a
incomprensibles y quizás ridículos antojos.

---¡Morir!---exclamó ella con tristeza.---¿Y a qué viene ahora eso? ¿Se
puede saber, señor mío querido?

---Quiero morir para verte llorar por mí\ldots{} pero en verdad esto es
absurdo, porque si muriera, ¿cómo podría verte? Dime que me amas,
dímelo.

---Esto sí que está bueno. Al cabo de la vejez\ldots{}

---Si nunca me lo has dicho\ldots{} Puede que quieras sostener que me lo
has dicho.

---¿Que no?---exclamó con jovialidad encantadora.---Pues no.

No sé qué más iba a decir ella; pero indudablemente pensó decir algo,
más dulce para mí que las palabras de los ángeles, cuando sonó en la
estancia una ronca voz.

---No, no te vas, paloma, sin abrazar a tu marido---exclamé estrujando
aquel lindo cuerpo, que se escapó de mis brazos para volar al lado del
enfermo.

\hypertarget{xxx}{%
\chapter{XXX}\label{xxx}}

Acerqueme a la puerta de la triste alcoba. Santorcaz no me veía, porque
su observación estaba fatigada y torpe a causa del mal, y la estancia
medio a oscuras.

---Alguien estaba ahí---dijo el enfermo besando las manos de su
hija.---Me pareció sentir la voz de ese tunante de Gabriel.

---Padre, no hables mal de los que nos han hecho un beneficio, no
tientes a Dios, no le provoques.

---Yo también le he hecho beneficios, y ya ves cómo me paga:
prendiéndome.

---Araceli es un buen muchacho.

---¡Sabe Dios lo que harán conmigo esos verdugos!---exclamó el anciano
dando un suspiro.---Esto se acabó, hija mía.

---Se acabaron, sí, las locuras, los viajes, las logias que sólo sirven
para hacer daño---afirmó Inés abrazando a su padre.---Pero subsistirá el
amor de tu hija, y la esperanza de que viviremos todos, todos felices y
tranquilos.

---Tú vives de dulces esperanzas---dijo---yo de tristes o funestos
recuerdos. Para ti se abre la vida; para mí, lo contrario. Ha sido tan
horrible, que ya deseo se cierre esa puerta negra y sombría, dejándome
fuera de una vez\ldots{} Hablas de esperanzas; ¿y si estos déspotas me
encierran en una cárcel, si me envían a que muera a cualquiera de esos
muladares del África\ldots?

---No te llevarán, respondo de que no te llevarán, padrito.

---Pero cualquiera que sea mi suerte, será muy triste, niña de mi
alma\ldots{} Viviré encerrado, y tú\ldots{} ¿tú qué vas a hacer? Te
verás obligada a abandonarme\ldots{} Pues qué, ¿vas a encerrarte en un
calabozo?

---Sí: me encerraré contigo. Donde tú estés allí estaré yo---dijo la
muchacha con cariño.---No me separaré de ti, no te abandonare jamas, ni
iré\ldots{} no, no iré a ninguna parte donde tú no puedas ir también.

No oí voz alguna, sino los sollozos del pobre enfermo.

---Pero en cambio, padrito---continuó ella en tono de amonestación
afectuosa,---es preciso que seas bueno, que no tengas malos
pensamientos, que no odies a nadie, que no hables de matar gente, pues
Dios tiene buena mano para hacerlo; que desistas de todas esas
majaderías que te han trastornado la cabeza, y no pierdas la
tranquilidad y la salud porque haya un rey de más o de menos en el
mundo; ni hagas caso de los frailes ni de los nobles, los cuales, padre
querido, no se van a suprimir y a aniquilarse porque tú lo desees, ni
porque así lo quiera el mal humor del Sr.~Canencia, del Sr.~Monsalud y
del Sr. Ciruelo\ldots{} He aquí tres que hablan mal de los nobles, de
los poderosos y de los reyes, porque hasta ahora ningún rey, ni ningún
señor han pensado en arrojarles un pedazo de pan para que callen, y otro
para que griten en favor suyo\ldots{} ¿Conque serás bueno? ¿Harás lo que
te digo? ¿Olvidarás esas majaderías?\ldots{} ¿Me querrás mucho a mí y a
todos los que me quieren?

Diciendo esto, arreglaba las ropas del lecho, acomodaba en las almohadas
la venerable y hermosa cabeza de Santorcaz, destruía los dobleces y
durezas que pudieran incomodarle, todo con tanto cariño, solicitud,
bondad y dulzura, que yo estaba encantado de lo que veía. Santorcaz
callaba y suspiraba, dejándose tratar como un chico. Allí la hija
parecía más que una hija una tierna madre, que se finge enojada con el
precioso niño porque no quiere tomar las medicinas.

---Me convertirás en un chiquillo, querida---dijo el enfermo.---Estoy
conmovido\ldots{} quiero llorar. Pon tu mano sobre mi frente para que no
se me escape esa luz divina que tengo dentro del cerebro\ldots{} pon tu
mano sobre mi corazón y aprieta. Me duele de tanto sentir. ¿Has dicho
que no te separarás de mí?

---No, no me separaré.

---¿Y si me llevan a Ceuta?

---Iré contigo.

---¡Irás conmigo!

---Pero es preciso ser bueno y humilde.

---¿Bueno? ¿Tú lo dudas? Te adoro, hija mía. Dime que soy bueno, dime
que no soy un malvado y te lo agradeceré más que si me vinieras a llamar
de parte del Ser Sup\ldots{} de parte de Dios, decimos los cristianos.
Si tú me dices que soy un hombre bueno, que no soy malo, tendré por
embusteros a los que se empeñan en llamarme malvado.

---¿Quién duda que eres bueno? Para mí al menos.

---Pero a ti te he hecho algún daño.

---Te lo perdono, porque me amas, y sobre todo porque me sacrificas tus
pasiones, porque consientes que sea yo la destinada a quitarte esas
espinas que desde hace tanto tiempo tienes clavadas en el corazón.

---¡Y cómo punzan!---exclamó con profunda pena el infeliz masón.---Sí,
quítamelas, quítamelas todas con tus manos de ángel; quítalas una a una,
y esas llagas sangrientas se restañarán por sí\ldots{} ¿De modo que yo
soy bueno?

---Bueno, sí: yo lo diré así a quien crea lo contrario, y espero que se
convencerán cuando yo lo diga. Pues no faltaba más\ldots{} La verdad es
lo primero. Ya verás cuánto te van a querer todos, y qué buenas cosas
dirán de ti. Has padecido: yo les contaré todo lo que has padecido.

---Ven---murmuró Santorcaz con voz balbuciente, alargando los brazos
para coger en sus manos trémulas la cabeza de su hija.---Trae acá esa
preciosa cabeza que adoro. No es una cabeza de mujer, es de ángel. Por
tus ojos mira Dios a la tierra y a los hombres, satisfecho de su obra.

El anciano cubrió de besos la hermosa frente, y yo por mi parte no
ocultaré que deseaba hacer otro tanto. En aquel momento di algunos pasos
y Santorcaz me vio. Advertí súbita mudanza en la expresión de su
semblante, y me miró con disgusto.

---Es Gabriel, nuestro amigo, que nos defiende y nos protege---dijo
Inés,---¿por qué te asustas?

---Mi carcelero---murmuró Santorcaz con tristeza\ldots---Me había
olvidado de que estoy preso.

---No soy carcelero, sino amigo---afirmé adelantándome.

---Sr.~Araceli---continuó él con voz grave,---¿a dónde me llevan? ¡Oh,
miserable de mí! Malo es caer en las garras de los satélites del
despotismo\ldots{} no, no, hija mía, no he dicho nada; quise decir que
los soldados\ldots{} no puedo negar que odio un poquillo a los soldados,
porque sin ellos, ya ves, sin ellos no podrían los reyes\ldots{}
¡malditos sean los reyes!\ldots{} no, no, a mí no me importa que haya
reyes, hija mía; allá se entiendan. Sólo que\ldots{} francamente, no
puedo menos de aborrecer un poco a ese muchacho que quiso separarte de
mí. Ya se ve, le mandaban sus amos\ldots{} estos militares son gente
servil que los grandes emplean para oprimir a los hijos del
pueblo\ldots{} No le puedo ver, ni tú tampoco, ¿es verdad?

---No sólo le puedo ver, sino que le estimo mucho.

---Pues que entre\ldots{} Araceli\ldots{} también yo te estimé en otro
tiempo. Inés dice que eres un buen muchacho\ldots{} Será preciso
creerlo\ldots{} Puesto que ella te estima, ¿sabes lo que yo haría?
exceptuarte a ti solo, a ti solito; ponerte a un lado, y a todos los
demás enviarles a la guillot\ldots{} no, no he dicho nada\ldots{} Si
otros la quieren levantar, háganlo en buen hora; yo no haré más que ver
y aplaudir\ldots{} No, no, no aplaudiré tampoco: váyanse al diablo las
guillotinas.

---Padre---dijo Inés,---da la mano a Araceli, que se marchará a sus
quehaceres, y ruégale que vuelva a vernos después. ¡Ay! dicen que va a
darse una batalla: ¿no sientes que le suceda alguna desgracia?

---Sí, seguramente---dijo Santorcaz estrechándome la mano.---¡Pobre
joven! La batalla será muy sangrienta, y lo más probable es que muera en
ella.

---¿Qué dices, padre?---preguntó Inés con terror.

---La mejor batalla del mundo, hija mía, será aquélla en que perezcan
todos los soldados de los dos ejércitos contendientes.

---¡Pero él no, él no! Me estás asustando.

---Bueno, bueno, que viva él\ldots{} que viva Araceli. Joven, mi hija te
estima, y yo\ldots{} yo también\ldots{} también te estimo. Así es que
Dios hará muy bien en conservar tu preciosa vida. Pero no servirás más a
los verdugos del linaje humano, a los opresores del pueblo, a los que
engordan con la sangre del pueblo, a los pícaros frailes
y\ldots---¡Jesús! estás hablando como Canencia, ni más ni menos.

---No he dicho nada; pero este Araceli\ldots{} a quien estimo\ldots{}
nos aborrece, querida mía, quiere separarnos, es agente y servidor de
una persona\ldots{}

---A quien estimas también, padre.

---De una persona\ldots---continuó el masón, poniéndose tan pálido que
parecía un cadáver.

---A quien amas, padre---añadió la muchacha rodeando con sus brazos la
cabeza del pobre enfermo,---a quien pedirás perdón\ldots{} por\ldots{}

El rostro de Santorcaz encendiose de repente con fuerte congestión; sus
ojos despidieron rayo muy vivo, incorporose en el lecho y estirando los
brazos y cerrando los puños y frunciendo el terrible ceño gritó:

---¡Yo!\ldots{} pedirle perdón\ldots{} pedirle perdón yo\ldots{} ¡Jamás,
jamás!

Diciendo esto cayó en el lecho como cuerpo del que súbitamente y con
espanto huye la vida.

Inés y yo acudimos a socorrerle. Balbucía frases ardorosas\ldots{}
llamaba a Inés creyéndola ausente, la miraba con extravío; me despedía
con gritos y amenazas; y, finalmente, se tranquilizó cayendo en pesado
sopor.

---Otra vez será---me dijo Inés con los ojos llenos de lágrimas.---No
desconfío. Haz lo que dijimos. Escríbele esta tarde mismo.

---Le escribiré y vendrá en seguida a Salamanca. Prepárate a marchar
allá con tu enfermo.

\hypertarget{xxxi}{%
\chapter{XXXI}\label{xxxi}}

Haciendo mucho ruido, llamándome a voces y azotando con su látigo las
puertas y los muebles, entró en la casa miss Fly. Recibila en la sala y
al verme sonrió con gracia incomparable, no exenta en verdad de
coquetería. Llamó mi atención ver que se había acicalado y compuesto,
cosa verdaderamente extraña en aquel lugar y ocasión. Su rostro
resplandecía de belleza y frescura. Habíase peinado cual si tuviese a
mano los más delicados enseres de tocador, y el vestido, limpio ya de
polvo y lodo, disimulaba sus desgarrones y arrugas no sé por qué arte
singular, sólo revelado a las mujeres. ¿Por qué no decirlo? Detesto las
gazmoñerías y melindres. Sí, lo diré: Athenais estaba encantadora,
hechicera, lindísima.

Como le manifestase mi sorpresa por aquella restauración de su
interesante persona, me dijo:

---Caballero Araceli, después que vuestros soldados han apagado el
incendio, quedó un poco de agua para mí. En casa de unos aldeanos me
proporcionaron lo preciso para peinarme\ldots{} Pero, señor comandante,
¿así cumplís con vuestros deberes? ¿No estaréis mejor al frente de
vuestras tropas? Hace un rato que ha llegado Leith con su división, y
pregunta por vos.

Al saber la noticia, no quise detenerme. Despedime de Inés, y después de
asegurar bien la entrada de la casa y de encomendar a Tribaldos que
cuidase a los dos prisioneros, bajé a la plaza, donde miss Fly se separó
de mí sin motivo aparente. Empezaban a llegar tropas inglesas. El
general Leith, a quien indiqué que España me había mandado proseguir,
cuando llegaron los ingleses me ordenó que esperase hasta la noche.

---Es imposible perseguir a los franceses de cerca---dijo.---Van muy
adelantados, y nos será difícil hacerles daño. Nuestras tropas están
cansadas.

Quedeme allí no sin gozo, y dispuse lo necesario para que Santorcaz y su
hija fuesen trasladados a Salamanca. Felizmente regresaba aquella tarde
para quedar allí de guarnición, Buenaventura Figueroa, mi más íntimo y
querido amigo, y le di instrucciones prolijas sobre lo que debía hacer
con mis prisioneros en la ciudad y durante el viaje. Verificose este por
la noche en un convoy que se envió a \emph{Roma la chica}, y no sin
trabajo logré un carromato de regular comodidad, en cuyo interior
acomodé a padre e hija, acompañados de Tribaldos y de buen repuesto de
víveres para el viaje. Quise darles también dinero, mas rehusolo Inés, y
a la verdad no lo necesitaban, porque el Sr.~Santorcaz (no sé si lo he
dicho), que un año antes heredara íntegro su patrimonio, poseía regular
hacienda, sobrada para su modesto traer.

Di también a Inés instrucciones para que contribuyese a impedir nuevas
salidas de su infeliz padre al campo de Montiel de las masónicas
aventuras, y ella prometiome con inequívoca seguridad que le
encarcelaría convenientemente sin mortificarle, con lo cual, muy
apenados nos despedimos los dos, yo por aquella nueva separación, cuyos
límites no sabía, y ella por presentimientos del peligro a que expuesto
quedaba en la terrible campaña emprendida. En esto, y en escribir a la
condesa lo que el lector supone, entretuve gran parte de las últimas
horas del día.

Partimos al amanecer del siguiente, persiguiendo a los franceses, que no
pararon hasta pasar el Duero por Tordesillas, extendiéndose hasta
Simancas. Allí reforzó Marmont su ejército con la división de Bonnet, y
nosotros le aguardamos en la orilla izquierda vigilando sus movimientos.
La cuestión era saber por qué sitio quería el francés pasar el río, para
venir al encuentro del ejército aliado, cuyo cuartel general estaba en
La Seca.

No quería Marmont, como es fácil suponer, darnos gusto, y sin avisarnos,
cosa muy natural también, partió de improviso hacia Toro\ldots{} ¡En
marcha todo el mundo hacia la izquierda, ingleses, españoles, lusitanos,
en marcha otra vez hacia el Guareña y hacia los perversos pueblos de
Babilafuente y Villorio!

---¡Y a esto llaman hacer la guerra!---decía uno.---Por el mucho
ejercicio que hacen, tienen tan buenas piernas los ingleses. Ahora
resultará que Marmont no acepta tampoco la batalla en el Guareña y lo
buscaremos en el Pisuerga, en el Adaja o tal vez en el Manzanares o en
el Abroñigal a las puertas de Madrid.

Tan sólo resultó que después de dos semanas de marchas y contramarchas,
nos encontramos otra vez en las inmediaciones de Salamanca. Pero lo más
gracioso fue cuando bailamos el minueto, como decían los españoles, pues
aconteció que ambos ejércitos marcharon todo un día paralelamente, ellos
sobre la izquierda y nosotros sobre la derecha, viéndonos muy bien a
distancia de medio tiro de cañón y sin gastar un cartucho. Esto pasó no
muy lejos de Salamanca; y cuando nos detuvimos en San Cristóbal, allí
eran de ver las burlas motivadas por la tal maniobra y marcha
estratégica que los chuscos calificaban de contradanza.

Desde San Cristóbal quise ir a Salamanca: pero me fue imposible, porque
no se concedían licencias largas ni cortas. Tuve, sin embargo, el gusto
de saber que nada singular había ocurrido en la casa de la calle del
Cáliz durante mi ausencia y las marchas y minuetos del ejército
aliado\ldots{} En cuanto a miss Fly (me apresuro a nombrarla, porque
oigo una misma pregunta en los labios de cuantos me escuchan), me había
honrado no pocas veces con su encantadora palabra durante los viajes a
Tordesillas, a la Nava y al Guareña; pero siempre en cortas y
disimuladas entrevistas, cual si existiese algún desconocido estorbo,
algún impedimento misterioso de su antes ilimitada libertad. En estas
breves entrevistas advertía siempre en ella sin igual dulzura y
melancólico abandono, y además una admiración injustificada hacia todas
mis acciones, aunque fuesen de las más comunes e insignificantes.

Por lo demás si las entrevistas pecaban de cortas, eran frecuentísimas.
No hacíamos alto en punto alguno, sin que se me presentase Athenais,
cual mi propia sombra y recatadamente me hablase, diciéndome por lo
general cosas alambicadas y sutiles, cuando no melifluas y apasionadas.
La más refinada cortesía y un excelente humor de bromas inspiraban mis
contestaciones. Regalábame a cada momento mil monerías, golosinas o
cachivaches de poco valor, que adquiría en los diversos pueblos de la
carrera.

Entre tanto (suplico a mis oyentes se fijen bien en esto, porque sirve
de lamentable antecedente a uno de los principales contratiempos de mi
vida), yo notaba que no se había disipado entre mis compañeros ingleses
y españoles la infundada sospecha que el viaje de Athenais a Salamanca
despertara. En suma, la Pajarita había vuelto al cuartel general, y mi
buena opinión y fama de caballerosidad continuaban tan problemáticas
como el día que aparecí en Bernuy. En dos ocasiones en que tuve el alto
honor de hablar con el señor duque, experimenté mortal pena, hallándole
no sólo desdeñoso sino en extremo austero y desapacible conmigo. Los
espejuelos del coronel Simpson despedían rayos olímpicos contra mí y en
general cuantas personas conocía en las filas inglesas demostraban de
diversos modos poca o ninguna afición a mi honrada persona.

---Sr.~Araceli, Sr.~Araceli---me dijo Athenais presentándose de
improviso ante mí el 21 de Julio cuando acabábamos de ocupar el cerro
comúnmente llamado Arapil Chico,---venid a mi lado. Simpson no ha salido
aún de Salamanca. ¿Os ha pasado algo desde ayer que no nos hemos visto?

---Nada, señora, no me ha pasado nada. ¿Y a usted?

---A mí sí; pero ya os lo contaré más adelante. ¿Por qué me miráis de
ese modo?\ldots{} Vos también dais en creer, como los demás, que estoy
triste, que estoy pálida, que he cambiado mucho\ldots{}

---En efecto, miss Fly, se me figura que esa cara no es la misma.

---No me siento bien---dijo con sonrisa graciosa.---No sé lo que
tengo\ldots{} ¡Ah!, ¿no sabéis? Dicen que va a darse una gran batalla.

---No lo dudo. Los franceses están hacia Cavarrasa. ¿Cuándo será?

---Mañana\ldots{} Parece que os alegráis---dijo mostrando un temor
femenino que me sorprendió, conociendo como conocía su varonil arrojo.

---Y usted también se alegrará, señora. Un alma como la de usted, para
sostenerse a su propia altura, necesita estos espectáculos grandiosos,
el inmenso peligro seguido de la colosal gloria. Nos batiremos, señora,
nos batiremos con el Imperio, con el enemigo común, como dicen en
Inglaterra, y le derrotaremos.

Athenais no me contestó, como esperaba, con ningún arrebato de
entusiasmo, y la poesía de los romances parecía haberse replegado con
timidez y vergüenza quizás en lo más escondido de su alma.

---Será una gran batalla y ganaremos---dijo con
abatimiento;---pero\ldots{} morirá mucha gente. ¿No os ocurre que podéis
morir vos?

---¿Yo?\ldots{} ¿y qué importa? ¿Qué importa la vida de un miserable
soldado, con tal que quede triunfante la bandera?

---Es verdad; pero no debéis exponeros\ldots---dijo con cierta
emoción.---Dicen que la división española no se batirá.

---Señora, no conozco a usted, no es usted miss Fly.

---Voy creyendo lo que decís---afirmó clavando en mí los dulces ojos
azules;---voy creyendo que no soy yo miss Fly\ldots{} Oíd bien, Araceli,
lo que voy a deciros. Si no entráis en fuego mañana, como espero,
avisádmelo\ldots{} Adiós, adiós.

---Pero aguarde usted un momento, miss Fly---dije procurando detenerla.

---No, no puedo. Sois muy indiscreto\ldots{} Si supierais lo que
dicen\ldots{} adiós, adiós.

Dando algunos pasos hacia ella, la llamé repetidas veces; mas en el
mismo instante vi un coche o silla de postas que se paraba delante de mí
en mitad del camino; vi que por la portezuela aparecía una cara, una
mano, un brazo. Si era la condesa\ldots{} ¡Dios poderoso, qué inmensa
alegría! Era la condesa, que detenía su coche delante de mí, que me
buscaba con la vista, que me llamaba con un lindo gesto, que iba a decir
sin duda dulcísimas cosas. Corrí hacia ella loco de alegría.

\hypertarget{xxxii}{%
\chapter{XXXII}\label{xxxii}}

Antes de referir lo que hablamos, conviene que diga algo del lugar y
momento en que tales hechos pasaban, porque una cosa y otra interesan
igualmente a la historia y a la relación de los sucesos de mi vida que
voy refiriendo. El 21 por la tarde pasamos el Tormes, los unos por el
puente de Salamanca, los otros por los vados inmediatos. Los franceses,
según todas las conjeturas, habían pasado el mismo río por Alba de
Tormes, y se encontraban al parecer en los bosques que hay más allá de
Cavarrasa de Arriba. Formamos nosotros una no muy extensa línea cuya
izquierda se apoyaba junto al vado de Santa Marta, y la derecha en el
Arapil Chico, junto al camino de Madrid. Una pequeña división inglesa
con algunas tropas ligeras ocupaba el lugar de Cavarrasa de Abajo, punto
el más avanzado de la línea anglo-hispano-portuguesa.

En la falda del Arapil Chico, y al borde del camino, fue donde se me
apareció Athenais, que volvía a caballo de Cavarrasa, y pocos instantes
después la señora condesa, mi adorada protectora y amiga. Corrí hacia
ella, como he dicho, y con la más viva emoción besé sus hermosas manos
que aún asomaban por la portezuela. El inmenso gozo que experimenté
apenas me dejó articular otras voces que las de «madre y señora mía,»
voces en que mi alma, con espontaneidad y confianza sumas, esperaban
iguales manifestaciones cariñosas de parte de ella. Mas con amargura y
asombro advertí en los ojos de la condesa desdén, enojo, ira, ¡qué sé
yo!\ldots{} una severidad inexplicable que me dejó absorto y helado.

---¿Y mi hija?---preguntó con sequedad.

---En Salamanca, señora---repuse.---No podría usted llegar más a tiempo.
Tribaldos, mi asistente, acompañará a usted. Ha sido casualidad que nos
hayamos encontrado aquí.

---Ya sabía que estabas en este sitio que llaman el Arapil Chico---me
dijo con el mismo tono severo, sin una sonrisa, sin una mirada cariñosa,
sin un apretón de manos .---En Cavarrasa de Abajo, donde me detuve un
instante, encontré a sir Tomás Parr, el cual me dijo dónde estabas, con
otras cosas acerca de tu conducta, que me han causado tanto asombro como
indignación.

---¡Acerca de mi conducta, señora!---exclamé con dolor tan vivo como si
una hoja de acero penetrara en mi corazón.---Yo creía que en mi conducta
no había nada que pudiera desagradar a usted.

---Conocí en Cádiz a sir Tomás Parr, y es un caballero incapaz de
mentir---añadió ella con indecible resplandor de ira en los ojos que
tanta ternura habían tenido en otro tiempo para mí.---Has seducido a una
joven inglesa, has cometido una iniquidad, una violencia, una acción
villana.

---¡Yo, señora!, ¡yo!\ldots{} ¿Este hombre honrado que ha dado tantas
pruebas de su lealtad\ldots? ¿Este hombre ha hecho tales maldades?

---Todos lo dicen\ldots{} No me lo ha dicho sólo sir Tomás Parr, sino
otros muchos; me lo dirá también Wellesley.

---Pues si Wellesley lo afirmara---repliqué con desesperación,---si
Wellesley lo afirmara, yo le diría\ldots{}

---Que miente\ldots{}

---No: el primer caballero de Inglaterra, el primer general de Europa no
puede mentir; es imposible que el duque diga semejante cosa.

---Hay hechos que no pueden disimularse---añadió con pena,---que no
pueden desfigurarse. Dicen que la persona agraviada se dispone a pedir
que se te obligue al cumplimiento de las leyes inglesas sobre el
matrimonio.

Al oír esto, una hilaridad expansiva y una indignación terrible cruzaron
sus diversos efectos en mi alma, como dos rayos que se encuentran al
caer sobre un mismo objeto, y por un instante se lo disputan. Me reí y
estuve a punto de llorar de rabia.

---Señora, me han calumniado, es falso, es mentira que yo\ldots---grité
introduciendo por la portezuela del coche primero la cabeza y después
medio cuerpo .---Me volveré loco si usted, si esta persona a quien
respeto y adoro a quien no podré jamás engañar, da valor a tan infame
calumnia.

---¿Con que es calumnia?\ldots---dijo con verdadero dolor.---Jamás lo
hubiera creído en ti\ldots{} Vivimos para ver cosas horribles\ldots{}
Pero dime, ¿veré a mi hija en seguida?

---Repito que es falso. Señora, me está usted matando, me impulsará
usted a extremos de locura, de desesperación.

---¿Nadie me estorbará que la recoja, que la lleve conmigo?---preguntó
con afán y sin hacer caso del frenesí que me dominaba.---Que venga tu
asistente. No puedo detenerme. ¿No decías en tu carta que todo estaba
arreglado? ¿Ha muerto ese verdugo? ¿Está mi hija sola?\ldots{} ¿Me
espera?\ldots{} ¿Puedo llevármela?\ldots{} Responde.

---No sé, señora; no sé nada; no me pregunte usted nada---dije
confundido y absorto.---Desde el momento que usted duda de mí\ldots{}

---Y mucho\ldots{} ¿En quién puede tenerse confianza?\ldots{} Déjame
seguir\ldots{} Tú ya no eres el mismo para mí.

---Señora, señora, no me diga usted eso, porque me muero---exclamé con
inmensa aflicción.

---Bueno, si eres inocente, tiempo tienes de probármelo.

---No\ldots{} no\ldots{} Mañana se da una gran batalla. Puedo morir.
Moriré irritado y me condenaré\ldots{} ¡Mañana!, ¡sabe Dios dónde estaré
mañana! Usted va a Salamanca, verá y hablará a su hija; entre las dos
fraguarán una red de sospechas y falsos supuestos, donde se enmarañe
para siempre la memoria del infeliz soldado, que agonizará quizás dentro
de algunas horas en este mismo sitio donde nos encontramos. Es posible
que no nos veamos más\ldots{} Estamos en un campo de batalla. ¿Distingue
usted aquellos encinares que hay hacia abajo? Pues allí detrás están los
franceses. ¡Cuarenta y siete mil hombres, señora! Mañana este sitio
estará cubierto de cadáveres. Dirija usted la vista por estos contornos.
¿Ve usted esa juventud de tres naciones? ¿Cuántos de estos tendrán vida
mañana? Me creo destinado a perecer, a perecer rabiando, porque
precipitará mi muerte la idea de haber perdido el amor de las dos
personas a quienes he consagrado mi vida.

Mis palabras, ardientes como la voz de la verdad, hicieron algún efecto
en la condesa, y la observé suspensa y conmovida. Tendió la vista por el
campo, ocupado por tanta tropa, y luego cubriose el rostro con las
manos, dejándose caer en el fondo del coche.

---¡Qué horror!---dijo.---¡Una batalla! ¿No tienes miedo?

---Más miedo tengo a la calumnia.

---Si pruebas tu inocencia, creeré que he recobrado un hijo perdido.

---Sí, sí, lo recobrará usted---afirmé.---¿Pero no basta que yo lo diga,
no basta mi palabra?\ldots{} ¿Nos conocemos de ayer? ¡Oh! Si a Inés se
le dijera lo que a Vd. han dicho, no lo creería. Su alma generosa me
habría absuelto sin oírme.

Una voz gritó:

---¡Ese coche, adelante o atrás!

---Adiós---dijo la condesa,---me echan de aquí.

---Adiós, señora---respondí con profunda tristeza.---Por si no nos vemos
más, nunca más, sepa usted que en el último día de mi vida conservo
todos, absolutamente todos los sentimientos de que he hecho gala en
todos los instantes de mi vida ante usted y ante otra persona que a
entrambos nos es muy cara. Agradezco a usted, hoy como ayer, el amor que
me ha mostrado, la confianza que ha puesto en mí, la dignidad que me ha
infundido, la elevación que ha dado a mi conciencia\ldots{} No quiero
dejar deudas\ldots{} Si no nos vemos más\ldots{}

El coche partió, obligado a ello por una batería a la cual era forzoso
ceder el paso. Cuando dejé de ver a la condesa, llevaba ella el pañuelo
a los ojos para ocultar sus lágrimas.

Sofocado y aturdido por la pena angustiosa que llenaba mi alma, no
reparé que el cuartel general venía por el camino adelante en dirección
al Arapil Chico. El duque y los de su comitiva echaron pie a tierra en
la falda del cerro, dirigiendo sus miradas hacia Cavarrasa de Arriba.
Llamó el lord a los oficiales del regimiento de Ibernia, uno de los
establecidos allí, y habiéndome yo presentado el primero, me dijo:

---¡Ah! Es usted el caballero Araceli\ldots{}

---El mismo, mi general---contesté,---y si vuecencia me permite en esta
ocasión hablar de un asunto particular, le suplicaré que haga luz sin
pérdida de tiempo sobre las calumnias que pesan sobre mí después de mi
viaje a Salamanca. No puedo soportar que se me juzgue con ligereza, por
las hablillas de gente malévola.

Lord Wellington, ocupado sin duda con asunto más grave, apenas me hizo
caso. Después de registrar rápidamente todo el horizonte con su anteojo,
me dijo casi sin mirarme:

---Señor Araceli, no puedo contestar a usted que estoy decidido a que la
Gran Bretaña sea respetada.

Como yo no había dejado nunca de respetar a la Gran Bretaña, ni a las
demás potencias europeas, aquellas palabras que encerraba sin duda una
amenaza, me desconcertó un poco. Los oficiales generales que rodeaban al
duque, trabaron con él coloquio muy importante sobre el plan de batalla.
Pareciéronme entonces inoportunas y aun ridículas mis reclamaciones, por
lo cual un poco turbado, contesté de este modo:

---¡La Gran Bretaña! no deseo otra cosa que morir por ella.

---Brigadier Pack---dijo vivamente Wellington a uno de los que le
acompañaban ,---en la ayudantía del 23 de línea que está vacante, ponga
usted a este joven español, que desea morir por la Gran Bretaña.

---Por la gloria y honor de la Gran Bretaña---añadí.

El brigadier Pack me honró con una mirada de protectora simpatía.

---La desesperación---me dijo luego Wellington---no es la principal
fuente del valor; pero me alegaré de ver mañana al señor de Araceli en
la cumbre del Arapil Grande. Señor D. José Olawlor---añadió dirigiéndose
a su íntimo amigo, que le acompañaba,---creo que los franceses se están
disponiendo para adelantársenos mañana a ocupar el Arapil Grande.

El duque manifestó cierta inquietud, y por largo tiempo su anteojo
exploró los lejanos encinares y cerros hacia Levante. Poco se veía ya,
porque vino la noche. Los cuerpos de ejército seguían moviéndose para
ocupar las posiciones dispuestas por el general en jefe, y me separé de
mis compañeros de Ibernia y de la división española.

---Nosotros---me dijo España---vamos al lugar de Torres, en la extrema
derecha de la línea, más bien para observar al enemigo que para
atacarle. ¡Plan admirable! El general Picton y el portugués d'Urban
parece que están encargados de guardar el paso del Tormes, de modo que
la situación de los franceses no puede ser más desventajosa. No falta
más que ocupar el Arapil Grande.

---De eso se trata, mi general. La brigada Pack, a la cual desde hace un
momento pertenezco, amanecerá mañana, con la ayuda de Dios en la ermita
de Santa María de la Peña, y después\ldots{} Así lo exige el honor de la
Gran Bretaña\ldots{}

---Adiós, mi querido Araceli, pórtate bien.

---Adiós, mi querido general. Saludo a mis compañeros desde la cumbre
del Arapil Grande.

\hypertarget{xxxiii}{%
\chapter{XXXIII}\label{xxxiii}}

¡El Arapil Grande! Era la mayor de aquellas dos esfinges de tierra,
levantadas la una frente a la otra, mirándose y mirándonos. Entre las
dos debía desarrollarse al día siguiente uno de los más sangrientos
dramas del siglo, el verdadero prefacio de Waterloo, donde sonaron por
última vez las trompas de la Ilíada del Imperio. A un lado y otro del
lugar llamado de Arapiles se elevaban los dos célebres cerros, pequeño
el uno, grande el otro. El primero nos pertenecía, el segundo no
pertenecía a nadie en la noche del 21. No pertenecía a nadie por lo
mismo que era la presa más codiciada; y el leopardo de un lado y el
águila del otro le miraban con anhelo deseando tomarlo y temiendo
tomarlo. Cada cual temía encontrarse allí al contrario en el momento de
poner la planta sobre la preciosa altura.

A la derecha del Arapil Grande, y más cerca de nuestra línea, estaba
Huerta, y a la izquierda en punto avanzado, formando el vértice de la
cuña, Cavarrasa de Arriba. El de abajo, mucho más distante y a espaldas
del gran Arapil, estaba en poder de los franceses.

La noche era como de Julio, serena y clara. Acampó la brigada Pack en un
llano, para aguardar el día. Como no se permitía encender fuego, los
pobrecitos ingleses tuvieron que comer carne fría; pero las mujeres, que
en esto eran auxiliares poderosos de la milicia británica, traían de
Aldeatejada y aun de Salamanca fiambres muy bien aderezados, que con el
rom abundante devolvieron el alma a aquellos desmadejados cuerpos. Las
mujeres (y no bajaban de veinte las que vi en la brigada), departían con
sus esposos cariñosamente, y según pude entender, rezaban o se
fortalecían el espíritu con recuerdos de la Verde Erín y de la bella
Escocia. Gran martirio era para los \emph{highlanders}, que no se les
consintiera en aquel sitio tocar la zampoña, entonando las melancólicas
canciones de su país; y formaban animados corrillos, en los cuales me
metí bonitamente, para tener el extraño placer de oírles sin
entenderles. Érame en extremo agradable ver la conformidad y alegría de
aquella gente, transportada tan lejos de su patria, sostenida en su
deber y conducida al sacrificio por la fe de la misma patria\ldots{} Yo
escuchaba con delicia sus palabras y aun entendiendo muy poco de ellas,
creí comprender el espíritu de las ardientes conversaciones. Un escocés
fornido, alto, hermoso, de cabellos rubios como el oro y de mejillas
sonrosadas como una doncella, levantose al ver que me acercaba al
corrillo, y en chapurreado lenguaje, mitad español, mitad portugués, me
dijo:

---Señor oficial español, dignaos honrarnos aceptando este pedazo de
carne y este vaso de rom, y brindemos a la salud de España y de la vieja
Escocia.

---¡A la salud del rey Jorge III!---exclamé aceptando sin vacilar el
obsequio de aquellos valientes.

Sonoros \emph{hurras} me contestaron.

---El hombre muere y las naciones viven---dijo dirigiéndose a mí otro
escocés que llevaba bajo el brazo el enorme pellejo henchido de una
zampoña.---\emph{¡Hurra por Inglaterra!} ¡Qué importa morir! Un grano de
arena que el viento lleva de aquí para allá no significa nada en la
superficie del mundo. Dios nos está mirando, amigos, por los bellos ojos
de la madre Inglaterra.

No pude menos de abrazar al generoso escocés, que me estrechó contra su
pecho, diciendo:

---¡Viva España!

---¡Viva lord Wellington!---grité yo.

Las mujeres lloraban, charlando por lo bajo. Su lenguaje incomprensible
para mí, me pareció un coro de pájaras picoteando alrededor del nido.

Los escoceses se distinguían por el pintoresco traje de cuadros rojos y
negros, la pierna desnuda, las hermosas cabezas osiánicas cubiertas con
el sombrero de piel, y el cinto adornado con la guedeja que parecía
cabellera arrancada del cráneo del vencedor en las salvajes guerras
septentrionales. Mezclábanse con ellos los ingleses, cuyas casacas rojas
les hacían muy visibles a pesar de la oscuridad. Los oficiales envueltos
en capas blancas y cubiertos con los sombreritos picudos y emplumados,
nada airosos por cierto, semejaban pájaros zancudos de anchas alas y
movible cresta.

Con las primeras luces del día la brigada se puso en marcha hacia el
Arapil Grande. A medida que nos acercábamos, más nos convencíamos de que
los franceses se nos habían anticipado por hallarse en mejores
condiciones para el movimiento, a causa de la proximidad de su línea. El
brigadier distribuyó sus fuerzas, y las guerrillas se desplegaron. Los
ojos de todos fijábanse en la ermita situada como a la mitad del cerro,
y en las pocas casas dispersas, únicos edificios que interrumpían a
larguísimos trechos la soledad y desnudez del paisaje.

Subieron algunas columnas sin tropiezo alguno, y llegábamos como a cien
varas de Santa María de la Peña cuando la ondulación del terreno,
descendiendo a nuestros ojos a medida que adelantábamos, nos dejó ver,
primero, una línea de cabezas, luego una línea de bustos, después los
cuerpos enteros. Eran los franceses. El sol naciente que aparecía a
espaldas de nuestros enemigos nos deslumbraba, siendo causa de que los
viésemos imperfectamente. Un murmullo lejano llegó a nuestros oídos, y
del lado acá también los escoceses profirieron algunas palabras; no fue
preciso más para que brotase la chispa eléctrica. Rompiose el fuego. Las
guerrillas lo sostenían, mientras algunos corrieron a ocupar la ermita.

Precedía a esta un patio, semejante a un cementerio. Entraron en él los
ingleses; pero los imperiales, que se habían colado por el ábside,
dominaron pronto lo principal del edificio con los anexos posteriores;
así es que aún no habían forzado la puerta los nuestros cuando ya les
hacían fuego desde la espadaña de las campanas y desde la claraboya
abierta sobre el pórtico.

El brigadier Pack, uno de los hombres más valientes, más serenos y más
caballerosos que he conocido, arengó a los \emph{highlanders}. El
coronel que mandaba el 3.º de cazadores arengó a los suyos, y todos
arengaron, en suma, incluso yo, que les hablé en español el lenguaje más
apropiado a las circunstancias. Tengo la seguridad de que me
entendieron.

El 23 de línea no había entrado en el patio, sino que flanqueaba la
ermita por su izquierda, observando si venían más fuerzas francesas. En
caso contrario, la partida era nuestra, por la sencilla razón de que
éramos más hasta entonces. Pero no tardó en aparecer otra columna
enemiga. Esperarla, darle respiro, es decir, aparentar siquiera fuese
por un momento que se la temía, habría sido renunciar de antemano a toda
ventaja.

---¡A ellos!---grité a mi coronel.

---\emph{All right}!---exclamó este.

Y el 23 de línea cayó como una avalancha sobre la columna francesa.
Trabose un vivo combate cuerpo a cuerpo; vacilaron un poco nuestros
ingleses, porque el empuje de los enemigos era terrible en el primer
momento; pero tornando a cargar con aquella constancia imperturbable
que, si no es el heroísmo mismo, es lo que más se le parece, toda la
ventaja estuvo pronto de nuestra parte. Retiráronse en desorden los
imperiales, o mejor dicho, variaron de táctica, dispersándose en
pequeños grupos, mientras les venían refuerzos. Habíamos tenido pérdidas
casi iguales en uno y otro lado, y bastantes cuerpos yacían en el suelo;
pero aquello no era nada todavía, un juego de chicos, un prefacio
inocente que casi hacía reír.

Nuestra desventaja real consistía en que ignorábamos la fuerza que
podían enviar los franceses contra nosotros. Veíamos enfrente el espeso
bosque de Cavarrasa, y nadie sabía lo que se ocultaba bajo aquel manto
de verdura. ¿Serán muchos, serán pocos? Cuando la intuición, la
inspiración o el genio zahorí de los grandes capitanes no sabe contestar
a estas preguntas, la ciencia militar está muy expuesta a resultar vana
y estéril como jerga de pedantes. Mirábamos al bosque, y el oscuro
ramaje de las encinas no nos decía nada. No sabíamos leer en aquella
verdinegra superficie que ofrecía misteriosos cambiantes de color y de
luz, fajas movibles y oscilantes signos en su vasta extensión. Era una
masa enorme de verdura, un monstruo chato y horrible que se aplanaba en
la tierra con la cabeza gacha y las alas extendidas, empollando quizás
bajo ellas innumerables guerreros.

Al ver en retirada la segunda columna francesa, mandó Pack redoblar la
tentativa contra la ermita, y los \emph{highlanders} intentaron
asaltarla por distintos puntos, lo cual hubiera sido fácil si al sonar
los primeros tiros no ocurriese del lado del bosque algo de particular.
Creeríase que el monstruo se movía; que alzaba una de las alas; que
echaba de sí un enjambre de homúnculos, los cuales distinguíanse allá
lejos al costado de la madre, pequeños como hormigas. Luego iban
creciendo, íbanse acercando\ldots{} de pigmeos tornábanse en gigantes;
lucían sus cascos: sus espadas semejaban rayos flamígeros; subían en
ademán amenazador columna tras columna, hombre tras hombre.

El coronel me miró y nos miramos los jefes todos sin decirnos nada. Con
la presteza del buen táctico, Pack, sin abandonar el asedio de la
ermita, nos mandó más gente y esperamos tranquilos. El bosque seguía
vomitando gente.

---Es preciso combatir a la defensiva---dijo el coronel.

---A la defensiva, sí. ¡Viva Inglaterra!

---¡Viva el emperador!---repitieron los ecos allá lejos.

---¡Ingleses, la Inglaterra os mira!

El clamor que antes nos contestara de lejos diciendo: ¡viva el
emperador! resonó con más fuerza. El animal se acercaba y su feroz
bramido infundía zozobra.

\hypertarget{xxxiv}{%
\chapter{XXXIV}\label{xxxiv}}

Ocupáronse al instante unas casas viejas y unos tejares que había como a
60 varas a un lado y otro de la ermita, estableciéndose imaginaria línea
defensiva, cuyo único apoyo material era una depresión del terreno, una
especie de zanja sin profundidad que parecía marcar el linde entre dos
heredades. Si yo hubiera mandado toda la fuerza del brigadier Pack,
habría intentado jugar el todo por el todo y desconcertar al enemigo
antes que embistiera; pero los ingleses no hacían nunca estas locuras
que salen bien una vez, y veinte se malogran. Por el contrario, Pack
dispuso sus fuerzas a la defensiva; con ojo admirable y rápido se hizo
cargo de todos los accidentes del terreno, de las suaves ondulaciones
del cerro por aquella parte, del peñón aislado, del árbol solitario, de
la tapia ruinosa, y todo lo aprovechó.

Llegaron los franceses. Nos miraban desde lejos con recelo, nos olían,
nos escuchaban.

¿Habéis visto a la cigüeña alargar el cuello a un lado y otro, de tal
modo que no se sabe si mira o si oye, sostenerse en un pie, alzando el
otro con intento de no fijarlo en tierra hasta no hallar suelo seguro?
Pues así se acercaban los franceses. Entre nosotros, algunos reían.

No puedo dar idea del silencio que reinaba en las filas en aquel
momento. ¿Eran soldados en acecho o monjes en oración?\ldots{} Pero
instantáneamente, la cigüeña puso los dos pies en tierra. Estaba en
terreno firme. Sonaron mil tiros a la vez y se nos vino encima una
oleada humana compuesta de bayonetas, de gritos, de patadas, de
ferocidades sin nombre.

---¡Fuego!, ¡muerte!, ¡sangre!, ¡canallas!---tales son las palabras con
que puedo indicar, por lo poco que entendía, aquella algazara de la
indignación inglesa, que mugía en torno mío, un concierto de
articulaciones guturales, un graznido al mismo tiempo discorde y sublime
como de mil celestiales loros y cotorras charlando a la vez.

Yo había visto cosas admirables en soldados españoles y franceses,
tratándose de atacar; pero no había visto nada comparable a los ingleses
tratando de resistir. Yo no había visto que las columnas se dejaran
acuchillar. El viejo tronco inerte no recibe con tanta paciencia el
golpe de la segur que lo corta, como aquellos hombres la bayoneta que
los destrozaba. Repetidas veces rechazaron a los franceses haciéndoles
correr mucho más allá de la ermita. Había gente para todo; para morir
resistiendo y para matar empujando. Por momentos parecía que les
rechazábamos definitivamente; pero el bosque, sacando de su plumaje
nuevas empolladuras de gente, nos ponía en desventaja numérica, pues si
bien del Arapil Chico venían a ayudarnos algunas compañías, no eran en
número suficiente.

La mortandad era grande por un lado y por otro, más por el nuestro, y a
tanto llegó que nos vimos en gran apuro para retirar los muchos muertos
y heridos que imposibilitaban los movimientos. El combate se suspendía y
se trababa en cortos intervalos. No retrocedíamos ni una línea; pero
tampoco avanzábamos, y habíamos abandonado el patio de la ermita por ser
imposible sostenerse allí. Las casas de labor y tejares sí eran nuestros
y no parecían los \emph{highlanders} dispuestos a dejárselos quitar,
pero esta serie de ventajas y desventajas que equilibraba las dos
potencias enemigas, este contrapeso sostenido a fuerza de arrojo no
podía durar mucho. Que los franceses enviasen gente, que, por el
contrario, las enviase lord Wellington, y la cuestión había de decidirse
pronto; que la enviasen los dos al mismo tiempo y entonces\ldots{} sólo
Dios sabía el resultado.

El brigadier Pack me llamó, diciéndome:

---Corred al cuartel general y decid al lord lo que pasa.

Monté a caballo y a todo escape me dirigí al cuartel general. Cuando
bajaba la pendiente en dirección a las líneas del ejército aliado,
distinguí muy bien las masas del ejército francés moviéndose sin cesar;
pero entre el centro de uno y otro ejército no se disparaba aún ni un
solo tiro. Todo el interés estaba todavía en aquella apartada escena del
Arapil Grande, en aquello que parecía un detalle insignificante, un
capricho del genio militar que a la sazón meditaba la gran batalla.

Cuando pasé junto a los diversos cuerpos de la línea aliada, llamó mi
atención verles quietos y tranquilos, esperando órdenes mano sobre mano.
No había batalla: es más, no parecía que iba a haber batalla, sino
simulacro. Pero los jefes, todos en pie sobre las elevaciones del
terreno, sobre los carros de municiones y aun sobre las cureñas,
observaban, ayudados de sus anteojos, la peripecia del Arapil Grande,
junto a la ermita.

---¿Por qué toda esta gente no corre a ayudar al brigadier Pack?---me
preguntaba yo lleno de confusiones.

Era que ni Wellington ni Marmont querían aparentar gran deseo de ocupar
el Arapil Grande, por lo mismo que uno y otro consideraban aquella
posición como la clave de la batalla. Marmont fingía movimientos
diversos para desconcertar a Wellington: amenazaba correr hacia el
Tormes para que el ojo imperturbable del capitán inglés se apartase del
Arapil; luego afectaba retirarse como si no quisiera librar batalla, y
en tanto Wellington, quieto, inmutable, sereno, atento, vigilante,
permanecía en su puesto observando las evoluciones del francés, y
sostenía con poderosa mano las mil riendas de aquel ejército que quería
lanzarse antes de tiempo.

Marmont quería engañar a Wellington; pero Wellington no sólo quería
engañar sino que estaba engañando a Marmont. Este se movía para
desconcertar a su enemigo, y el inglés atento a las correrías del otro,
espiaba la más ligera falta del francés para caerle encima. Al mismo
tiempo afectaba no hacer caso del Arapil Grande y colocó bastantes
tropas en la derecha del Tormes para hacer creer que allí quería poner
todo el interés de la batalla. En tanto tenía dispuestas fuerzas enormes
para un caso de apuro en el gran cerro. Pero ese caso de apuro, según
él, no había llegado todavía, ni llegaría, mientras hubiera carne viva
en Santa María de la Peña. Eran las diez de la mañana y fuera de la
breve acción que he descrito, los dos ejércitos no habían disparado un
tiro.

Cuando atravesé las filas, muchos jefes apostados en distintos puntos me
dirigían preguntas a que era imposible contestar, y cuando llegué al
cuartel general, vi a Wellington a caballo, rodeado de multitud de
generales.

Antes de acercarme a él, ya había dicho yo expresivamente con el gesto,
con la mirada:

---No se puede.

---¿Qué no se puede?---exclamó con calma imperturbable, después que
verbalmente le manifesté lo que pasaba allá.

---Dominar el Arapil Grande.

---Yo no he mandado a Pack que dominara el Arapil Grande, porque es
imposible ---repuso.---Los franceses están muy cerca y desde ayer tienen
hechos mil preparativos para disputarnos esa posición, aunque lo
disimulan.

---Entonces\ldots{}

---Yo no he mandado a Pack que dominase por completo el cerro, sino que
impidiese a los franceses que se establecieran allí definitivamente. ¿Se
establecerán? ¿No existen ya el 23 de línea, ni el 3.º de cazadores, ni
el 7.º de \emph{highlanders?}

---Existen\ldots{} un poco todavía, mi general.

---Con las fuerzas que han ido después basta para el objeto, que es
resistir, nada más que resistir. Basta con que ni un francés pise la
vertiente que cae hacia acá. Si no se puede dominar la ermita, no creo
que falte gente para entretener al enemigo unas cuantas horas.

---En efecto, mi general---dije.---Por muy aprisa que se muera,
ochocientos cuerpos dan mucho de sí. Se puede conservar hasta el medio
día lo que poseemos.

Cuando esto decía, atendiendo más a las lejanas líneas enemigas que a
mí, observé en él un movimiento súbito; volviose al general Álava, que
estaba a su lado y dijo:

---Esto cambia de repente. Los franceses extienden demasiado su línea.
Su derecha quiere envolverme\ldots{}

Una formidable masa de franceses se extendía hacia el Tormes, dejando un
claro bastante notable entre ella y Cavarrasa. Era necesario ser ciego
para no comprender que por aquel claro, por aquella juntura iba a
introducir su terrible espada hasta la empuñadura el genio del ejército
aliado.

\hypertarget{xxxv}{%
\chapter{XXXV}\label{xxxv}}

El cuartel general retrocedió, diéronse órdenes, corrieron los oficiales
de un lado para otro, resonó un murmullo elocuente en todo el ejército,
avanzaron los cañones, piafaron los caballos. Sin esperar más, corrí al
Arapil para anunciar que todo cambiaba. Veíanse oscilar las líneas de
los regimientos, y los reflejos de las bayonetas figuraban movibles
ondas luminosas; los cuerpos de ejército se estremecían conmovidos por
las palpitaciones íntimas de ese miedo singular que precede siempre al
heroísmo. La respiración y la emoción de tantos hombres daba a la
atmósfera no sé qué extraño calor. El aire ardiente y pesado no bastaba
para todos.

Las órdenes trasmitidas con rapidez inmensa llevaban en sí el
pensamiento del general en jefe. Todos lo adivinamos en virtud de la
extraña solidaridad que en momentos dados se establece entre la voluntad
y los miembros, entre el cerebro que piensa y las manos que ejecutan. El
plan era precipitar el centro contra el claro de la línea enemiga y al
mismo tiempo arrojar sobre el Arapil Grande toda la fuerza de la
derecha, que hasta entonces había permanecido en el llano en actitud
expectativa.

Hallábame cerca del lugar de partida, cuando un estrépito horrible hirió
mis oídos. Era la artillería de la izquierda enemiga, que tronaba contra
el gran cerro. Le atacaba con empuje colosal. Nuestra derecha, compuesta
de valientes cuerpos de ejército, subía en el mismo instante a sacar de
su aprieto a los incomparables \emph{highlanders}, 23 de línea y 3.º de
ligeros, cuyas proezas he descrito.

Pasé por entre la quinta división al mando del general Leith, que desde
el pueblo de los Arapiles marchaba al cerro; pasé por entre la tercera
división, mandada por el mayor general Packenham, la caballería del
general d'Urban y los dragones del decimocuarto regimiento, que iban en
cuatro columnas a envolver la izquierda del enemigo en la famosa altura;
y vi desde lejos la brigada del general Bradford, la de Cole y la
caballería de Stapleton Cotton, que marchaban en otra dirección contra
el centro enemigo; distinguí asimismo a lo lejos a mis compañeros de la
división española formando parte de la reserva mandada por Hope.

La ermita antes nombrada no coronaba el Arapil Grande, pues había
alturas mucho mayores. Era en realidad aquella eminencia regular y
escalonada, y si desde lejos no lo parecía, al aventurarse en ella
hallábanse grandes depresiones del terreno, ondulaciones, pendientes,
ora suaves ora ásperas, y suelo de tierra ligeramente pedregoso.

Los franceses, desde el momento en que creyeron oportuno no disimular su
pensamiento, aparecieron por distintos puntos y ocuparon la parte más
alta y sitios eminentes, amenazando de todos ellos las escasas fuerzas
que operaban allí desde por la mañana. La primera división que rompió el
fuego contra el enemigo fue la de Packenham, que intentó subir y subió
por la vertiente que cae al pueblo. Sostúvole la caballería portuguesa
de d'Urban; pero sus progresos no fueron grandes, porque los franceses,
que acababan de salir del bosque, habían tomado posiciones en lo más
alto, y aunque la pendiente era suave, dábales bastante ventaja.

Cuando llegué a las inmediaciones de la ermita, el brigadier Pack no
había perdido una línea de sus anteriores posiciones; pero sus bravos
regimientos estaban reducidos a menos de la mitad. El general Leith
acababa de llegar con la quinta división, y el aspecto de las cosas
había cambiado completamente porque si el enemigo enviaba numerosas
fuerzas a la cumbre del cerro, nosotros no le íbamos en zaga en número
ni en bravura.

Pero no había tiempo que perder. Era preciso arrojar hombres y más
hombres sobre aquel montón de tierra, despreciando los fuegos de la
artillería francesa, que nos cañoneaba desde el bosque, aunque sin
hacernos gran daño. Era preciso echar a los franceses de Santa María de
la Peña y después seguir subiendo, subiendo hasta plantar los pabellones
ingleses en lo más alto del Arapil Grande.

---El refuerzo ha venido casi antes que la contestación---dije al
brigadier Pack.---¿Qué debo hacer?

---Tomar el mando del 23 de línea, que ha quedado sin jefes. ¡Arriba,
siempre arriba! Ya veo lo que tenemos que hacer. Sostenernos aquí,
atraer el mayor número posible de tropas enemigas, para que Cole y
Bradford no hallen gran resistencia en el centro. Esta es la llave de la
batalla. ¡Arriba, siempre arriba!

Los franceses parecían no dar ya gran importancia a Santa María de la
Peña, y coronaron la altura. Las columnas escalonadas con gran arte, nos
esperaban a pie firme. Allí no había posibilidad de destrozarlas con la
caballería, ni de hacerles gran daño con los cañones situados a mucha
distancia. Era preciso subir a pecho descubierto y echarles de allí como
Dios nos diera a entender. El problema era difícil, la tarea inmensa, el
peligro horrible.

Tocó al 23 de línea la gloria de avanzar el primero contra las inmóviles
columnas francesas que ocupaban la altura. ¡Espantoso momento! La
escalera, señores, era terrible, y en cada uno de sus fúnebres peldaños,
el soldado se admiraba de encontrarse con vida. Si en vez de subir
bajase, aquélla sería la escalera del infierno. Y sin embargo, las
tropas de Pack y de Leith subían. ¿Cómo? No lo sé. En virtud de un
prodigio inexplicable. Aquellos ingleses no se parecían a los hombres
que yo había visto. Se les mandaba una cosa, un absurdo, un imposible, y
lo hacían, o al menos lo intentaban.

Al referir lo que allí pasó, no me es posible precisar los movimientos
de cada batallón, ni las órdenes de cada jefe, ni lo que cada cual hacía
dentro de su esfera. La imaginación conserva con caracteres indelebles y
pavorosos lo principal; pero lo accesorio no, y lo principal era
entonces que subíamos empujados por una fuerza irresistible, por no sé
qué manos poderosas que se agarraban a nuestra espalda. Veíamos la
muerte delante, arriba; pero la propia muerte nos atraía. ¡Oh! Quien no
ha subido nunca más que las escaleras de su casa, no comprenderá esto.

Como el terreno era desigual, había sitios en que la pendiente
desaparecía. En aquellos escalones se trababan combates parciales de un
encarnizamiento y ferocidad inauditos. Los valientes del Mediodía, que
conocen rara vez el heroísmo pasivo de dejarse matar antes que
descomponer las filas separándose de ellas, no comprenderán aquella
locura imperturbable a que nos conducía la separación convertida en
virtud. Fácil es a la alta cumbre desprenderse y precipitarse,
aumentando su velocidad con el movimiento, y caer sobre el llano y
arrollarlo e invadirlo; pero nosotros éramos el llano, empeñado en subir
a la cumbre, y deseoso de aplastarla, y hundirla y abollarla. En la
guerra como en la naturaleza, la altura domina y triunfa, es la
superioridad material, y una forma simbólica de la victoria, porque la
victoria es realmente algo que con flamígera velocidad baja rodando y
atropellando, hendiendo y destruyendo. El que está arriba tiene la
fuerza material y moral, y por consiguiente el pensamiento de la lucha,
que puede dirigir a su antojo. Como la cabeza en el cuerpo humano,
dispone de los sentidos y de la idea\ldots{} nosotros éramos pobres
fuerzas rastreras que arañando el suelo, estábamos a merced de los de
arriba, y sin embargo queríamos destronarlos. Figuraos que los pies se
empeñaran en arrojar la cabeza de los hombros para ponerse encima ellos,
¡estúpidos que no saben más que andar!

Los primeros escalones no ofrecieron gran dificultad. Moría mucha gente;
pero se subía. Después ya fue distinto. Creeríase que los franceses nos
permitían el ascenso a fin de cogernos luego más a mano. Las
disposiciones de Pack para que sufriésemos lo menos posible eran
admirables. Inútil es decir que todos los jefes habían dejado sus
caballos, y unos detrás, otros a la cabeza de las líneas, llevaban, por
decirlo así, de la mano a los obedientes soldados. Un orden preciso en
medio de las muertes, un paso seguro, un aplomo sin igual regimentando
la maniobra, impedían que los estragos fuesen excesivos. Con las armas
modernas, aquel hecho hubiera sido imposible.

Era indispensable aprovechar los intervalos en que el enemigo cargaba
los fusiles, para correr nosotros a la bayoneta. Teníamos en contra
nuestra el cansancio, pues si en algunos sitios la inclinación era poco
más que rampa, en otros era regular cuesta. Los franceses reposados,
satisfechos y seguros de su posición, nos abrasaban a fuego certero y
nos recibían a bayoneta limpia. A veces una columna nuestra lograba, con
su constancia abrumadora, abrirse paso por encima de los cadáveres de
los enemigos; mas para esto se necesitaba duplicar y triplicar los
empujes, duplicar y triplicar los muertos, y el resultado no
correspondía a la inmensidad del esfuerzo.

¡Qué espantosa ascensión! Cuando se empeñaban en algún descanso combates
parciales, las voces, el tumulto, el hervidero de aquellos cráteres no
son comparables a nada de cuanto la cólera de los hombres ha inventado
para remedar la ferocidad de las bestias. Entre mil muertes se
conquistaba el terreno palmo a palmo, y una vez que se le dominaba, se
sostenía con encarnizamiento el pedazo de tierra necesario para poner
los pies. Inglaterra no cedía el espacio en que fijaba las suelas de sus
zapatos, y para quitárselo y vencer aquel prodigio de constancia, era
preciso a los franceses desplegar todo su arrojo favorecido por la
altura. Aun así no lograban echar a los británicos por la pendiente
abajo. ¡Ay del que rodase primero! Conociendo el peligro inmenso de un
pasajero desmayo, de un retroceso, de una mirada atrás, los pies de
aquellos hombres echaban raíces. Aun después de muertas, parecía que sus
largas piernas se enclavaban en el suelo hasta las rodillas, como
jalones que debían marcar eternamente la conquista del poderoso genio de
Inglaterra.

Mas al fin llegó un momento terrible; un momento en que las columnas
subían y morían, en que la mucha gente que se lanzaba por aquel talud,
destrozada, abrasada, diezmada, sintiéndose mermar a cada paso, entendió
que sus fuerzas no traían gran ventaja. Tras las columnas francesas
arrolladas, aparecían otras. Como en el espantoso bosque de Macbeth, en
la cresta del Grande Arapil cada rama era un hombre. Nos acercábamos
arriba, y aquel cráter superior vomitaba soldados. Se ignoraba de dónde
podía salir tanta gente, y era que la meseta del cerro tenía cabida para
un ejército. Llegó, pues, un momento, en que los ingleses vieron venir
sobre ellos la cima del cerro mismo, una monstruosidad horrenda que
esgrimía mil bayonetas y apuntaba con miles de cañones de fusil. El
pánico se apoderó de todos, no aquel pánico nervioso que obliga a
correr, sino una angustia soberana y grave que quita toda esperanza,
dando resignación. Era imposible, de todo punto imposible, seguir
subiendo.

Pero bajar era el punto más difícil. Nada más fácil si se dejaban
acuchillar por los franceses, resignándose a rodar sobre la tierra vivos
o muertos. Una retirada en declive paso a paso y dando al enemigo cada
palmo de terreno con tanta parsimonia como se le quitó, es el colmo de
la dificultad. Pack bramaba de ira, y la sangre agolpada en la carnaza
encendida de su rostro parecía querer brotar por cada poro. Era hombre
que tenía alma para plantarse solo en la cumbre del cerro. Daba órdenes
con ronca voz; pero sus órdenes no se oían ya: esgrimía la espada
acuchillando al cielo, porque el cielo tenía sin duda la culpa de que
los ingleses no pudiesen continuar adelante.

Había llegado la ocasión de que muriese estoicamente uno para resguardar
con su cuerpo al que daba un paso atrás. De este modo se salvaba la
mitad de la carne. Una mala retirada arroja en las brasas todo cuanto
hay en el asador. Las columnas se escalonaban con arte admirable; el
fuego era más vivo, y cada vez que descendía de lo alto desgajándose uno
de aquellos pesados aludes, creeríase que todo había concluido; pero la
confusión momentánea desaparecía al instante, las masas inglesas
aparecían de nuevo compactas y formidables, y la muerte tenía que
contentarse con la mitad. Así se fue cediendo lentamente parte del
terreno, hasta que los imperiales dejaron de atacarnos. Habían llegado a
un punto en que el cañón inglés les molestaba mucho, y además los
progresos de Packenham por el flanco del Grande Arapil les inquietaban
bastante. Reconcentráronse y aguardaron.

En tanto, por otro lado ocurrían sucesos admirables y gloriosos. Todo
iba bien en todas partes menos en nuestro malhadado cerro. El general
Cole destrozaba el centro francés. La caballería de Stapleton Cotton,
penetrando por entre las descompuestas filas, daba una de las cargas más
brillantes, más sublimes y al mismo tiempo más horrorosas que pueden
verse. Desde la posición a que nos retiramos, no avergonzados pero sí
humillados, distinguíamos a lo lejos aquella admirable función que nos
causaba envidia. Las columnas de dragones, las falanges de caballos, los
más ligeros, los más vivos, los más guerreros que pueden verse,
penetraban como inmensas culebras por entre la infantería francesa. Los
golpes de los sables ofrecían a la vista un salpicar perenne de pequeños
rayos, menuda lluvia de acero que destrozaba pechos, aniquilaba gente,
atropellaba y deshacía como el huracán. Los gritos de los jinetes, el
brillo de sus cascos, el relinchar de los corceles que regocijaban en
aquella fiesta sangrienta sus brutales e imperfectas almas, ofrecían
espectáculo aterrador. Indiferentes como es natural, a las desdichas del
enemigo, los corazones guerreros se endiosaban con aquel espectáculo. La
confianza huye de los combates, deidad asustada y llorosa, conducida por
el miedo; no queda más que la ira guerrera que nada perdona, y el
bárbaro instinto de la fuerza, que por misterioso enigma del espíritu se
convierte en virtud admirable.

Los escuadrones de Stapleton Cotton, como he dicho, estaban realizando
el gran prodigio de aquella batalla. En vano los franceses alcanzaban
algunas ventajas por otro lado; en vano habían logrado apoderarse de
algunas casas del pueblo de Arapiles. Creyendo que poseer la aldea era
importante, tomaron briosamente los primeros edificios y los defendieron
con bravura. Se agarraban a las paredes de tierra y se pegaban a ella,
como los moluscos a la piedra; se dejaban espachurrar contra las tapias
antes que abandonarlas, barridos por la metralla inglesa. Precisamente
cuando los franceses creían obtener gran ventaja poseyendo el pueblo, y
cuando nosotros descendíamos del Arapil Grande, fue cuando la caballería
de Cotton penetró como un gran puñal en el corazón del ejército
imperial; viose el gran cuerpo partido en dos, crujiendo y estallando al
violento roce de la poderosa cuña. Todo cedía ante ella, fuerza,
previsión, pericia, valor, arrojo, porque era una potencia admirable,
una unidad abrumadora, compuesta de miles de piezas que obraban
armónicamente sin que una sola discrepara. Las miles de corazas daban
idea del \emph{testudo} romano, pero aquella inmensa tortuga con conchas
de acero tenía la ligereza del reptil y millares de patas y millares de
bocas para gritar y morder. Sus dentelladas ensanchaban el agujero en
que se había metido; todo caía ante ella. Gimieron con espanto los
batallones enemigos. Corrió Marmont a poner orden y una bala de cañón le
quitó el brazo derecho. Corrió luego Bonnet a sustituirle y cayó
también. Ferey, Thomieres y Desgraviers, generales ilustres, perecieron
con millares de soldados.

En la falda de nuestro cerro se había suspendido el fuego. Un oficial
que había caído junto a mí al verificar el descenso, era transportado
por dos soldados. Le vi al pasar y él casi moribundo, me llamó con una
seña. Era sir Thomas Parr. Puesto en el suelo, el cirujano, examinando
su pecho destrozado, dio a entender que aquello no tenía remedio. Otros
oficiales ingleses, la mayor parte heridos también, le rodeaban. El
pobre Parr volvió hacia mí los ojos en que se extinguían lentamente los
últimos resplandores de la vida, y con voz débil me habló así:

---Me han dicho antes de la batalla que tenéis resentimientos contra mí
y que os disponíais a pedirme satisfacción por no sé qué agravios.

---Amigo---exclamé conmovido,---en esta ocasión no puede quedar en mi
pecho ni rastro de cólera. Lo perdono y lo olvido todo. La calumnia de
que usted se ha hecho eco, seguramente sin malicia, no puede dañar a mi
honor; es una ligereza de esas que todos cometemos.

---¿Quién no comete alguna, caballero Araceli?---dijo con voz
grave.---Reconoced, sin embargo, que no he podido ofenderos. Muero sin
la zozobra de ser odiado\ldots{} ¿Decís que os calumnié? ¿Os referís al
caso de miss Fly? ¿Y a eso llamáis calumnia? Yo he repetido lo que he
oído.

---¿Miss Fly?

---Como se dice que forzosamente os casaréis con ella, nada tengo que
echaros en cara. ¿Reconocéis que no os he ofendido?

---Lo reconozco---respondí sin saber lo que respondía.

Parr, volviéndose a sus compatriotas, dijo:

---Parece que perdemos la batalla.

---La batalla se ganará---le respondieron.

Sacó su reloj y lo entregó a uno de los presentes.

---¡Que la Inglaterra sepa que muero por ella! ¡Que no se olvide mi
nombre!\ldots---murmuró con voz que se iba apagando por grados.

Nombró a su mujer, a sus hijos, pronunció algunas palabras cariñosas,
estrechando la mano de sus amigos.

---La batalla se ganará\ldots{} ¡Muero por Inglaterra!\ldots---dijo
cerrando los ojos.

Algunos leves movimientos y ligeras oscilaciones de sus labios fueron
las últimas señales de la vida en el cuerpo de aquel valiente y generoso
soldado. Un momento después se añadía un número a la cifra espantosa de
los muertos que se había tragado el Arapil Grande.

\hypertarget{xxxvi}{%
\chapter{XXXVI}\label{xxxvi}}

A tremenda carga de Stapleton Cotton había variado la situación de las
cosas. Leith se apareció de nuevo entre nosotros, acompañado del
brigadier Spry. En sus semblantes, en sus gestos lo mismo que en las
vociferaciones de Pack comprendí que se preparaba un nuevo ataque al
cerro. La situación del enemigo era ya mucho menos favorable que
anteriormente, porque las ventajas obtenidas en nuestro centro con el
avance de la caballería y los progresos del general Cole modificaban
completamente el aspecto de la batalla. Packenham, después de
rechazarles del pueblo, les apretaba bastante por la falda oriental del
cerro, de modo que estaban expuestos a sufrir las consecuencias de un
movimiento envolvente. Pero tenía poderosa fuerza en la vasta colina y
además retirada segura por los montes de Cavarrasa. La brigada de Spry
que antes maniobrara en las inmediaciones del pueblo, corriose a la
derecha para apoyar a Packenham. La división de Leith, la brigada de
Pack con el 23 de línea, el 3.º y 5.º de ligeros entraron de nuevo en
fuego.

Los franceses reconcentrándose en sus posiciones de la ermita para
arriba, esperaban con imponente actitud. Sonó el tiroteo por diversos
puntos; las columnas marcharon en silencio. Ya conocíamos el terreno, el
enemigo y los tropiezos de aquella ascensión. Como antes, los franceses
parecían dispuestos a dejarnos que avanzáramos, para recibirnos a lo
mejor con una lluvia de balas; pero no fue así, porque de súbito
desgajáronse con ímpetu amenazador sobre Packenham y sobre Leith
atacando con tanto coraje que era preciso ser inglés para resistirlo.
Las columnas de uno y otro lado habían perdido su alineación, y formadas
de irregulares y deformes grupos ofrecían frentes erizados de picos, si
se me permite expresarlo así, los cuales se engastaban unos en otros.
Los dos ejércitos se clavaban mutuamente las uñas desgarrándose. Arroyos
de sangre surcaban el suelo. Los cuerpos que caían eran a veces el
principal obstáculo para avanzar; a ratos se interrumpían aquellos al
modo de abrazos de muerte y cada cual se retiraba un poco hacia atrás a
fin de cobrar nueva fuerza para una nueva embestida. Observábamos los
claros del suelo ensangrentado y lleno de cadáveres, y lejos de desmayar
ante aquel espectáculo terrible, reproducíamos con doble furia los
mismos choques. Cubierto de sangre, que ignoraba si había salido de mis
propias venas o de las de otro, yo me lanzaba a los mismos delirios que
veía en los demás, olvidado de todo, sintiendo (y esto es evidente),
como una segunda, o mejor dicho, una nueva alma que no existía más que
para regocijarse en aquellas ferocidades sin nombre, una nueva alma, en
cuyas potencias irritadas se borraba toda memoria de lo pasado, toda
idea extraña al frenesí en que estaba metida. Bramaba como los
\emph{highlanders}, y ¡cosa extraordinaria! en aquella ocasión yo
hablaba inglés. Ni antes ni después supe una palabra de ese lenguaje;
pero es lo cierto que cuanto aullé en la batalla me lo entendían, y a mi
vez les entendía yo.

El poderoso esfuerzo de los escoceses desconcertó un poco las líneas
imperiales, precisamente en el instante en que llegó a nuestro campo la
división de Clinton, que hasta entonces había estado en la reserva.
Tropas frescas y sin cansancio entraron en acción, y desde aquel momento
vimos que las horribles filas de franceses se mantuvieron inactivas
aunque firmes. Poco después las vimos replegarse, sin dejar de hacer
fuego muy vivo. A pesar de esto, los ingleses no se lanzaban sobre
ellos. Corrió algún tiempo más, y entonces observamos que las tropas que
ocupaban lo alto del cerro lo abandonaban lentamente, resguardadas por
el frente que seguía haciendo fuego.

No sé si dieron órdenes para ello; lo que sé es que súbitamente los
regimientos ingleses, que en distintos puntos ocupaban la pendiente,
avanzaron hacia arriba con calma, sin precipitación. La cumbre del
Grande Arapil era una extensión irregular y vasta, compuesta de otros
pequeños cerros y vallecitos. Inmenso número de soldados cabían en ella,
pero venía la noche, el centro del ejército enemigo estaba derrotado, su
izquierda hacia el Tormes también, de modo que les era imposible
defender la disputada altura. Francia empezaba a retirarse, y la batalla
estaba ganada.

Sin embargo, no era fácil acuchillar, como algunos hubieran querido, a
los franceses que aún ocupaban varias alturas, porque se defendían con
aliento y sabían cubrir la retirada. Por nuestro lado fue donde más daño
se les hizo. Mucho se trabajó para romper sus filas, para quebrantar y
deshacer aquella muralla que protegía la huida de los demás hacia el
bosque; pero al principio no fue fácil. El espectáculo de las
considerables fuerzas que se retiraban casi ilesas y tranquilamente nos
impulsó a cargar con más brío sobre ellas, y al cabo, tanto se golpeó y
machacó en la infortunada línea francesa, que la vimos agrietarse,
romperse, desmenuzarse, y en sus innúmeros claros penetraron el puño y
la garra del vencedor para no dejar nada con vida. ¡Terrible hora
aquella en que un ejército vencido tiene que organizar su fuga ante la
amenazadora e implacable saña del vencedor, que si huye le destroza y si
se queda le destroza también!

Caía la tarde; iba oscureciéndose lentamente el paisaje. Los
desparramados grupos del ejército enemigo, rayas fugaces que
serpenteaban en el suelo a lo lejos, se desvanecían absorbidos por la
tierra y los bosques, entre la triste música de los roncos tambores.
Estos y la algazara cercana y el ruido del cañón, que aún cantaba las
últimas lúgubres estrofas del poema, producían un estrépito loco que
desvanecía el cerebro. No era posible escuchar ni la voz del amigo
gritando en nuestro oído. Había llegado el momento en que todo lo dicen
las facciones y los gestos, y era inútil dar órdenes, porque no se
entendían. El soldado veía llegada la ocasión de las proezas
individuales, para lo cual no necesitaba de los jefes, y todo estaba ya
reducido a ver quién mataba más enemigos en fuga, quién cogía más
prisioneros, quién podía echar la zarpa a un general, quién lograba
poner la mano en una de aquellas veneradas águilas que se habían
pavoneado orgullosas por toda Europa, desde Berlín hasta Lisboa.

El rugido que atronó los espacios cuando el vencedor, lleno de ira y
sediento de venganza se precipitó sobre el vencido para ahogarle, no es
susceptible de descripción. Quien no ha oído retumbar el rayo en el seno
de las tempestades de los hombres, ignorará siempre lo que son tales
escenas. Ciegos y locos, sin ver el peligro ni la muerte, sin oír más
que el zumbar del torbellino, nos arrojábamos dentro de aquel volcán de
rabia. Nos confundíamos con ellos: unos eran desarmados, otros tendían a
sus pies al atrevido que les quería coger prisioneros, cuál moría
matando, cuál se dejaba atrapar estoicamente. Muchos ingleses eran
sacrificados en el último pataleo de la bestia herida y desesperada: se
acuchillaban sin piedad: miles de manos repartían la muerte en todas
direcciones, y vencidos y vencedores caían juntos revueltos y enlazados,
confundiendo la abrasada sangre.

No hay en la historia odio comparable al de ingleses y franceses en
aquella época. Güelfos y gibelinos, cartagineses y romanos, árabes y
españoles se perdonaban alguna vez; pero Inglaterra y Francia en tiempo
del Imperio se aborrecían como Satanes. La envidia simultánea de estos
dos pueblos, de los cuales uno dominaba los mares del globo y otro las
tierras, estallaba en los campos de batalla de un modo horrible. Desde
Talavera hasta Waterloo, los duelos de estos dos rivales tendieron en
tierra un millón de cuerpos. En los Arapiles, una de sus más
encarnizadas reyertas, llegaron ambos al colmo de la ferocidad.

Para coger prisioneros, se destrozaba todo lo que se podía en la vida
del enemigo. Con unos cuantos portugueses e ingleses, me interné tal vez
más de lo conveniente en el seno de la desconcertada y fugitiva
infantería enemiga. Por todos lados presenciaba luchas insanas y oía los
vocablos más insultantes de aquellas dos lenguas que peleaban con sus
injurias como los hombres con las armas. El torbellino, la espiral me
llevaba consigo, ignorante yo de lo que hacía; el alma no conservaba más
conocimiento de sí misma que un anhelo vivísimo de matar algo. En
aquella confusión de gritos, de brazos alzados, de semblantes
infernales, de ojos desfigurados por la pasión, vi un águila dorada
puesta en la punta de un palo, donde se enrollaba inmundo trapo, una
arpillera sin color, cual si con ella se hubieran fregado todos los
platos de la mesa de todos los reyes europeos. Devoré con los ojos aquel
harapo, que en una de las oscilaciones de la turba fue desplegado por el
viento y mostró una N que había sido de oro y se dibujaba sobre tres
fajas cuyo matiz era un pastel de tierra, de sangre, de lodo y de polvo.
Todo el ejército de Bonaparte se había limpiado el sudor de mil combates
con aquel pañuelo agujereado que ya no tenía forma ni color.

Yo vi aquel glorioso signo de guerra a una distancia como de cinco
varas. Yo no sé lo que pasó: yo no sé si la bandera vino hasta mí, o si
yo corrí hacia la bandera. Si creyese en milagros, creería que mi brazo
derecho se alargó cinco varas, porque sin saber cómo, yo agarré el palo
de la bandera, y lo así tan fuertemente, que mi mano se pegó a él y lo
sacudió y quiso arrancarlo de donde estaba. Tales momentos no caben
dentro de la apreciación de los sentidos. Yo me vi rodeado de gente;
caían, rodaban, unos muriendo, otros defendiéndose. Hice esfuerzos para
arrancar el asta, y una voz gritó en francés:

---Tómala.

En el mismo segundo una pistola se disparó sobre mí. Una bayoneta
penetró en mi carne; no supe por dónde, pero sí que penetró. Ante mí
había una figura lívida, un rostro cubierto de sangre, unos ojos que
despedían fuego, unas garras que hacían presa en el asta de la bandera y
una boca contraída que parecía iba a comerse águila, trapo y asta, y a
comerme también a mí. Decir cuánto odié a aquel monstruo, me es
imposible; nos miramos un rato y luego forcejeamos. Él cayó de rodillas;
una de sus piernas, no era pierna, sino un pedazo de carne. Pugné por
arrancar de sus manos la insignia. Alguien vino en auxilio mío, y
alguien le ayudó a él. Me hirieron de nuevo, me encendí en ira más
salvaje aún, y estreché a la bestia apretándola contra el suelo con mis
rodillas. Con ambas manos agarraba ambas cosas, el palo de la bandera y
la espada. Pero esto no podía durar así, y mi mano derecha se quedó sólo
con la espada. Creí perder la bandera; pero el acero empujado por mí se
hundía más cada vez en una blandura inexplicable, y un hilo de sangre
vino derecho a mi rostro como una aguja. La bandera quedó en mi poder;
pero de aquel cuerpo que se revolvía bajo el mío surgieron al modo de
antenas, garras, o no sé qué tentáculo rabioso y pegajoso, y una boca se
precipitó sobre mí clavando sus agudos dientes en mi brazo con tanta
fuerza, que lancé un grito de dolor.

Caí, abrazado y constreñido por aquel dragón, pues dragón me parecía. Me
sentí apretado por él, y rodamos por no sé qué declives de tierra, entre
mil cuerpos, los unos muertos e inertes, los otros vivos y que corrían.
Yo no vi más; sólo sentí que en aquel rodar veloz, llevaba el águila
fuertemente cogida entre mis brazos. La boca terrible del monstruo
apretaba cada vez más mi brazo, y me llevaba consigo, los dos envueltos,
confundidos, el uno sobre el otro y contra el otro, bajo mil patas que
nos pisaban; entre la tierra que nos cegaba los ojos; entre una
oscuridad tenebrosa, entre un zumbido tan grande, como si todo el mundo
fuese un solo abejón; sin conciencia de lo que era arriba y abajo, con
todos los síntomas confusos y vagos de haberme convertido en
constelación, en una como criatura circunvoladora, en la cual todos los
miembros, todas las entrañas, toda la carne y sangre y nervios dieron
vueltas infinitas y vertiginosas alrededor del ardiente cerebro.

Yo no sé cuánto tiempo estuve rodando; debió de ser poco; pero a mí me
pareció algo al modo de siglos. Yo no sé cuándo paré; lo que sé es que
el monstruo no dejaba de formar conmigo una sola persona, ni su feroz
boca de morderme\ldots{} por último, no se contentaba con comerme el
brazo, sino que, al parecer, hundía su envenenado diente en mi corazón.
Lo que también sé es que el águila seguía sobre mi pecho, yo la sentía.
Sentía el asta cual si la tuviera clavada en mis entrañas. Mi
pensamiento se hacía cargo de todo con extravío y delirio, porque él
mismo era una luz ardiente que caía no sé de dónde, y en la inapreciable
velocidad de su carrera describía una raya de fuego, una línea sin fin,
que\ldots{} tampoco sé a dónde iba. ¡Tormento mayor no lo experimenté
jamás! Este se acabó cuando perdí toda noción de existencia. La batalla
de los Arapiles concluyó, al menos para mí.

\hypertarget{xxxvii}{%
\chapter{XXXVII}\label{xxxvii}}

Dejadme descansar un instante y luego contestaré a las preguntas que se
me dirigen. Yo no recobré el sentido en un momento, sino que fui
entrando poco a poco en la misteriosa claridad del conocer; fui
renaciendo poco a poco con percepciones vagas; fui recobrando el uso de
algunos sentidos y había dentro de mí una especie de aurora; pero muy
lenta, sumamente lenta y penosa. Me dolía la nueva vida, me mortificaba
como mortifica al ciego la luz que en mucho tiempo no ha visto. Pero
todo era turbación. Veía algunos objetos y no sabía lo que eran; oía
voces y tampoco sabía lo que eran. Parecía haber perdido completamente
la memoria.

Yo estaba en un sitio (porque indudablemente era un sitio del globo
terráqueo); yo veía en torno a mí formas; pero no sabía que las paredes
fueran paredes, ni que el techo fuese techo; oía los lamentos, pero
desconocía aquellas vibraciones quejumbrosas que lastimaban mi oído.
Delante, muy cerca, frente por frente a mí, vi una cara. Al verla, mi
espíritu hizo un esfuerzo para apreciar la forma visible; pero no pudo.
Yo no sabía qué cara era aquella; lo ignoraba como se ignora lo que
piensa otro. Pero la cara tenía dos ojos hermosísimos que me miraban
amorosamente. Todo esto se determinaba en mí por sentimiento, porque
¿entender?\ldots{} no entendía nada. Así es que por sentimiento adiviné
en la persona que tenía delante una como tendencia compasiva y tierna y
cariñosa hacia mí.

Pero lo más extraño es que aquel cariño que pendía sobre mí y me
protegía como un ángel de la Guarda, tenía también voz y la voz vibró en
los espacios, agitando todas las partículas del aire y con las
partículas del aire todos los átomos de mi ser desde el centro del
corazón hasta la punta del cabello. Oí la voz que decía:

---Estáis vivo, estáis vivo\ldots{} y estaréis también sano.

El hermoso semblante se puso tan alegre que yo también me alegré.

---¿Me conocéis?---dijo la voz.

No debí de contestar nada, porque la voz repitió la pregunta. Mi
sensibilidad era tan grande, que cada palabra cual hoja acerada me
atravesaba el pecho. El dolor, la debilidad me vencieron de nuevo, sin
duda porque había hecho esfuerzos de atención superiores a mi estado, y
recaí en el desvanecimiento. Cerrando los ojos, dejé de oír la voz.
Entonces experimenté una molestia material. Un objeto extraño rozaba mi
frente cayéndome sobre los ojos. Como si el ángel protector lo
adivinara, al punto noté que me quitaban aquel estorbo. Era el cabello
en desorden que me caía sobre la frente y las cejas. Sentí una tibia
suavidad cariñosa que debía de ser una mano, la cual desembarazó mi
frente del contacto enojoso.

Poco después (continuaba con los ojos cerrados) me pareció que por
encima de mi cabeza revoloteaba una mariposa, y que después de trazar
varias curvas y giros, en señal de indecisión, se posaba sobre mi
frente. Sentí sus dos alas abatidas sobre mi piel; pero las alas eran
calientes, pesadas y carnosas: estuvieron largo rato impresas en mí, y
luego se levantaron produciendo cierto rumor, un suave estallido que me
hizo abrir los ojos.

Si rápidamente los abrí, más rápidamente huyó el alado insecto. Pero la
misma cara de antes estaba tan cerca de la mía, tan cerca, que su calor
me molestaba un poco. Había en ella cierto rubor. Al verla, mi espíritu
hizo un esfuerzo, un gran esfuerzo, y se dijo:---¿Qué rostro es este?
Creo que conozco este rostro.

Pero no habiendo resuelto el problema, se resignó a la ignorancia. La
voz sonó entonces de nuevo, diciendo con acento patético:

---¡Vivid, vivid por Dios!\ldots{} ¿Me conocéis? ¿Qué tal os sentís? No
tenéis heridas graves\ldots{} habéis contraído un ataque cerebral, pero
la fiebre ha cedido\ldots{} Viviréis, viviréis sin remedio, porque yo lo
quiero\ldots{} Si la voluntad humana no resucitara a los muertos, ¿de
qué serviría?

En el fondo, allá en el fondo de mi ser, no sé qué facultad, saliendo
entumecida de profundo sopor, emitió misteriosas voces de asentimiento.

---¿No me veis?---continuó ella (repito que no sabía quién era).---¿Por
qué no me habláis? ¿Estáis enfadado conmigo? Imposible, porque no os he
ofendido\ldots{} Si no os vi, si no os hablé con más frecuencia en los
últimos días, fue porque no me lo permitían. Ha faltado poco para que me
enviasen a mi país dentro de una jaula\ldots{} Pero no me pueden impedir
que cuide a los heridos, y estoy aquí velando por vos\ldots{} ¡Cuánto he
penado esperando a que abrieseis los ojos!

Sentí mi mano estrechada con fuerza. El rostro se apartó de mí.

---¿Tenéis sed?---dijo la voz.

Quise contestar con la lengua; pero el don de la palabra me era negado
todavía. De algún modo, empero, me expliqué afirmativamente, porque el
ángel tutelar aplicó una taza a mis labios. Aquello me produjo un
bienestar inmenso. Cuando bebía apareció otra figura delante de mí.
Tampoco sabía precisamente quién era; pero dentro, muy dentro de mí
bullía inquieta una chispa de memoria, esforzándose en explicarme con su
indeciso resplandor el enigma de aquel otro ser flaco, escuálido,
huesoso, triste, de cuyo esqueleto pendía negro traje talar semejante a
una mortaja. Cruzando sus manos, me miró con lástima profunda. La mujer
dijo entonces:

---Hermano, podéis retiraros a cuidar de los otros heridos y enfermos.
Yo le velaré esta noche.

De dentro de aquella funda negra que envolvía los huesos vivos de un
hombre, salió otra voz que dijo:

---¡Pobre Sr.~D. Gabriel de Araceli! ¡En qué estado tan lastimoso se
halla!

Al oír esto, mi espíritu experimentó un gran alborozo. Se regocijó, se
conmovió todo, como debió de conmoverse el de Colón al descubrir el
Nuevo Mundo. Gozándose en su gran conquista, pensó mi espíritu así:

---¿Con que yo me llamo Gabriel Araceli?\ldots{} Luego yo soy uno que se
halló en la batalla de Trafalgar y en el 2 de Mayo\ldots{} Luego yo soy
aquel que\ldots{}

Este esfuerzo, el mayor de los que hasta entonces había hecho, me postró
de nuevo. Sentime aletargado. Se extinguía la claridad: venía la noche.
Luz rojiza, procedente de triste farol, iluminaba aquel hueco donde yo
estaba. El hombre había desaparecido, y sólo quedó la hermosa mujer. Por
largo rato me estuvo mirando sin decirme cosa alguna. Su imagen muda,
triste y fija delante de mí, cual si estuviese pintada en un lienzo, fue
borrándose y desvaneciéndose a medida que yo me sumergía de nuevo en
aquella noche oscura de mi alma, de cuyo seno sin fondo poco antes
saliera. Dormí no sé cuánto tiempo, y al volver en mi acuerdo, había
ganado poco en la claridad de mis facultades. El estupor seguía, aunque
no tan denso. El deshielo iba muy despacio.

Mi protectora angelical no se había apartado de mí, y después de darme
de beber una sustancia que me causara gran alivio y reanimación, acomodó
mi cabeza en la almohada, y me dijo:

---¿Os sentís mejor?

Un soplo corrió de mi cerebro a mis labios, que articularon:---Sí.

---Ya se conoce---añadió la voz.---Vuestra cara es otra. Creo que va
desapareciendo la fiebre.

Contesté segunda vez que sí. En la estupidez que me dominaba no sabía
decir otra cosa, y me deleitaba el usar constantemente el único tesoro
adquirido hasta entonces en los inmensos dominios de la palabra. El sí
es vocabulario completo de los idiotas. Para contestar a todo que sí,
para dar asentimiento a cuanto existe, no es necesario raciocinio ni
comparación, ni juicio siquiera. Otro ha hecho antes el trabajo. En
cambio para decir \emph{no} es preciso oponer un razonamiento nuevo al
de aquel que pregunta, y esto exige cierto grado de inteligencia. Como
yo me encontraba en los albores del raciocinio, contestar negativamente
habría sido un portento de genio, de precocidad, de inspiración.

---Esta noche habéis dormido muy tranquilo---dijo la voz de mi
enfermera.---Pronto estaréis bien. Dadme vuestras manos que están algo
frías: os las calentaré.

Cuando lo hacía, un rayo pasó por mi mente, pero tan débil, tan rápido,
que no era todavía certeza, sino un presentimiento, una esperanza de
conocer, un aviso precursor. En mi cerebro se desembrollaba la madeja;
pero tan despacio, tan despacio\ldots{}

---Me debéis la vida\ldots---continuó la voz perteneciente a la persona
cuyas manos apretaban y calentaban las mías,---me debéis la vida.

La madeja de mi cerebro agitó sus hilos; tal esfuerzo hacía por
desenredarlos que estuvo a punto de romperlos.

---En vuestro delirio---prosiguió---se os han escapado palabras muy
lisonjeras para mí. El alma cuando se ve libre del imperio de la razón
se presenta desnuda y sin mordaza; enseña todas sus bellezas y dice todo
lo que sabe. Así la vuestra no me ha ocultado nada\ldots{} ¿Por qué me
miráis con esos ojos fijos, negros y tristes como noches? Si con ellos
me suplicáis que lo diga, lo diré, aunque atropelle la ley de las
conveniencias. Sabed que os amo.

La madeja entonces tiró tan fuertemente de sus hilos, que se iba a
romper, se rompía sin remedio.

---No necesitaría decíroslo porque ya lo sabéis---continuó después de
larga pausa.---Lo que no sabéis es que os amaba antes de
conoceros\ldots{} Yo tenía una hermana gemela más hermosa y más pura que
los ángeles. Apuesto a que no sabéis nada de esto\ldots{} Pues bien, un
libertino la engañó, la sedujo, la robó a Dios y a su familia, y mi
pobrecita, mi adorada, mi idolatrada Lillian, tuvo un momento de
desesperación y se dio a sí propia la muerte. El mayor de mis hermanos
persiguió al malvado, autor de nuestra vergüenza: ambos fueron una noche
a orillas del mar, se batieron y mi pobre Carlos cayó para no levantarse
más. Poco después mi madre, trastornada por el dolor se fue
desprendiendo de la tierra y en una mañana del mes de Mayo nos dijo
adiós y huyó al cielo. Seguramente nada sabíais de esto.

Continuaba siendo idiota y contesté que sí.

---Después de estos acontecimientos, sobre la haz de la tierra existía
un hombre más aborrecido que Satanás. Para mí su sólo nombre era una
execración. Le odiaba de tal modo que si le viera arrepentido y
caminando al cielo, mis labios no hubieran pronunciado para él una
palabra de perdón. Figurándomelo cadáver, le pisoteaba\ldots{}

La madeja daba unas vueltas, unos giros, y hacía tales enredos y
embrollos, que me dolía el cerebro vivamente. Allí había un hilo tirante
y rígido, el cual, doliéndome más que los demás me hizo decir:

---Soy Araceli, el mismo que se halló en Trafalgar y naufragó en el
\emph{Rayo} y vivió en Cádiz\ldots{} En Cádiz hay una taberna, de que es
amo el Sr.~Poenco.

---Un día---prosiguió,---hallándome en España, a donde vine siguiendo a
mi segundo hermano, dijéronme que aquel hombre había sido muerto por
otro en duelo de honor. Pregunté con tanto anhelo, con tan profunda
curiosidad el nombre del vencedor, que casi lo supe antes que lo
revelaran. Me dijeron vuestro nombre; me refirieron algunos pormenores
del caso, y desde aquel momento ¿por qué ocultarlo? os adoré.

Mi espíritu hizo inexplicables equilibrios sobre dos imágenes grotescas,
y puestos en una balanza dos figurones llamados Poenco y D. Pedro del
Congosto, el uno subía mientras el otro bajaba. En aquel instante debí
de decir algo más sustancioso que los primitivos \emph{sís}, porque ella
(yo continuaba ignorando quién era) puso la mano sobre mi frente, y
habló así:

---Me adivinabais sin duda, me veíais desde lejos con los ojos del
corazón. Yo os busqué durante muchos meses. Tanto tardasteis en
aparecer, que llegué a creeros desprovisto de existencia real. Yo leía
romances y todos a vos los aplicaba. Erais el Cid, Bernardo del Carpio,
Zaide, Abenamar, Celindos, Lanzarote del Lago, Fernán González y Pedro
Ansúrez\ldots{} Tomabais cuerpo en mi fantasía y yo cuidaba de haceros
crecer en ella; pero mis ojos registraban la tierra y no podían
encontraros. Cuando os encontré, me pareció que ibais a achicaros; pero
os vi subir de pronto y tocar el altísimo punto de talla con que yo os
había medido. Hasta entonces cuantos hombres traté, o se burlaban de mí
o no me comprendían. Vos tan sólo me mirasteis cara a cara y
afrontasteis las excelsas temeridades de mi pensamiento sin asustaros.
Os vi espontáneamente inclinado a la realización de acciones no comunes.
Asocieme a ellas, quise llevaros más adelante todavía y me seguisteis
ciegamente. Vuestra alma y la mía se dieron la mano y tocaron su frente
la una con la otra, para convencerse de que eran las dos de un mismo
tamaño. La luz de entrambos se confundía en una sola.

Al oir esto, la madeja de mi conocimiento se revolvió de un modo
extraordinario. Los hilos entraban, salían los unos por entre los otros
y culebreaban para separarse y ponerse en orden. Ya aparecían en grupos
de distintos colores, y aunque harto enmarañados todavía, muchos de
ellos, si no todos, parecían haber encontrado su puesto.

---Vos amabais a otra---prosiguió aquélla que empezaba ya a no serme
desconocida.---La vi y la observé. Quise tratarla por algún tiempo y la
traté y la conocí; la hallé tan indigna de vos, que desde luego me
consideré vencedora. Es imposible que me equivoque.

Al oír esto, el corazón mío, que hasta entonces había permanecido quieto
y mudo, y dormido como un niño en su cuna, empezó a dar unos saltitos
tan vivarachos, y a llamarme con una vocecita tan dulce que realmente me
hacía daño. Dentro de mí se fue levantando no sé si diré un vapor, una
onda que fue primero tibia y después ardiente, y me subía desde el fondo
a la superficie del ser, despertando a su paso todo lo que dormía; una
oleada invasora, dominante, que poseía el don de la palabra, y al
ascender por mí iba diciendo: «Arriba, arriba todo.»

---¿Qué tenéis?---continuó aquella mujer.---Estáis agitado. Vuestro
rostro se enciende\ldots{} ahora palidece\ldots{} ¿Vais a llorar? Yo
también lloro. La salud vuelve a vuestro cuerpo, como la sensibilidad a
vuestra noble alma. ¿Será posible que os haya conmovido la revelación
que he hecho? No juzguéis mi atrevimiento con criterio vulgar, creyendo
que no falto al decoro, a las conveniencias y al pudor diciendo a un
hombre que le amo. Yo, al mismo tiempo soy pura como los ángeles y libre
como el aire. Los necios que me rodean podrán calumniarme y calumniaros;
pero no mancharán mi honra, como no la mancha un amor ideal y celeste al
pasar del pensamiento a la palabra\ldots{} Si durante mucho tiempo he
disimulado y aparentado huir de vos, no ha sido por temor a los tontos,
sino por provecho de entrambos. Cuando os he visto casi muerto, cuando
os he recogido en mis brazos del campo de batalla, cuando os traje aquí
y os atendí y os cuidé, tratando de devolveros la vida, tenía gran pena
de que murieseis ignorando mi secreto.

El estupor mío tocaba a su fin. Pensamiento y corazón recobraban su
prístino ser; pero la palabra tardaba; vaya si tardaba\ldots{}

---Dios me ha escuchado---añadió ella.---No sólo podéis oírme, sino que
vivís; y podréis hablarme y contestarme. Decidme que me amáis, y si
morís después, siempre me quedará algo vuestro.

Una figura celestial, tan celestial que no parecía de este mundo, se
entró dentro de mí, agasajándome y plegándose toda para que no hubiese
en mi interior un solo hueco que no estuviese lleno con ella.

---No me contestáis una sola palabra---dijo la voz de mi enfermera.---Ni
siquiera me miráis. ¿Por qué cerráis los ojos\ldots? ¿Así se contesta,
caballero\ldots? Sabed que no sólo tengo dudas, sino también celos. ¿Os
habré desagradado en lo que últimamente he hecho? No os lo ocultaré,
porque jamás he mentido. Mi lengua nació para la verdad\ldots{}
¿Ignoráis tal vez que vuestra princesa encantada y el bribón de su padre
estaban en Salamanca? Quien los trajo, es cosa que ignoro. El
desgraciado masón anhelaba la libertad y se la he dado con el mayor
gusto, consiguiendo del general un salvo conducto para que saliese de
aquí y pudiese atravesar toda España sin ser molestado.

Al oí r esto, razón, memoria, sentimientos, palabra, todo volvió súbito
a mí con violencia, con ímpetu, con estrépito, como una catarata
despeñándose de las alturas del cielo. Di un grito, me incorporé en el
lecho, agité los brazos, arrojé lejos de mí con instintiva brutalidad
aquella hermosa figura que tenía delante, y prorrumpí en exclamaciones
de ira. Miré a la dama y la nombré, porque ya la había conocido.

\hypertarget{xxxviii}{%
\chapter{XXXVIII}\label{xxxviii}}

El hospitalario que antes vi, entró al oír mis gritos, y ambos
procuraron calmarme.

---Otra vez le empieza el delirio---dijo Juan de Dios.

---Yo he sido la causa de esta alteración---dijo miss Fly muy afligida.

Mi propia debilidad me rindió, y caí en el lecho, sofocado por la
indignación que sordamente se reconcentraba en mí, no encontrando ni voz
suficiente ni fuerzas para expresarse fuera.

---El pobre Sr.~Araceli---dijo Juan de Dios con sentimiento piadoso---se
volverá loco como yo. El demonio ha puesto su mano en él.

---Callad, hermano, y no digáis tonterías---dijo miss Fly cubriendo mis
brazos con la manta y limpiando el sudor de mi frente.---¿Qué habláis
ahí de demonios?

---Sé lo que me digo---añadió el agustino, mirándome con profunda
lástima.---El pobre D. Gabriel está bajo una influencia maléfica\ldots{}
Lo he visto, lo he visto.

Diciendo esto, destacaba de su puño cerrado dos dedos flacos y
puntiagudos, y con ellos se señalaba los ojos.

---Marchad fuera a cuidar de los otros enfermos---dijo miss Fly
jovialmente---y no vengáis a fastidiarnos con vuestras necedades.

Fuese Juan de Dios y nos quedamos de nuevo solos Athenais y yo.
Hallándome ya en posesión completa de mi pensamiento, le hablé así:

---Señora, repítame usted lo que hace poco ha dicho. No entendí bien.
Creo que ni mis sentidos ni mi razón están serenos. Estoy delirando,
como ha dicho aquel buen hombre.

---Os he hablado largo rato---dijo miss Fly con cierta turbación.

---Señora, no puedo apreciar sino de un modo muy confuso lo que he visto
y oído esta noche\ldots{} Efectivamente, he visto delante de mí una
figura hermosa y consoladora; he oído palabras\ldots{} no sé qué
palabras. En mi cerebro se confunden el eco de voces lejanas y el son
misterioso de otras que yo mismo habré pronunciado\ldots{} No distingo
bien lo real de lo verdadero; durante algún tiempo he visto los objetos
y los semblantes sin conocerlos.

---¡Sin conocerlos!

---He oído palabras. Algunas las recuerdo, otras no.

---Tratad de repetir lo sustancial de lo mucho que os he dicho---murmuró
Athenais, pálida y grave.---Y si no habéis entendido bien, os lo
repetiré.

---En verdad no puedo repetir nada. Hay dentro de mí una confusión
espantosa\ldots{} He creído ver delante de mí a una persona, cuya
representación ideal no me abandona jamás en mis sueños, una figura que
quiero y respeto, porque la creo lo más perfecto que ha puesto Dios
sobre la tierra\ldots{} He creído oír no sé qué palabras dulces y
claras, mezcladas con otras que no comprendía\ldots{} He creído escuchar
tan pronto una música del cielo, tan pronto el fragor de cien
tempestades que bramaban dentro de un corazón\ldots{} Nada puedo
precisar\ldots{} al fin he visto claramente a usted, la he
conocido\ldots{}

---¿Y me habéis oído claramente también?---preguntó acercando su rostro
al mío .---Ya sé que no debe darse conversación a los enfermos. Os habré
molestado. Pero es lo cierto que yo esperaba con ansia que pudierais
oírme. Si por desgracia murierais\ldots{}

---De lo que he oído, señora, sólo recuerdo claramente que había usted
puesto en libertad a una persona a quien yo aprisioné.

---¿Y esto os disgusta?---preguntó la Mosquita con terror.

---No sólo me disgusta, sino que me contraría mucho, pero
mucho---exclamé con inquietud, sacudiendo las ropas del lecho para sacar
los brazos.

Athenais gimió. Después de breve pausa, mirome con fijeza y orgullo y
dijo:

---Caballero Araceli, ¿tanto coraje es porque se os ha escapado el ave
encantada de la calle del Cáliz?

---Por eso, por eso es---repetí.

---¿Y seguramente la amáis?\ldots{}

---La adoro, la he adorado toda mi vida. Ha tiempo que mi existencia y
la suya están tan enlazadas como si fueran una sola. Mis alegrías son
sus alegrías, y sus penas son mis penas. ¿En dónde está? Si ha
desaparecido otra vez, señora Athenais de mi alma, juro a usted que
todos los romances de Bernardo, del Cid, de Lanzarote y de Celindos, me
parecerían pocos para buscarla.

Athenais estaba lastimosamente desfigurada. Diríase que era ella el
enfermo y yo el enfermero. Largo rato la vi como sosteniendo no sé qué
horrible lucha consigo misma. Volvía el rostro para que no viese yo su
emoción: me miraba después con ira violentísima que se trocaba sin
quererlo ella misma en inexplicable dulzura, hasta que levantándose con
ademán de majestuosa soberbia, me dijo:

---Caballero Araceli, adiós.

---¿Se va usted?---dije con tristeza y tomando su mano que ella separó
vivamente de la mía.---Me quedaré solo\ldots{} Merezco que usted me
desprecie, porque he vuelto a la vida, y mi primera palabra no ha sido
para dar las gracias a esta amiga cariñosa, a esta alma caritativa que
me recogió sin duda del campo de batalla, que me ha curado y
asistido\ldots{} ¡Señora, señora mía! La vida que usted ha ganado a la
muerte vería con gusto el momento en que tuviera que volverse a perder
por usted.

---Palabras hermosas, caballero Araceli---me dijo con acento solemne,
sin acercarse a mí, mirándome pálida y triste y seria desde lejos, como
una sibila sentenciosa que pronunciase las revelaciones de mi
destino.---Palabras hermosas; pero no tanto que encubran la vulgaridad
de vuestra alma vacía. Yo aparto esa hojarasca y no encuentro nada.
Estáis compuesto de grandeza y pequeñez.

---Como todo, como todo lo creado, señora---interrumpí.

---No, no---dijo con viveza.---Yo conozco algo que no es así; yo conozco
algo donde todo es grande. Habéis hecho en vuestra vida y aun en estos
mismos días cosas admirables. Pero el mismo pensamiento que concibió la
muerte de lord Gray, lo entregáis a una vulgar y prosaica ama de casa
como un papel en blanco para que escriba las cuentas de la lavandera.
Vuestro corazón, que tan bien sabe sentir en algunos momentos, no os
sirve para nada y lo entregáis a las costureras para que hagan de él un
cojincillo en que clavar sus alfileres. Caballero Araceli, me fastidio
aquí.

---¡Señora, señora, por Dios, no me deje usted! Estoy muy enfermo
todavía.

---¿Acaso no tengo yo rango más alto que el de enfermera? Soy muy
orgullosa, caballero. El hermano hospitalario os cuidará.

---Usted bromea, apreciable amiga, encantadora Athenais, usted se burla
del verdadero afecto, de la admiración que me ha inspirado. Siéntese
usted a mi lado; hablaremos de cosas diversas, de la batalla, del pobre
sir Thomas Parr a quien vi morir\ldots{}

---Todavía creo que valgo para algo más que para dar conversación a los
ociosos y a los aburridos---me contestó con desdén.---Caballero, me
tratáis con una familiaridad que me causa sorpresa.

---¡Oh! Recordaremos las proezas inauditas que hemos realizado juntos.
¿Se acuerda usted de Jean-Jean?

---En verdad sois impertinente. Bastante os he asistido; bastantes horas
he pasado junto a vos. Mientras delirabais, me he reído, oyendo las
necedades y graciosos absurdos que continuamente decíais; pero ya estáis
en vuestro sano juicio y de nuevo sois tonto.

---Pues bien, señora, deliraré, deliraré y diré todas las majaderías que
usted quiera, con tal que me acompañe---exclamé jovialmente.---No quiero
que usted se marche enojada conmigo.

Miss Fly se apoyó en la pared para no caer. Advertí que la expresión de
su rostro pasaba de una furia insensata a una emoción profunda. Sus ojos
se inundaron de lágrimas, y como si no le pareciese que sus manos las
ocultaban bien, corrió rápidamente hacia afuera. Su intención primera
fue sin duda salir; mas se quedó junto a la puerta y en sitio donde
difícilmente la veía. Con todo, bastaron a revelarme su presencia,
ignoro si los suspiros que creí oír o la sombra que se proyectaba en la
pared y subía hasta el techo. Lo que sí no tiene duda alguna para mí, es
que después de estar largo tiempo sumergido en tristes cavilaciones, me
sentí con sueño, y lentamente caí en uno profundísimo que duró hasta por
la mañana. ¿Debo decir que cuando me hallaba próximo a perder
completamente el uso de los sentidos, se repitieron los fenómenos
extraños que habían acompañado mi penoso regreso a la vida? ¿Debo decir
que me pareció ver volar encima y alrededor de mi cabeza un insecto
alado, que después vino a posar sobre mi frente sus dos alas blandas,
pesadas y ardientes?

Eso no era más que repetición de lo que antes había soñado: el fenómeno
más raro entre todos los de aquella rarísima noche vino después,
poniendo digno remate a mis confusiones, y fue, señores míos, que no
desvanecida aún mi confusión por aquello de la Pajarita, advertí que se
cernía sobre mi frente una cosa negra, larga, no muy grande, aunque me
era muy difícil precisar su tamaño, el cual objeto o animalucho tenía
dos largas piernas y dos picudas alas, que abría y cerraba
alternativamente, todo negro, áspero, rígido y extremadamente feo. Aquel
horrible crustáceo se replegaba, y entonces parecía un puñal negro;
después abría sus patas y sus alas y parecía un escorpión. Lentamente
bajaba acercándose a mí, y cuando tocó mi frente sentí frío en todo mi
cuerpo. Agitose mucho, meneó las horribles extremidades repetidas veces,
emitiendo un chillido estridente, seco, áspero, que estremecía los
nervios, y después huyó.

\hypertarget{xxxix}{%
\chapter{XXXIX}\label{xxxix}}

Tras un sueño tan largo como profundo, desperté en pleno día
notablemente mejorado. La hermosa claridad del sol me produjo bienestar
inmenso, y además del alivio corporal experimentaba cierto apacible
reposo del alma. Me recreaba en mi salud como un fatuo en su hermosura.

A mi lado estaban dos hombres, el hospitalario y un médico militar, que
después de reconocerme, hizo alegres pronósticos acerca de mi enfermedad
y me mandó que comiese algo suculento si encontraba almas caritativas
que me lo proporcionasen. Marchose a cortar no sé cuántas piernas, y el
hermano, luego que nos quedamos solos, se sentó junto a mí, y
compungidamente me dijo:

---Siga usted los consejos de un pobre penitente, Sr.~D. Gabriel, y en
vez de cuidarse del alimento del cuerpo, atienda al del alma, que harto
lo ha menester.

---¿Pues qué, Sr.~Juan de Dios, acaso voy a morir?---le dije recelando
que quisiera ensayar en mí el sistema de las silvestres yerbecillas.

---Para vivir como usted vive---afirmó el fraile con acento
lúgubre,---vale más mil veces la muerte. Yo al menos la preferiría.

---No entiendo\ldots{}

---Sr.~Araceli, Sr.~Araceli---exclamó, no ya inquieto sino con verdadera
alarma ,---piense usted en Dios, llame usted a Dios en su ayuda, elimine
usted de su pensamiento toda idea mundana, abstráigase usted. Para
conseguirlo recemos, amigo mío, recemos fervorosamente por espacio de
cuatro, cinco o seis horas, sin distraernos un momento, y nos veremos
libres del inmenso, del horrible peligro que nos amenaza.

---Pero este hombre me va a matar---dije con miedo.---Me manda el médico
que coma, y ahora resulta que necesito una ración de seis horas de rezo.
Hermanuco, por amor de Dios, tráigame una gallina, un pavo, un carnero,
un buey.

---¡Perdido, irremisiblemente perdido!\ldots---exclamó con aflicción
suma, elevando los ojos al cielo y cruzando las manos.---¡Comer, comer!
Regalar el cuerpo con incitativos manjares cuando el alma está
amenazada; amenazada, Sr.~Araceli\ldots{} Vuelva usted en sí\ldots{}
recemos juntos, nada más que seis horas, sin un instante de
distracción\ldots{} con el pensamiento clavado en lo alto\ldots{} De
esta manera el pérfido se ahuyentará, vacilará al menos antes de poner
su infernal mano en un alma inocente, la encontrará atada al cielo con
la santas cadenas de la oración, y quizás renuncie a sus execrables
propósitos.

---Hermano Juan de Dios, quíteseme de delante o no sé lo que haré. Si
usted es loco de atar, yo por fortuna no lo soy, y quiero alimentarme.

---Por piedad, por todos los santos, por la salvación de su alma, amado
hermano mío, modérese usted, refrene esos livianos apetitos, ponga cien
cadenas a la concupiscencia del mascar, pues por la puerta de la
gastronomía entran todos los melindres pecaminosos.

Le miré entre colérico y risueño, porque su austeridad, que había
empezado a ser grotesca, me enfadaba, y al mismo tiempo me divertía. No,
no me es posible pintarle tal como era, tal como le vi en aquel momento.
Para reproducir en el lienzo la extraña figura de aquel hombre, a quien
los ayunos y la exaltación de la fantasía llevaran a estado tan
lastimoso, no bastaría el pincel de Zurbarán, no; sería preciso revolver
la paleta del gran Velázquez para buscar allí algo de lo que sirvió para
la hechura de sus inmortales bobos.

Me reí de él, diciéndole:

---Tráigame usted de comer y después rezaremos.

Por única contestación, el hospitalario se arrodilló, y sacando un libro
de rezos, me dijo:

---Repita usted lo que yo vaya leyendo.

---¡Que me mata este hombre, que me mata! ¡Favor!---grité encolerizado.

Juan de Dios se levantó, y poniendo su mano sobre mi pecho, espantado y
tembloroso, me habló así:

---¡Que viene!, ¡que va a venir!

---¿Quién?---pregunté cansado de aquella farsa.

---¿Quién ha de ser, desgraciado, quién ha de ser?---dijo en voz baja y
con abatimiento.---¿Quién ha de ser sino el torpe enemigo del linaje
humano, el negro rey que gobierna el imperio de las tinieblas como Dios
el de la luz; aquel que odia la santidad y tiende mil lazos a la virtud
para que se enrede? ¿Quién ha de ser sino la inmunda bestia que posee el
arte de mudarse y embellecerse, tomando la figura y traje que más
fácilmente seducen al descuidado pecador? ¿Quién ha de ser? ¡Extraña
pregunta por cierto! ¡Me asombro de la inocente calma con que usted me
habla, hallándose, como se halla, en el mismo estado que yo!

Mis carcajadas atronaban la estancia.

---Me alegraré en extremo de que venga---le dije.---¿Cómo sabe usted que
va a venir?

---Porque ya ha estado, pobrecito; porque ya ha puesto sus aleves manos
sobre usted en señal de posesión y dominio, porque dijo que iba a
volver.

---Eso me alegra sobremanera. ¿Y cuándo he tenido el honor de tal
visita? No he visto nada.

---¡Cómo había usted de verlo si dormía, desgraciado!---exclamó con
lástima.---¡Dormir, dormir! he aquí el gran peligro. Él aprovecha las
ocasiones en que el alma está suelta y haciendo travesuras, libre de la
vigilancia de la oración. Por eso yo no duermo nunca, por eso velo
constantemente.

---¿Vino mientras yo dormía?

---Sí: anoche\ldots{} ¡horrible momento! La señora inglesa que tan bien
ha cuidado a usted había salido. Yo estaba solo y me distraje un poco en
mis rezos. Sin saber cómo, había dejado volar el pensamiento por
espacios voluptuosos y sonrosados\ldots{} ¡pecador indigno, mil veces
indigno!\ldots{} Yo había puesto el libro sobre mis rodillas, y cerrado
los ojos, y dejádome aletargar en sabroso desvanecimiento, cuya vaporosa
niebla y blando calor recreaban mi cuerpo y mi espíritu\ldots{}

---Y entonces, cuando mi bendito hermanuco se regocijaba con tales
liviandades; abriose la tierra, salió una llama de azufre\ldots{}

---No se abrió la tierra, sino la puerta, y apareció\ldots{} ¡Ay!
apareció en aquella forma celestial, robada a las criaturas de la más
alta esfera angélica; apareció cual siempre le ven mis pecadores ojos.

---Hermano, hermano, soy feliz y sentiría que estuviera usted cuerdo.

---Apareció, como he dicho, y su vista me convirtió en estatua. Otra de
igual catadura le acompañaba, también en forma mujeril, representando
más edad que la primera, la tan aborrecida como adorada, que es el
terror de mis noches y el espanto de mis días, y el abismo que se traga
mi alma.

---¿Y en cuanto me vieron\ldots? Adoro a esos demonios, Sr.~Juan de
Dios, y ahora mismo voy a mandarles un recadito con usted.

---¿Conmigo? ¡Infeliz precito! Ya vendrán por usted y se lo llevarán con
sus satánicas artes.

---Quiero saber qué hicieron, qué dijeron.

---Dijeron: «aquí nos han asegurado que está,» y luego sus ojos, que
todo lo ven en la lobreguez de la horrenda noche, vieron el miserable
cuerpo, y se abalanzaron hacia él con aullidos que parecían sollozos
tiernísimos, con lamentos que parecían la dulce armonía del amor
materno, llorando junto a la cuna del niño moribundo.

---¡Y yo dormido como un poste! ¡Padre Juan, es usted un imbécil, un
majadero! ¿Por qué no me despertó?

---Usted deliraba aún; las dos ¡ay! aquellas dos apariencias
hermosísimas, y tan acabadas y perfectas que sólo yo con los perspicuos
ojos del alma podía adivinar bajo su deslumbradora estructura la mano
del infernal artífice; las dos mujeres, digo, derramaron sobre el pecho
y la frente de usted demoníacas chispas, con tan ingeniosa alquimia
desfiguradas, que parecían lágrimas de ternura. Pusieron sus labios de
fuego en las manos de usted como si las besaran, le arreglaron las ropas
del lecho, y después\ldots{}

---¿Y después?

---Y después, buscáronme con los ojos como para preguntarme algo; mas
yo, más muerto que vivo, habíame escondido bajo aquella mesa y temblaba
allí y me moría. Sr.~D. Gabriel, me moría queriendo rezar y sin poder
rezar, queriendo dejar de ver aquel espectáculo y viéndolo
siempre\ldots{} Por fin, resolvieron marcharse\ldots{} ya eran dueños
del alma de usted y no necesitaban más.

---Se fueron, pues.

---Se fueron diciendo que iban a pedir licencia a no sé quién para
trasladar a usted a otro punto mejor\ldots{} al infierno cuando menos.
De esta manera desapareció de entre los vivos un hermano hospitalario
que era gran pecador; se lo llevaron una mañana enterito y sin dejar una
sola pieza de su corporal estructura.

---¿Y después\ldots? Estoy muy alegre, hermano Juan.

---Después vino esa señora a quien llaman \emph{Doña Flay}, la cual es
una criatura angelical, que le quiere a usted mucho. Usted empezó a
salir de aquel marasmo o trastorno en que le dejaron las embajadoras del
negro averno: la señora inglesa habló largamente con usted y yo, que me
puse a escuchar tras la puerta, oí que le decía mil cositas tiernas,
melosas y hechiceras.

---¿Y después?

---Y después usted se puso furioso y entré yo, y la inglesa me mandó
salir, y a lo que entendí, mi don Gabriel se durmió. La inglesa entraba
y salía, sin cesar de llorar.

---¿Y nada más?

---Algo más hay, sí, sin duda lo más terrible y espantoso, porque el
atormentador del linaje humano, aquél que, según un santo Padre, tiene
por cómplice de su infame industria a la mujer, la cual es hornillo de
sus alquimias, y fundamento de sus feas hechuras; aquel que me atormenta
y quiere perderme, entró de nuevo en la misma duplicada forma de mujer
linda\ldots{}

---Y yo, ¿dormía también?

---Dormía usted con sueño tranquilo y reposado. La señora inglesa estaba
junto a aquella mesa envolviendo no sé qué cosa en un papel. Entraron
ellas\ldots{} no expiré en aquel momento por milagro de Dios\ldots{} se
acercaron a usted y vuelta a los aullidos que parecían llantos, y a los
signos quirománticos semejantes a blandas y amorosas caricias.

---¿Y no dijeron nada? ¿No dijeron nada a miss Fly ni a usted?

---Sí---continuó después de tomar aliento, porque la fatiga de su
oprimido pecho apenas le permitía hablar,---dijeron que ya tenían la
licencia y que iban a buscar una litera para trasladar a usted a un
sitio que no nombraron\ldots{} Pero lo más extraño es que al oír esto la
señora inglesa, que no estaba menos absorta, ni menos suspendida, ni
menos espantada que yo, debió de conocer que las tan aparatosas beldades
eran obra de aquel que llevó a Jesús a la cima de la montaña y a la
cúspide de la ciudad; y sobrecogida como yo, lanzó un grito agudísimo
precipitándose fuera de la habitación. Seguila y ambos corrimos largo
trecho, hasta que ella puso fin a su atropellada carrera, y apoyando la
cabeza contra una pared, allí fue el verter lágrimas, el exhalar hondos
suspiros y el proferir palabras vehementes, con las cuales pedía a Dios
misericordia. Una hora después volví, despertó usted, y nada más. Sólo
falta que recemos, como antes dije, porque sólo la oración y la
vigilancia del espíritu ahuyenta al Malo, así como el pérfido sueño, las
regaladas comidas y las conversaciones mundanas le llaman.

Juan de Dios no dijo más; atendía a extraños ruidos que sonaban fuera, y
estaba trémulo y lívido.

---¡Aquí, aquí estoy, Inesilla\ldots{} señora condesa!---exclamé
reconociendo las dulces voces que desde mi lecho oía.---Aquí estoy vivo
y sano y contento, y queriéndolas a las dos más que a mi vida.

¡Ay! Entraron ambas y desoladas corrieron hacia mí. Una me abrazó por un
costado y otra por otro. Casi me desvanecí de alegría cuando las dos
adoradas cabezas oprimían mi pecho.

Juan de Dios huyó de un salto, de un vuelo o no sé cómo.

Quise hablar y la emoción me lo impedía. Ellas lloraban y no decían nada
tampoco. Al fin, Inés levantó los ojos sobre mi frente y la observé con
curiosidad y atención.

---¿Qué miras?---le dije.---¿Estoy tan desfigurado que no me conoces?

---No es eso.

La condesa miró también.

---Es que noto que te falta algo---dijo Inés sonriendo.

Me llevé la mano a la frente, y en efecto, algo me faltaba.

---¿Dónde han ido a parar los dos largos mechones de pelo que tenías
aquí?

Al decir esto, con sus deditos tocaba mi cabeza.

---Pues no sé\ldots{} tal vez en la batalla\ldots{}

Las dos se rieron.

---Queridas mías, recuerdo haber visto en sueños encima de mi cabeza un
animalejo frío y negro, y ahora comprendo lo que era aquello: unas
tijeras. Tengo aquí sobre la sien una rozadura\ldots{} ¿la ven
ustedes?\ldots{} Esos pelos me molestaban, y aquí del cirujano. Es
hombre entendido que no olvida el más mínimo detalle.

Tantas preguntas tenía que hacer, que no sabía por cuál empezar.

---¿Y en qué paró esa batalla?---dije.---¿Dónde está lord Wellington?

---La batalla paró en lo que paran todas, en que se acabó cuando se
cansaron de matarse---me respondió una de ellas, no sé cuál.

---Pero los franceses se retiraban cuando yo caí.

---Tanto se retiraron---dijo la condesa,---que todavía están corriendo.
Wellington les va a los alcances. No tengas cuidado por eso, que ya lo
harán bien sin ti\ldots{} Veremos si te dan algún grado por haber cogido
el águila.

---Con que yo cogí un águila\ldots---Un águila toda dorada, con las alas
abiertas y el pico roto, puesta sobre un palo, y con rayos en las
garras: la he visto---dijo Inés con satisfacción, extendiéndose en
pomposas descripciones de la insignia imperial.

---Te encontraron---añadió la condesa,---entre muchos muertos y heridos,
abrazado con el cadáver de un abanderado francés, el cual te mordía el
brazo.

Era la parte de mi cuerpo que más me dolía.

---Te hemos buscado desde el 22---dijo Inés,---y hasta anoche todo ha
sido correr y más correr sin resultado alguno. Creímos que habías
muerto. Fui a la zanja grande donde están enterrando los pobres cuerpos.
Había tantos, tantos, que no los pude ver todos\ldots{} Aquello parecía
una maldición de Dios. Si cuando tal vi hubiera tenido en mi mano el
águila que cogiste, la habría echado también en la zanja, y luego
tierra, mucha tierra encima.

---Bien, Inesilla, nadie mejor que tú dice las mayores verdades de un
modo más sencillo. La gloria militar y los muertos de las batallas
debieran enterrarse en una misma fosa\ldots{} En fin, adoradas mías,
vivo estoy para quererlas muchísimo, y para casarme con la una, previo
el consentimiento de la otra.

La condesa frunció ligeramente el ceño e Inés me miró el cabello. La
felicidad que inundaba mi alma se desbordó en francas risas y
expresiones gozosas, a que Inés habría contestado de algún modo, si la
seriedad de su madre se lo hubiera permitido.

---Saquemos ahora de aquí a este bergante---dijo la condesa---y después
se verá. Debemos dar gracias a esa señora inglesa que te recogió en el
campo de batalla y que te ha cuidado tan bien, según nos han dicho. Sé
quien es y la hemos visto. La conocí en el Puerto\ldots{} Por cierto,
caballerito, que tenemos que hablar tú y yo.

---¿No está por aquí? ¡Athenais, Athenais!\ldots{} Se empeñará en no
venir cuando la necesitamos. Me alegro infinito de que se conozcan
ustedes, creo que este conocimiento me ahorra un disgusto. Miss Fly es
persona leal y generosa. ¡Sr. Juan de Dios!\ldots{} Ese no vendrá aunque
le ahorquen. Ha dado en decir que son ustedes el demonio.

---¿Ese bendito hospitalario?---indicó la condesa.---El médico nos dijo
que se había ya escapado dos veces de la casa de locos\ldots{} Vamos, a
ver cómo te arreglamos en la camilla. Llamaremos a otro enfermero.

Cuando salió la condesa, dije a Inés:

---No me has dicho nada de aquella persona\ldots{}

---Ya lo sabrás todo---me contestó, sin oponerse a que le comiese a
besos las manos.---Ven pronto a casa\ldots{} prueba a levantarte.

---No puedo, hijita, estoy muy débil. Ese hospitalario de mil demonios
se propuso hoy matarme de hambre. El agustino empeñado en que no había
de comer, y miss Fly volviéndome loco con sus habladurías\ldots{}

---¡Oh!---dijo Inés con encantadora expresión de amenaza.---¿Esa inglesa
ha de estar contigo en todas partes\ldots? Tengo una sospecha, una
sospecha terrible, y si fuera cierto\ldots{} ¿Seré yo demasiado buena,
demasiado confiada e inocente, y tú un grandísimo tunante?

Miró de nuevo mi frente, no ya con inquietud, sino con verdadera alarma.

---¡Inesilla de mi corazón!---exclamé.---¡Si tienes sospechas, yo las
disiparé! ¿Dudas de mí? Eso no puede ser. No ha sucedido nunca y no
sucederá ahora. ¿Puedo yo dudar de ti? ¿Puede quebrantarse la fe de esta
religión mutua en que ha mucho tiempo vivimos y entrañablemente nos
adoramos?

---Así ha sido hasta aquí; pero ahora\ldots{} tú me ocultas algo\ldots{}
mi madre ha pronunciado al descuido algunas palabras\ldots{} No,
Gabriel, no me engañes. Dímelo, dímelo pronto. Miss Fly te recogió del
campo de batalla. Ella lo ha negado; pero es verdad. Nos lo han dicho.

---¡Engañarte yo!\ldots{} Eso sí que es gracioso. Aunque fuese malo y
quisiera hacerlo no podría\ldots{} Pero te debo decir la verdad, toda la
verdad, mujer mía, y empiezo desde este momento\ldots{} ¿por qué me
miras la frente?

---Porque\ldots{} porque---dijo pálida, grave y amenazadora---porque ese
mechón de pelo te lo ha quitado miss Fly. Yo lo adivino.

---Pues sí, ella misma ha sido---contesté con serenidad imperturbable.

---¡Ella misma!\ldots{} ¡Y lo confiesa!---exclamó entre suspensa y
aterrada.

Sus ojos se llenaron de lágrimas. Yo no sabía qué decirle. Pero la
verdad salía en onda impetuosa de mi corazón a mis labios. Mentir,
fingir, tergiversar, disimular era indigno de mí y de ella.
Incorporándome con dificultad le dije:

---Yo te contaré muchas cosas que te sorprenderán, querida mía. Demos tú
y yo las gracias a esa generosa mujer que me recogió de entre los
muertos en el Arapil Grande, para que no te quedases viuda.

---En marcha, vamos---dijo la condesa entrando de súbito e
interrumpiéndome.---En esta litera irás bien.

\hypertarget{xl}{%
\chapter{XL}\label{xl}}

A casa de la calle del Cáliz, a donde por dos veces he transportado a
mis oyentes, y a cuyo recinto de nuevo me han de seguir, si quieren
saber el fin de esta puntual historia, era la habitación patrimonial de
Santorcaz, que la había heredado de su padre un año antes, con algunas
tierras productivas. Componíase el tal caserón de dos o tres edificios
diversos en tamaño y estructura, que compró, unió y comunicó entre sí el
Sr.~D. Juan de Santorcaz, aldeano enriquecido a principios del siglo
pasado. Faltaba a aquella vivienda elegancia y belleza; pero no solidez,
ni magnitud, ni comodidades, aunque algunas piezas se hallaban demasiado
distantes unas de otras y era excesiva la longitud de los corredores,
así como el número de escalones que al discurrir de una parte a otra se
encontraban.

En los aposentos donde anteriormente les vimos estaba Santorcaz con su
hija el 22 de Julio durante la batalla. Esta última circunstancia hará
comprender a mis oyentes que no presencié lo que voy a contar, mas si lo
cuento de referencia, si lo pongo en el lugar de los hechos presenciados
por mí es porque doy tanta fe a la palabra de quien me los contó, como a
mis propios ojos y oídos; y así téngase esto por verídico y real.

Estaban, pues, según he dicho, el infortunado D. Luis y su hija en la
sala; lamentábase ella de que existieran guerras y maldecía él su triste
estado de salud que no le permitía presenciar el espectáculo de aquel
día, cuando sonó con terrible estruendo la famosa aldaba del culebrón, y
al poco rato el único criado que les servía y el militar que les
guardaba anunciaron a los solitarios dueños que una señora quería
entrar. Como miss Fly había estado allí algunos días antes, ofreciendo
al masón un salvo-conducto para salir de Salamanca y de España,
alegrósele a aquel el alma y dio orden de que al punto dejasen pasar e
internasen hasta su presencia a la generosa visitante. Transcurridos
algunos minutos, entró en la sala la condesa.

Santorcaz rugió como la fiera herida cuando no puede defenderse. Largo
rato estuvieron abrazadas madre e hija, confundiendo sus lágrimas, y tan
olvidadas del resto de la creación, cual si ellas solas existieran en el
mundo. Vueltas al fin en su acuerdo, la madre, observando con terror a
aquel hombre rabioso y sombrío que clavaba los ojos en el suelo como si
quisiera con la sola fuerza de su mirada abrir un agujero en que
meterse, quiso llevar a su hija consigo, y dijo palabras muy parecidas a
las que yo pronuncié en circunstancias semejantes.

Los que vieron mi sorpresa, juzguen cuál sería la de Amaranta cuando
Inés se separó de ella, y hecha un mar de lágrimas corrió con los brazos
abiertos hacia el anciano, en ademán cariñoso. Absorta miró tan
increíble movimiento la condesa. Santorcaz, cuando su hija estuvo
próxima, volvió el rostro y alargó los brazos para rechazarla.

---Vete de aquí---dijo,---no quiero verte, no te conozco.

---¡Loco!---gritó la muchacha con dolor.---Si dices otra vez que me
marche, me marcharé.

Revolvió Santorcaz los fieros ojos de un lado a otro de la estancia,
miró con igual rencor a la condesa y a su hija, y temblando de cólera,
repitió:

---Vete, vete, te he dicho que te vayas. No quiero verte más. Sal de
esta casa con esa mujer, y no vuelvas.

---Padre---dijo Inés sin dar gran importancia al frenesí del
anciano.---¿No me has dicho que esta casa es mía? ¿No me has entregado
las llaves? Pues voy a acomodar a esta señora en una habitación de las
de la calle, porque hoy es imposible que encuentre posada, y mañana las
dos nos iremos, dejándote tranquilo.

Tomando un manojo de llaves y repiqueteando con él, no sin cierta
intención zumbona, Inés salió de la estancia seguida de Amaranta, que
nada comprendía de aquella tragicomedia.

Luego que se quedó solo, Santorcaz dio algunos paseos por la habitación,
recorriéndola en giros y vueltas sin fin, cual macho de noria. Su
fisonomía expresaba todo cuanto puede expresar la fisonomía humana,
desde la saña más terrible a la emoción más tierna. Tomó después un
libro, pero lo arrojó en el suelo a los pocos minutos. Cogió luego una
pluma, y después de rasguñar el papel breve rato, la destrozó y la
pisoteó. Levantose, y con pasos vacilantes e inseguro ademán dirigiose a
la puerta vidriera, penetró en la estancia próxima, donde había un
tocador de mujer y un lecho blanco. De rodillas en el suelo, hizo de la
cama reclinatorio, y apoyando el rostro sobre ella, estuvo llorando todo
el día.

Si Santorcaz hubiera tenido un oído agudo y finísimo, como el de algunas
especies ornitológicas, habría percibido el rumor de tenues pasos en el
corredor cercano; si Santorcaz hubiera poseído la doble vista, que es un
absurdo para la fisiología, pero que no lo parecería si se llegaran a
conocer los misteriosos órganos del espíritu, habría visto que no estaba
enteramente solo; que una figura celestial batía sus alas en las
inmediaciones de la triste alcoba; que sin tocar el suelo con su ligero
paso, venía y se acercaba, y aplicaba con gracioso gesto su linda cabeza
a la puerta para escuchar, y luego introducía un rayo de sus ojos por un
resquicio para observar lo que dentro pasaba; y como si lo que veía y
oía la contentase, iluminaba aquellos sombríos espacios con una sonrisa,
y se marchaba para volver al poco rato y atender lo mismo. Pero el pobre
masón no veía nada de esto. Aquella tarde un ordenanza inglés le trajo
un salvo-conducto para salir de Salamanca; pero el masón lo rompió. La
condesa e Inés, excepto en los intervalos que esta salía, hablaban por
los codos en las habitaciones de la calle. Figuraos la tarea de dos
lenguas de mujer que quieren decir en un día todo lo que han callado en
un año. Hablaban sin cesar, pasando de un asunto a otro, sin agotar
ninguno, experimentando emociones diversas, siempre sorprendidas,
siempre conmovidas, quitándose una a otra la palabra, refiriendo,
ponderando, encareciendo, comentando, afirmando y negando.

Esto pasaba el 22 de Julio. De vez en cuando las interrumpía zumbido
lejano, estremecimiento sordo de la tierra y del aire. Era la voz de los
cañones de Inglaterra y Francia que estaban batiéndose donde todos
sabemos. Las dos mujeres cruzaban las manos, elevando los ojos al
cielo\ldots{} Los cañonazos se repetían cada vez más. Por la tarde era
un mugido incesante como el del Océano tempestuoso. En madre e hija pudo
tanto el terror, que se callaron: es cuanto hay que decir. Pensaban en
la cantidad de hombres que se tragaría en cada una de sus sacudidas el
mar irritado que bramaba a lo lejos.

Llegó la noche y los cañonazos cesaron. Muy tarde entró Tribaldos en la
casa. El pobre muchacho estaba consternado, y aunque se la echaba de
valiente, derramó algunas lágrimas.

---¿A dónde vas?---preguntó con inquietud la madre a la hija, viendo que
esta se ponía el manto sin decir para qué.

---Al Arapil---contestó Inés entregando otro manto a la condesa, que se
lo puso también sin decir nada.

Visitó Inés por breves momentos al anciano y salió de la casa y de la
ciudad, acompañada de su madre y del fiel Tribaldos. Inmenso gentío de
curiosos llenaba el camino. La batalla había sido horrenda, y querían
ver las sobras todos los que no pudieron ver el festín. Anduvieron largo
tiempo, toda la noche, hacia arriba y hacia abajo, y de acá para allá
sin encontrar lo que buscaban, ni quien razón les diera de ello. Cerca
del día vieron a miss Fly que regresaba del campo de batalla delante de
una camilla bien arreglada y cubierta, donde traían a un hombre que fue
encontrado en el Arapil Grande, lleno de heridas, sin conocimiento y con
una horrible mordida en el brazo.

Acercáronse Inés, la condesa y Tribaldos a miss Fly para hacerle
preguntas; pero esta, impaciente por seguir, les contestó:

---No sé una palabra. Dejadme continuar; llevo en esta camilla al pobre
sir Thomas Parr, que está herido de gravedad.

Siguieron ellas y Tribaldos y recorrieron el campo de batalla, que la
luz del naciente día les permitió ver en todo su horror; vieron los
cuerpos tendidos y revueltos, conservando en sus fisonomías la expresión
de rabia y espanto con que les sorprendiera la muerte. Miles de ojos sin
brillo y sin luz, como los ojos de las estatuas de mármol, miraban al
cielo sin verlo. Las manos se agarrotaban en los fusiles y en las
empuñaduras de los sables, como si fueran a alzarse para disparar y
acuchillar de nuevo. Los caballos alzaban sus patas tiesas y mostraban
los blancos dientes con lúgubre sonrisa. Las dos desconsoladas mujeres
vieron todo esto, y examinaron los cuerpos uno a uno; vieron los
charcos, las zanjas, los surcos hechos por las ruedas y los hoyos que
tantos millares de pies abrieran en el bailoteo de la lucha; vieron las
flores del campo machacadas, y las mariposas que alzaban el vuelo con
sus alas teñidas de sangre. Regresaron a Salamanca, volvieron por la
noche al campo de batalla, no ya conmovidas sino desesperadas; rezaban
por el camino, preguntaban a todos los vivos y también a los muertos.

Por último, después de repetidos viajes y exploraciones dentro y fuera
de la ciudad, en los cuales emplearon tres días, con ligeros intervalos
de residencia y descanso en la casa de la calle del Cáliz, encontraron
lo que buscaban en el hospital de sangre; improvisado en la Merced. Lo
hallaron separado de los demás, en una habitación solitaria y en poder
de un pobre fraile demente. Hicieron diligencias cerca de la autoridad
militar, y, por último, consiguieron poder llevarle, es decir, llevarme
consigo.

\hypertarget{xli}{%
\chapter{XLI}\label{xli}}

Acomodáronme en una estancia clara y bonita y en un buen lecho, que
atropelladamente dispusieron para mí. Me dieron de comer, lo cual
agradecí con toda mi alma, y empecé a encontrarme muy bien. Lo que más
contribuía a precipitar mi restablecimiento era la alegría inexplicable
que llenaba mi alma. Síntoma externo de este gozo era una jovialidad
expansiva que me impulsaba a reír por cualquier frívolo motivo.

La noche de mi entrada en la casa, mientras la condesa escribía cartas a
todo ser viviente en la sala inmediata, Inés me daba de cenar.

Nos hallábamos solos, y le conté toda, absolutamente toda la casi
increíble novela de miss Fly, sin omitir nada que me perjudicase o me
engrandeciese a los ojos de mi interlocutora. Oyome esta con atención
profunda, mas no sin tristeza, y cuando concluí, diríase que mi
constante amiga había perdido el uso de la palabra. No sé en qué vagas
perplejidades se quedó suspenso y flotante su grande ánimo. En su
fisonomía observé el enojo luchando con la compasión, y el orgullo tal
vez en pugna con la hilaridad. Pero no decía nada, y sus grandes ojos se
cebaban en mí. Por mi parte, mientras más duraba su abstracción
contemplativa, más inclinado me sentía yo a burlarme de las nubes que
oscurecían mi cielo.

---¿Es posible que pienses todavía en eso?---le dije.

---Espero que me enseñes el mechón rubio con que te han pagado el
negro\ldots{} Buena pieza, piensas que me casaré contigo, con un
perdido, con un bribón\ldots{} Te cuidaremos, y luego que estés bueno te
marcharás con tu adorada inglesa. Ninguna falta me haces.

Quería ponerse seria, y casi, casi lo lograba.

---No me marcharé, no---le dije,---porque te quiero más que a las niñas
de mis ojos; me has enamorado porque eres una criatura de otros tiempos,
porque vuestra alma, señora (me gusta tratar de vos a las personas) da
la mano a la mía y ambas suben a las alturas donde jamás llega la
vulgaridad y bajeza de los nacidos. Por vos, señora, seré Bernardo del
Carpio, el Cid y Lanzarote del Lago, acometeré las empresas más
absurdas, mataré a medio mundo y me comeré al otro medio.

---Si piensas embobarme con tales tonterías\ldots---dijo sin quererse
reír pero riendo.

---Señora---exclamé con dramático acento,---vos sois el imán de mi
existencia, la única pareja digna de la inmensidad de mi alma; adoro las
águilas que vuelan mirando cara a cara al sol, y no las gallinas que
sólo saben poner huevos, criar pollos, cacarear en los corrales y morir
por el hombre. Llevadme, llevadme con vos, señora, a los espacios de las
grandes emociones y a las excelsitudes del pensamiento. Si me
abandonáis, yo os lloraré en las ruinas; si me amáis, seré vuestro
esclavo y conquistaré diez reinos para poneros uno en cada dedo de las
manos.

---Calla, calla, tonto, farsante---dijo Inés defendiéndose como podía
contra la hilaridad que la ahogaba.

---¡Ah, señora y dueña mía!---proseguí yo reforzando mi entonación.---Me
rechazáis. Vuestro corazón es indigno del mío. Yo lo creí templado en el
fuego de la pasión, y es un pedazo de carne fofa y blanda. Os lo pedía
yo para unirlo al mío y vos le arrojáis a los soldados para que claven
en él sus bayonetas. Sois indigna de mí, señora. Os digo estas
sublimidades, y en vez de oírme, os estáis cosiendo todo el día;
tembláis cuando voy a la guerra, no pensáis más que en vuestros
chiquillos, en vez de pensar en mi gloria; y os ocupáis en hacer
guisotes y platos diversos para darme de comer: yo no como, señora; en
la región donde yo habito no se come\ldots{} De veras sois tonta: os
habéis empeñado en amarme con cariño dulce y tranquilo propio de
costureras, boticarios, sargentos, covachuelistas y sastres de portal.
¡Oh! amadme con exaltación, con frenesí, con delirio, como amaba
Bernardo del Carpio a doña Estela, y cantad las hazañas de los héroes
que son norte y faro de mi vida, y poneos delante de mí cual figura
histórica, sin cuidaros de que mi ropa esté hecha pedazos, mi mesa sin
comida, y mis hijos desnudos. ¿Qué veo? ¿Os reís? ¡Miseria! ¡Yo me muero
por vos y os reís! ¡Yo peno y vos os regocijáis! ¡Yo enflaquezco y vos
os presentáis a mí fresca, alegre y gordita!

Inés lloraba de risa, pero de una manera tan franca y natural, que todo
el enojo se iba desvaneciendo en aquellas chispas de alegría. Mi corazón
se entendió con el suyo, como los hermanos que por un momento riñen,
para quererse más.

---Os abandono, porque amáis a otro, a una criatura vulgar y
antipoética, señora ---continué mirando su frente y haciendo con mis
dedos movimiento semejante al abrir y cerrar de unas tijeras;---pero
quiero llevarme un recuerdo vuestro, y así os corto ese mechón que os
cuelga sobre la frente.

Diciéndolo, cogí la preciosa cabeza y le di mil besos.

---Que me lastimas, bárbaro---gritó sin cesar de reír.

Acudió la condesa que en la cercana habitación estaba, y al verla, Inés,
más roja que una amapola, le dijo:

---Es Gabriel, que la está echando de gracioso.

---No hagáis ruido que estoy escribiendo. Todavía me faltan muchas
cartas, pues tengo que escribir a Wellington, a Graham, a Castaños, a
Cabarrús, a Azanza, a Soult, a O'Donnell y al Rey José.

Mi adorada suegra tenía la manía de las cartas. Escribía a todo el
mundo, y de todos lograba respuesta. Su colección epistolar era un
riquísimo archivo histórico, del cual sacaré algún día no pocas
preciosidades.

Al día siguiente mi suegra fue a visitar a miss Fly, a quien como he
dicho, había tratado en el Puerto y reconocido últimamente en Salamanca.
Athenais pagó la visita a la condesa en el mismo día. Vino elegantemente
vestida, deslumbradora de hermosura y de gracia. Servíale de caballero
el coronel Simpson, siempre encarnadito, vivaracho, acicalado y
compuesto como un figurín, y siempre honrando todos los objetos y
personas con la cuádruple mirada de dos ojos y dos vidrios que jamás
descansaban en su investigadora observación. Yo me había levantado y
desde un sillón asistí sin moverme a la visita, que no fue larga, aunque
sí digna de ocupar el penúltimo lugar en esta verídica historia.

---¿De modo que parte usted definitivamente para Inglaterra?---dijo la
condesa.

---Sí, señora---repuso Athenais, que no se dignaba mirarme---estoy
cansada de la guerra y de España, y deseo abrazar a mi padre y hermanas.
Si alguna vez vuelvo a España tendré el gusto de visitaros.

---Antes quizás tenga yo el de escribir a usted---dijo mi suegra
acordándose de que había papel y plumas en el mundo.---Por falta de
tiempo no he escrito ya a lord Byron a quien conocí en Cádiz. No llevará
usted malos recuerdos de España.

---Muy buenos. Me he divertido mucho en este extraño país; he estudiado
las costumbres, he hecho muchos dibujos de los trajes y gran número de
paisajes en lápiz y acuarela. Espero que mi álbum llame la atención.

---También llevará usted memoria de las tristes escenas de la
guerra---dijo Amaranta con emoción.

---Los franceses nada respetan---indicó miss Fly con la indiferencia que
se emplea en las visitas para hablar del tiempo.

---En su retirada---afirmó Simpson---han destruido todos los pueblos de
la ribera del Tormes. No nos perdonan que les hayamos matado cinco mil
hombres y cogido siete mil prisioneros con dos águilas, seis banderas y
once cañones\ldots{} ¡Grandiosa e importante batalla! No puedo menos de
felicitar al Sr.~de Araceli---añadió haciéndome el honor de dirigirse a
mí---por su buen comportamiento durante la acción. El brigadier Pack y
el honorable general Leith han hecho delante de mí grandes elogios de
usted. Me consta que su excelencia el gran Wellington no ignora nada de
lo que tanto os favorece.

---En ese caso---dije---tal vez se disipe la prevención que su
excelencia tenía contra mí por motivos que nunca pude saber.

Athenais se puso pálida; mas dominándose al instante, no sólo se atrevió
a fijar en mí sus lindos ojos de cielo, sino que se rió y de muy buena
gana, según parecía.

---Este caballero---contestó con jovialidad asombrosa por lo bien
fingida---ha tenido la desgracia y la fortuna de pasar por mi amante a
los ojos de los ociosos del campamento. En España, el honor de las damas
está a merced de cualquier malicioso.

---¡Pero cómo! ¿Es posible, señora?---exclamé fingiéndome sorprendido y
además de sorprendido encolerizado.---¿Es posible que por aquel
felicísimo encuentro nuestro\ldots? No sabía nada ciertamente. ¡Y se han
atrevido a calumniar a usted!\ldots{} ¡Qué horror!

---Y poco ha faltado para que me supusieran casada con vos---añadió
apartando los ojos de mí, contra lo que las conveniencias del diálogo
exigían.---Me ha servido de gran diversión, porque a la verdad, aunque
os tengo por persona estimable\ldots{}

---No tanto que pudiera merecer el honor\ldots---añadí completando la
frase.---Eso es claro como el agua.

---Todo provino de que alguien nos vio juntos en la ciudad, cuando para
salvaros de aquellos infames soldados, pasasteis por mi criado durante
unas cuantas horas---dijo Athenais, coqueteando y haciendo monerías.

---Ahora falta saber si por vanidad pueril fuisteis vos mismo quien se
atrevió a propalar rumores tan ridículos acerca de una noble dama
inglesa, que jamás ha pensado enamorarse en España, y menos de un hombre
como vos.

---¡Yo, señora! El coronel Simpson es testigo de lo que pensaba yo sobre
el particular.

---Los rumores---dijo el simpático Abraham,---partieron de la
oficialidad inglesa y empezaron a circular cuando Araceli volvió de
Salamanca y Athenais no.

---Y vos, mi querido sir Abraham Simpson---dijo miss Fly con cierto
enojo,---disteis circulación a las groserías que corrían acerca de mí.

---Permitidme decir, mi querida Athenais---indicó Simpson en
español---que vuestra conducta ha sido algo extraña en este asunto. Sois
orgullosa\ldots{} lo sé\ldots{} creíais rebajaros sólo ocupándoos del
asunto\ldots{} Lo cierto es que oíais todo, y callabais. Vuestra
tristeza, vuestro silencio hacían creer\ldots{}

---Me parece que no conocéis bien los hechos---dijo Athenais empezando a
ruborizarse.

---Todos hablaban del asunto; el mismo Wellington se ocupó de él. Os
interrogaron con delicadeza, y contestasteis de un modo vago. Se dijo
que pensabais pedir el cumplimiento de las leyes inglesas sobre el
matrimonio; calumnia, pura calumnia; pero ello es que lo decían y vos no
lo negabais\ldots{} yo mismo os llamé la atención sobre tan grave
asunto, y callasteis\ldots{}

---Conocéis mal los hechos---repitió Athenais más ruborizada,---y además
sois muy indiscreto.

---Es que, según mi opinión---dijo Simpson,---llevasteis la delicadeza
hasta un extremo lamentable, mi querida Athenais\ldots{} Os sentíais
ultrajada sólo por la idea de que creyeran\ldots{} pues\ldots{} una
mujer de vuestra clase\ldots{} No quiero ofender al señor; pero\ldots{}
es absurdo, monstruoso. La Inglaterra, señora, se hubiera estremecido en
sus cimientos de granito.

---¡Sí, en sus cimientos de granito!---repetí yo.---¡Qué hubiera sido de
la Gran Bretaña!\ldots{} Es cosa que espanta.

Miss Fly me dirigió una mirada terrible.

---En fin---dijo la condesa,---los rumores circularon\ldots{} yo misma
lo supe\ldots{} Pero la cosa no vale la pena. Si la Gran Bretaña se
mantiene sin mancilla\ldots{}

Miss Fly se levantó.

---Señora---le dije con el mayor respeto,---sentiría que usted dejase a
España sin que yo pudiese manifestarle la profundísima gratitud que
siento\ldots{}

---¿Por qué, caballero?---preguntó llevando el pañuelo a su agraciada
boca.

---Por su bondad, por su caridad. Mientras viva, señora, bendeciré a la
persona que me recogió del campo de batalla con otros infelices
compañeros.

---Estáis en gran error---exclamó riendo.---Yo no he pensado en tal
cosa. Vos sin duda lo deseabais. Recogí a varios, sí; pero no a vos. Os
han engañado. Me visteis en la Merced recorriendo las salas y
dormitorios\ldots{} No quiero que me atribuyan el mérito de obras que no
me pertenecen.

---Entonces, señora, permítame usted que le dé las gracias por\ldots{}
No, lo que quiero decir es que ruego a usted no me guarde rencor por
haber sido causa, aunque inocente, de esos ridículos rumores.

---¡Oh, oh!\ldots{} No haga caso de semejante necedad. Soy muy superior
a tales miserias\ldots{} ¡La calumnia! Acaso me importa algo\ldots{}
¡Vuestra persona! ¿Significa algo para mí? Sois vanidoso y petulante.

Miss Fly hacía esfuerzos extraordinarios por conservar en su semblante
aquella calma inglesa que sirve de modelo a la majestuosa impasibilidad
de la escultura. Miraba a los cristales, a los viejos cuadros, al suelo,
a Inés, a todos menos a mí.

---Entonces, señora---añadí,---puesto que ningún daño ha padecido usted
por causa mía\ldots{}

---Ninguno, absolutamente ninguno. Os hacéis demasiado honor, caballero
Araceli, y sólo con pedirme excusas por la vil calumnia, sólo con
asociar vuestra persona a la mía, estáis faltando al comedimiento, sí,
faltando a la consideración que debe inspirar en todo lo habitado una
hija de la Gran Bretaña.

---Perdón, señora, mil veces perdón. Sólo me resta decir a usted que
deseo ser su humildísimo servidor y criado aquí y en todas partes y en
todas las ocasiones de mi vida. ¿También así falto al comedimiento?

---También\ldots{} pero, en fin, admito vuestros homenajes. Gracias,
gracias---dijo con altivez.---Adiós.

Al fin de la visita, aunque repetidas veces se empeñó en reír, no pudo
conseguirlo sino a medias. Sus manos temblaban, destrozando las puntas
del chal amarillo. Despidiose cariñosamente de la condesa, y con mucha
ceremonia de Inés y de mí.

---¿Y no será usted tan buena que nos escriba alguna vez para enterarnos
de su salud?---le dije.

---¿Os importa algo?

---¡Mucho, muchísimo!---respondí con vehemencia y sinceridad profunda.

---¡Escribiros! Para eso necesitaría acordarme de vos. Soy muy
desmemoriada, señor de Araceli.

---Yo, mientras viva, no olvidaré la generosidad de usted, Athenais. Me
cuesta mucho trabajo olvidar.

---Pues a mí no,---dijo mirándome por última vez.

Y en aquella mirada postrera que sus ojos me echaron, puso tanto
orgullo, tanta soberbia, tanta irritación que sentí verdadera pena. Al
fin salió de la sala. La palidez de su rostro y la furia de su alma la
hacían terrible y majestuosamente bella.

Pocos momentos después aquel hermoso insecto de mil colores, que por
unos días revoloteara en caprichosos círculos y juegos alrededor de mí,
había desaparecido para siempre.

Muchas personas que anteriormente me han oído contar esto sostienen que
jamás ha existido miss Fly; que toda esta parte de mi historia es una
invención mía para recrearme a mí propio y entretener a los demás; pero
¿no debe creerse ciegamente la palabra de un hombre honrado?

Por ventura, quien de tanta rectitud dio pruebas, ¿será capaz ahora de
oscurecer su reputación con ficciones absurdas y con fábricas de la
imaginación que no tengan por base y fundamento a la misma verdad, hija
de Dios?

Poco después de que los dos ingleses nos dejaron solos, la condesa dijo
a Inés:

---Hija mía, ¿tienes inconveniente en casarte con Gabriel?

---No, ninguno---repuso ella con tanto aplomo, que me dejó sorprendido.

Con inefable afecto besé su hermosa mano que tenía entre las mías.

---¿Está tranquila y satisfecha tu alma, hija mía?

---Tranquila y satisfecha---repuso.---¡Pobrecita miss Fly!

Ambos nos miramos. Un cielo lleno de luz divina, y de inexplicable
música de ángeles flotaba entre uno y otro semblante\ldots{} Si es
posible ver a Dios, yo lo veía, yo.

---¡Qué hermoso es vivir!---exclamé.---¡Qué bien hizo Dios en criarnos a
los dos, a los tres! ¿Hay felicidad comparable a la mía? ¿Pero esto qué
es, es vivir o es morir?

Al oír esto, la condesa, que había corrido a abrazamos, se apartó de
nosotros. Fijó los ojos en el suelo con tristeza. Inés y yo pensamos al
mismo tiempo en lo mismo y sentimos la misma pena, una lástima íntima y
honda que turbaba nuestra dicha.

---¿Qué tal está hoy?---preguntó Amaranta.

---Muy mal---repuso Inés.---Sólo vive su espíritu.

Amaranta dio un suspiro y nos abrazó de nuevo.

---Levántate---me dijo Inés. Vamos los dos allá. Hace ya hora y media
que no me ha visto, y estará muy taciturno.

Aunque extenuado y débil, me levanté y la seguí apoyado en su brazo.

---Haré la última tentativa y venceré---dijo cerca de la guarida del
masón.---Le he observado muy bien todo el día, y el pobrecito no desea
ya sino rendirse.

\hypertarget{xlii}{%
\chapter{XLII}\label{xlii}}

Al entrar en la solitaria y triste estancia, vimos a Santorcaz
apoltronado en el sillón y leyendo atentamente un libro. Alzó la vista
para mirarnos. Inés, poniendo la mano en su hombro, le dijo con cariñoso
gracejo:

---Padre, ¿sabes que me caso?

---¿Te casas?---dijo con asombro el anciano soltando el libro y
devorándonos con los ojos.---¡Tú!\ldots{}

---Sí---continuó Inés en el mismo tono.---Me caso con este pícaro
Gabriel, con un opresor del pueblo, con un verdugo de la humanidad, con
un satélite del despotismo.

Santorcaz quiso hablar, pero la emoción entorpecía su lengua. Quiso
reír, quiso después ponerse serio y aun colérico; mas su semblante no
podía expresar más que turbación, vacilación y desasosiego.

---Y como mi marido tendrá que servir a los reyes, porque éste es su
oficio---prosiguió Inés,---me veré obligada, querido padre, a reñir
contigo. Ahora me ha dado por la nobleza; quiero ir a la corte, tener
palacio, coches y muchos y muy lujosos criados\ldots{} Yo soy así.

---Bromea usted, señora doña Inesita---dijo Santorcaz en tono
agri-dulce, recobrando al fin el uso de la palabra.---¿No hay más que
casarse con el primero que llega?

---Hace tiempo que le conozco, bien lo sabes---dijo ella
riendo.---Muchas veces te lo he dicho\ldots{} Ahora, padre, tú te
quedarás aquí con Juan y Ramoncilla, y yo me voy a Madrid con mi marido.
Te entretendrás en fundar una gran logia y en leer libros de
revoluciones y guillotinas para que acabes de volverte loco, como D.
Quijote con los de caballerías.

Diciendo esto abrazó al anciano y se dejó besar por él.

---¡Adiós, adiós!---repitió ella---puesto que no nos hemos de ver más,
despidámonos bien.

---Picarona---dijo él estrechándola amorosamente contra su pecho y
sentándola sobre sus rodillas.---¿Piensas que te voy a dejar marchar?

---¿Y piensas que yo voy a esperar a que tú me dejes salir? Padre, ¿te
has vuelto tonto? ¿Has olvidado a la persona que ha estado en casa y que
tiene tanto poder?\ldots{} ¿No sabes que estás preso?\ldots{} ¿crees que
no hay justicia ni leyes, ni corregidores? Atrévete a respirar\ldots{}

El masón apartó de sí a la muchacha, trató de levantarse, mas
impidiéronselo sus doloridas piernas, y golpeando los brazos del sillón,
habló así:

---Pues no faltaba más\ldots{} marcharte tú y dejarme\ldots{}
Araceli---añadió dirigiéndose a mí con bondad.---Ya que mi hija tiene la
debilidad de quererte, te permito que seas su marido; pero tú y ella os
quedaréis conmigo.

---A buena parte vas con súplicas---dijo Inés riendo.---A fe que mi
marido hace buenas migas con los masones. Él y yo detestamos el
populacho y adoramos a reyes y frailes.

---Bueno, me quedaré---dijo Santorcaz con ligera inflexión de broma en
su tono .---Me moriré aquí. Ya sabes cómo está mi salud, hija mía: vivo
de milagro. En estos días que has estado enojada conmigo, yo sentía que
la vida se me iba por momentos, como un vaso que se vacía. ¡Ay! queda
tan poco, que ya veo, ya estoy viendo el fondo negro.

---Todo se arreglará---dije yo acercando mi asiento al del
enfermo.---Nos llevaremos con nosotros al enemigo de los reyes.

---Eso es, eso\ldots{} Gabriel ha hablado con tanto talento como
Voltaire---dijo el masón con repentino brío.---Me llevaréis con
vosotros\ldots{} No tengo inconveniente, la verdad.

---Bueno, le llevaremos---dijo Inés abrazando a su padre,---le
llevaremos a Madrid, donde tenemos una casa muy grande, grandísima, y en
la cual estaremos muy anchos, porque mi madre se va con todos sus
criados a vivir a Andalucía para no volver más.

---¡Para no volver más!---dijo el enfermo con turbación.---¿Quién te lo
ha dicho?

---Ella misma. Se separa de mí mientras tú vivas.

---¡Mientras yo viva!\ldots{} Ya lo ves. Por eso conocerás la inmensidad
de su aborrecimiento.

---Al contrario, padre---dijo Inés con dulzura,---se marcha porque tú no
la puedes ver, y para dejarme en libertad de que te cuide y esté contigo
en tu enfermedad. Lo que te decía hace poco de abandonarte y marcharme
sola con mi marido era una broma.

En los párpados del anciano asomaban algunas lágrimas que él hubiera
deseado poder contener:

---Lo creo; pero eso de que tu madre se separe de ti por concederme el
inestimable beneficio de tu compañía, me parece una farsa.

---¿No lo crees?

---No: ¿a que no se atreve a venir aquí y a decirlo delante de mí?

---Eso quisieras tú, padrito. ¿Cómo ha de venir a decirte eso, ni
ninguna otra cosa, cuando se ha marchado?

---¡Se ha marchado! ¡Se ha marchado!---exclamó Santorcaz con un
desconsuelo tan profundo que por largo rato quedó estupefacto.

---¿Pues no lo sabes? ¿No sentiste la voz de unos señores ingleses? Esos
la acompañan hasta Madrid, de donde partirá para Andalucía.

El dominio de aquella hermosa y excelente criatura sobre su padre era
tan grande que Santorcaz pareció creerlo todo tal como ella lo decía.
Clavaba los ojos en el suelo y lentamente se acariciaba la barba.

---Búscala por toda la casa---prosiguió Inés.---A fe que tendría gusto
la señora en vivir dentro de esta jaula de locos.

---¡Se ha marchado!---repitió sombríamente Santorcaz, hablando consigo
mismo.

---Y no me costó poco quedarme---añadió ella haciendo con manos y rostro
encantadoras monerías.---Su deseo era llevarme consigo. Allá le dijo no
sé quién\ldots{} nada se puede tener oculto\ldots{} que yo te había
tomado gran cariño. Sólo por esta razón venía dispuesta a perdonarte, a
reconciliarse contigo\ldots{} Esto era lo más natural, pues tú la habías
amado mucho, y ella te había amado a ti\ldots{} Pero tú estás
loco\ldots{} la recibiste como se recibe a un enemigo\ldots{} te pusiste
furioso\ldots{} te negaste a ser bueno con ella. Me has hecho pasar unos
ratos que no te perdono.

Las lágrimas corrieron hilo a hilo por la cara de Santorcaz.

---Mi deber era huir de esta casa aborrecida, huir con ella,
abandonándote a las perversidades y rencores de tu corazón---dijo Inés
que reunía a la santidad de los ángeles cierta astucia de
diplomático.---Pero me acordé de que estabas enfermo y postrado; se lo
dije\ldots{}

El masón miró a su hija, preguntándole con los ojos cuanto es posible
preguntar.

---Se lo dije, sí---prosiguió ella,---y como esa señora tiene un corazón
bueno, generoso y amante; como nunca, nunca ha deseado el mal ajeno, ni
ha vivido del odio; como sabe perdonar las ofensas y hacer bien a los
que la aborrecen\ldots{} ¡ay! no lo creerás ni lo comprenderás, porque
un corazón de hierro como el tuyo, no puede comprender esto.

---Sí, lo creo, lo comprendo---dijo Santorcaz secando sus lágrimas.

---Pues bien; ella misma convino en que no me separase de ti, para
consolarte y fortalecerte en tus últimos días; y como ella y tú no
podéis estar juntos en un mismo sitio, determinó retirarse. Acordamos
que me case con el verdugo de la humanidad y que Gabriel y yo te
llevemos a vivir con nosotros\ldots{}

---¿Y se marchó?\ldots{} ¿pero se marchó?---preguntó Santorcaz con un
resto de esperanza.

---Y se marchó, sí señor. Venía dispuesta a reconciliarse contigo, a
quererte como yo te quiero. Ha llorado mucho la pobrecita, al ver que
después de tantos años, después de tantas desgracias como le han
ocurrido por ti, después de tanto daño como le has hecho, aún te niegas
a pronunciar una palabra cristiana, a borrar con un momento de
generosidad todas las culpas de tu vida, a descargar tu conciencia y
también la suya del peso de un resentimiento insoportable. Se ha
marchado perdonándote. Dios se encargará de juzgarte a ti, cuando en el
momento del juicio le presentes como únicos méritos de tu existencia,
ese corazón insensible y perverso, o mejor dicho, ese nido de culebras,
a las cuales has criado, a las cuales echas de comer todos los días para
que crezcan y vivan siempre, y te muerdan aquí y en la eternidad de la
otra vida.

El anciano se revolvía con angustia en su sillón; el llanto había cesado
de afluir de sus ojos; tenía el rostro encendido, las manos crispadas,
echada la cabeza hacia atrás, y entrecortaba su aliento una sofocación
fatigosa.

---Padre---exclamó Inés echándole los brazos al cuello.---Sé bueno, sé
generoso y te querré más todavía. Ya sabes mi deseo: prepárate a
cumplirlo, y mi madre volverá. Yo la llamaré y volverá.

Los músculos de Santorcaz se tendieron, poniéndose rígidos, cerró los
ojos, inclinó la cabeza, y su aspecto fue el de un cadáver. En aquel
mismo instante abriose la puerta y penetró la condesa, pálida, llorosa.
Andando lentamente, adelantó hasta llegar al lado del enfermo que seguía
inerte, mudo y aparentemente sin vida. Alarmados todos, acudimos a él, y
con ayuda de Juan y Ramoncilla le acostamos en su lecho; al instante
hicimos venir el médico que ordinariamente le asistía.

Inés y la condesa le observaban atentamente, y fijaban sus ojos en el
semblante demacrado, pero siempre hermoso, del desgraciado masón.
Miraban con espanto aquella sima, aterradas de lo que en su profundidad
había, sin comprenderlo bien.

El médico, luego que le examinara, anunció su próximo fin, añadiendo que
se maravillaba de que alargase tanto su vida, pues el día anterior casi
le diputó por muerto, aunque ocultó a Inés el fatal pronóstico. Cerca ya
de la noche, un hondo suspiro nos anunció que recobraba de nuevo el
conocimiento; abrió los ojos, y revolviéndolos con espanto por todo el
recinto de la estancia, fijolos en la condesa, cuyo semblante iluminaba
la triste luz.

---¡Otra vez estás aquí!---exclamó con voz torpe y expresión de hastío y
cólera ;---¿otra vez aquí? Mujer, sabe que te aborrezco. ¡La cárcel, el
destierro, el patíbulo\ldots{} todo te ha parecido poco para
perseguirme!\ldots{} ¿Por qué vienes a turbar mi felicidad? Vete, ¿por
qué agarras a mi hija con esa mano amarilla como la de la muerte? ¿Por
qué me miras con esos ojos plateados que parecen rayos de luna?

---Padre, no hables así, que me das miedo---gritó Inés abrazándole,
llenos los ojos de lágrimas.

La condesa no decía nada y lloraba también.

Santorcaz, después de aquella crisis de su espíritu, cayó en nuevo sopor
profundísimo, y cerca de la madrugada, recobró el conocimiento con un
despertar sereno y sosegado. Su mirar era tranquilo, su voz clara y
entera, cuando dijo: ---Inés, niña mía, ángel querido ¿estás aquí?

---Aquí estoy, padre---respondió ella acudiendo cariñosamente a su
lado.---¿No me ves?

Inés tembló al observar que los ojos de su padre se fijaban en los de la
condesa.

---¡Ah!---dijo Santorcaz sonriendo ligeramente.---Está ahí\ldots{} la
veo\ldots{} viene hacia acá\ldots{} ¿Pero por qué no habla?

La condesa había dado algunos pasos hacia el lecho, pero permanecía
muda.

---¿Por qué no habla?---repitió el enfermo.

---Porque te tiene miedo---dijo Inés---como te lo tengo yo, y no se
atreve la pobrecita a decirte nada. Tú tampoco le dices nada.

---¿Qué no?---indicó el masón con asombro.---Hace dos horas que estoy
dirigiéndole la palabra\ldots{} tengo la boca seca de tanto hablar, y no
me contesta. ¡Ay!---añadió con dolor y volviendo el rostro---es
demasiado cruel con este infeliz.

---¿La quieres mucho, padre?---preguntó Inés tan conmovida que apenas
entendimos sus palabras.

---¡Oh, mucho, muchísimo!---exclamó el enfermo oprimiéndose el corazón.

---Por eso desde que la has visto---continuó la muchacha---le has pedido
perdón por los ligeros perjuicios que sin querer le has causado. Todos
te hemos oído y hemos alabado a Dios por tu buen comportamiento.

---¿Me habéis oído?\ldots---dijo él con asombro, mirándonos a
todos.---¿Me has oído tú\ldots{} me ha oído ella\ldots{} me ha oído
también Araceli? Lo había dicho bajo, muy bajito para que sólo Dios me
oyera, y lo ignorara todo ser.

Amaranta, tomando la mano de Santorcaz, dijo:

---Hace mucho, mucho tiempo que deseaba perdonarte; si en cualquiera
ocasión, desde que Inés vino a mi poder, te hubieras presentado a mí
como amigo\ldots{} Yo también he tenido resentimientos; pero la
desgracia me ha enseñado pronto a sofocarlos\ldots{}

Lágrimas abundantes cortaron su voz.

---Y yo---dijo Santorcaz con voz apacible y ademán sereno.---Yo que voy
a morir, no sé lo que pasa en mi corazón. Él nació para amar. Él mismo
no sabe si ha amado o ha aborrecido toda su vida.

Después de estas palabras todos callaron por breve rato. Las almas de
aquellos tres individuos, tan unidos por la Naturaleza y tan separados
por las tempestades del mundo, se sumergían, por decirlo así, en lo
profundo de una meditación religiosa y solemne sobre su respectiva
situación. Inés fue la primera que rompió el grave silencio, diciendo:

---Bien se conoce, querido padre, que eres un hombre bueno, honrado,
generoso. Si has tenido fama de lo contrario, es porque te han
calumniado. Pero nosotras, nosotras dos y también Araceli, te conocemos
bien. Por eso te amamos tanto.

---Sí---respondió el masón, como responde el moribundo a las preguntas
del confesor.

---Si has hecho algunas cosas malas---continuó Inés---es decir, que
parecen malas, ha sido por broma\ldots{} Esto lo comprendo
perfectamente. Por ejemplo: cuando te perseguían\ldots{} apuesto a que
la persecución no era ni la mitad de lo que tú te figurabas\ldots{}
pero, en fin, sea lo que quiera. Lo cierto es que te enfadaste, y con
muchísima razón, porque tú estabas enamorado, querías ser bueno,
querías\ldots{} Pero hay familias orgullosas\ldots{} Es preciso también
considerar que una familia noble debe tener cierto punto\ldots{} Dios
primero y el mundo después no han querido que todos sean iguales.

---Pero se ven castigos, o si no castigos, justicias providenciales en
la tierra---dijo Santorcaz bruscamente, mirando a Amaranta.---Señora
condesa, hoy mismo ha consentido usted que su hija única y noble
heredera se case con un chico de las playas de la Caleta. ¡Bravo
abolengo, por cierto!

---Mejor sería---repuso la condesa---decir con un joven honrado, digno,
generoso, de mérito verdadero y de porvenir.

---¡Oh! señora mía, eso mismo era yo hace veinte años---afirmó Santorcaz
con tristeza.

Después cerró los ojos, como para apartar de sí imágenes dolorosas.

---Es verdad---dijo Inés entre broma y veras;---pero tú te entregaste a
la desesperación, padre querido, tú no tuviste la fortaleza de ánimo de
este opresor de los pueblos, tú no luchaste como él contra la
adversidad, ni conquistaste escalón por escalón un puesto honroso en el
mundo. Tú te dejaste vencer por la desgracia; corriste a París, te
uniste a los pícaros revolucionarios que entonces se divertían en matar
gente. Agraviados ellos como tú y tú como ellos, todos creíais que
cortando cabezas ajenas ganabais alguna cosa y valían más los que se
quedaran con ella sobre los hombros\ldots{} Viniste luego a España con
el corazón lleno de venganza. Tú querías que nos divirtiéramos aquí con
lo que se divertían allá; la gente no ha querido darte gusto y te
entretuviste con las mojigangas y gansadas de los masones, que según
ellos dicen, hacen mucho, y según yo veo, no hacen nada\ldots{}

---Sí---dijo el anciano.

---Al mismo tiempo procurabas hacer daño a la persona que más debías
amar\ldots{} Yo sé que si ella no te hubiera despreciado como te
despreciaba, tú habrías sido bueno, muy bueno, y te habrías desvivido
por ella\ldots{}

---Sí, sí---repitió él.

---Esto es claro: Dios consiente tales cosas. A veces dos personas
buenas parece que se ponen de acuerdo para hacer maldades, sin caer en
la cuenta de que diciéndose dos palabras, concluirían por abrazarse y
quererse mucho.

---Sí, sí.

---Y no me queda duda---continuó Inés derramando sin cesar aquel
torrente de generosidad sobre el alma del pobre enfermo,---no me queda
duda de que te apoderaste de mí porque me querías mucho y deseabas que
te acompañara.

Santorcaz no afirmó ni negó nada.

---Lo cual me place mucho---prosiguió ella.---Has sido para mí un padre
cariñoso. Declaro que eres el mejor de los hombres, que me has amado,
que eres digno de ser respetado y querido, como te quiero y te respeto
yo, dando el ejemplo a todos los que están presentes.

El revolucionario miró a su hija con inefable expresión de
agradecimiento. La religión no hubiera ganado mejor un alma.

---Muero---dijo con voz conmovida D. Luis, alargando la mano derecha a
Amaranta y la izquierda a su hija---sin saber cómo me recibirá Dios. Me
presentaré con mi carga de culpas y con mi carga de desgracias, tan
grandes la una y la otra, que ignoro cuál será de más peso\ldots{} Mi
pecho ha respirado venganza y aborrecimiento por mucho tiempo\ldots{} he
creído demasiado en las justicias de la tierra: he desconfiado de la
Providencia; he querido conquistar con el terror y la violencia lo que a
mi entender me pertenecía; he tenido más fe en la maldad que en la
virtud de los hombres; he visto en Dios una superioridad irritada y
tiránica, empeñada en proteger las desigualdades del mundo; he carecido
por completo de humildad; he sido soberbio como Satán, y me he burlado
del paraíso a que no podía llegar; he hecho daño, conservando en el
fondo de mi alma cierto interés inexplicable por la persona ofendida; he
corrido tras el placer de la venganza, como corre en el desierto el
sediento tras un agua imaginaria; he vivido en perpetua cólera,
despedazándome el corazón con mis propias uñas. Mi espíritu no ha
conocido el reposo hasta que traje a mi lado un ángel de paz que me
consoló con su dulzura, cuando yo la mortificaba con mi cólera. Hasta
entonces no supe que existían las dos virtudes consoladoras del corazón,
la caridad y la paciencia. Que las dos llenen mi alma, que cierren mis
ojos y me lleven delante de Dios.

Diciendo esto, se desvaneció poco a poco. Parecía dormido. Las dos
mujeres, arrodilladas a un lado y otro, no se movían. Creí que había
muerto; pero acercándome, observe su respiración tranquila. Retireme a
la sala inmediata, e Inés me siguió poco después. Entre los dos
convenimos en llamar al prior de Agustinos, varón venerable, que había
sido amigo muy querido del padre de Santorcaz.

~

Por la mañana, después de la piadosa ceremonia espiritual, Santorcaz nos
rogó que le dejásemos solo con la condesa. Largo rato hablaron a solas
los dos; mas como de pronto sintiéramos ruido, entramos y vimos a
Amaranta de rodillas al pie del lecho, y a él incorporado, inquieto, con
todos los síntomas de un delirio atormentador. Con sus extraviados ojos
miraba a todos lados, sin vernos, atento sólo a los objetos imaginados
con que su espíritu poblaba la oscura estancia.

---Ya me voy---decía,---ya me voy\ldots{} ¡adiós! es de día\ldots{} No
tiembles\ldots{} esos pasos que se sienten son los de tu padre que viene
con un ejército de lacayos armados para matarme\ldots{} No me
encontrarán\ldots{} Saldré por la ventana del torreón\ldots{} ¡Cielo
santo! han quitado la escala me arrojaré aunque muera\ldots{} Dices
bien, mi cuerpo, encontrado al pie de estos muros, será tu vergüenza y
la deshonra de esta casa\ldots{} ¿Esperaré? ¿No quieres que
aguarde?\ldots{} Ya están ahí; tu padre golpea la puerta y te
llama\ldots{} Adiós: me arrojaré al campo\ldots{} También allá abajo hay
criados con palos y escopetas. Dios nos abandona porque somos
criminales. Me ocurre una idea feliz. Estás salvada\ldots{} escóndete
allí\ldots{} pasa a tu alcoba. Déjame recoger estos vasos de valor,
estos candelabros de plata. Los llevaré conmigo, y procuraré escurrirme
con mi tesoro robado por la cornisa del torreón hasta llegar al techo de
las cuadras. Adiós\ldots{} saldré; abre la puerta y grita: \emph{¡al
ladrón, al ladrón!} Conocerán tu deshonra Dios y tu padre, si quieres
revelársela; pero no esa turba soez. Vieron entrar un hombre, pero
ignoran quién es y a lo que vino. Alma mía, ten valor; haz bien tu
papel. Grita \emph{¡al ladrón, al ladrón!\ldots{}} Adiós\ldots{} Ya
salgo; me escurro por estas piedras resbaladizas y verdosas\ldots{} Aún
no me han visto los de abajo. Es preciso que me vean\ldots{} ¡Oh! Ya me
ven los miserables con mi carga de preciosidades, y todos gritan:
\emph{¡al ladrón, al ladrón!} ¡Qué inmensa alegría siento! Nadie sabrá
nada, vida y corazón mío; nadie sabrá nada, nada\ldots{} Cayó hacia
atrás, estremeciéndose ligeramente, y su alma hundiose en el piélago sin
fondo y sin orillas. Inés y yo nos acercamos con religioso respeto al
exánime cuerpo. En nuestro estupor y emoción creímos sentir el rumor de
las aguas negras y eternas, agitándose al impulso de aquel ser que había
caído en ellas; pero lo que oíamos era la agitada respiración de la
condesa, que lloraba con amargura, sin atreverse a alzar su frente
pecadora.

\hypertarget{xliii}{%
\chapter{XLIII}\label{xliii}}

Los que quieran saber cómo y cuándo me casé, con otras particularidades
tan preciosas como ignoradas acerca de mi casi inalterable tranquilidad
durante tantos años, lean, si para ello tienen paciencia, lo que otras
lenguas menos cansadas que la mía narrarán en lo sucesivo. Yo pongo aquí
punto final, con no poco gusto de mis fatigados oyentes y gran placer
mío por haber llegado a la más alta ocasión de mi vida, cual fue el
suceso de mis bodas, primer fundamento de los sesenta años de
tranquilidad que he disfrutado, haciendo todo el bien posible, amado de
los míos y bienquisto de los extraños. Dios me ha dado lo que da a todos
cuando lo piden buscándolo, y lo buscan sin dejar de pedirlo. Soy hombre
práctico en la vida y religioso en mi conciencia. La vida fue mi
escuela, y la desgracia mi maestra. Todo lo aprendí y todo lo tuve.

Si queréis que os diga algo más (aunque otros se encargarán de sacarme
nuevamente a plaza, a pesar de mi amor a la oscuridad), sabed que una
serie de circunstancias, difíciles de enumerar por su muchedumbre y
complicación, hicieron que no tomase parte en el resto de la guerra;
pero lo más extraño es que desde mi alejamiento del servicio empecé a
ascender de tal modo que aquello era una bendición.

Habiendo recobrado el aprecio y la consideración de lord Wellington,
recibí de este hombre insigne pruebas de cordial afecto, y tanto me
atendió y agasajó en Madrid que he vivido siempre profundamente
agradecido a sus bondades. Uno de los días más felices de mi vida fue
aquel en que supimos que el duque de Ciudad-Rodrigo había ganado la
batalla de Waterloo.

Obtuve poco después de los Arapiles el grado de teniente coronel. Pero
mi suegra, con el talismán de su jamás interrumpida correspondencia, me
hizo coronel, luego brigadier, y aún no me había repuesto del susto,
cuando una mañana me encontré hecho general.

---Basta---exclamé con indignación después de leer mi hoja de
servicios.---Si no pongo remedio, serán capaces de hacerme capitán
general sin mérito alguno.

Y pedí mi retiro.

Mi suegra seguía escribiendo para aumentar por diversos modos nuestro
bienestar, y con esto y un trabajo incesante, y el orden admirable que
mi mujer estableció en mi casa (porque mi mujer tenía la manía del orden
como mi suegra la de las cartas) adquirí lo que llamaban los antiguos
\emph{aurea mediocritas}; viví y vivo con holgura, casi fui y soy rico,
tuve y tengo un ejército brillante de descendientes entre hijos, nietos
y biznietos.

Adiós, mis queridos amigos. No me atrevo a deciros que me imitéis,
porque sería inmodestia; pero si sois jóvenes, si os halláis postergados
por la fortuna, si encontráis ante vuestros ojos montañas escarpadas,
inaccesibles alturas, y no tenéis escalas ni cuerdas, pero sí manos
vigorosas; si os halláis imposibilitados para realizar en el mundo los
generosos impulsos del pensamiento y las leyes del corazón, acordaos de
Gabriel Araceli, que nació sin nada y lo tuvo todo.

\flushright{Febrero-Marzo de 1875.}

~

\bigskip
\bigskip
\begin{center}
\textsc{Fin de la Batalla de los Arapiles}
\end{center}

\end{document}
