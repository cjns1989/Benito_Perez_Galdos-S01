\PassOptionsToPackage{unicode=true}{hyperref} % options for packages loaded elsewhere
\PassOptionsToPackage{hyphens}{url}
%
\documentclass[oneside,12pt,spanish,]{extbook} % cjns1989 - 27112019 - added the oneside option: so that the text jumps left & right when reading on a tablet/ereader
\usepackage{lmodern}
\usepackage{amssymb,amsmath}
\usepackage{ifxetex,ifluatex}
\usepackage{fixltx2e} % provides \textsubscript
\ifnum 0\ifxetex 1\fi\ifluatex 1\fi=0 % if pdftex
  \usepackage[T1]{fontenc}
  \usepackage[utf8]{inputenc}
  \usepackage{textcomp} % provides euro and other symbols
\else % if luatex or xelatex
  \usepackage{unicode-math}
  \defaultfontfeatures{Ligatures=TeX,Scale=MatchLowercase}
%   \setmainfont[]{EBGaramond-Regular}
    \setmainfont[Numbers={OldStyle,Proportional}]{EBGaramond-Regular}      % cjns1989 - 20191129 - old style numbers 
\fi
% use upquote if available, for straight quotes in verbatim environments
\IfFileExists{upquote.sty}{\usepackage{upquote}}{}
% use microtype if available
\IfFileExists{microtype.sty}{%
\usepackage[]{microtype}
\UseMicrotypeSet[protrusion]{basicmath} % disable protrusion for tt fonts
}{}
\usepackage{hyperref}
\hypersetup{
            pdftitle={ZARAGOZA},
            pdfauthor={Benito Pérez Galdós},
            pdfborder={0 0 0},
            breaklinks=true}
\urlstyle{same}  % don't use monospace font for urls
\usepackage[papersize={4.80 in, 6.40  in},left=.5 in,right=.5 in]{geometry}
\setlength{\emergencystretch}{3em}  % prevent overfull lines
\providecommand{\tightlist}{%
  \setlength{\itemsep}{0pt}\setlength{\parskip}{0pt}}
\setcounter{secnumdepth}{0}

% set default figure placement to htbp
\makeatletter
\def\fps@figure{htbp}
\makeatother

\usepackage{ragged2e}
\usepackage{epigraph}
\renewcommand{\textflush}{flushepinormal}

\usepackage{indentfirst}

\usepackage{fancyhdr}
\pagestyle{fancy}
\fancyhf{}
\fancyhead[R]{\thepage}
\renewcommand{\headrulewidth}{0pt}
\usepackage{quoting}
\usepackage{ragged2e}

\newlength\mylen
\settowidth\mylen{...................}

\usepackage{stackengine}
\usepackage{graphicx}
\def\asterism{\par\vspace{1em}{\centering\scalebox{.9}{%
  \stackon[-0.6pt]{\bfseries*~*}{\bfseries*}}\par}\vspace{.8em}\par}

 \usepackage{titlesec}
 \titleformat{\chapter}[display]
  {\normalfont\bfseries\filcenter}{}{0pt}{\Large}
 \titleformat{\section}[display]
  {\normalfont\bfseries\filcenter}{}{0pt}{\Large}
 \titleformat{\subsection}[display]
  {\normalfont\bfseries\filcenter}{}{0pt}{\Large}

\setcounter{secnumdepth}{1}
\ifnum 0\ifxetex 1\fi\ifluatex 1\fi=0 % if pdftex
  \usepackage[shorthands=off,main=spanish]{babel}
\else
  % load polyglossia as late as possible as it *could* call bidi if RTL lang (e.g. Hebrew or Arabic)
%   \usepackage{polyglossia}
%   \setmainlanguage[]{spanish}
%   \usepackage[french]{babel} % cjns1989 - 1.43 version of polyglossia on this system does not allow disabling the autospacing feature
\fi

\title{ZARAGOZA}
\author{Benito Pérez Galdós}
\date{}

\begin{document}
\maketitle

\hypertarget{i}{%
\chapter{I}\label{i}}

Me parece que fue al anochecer del 18 cuando avistamos a Zaragoza.
Entrando por la puerta de Sancho, oímos que daba las diez el reloj de la
Torre Nueva. Nuestro estado era excesivamente lastimoso en lo tocante a
vestido y alimento, porque las largas jornadas que habíamos hecho desde
Lerma por Salas de los Infantes, Cervera, Agreda, Tarazona y Borja,
escalando montes, vadeando ríos, franqueando atajos y vericuetos hasta
llegar al camino real de Gallur y Alagón, nos dejaron molidos,
extenuados y enfermos de fatiga. Con todo, la alegría de vernos libres
endulzaba todas nuestras penas.

Éramos cuatro los que habíamos logrado escapar entre Lerma y Cogollos,
divorciando nuestras inocentes manos de la cuerda que enlazaba a tantos
patriotas. El día de la evasión reuníamos entre los cuatro un capital de
once reales; pero después de tres días de marcha, y cuando entramos en
la metrópoli aragonesa, hízose un balance y arqueo de la caja social, y
nuestras cuentas sólo arrojaron un activo de treinta y un cuartos.
Compramos pan junto a la Escuela Pía, y nos lo distribuimos.

D. Roque, que era uno de los expedicionarios, tenía buenas relaciones en
Zaragoza; pero aquella no era hora de presentarnos a nadie. Aplazamos
para el día siguiente el buscar amigos, y como no podíamos alojarnos en
una posada, discurrimos por la ciudad buscando un abrigo donde pasar la
noche. Los portales del Mercado no nos parecían tener las comodidades y
el sosiego que nuestros cansados cuerpos exigían. Visitamos la torre
inclinada, y aunque alguno de mis compañeros propuso que nos
guareciéramos al amor de su zócalo, yo opiné que allí estábamos como en
campo raso. Sirvionos, sin embargo, de descanso aquel lugar, y también
de refectorio para nuestra cena de pan seco, la cual despachamos
alegremente, mirando de rato en rato la mole amenazadora, cuya
desviación la asemeja a un gigante que se inclina para mirar quién anda
a sus pies. A la claridad de la luna, aquel centinela de ladrillo
proyecta sobre el cielo su enjuta figura, que no puede tenerse derecha.
Corren las nubes por encima de su aguja, y el espectador que mira desde
abajo, se estremece de espanto, creyendo que las nubes están quietas y
que la torre se le viene encima. Esta absurda fábrica bajo cuyos pies ha
cedido el suelo cansado de soportarla, parece que se está siempre
cayendo, y nunca acaba de caer.

Recorrimos luego el Coso desde la casa de los Gigantes hasta el
Seminario; nos metimos por la calle Quemada y la del Rincón, ambas
llenas de ruinas, hasta la plazuela de San Miguel, y de allí, pasando de
callejón en callejón, y atravesando al azar angostas e irregulares vías,
nos encontramos junto a las ruinas del monasterio de Santa Engracia,
volado por los franceses al levantar el primer sitio. Los cuatro
lanzamos una misma exclamación, que indicaba la conformidad de nuestros
pensamientos. Habíamos encontrado un asilo, y excelente alcoba donde
pasar la noche.

La pared de la fachada continuaba en pie con su pórtico de mármol,
poblado de innumerables figuras de santos, que permanecían enteros y
tranquilos como si ignoraran la catástrofe. En el interior vimos arcos
incompletos, machones colosales, irguiéndose aún entre los escombros, y
que al destacarse negros y deformes sobre la claridad del espacio,
semejaban criaturas absurdas, engendradas por una imaginación en
delirio; vimos recortaduras, ángulos, huecos, laberintos, cavernas y
otras mil obras de esa arquitectura del acaso trazada por el desplome.
Había hasta pequeñas estancias abiertas entre los pedazos de la pared
con un arte semejante al de las grutas en la naturaleza. Los trozos de
retablo podridos a causa de la humedad, asomaban entre los restos de la
bóveda, donde aún subsistía la roñosa polea que sirvió para suspender
las lámparas, y precoces yerbas nacían entre las grietas de la madera y
de la piedra. Entre tanto destrozo había objetos completamente intactos,
como algunos tubos del órgano y la reja de un confesonario. El techo se
confundía con el suelo, y la torre mezclaba sus despojos con los del
sepulcro. Al ver semejante aglomeración de escombros, tal multitud de
trozos caídos sin perder completamente su antigua forma, las masas de
ladrillo enyesado que se desmoronaban como objetos de azúcar, creeríase
que los despojos del edificio no habían encontrado posición definitiva.
La informe osamenta parecía palpitar aún con el estremecimiento de la
voladura.

D. Roque nos dijo que bajo aquella iglesia había otra, donde se
veneraban los huesos de los Santos Mártires de Zaragoza; pero la entrada
del subterráneo estaba obstruida. Profundo silencio reinaba allí; mas
internándonos, oímos voces humanas que salían de aquellos misteriosos
antros. La primera impresión que al escucharlas nos produjo fue como si
hubieran aparecido las sombras de los dos famosos cronistas, de los
mártires cristianos, y de los patriotas sepultados bajo aquel polvo, y
nos increparan por haber turbado su sueño. En el mismo instante, al
resplandor de una llama que iluminó parte de la escena, distinguimos un
grupo de personas que se abrigaban unas contra otras en el hueco formado
entre dos machones derruidos. Eran mendigos de Zaragoza que se habían
arreglado un palacio en aquel sitio, resguardándose de la lluvia con
vigas y esteras. También nosotros nos pudimos acomodar por otro lado, y
tapándonos con manta y media, llamamos al sueño. D. Roque me decía así:

---Yo conozco a D. José de Montoria, uno de los labradores más ricos de
Zaragoza. Ambos somos hijos de Mequinenza, fuimos juntos a la escuela y
juntos jugábamos al truco en el altillo del Corregidor. Aunque hace
treinta años que no le veo, creo que nos recibirá bien. Como buen
aragonés, todo él es corazón. Le veremos, muchachos; veremos a D. José
de Montoria\ldots{} Yo también tengo sangre de Montoria por la línea
materna. Nos presentaremos a él; le diremos\ldots{}

Durmiose D. Roque y también me dormí.

\hypertarget{ii}{%
\chapter{II}\label{ii}}

El lecho en que yacíamos no convidaba por sus blanduras a dormir
perezosamente la mañana, antes bien, colchón de guijarros hace buenos
madrugadores. Despertamos, pues, con el día, y como no teníamos que
entretenernos en melindres de tocador, bien pronto estuvimos en
disposición de salir a hacer nuestras visitas. A los cuatro nos ocurrió
simultáneamente la idea de que sería muy bueno desayunarnos; pero al
punto convinimos con igual unanimidad, en que no era posible por carecer
de los fondos indispensables para tan alta empresa.

---No os acobardéis, muchachos---dijo D. Roque,---que al punto os he de
llevar a todos a casa de mi amigo, el cual nos amparará.

Cuando esto decía, vimos salir a dos hombres y una mujer de los que
fueron durante la noche nuestros compañeros de posada, y parecían gente
habituada a dormir en aquel lugar. Uno de ellos, era un infeliz lisiado,
un hombre que acababa en las rodillas y se ponía en movimiento con ayuda
de muletas o bien andando a cuatro remos, viejo, de rostro jovial y muy
tostado por el sol. Como nos saludara afablemente al pasar, dándonos los
buenos días, D. Roque le preguntó hacia qué parte de la ciudad caía la
casa de D. José de Montoria, oyendo lo cual repuso el cojo:

---¿D. José de Montoria? Le conozco más que a las niñas de mis ojos.
Hace veinte años vivía en la calle de la Albardería; después se mudó a
la de la Parra, después\ldots{} Pero ustés son forasteros por lo que
veo.

---Sí, buen amigo, forasteros somos, y venimos a afiliarnos en el
ejército de esta valiente ciudad.

---¿De modo que no estaban ustés aquí el 4 de Agosto?

---No, amigo---le respondí,---no hemos presenciado ese gran hecho de
armas.

---¿Ni tampoco vieron la batalla de las Eras?---preguntó el mendigo
sentándose frente a nosotros.

---Tampoco hemos tenido esa felicidad.

---Pues allí estuvo D. José de Montoria; fue de los que llevaron
arrastrando el cañón hasta enfilarlo\ldots{} pues. Veo que ustés no han
visto nada. ¿De qué parte del mundo vienen ustés?

---De Madrid---dijo D. Roque.---¿Con que Vd. nos podrá decir dónde vive
mi gran amigo D. José?

---Pues no he de poder, hombre, pues no he de poder---repuso el cojo,
sacando un mendrugo para desayunarse.---De la calle de la Parra se mudó
a la de Enmedio. Ya saben ustés que todas las casas volaron\ldots{}
pues. Allí estaba Esteban López, soldado de la décima compañía del
primer tercio de voluntarios de Aragón, y él solo con cuarenta hombres
hizo retirar a los franceses.

---Eso sí que es cosa admirable---dijo D. Roque.

---Pero si no han visto ustés lo del 4 de Agosto, no han visto
nada---continuó el mendigo.---Yo vi también lo del 4 de Junio, porque me
fui arrastrando por la calle de la Paja, y vi a la \emph{artillera}
cuando dio fuego al cañón de 24.

---Ya, ya tenemos noticia del heroísmo de esa insigne mujer---manifestó
D. Roque.---Pero si Vd. nos quisiera decir\ldots{}

---Pues sí; D. José de Montoria es muy amigo del comerciante D. Andrés
Guspide, que el 4 de Agosto estuvo haciendo fuego desde la visera del
callejón de la Torre del Pino, y por allí llovían granadas, balas,
metralla, y mi D. Andrés fijo como un poste. Más de cien muertos había a
su lado, y él solo mató cincuenta franceses.

---Gran hombre es ese; ¿y es amigo de mi amigo?

---Sí señor---respondió el cojo.---Y ambos son los mejores caballeros de
toda Zaragoza, y me dan limosna todos los sábados. Porque han de saber
ustés que yo soy Pepe Pallejas, y me llaman por mal nombre \emph{Sursum
Corda}, pues como fui hace veinte y nueve años sacristán de Jesús y
cantaba\ldots{} pero esto no viene al caso, y sigo diciendo que yo soy
\emph{Sursum Corda} y \emph{pue} que hayan ustés oído hablar de mí en
Madrid.

---Sí---dijo D. Roque cediendo a un impulso de amabilidad;---me parece
que allá he oído nombrar al señor de \emph{Sursum Corda}. ¿No es verdad,
muchachos?

---Pues ello\ldots---prosiguió el mendigo.---Y sepan también que antes
del sitio yo pedía limosna en la puerta de este monasterio de Santa
Engracia, volado por los bandidos el 13 de Agosto. Ahora pido en la
puerta de Jerusalem, donde me podrán hallar siempre que gusten\ldots{}
Pues como iba diciendo, el día 4 de Agosto estaba yo aquí, y vi salir de
la iglesia a Francisco Quílez, sargento primero de la primera compañía
del primer batallón de fusileros, el cual ya saben ustés que fue el que
con treinta y cinco hombres echó a los bandidos del Convento de la
Encarnación\ldots{} Veo que se asombran ustés\ldots{} ya. Pues en la
huerta de Santa Engracia, que está aquí detrás, murió el subteniente D.
Miguel Gila. Lo menos había doscientos cadáveres en la tal huerta, y
allí perniquebraron a D. Felipe San Clemente y Romeu, comerciante de
Zaragoza. Verdad es que si no hubiera estado presente D. Miguel
Salamero\ldots{} ¿ustés no saben nada de esto?

---No, amigo y señor mío---dijo D. Roque,---nada de esto sabemos, y
aunque tenemos el mayor gusto en que Vd. nos cuente tantas maravillas,
lo que es ahora, más nos importa saber dónde encontraremos al D. José mi
antiguo amigo, porque padecemos los cuatro de un mal que llaman hambre y
que no se cura oyendo contar sublimidades.

---Ahora mismo les llevaré a donde quieren ir---repuso \emph{Sursum
Corda}, después de ofrecernos parte de sus mendrugos.---Pero antes les
quiero decir una cosa, y es que si D. Mariano Cereso no hubiera
defendido la Aljafería como la defendió, nada se habría hecho en el
Portillo. ¡Y que es hombre de mantequillas en gracia de Dios el tal D.
Mariano Cereso! En la del 4 de Agosto andaba por las calles con su
espada y rodela antigua y daba miedo verle. Esto de Santa Engracia
parecía un horno, señores. Las bombas y las granadas llovían; pero los
patriotas no les hacían más caso que si fueran gotas de agua. Una buena
parte del Convento se desplomó; las casas temblaban y todo esto que
estamos viendo parecía un barrio de naipes, según la prontitud con que
se incendiaba y se desmoronaba. Fuego en las ventanas, fuego arriba,
fuego abajo: los franceses caían como moscas, señores, y a los
zaragozanos lo mismo les daba morir que nada. D. Antonio Quadros embocó
por allí, y cuando miró a las baterías francesas, se las quería comer.
Los bandidos tenían sesenta cañones echando fuego sobre estas paredes.
¿Ustés no lo vieron? Pues yo sí, y los pedazos del ladrillo de las
tapias y la tierra de los parapetos salpicaban como miajas de un bollo.
Pero los muertos servían de parapeto, y muertos arriba, muertos abajo,
aquello era una montaña. D. Antonio Quadros echaba llamas por los ojos.
Los muchachos hacían fuego sin parar; su alma era toda balas, ¿ustés no
lo vieron? Pues yo sí, y las baterías francesas se quedaban limpias de
artilleros. Cuando vio que un cañón enemigo había quedado sin gente, el
comandante gritó: «¡Una charretera al que clave aquel cañón!» y Pepillo
Ruiz echa a andar como quien se pasea por un jardín entre mariposas y
flores de Mayo; sólo que aquí las mariposas eran balas, y las flores
bombas. Pepillo Ruiz clava el cañón y se vuelve riendo. Pero velay que
otro pedazo de Convento se viene al suelo. El que fue aplastado,
aplastado quedó. D. Antonio Quadros dijo que aquello no importaba nada,
y viendo que la artillería de los bandidos había abierto un gran boquete
en la tapia, fue a taparlo él mismo con una saca de lana. Entonces una
bala le dio en la cabeza. Retiráronle aquí; dijo que tampoco aquello
importaba nada, y expiró.

---¡Oh!---dijo D. Roque con impaciencia.---Estamos encantados, señor
\emph{Sursum Corda}, y el más puro patriotismo nos inflama al oírle
contar a Vd. tan grandes hazañas; pero si Vd. nos quisiera decir
dónde\ldots{}

---Hombre de Dios---contestó el mendigo,---¿pues no se lo he de decir?
Si lo que más sé y lo que más visto tengo en mi vida es la casa de D.
José de Montoria. Como que está cerca de San Pablo. ¡Oh! ¿Ustés no
vieron lo del hospital? Pues yo sí: allí caían las bombas como el
granizo. Los enfermos viendo que los techos se les venían encima, se
arrojaban por las ventanas a la calle. Otros se iban arrastrando y
rodaban por las escaleras. Ardían los tabiques, oíanse lamentos, y los
locos mugían en sus jaulas como fieras rabiosas. Otros se escaparon y
andaban por los claustros riendo, bailando y haciendo mil gestos
graciosos que daban espanto. Algunos salieron a la calle como en día de
Carnaval, y uno se subió a la cruz del Coso, donde se puso a sermonear,
diciendo que él era el Ebro y que anegando la ciudad iba a sofocar el
fuego. Las mujeres corrían a socorrer a los enfermos, y todos eran
llevados al Pilar y a la Seo. No se podía andar por las calles. La Torre
Nueva hacía señales para que se supiera cuándo venía una bomba; pero el
griterío de la gente no dejaba oír las campanas. Los franceses avanzan
por esta calle de Santa Engracia; se apoderan del hospital y del
Convento de San Francisco; empieza la guerra en el Coso y en las calles
de por allí. Don Santiago Sas, D. Mariano Cereso, D. Lorenzo Calvo, D.
Marcos Simonó, Renovales, el albéitar Martín Albantos, Vicente Codé, D.
Vicente Marraco y otros atacan a los franceses a pecho descubierto; y
detrás de una barricada hecha por ella misma, les espera llena de furor
y fusil en mano, la señora condesa de Bureta.

---¿Cómo, una mujer, una condesa---preguntó con entusiasmo D.
Roque,---levantaba barricadas y apuntaba fusiles?

---¿Ustés no lo sabían?---dijo \emph{Sursum.---} ¿Pues en dónde viven
ustés? La señora doña María Consolación Azlor y Villavicencio, que vive
allá por el Ecce-Homo, andaba por las calles, y a los desanimados les
decía mil lindezas, y luego haciendo cerrar la entrada de la calle, se
puso al frente de una partida de paisanos, gritando: «¡Aquí moriremos
todos, antes que dejarles pasar!»

---¡Oh, cuánta sublimidad!---exclamó D. Roque bostezando de hambre.---¡Y
cuánto me agradaría oír contar hazañas de esa naturaleza con el estómago
lleno! Conque decía Vd., buen amigo, que la casa de D. José cae
hacia\ldots{}

---Hacia allá---repuso el cojo.---Ya saben ustés que los franceses se
enredaron y se atascaron en el arco de Cineja. ¡Virgen mía del Pilar!
Aquello era matar franceses, lo demás es aire. En la calle de la Parra,
en la plazuela de Estrevedes, en la calle de los Urreas, en la de Santa
Fe y en la del Azoque los paisanos despedazaban a los franceses. Todavía
me zumban en las orejas el cañoneo y el gritar de aquel día. Los
gabachos quemaban las casas que no podían defender y los zaragozanos
hacían lo mismo. Fuego por todos lados\ldots{} Hombres, mujeres,
chiquillos\ldots{} Basta tener dos manos para trabajar contra el
enemigo. ¿Ustés no lo vieron? Pues no han visto nada. Pues como les iba
diciendo, aquel día salió Palafox de Zaragoza para\ldots{}

---Basta, amigo mío---dijo D. Roque perdiendo la paciencia;---estamos
encantados con su conversación; pero si no nos guía al instante a casa
de mi paisano o nos indica cómo podemos encontrar su casa, nos iremos
solos.

---Al instante, señores, no apurarse---repuso \emph{Sursum Corda}
echando a andar delante de nosotros con toda la agilidad de sus
muletas.---Vamos allá, vamos con mil amores. ¿Ven ustés esta casa? Pues
aquí vive Antonio Laste, sargento primero de la compañía del cuarto
tercio, y ya sabrán que salvó de la tesorería los diez y seis mil
cuatrocientos pesos, y quitó a los franceses la cera que habían robado.

---Adelante, adelante, amigo---dije, viendo que el incansable hablador
se detenía para contar de un modo minucioso las hazañas de Antonio
Laste.

---Ya pronto llegaremos---repuso \emph{Sursum.---} Por aquí iba yo en la
mañana del 1.º de Julio, cuando encontré a Hilario Lafuente, cabo
primero de la compañía de escopeteros del presbítero Sas, y me dijo:
«Hoy van a atacar el Portillo.» Entonces yo me fui a ver lo que había
y\ldots{}

---Ya estamos enterados de todo---le indicó don Roque.---Vamos aprisa, y
después hablaremos.

---Esta casa que ven ustés toda quemada y hecha escombros---continuó el
cojo volviendo una esquina,---es la que ardió el día 4, cuando D.
Francisco Ipas, subteniente de la segunda compañía de escopeteros de la
parroquia de San Pablo, se puso aquí con un cañón, y luego\ldots{}

---Ya sabemos lo demás, buen hombre---dijo don Roque.---Adelante, y más
que de prisa.

---Pero mucho mejor fue lo que hizo Codé, labrador de la parroquia de la
Magdalena, con el cañón de la calle de la Parra---continuó el mendigo
deteniéndose otra vez.---Pues al ir a disparar, los franceses se echan
encima; huyen todos; pero Codé se mete debajo del cañón; pasan los
franceses sin verle, y después, ayudado de una vieja que le dio una
cuerda, arrastra la pieza hasta la bocacalle. Vengan ustés y les
enseñaré.

---No, no queremos ver nada: adelante, adelante en nuestro camino.

Tanto le azuzamos, y con tanta obstinación cerramos nuestros oídos a sus
historias, que al fin, aunque muy despacio, nos llevó por el Coso y el
Mercado a la calle de la Hilarza, donde la persona a quien queríamos ver
tenía su casa.

\hypertarget{iii}{%
\chapter{III}\label{iii}}

Pero ¡ay!, D. José de Montoria no estaba en ella y nos fue preciso
buscarle en los alrededores de la ciudad. Dos de mis compañeros,
aburridos de tantas idas y venidas, se separaron de nosotros, aspirando
a buscar con su propia iniciativa un acomodo militar o civil. Nos
quedamos solos D. Roque y un servidor, y así emprendimos con más
desembarazo el viaje a la torre de nuestro amigo (llaman en Zaragoza
\emph{torres} a las casas de campo) situada a poniente, lindando con el
camino de Muela y a poca distancia de la Bernardona. Un paseo tan largo
a pie y en ayunas no era lo más satisfactorio para nuestros fatigados
cuerpos; pero la necesidad nos obligaba a tan inoportuno ejercicio y por
bien servidos nos dimos encontrando al deseado zaragozano, y siendo
objeto de su cordial hospitalidad.

Ocupábase Montoria cuando llegamos en talar los frondosos olivos de su
finca, porque así lo exigía el plan de obras de defensa establecido por
los jefes facultativos ante la inminencia de un segundo sitio. Y no era
sólo nuestro amigo el que por sus propias manos destruía sin piedad la
hacienda heredada: todos los propietarios de los alrededores se ocupaban
en la misma faena y presidían los devastadores trabajos con tanta
tranquilidad como si fuera un riego, un replanteo o una vendimia.
Montoria nos dijo:

---En el primer sitio talé la heredad que tengo al lado allá de Huerva;
pero este segundo asedio que se nos prepara dicen que será más terrible
que aquel, a juzgar por el gran aparato de tropas que traen los
franceses.

Contámosle la capitulación de Madrid, lo cual pareció causarle mucha
pesadumbre, y como elogiáramos con exclamaciones hiperbólicas las
ocurrencias de Zaragoza desde el 15 de Junio al 14 de Agosto, encogiose
de hombros y contestó:

---Se ha hecho todo lo que se ha podido.

Acto continuo D. Roque pasó a hacer elogios de mi personalidad, militar
y civilmente considerada, y de tal modo se le fue la mano en este
capítulo, que me hizo sonrojar, mayormente considerando que algunas de
sus afirmaciones eran estupendas mentiras. Díjole primero que yo
pertenecía a una de las más alcurniadas familias de la \emph{baja
Andalucía en tierra de Doñana}, y que había asistido al glorioso combate
de Trafalgar en clase de guardia marina. Le dijo también que la junta me
había concedido un destino en el Perú y que durante el sitio de Madrid
había hecho prodigios de valor en la Puerta de los Pozos, siendo tanto
mi ardor, que los franceses, después de la rendición, creyeron
conveniente deshacerse de tan terrible enemigo, enviándome con otros
patriotas a Francia. Añadió que mis ingeniosas invenciones habían
proporcionado la fuga a los cuatro compañeros refugiados en Zaragoza, y
puso fin a su panegírico asegurando que por mis cualidades personales
era yo acreedor a las mayores distinciones.

Montoria en tanto me examinaba de pies a cabeza, y si llamaba su
atención mi mal traer y las infinitas roturas de mi vestido, también
debió advertir que este era de los que usan las personas de calidad,
revelando su finura, buen corte y aristocrático origen en medio de la
multiplicidad abrumadora de sus desperfectos. Luego que me examinó, me
dijo:

---¡Porra! No le podré afiliar a Vd. en la tercera escuadra de la
segunda compañía de escopeteros de D. Santiago Sas, de cuya compañía soy
capitán; pero entrará en el cuerpo en que está mi hijo; y si no quiere
Vd., largo de Zaragoza, que aquí no admitimos gente haragana. Y a Vd.,
D. Roque amigo, puesto que no está para coger el fusil ¡porra!, le
haremos practicante de los hospitales del ejército.

Luego que esto oyó D. Roque, expuso por medio de circunlocuciones
retóricas y de graciosas elipsis la gran necesidad en que nos
encontrábamos y lo bien que recibiríamos sendas magras y un par de panes
cada uno. Entonces vimos que frunció el ceño el gran Montoria,
mirándonos de un modo severo, lo cual nos hizo temblar, y parecionos que
íbamos a ser despedidos por la osadía de pedir de comer. Balbucimos
tímidas excusas y entonces nuestro protector con rostro encendido, nos
habló así:

---¿Con que tienen hambre? ¡Porra, váyanse al demonio con cien mil pares
de porras! ¿Y por qué no lo habían dicho? ¿Con que yo soy hombre capaz
de consentir que los amigos tengan hambre, porra? Sepan que no me faltan
diez docenas de jamones colgados en el techo de la despensa, ni veinte
cubas de lo de Rioja, sí señor; y tener hambre y no decírmelo en mi cara
sin retruécanos, es ofender a un hombre como yo. Ea, muchachos, entrad
adentro y mandar que frían obra de cuatro libras de lomo, y que
estrellen dos docenas de huevos, y que maten seis gallinas, y saquen de
la cueva siete jarros de vino, que yo también quiero almorzar. Vengan
todos los vecinos, los trabajadores y mis hijos si están por ahí. Y
ustedes, señores, prepárense a hacer penitencia conmigo. ¡Nada de
melindres, porra! Comerán de lo que hay sin dengues ni boberías. Aquí no
se usan cumplidos. Vd., Sr.~D. Roque, y Vd., Sr.~de Araceli, están en su
casa hoy y mañana y siempre, ¡porra! José de Montoria es muy amigo de
los amigos. Todo lo que tiene es de los amigos.

La ruda generosidad de aquel insigne varón nos tenía anonadados. Como
recibiera muy mal los cumplimientos, resolvimos dejar a un lado el
formulario artificioso de la corte, y vierais allí cómo la llaneza más
primitiva reinó durante el almuerzo.

---Qué, ¿no come Vd. más?---me dijo D. José.---Me parece que es Vd. un
boquirrubio que se anda con enjuagues y finuras. A mí no me gusta eso,
caballerito; me parece que me voy a enfadar y tendré que pegar palos
para hacerles comer. Ea, despache Vd. este vaso de vino. ¿Acaso es mejor
el de la corte? Ni a cien leguas. Con que, porra, beba Vd., porra, o nos
veremos las caras.

Esto fue causa de que comiera y bebiera mucho más de lo que cabía en mi
cuerpo; pero había que corresponder a la generosa franqueza de Montoria,
y no era cosa de que por una indigestión más o menos se perdiera tan
buena amistad.

Después del almuerzo, siguieron los trabajos de tala, y el rico labrador
los dirigía como si fuera una fiesta.

---Veremos---decía,---si esta vez se atreven a atacar el castillo. ¿No
ha visto Vd. las obras que hemos hecho? Menudo trabajo van a tener. Yo
he dado doscientas sacas de lana, una friolera, y daré hasta el último
mendrugo.

Cuando nos retirábamos a la ciudad, llevonos Montoria a examinar las
obras defensivas que a la sazón se estaban construyendo en aquella parte
occidental. Había en la puerta del Portillo una gran batería
semicircular que enlazaba las tapias del Convento de los Fecetas con las
del de Agustinos descalzos. Desde este edificio al de Trinitarios corría
otra muralla recta, aspillerada en toda su extensión y con un buen
reducto en el centro, todo resguardado por profundo foso que se abría
hacia el famoso campo de las Eras o del Sepulcro, teatro de la heroica
jornada del 15 de Junio. Más al Norte y hacia la puerta de Sancho, que
da paso al pretil del Ebro, seguían las fortificaciones, terminando en
otro baluarte. Todas estas obras, como hechas a prisa, aunque con
inteligencia, no se distinguían por su solidez. Cualquier general
enemigo, ignorante de los acontecimientos del primer sitio y de la
inmensa estatura moral de los zaragozanos al ponerse detrás de aquellos
montones de tierra, se habría reído de fortificaciones tan despreciables
para un buen material de sitio; pero Dios ha dispuesto que alguien
escape de vez en cuando a las leyes físicas establecidas por la guerra.
Zaragoza, comparada con Amberes, Dantzig, Metz, Sebastopol, Cartagena,
Gibraltar y otras célebres plazas fuertes tomadas o no, era entonces una
fortaleza de cartón. Y sin embargo\ldots{}

\hypertarget{iv}{%
\chapter{IV}\label{iv}}

En su casa, Montoria se enfadó otra vez con don Roque y conmigo, porque
no quisimos admitir el dinero que nos ofrecía para nuestros primeros
gastos en la ciudad, y aquí se repitieron los puñetazos en la mesa y la
lluvia de\emph{porras} y otras palabras que no cito; pero al fin
llegamos a una transacción honrosa para ambas partes. Y ahora caigo en
que me ocupo demasiado de hombre tan singular sin haber anticipado
algunas observaciones acerca de su persona. Era D. José un hombre de
sesenta años, fuerte, colorado, rebosando salud, bienestar, contento de
sí mismo, conformidad con la suerte y conciencia tranquila. Lo que le
sobraba en patriarcales virtudes y en costumbres ejemplares y pacíficas
(si es que esto puede estar de sobra en algún caso), le faltaba en
educación, es decir, en aquella educación atildada y distinguida que
entonces empezaban a recibir algunos hijos de familias ricas. D. José no
conocía los artificios de la etiqueta, y por carácter y por costumbres
era refractario a la mentira discreta y a los amables embustes que
constituyen la base fundamental de la cortesía. Como él llevaba siempre
el corazón en la mano, quería que asimismo lo llevasen los demás, y su
bondad salvaje no toleraba las coqueterías frecuentemente falaces de la
conversación fina. En los momentos de enojo era impetuoso y dejábase
arrastrar a muy violentos extremos, de que por lo general se arrepentía
más tarde.

En él no había disimulo, y tenía las grandes virtudes cristianas, en
crudo y sin pulimento, como un macizo canto del más hermoso mármol,
donde el cincel no ha trazado una raya siquiera. Era preciso saberlo
entender, cediendo a sus excentricidades, si bien en rigor no debe
llamarse excéntrico el que tanto se parecía a la generalidad de sus
paisanos. No ocultar jamás lo que sentía era su norte, y si bien esto le
ocasionaba algunas molestias en el curso de la vida ordinaria y en
asuntos de poca monta, era un tesoro inapreciable siempre que se tratase
con él un negocio grave, porque puesta a la vista toda su alma, no había
que temer malicia alguna. Perdonaba las ofensas, agradecía los
beneficios y daba gran parte de sus cuantiosos bienes a los
menesterosos.

Vestía con aseo, comía abundantemente, ayunando con todo escrúpulo la
Cuaresma entera, y amaba a la Virgen del Pilar con fanático amor de
familia. Su lenguaje no era, según se ha visto, un modelo de
comedimiento, y él mismo confesaba como el mayor de sus defectos lo de
soltar a todas horas \emph{porra} y más \emph{porra}, sin que viniese al
caso; pero más de una vez le oí decir, que conocedor de la falta, no la
podía remediar, porque aquello de las \emph{porras} le salía de la boca
sin que él mismo se diera cuenta de ello.

Tenía mujer y tres hijos. Era aquélla doña Leocadia Sarriera, navarra de
origen. De los vástagos, el mayor y la hembra estaban casados y habían
dado a los viejos algunos nietos. El más pequeño de los hijos llamábase
Agustín y era destinado a la Iglesia, como su tío del mismo nombre,
arcediano de la Seo. A todos les conocí en el mismo día, y eran la mejor
gente del mundo. Fui tratado con tanto miramiento, que me tenía absorto
su generosidad, y si me conocieran desde el nacer no habrían sido más
rumbosos. Sus obsequios, espontáneamente sugeridos por corazones
generosos, me llegaban al alma, y como yo siempre he sido fácil en
dejarme querer, les correspondí desde el principio con muy sincero
afecto.

---Sr.~D. Roque---dije aquella noche a mi compañero cuando nos
acostábamos en el cuarto que nos destinaron,---yo jamás he visto gente
como esta. ¿Son así todos los aragoneses?

---Hay de todo---me respondió,---pero hombres de la madera de D. José de
Montoria, y familias como esta familia abundan mucho en esta tierra de
Aragón.

Al siguiente día nos ocupamos en mi alistamiento. La decisión de aquella
gente me entusiasmaba de tal modo, que nada me parecía tan honroso como
seguir tras ella, aunque fuera a distancia, husmeando su rastro de
gloria. Ninguno de Vds. ignora que en aquellos días Zaragoza y los
zaragozanos habían adquirido un renombre fabuloso; que sus hazañas
enardecían las imaginaciones y que todo lo referente al sitio famoso de
la inmortal ciudad, tomaba en boca de los narradores las proporciones y
el colorido de una leyenda de los tiempos heroicos. Con la distancia,
las acciones de los zaragozanos adquirían dimensiones mayores aún, y en
Inglaterra y en Alemania, donde les consideraban como los numantinos de
los tiempos modernos, aquellos paisanos medio desnudos, con alpargatas
en los pies y un pañizuelo enrollado en la cabeza, eran figuras de
coturno. \emph{Capitulad y os vestiremos}---decían los franceses en el
primer sitio, admirados de la constancia de unos pobres aldeanos
vestidos de harapos.---\emph{No sabemos
rendirnos}---contestaban,---\emph{y nuestras carnes sólo se cubren de
gloria}.

Esta y otras frases habían dado la vuelta al mundo.

Pero volvamos a lo de mi alistamiento. Era un obstáculo para este el
manifiesto de Palafox de 13 de Diciembre, en que ordenaba la expulsión
de forasteros mandándoles salir en el término de veinticuatro horas,
acuerdo tomado en razón de la mucha gente que iba a alborotar sembrando
discordias y desavenencias; pero precisamente en los días de mi llegada
se publicó otra proclama llamando a los soldados dispersos del ejército
del Centro, desbaratado en Tudela, y en esto hallé una buena coyuntura
para afiliarme, pues aunque no pertenecí a dicho ejército, había
concurrido a la defensa de Madrid, y a la batalla de Bailén, razones que
con el apoyo de mi protector Montoria, me valieron el ingreso en las
huestes zaragozanas. Diéronme un puesto en el batallón de voluntarios de
las Peñas de San Pedro, bastante mermado en el primer sitio, y recibí un
uniforme y un fusil. No formé, como había dicho mi protector, en las
filas de mosén Santiago Sas, fogoso clérigo, puesto al frente de un
batallón de escopeteros, porque esta valiente partida se componía
exclusivamente de vecinos de la parroquia de San Pablo. Tampoco querían
gente moza en su batallón, por cuya causa ni el ni mismo hijo de D. José
de Montoria, Agustín Montoria, pudo servir a las ordenes de Sas, y se
afilió como yo en el batallón de las Peñas de San Pedro. La suerte me
deparaba un buen compañero y un excelente amigo.

Desde el día de mi llegada, oí hablar de la aproximación del ejército
francés; pero esto no fue un hecho incontrovertible hasta el 20. Por la
tarde una división llegó a Zuera, en la orilla izquierda, para amenazar
el arrabal; otra mandada por Suchet acampó en la derecha sobre San
Lamberto. Moncey, que era el general en jefe, situose con tres
divisiones hacia el Canal y en las inmediaciones de la Huerva. Cuarenta
mil hombres nos cercaban.

Sabido es que impacientes por vencernos, los franceses comenzaron sus
operaciones el 21 desde muy temprano, embistiendo con gran furor y
simultáneamente el monte Torrero y el arrabal de la izquierda del Ebro,
puntos sin cuya posesión era excusado pensar en someter la valerosa
ciudad; pero si bien tuvimos que abandonar a Torrero, por ser peligrosa
su defensa, en el arrabal desplegó Zaragoza tanto y tan temerario
arrojo, que es aquel día uno de los más brillantes de su brillantísima
historia.

Desde las cuatro de la madrugada, el batallón de las Peñas de San Pedro
fue destinado a guarnecer el frente de fortificaciones desde Santa
Engracia hasta el Convento de Trinitarios, línea que me pareció la menos
endeble en todo el circuito de la ciudad. A espaldas de Santa Engracia
estaba la batería de los Mártires: corría luego la tapia, aspillerada
hasta el puente de la Huerva, defendido por un reducto: desviábase luego
hacia Poniente, formando un ángulo obtuso, y enlazándose con otro
reducto levantado en la torre del Pino, seguía casi en línea recta hasta
el Convento de Trinitarios dejando dentro la puerta del Carmen. El que
haya visto a Zaragoza, comprenderá perfectamente mi ligera descripción,
pues todavía existen las ruinas de Santa Engracia, y la puerta del
Carmen ostenta aún no lejos de la Glorieta su despedazado umbral y sus
sillares carcomidos.

Estábamos, como he dicho, guarneciendo la extensión descrita, y parte de
los soldados teníamos nuestro vivac en una huerta inmediata al colegio
del Carmen. Agustín de Montoria y yo no nos separábamos, porque su
apacible carácter, el afecto que me mostró desde que nos conocimos, y
cierta conformidad, cierta armonía inexplicable en nuestras ideas, me
hacían muy agradable su compañía. Era él un joven de hermosísima figura,
con ojos grandes y vivos, despejada frente y cierta gravedad melancólica
en su fisonomía. Su corazón, como el del padre, estaba lleno de aquella
generosidad que se desbordaba al menor impulso; pero tenía sobre él la
ventaja de no lastimar al favorecido, porque la educación le había
quitado gran parte de la rudeza nacional. Agustín entraba en la edad
viril con la firmeza y la seguridad de un corazón lleno, de un
entendimiento rico y no gastado, de un alma vigorosa y sana, a la cual
no faltaba sino ancho mundo, ancho espacio para producir bondades sin
cuento. Estas cualidades eran realzadas por una imaginación brillante,
pero de vuelo seguro y derecho, no parecida a la de nuestros modernos
geniecillos, que las más de las veces ignoran por dónde van, sino serena
y majestuosa, como educada en la gran escuela de los latinos.

Aunque con gran inclinación a la poesía (pues Agustín era poeta), había
aprendido la ciencia teológica, descollando en ella como en todo. Los
padres del Seminario, hombres de mucha ciencia y muy cariñosos con la
juventud, le tenían por un prodigio en las letras humanas y en las
divinas, y se congratulaban de verle con un pie dentro de la Iglesia
docente. La familia de Montoria no cabía en sí de gozo y esperaba el día
de la primera misa como el santo advenimiento.

Sin embargo (me veo obligado a decirlo desde el principio), Agustín no
tenía vocación para la iglesia. Su familia, lo mismo que los buenos
padres del Seminario, no lo comprendían así ni lo comprendieran aunque
bajara a decírselo el Espíritu Santo en persona. El precoz teólogo, el
humanista que tenía a Horacio en las puntas de los dedos, el dialéctico
que en los ejercicios semanales dejaba atónitos a los maestros con la
intelectual gimnasia de la ciencia escolástica, no tenía más vocación
para el sacerdocio que la que tuvo Mozart para la guerra, Rafael para
las matemáticas o Napoleón para el baile.

\hypertarget{v}{%
\chapter{V}\label{v}}

---Gabriel---me decía aquella mañana,---¿tienes ganas de batirte?

---Agustín, ¿tienes tú ganas de batirte?---le respondí. (Como se ve nos
tuteábamos a los tres días de conocernos.)---No
muchas---dijo.---Figúrate que la primera bala nos matara\ldots{}

---Moriríamos por la patria, por Zaragoza, y aunque la posteridad no se
acordara de nosotros, siempre es un honor caer en el campo de batalla
por una causa como esta.

---Dices bien---repuso con tristeza;---pero es una lástima morir. Somos
jóvenes. ¿Quién sabe lo que nos está destinado en la vida?

---La vida es una miseria, y para lo que vale, mejor es no pensar en
ella.

---Eso que lo digan los viejos; pero no nosotros que empezamos a vivir.
Francamente, yo no quisiera ser muerto en este terrible cerco que nos
han puesto los franceses. En el otro sitio también tomamos las armas
todos los alumnos del Seminario, y te confieso que estaba yo más
valiente que ahora. Un fuego particular enardecía mi sangre, y me
lanzaba a los puestos de mayor peligro sin temer la muerte. Hoy no me
pasa lo mismo: estoy medroso y el disparo de un fusil me hace
estremecer.

---Eso es natural---contesté.---El miedo no existe cuando no se conoce
el peligro. Por eso dicen que los más valientes soldados son los
bisoños.

---No es nada de eso. Francamente, Gabriel, te confieso que esto de
morir sin más ni más me sabe muy mal. Por si muero voy a hacerte un
encargo, que espero cumplirás con la solicitud de un buen amigo. Atiende
bien a lo que te digo. ¿Ves aquella torre que se inclina de un lado y
parece alongarse hacia acá para ver lo que aquí pasa u oír lo que
estamos diciendo?

---La Torre Nueva. Ya la veo; ¿qué encargo me vas a dar para esa señora?

Amanecía, y entre los irregulares tejados de la ciudad, entre las
espadañas, minaretes, miradores y cimborrios de las iglesias, se
destacaba la Torre Nueva, siempre \emph{vieja} y nunca derecha.

---Pues oye bien---continuó Agustín.---Si me matan a los primeros tiros
en este día que ahora comienza, cuando acabe la acción y rompan filas,
te vas allá\ldots{}

---¿A la Torre Nueva? Llego, subo\ldots{}

---No hombre, subir no. Te diré: llegas a la plaza de San Felipe, donde
está la Torre\ldots{} Mira hacia allá: ¿ves que junto a la gran mole hay
otra torre, un campanario pequeñito? Parece un monaguillo delante del
señor canónigo, que es la torre grande.

---Sí, ya veo al monaguillo. Y si no me engaño, es el campanario de San
Felipe. Y ahora toca el maldito.

---A misa, está tocando a misa---dijo Agustín con grande emoción.---¿No
oyes el esquilón rajado?

---Pues bien, sepamos lo que tengo que decir a ese señor monaguillo que
toca el esquilón rajado.

---No, no es nada de eso. Llegas a la plaza de San Felipe. Si miras al
campanario, verás que está en una esquina: de esta esquina parte una
calle angosta: entras por ella y a la izquierda encontrarás al poco
trecho otra calle angosta y retirada que se llama de Antón Trillo.
Sigues por ella hasta llegar a espaldas de la iglesia. Allí verás una
casa: te paras\ldots{}

---Y luego me vuelvo.

---No: junto a la casa de que te hablo hay una huerta, con un portalón
pintado de color de chocolate. Te paras allí\ldots{}

---Me paro allí, y allí me estoy.

---No hombre: verás\ldots{}

---Estás más blanco que la camisa, Agustinillo. ¿Qué significan esas
torres y esas paradas?

---Significan---continuó mi amigo con más embarazo cada vez,---que en
cuanto estés allí\ldots{} Te advierto que debes ir de noche\ldots{}
Bueno; llegas, te paras; aguardas un poquito; luego pasas a la acera de
enfrente, alargas el cuello y verás por sobre la tapia de la huerta una
ventana. Coges una piedrecita y la tiras contra los vidrios de modo que
no haga mucho ruido.

---Y en seguida saldrá ella.

---No, hombre, ten paciencia. ¿Qué sabes tú si saldrá o no saldrá?

---Bueno: pongamos que sale.

---Antes te diré otra cosa, y es que allí vive el tío Candiola. ¿Tú
sabes quién es el tío Candiola? Pues es un vecino de Zaragoza, hombre
que según dicen, tiene en su casa un sótano lleno de dinero. Es avaro y
usurero y cuando presta saca las entrañas. Sabe de leyes, y moratorias y
ejecuciones más que todo el Consejo y Cámara de Castilla. El que se mete
en pleito con él está perdido. Es riquísimo.

---De modo que la casa del portalón pintado de color de chocolate será
un magnífico palacio.

---Nada de eso: verás una casa miserable, que parece se está cayendo. Te
digo que el tío Candiola es avaro. No gasta un real aunque lo fusilen, y
si le vieras por ahí, le darías una limosna. Te diré otra cosa, y es que
en Zaragoza nadie le puede ver, y le llaman tío Candiola por mofa y
desprecio de su persona. Su nombre es D. Jerónimo de Candiola, natural
de Mallorca, si no me engaño.

---Y ese tío Candiola tiene una hija.

---Hombre, espera. ¡Qué impaciente eres! ¿Qué sabes tú si tiene o no
tiene una hija?---me dijo, disimulando con estas evasivas su
turbación.---Pues como te iba contando, el tío Candiola es muy
aborrecido en la ciudad por su gran avaricia y mal corazón. A muchos
pobres ha metido en la cárcel después de arruinarlos. Además en el otro
sitio no dio un cuarto para la guerra, ni tomó las armas, ni recibió
heridos en su casa, ni le pudieron sacar una peseta, y como un día
dijera que a él lo mismo le daba Juan que Pedro, estuvo a punto de ser
arrastrado por los patriotas.

---Pues es una buena pieza el hombre de la casa de la huerta del
portalón color de chocolate. ¿Y si cuando arroje la piedra a la ventana,
sale el tío Candiola con un garrote y me da una solfa por hacerle
chicoleos a su hija?

---No seas bestia, y calla. ¿No sabes que desde que oscurece, Candiola
se encierra en un cuarto subterráneo y se está contando su dinero hasta
más de medianoche? ¡Bah! Ahora va él a ocuparse\ldots{} Los vecinos
dicen que sienten un cierto rumorcillo o sonsonete como si estuvieran
vaciando sacos de onzas.

---Bien: llego, arrojo la piedra, espero, ella sale y le digo\ldots{}

---Le dices que he muerto\ldots{} no, no seas bárbaro. Le das este
escapulario\ldots{} no, le dices\ldots{} no, más vale que no le digas
nada.

---Entonces, le daré el escapulario.

---Tampoco: no le lleves el escapulario.

---Ya, ya comprendo. Luego que salga, le daré las buenas noches y me
marcharé cantando \emph{La Virgen del Pilar dice}\ldots{}

---No: es preciso que sepa mi muerte. Tú haz lo que yo te mando.

---Pero si no me mandas nada.

---¿Pero qué prisa tienes? Deja tú. Todavía puede ser que no me maten.

---Ya. ¡Cuánto ruido para nada!

---Es que me pasa una cosa, Gabriel, y te la diré francamente. Tenía
muchos, muchísimos deseos de confiarte este secreto, que se me sale del
pecho. ¿A quién lo había de revelar sino a ti, que eres mi amigo? Si no
te lo dijera, me reventaría el corazón como una granada. Temo mucho
decirlo de noche en sueños, y por este temor no duermo. Si mi padre, mi
madre o mi hermano lo supieran, me matarían.

---¿Y los padres del Seminario?

---No nombres a los padres. Verás: te contaré lo que me ha pasado.
¿Conoces al padre Rincón? Pues el padre Rincón me quiere mucho, y todas
las tardes me sacaba a paseo por la ribera o hacia Torrero o camino de
Juslibol. Hablábamos de teología y de letras humanas. Rincón es tan
entusiasta del gran poeta Horacio que suele decir: «Es lástima que ese
hombre no haya sido cristiano para canonizarlo.» Lleva siempre consigo
un pequeño Elzevirius, a quien ama más que a las niñas de sus ojos, y
cuando nos cansamos en el paseo, él se sienta, lee y entre los dos
hacemos los comentarios que se nos ocurren\ldots{} Bueno\ldots{} ahora
te diré que el padre Rincón era pariente de doña María Rincón, difunta
esposa de Candiola, y que este tiene una heredad en el camino de
Monzalbarba, con una \emph{torre} miserable, más parecida a cabaña que a
\emph{torre}, pero rodeada de frondosos árboles y con deliciosas vistas
al Ebro. Una tarde, después que leímos el \emph{Quis multa gracilis te
puer in rosa}, mi maestro quiso visitar a su pariente. Fuimos allá,
entramos en la huerta, y Candiola no estaba. Pero nos salió al encuentro
su hija, y Rincón le dijo:---Mariquilla, da unos melocotones a este
joven y saca para mí una copita de lo que sabes.

---¿Y es guapa Mariquilla?

---No preguntes eso. ¿Que si es guapa? Verás\ldots{} El padre Rincón le
tomó la barba, y haciéndole volver la cara hacia mí, me dijo: «Agustín,
confiesa que en tu vida has visto una cara más linda que esta. Mira qué
ojos de fuego, qué boca de ángel y qué pedazo de cielo por frente.» Yo
temblaba, y Mariquilla, con el rostro encendido como la grana, se reía.
Luego Rincón continuó diciendo: «A ti que eres un futuro padre de la
Iglesia, y un joven ejemplar sin otra pasión que la de los libros, se te
puede enseñar esta divinidad. Joven, admira aquí las obras admirables
del Supremo Creador. Observa la expresión de ese rostro, la dulzura de
esas miradas, la gracia de esa sonrisa, el frescor de esa boca, la
suavidad de esa tez, la elegancia de ese cuerpo, y confiesa que si es
hermoso el cielo, y la flor, y las montañas, la luz, todas las
creaciones de Dios se oscurecen al lado de la mujer, la más perfecta y
acabada hechura de las inmortales manos.» Esto me dijo mi maestro, y yo,
mudo y atónito, no cesaba de contemplar aquella obra maestra, que era
sin disputa mejor que la Eneida. No puedo explicarte lo que sentí.
Figúrate que el Ebro, ese gran río que baja desde Fontibre hasta dar en
el mar por los Alfaques, se detuviera de improviso en su curso, y
empezase a correr hacia arriba volviendo a las Asturias de Santillana:
pues una cosa así pasó en mi espíritu. Yo mismo me asombraba de ver cómo
todas mis ideas se detuvieron en su curso sosegado, y volvieron atrás,
echando no sé por qué nuevos caminos. Te digo que estaba asombrado y lo
estoy todavía. Mirándola sin saciar nunca la ansiedad tanto de mi alma
como de mis ojos, yo me decía: «La amo de un modo extraordinario. ¿Cómo
es que hasta ahora no había caído en ello?» Yo no había visto a
Mariquilla hasta aquel momento.

---¿Y los melocotones?

---Mariquilla estaba tan turbada delante de mí como yo delante de ella.
El padre Rincón se puso a hablar con el hortelano sobre los desperfectos
que habían hecho en la finca los franceses (pues esto pasaba a
principios de Setiembre, un mes después de levantado el primer sitio) y
Mariquilla y yo nos quedamos solos. ¡Solos! Mi primer impulso fue echar
a correr, y ella, según me ha dicho, también sintió lo mismo. Pero ni
ella ni yo corrimos, sino que nos quedamos allí. De pronto sentí una
grande y extraña energía en mi cerebro. Rompiendo el silencio, comencé a
hablar con ella. Dijimos varias cosas indiferentes al principio, pero a
mí me ocurrían pensamientos que según mi entender, sobresalían de lo
vulgar, y todos, todos los dije. Mariquilla me respondía poco; pero sus
ojos eran más elocuentes que cuanto yo le estaba diciendo. Al fin,
llamonos el padre Rincón, y nos marchamos. Me despedí de ella y en voz
baja le dije que pronto nos volveríamos a ver. Volvimos a Zaragoza. ¡Ay!
Por el camino, los árboles, el Ebro, las cúpulas del Pilar, los
campanarios de Zaragoza, los transeúntes, las casas, las tapias de las
huertas, el suelo, el rumor del viento, los perros del camino, todo me
parecía distinto; todo, cielo y tierra habían cambiado. Mi buen maestro
volvió a leer a Horacio, y yo dije que Horacio no valía nada. Me quiso
comer, y amenazome con retirarme su amistad. Yo elogié a Virgilio con
entusiasmo, y repetí aquellos célebres versos:

\small
\newlength\mlena
\settowidth\mlena{interea, et tacitum vivit sub pectore vulnus.}
\begin{center}
\parbox{\mlena}{\quad \quad\quad\quad \quad \textit{Est mollis flamma medullas  \\
                interea, et tacitum vivit sub pectore vulnus.}}                 \\
\end{center}
\normalsize

---Eso pasó a principios de Setiembre---le dije.---¿Y de entonces acá?

---Desde aquel día ha empezado para mí la nueva vida. Comenzó por una
inquietud ardiente que me quitaba el sueño, haciéndome aborrecible todo
lo que no fuera Mariquilla. La propia casa paterna me era odiosa, y
vagando por los alrededores de la ciudad sin compañía alguna, buscaba en
la soledad la paz de mi espíritu. Aborrecí el colegio, los libros todos
y la teología, y cuando llegó Octubre y me querían obligar a vivir
encerrado en la santa casa, me fingí enfermo para quedarme en la mía.
Gracias a la guerra, que a todos nos ha hecho soldados, puedo vivir
libremente, salir a todas horas, incluso de noche, y verla y hablarle
con frecuencia. Voy a su casa, hago la seña convenida, baja, abre una
ventana con reja, y hablamos largas horas. Los transeúntes pasan; pero
como estoy embozado en mi capa hasta los ojos, con esto y la oscuridad
de la noche, nadie me conoce. Por eso los muchachos del pueblo se
preguntan unos a otros: «¿Quién será el novio de la Candiola?» De
algunas noches a esta parte, recelando que nos descubran, hemos
suprimido la conversación por la reja. María baja, abre el portalón de
la huerta y entro. Nadie puede descubrirnos, porque D. Jerónimo,
creyéndola acostada, se retira a su cuarto a contar el dinero, y la
criada vieja, única que hay en la casa, nos protege. Solos en la huerta,
nos sentamos en una escalera de piedra que allí existe, y al través de
las ramas de un álamo negro y corpulento, vemos a pedacitos la claridad
de la luna. En aquel silencio majestuoso nuestras almas comprenden lo
divino y sentimos con un sentimiento inmenso, que no puede expresarse
por el lenguaje. Nuestra felicidad es tan grande que a veces es un
tormento vivísimo; y si hay momentos en que uno desearía centuplicarse,
también los hay en que uno desearía no existir. Pasamos allí largas
horas. Anteanoche estuve hasta cerca del día, pues como mis padres me
creen en el cuerpo de guardia, no tengo prisa por retirarme. Cuando
principiaba a aclarar la aurora, nos despedimos. Por encima de la tapia
de la huerta se ven los techos de las casas inmediatas, y el pico de la
Torre Nueva. María, señalándole, me dijo:

---Cuando esa torre se ponga derecha, dejaré de quererte.

No dijo más Agustín, porque sonó un cañonazo del lado de Monte Torrero,
y ambos volvimos hacia allá la vista.

\hypertarget{vi}{%
\chapter{VI}\label{vi}}

Los franceses habían embestido con gran empeño las posiciones
fortificadas de Torrero. Defendían estas diez mil hombres mandados por
D. Felipe Saint-March y por O'Neille, ambos generales de mucho mérito.
Los voluntarios de Borbón, de Castilla, del Campo Segorbino, de Alicante
y el provincial de Soria: los cazadores de Fernando VII, el regimiento
de Murcia y otros cuerpos que no recuerdo, rompieron el fuego. Desde el
reducto de los Mártires vimos el principio de la acción y las columnas
francesas que corrían a lo largo del Canal para flanquear a Torrero.
Duró gran rato el fuego de fusilería; mas la lucha no podía prolongarse
mucho tiempo, porque aquel punto no se prestaba a una defensa enérgica,
sin la ocupación y fortificación de otros inmediatos como Buenavista,
Casa-Blanca y el cajero del Canal. Sin embargo, nuestras tropas no se
retiraron sino muy tarde y con el mayor orden, volando el puente de
América y trayéndose todas las piezas, menos una, que había sido
desmontada por el fuego enemigo.

Entre tanto sentíamos fuertísimo estruendo que resonaba a lo lejos, y
como por allí casi había cesado el fuego, supusimos trabada otra acción
en el Arrabal.

---Allá está el brigadier D. José Manso---me dijo Agustín,---con el
regimiento suizo de Aragón, que manda D. Mariano Walker, los voluntarios
de Huesca, de que es jefe D. Pedro Villacampa; los voluntarios de
Cataluña y otros valientes cuerpos. ¡Y nosotros aquí, mano sobre mano!
Por este lado parece que ha concluido. Los franceses se contentarán hoy
con la conquista de Torrero.

---O yo me engaño mucho---repuse,---o ahora van a atacar a San José.

Todos miramos al punto indicado, edificio de grandes dimensiones, que se
alzaba a nuestra izquierda, separado de Puerta Quemada por la hondonada
de la Huerva.

---Allí está Renovales---me dijo Agustín,---el valiente D. Mariano
Renovales, que tanto se distinguió en el otro sitio, y manda ahora los
cazadores de Orihuela y de Valencia.

En nuestra posición todo estaba preparado para una defensa enérgica. En
el reducto del Pilar, en la batería de los Mártires, en la torre del
Pino, lo mismo que en Trinitarios, los artilleros aguardaban con mecha
encendida, y los de infantería escogíamos tras los parapetos las
posiciones que nos parecían más seguras para hacer fuego, si alguna
columna intentaba asaltarnos. Se sentía mucho frío, y los más
tiritábamos. Alguien habría creído que era de miedo; pero no, era de
frío, y quien dijese lo contrario, miente.

No tardó en verificarse el movimiento que yo había previsto, y el
Convento de San José fue atacado por una fuerte columna de infantería
francesa, mejor dicho, fue objeto de una tentativa de ataque o más bien
sorpresa. Al parecer, los enemigos tenían mala memoria y en tres meses
se les había olvidado que las sorpresas eran imposibles en Zaragoza.
Llegaron, sin embargo, con mucha confianza hasta tiro de fusil, y sin
duda aquellos desgraciados creían que sólo con verlos, caerían muertos
de miedo nuestros guerreros. Los pobrecitos acababan de llegar de la
Silesia y no sabían qué clase de guerra era la de España. Además como
ganaran a Torrero con tan poco trabajo, creyéronse en disposición de
tragarse el mundo. Ello es que avanzaban como he dicho, sin que San José
hiciera demostración alguna, hasta que hallándose a tiro de fusil o poco
menos, vomitaron de improviso tan espantoso fuego las troneras y
aspilleras de aquel edificio, que mis bravos franceses tomaron soleta
con precipitación. Bastantes, sin embargo, quedaron tendidos, y al ver
este desenlace de su valentía, los que contemplábamos el lance desde la
batería de los Mártires, prorrumpimos en exclamaciones, gritos y
palmadas. De este modo celebra el feroz soldado en la guerra la muerte
de sus semejantes, y el que siente instintiva compasión al matar un
conejo en una cacería, salta de júbilo viendo caer centenares de hombres
robustos, jóvenes y alegres que después de todo no han hecho mal a
nadie.

Tal fue el ataque de San José, una intentona rápidamente castigada.
Desde entonces debieron de comprender los franceses, que si se abandonó
a Torrero fue por cálculo y no por flaqueza. Sola, aislada, desamparada,
sin baluartes exteriores, sin fuertes ni castillos, Zaragoza alzaba de
nuevo sus murallas de tierra, sus baluartes de ladrillos crudos, sus
torreones de barro amasado la víspera para defenderse otra vez contra
los primeros soldados, la primera artillería y los primeros ingenieros
del mundo. Grande aparato de gente, formidables máquinas, enormes
cantidades de pólvora, preparativos científicos y materiales, la fuerza
y la inteligencia en su mayor esplendor, traen los invasores para atacar
el recinto fortificado que parece juego de muchachos, y aun así es poco,
todo sucumbe y se reduce a polvo ante aquellas tapias que se derriban de
una patada. Pero detrás de esta deleznable defensa material está el
acero de las almas aragonesas, que no se rompe, ni se dobla, ni se
funde, ni se hiende, ni se oxida y circunda todo el recinto como una
barra indestructible por los medios humanos.

La campana de la Torre Nueva suena con clamor de alarma. Cuando esta
campana da al viento su lúgubre tañido la ciudad está en peligro y
necesita de todos sus hijos. ¿Qué será? ¿Qué pasa? ¿Qué hay?

---En el arrabal---dijo Agustín,---debe de andar mala la cosa.

---Mientras nos atacan por aquí para entretener mucha gente de este
lado, embisten también por la otra parte del río.

---Lo mismo fue en el primer sitio.

---¡Al arrabal, al arrabal!

Y cuando decíamos esto, la línea francesa nos envió algunas balas rasas
para indicarnos que teníamos que permanecer allí. Felizmente Zaragoza
tenía bastante gente en su recinto y podía acudir con facilidad a todas
partes. Mi batallón abandonó la cortina de Santa Engracia y púsose en
marcha hacia el Coso. Ignorábamos a dónde se nos conducía; pero era
probable que nos llevaran al arrabal. Las calles estaban llenas de
gente. Los ancianos, las mujeres salían impulsados por la curiosidad,
queriendo ver de cerca los puntos de peligro, ya que no les era posible
situarse en el peligro mismo. Las calles de San Gil, de San Pedro y la
Cuchillería\footnote{Esta calle, unida a las de San Pedro y la
  Cuchillería, se llama hoy de D. Jaime I.}, que son camino para el
puente, estaban casi intransitables; inmensa multitud de mujeres las
cruzaba, marchando todas a prisa en dirección al Pilar y a la Seo. El
estrépito del lejano canon más bien animaba que entristecía al fervoroso
pueblo, y todo era gritar disputándose el paso para llegar más pronto.
En la plaza de la Seo vi la caballería, que con el gran gentío casi
obstruía la salida del puente, lo cual obligó a mi batallón a buscar más
fácil salida por otra parte. Cuando pasamos por delante del pórtico de
este santuario sentimos desde fuera el clamor de las plegarias con que
todas las mujeres de la ciudad imploraban a la santa patrona. Los pocos
hombres que querían penetrar en el templo eran expulsados por ellas.

Salimos a la orilla del río por junto a San Juan de los Panetes y nos
situaron en el malecón esperando órdenes. Enfrente y al otro lado del
río se divisaba el campo de batalla. Veíase en primer término la
arboleda de Macanaz, más allá y junto al puente el pequeño monasterio de
Altabás, más allá el de San Lázaro y a continuación el de Jesús. Detrás
de esta decoración, reflejada en las aguas del gran río, la vista
distinguía un fuego horroroso, un cruzamiento interminable de
trayectorias, un estrépito ronco, de las voces del cañón y de humanos
gritos formado, y densas nubes de humo que se renovaban sin cesar y
corrían a confundirse con las del cielo. Todos los parapetos de aquel
sitio estaban construidos con los ladrillos de los cercanos tejares,
formando con el barro y la tierra de los hornos una masa rojiza.
Creeríase que la tierra estaba amasada con sangre.

Los franceses tenían su frente desde el camino de Barcelona al de
Juslibol, más allá de los tejares y de las huertas que hay a mano
izquierda de la segunda de aquellas dos vías. Desde las doce habían
atacado con furia nuestras trincheras, internándose por el camino de
Barcelona y desafiando con impetuoso arrojo los fuegos cruzados de San
Lázaro y del sitio llamado el Macelo. Consistía su empeño en tomar por
audaces golpes de mano las baterías, y esta tenacidad produjo una
verdadera hecatombe. Caían muchísimos, clareábanse las filas, y llenadas
al instante por otros, repelían la embestida. A veces llegaban hasta
tocar los parapetos y mil luchas individuales acrecían el horror de la
escena. Iban delante los jefes blandiendo sus sables, como hombres
desesperados que han hecho cuestión de honor el morir ante un montón de
ladrillos, y en aquella destrucción espantosa que arrancaba a la vida
centenares de hombres en un minuto, desaparecían, arrojados por el suelo
el soldado y el sargento y el alférez y el capitán y el coronel. Era una
verdadera lucha entre dos pueblos, y mientras los furores del primer
sitio inflamaban los corazones de los nuestros, venían los franceses
frenéticos, sedientos de venganza, con toda la saña del hombre ofendido,
peor acaso que la del guerrero.

Precisamente este prematuro encarnizamiento les perdió. Debieron
principiar batiendo cachazudamente con su artillería nuestras obras;
debieron conservar la serenidad que exige un sitio, y no desplegar
guerrillas contra posiciones defendidas por gente como la que habían
tenido ocasión de tratar el 15 de Julio y el 4 de Agosto; debieron haber
reprimido aquel sentimiento de desprecio hacia las fuerzas del enemigo,
sentimiento que ha sido siempre su mala estrella, lo mismo en la guerra
de España que en la moderna contra Prusia; debieron haber puesto en
ejecución un plan calmoso, que produjera en el sitiado antes el fastidio
que la exaltación. Es seguro que de traer consigo la mente pensadora de
su inmortal jefe, que vencía siempre con su lógica admirable lo mismo
que con sus cañones, habrían empleado en el sitio de Zaragoza no poco
del conocimiento del corazón humano, sin cuyo estudio la guerra, la
brutal guerra, ¡parece mentira!, no es más que una carnicería salvaje.
Napoleón, con su penetración extraordinaria. hubiera comprendido el
carácter zaragozano y se habría abstenido de lanzar contra él columnas
descubiertas, haciendo alarde de valor personal. Esta es una cualidad de
difícil y peligroso empleo, sobre todo delante de hombres que se baten
por un ideal, no por un ídolo.

No me extenderé en pormenores sobre esta espantosa acción del 21 de
Diciembre, una de las más gloriosas del segundo sitio de la capital de
Aragón. Sobre que no la presencié de cerca, y sólo podría dar cuenta de
ella por lo que me contaron, me mueve a no ser prolijo la circunstancia
de que son tantos y tan interesantes los encuentros que más adelante
habré de narrar, que conviene cierta sobriedad en la descripción de
estos sangrientos choques. Baste saber por ahora que los franceses al
caer de la tarde creyeron oportuno desistir de su empeño, y que se
retiraron dejando el campo cubierto de cadáveres. Era la ocasión muy
oportuna para perseguirlos con la caballería; pero después de una breve
discusión, según se dijo, acordaron los jefes no arriesgarse en una
salida que podía ser peligrosa.

\hypertarget{vii}{%
\chapter{VII}\label{vii}}

Llegada la noche, y cuando parte de nuestras tropas se replegaron a la
ciudad, todo el pueblo corrió hacia el arrabal para contemplar de cerca
el campo de batalla, ver los destrozos hechos por el fuego, contar los
muertos y regocijar la imaginación, representándose una por una las
heroicas escenas. La animación, el movimiento y bulla hacia aquella
parte de la ciudad eran inmensos. Por un lado grupos de soldados
cantando con febril alegría, por otro las cuadrillas de personas
piadosas que trasportaban a sus casas los heridos, y en todas partes una
general satisfacción, que se mostraba en los diálogos vivos, en las
preguntas, en las exclamaciones jactanciosas y con lágrimas y risas,
mezclando la jovialidad al entusiasmo.

Serían las nueve cuando rompimos filas los de mi batallón, porque faltos
de acuartelamiento, se nos permitía dejar el puesto por algunas horas,
siempre que no hubiera peligro. Corrimos Agustín y yo hacia el Pilar,
donde se agolpaba un gentío inmenso, y entramos difícilmente. Quedeme
sorprendido al ver cómo forcejeaban unas contra otras las personas allí
reunidas para acercarse a la capilla en que mora la Virgen del Pilar.
Los rezos, las plegarias y las demostraciones de agradecimiento formaban
un conjunto que no se parecía a los rezos de ninguna clase de fieles.
Más que rezo era un hablar continuo, mezclado de sollozos, gritos,
palabras tiernísimas y otras de íntima e ingenua confianza, como suele
usarlas el pueblo español con los santos que le son queridos. Caían de
rodillas, besaban el suelo, se asían a las rejas de la capilla, se
dirigían a la santa imagen, llamándola con los nombres más familiares y
más patéticos del lenguaje. Los que por la aglomeración de la gente no
podían acercarse, hablaban con la Virgen desde lejos agitando sus
brazos. Allí no había sacristanes que prohibieran los modales
descompuestos y los gritos irreverentes, porque estos y aquellos eran
hijos del desbordamiento de la devoción, semejante a un delirio. Faltaba
el silencio solemne de los lugares sagrados, y todos estaban allí como
en su casa, como si la casa de la Virgen querida, la madre, ama y reina
de los zaragozanos, fuese también la casa de sus hijos, siervos y
súbditos.

Asombrado de aquel fervor, a quien la familiaridad hacía más
interesante, pugné por abrirme paso hasta la reja, y vi la célebre
imagen. ¿Quién no la ha visto, quién no la conoce al menos por las
innumerables esculturas y estampas que la han reproducido hasta lo
infinito de un extremo a otro de la Península? A la izquierda del
pequeño altar que se alza en el fondo de la capilla, dentro de un nicho
adornado con lujo oriental, estaba entonces como ahora la pequeña
escultura. Gran profusión de velas de cera la alumbraban, y las piedras
preciosas pegadas a su vestido y corona, despiden deslumbradores
reflejos. Brillan el oro y los diamantes en el cerquillo de su rostro,
en la ajorca de su pecho, en los anillos de sus manos. Una criatura viva
rendiríase sin duda al peso de tan gran tesoro. El vestido sin pliegues,
rígido y estirado de arriba a abajo como una funda, deja asomar
solamente la cara y las manos; y el Niño Jesús, sostenido en el lado
izquierdo, muestra apenas su carita morena entre el brocado y las
pedrerías. El rostro de la Virgen, bruñido por el tiempo, es también
moreno. Posee una apacible serenidad, emblema de la beatitud eterna.
Dirígese al exterior, y su dulce mirada escruta perpetuamente el devoto
concurso. Brilla en sus pupilas un rayo de las cercanas luces, y aquel
artificial fulgor de los ojos remeda la intención y fijeza de la mirada
humana. Era difícil, cuando la vi por primera vez, permanecer
indiferente en medio de aquella manifestación religiosa, y no añadir una
palabra al concierto de lenguas entusiastas que hablaban en distintos
tonos con la Señora.

Yo contemplaba la imagen, cuando Agustín me apretó el brazo, diciéndome:

---Mírala, allí está.

---¿Quién, la Virgen? Ya la veo.

---No, hombre, Mariquilla. ¿La ves? Allá enfrente junto a la columna.

Miré y sólo vi mucha gente: al instante nos apartamos de aquel sitio,
buscando entre la multitud un paso para transportarnos al otro lado.

---No está con ella el tío Candiola---dijo Agustín muy alegre.---Viene
con la criada.

Y diciendo esto, codeaba a un lado y otro para hacerse camino,
estropeando pechos y espaldas, pisando pies, chafando sombreros y
arrugando vestidos. Yo seguía tras él, causando iguales estragos a
derecha e izquierda, y por fin llegamos junto a la hermosa joven, que lo
era realmente, según pude reconocerlo en aquel momento por mis propios
ojos. La entusiasta pasión de mi buen amigo no me engañó, y Mariquilla
valía la pena de ser desatinadamente amada. Llamaban la atención en ella
su tez morena y descolorida, sus ojos de profundo negror, la nariz
correctísima, la boca incomparable y la frente hermosa aunque pequeña.
Había en su rostro, como en su cuerpo delgado y ligero, cierto abandono
voluptuoso; cuando bajaba los ojos parecíame que una dulce y amorosa
oscuridad envolvía su figura, confundiéndola con las nuestras. Sonreía
con gravedad, y cuando nos acercamos, sus miradas revelaban temor. Todo
en ella anunciaba la pasión circunspecta y reservada de las mujeres de
cierto carácter, y debía de ser, según me pareció en aquel momento, poco
habladora, falta de coquetería y pobre de artificios. Después tuve
ocasión de comprobar aquel mi prematuro juicio. Resplandecía en el
rostro de Mariquilla una calma platónica y cierta seguridad de sí misma.
A diferencia de la mayor parte de las mujeres, y semejante al menor
número de las mismas, aquella alma se alteraba difícilmente, pero al
verificarse la alteración, la cosa iba de veras. Blandas y sensibles
otras como la cera, ante un débil calor sin esfuerzo se funden; pero
Mariquilla, de durísimo metal compuesta, necesitaba la llama de un gran
fuego para perder la compacta conglomeración de su carácter, y si este
momento llegaba, había de ser como el metal derretido que abrasa cuanto
toca.

Además de su belleza, me llamó la atención la elegancia y hasta cierto
punto el lujo con que vestía; pues acostumbrado a oír exagerar la
avaricia del tío Candiola, supuse que tendría reducida a su hija a los
últimos extremos de la miseria en lo relativo a traje y tocado. Pero no
era así. Según Montoria me dijo después, el tacaño de los tacaños no
sólo permitía a su hija algunos gastos, sino que la obsequiaba de peras
a higos, con tal cual prenda, que a él le parecía el non plus ultra de
las pompas mundanas. Si Candiola era capaz de dejar morir de hambre a
parientes cercanos, tenía con su hija condescendencias de bolsillo
verdaderamente escandalosas y fenomenales; pero aunque avaro, era padre:
amaba regularmente, quizás mucho, a la infeliz muchacha, hallando por
esto en su generosidad el primero, tal vez el único agrado de su árida
existencia.

Algo más hay que hablar en lo referente a este punto, pero irá saliendo
poco a poco durante el curso de la narración, y ahora me concretaré a
decir que mi amigo no había dicho aun diez palabras a su adorada María,
cuando un hombre se nos acercó de súbito, y después de mirarnos un
instante a los dos con centelleantes ojos, dirigiose a la joven, la tomó
por el brazo, y enojadamente le dijo:

---¿Qué haces aquí? Y Vd., tía Guedita, ¿por qué la ha traído al Pilar a
estas horas? A casa, a casa pronto.

Y empujándolas a ambas, ama y criada, llevolas hacia la puerta y a la
calle, desapareciendo los tres de nuestra vista.

Era Candiola. Lo recuerdo bien, y su recuerdo me hace estremecer de
espanto. Más adelante sabréis por qué. Desde la breve escena en el
templo del Pilar, la imagen de aquel hombre quedó grabada en mi memoria,
y no era ciertamente su figura de las que prontamente se olvidan. Viejo,
encorvado, con aspecto miserable y enfermizo, de mirar oblicuo y
desapacible, flaco de cara y hundido de mejillas, Candiola se hacía
antipático desde el primer momento. Su nariz corva y afilada como el
pico de un pájaro lagartijero, la barba igualmente picuda, los largos
pelos de las cejas blanquinegras, la pupila verdosa, la frente vasta y
surcada por una pauta de paralelas arrugas, las orejas cartilaginosas,
la amarilla tez, el ronco metal de la voz, el desaliñado vestir, el
gesto insultante, toda su persona, desde la punta del cabello, mejor
dicho, desde la bolsa de su peluca hasta la suela del zapato, producía
repulsión invencible. Se comprendía que no tuviera ningún amigo.

Candiola no tenía barbas; llevaba el rostro, según la moda,
completamente rasurado, aunque la navaja no entraba en aquellos campos
sino una vez por semana. Si D. Jerónimo hubiera tenido barbas, le
compararía por su figura a cierto mercader veneciano que conocí mucho
después, viajando por el vastísimo continente de los libros, y en quien
hallé ciertos rasgos de fisonomía que me hicieron recordar los de aquel
que bruscamente se nos presentó en el templo del Pilar.

---¿Has visto qué miserable y ridículo viejo?---me dijo Agustín cuando
nos quedamos solos, mirando a la puerta por donde las tres personas
habían desaparecido.

---No gusta que su hija tenga novios.

---Pero estoy seguro de que no me vio hablando con ella. Tendrá
sospechas; pero nada más. Si pasara de la sospecha a la certidumbre,
María y yo estaríamos perdidos. ¿Viste qué mirada nos echó? ¡Condenado
avaro, alma negra hecha de la piel de Satanás!

---Mal suegro tienes.

---Tan malo---dijo Montoria con tristeza,---que no doy por él dos
cuartos con cardenillo. Estoy seguro de que esta noche la pone de vuelta
y media, y gracias que no acostumbra a maltratarla de obra.

---Y el Sr.~Candiola---le pregunté,---¿no tendrá gusto en verla casada
con el hijo de D. José de Montoria?

---¿Estás loco? Sí\ldots{} ve a hablarle de eso. Además de que ese
miserable avariento guarda a su hija como si fuera un saco de onzas y no
parece dispuesto a darla a nadie, tiene un resentimiento antiguo y
profundo contra mi buen padre porque este libró de sus garras a unos
infelices deudores. Te digo que si él llega a descubrir el amor que su
hija me tiene, la guardará dentro de un arca de hierro en el sótano
donde esconde los pesos duros. Pues no te digo nada, si mi padre lo
llega a saber\ldots{} Me tiemblan las carnes sólo de pensarlo. La
pesadilla más atroz que puede turbar mi sueño, es aquella que me
representa el instante en que mi señor padre y mi señora madre se
enteren de este inmenso amor que tengo por Mariquilla. ¡Un hijo de D.
José de Montoria enamorado de la hija del tío Candiola! ¡Qué horrible
pensamiento! ¡Un joven que formalmente está destinado a ser
obispo\ldots{} obispo, Gabriel, yo voy a ser obispo, en el sentir de mis
padres!

Diciendo esto, Agustín dio un golpe con su cabeza en el sagrado muro en
que nos apoyábamos.

---¿Y piensas seguir amando a Mariquilla?

---No me preguntes eso---me respondió con energía.---¿La viste? Pues si
la viste ¿a qué me dices si seguiré amándola? Su padre y los míos antes
me quieren ver muerto que casado con ella. ¡Obispo, Gabriel, quieren que
yo sea obispo! Compagina tú el ser obispo y el amar a Mariquilla durante
toda la vida terrenal y la eterna: compagina tú esto, y ten lástima de
mí.

---Dios abre caminos desconocidos---le dije.

---Es verdad. Yo tengo a veces una confianza sin límites. ¡Quién sabe lo
que nos traerá el día de mañana! Dios y la Virgen del Pilar me sacarán
adelante.

---¿Eres devoto de esta imagen?

---Sí. Mi madre pone velas a la que tenemos en casa para que no me
hieran en las batallas; y yo la miro, y para mis adentros le
digo:---¡Señora, que esta ofrenda de velas sirva también para recordaros
que no puedo dejar de amar a la Candiola!

Estábamos en la nave a que corresponde el ábside de la capilla del
Pilar. Hay allí una abertura en el muro, por donde los devotos, bajando
dos o tres peldaños, se acercan a besar el pilar que sustenta la
venerada imagen. Agustín besó el mármol rojo: beselo yo también y luego
salimos de la iglesia para ir a nuestro vivac.

\hypertarget{viii}{%
\chapter{VIII}\label{viii}}

El día siguiente, 22, fue cuando Palafox dijo al parlamentario de Moncey
que venía a proponerle la rendición: \emph{No sé rendirme: después de
muerto hablaremos de eso.} Contestó en seguida a la intimación en un
largo y elocuente pliego, que publicó la \emph{Gaceta} (pues también en
Zaragoza había \emph{Gaceta}); pero según opinión general ni aquel
documento ni ninguna de las proclamas que aparecían con la firma del
capitán general eran obra de este, sino de la discreta pluma de su
maestro y amigo el padre Basilio Boggiero, hombre de mucho
entendimiento, a quien se veía con frecuencia en los sitios de peligro
rodeado de patriotas y jefes militares.

Excusado es decir que los defensores estaban muy envalentonados con la
gloriosa acción del 21. Era preciso para dar desahogo a su ardor,
disponer alguna salida. Así se hizo en efecto; pero ocurrió que todos
querían tomar parte en ella al mismo tiempo, y fue preciso sortear los
cuerpos. Las salidas, dispuestas con prudencia eran convenientes, porque
los franceses, extendiendo su línea en derredor de la ciudad, se
preparaban para un sitio en regla, y habían comenzado las obras de su
primera paralela. Además el recinto de Zaragoza encerraba mucha tropa,
lo cual a los ojos del vulgo era una ventaja, pero un gran peligro para
los inteligentes, no sólo por el estorbo que esta causaba, sino porque
el gran consumo de víveres traería pronto el hambre, ese terrible
general que es siempre el vencedor de las plazas bloqueadas. Por esta
misma causa del exceso de gente eran oportunas las salidas. Hizo una
Renovales el 24 con las tropas del fortín de San José, y cortó un olivar
que ocultaba los trabajos del enemigo; por el arrabal salió el 25 D.
Juan O'Neille con los voluntarios de Aragón y Huesca, y tuvo la suerte
de coger desprevenido al enemigo, matándole bastante gente, y el 31 se
hizo la más eficaz de todas por dos puntos distintos y con fuerzas
considerables.

Durante el día, en los anteriores, habíamos divisado perfectamente las
obras de su primera paralela, establecida como a ciento sesenta toesas
de la muralla. Trabajaban con mucha actividad, sin descansar de noche, y
notamos que se hacían señales en toda la línea con farolitos de colores.
De vez en cuando disparábamos nuestros morteros; pero les causábamos muy
poco daño. En cambio si se les antojaba destacar guerrillas para un
reconocimiento, eran despachadas por las nuestras en menos que canta un
gallo. Llegó la mañana del 31, y a mi batallón le tocó marchar a las
órdenes de Renovales, encargado de mortificar al enemigo en su centro,
desde Torrero al camino de la Muela, mientras el brigadier Butrón lo
hacía por la Bernardona, es decir por la izquierda francesa, saliendo
con bastantes fuerzas de infantería y caballería por las puertas de
Sancho y del Portillo.

Para distraer la atención de los franceses, el jefe mandó que un
batallón se desplegase en guerrillas por las Tenerías llamando hacia
allí la atención del enemigo, y entre tanto con algunos cazadores de
Olivenza, y parte de los de Valencia, avanzamos por el camino de Madrid,
derechos a la línea francesa. Desplegadas guerrillas a un lado y otro
del camino, cuando los enemigos se percataron de nuestra presencia, ya
estábamos encima, veloces como gamos, y arrollábamos la primera tropa de
infantería francesa que nos salió al paso. Tras una torre medio
destruida se hicieron fuertes algunos, y dispararon con encarnizamiento
y buena puntería. Por un instante permanecimos indecisos, pues
flanqueábamos la torre unos veinte hombres, mientras los demás seguían
por la carretera, persiguiendo a los fugitivos; pero Renovales se lanzó
delante y nos llevó, matando a boca de jarro y a bayonetazos a cuantos
defendían la casa. En el momento en que pusimos el pie dentro del
patiecillo delantero, advertí que mi fila se clareaba, vi caer exhalando
el último gemido a algunos compañeros; miré a mi derecha temiendo no
encontrar entre los vivos a mi querido amigo; pero Dios le había
conservado. Montoria y yo salimos ilesos.

No podíamos emplear mucho tiempo en comunicarnos la satisfacción que
experimentábamos al ver que vivíamos, porque Renovales dio orden de
seguir adelante en dirección hacia la línea de atrincheramientos que
estaban levantando los franceses; pero abandonamos la carretera y
torcimos hacia la derecha con intento de unirnos a los voluntarios de
Huesca, que acometían por el camino de la Muela. Se comprende por lo que
llevo referido, que los franceses no esperaban aquella salida y que
completamente desprevenidos, sólo tenían allí, además de la escasa
fuerza que custodiaba los trabajos, las cuadrillas de ingenieros que
abrían las zanjas de la primera paralela. Les embestimos con ímpetu,
haciéndoles un fuego horroroso, aprovechando muy bien los minutos antes
que llegasen fuerzas temibles; cogíamos prisioneros a los que
encontrábamos sin armas; matábamos a los que las tenían; recogíamos los
picos y azadas, todo esto con una presteza sin igual, animándonos con
palabras ardientes, y exaltados por la idea de que nos estaban viendo
desde la ciudad.

En aquel lance todo fue afortunado, porque mientras nosotros
destrozábamos tan sin piedad a los trabajadores de la primera paralela,
las tropas que por la izquierda habían salido a las órdenes del
brigadier Butrón, empeñaban un combate muy feliz contra los
destacamentos que tenía el enemigo en la Bernardona. Mientras los
voluntarios de Huesca, los granaderos de Palafox y las guardias walonas
arrollaban la infantería francesa, aparecieron los escuadrones de
caballería de Numancia y Olivenza, cautelosamente salidos por la puerta
de Sancho, y que describiendo una gran vuelta, habían venido a ocupar el
camino de Alagón por una parte y el de la Muela por otra, precisamente
cuando los franceses retrocedían de la izquierda al centro, en demanda
de mayores fuerzas que les auxiliaran. Hallándose en su elemento
aquellos briosos caballos, lanzáronse por el arrecife, destruyendo
cuanto encontraban al paso, y allí fue el caer y el atropellarse de los
desgraciados infantes que huían hacia Torrero. En su dispersión muchos
fueron a caer precisamente entre nuestras bayonetas, y si grande era su
ansiedad por huir de los caballos, mayor era nuestro anhelo de
recibirlos dignamente a tiros. Unos corrían, arrojándose en las acequias
por no poder saltarlas, otros se entregaban a discreción, soltando las
armas, algunos se defendían con heroísmo, dejándose matar antes que
rendirse, y por último no faltaron unos pocos que, encerrándose dentro
de un horno de ladrillos cargado de ramas secas y de leña, le pegaron
fuego, prefiriendo morir asados a caer prisioneros.

Todo esto que he referido con la mayor concisión posible pasó en
brevísimo tiempo, y sólo mientras pudo el cuartel general, harto
imprevisor en aquella hora, destacar fuerzas suficientes para contener y
castigar nuestra atrevida expedición. Tocaron a generala en monte
Torrero, y vimos que venía contra nosotros mucha caballería. Pero los de
Renovales, lo mismo que los de Butrón, habíamos conseguido nuestro deseo
y no teníamos para qué esperar a aquellos caballeros que llegaban al fin
de la función; así es que nos retiramos dándoles desde lejos los buenos
días, con las frases más pintorescas y más agudas de nuestro repertorio.
Tuvimos aún tiempo de inutilizar algunas piezas de las dispuestas para
su colocación al día siguiente; recogimos una multitud de herramientas
de zapa, y destruimos a toda prisa lo que pudimos en las obras de la
paralela, sin dejar de la mano las docenas de prisioneros a quienes
habíamos echado el guante.

Juan Pirli, uno de nuestros compañeros en el batallón, traía al volver a
Zaragoza un morrión de ingeniero, que se puso para sorprender al
público, y además una sartén en la cual aún había restos de almuerzo,
comenzado en el campamento frente a Zaragoza, y terminado en el otro
mundo.

Habíamos tenido en nuestro batallón nueve muertos y ocho heridos. Cuando
Agustín se reunió a mí, cerca ya de la puerta del Carmen, noté que tenía
una mano ensangrentada.

---¿Te han herido?---le dije, examinándole.---No es más que una
rozadura.

---Una rozadura es---me contestó,---pero no de bala, ni de lanza, ni de
sable, sino de dientes, por que cuando le eché la zarpa a aquel francés
que alzó el azadón para descalabrarme, el condenado me clavó los dientes
en esta mano como un perro de presa.

Cuando entrábamos en la ciudad, unos por la puerta del Carmen, otros por
el Portillo, todas las piezas de los reductos y fuertes del Mediodía
hicieron fuego contra las columnas que venían en nuestra persecución.
Las dos salidas combinadas habían hecho bastante daño a los franceses.
Sobre que perdieron mucha gente, se les inutilizó una parte, aunque no
grande, de los trabajos de su primera paralela, y nos apoderamos de un
número considerable de herramientas. Además de esto, los oficiales de
ingenieros que llevó Butrón en aquella osada aventura habían tenido
tiempo de examinar las obras de los sitiadores y explorarlas y medirlas
para dar cuenta de ellas al Capitán General.

La muralla estaba invadida por la gente. Habíase oído desde dentro de la
ciudad el tiroteo de las guerrillas, y hombres, mujeres, ancianos y
niños, todos acudieron a ver qué nueva acción gloriosa era aquella
entablada fuera de la plaza. Fuimos recibidos con exclamaciones de gozo,
y desde San José hasta más allá de Trinitarios, la larga fila de hombres
y mujeres mirando hacia el campo, encaramados sobre la muralla y
batiendo palmas a nuestra llegada o saludándonos con sus pañuelos,
presentaba un golpe de vista magnífico. Después tronó el cañón, los
reductos hicieron fuego a la vez sobre el llano que acabábamos de
abandonar, y aquel estruendo formidable parecía una salva triunfal,
según se mezclaban con él los cantos, los vítores, las exclamaciones de
alegría. En las cercanas casas, las ventanas y balcones estaban llenos
de mujeres, y la curiosidad, el interés de algunas era tal que se las
veía acercarse en tropel a los fuertes y a los cañones para regocijar
sus varoniles almas y templar sus acerados nervios con el ruido, a
ningún otro comparable, de la artillería. En el fortín del Portillo fue
preciso mandar salir a la muchedumbre. En Santa Engracia la concurrencia
daba a aquel sitio el aspecto de un teatro, de una fiesta pública. Cesó
al fin el fuego de cañón, que no tenía más objeto que proteger nuestra
retirada, y sólo la Aljafería siguió disparando de tarde en tarde contra
las obras del enemigo.

En recompensa de la acción de aquel día se nos concedió en el siguiente
llevar una cinta encarnada en el pecho a guisa de condecoración; y
haciendo justicia a lo arriesgado de aquella salida, el padre Boggiero
nos dijo, entre otras cosas, por boca del General: «Ayer sellasteis el
último día del año con una acción digna de vosotros\ldots{} Sonó el
clarín y a un tiempo mismo los filos de vuestras espadas arrojaban al
suelo las altaneras cabezas, humilladas al valor y al patriotismo.
¡Numancia! ¡Olivenza! ¡Ya he visto que vuestros ligeros caballos sabrán
conservar el honor de este ejército y el entusiasmo de estos sagrados
muros!\ldots{} Ceñid esas espadas ensangrentadas, que son el vínculo de
vuestra felicidad y el apoyo de la patria!\ldots»

\hypertarget{ix}{%
\chapter{IX}\label{ix}}

Desde aquel día, tan memorable en el segundo sitio como el de las Eras
en el primero, empezó el gran trabajo, el gran frenesí, la exaltación
ardiente, en que vivieron por espacio de mes y medio sitiadores y
sitiados. Las salidas verificadas en los primeros días de Enero no
fueron de gran importancia. Los franceses, concluida la primera
paralela, avanzaron en zig-zag para abrir la segunda, y con tanta
actividad trabajaron en ella, que bien pronto vimos amenazadas nuestras
dos mejores posiciones del mediodía, San José y el reducto del Pilar,
por imponentes baterías de sitio, cada una con diez y seis cañones.
Excusado es decir que no cesábamos en mortificarles, ya enviándoles un
incesante fuego, ya sorprendiéndoles con audaces escaramuzas; pero así y
todo, Junot, que por aquellos días sustituyó a Moncey, llevaba adelante
los trabajos con mucha diligencia.

Nuestro batallón continuaba en el reducto, obra levantada en la cabecera
del puente de la Huerva y a la parte de fuera. El radio de sus fuegos
abrazaba una extensión considerable cruzándose con los de San José. Las
baterías de los Mártires, del jardín Botánico y de la torre del Pino,
más internadas en el recinto de la ciudad tenían menos importancia que
aquellas dos sólidas posiciones avanzadas, y le servían de auxiliares.
Nos acompañaban en la guarnición muchos voluntarios zaragozanos, algunos
soldados del resguardo, y varios paisanos armados de los que
espontáneamente se adherían al cuerpo más de su gusto. Ocho cañones
tenía el reducto. Era su jefe D. Domingo Larripa, mandaba la artillería
D. Francisco Betbezé, y hacía de jefe de ingenieros el gran Simonó,
oficial de este distinguido cuerpo, y hombre de tal condición que se le
puede citar como modelo de buenos militares, así en el valor como en la
pericia.

Era el reducto una obra, aunque de circunstancias, bastante fuerte, y no
carecía de ningún requisito material para ser bien defendida. Sobre la
puerta de entrada, al extremo del puente habían puesto sus constructores
una tabla con la siguiente inscripción: Reducto inconquistable de
\emph{Nuestra Señora del Pilar. Zaragozanos: ¡morir por la Virgen del
Pilar o vencer!}

Allí dentro no teníamos alojamiento, y aunque la estación no era muy
cruda, lo pasábamos bastante mal. El suministro de provisiones de boca
se hacía por una junta encargada de la administración militar; pero esta
junta a pesar de su celo no podía atendernos de un modo eficaz. Por
nuestra fortuna y para honor de aquel magnánimo pueblo, de todas las
casas vecinas nos mandaban diariamente lo mejor de sus provisiones y
frecuentemente éramos visitados por las mismas mujeres caritativas que
desde la acción del 31 se habían encargado de cuidar en su propio
domicilio a nuestros pobres heridos.

No sé si he hablado de Pirli. Pirli era un muchacho de los arrabales,
labrador, como de veinte años y de condición tan festiva, que los lances
peligrosos desarrollaban en él una alegría nerviosa y febril. Jamás le
vi triste; acometía a los franceses cantando, y cuando las balas
silbaban en torno suyo, sacudía manos y pies haciendo mil grotescos
gestos y cabriolas. Llamaba al fuego graneado \emph{pedrisco}; a las
balas de cañón \emph{las tortas calientes}; a las granadas \emph{las
señoras}, y a la pólvora la harina negra, usando además otros
terminachos de que no hago memoria en este momento. Pirli, aunque poco
formal, era un cariñoso compañero.

No sé si he hablado del tío Garcés. Era este un hombre de cuarenta y
cinco años, natural de Garrapinillos, fortísimo, atezado, con semblante
curtido y miembros de acero, ágil cual ninguno en los movimientos e
imperturbable como una máquina ante el fuego; poco hablador y bastante
desvergonzado cuando hablaba, pero con cierto gracejo en su garrulería.
Tenía una pequeña hacienda en los alrededores, y casa muy modesta; mas
con sus propias manos había arrasado la casa, y puesto por tierra los
perales, para quitar defensas al enemigo. Oí contar de él mil proezas
hechas en el primer sitio y ostentaba bordado en la manga derecha el
\emph{escudo de premio y distinción} de 16 de Agosto. Vestía tan mal que
casi iba medio desnudo, no porque careciera de traje, sino por no haber
tenido tiempo para ponérselo. Él y otros como él, fueron sin duda los
que inspiraron la célebre frase de que antes he hecho mención. Sus
carnes \emph{sólo se vestían de gloria}. Dormía sin abrigo y comía menos
que un anacoreta, pues con dos pedazos de pan acompañados de un par de
mordiscos de cecina, dura como cuero, tenía bastante para un día. Era
hombre algo meditabundo, y cuando observaba los trabajos de la segunda
paralela, decía mirando a los franceses: \emph{gracias a Dios que se
acercan, ¡cuerno!\ldots{} ¡Cuerno!, esta gente le acaba a uno la
paciencia.}

---¿Qué prisa tiene Vd., tío Garcés?---le decíamos.

---¡Recuerno! Tengo que plantar los árboles otra vez antes que pase el
invierno ---contestaba,---y para el mes que entra quisiera volver a
levantar la casita.

En resumen, el tío Garcés, como el reducto, debía llevar un cartel en la
frente que dijera: \emph{Hombre inconquistable}.

Pero ¿quién viene allí, avanzando lentamente por la hondonada de la
Huerva, apoyándose en un grueso bastón, y seguido de un perrillo
travieso, que ladra a todos los transeúntes por pura fanfarronería y sin
intención de morderles? Es el padre fray Mateo del Busto, lector y
calificador de la orden de mínimos, capellán del segundo tercio de
voluntarios de Zaragoza, insigne varón a quien, a pesar de su
ancianidad, se vio durante el primer sitio en todos los puestos de
peligro, socorriendo heridos, auxiliando moribundos, llevando municiones
a los sanos y animando a todos con el acento de su dulce palabra.

Al entrar en el reducto, nos mostró una cesta grande y pesada que
trabajosamente cargaba, y en la cual traía algunas vituallas algo
mejores que las de nuestra ordinaria mesa.

---Estas tortas---dijo sentándose en el suelo y sacando uno por uno los
objetos que iba nombrando,---me las han dado en casa de la Excma. Sra.
condesa de Bureta, y esta en casa de D. Pedro Ric. Aquí tenéis también
un par de lonjas de jamón, que son de mi Convento, y se destinaban al
padre Loshoyos, que está muy enfermito del estómago; pero él,
renunciando a este regalo, me lo ha dado para traéroslo. ¿A ver qué os
parece esta botella de vino? ¿Cuánto darían por ella los gabachos que
tenemos enfrente?

Todos miramos hacia el campo. El perrillo saltando denodadamente a la
muralla, empezó a ladrar a las líneas francesas.

---También os traigo un par de libras de orejones, que se han conservado
en la despensa de nuestra casa. Íbamos a ponerlos en aguardiente; pero
primero que nadie sois vosotros, valientes muchachos. Tampoco me he
olvidado de ti, querido Pirli---añadió, volviéndose al chico de este
nombre,---y como estás casi desnudo y sin manta, te he traído un
magnífico abrigo. Mira este lío. Pues es un hábito viejo que tenía
guardado para darlo a un pobre; ahora te lo regalo para que cubras y
abrigues tus carnes. Es vestido impropio de un soldado; pero si el
hábito no hace al monje, tampoco el uniforme hace al militar. Póntelo y
estarás muy holgadamente con él.

El fraile dio a nuestro amigo su lío, y este se puso el hábito entre
risas y jácara de una y otra parte, y como conservaba aún, llevándolo
constantemente en la cabeza, el alto sombrero de piel que el día 31
había cogido en el campamento enemigo, hacía la figura más extraña que
puede imaginarse.

Poco después llegaron algunas mujeres también con cestas de provisiones.
La aparición del sexo femenino trasformó de súbito el aspecto del
reducto. No sé de dónde sacaron la guitarra; lo cierto es que la sacaron
de alguna parte; uno de los presentes empezó a rasguear primorosamente
los compases de la incomparable, de la divina, de la inmortal jota, y en
un momento se armó gran jaleo de baile. Pirli, cuya grotesca figura
empezaba en ingeniero francés y acababa en fraile español, era el más
exaltado de los bailarines, y no se quedaba atrás su pareja, una
muchacha graciosísima, vestida de serrana, y a quien desde el primer
momento oí que llamaban Manuela. Representaba veinte o veinte y dos
años, y era delgada, de tez pálida y fina. La agitación del baile
inflamó bien pronto su rostro, y por grados avivaba sus movimientos,
insensible al cansancio. Con los ojos medio cerrados, las mejillas
enrojecidas, agitando los brazos al compás de la grata cadencia,
sacudiendo con graciosa presteza sus faldas, cambiando de lugar con
ligerísimo paso, presentándosenos ora de frente, ora de espaldas,
Manuela nos tuvo encantados durante largo rato. Viendo su ardor
coreográfico, más se animaban el músico y los demás bailarines, y con el
entusiasmo de estos aumentábase el suyo, hasta que al fin, cortado el
aliento y rendida de fatiga, aflojó los brazos y cayó sentada en tierra
sin respiración y encendida como la grana.

Pirli se puso junto a ella y al punto formose un corrillo cuyo centro
era la cesta de provisiones.

---A ver qué nos traes, Manuelilla---dijo Pirli.---Si no fuera por ti y
el padre Busto, que está presente, nos moriríamos de hambre. Y si no
fuera por este poco de baile con que quitamos el mal gusto de las
\emph{tortas calientes} y de \emph{las señoras}, ¡qué sería de estos
pobres soldados!

---Os traigo lo que hay---repuso Manuela sacando las
provisiones.---Queda poco y si esto dura, comeréis ladrillos.

---Comeremos metralla amasada con harina negra---dijo
Pirli.---Manuelilla, ¿ya se te ha quitado el miedo a los tiros?

Al decir esto, tomó con presteza su fusil disparándolo al aire. La
muchacha dio un grito y sobresaltada huyó de nuestro grupo.

---No es nada, hija---dijo el fraile.---Las mujeres valientes no se
asustan del ruido de la pólvora, antes al contrario deben encontrar en
él tanto agrado como en el son de las castañuelas y bandurrias.

---Cuando oigo un tiro---dijo Manuela, acercándose llena de miedo,---no
me queda gota de sangre en las venas.

En aquel instante los franceses que sin duda querían probar la
artillería de su segunda paralela, dispararon un cañón y la bala vino a
rebotar contra la muralla del reducto, haciendo saltar en pedazos mil
los deleznables ladrillos.

Levantáronse todos a observar el campo enemigo; la serrana lanzó una
exclamación de terror, y el tío Garcés púsose a dar gritos desde una
tronera contra los franceses, prodigándoles los más insolentes vocablos
acompañados de mucho \emph{cuerno y recuerno}. El perrillo recorriendo
la cortina de un extremo a otro ladraba con exaltada furia.

---Manuela, echemos otra jota al son de esta música, y ¡viva la Virgen
del Pilar! ---exclamó Pirli saltando como un insensato.

Impulsada por la curiosidad, alzábase Manuela lentamente, alargando el
cuello para mirar el campo por encima de la muralla. Luego al extender
los ojos por la llanura, parecía disiparse poco a poco el miedo en su
espíritu pusilánime, y al fin la vimos observando la línea enemiga con
cierta serenidad y hasta con un poco de complacencia.

---Uno, dos, tres cañones---dijo contando las bocas de fuego que a lo
lejos se divisaban.---Vamos, chicos, no tengáis miedo. Eso no es nada
para vosotros.

Oyose hacia San José estrépito de fusilería, y en nuestro reducto sonó
el tambor, mandando tomar las armas. Del fuerte cercano había salido una
pequeña columna que se tiroteaba de lejos con los trabajadores
franceses. Algunos de estos corriéndose hacia su izquierda, parecían
próximos a ponerse al alcance de nuestros fuegos: corrimos todos a las
aspilleras, dispuestos a enviarles un poco de pedrisco, y sin esperar la
orden del jefe, algunos dispararon sus fusiles con gran algazara.
Huyeron en tanto por el puente y hacia la ciudad todas las mujeres,
excepto Manuela. ¿El miedo le impedía moverse? No: su miedo era inmenso
y temblaba, dando diente con diente, desfigurado el rostro por repentina
amarillez; pero una curiosidad irresistible la retenía en el reducto, y
fijaba los atónitos ojos en los tiradores y en el cañón que en aquel
instante iba a ser disparado.

---Manuela---le dijo Agustín.---¿No te vas? ¿No te causa temor esto que
estás mirando?

La serrana con la atención fija en aquel espectáculo, asombrada,
trémula, con los labios blancos y el pecho palpitante, ni se movía, ni
hablaba.

¡Manuelilla---dijo Pirli corriendo hacia ella,---toma mi fusil y
dispáralo!

Contra lo que esperábamos, Manuelilla no hizo movimiento alguno de
terror.

---Tómalo, prenda---añadió Pirli haciéndole tomar el arma;---pon el dedo
aquí, apunta afuera y tira. ¡Viva la segunda artillera Manuela Sancho y
la Virgen del Pilar!

La serrana tomó el arma, y a juzgar por su actitud y el estupor inmenso
revelado en su mirar, parecía que ella misma no se daba cuenta de su
acción. Pero alzando el arma con mano temblorosa, apuntó hacia el campo,
tiró del gatillo e hizo fuego.

Mil gritos y ardientes aplausos acogieron este disparo, y la serrana
soltó el fusil. Estaba radiante de satisfacción y el júbilo encendió de
nuevo sus mejillas.

---Ves: Ya has perdido el miedo---dijo el mínimo.---Si a estas cosas no
hay más que tomarlas el gusto. Lo mismo debieran hacer todas las
zaragozanas, y de ese modo la Agustina y Casta Álvarez no serían una
gloriosa excepción entre las de su sexo.

---Venga otro fusil---exclamó la serrana,---que quiero tirar otra vez.

---Se han marchado ya, prenda. ¿Te ha sabido a bueno?---dijo Pirli,
preparándose a hacer desaparecer algo de lo que contenían las
cestas.---Mañana, si quieres, estás convidada a un poco de \emph{torta
caliente}. Ea, sentémonos y a comer.

El fraile, llamando a su perrillo, le decía:

---Basta, hijo; no ladres tanto, ni lo tomes tan a pechos, que vas a
quedarte ronco. Guarda ese arrojo para mañana: por hoy, no hay en qué
emplearlo, pues, si no me engaño van a toda prisa a guarecerse detrás de
sus parapetos.

En efecto, la escaramuza de los de San José había concluido, y por el
momento no teníamos franceses a la vista. Un rato después sonó de nuevo
la guitarra, y regresando las mujeres, comenzaron los dulces vaivenes de
la jota, con Manuela Sancho y el gran Pirli en primera línea.

\hypertarget{x}{%
\chapter{X}\label{x}}

Cuando desperté al amanecer del día siguiente, vi a Montoria, que se
paseaba por la muralla.

---Creo que va a empezar el bombardeo---me dijo.---Se nota gran
movimiento en la línea enemiga.

---Empezarán por batir este reducto---indiqué yo, levantándome con
pereza.---¡Qué feo está el cielo, Agustín! El día amanece muy triste.

---Creo que atacarán por todas partes a la vez, pues tienen hecha su
segunda paralela. Ya sabes que Napoleón, hallándose en París, al saber
la resistencia de esta ciudad en el primer sitio, se puso furioso contra
Lefebvre Desnouettes porque había embestido la plaza por el Portillo y
la Aljafería. Luego pidió un plano de Zaragoza, se lo dieron e indicó
que la ciudad debía ser atacada por Santa Engracia.

---¿Por aquí? Pronto lo veremos. Mal día se nos prepara si se cumplen
las órdenes de Napoleón. Dime, ¿tienes por ahí algo que comer?

---No te lo enseñé antes, porque quise sorprenderte---me dijo
mostrándome un cesto, que servía de sepulcro a dos aves asadas fiambres,
con algunas confituras y conservas finas.

---¿Lo has traído anoche?\ldots{} Ya. ¿Cómo pudiste salir del reducto?

---Pedí licencia al jefe, y me la concedió por una hora. Mariquilla
tenía preparado este festín. Si el tío Candiola sabe que dos de las
gallinas de su corral han sido muertas y asadas para regalo de los
defensores de la ciudad, se lo llevarán los demonios. Comamos, pues,
Sr.~Araceli, y esperemos ese bombardeo\ldots{} ¡Eh! ¡Aquí está!\ldots{}
una bomba, otra, otra\ldots{}

Las ocho baterías que embocaban sus tiros contra San José y el reducto
del Pilar, empezaron a hacer fuego; ¡pero qué fuego! ¡Todo el mundo a
las troneras, o al pie del cañón! ¡Fuera almuerzos, fuera desayunos,
fuera melindres! Los aragoneses no se alimentan sino de gloria. El
fuerte inconquistable contestó al insolente sitiador con orgulloso
cañoneo, y bien pronto el gran aliento de la patria dilató nuestros
pechos. Las balas rasas rebotando en la muralla de ladrillo y en los
parapetos de tierra, destrozaban el reducto, cual si fuera un juguete
apedreado por un niño; las granadas cayendo entre nosotros reventaban
con estrépito, y las bombas pasando con pavorosa majestad por sobre
nuestras cabezas, iban a caer en las calles y en los techos de las
casas.

¡A la calle todo el mundo! No haya gente cobarde ni ociosa en la ciudad.
Los hombres a la muralla, las mujeres a los hospitales de sangre, los
chiquillos y los frailes a llevar municiones. No se haga caso de estas
terribles masas inflamadas que agujerean los techos, penetran en las
habitaciones, abren las puertas, horadan los pisos, bajan al sótano, y
al reventar desparraman las llamas del infierno en el hogar tranquilo,
sorprendiendo con la muerte al anciano inválido en su lecho y al niño en
su cuna. Nada de esto importa. A la calle todo el mundo y con tal que se
salve el honor, perezca la ciudad y la casa, y la iglesia, y el
Convento, y el hospital, y la hacienda, que son cosas terrenas. Los
zaragozanos, despreciando los bienes materiales como desprecian la vida,
viven con el espíritu en los infinitos espacios de lo ideal.

En los primeros momentos nos visitó el capitán general, con otras muchas
personas distinguidas, tales como D. Mariano Cereso\footnote{Se llamaba
  Cereso y no Cerezo, como en muchas historias se estampa, y aun en el
  letrero de la calle que en Zaragoza lleva su nombre.}, el cura Sas, el
general O'Neilly, San Genis y D. Pedro Ric. También estuvo allí el bravo
y generoso y campechano D. José de Montoria, que abrazó a su hijo,
diciéndole: «Hoy es día de vencer o morir. Nos veremos en el cielo.»
Tras de Montoria se nos presentó don Roque, el cual estaba hecho un
valiente, y como empleado en el servicio sanitario, desde antes que
existieran heridos había comenzado a desplegar de un modo febril su
actividad, y nos mostró un mediano montón de hilas. Varios frailes se
mezclaron asimismo entre los combatientes durante los primeros disparos,
exhortándonos con un furor místico, inspirado en el libro de los
Macabeos.

A un mismo tiempo y con igual furia atacaban los franceses el reducto
del Pilar y el fortín de San José. Este, aunque ofrecía un aspecto más
formidable había de resistir menos, quizás por presentar mayor blanco al
fuego enemigo. Pero allí estaba Renovales con los voluntarios de Huesca,
los voluntarios de Valencia, algunos guardias walonas y varios
individuos de milicias de Soria. El gran inconveniente de aquel fuerte
consistía en estar construido al amparo de un vasto edificio, que la
artillería enemiga convertía paulatinamente en ruinas; y desplomándose
de rato en rato pedazos de paredón, muchos defensores morían aplastados.
Nosotros estábamos mejor; sobre nuestras cabezas no teníamos más que
cielo, y si ningún techo nos guarecía de las bombas, tampoco se nos
echaban encima masas de piedra y ladrillo. Batían la muralla por el
frente y los costados, y era un dolor ver cómo aquella frágil masa se
desmoronaba, poniéndonos al descubierto. Sin embargo, después de cuatro
horas de fuego incesante con poderosa artillería apenas pudieron abrir
una brecha practicable.

Así pasó todo el día 10, sin ventaja alguna para los sitiadores por
nuestro lado, si bien hacia San José habían logrado acercarse y abrir
una brecha espantosa, lo cual unido al estado ruinoso del edificio
anunciaba la dolorosa necesidad de su rendición. Sin embargo, mientras
el fuerte no estuviese reducido a polvo y muertos o heridos sus
defensores había esperanza. Renováronse allí las tropas, porque los
batallones que trabajaban desde por la mañana estaban diezmados, y
cuando anocheció, después de abierta la brecha e intentado sin fruto un
asalto, aún se sostuvo Renovales sobre las ruinas empapadas en sangre,
entre montones de cadáveres y con la tercera parte tan sólo de su
artillería.

No interrumpió la noche el fuego, antes bien siguió con encarnizamiento
en los dos puntos. Nosotros habíamos tenido buen número de muertos y
muchos heridos. Estos eran al punto recogidos y llevados a la ciudad por
los frailes y las mujeres; pero aquellos aún prestaban el último
servicio con sus helados cuerpos, porque estoicamente los arrojábamos a
la brecha abierta, que luego se acababa de tapar con sacos de lana y
tierra.

Durante la noche no descansamos ni un solo momento, y la mañana del 11
nos vio poseídos del mismo frenesí, ya apuntando las piezas contra la
trinchera enemiga, ya acribillando a fusilazos a los pelotones que
venían a flanquearnos, sin abandonar ni un instante la operación de
tapar la brecha, que de hora en hora iba agrandando su horroroso espacio
vacío. Así nos sostuvimos toda la mañana, hasta el momento en que dieron
el asalto a San José, ya convertido en un montón de ruinas, y con gran
parte de su guarnición muerta. Aglomerando contra los dos puntos grandes
fuerzas, mientras caían sobre el Convento, dirigieron sobre nosotros un
atrevido movimiento; y fue que con objeto de hacer practicable la brecha
que nos habían abierto, avanzaron por el camino de Torrero con dos
cañones de batalla, protegidos por una columna de infantería.

En aquel instante nos consideramos perdidos: temblaron los endebles
muros, y los ladrillos mal pegados se desbarataban en mil pedazos.
Acudimos a la brecha que se abría y se abría cada vez más, y nos
abrasaron con un fuego espantoso, porque viendo que el reducto se
deshacía pedazo a pedazo, cobraron ánimo llegando al borde mismo del
foso. Era una locura tratar de tapar aquel hueco formidable; y hacerlo a
pecho descubierto era ofrecer víctimas sin fin al furioso enemigo.
Abalanzáronse muchos con sacos de lana y paletadas de tierra, y más de
la mitad quedaron yertos en el sitio. Cesó el fuego de cañón, porque ya
parecía innecesario; hubo un momento de pánico indefinible; se nos caían
los fusiles de las manos; nos vimos destrozados, deshechos, aniquilados
por aquella lluvia de disparos que parecían incendiar el aire, y nos
olvidamos del honor, de la muerte gloriosa, de la patria y de la Virgen
del Pilar, cuyo nombre decoraba la puerta del baluarte inconquistable.
La confusión más espantosa reinó en nuestras filas. Rebajado de
improviso el nivel moral de nuestras almas, todos los que no habíamos
caído, deseamos unánimemente la vida, y saltando por encima de los
heridos y pisoteando los cadáveres, huimos hacia el puente, abandonando
aquel horrible sepulcro antes que se cerrara, enterrándonos a todos.

En el puente nos agolpamos con pavor y desorden invencibles. Nada hay
más frenético que la cobardía: sus vilezas son tan vehementes como las
sublimidades del valor. Los jefes nos gritaban: «¡Atrás, canallas! ¡El
reducto del Pilar no se rinde!» Y al mismo tiempo sus sables azotaron de
plano nuestras viles espaldas. Nos revolvimos en el puente sin poder
avanzar, porque otras tropas venían a contenernos, y tropezamos unos con
otros, confundiendo la furia de nuestro miedo con el ímpetu de su
bravura.

---¡Atrás, canallas!---gritaban los jefes abofeteándonos.---¡A morir en
la brecha!

El reducto estaba vacío: no había en él más que muertos y heridos. De
repente vimos que entre el denso humo y el espeso polvo, y saltando
sobre los exánimes cuerpos, y los montones de tierra, y las ruinas, y
las cureñas rotas, y el material deshecho, avanzaba una figura impávida,
pálida, grandiosa, imagen de la serenidad trágica; era una mujer que se
había abierto paso entre nosotros, y penetrando en el recinto
abandonado, marchaba majestuosa ¡hasta la horrible brecha! Pirli, que
yacía en el suelo herido en una pierna, exclamó con terror:

---Manuela Sancho, ¿a dónde vas?

Todo esto pasó en mucho menos tiempo del que empleo en contarlo. Tras de
Manuela Sancho se lanzó uno, luego tres, luego muchos, y al fin todos
los demás, azuzados por los jefes que a sablazos nos llevaron otra vez
al puesto del deber. Ocurrió esta transformación portentosa, por un
simple impulso del corazón de cada uno, obedeciendo a sentimientos que
se comunicaban a todos sin que nadie supiera de qué misterioso foco
procedían. Ni sé por qué fuimos cobardes, ni sé por qué fuimos valientes
unos cuantos segundos después. Lo que sé es que movidos todos por una
fuerza extraordinaria, poderosísima, sobrehumana, nos lanzamos a la
lucha tras la heroica mujer, a punto que los franceses intentaban con
escalas el asalto; y sin que tampoco sepa decir la causa, nos sentimos
con centuplicadas energías, y aplastamos, arrojándolos en lo profundo
del foso, a aquellos hombres de algodón que antes nos parecieron de
acero. A tiros, a sablazos, con granadas de mano, a paletadas, a golpes,
a bayonetazos, murieron muchos de los nuestros para servir de defensa a
los demás con sus fríos cuerpos; defendimos el paso de la brecha, y los
franceses se retiraron, dejando mucha gente al pie de la muralla.
Volvieron a disparar los cañones, y el reducto inconquistable no cayó el
día 11 en poder de la Francia.

Cuando la tempestad de fuego se calmó, no nos conocíamos: estábamos
transfigurados, y algo nuevo y desconocido palpitaba en lo íntimo de
nuestras almas, dándonos una ferocidad inaudita. Al día siguiente decía
Palafox con elocuencia: «\emph{Las bombas, las granadas y las balas no
mudan el color de nuestros semblantes, ni toda la Francia lo
alteraría»}.

\hypertarget{xi}{%
\chapter{XI}\label{xi}}

El fuerte de San José se había rendido, mejor dicho, los franceses
entraron en él cuando la artillería lo hubo reducido a polvo, y cuando
yacían entre los escombros uno por uno todos sus defensores. Los
imperiales, al penetrar, encontraron inmenso número de cuerpos
destrozados, y montones de tierra y guijarros amasados con sangre. No
podían aún establecerse allí, porque eran flanqueados por la batería de
los Mártires y la del Jardín Botánico, y continuaron las operaciones de
zapa para apoderarse de estos dos puntos. Las fortificaciones que
conservábamos estaban tan destrozadas, que urgía una composición
general, y se dictaron órdenes terribles convocando a todos los
habitantes de Zaragoza para trabajar en ellas. La proclama dijo que
todos debían llevar el fusil en una mano y la azada en la otra.

El 12 y el 13 se trabajó sin descanso, disminuyendo bastante el fuego,
porque los sitiadores escarmentados, no querían arriesgarse en nuevos
golpes de mano, y comprendiendo que aquello era obra de paciencia y
estudio más que de arrojo, abrían despacio y con toda seguridad zanjas y
caminos cubiertos que les trajesen a la posesión del reducto, sin
pérdida de gente. Casi fue preciso hacer de nuevo las murallas, mejor
dicho, sustituirlas con sacos de tierra, operación en que además de toda
la tropa, se ocupaban muchos frailes, canónigos, magistrados de la
Audiencia, chicos y mujeres. La artillería estaba casi inservible, el
foso casi cegado, y era preciso continuar la defensa a tiro de fusil.
Así nos sostuvimos todo el 13 protegiendo los trabajos de recomposición,
padeciendo mucho y viendo que cada vez mermábamos en número, aunque
entraba gente nueva a cubrir las considerables bajas. El 14 la
artillería enemiga empezó a desbaratar de nuevo nuestra muralla de
sacos, abriéndonos brechas por el frente y los costados; mas no se
atrevían a intentar un nuevo asalto, contentándose con seguir abriendo
una zanja en tal dirección que no podíamos de modo alguno enfilarla con
nuestro fuego, ni con los de las baterías inmediatas.

El valeroso, el provocativo fuerte de tierra, iba a estar bien pronto
bajo los fuegos cubiertos de baterías cercanas que arrojarían a los
cuatro vientos el polvo de que estaba formado. En esta situación le era
forzoso rendirse más tarde o más temprano, pues se hallaba a merced de
los tiros del francés, como un barco a merced de las olas del Océano.
Flanqueado por caminos cubiertos y zig-zags, por cuyos huecos discurría
sin peligro un enemigo inteligente lleno de fuerza material y con todos
los recursos de la ciencia, el baluarte era como un hombre cercado por
un ejército. No teníamos cañones servibles ni podíamos traer otros
nuevos, porque las murallas no los hubieran resistido.

Nuestro único recurso era minar el reducto para volarlo en el momento en
que entraran en él los franceses, y destruir también el puente para
impedir que nos persiguieran. Así se hizo, y durante la noche del 14 al
15 trabajamos sin descanso en la mina, y pusimos los hornillos del
puente, esperando que los enemigos se echasen encima al día siguiente
por la mañana. Con todo, no fue así, porque, no atreviéndose a dar un
asalto sin tomar las precauciones y seguridades posibles, continuaron
sus trabajos de zapa hasta muy cerca del foso. En esta faena, nuestra
infatigable fusilería les hacía poco daño. Estábamos desesperados; sin
poder hacer nada, sin que la misma desesperación nos sirviera para la
defensa. Era una fuerza inútil como la cólera de un loco en su jaula.

Desclavamos también el tablón que decía \emph{Reducto inconquistable},
para llevarnos aquel testimonio de nuestra justificada jactancia, y al
anochecer fue abandonado el fuerte, quedando sólo cuarenta hombres para
custodiarlo hasta el fin y \emph{matar lo que se pudiera}, como decía
nuestro capitán, pues no debía perderse ninguna ocasión de hacer un par
de bajas al enemigo. Desde la torre del Pino presenciamos la retirada de
los cuarenta a eso de las ocho de la noche, después de haberla
emprendido a bayonetazos con los ocupadores y batiéndose en retirada con
bravura. La mina del interior del reducto hizo muy poco efecto; pero los
hornillos del puente desempeñaron tan bien su cometido, que el paso
quedó roto y el reducto aislado en la otra orilla de la Huerva.
Adquirido este sitio y San José, los franceses tenían el apoyo
suficiente para abrir su tercera paralela y batir cómodamente todo el
circuito de la ciudad.

Estábamos tristes, y un poco, un poquillo desanimados. Pero ¿qué
importaba un decaimiento momentáneo, si al día siguiente tuvimos una
fiesta divertidísima? Después de batirse uno como un frenético, no venía
mal un poco de holgorio y bullanga precisamente cuando faltaba tiempo
para enterrar los muchos muertos, y acomodar en las casas el inmenso
número de heridos. Verdad es que para todo había manos, gracias a Dios;
y el motivo de la general alegría fue que empezaron a circular noticias
estupendas sobre ejércitos españoles que venían a socorrernos, sobre
derrotas de los franceses en distintos puntos de la Península y otras
zarandajas. Agolpábase el pueblo en la plaza de la Seo, esperando a que
saliese la \emph{Gaceta}, y al fin salió a regocijar los ánimos y hacer
palpitar de esperanza todos los corazones. No sé si efectivamente
llegaron a Zaragoza tales noticias, o si las sacó de su cacumen el
redactor principal, que era D. Ignacio Asso: lo cierto es que en letras
de molde se nos dijo que Reding venía a socorrernos con un ejército de
sesenta mil hombres; que el marqués de Lazán, después de derrotar a la
canalla en el Norte de Cataluña, había entrado en Francia \emph{llevando
el espanto por todas partes}; que también venía en nuestro auxilio el
duque del Infantado; que entre Blake y la Romana habían derrotado a
Napoleón \emph{matándole veinte mil hombres}, inclusos Berthier, Ney y
Savary, y que a Cádiz habían llegado \emph{diez y seis millones de
duros}, enviados por los ingleses para gastos de guerra. ¿Qué tal? ¿Se
explicaba la \emph{Gaceta}?

A pesar de ser tantas y tan gordas, nos las tragamos, y allí fueron las
demostraciones de alegría, el repicar campanas, y el correr por las
calles cantando la jota con otros muchos excesos patrióticos que por lo
menos tenían la ventaja de proporcionarnos un poco de aquel refrigerio
espiritual que necesitábamos. No crean Vds. que por consideración a
nuestra alegría había cesado la lluvia de bombas. Muy lejos de eso,
aquellos condenados parecían querer mofarse de las noticias de nuestra
\emph{Gaceta}, repitiendo la dosis.

Sintiendo un deseo vivísimo de reírnos en sus barbas, corrimos a la
muralla, y allí las músicas de los regimientos tocaron con cierta
afectación provocativa, cantando todos en inmenso coro el famoso tema:

\small
\newlength\mlenb
\settowidth\mlenb{que no quiere ser francesa...}
\begin{center}
\parbox{\mlenb}{\quad \textit{La Virgen del Pilar dice                  \\
                Que no quiere ser francesa...}}                         \\
\end{center}
\normalsize

También ellos estaban para burlas, y arreciaron el fuego de tal modo,
que la ciudad recibió en menos de dos horas mayor número de proyectiles
que en el resto del día. Ya no había asilo seguro, ya no había un palmo
de suelo ni de techo libre de aquel satánico fuego. Huían las familias
de sus hogares, o se refugiaban en los sótanos; los heridos que
abundaban en las principales casas eran llevados a las iglesias,
buscando reposo bajo sus fuertes bóvedas: otros salían arrastrándose;
algunos más ágiles llevaban a cuestas sus propias camas. Los más se
acomodaban en el Pilar y después de ocupar todo el pavimento, tendíanse
en los altares y obstruían las capillas. A pesar de tantos infortunios
se consolaban con mirar a la Virgen, la cual sin cesar con el lenguaje
de sus brillantes ojos les estaba diciendo \emph{que no quería ser
francesa}.

\hypertarget{xii}{%
\chapter{XII}\label{xii}}

Mi batallón no tomó parte en las salidas de los días 22 y 24, ni en la
defensa del Molino de aceite y de las posiciones colocadas a espaldas de
San José, hechos gloriosos en que se perdió bastante gente, pero donde
se les sentó la mano con firmeza a los franceses. Y no era porque estos
se descuidaran en tomar precauciones, pues en la tercera paralela, desde
la embocadura de la Huerva hasta la puerta del Carmen, colocaron 50
cañones, los más de grueso calibre, dirigiendo sus bocas con mucho arte
contra los puntos más débiles. De todo esto nos reíamos o aparentábamos
reírnos, como lo prueba la vanagloriosa respuesta de Palafox al mariscal
Lannes (que desde el 22 se puso al frente del ejército sitiador), en la
cual le decía: «\emph{La conquista de esta ciudad hará mucho honor al
señor Mariscal si la ganase a cuerpo descubierto, no con bombas y
granadas que sólo aterran a los cobardes}.»

Por supuesto en cuanto pasaron algunos días se conoció que los refuerzos
esperados y los poderosos ejércitos que venían a libertarnos eran puro
humo de nuestras cabezas y principalmente de la del diarista que en
tales cosas se entretenía. No había tales auxilios, ni ejércitos de
ninguna clase andaban cerca para ayudarnos.

Yo comprendí bien pronto que lo publicado en la \emph{Gaceta} del 16 era
una filfa, y así lo dije a D. José de Montoria y a su mujer, los cuales
en su optimismo atribuyeron mi incredulidad a falta de sentido común. Yo
había ido con Agustín y otros amigos a la casa de mis protectores para
ayudarles en una tarea que les traía muy apurados, pues destruido por
las bombas parte del techo, y amenazada de ruina una pared maestra,
estaban mudándose a toda prisa. El hijo mayor de Montoria, herido en la
acción del Molino de aceite, se había albergado, con su mujer e hijo en
el sótano de una casa inmediata, y doña Leocadia no daba paz a los pies
y a las manos para ir y venir de un sitio a otro trayendo y llevando lo
que era menester.

---No puedo fiarme de nadie---me decía.---Mi genio es así. Aunque tengo
criados, no quedo contenta si no lo hago todo yo misma. ¿Qué tal se ha
portado mi hijo Agustín?

---Como quien es, señora---le contesté.---Es un valiente muchacho, y su
disposición para las armas es tan grande, que no me asombraría verle de
general dentro de un par de años.

---¡General ha dicho Vd.!---exclamó con sorpresa.---Mi hijo cantará misa
en cuanto se acabe el sitio, pues ya sabe Vd. que para eso le hemos
criado. Dios y la Virgen del Pilar le saquen en bien de esta guerra, que
lo demás irá por sus pasos contados. Los padres del Seminario me han
asegurado que veré a mi hijo con su mitra en la cabeza y su báculo en la
mano.

---Así será, señora; no lo pongo en duda. Pero al ver cómo maneja las
armas, no puede acostumbrarse uno a considerar que con aquella misma
mano que tira del gatillo ha de echar bendiciones.

---Verdad es, Sr.~de Araceli; y yo siempre he dicho que a la gente de
iglesia no le cae bien esto del gatillo, pero qué quiere Vd. Ahí tenemos
hechos unos guerreros que dan miedo a D. Santiago Sas, a D. Manuel
Lasartesa, al beneficiado de San Pablo D. Antonio La Casa, al teniente
cura de la parroquia de San Miguel de los Navarros, D. José Martínez, y
también a D. Vicente Casanova, que tiene fama de ser el primer teólogo
de Zaragoza. Pues los demás lo hacen, guerree también mi hijo, aunque
supongo que él estará rabiando por volver al Seminario y meterse en la
balumba de sus estudios. Y no crea usted\ldots{} últimamente estaba
estudiando en unos libros tan grandes, tan grandes que pesan dos
quintales. Válgame Dios con el chico. Yo me embobo cuando le oigo
recitar una cosa larga, muy larga, toda en latín por supuesto, y que
debe de ser algo de nuestro divino Señor Jesucristo y el amor que tiene
a su Iglesia, porque hay mucho de \emph{amorem y de formosa}, y
\emph{pulcherrima, inflammavit} y otras palabrillas por el estilo.

---Justamente---le respondí,---y se me figura que lo que recita es el
libro cuarto de una obra eclesiástica, que llaman la Eneida, que
escribió un tal Fray Virgilio de la orden de Predicadores, y en cuya
obra se habla mucho del amor que Jesucristo tiene a su Iglesia.

---Eso debe de ser---repuso doña Leocadia.---Ahora, Sr.~de Araceli,
veamos si me ayuda Vd. a bajar esta mesa.

---Con mil amores, señora mía, la llevaré yo solo---contesté cargando el
mueble, a punto que entraba D. José de Montoria echando porras y cuernos
por su bendita boca.

---¿Qué es esto, porra?---exclamó.---¡Los hombres ocupados en faenas de
mujer! Para mudar muebles y trastos no se le ha puesto a Vd. un fusil en
la mano, Sr.~de Araceli. Y tú, mujer, ¿para qué distraes de este modo a
los hombres que hacen falta en otro lado? Tú y las chicas ¡porra!, ¿no
podéis bajar los muebles? Sois de pasta de requesón. Mira, por la calle
abajo va la condesa de Bureta con un colchón a cuestas, mientras sus dos
doncellas trasportan un soldado herido en una camilla.

---Bueno---dijo doña Leocadia,---para eso no es menester tanto ruido.
Váyanse fuera, pues, los hombres. A la calle todo el mundo, y déjennos
solas. Afuera tú también, Agustín, hijo mío, y Dios te conserve sano en
medio de este infierno.

---Hay que trasportar veinte sacos de harina del Convento de Trinitarios
al almacén de la junta de abastos---dijo Montoria.---Vamos todos.

Y cuando llegamos a la calle, añadió:

---La mucha tropa que tenemos dentro de Zaragoza hará que pronto no
podamos dar sino media ración. Verdad es, amigos míos, que hay muchos
víveres escondidos, y aunque se ha mandado que todo el mundo declare lo
que tiene, muchos no hacen caso y están acaparando para vender a precios
fabulosos. ¡Mal pecado! Si les descubro y caen bajo mis manos, les haré
entender quién es Montoria, presidente de la junta de abastos.

Llegábamos al Mercado, cuando nos salió al encuentro el padre fray Mateo
del Busto, que venía muy fatigado, forzando su débil paso, y le
acompañaba otro fraile a quien nombraron el padre Luengo.

---¿Qué noticias nos traen sus paternidades?---les preguntó Montoria.

---Efectivamente, D. Juan Gallart tenía algunas arrobas de embutidos que
pone a disposición de la junta.

---Y D. Pedro Pizcueta, el tendero de la calle de las Moscas, entrega
generosamente sesenta sacos de lana y toda la harina y la sal de sus
almacenes ---añadió Luengo.

---Pero acabamos de librar con el tío Candiola---dijo el fraile,---una
batalla, que ni la de las Eras se le compara.

---Pues qué---preguntó D. José con asombro,---¿no ha entendido ese
miserable cicatero que le pagaremos su harina, ya que es el único de
todos los vecinos de Zaragoza que no ha dado ni un higo para el
abastecimiento del ejército?

---Váyale Vd. con esos sermones al tío Candiola---repuso Luengo.---Ha
dicho terminantemente que no volvamos por allá si no le llevamos ciento
y veinticuatro reales por cada costal de harina, de sesenta y ocho que
tiene en su almacén.

---¡Hay infamia igual!---exclamó Montoria soltando una serie de porras
que no copio por no cansar al lector.---¡Con que a ciento veinticuatro
reales! Es preciso hacer entender a ese avaro empedernido cuáles son los
deberes de un hijo de Zaragoza en estas circunstancias. El capitán
general me ha dado autoridad para apoderarme de los abastecimientos que
sean necesarios, pagando por ellos la cantidad establecida.

---¿Pues sabe Vd. lo que dice, Sr.~D. José de mis pecados?---indicó
Busto.---Pues dice que el que quiera harina que la pague. Dice que si la
ciudad no se puede defender que se rinda, y que él no tiene obligación
de dar nada para la guerra, porque él no es quien la ha traído.

---Corramos allá---dijo Montoria lleno de enojo, que dejaba traslucir en
el gesto, en la alterada voz, en el semblante demudado y sombrío.---No
es esta la primera vez que le pongo la mano encima a ese canalla,
lechuzo, chupador de sangre.

Yo iba detrás con Agustín, y observando a este, le vi pálido y con la
vista fija en el suelo. Quise hablarle; pero me hizo señas de que
callara, y seguimos esperando a ver en qué pararía aquello. Pronto nos
hallamos en la calle de Antón Trillo, y Montoria nos dijo:

---Muchachos, adelantaos, tocad a la puerta de ese insolente judío:
echadla abajo si no os abren, entrad, y decidle que baje al punto y
venga delante de mí, porque quiero hablarle. Si no quiere venir, traedle
de una oreja; pero cuidado que no os muerda, que es perro con rabia y
serpiente venenosa.

Cuando nos adelantamos miré de nuevo a Agustín, y le observé lívido y
tembloroso.

---Gabriel---me dijo en voz baja,---yo quiero huir\ldots{} yo quiero que
se abra la tierra y me trague. Mi padre me matará, pero yo no puedo
hacer lo que nos ha mandado.

---Ponte a mi lado y haz como que se te ha torcido un pie y no puedes
seguir---le dije.

Y acto continuo los otros compañeros y yo empezamos a dar porrazos en la
puerta. Asomose al punto la vieja por la ventana y nos dijo mil
insolencias; trascurrió un breve rato y después vimos que una mano muy
hermosa levantaba la cortina dejando ver momentáneamente una cara
inmutada y pálida, cuyos grandes y vivos ojos negros dirigieron miradas
de terror hacia la calle. Era en el momento en que mis compañeros y los
chiquillos que nos seguían, gritaban en pavoroso concierto:

---¡Que baje el tío Candiola, que baje ese perro Caifás!

Contra lo que creímos, Candiola obedeció, mas lo hizo creyendo
habérselas con el enjambre de muchachos vagabundos que solían darle
tales serenatas, y sin sospechar que el presidente de la junta de
abastos con dos vocales de los más autorizados, estaban allí para hablar
de un asunto de importancia. Pronto tuvo ocasión de dar en lo cierto,
porque al abrir la puerta, y en el momento de salir, corriendo hacia
nosotros con un palo en la mano, y centelleando de ira sus feos ojos,
encaró con Montoria, y se detuvo amedrentado.

---¡Ah!, es Vd. Sr.~de Montoria---dijo con muy mal talante.---Siendo
Vd., como es, individuo de la junta de seguridad, ya podría mandar
retirar a esta canalla que viene a hacer ruido en la puerta de la casa
de un vecino honrado.

---No soy de la junta de seguridad---declaró Montoria,---sino de la de
abastos, y por eso vengo en busca del señor Candiola y le hago bajar;
que no entro yo en esa casa oscura, llena de telarañas y de ratones.

---Los pobres---repuso Candiola con desabrimiento,---no podemos tener
palacios como el Sr.~D. José de Montoria, administrador de bienes del
común y por largo tiempo contratista de arbitrios.

---Debo mi fortuna al trabajo, no a la usura---exclamó Montoria.---Pero
acabemos, señor don Jerónimo; vengo por esa harina\ldots{} ya le habrán
enterado a Vd. estos dos buenos religiosos\ldots{}

---Sí: la vendo, la vendo---contestó Candiola con taimada
sonrisa;---pero ya no la puedo dar al precio que indicaron esos señores.
Es demasiado barato. No la doy menos de ciento sesenta y dos reales
costal de a cuatro arrobas.

---Yo no pido precio---dijo D. José conteniendo la indignación.

---La junta podrá disponer de lo suyo; pero en mi hacienda no manda
nadie más que yo---contestó el avaro,---y está dicho todo\ldots{} conque
cada uno a su casa, que yo me meto en la mía.

---Ven acá, harto de sangre---exclamó Montoria asiéndole del brazo y
obligándole a dar media vuelta con mucha presteza.---Ven acá, Candiola
de mil demonios; he dicho que vengo por la harina y no me iré sin ella.
El ejército defensor de Zaragoza no se ha de morir de hambre ¡reporra!,
y todos los vecinos han de contribuir a mantenerlo.

---¡A mantenerlo, a mantener el ejército!---dijo el avariento, rebosando
veneno.---¿Acaso yo lo he parido?

---¡Miserable tacaño! ¿No hay en tu alma negra y vacía ni tanto así de
sentimiento patrio?

---Yo no mantengo vagabundos. Pues qué, ¿teníamos necesidad de que los
franceses nos bombardearan, destruyendo la ciudad? ¡Maldita guerra! ¿Y
quieren que yo les dé de comer? Veneno les daría.

---¡Canalla, sabandijo, polilla de Zaragoza y deshonra del pueblo
español!---exclamó mi protector, amenazando con el puño la arrugada cara
del avaro.---Más quisiera condenarme, ¡cuerno!, quemándome por toda la
eternidad en las llamas del infierno, que ser lo que tú eres, que ser el
tío Candiola por espacio de un minuto. Conciencia más negra que la
noche, alma perversa, ¿no te avergüenzas de ser el único que en esta
ciudad ha negado sus recursos al ejército libertador de la patria? El
odio general que por esta vil conducta has merecido, ¿no pesa sobre ti
más que si te hubieran echado encima todas las peñas del Moncayo?

---Basta de músicas y déjenme en paz---repuso don Jerónimo dirigiéndose
a la puerta.

---Ven acá, reptil inmundo---gritó Montoria, deteniéndole.---Te he dicho
que no me voy sin la harina. Si no la das de grado como todo buen
español, la darás por fuerza, y te la pagaré a razón de cuarenta y ocho
reales costal, que es el precio que tenía antes del sitio.

---¡Cuarenta y ocho reales!---exclamó Candiola con expresión
rencorosa.---Mi pellejo daría por ese precio antes que la harina. La
compré yo más cara. ¡Maldita tropa! ¿Me mantienen ellos a mí, Sr.~de
Montoria?

---Dales gracias, execrable usurero, porque no han puesto fin a tu vida
inútil. La generosidad de este pueblo ¿no te llama la atención? En el
otro sitio y cuando pasábamos los mayores apuros por reunir dinero y
efectos, tu corazón de piedra permaneció insensible, y no se te pudo
arrancar ni una camisa vieja para cubrir la desnudez del pobre soldado,
ni un pedazo de pan para matar su hambre. Zaragoza no ha olvidado tus
infamias. ¿Recuerdas que después de la acción del 4 de Agosto se
repartieron los heridos por la ciudad, y a ti te tocaron dos, que no
lograron traspasar el umbral de esa puerta de la miseria? Yo me acuerdo
bien: en la noche del 4 llegaron a tu puerta, y con sus débiles manos
tocaron para que les abrieras. Sus ayes lastimosos no conmovían tu
corazón de corcho; salistes a la puerta, y golpeándoles con el pie les
lanzaste en medio de la calle, diciendo que tu casa no era un hospital.
Indigno hijo de Zaragoza, ¿dónde tienes el alma, dónde tienes la
conciencia? Pero tú no tienes alma ni eres hijo de Zaragoza, sino que
naciste de un mallorquín con sangre de judío.

Los ojos de Candiola echaban chispas; temblábale la quijada, y con sus
dedos convulsos apretaba en la mano derecha el palo que le servía de
bastón.

---Sí, tú tienes sangre de judío mallorquín; tú no eres hijo de esta
noble ciudad. Los lamentos de aquellos dos pobres heridos ¿no resuenan
todavía en tus orejas de murciélago? Uno de ellos, desangrado
rápidamente, murió en este mismo sitio en que estamos. El otro
arrastrándose pudo llegar hasta el mercado, donde nos contó lo ocurrido.
¡Infame espantajo! ¿No te asombrastes de que el pueblo zaragozano no te
despedazara en la mañana del 5? Candiola, Candiolilla, dame la harina y
tengamos la fiesta en paz.

---Montoria, Montorilla---repuso el otro,---con mi hacienda y mi trabajo
no engordan los vagabundos holgazanes. ¡Ya! ¡Háblenme a mí de caridad y
de generosidad y de interés por los pobres soldados! Los que tanto
hablan de esto son unos miserables gorrones que están comiendo a costa
de la cosa pública. La junta de abastos no se reirá de mí. ¡Como si no
supiéramos lo que significa toda esta música de los socorros para el
ejército! Montoria, Montorilla, algo se queda en casa, ¿no es verdad?
Buenas cochuras se harán en los hornos de algún patriota con la harina
que dan los sandios bobalicones que la junta conoce. ¡A cuarenta y ocho
reales! ¡Lindo precio! ¡Luego en las cuentas que se pasan al capitán
general se le ponen como compradas a sesenta, diciendo que la
\emph{Virgen del Pilar no quiere ser francesa!}

D. José de Montoria que ya estaba sofocado y nervioso, luego que oyó lo
anterior, perdió los estribos como vulgarmente se dice, y sin poder
contener el primer impulso de su indignación, fuese derecho hacia el tío
Candiola con apariencia de aporrearle la cara; mas este, que sin duda
con su hábil mirada estratégica preveía el movimiento y se había
preparado para rechazarlo, tomó rápidamente la ofensiva, arrojándose con
salto de gato sobre mi protector, y le echó ambas manos al cuello,
clavándole en él sus dedos huesosos y fuertes, mientras apretaba los
dientes con tanta violencia cual si tuviera entre ellos la persona
entera de su enemigo. Hubo una brevísima lucha, en que Montoria trabajó
por deshacerse de aquella zarpa felina que tan súbitamente le había
hecho presa, y en un instante viose que la fuerza nerviosa del avaro no
podía nada contra la energía muscular del patriota aragonés. Sacudido
con violencia por este, Candiola cayó al suelo como un cuerpo muerto.

Oímos un grito de mujer en la ventana alta, y luego el chasquido de la
celosía al cerrarse. En aquel momento de dramática ansiedad, busqué en
torno mío a Agustín; pero había desaparecido.

D. José de Montoria, frenético de ira pateaba con saña el cuerpo del
caído, diciéndole al mismo tiempo con voz atropellada y balbuciente:

---Vil ladronzuelo, que te has enriquecido con la sangre de los pobres,
¿te atreves a llamarme ladrón, a llamar ladrones a los vocales de la
junta de abastos? Con mil porras, yo te enseñaré a respetar a la gente
honrada y agradéceme que no te arranco esa miserable lenguaza para
echarla a los perros.

Todos los circunstantes estábamos mudos de terror. Al fin sacamos al
infeliz Candiola de debajo de los pies de su enemigo, y su primer
movimiento fue saltar de nuevo sobre él; pero Montoria se había
adelantado hacia la casa, gritando:

---Ea, muchachos. Entrad en el almacén y sacad los sacos de harina.
Pronto, despachemos pronto.

La mucha gente que se había reunido en la calle impidió al viejo
Candiola entrar en su casa. Rodeándole al punto los chiquillos que en
gran número de las cercanías habían acudido, tomáronle por su cuenta.
Unos le empujaban hacia adelante; otros hacia atrás; hacíanle trizas el
vestido, y los más tomando la ofensiva desde lejos, le arrojaban en
grandes masas el lodo de la calle. En tanto, a los que penetramos en el
piso bajo, que era el almacén, nos salió al encuentro una mujer, en
quien al punto reconocí a la hermosa Mariquilla, toda demudada,
temblorosa, vacilando a cada paso, sin poderse sostener, ni hablar,
porque el terror la paralizaba. Su miedo era inmenso y a todos nos dio
lástima cuando la vimos, incluso a Montoria.

---¿Es usted la hija del Sr.~Candiola?---dijo este sacando del bolsillo
un puñado de monedas, y haciendo una breve cuenta en la pared con un
pedazo de carbón que tomó del suelo.---Sesenta y ocho costales de
harina, a cuarenta y ocho reales son tres mil doscientos sesenta y
cuatro. No valen ni la mitad, y me dan mucho olor a húmedo. Tome Vd.,
niña; aquí está la cantidad justa.

María Candiola no hizo movimiento alguno para tomar el dinero, y
Montoria lo depositó sobre un cajón, repitiendo:

---Ahí está.

Entonces la muchacha con brusco y enérgico movimiento que parecía, y lo
era ciertamente, inspiración de su dignidad ofendida, tomó las monedas
de oro, de plata y de cobre, y las arrojó a la cara de Montoria, como
quien apedrea. Desparramose el dinero por el suelo y en el quicio de la
puerta, sin que se haya podido averiguar en lo sucesivo dónde fue a
parar.

Inmediatamente después, la Candiola, sin decirnos nada, salió a la
calle, buscando con los ojos a su padre entre el apiñado gentío, y al
fin, ayudada de algunos mozos que no sabían ver con indiferencia la
desgracia de una mujer, rescató al anciano del cautiverio infame en que
los muchachos lo tenían.

Entraron padre e hija por el portalón de la huerta, cuando empezábamos a
sacar la harina.

\hypertarget{xiii}{%
\chapter{XIII}\label{xiii}}

Concluida la conducción, busqué a Agustín; pero no le encontraba en
ninguna parte, ni en casa de su padre, ni en el almacén de la junta de
abastos, ni en el Coso, ni en Santa Engracia. Al fin hallele a la caída
de la tarde en el molino de pólvora, hacia San Juan de los Panetes. He
olvidado decir que los zaragozanos, atentos a todo, habían improvisado
un taller donde se elaboraban diariamente de nueve a diez quintales de
pólvora. Ayudando a los operarios que ponían en sacos y en barriles la
cantidad fabricada en el día, vi a Agustín de Montoria trabajando con
actividad febril.

---¿Ves este enorme montón de pólvora?---me dijo cuando me acerqué a
él.---¿Ves aquellos sacos y aquellos barriles todos llenos de la misma
materia? Pues aún me parece poco, Gabriel.

---No sé lo que quieres decir.

---Digo que si esta inmensa cantidad de pólvora fuera del tamaño de
Zaragoza me gustaría aún más. Sí, y en tal caso quisiera yo ser el único
habitante de esta gran ciudad. ¡Qué placer! Mira, Gabriel; si así fuera,
yo mismo le pegaría fuego, volaría hasta las nubes escupido por la
horrorosa erupción, como la piedrecilla que lanza el cráter del volcán a
cien leguas de distancia. Subiría al quinto cielo; y de mis miembros
despedazados al caer después esparcidos en diferentes puntos no quedaría
memoria. La muerte, Gabriel, la muerte es lo que deseo. Pero yo quiero
una muerte\ldots{} no sé cómo explicártelo. Mi desesperación es tan
grande, que morir de un balazo, morir de una estocada no me satisface.
Quiero estallar y difundirme por los espacios en mil inflamadas
partículas; quiero sentirme en el seno de una nube flamígera y que mi
espíritu saboree, aunque sólo sea por un instante de inconmensurable
pequeñez, las delicias de ver reducida a polvo de fuego la carne
miserable. Gabriel, estoy desesperado. ¿Ves toda esta pólvora? Pues
supón dentro de mi pecho todas las llamas que pueden salir de
aquí\ldots{} ¿La viste cuando salió a recoger a su padre? ¿Viste cuando
arrojó las monedas\ldots? Yo estaba en la esquina observándolo todo.
María no sabe que aquel hombre que maltrató a su padre es el mío. Viste
cómo los chicos arrojaban lodo al pobre Candiola? Yo reconozco que
Candiola es un miserable; pero ella, ¿qué culpa tiene? Ella y yo, ¿qué
culpa tenemos? Nada, Gabriel, mi corazón destrozado anhela mil muertes;
yo no puedo vivir; yo correré al sitio de mayor peligro y me arrojaré a
buscar el fuego de los franceses, porque después de lo que he visto hoy,
yo y la tierra en que habito somos incompatibles.

Le saqué de allí llevándole a la muralla, y tomamos parte en las obras
de fortificación que se estaban haciendo en las Tenerías, el punto más
débil de la ciudad después de la pérdida de San José y de Santa
Engracia. Ya he dicho que desde la embocadura de la Huerva hasta San
José había 50 bocas de fuego. Contra esta formidable línea de ataque
¿qué valía nuestro circuito fortificado?

El arrabal de las Tenerías se extiende al Oriente de la ciudad, entre la
Huerva y el recinto antiguo perfectamente deslindado aún por la gran vía
que se llama el Coso. Componíase a principios del siglo el caserío de
edificios endebles, casi todos habitados por labradores y artesanos, y
las construcciones religiosas no tenían allí la suntuosidad de otros
monumentos de Zaragoza. La planta general de este barrio es
aproximadamente un segmento de círculo, cuyo arco da al campo y cuya
cuerda le une al resto de la ciudad, desde Puerta Quemada a la subida
del Sepulcro. Corrían desde esta línea hacia la circunferencia varias
calles, unas interrumpidas como las de Añón, Alcover y las Arcadas, y
otras prolongadas como las de Palomar y San Agustín. Con estas se
enlazaban sin plan ni concierto ni simetría alguna, estrechas vías como
la calle de la Diezma, Barrio Verde, de los Clavos y de Pabostre.
Algunas de estas se hallaban determinadas no por hileras de casas, sino
por largas tapias, y a veces faltando una cosa y otra, las calles se
resolvían en informes plazuelas, mejor dicho, corrales o patios donde no
había nada. Digo mal, porque en los días a que me refiero, los escombros
ocasionados por el primer sitio sirvieron para alzar baterías y
barricadas en los puntos donde las casas no ofrecían defensa natural.
Cerca del pretil del Ebro, existían algunos trozos de muralla antigua,
con varios cubos de mampostería, que algunos suponen hechos por manos de
gente romana, y otros juzgan obra de los árabes. En mi tiempo (no sé
cómo estará actualmente) estos trozos de muralla aparecían empotrados en
las manzanas de casas, mejor dicho, las casas estaban empotradas en
ellos, buscando apoyo en los recodos y ángulos de aquella obra secular,
ennegrecida, mas no quebrantada, por el paso de tantos siglos. Así, lo
nuevo se había edificado sobre y entre los restos de lo antiguo en
confuso amasijo, como la gente española se desarrolló y crió sobre
despojos de otras gentes con mezcladas sangres, hasta constituirse como
hoy lo está.

El aspecto general del barrio de las Tenerías traía a la imaginación,
acompañados de cierta idealidad risueña, los recuerdos de la dominación
arábiga. La abundancia del ladrillo, los largos aleros, el ningún orden
de las fachadas, las ventanuchas con celosías, la completa anarquía
arquitectural, aquello de no saberse dónde acababa una casa y empezaba
otra; la imposibilidad de distinguir si esta tenía dos pisos o tres, si
el tejado de aquella servía de apoyo a las paredes de las de más allá;
las calles que a lo mejor acababan en un corral sin salida, los arcos
que daban entrada a una plazuela, todo me recordaba lo que en otro
pueblo de España, de allí muy distante, había visto.

Pues bien: esta amalgama de casas que os he descrito muy a la ligera,
este arrabal fabricado por varias generaciones de labriegos y
curtidores, según el capricho de cada uno y sin orden ni armonía, estaba
preparado para la defensa, o se preparaba en los días 24 y 25 de Enero,
una vez que se advirtió la gran pompa de fuerzas ofensivas que desplegó
el francés por aquella parte. Y he de advertir que todas las familias
habitadoras de las casas del arrabal, procedían a ejecutar obras, según
su propio instinto estratégico, y allí había ingenieros militares con
faldas, que dieron muestras de un profundo saber de guerra al tabicar
ciertos huecos y abrir otros al fuego y a la luz. Los muros de Levante
estaban en toda su extensión aspillerados. Los cubos de la muralla
\emph{cesaraugustana}, hechos contra las flechas y las piedras de honda,
sostenían cañones.

Si la zona de acción de alguna de estas piezas era estrechada por
cualquier tejado colindante, azotea o casa entera, al punto se quitaba
el obstáculo. Muchos pasos habían sido obstruidos, y dos de los
edificios religiosos del arrabal, San Agustín y las Mónicas, eran
verdaderas fortalezas. La tapia había sido reedificada y reforzada; las
baterías se enlazaban unas con otras, y nuestros ingenieros habían
calculado hábilmente las posiciones y el alcance de las obras enemigas
para acomodar a ellas las defensivas. Dos puntos avanzados tenía la
línea, y eran el molino de Goicoechea y una casa, que por pertenecer a
un D. Victoriano González, ha quedado en la historia con el nombre de
\emph{Casa de González}. Recorriendo dicha línea desde Puerta Quemada,
se encontraba, primero, la batería de Palafox, luego, el Molino de la
ciudad; luego las eras de San Agustín, en seguida el molino de
Goicoechea, colocado fuera del recinto, después la tapia de la huerta de
las Mónicas, y a continuación, las de San Agustín; más adelante una gran
batería y la casa de González. Esto es todo lo que recuerdo de las
Tenerías. Había por allí un sitio que llamaban el Sepulcro, por la
proximidad de una iglesia de este nombre. Al arrabal entero, mejor que a
una parte de él, cuadraba entonces el nombre de \emph{sepulcro}. Y no os
digo más por no cansaros con estas menudencias descriptivas, que en
rigor son innecesarias para quien conoce aquellos gloriosísimos lugares,
e insuficientes para el que no ha podido visitarlos.

\hypertarget{xiv}{%
\chapter{XIV}\label{xiv}}

Agustín de Montoria y yo hicimos la guardia con nuestro batallón en el
Molino de la ciudad, hasta después de anochecido, hora en que nos
relevaron los voluntarios de Huesca, y se nos concedió toda la noche
para estar fuera de las filas. Mas no se crea que en estas horas de
descanso estábamos mano sobre mano, pues cuando concluía el servicio
militar empezaba otro no menos penoso en el interior de la ciudad, ya
conduciendo heridos a la Seo y al Pilar, ya desalojando casas
incendiadas o bien llevando material a los señores canónigos, frailes y
magistrados de la audiencia, que hacían cartuchos en San Juan de los
Panetes.

Pasábamos Montoria y yo por la calle de Pabostre. Yo iba comiéndome con
mucha gana un mendrugo de pan. Mi amigo, taciturno y sombrío, regalaba
el suyo a los perros que encontrábamos al paso, y aunque hice esfuerzos
de imaginación para alegrar un poco su ánimo contristado, él insensible
a todo, contestaba con tétricas expresiones a mi festivo charlar. Al
llegar al Coso, me dijo:

---Dan las diez en el reloj de la Torre Nueva. Gabriel, ¿sabes que
quiero ir allá esta noche?

---Esta noche no puede ser. Esconde entre ceniza la llama del amor
mientras atraviesan el aire esos otros corazones inflamados que llaman
bombas y que vienen a reventar dentro de las casas, matando medio
pueblo.

En efecto: el bombardeo, que no había cesado durante todo el día,
continuaba en la noche, aunque un poco menos recio; y de vez en cuando
caían algunos proyectiles, aumentando las víctimas que ya en gran número
poblaban la ciudad.

---Iré allá esta noche---me contestó.---¿Me vería Mariquilla entre el
gentío que tocó a las puertas de su casa? ¿Me confundiría con los que
maltrataron a su padre?

---No lo creo: esa niña sabrá distinguir a las personas formales. Ya
averiguarás eso más adelante, y ahora no está el horno para bollos ¿Ves?
De aquella casa piden socorro y por aquí van unas pobres mujeres. Mira,
una de ellas no se puede arrastrar y se arroja en el suelo. Es posible
que la señorita doña Mariquilla Candiola ande también socorriendo
heridos en San Pablo o en el Pilar.

---No lo creo.

---O quizá esté en la cartuchería.

---Tampoco lo creo. Estará en su casa y allá quiero ir, Gabriel; ve tú
al transporte de heridos, a la pólvora o a donde quieras, que yo voy
allá.

Diciendo esto, se nos presentó Pirli, con su hábito de fraile, ya en mil
partes agujereado, y el morrión francés tan lleno de abolladuras y
desperfectos en el pelo, chapa y plumero, que el héroe, portador de
tales prendas, más que soldado parecía una figura de Carnaval.

---¿Van Vds. al acarreo de heridos?---nos dijo.---Ahora se nos murieron
dos que llevábamos a San Pablo. Allá quieren gente para abrir la zanja
en que van a enterrar los muertos de ayer; pero yo he trabajado
bastante, y voy a descabezar un sueño en casa de Manuela Sancho. Antes
bailaremos un poco: ¿queréis venir?

---No: vamos a San Pablo---contesté,---y enterraremos muertos, pues todo
es trabajar.

---Dicen que los muchos difuntos envenenan el aire y que por eso hay
tanta gente con calenturas, las cuales despachan para el otro barrio más
pronto que los heridos. Yo más quiero el \emph{pastel caliente} que la
epidemia, y una \emph{señora} no me da miedo; pero el frío y la
calentura, sí. Conque ¿vais a enterrar muertos?

---Sí---dijo Agustín.---Enterremos muertos.

---En San Pablo hay lo menos cuarenta, todos puestos en una
capilla---añadió Pirli,---y al paso que vamos, pronto seremos más los
muertos que los vivos. ¿Queréis divertiros? Pues no vayáis a abrir la
zanja, sino a la cartuchería, donde hay unas mozas\ldots{} Todas las
muchachas principales del pueblo están allí y de cuando en cuando echan
algo de canto y bailoteo para alegrar las almas.

---Pero allí no hacemos falta. ¿Está también Manuela Sancho?

---No: todas son señoritas principales, que han sido llamadas por la
junta de seguridad. Y también hay muchas en los hospitales. Ellas se
brindan a este servicio, y la que falta es mirada con tan malos ojos,
que no encontrara novio con quien casarse en todo este año ni en el que
viene.

Sentimos detrás de nosotros pasos precipitados, y volviéndonos, vimos
mucha gente, entre cuyas voces reconocimos la de D. José de Montoria, el
cual al vernos, muy encolerizado nos dijo:

---¿Qué hacéis, papanatas? Tres hombres sanos y rollizos se están aquí
mano sobre mano, cuando hace tanta falta gente para el trabajo. Vamos,
largo de aquí. Adelante, caballeritos. Veis aquellos dos palos que hay
junto a la subida del Trenque, con una viga cruzada encima, de la que
penden seis dogales? ¿Veis la horca que se ha puesto esta tarde para los
traidores? Pues es también para los holgazanes. A trabajar, o a
puñetazos os enseñaré a mover el cuerpo.

Seguimos tras ellos, y pasamos junto a la horca, cuyos seis dogales se
balanceaban majestuosamente a impulso del viento, esperando gargantas de
traidores o cobardes.

Montoria, cogiendo a su hijo por un brazo, mostrole con enérgico ademán
el horrible aparato, y le habló así:

---Aquí tienes lo que hemos puesto esta tarde; ¡mira qué buen regalo
para los que no cumplen con su deber! Adelante: yo que soy viejo, no me
canso jamás, y vosotros jóvenes llenos de salud, parecéis de manteca. Ya
se acabó aquella gente invencible del primer sitio. Señores, nosotros
los viejos demos ejemplo a estos pisaverdes, que desde que llevan siete
días sin comer, se quejan y empiezan a pedir caldo. Caldo de pólvora os
daría yo y una garbanzada de cañón de fusil, ¡cobardes! Ea, adelante,
que hace falta enterrar muertos y llevar cartuchos a las murallas.

---Y asistir a los enfermos de esta condenada epidemia que se está
desarrollando ---dijo uno de los que acompañaban a Montoria.

---Yo no sé qué pensar de esto que llaman epidemia los facultativos, y
que yo llamo miedo, sí señores, puro miedo---añadió D. José;---porque
eso de quedarse uno frío, y entrarle calambres y calentura y ponerse
verde y morirse, ¿qué es si no efecto del miedo? Ya se acabó la gente
templada, sí señores, ¡qué gente aquella la del primer sitio! Ahora en
cuanto hacen fuego nutrido y lo reciben por espacio de diez horas ¡una
friolera!, ya se caen de fatiga y dicen que no pueden más. Hay hombre
que sólo por perder pierna y media se acobarda y empieza a llamar a
gritos a los santos Mártires diciendo que lo lleven a la cama. ¡Nada,
cobardía y pura cobardía! Como que hoy se retiraron de la batería de
Palafox varios soldados, entre los cuales había muchos que conservaban
un brazo sano y mondo. Y luego pedían caldo\ldots{} ¡Que se chupen su
propia sangre, que es el mejor caldo del mundo! ¡Cuando digo que se
acabó la gente de pecho, aquella gente, porra, mil porras!

---Mañana atacarán los franceses las Tenerías---dijo otro.---Si resultan
muchos heridos, no sé dónde los vamos a colocar.

---¡Heridos!---exclamó Montoria.---Aquí no se quieren heridos. Los
muertos no estorban, porque se hace con ellos un montón y\ldots{} pero
los heridos\ldots{} Como la gente no tiene ya aquel arrojo, pues\ldots{}
apuesto a que defenderán las posiciones mientras no se vean reducidos a
la décima parte; pero las abandonarán desde que encima de cada uno se
echen un par de docenas de franceses\ldots{} ¡Qué debilidad! En fin, sea
lo que Dios quiera, y pues hay heridos y enfermos, asistámoslos. ¿Qué
tal? ¿Se ha recogido hoy mucha gallina?

---Como unas doscientas, de las cuales más de la mitad son de donativo,
y las demás se han pagado a seis reales y medio. Algunos no las quieren
dar.

---Bueno. ¡Que un hombre como yo se ocupe de gallinas en estos
días!\ldots{} Han dicho Vds. que algunos no las querían dar, ¿eh? El
señor capitán general me ha autorizado para imponer multas a los que no
contribuyan a la defensa, y sin ruido ni violencia arreglaremos a los
tibios y a los traidores\ldots{} Alto, señores. Una bomba cae por las
inmediaciones de la Torre Nueva. ¿Veis? ¿Oís? ¡Qué horroroso estrépito!
Apuesto a que la divina Providencia, más que los morteros franceses, la
ha dirigido contra el hogar de ese judío empedernido y sin alma que ve
con indiferencia y hasta con desprecio las desgracias de sus convecinos.
Corre la gente hacia allá: parece que arde una casa o que se ha
desplomado\ldots{} No, no corráis, infelices; dejadla que arda, dejadla
que caiga al suelo en mil pedazos. Es la casa del tío Candiola, que no
daría una peseta por salvar al género humano de un nuevo diluvio\ldots{}
Eh, Agustín, ¿dónde vas? ¿Tú también corres hacia allá? Ven acá, y
sígueme, que hacemos más falta en otra parte.

Íbamos por junto a la Escuela Pía. Agustín, impulsado sin duda por un
movimiento de su corazón, tomó a toda prisa la dirección de la plazuela
de San Felipe, siguiendo a la mucha gente que corría hacia este sitio;
pero detenido enérgicamente por su padre, continuó mal de su grado en
nuestra compañía. Algo ardía indudablemente cerca de la Torre Nueva, y
en esta los preciosos arabescos y las facetas de los ladrillos brillaron
enrojecidos por la cercana llama. Aquel monumento elegante, aunque cojo,
descollaba en la negra noche, vestido de púrpura, y al mismo tiempo su
colosal campana lanzaba al aire prolongados lamentos.

Llegamos a San Pablo.

---Ea, muchachos, haraganes---nos dijo D. José,---ayudad a los que abren
esta zanja. Que sea holgadita, crecederita; es un traje con que se van a
vestir cuarenta cuerpos.

Y emprendimos el trabajo, sacando tierra de la zanja que se abría en el
patio de la iglesia. Agustín cavaba como yo, y a cada instante volvía
sus ojos a la Torre Nueva.

---Es un incendio terrible---me dijo.---Mira, parece que se extingue un
poco, Gabriel; yo me quiero arrojar en esta gran fosa que estamos
abriendo.

---No haya prisa---le respondí,---que tal vez mañana nos echen en ella
sin que lo pidamos. Con que dejarse de tonterías y a trabajar.

---¿No ves? Creo que se extingue el fuego.

---Sí: se habrá quemado toda la casa. El tío Candiola habrase encerrado
en el sótano con su dinero y allí no llegará el fuego.

---Gabriel, voy un momento allá; quiero ver si ha sido su casa. Si sale
mi padre de la iglesia, le dirás que\ldots{} le dirás que vuelvo en
seguida.

La repentina salida de D. José de Montoria impidió a Agustín la fuga que
proyectaba, y los dos continuamos cavando la gran sepultura. Comenzaron
a sacar cuerpos, y los heridos o enfermos que eran traídos a cada
instante veían el cómodo lecho que se les estaba preparando, quizás para
el día siguiente. Al fin se creyó que la zanja era bastante honda y nos
mandaron suspender la excavación. Acto continuo fueron traídos uno a uno
los cadáveres y arrojados en su gran sepultura, mientras algunos
clérigos, puestos de rodillas y rodeados de mujeres piadosas, recitaban
lúgubres responsos. Cayeron dentro todos; y no faltaba sino echar tierra
encima. D. José Montoria, con la cabeza descubierta y rezando en voz
alta un Padre Nuestro, echó el primer puñado, y luego nuestras palas y
azadas empezaron a cubrir la tumba a toda prisa. Concluida nuestra
operación, todos nos pusimos de rodillas y rezamos en voz baja. Agustín
de Montoria me decía al oído:

---Iremos ahora\ldots{} mi padre se marchará. Le dices que hemos quedado
en relevar a dos compañeros que tienen un enfermo en su familia y
quieren pasar a verle. Díselo, por Dios; yo no me atrevo\ldots{} y en
seguida iremos allá.

\hypertarget{xv}{%
\chapter{XV}\label{xv}}

Y engañamos al viejo y fuimos, ya muy avanzada la noche, porque la
inhumación que acabo de mencionar duró más de tres horas. La luz del
incendio por aquella parte había dejado de verse; la masa de la torre
perdíase en la oscuridad de la noche y su gran campana no sonaba sino de
tarde en tarde para anunciar la salida de una bomba. Pronto llegamos a
la plazuela de San Felipe, y al observar que humeaba el techo de una
casa cercana en la calle del Temple, comprendimos que no fue la del tío
Candiola sino aquella, la que tres horas antes habían invadido las
llamas.

---Dios la ha preservado---dijo Agustín con mucha alegría,---si la
ruindad del padre atrae sobre aquel techo la cólera divina, las virtudes
y la inocencia de Mariquilla la detienen. Vamos allá.

En la plazuela de San Felipe había alguna gente; pero la calle de Antón
Trillo estaba desierta. Nos detuvimos junto a la tapia de la huerta y
pusimos atento el oído. Todo estaba tan en silencio, que la casa parecía
abandonada. ¿Lo estaría realmente? Aunque aquel barrio era de los menos
castigados por el bombardeo, muchas familias le habían desalojado, o
vivían refugiadas en los sótanos.

---Si entro---me dijo Agustín,---tú entrarás conmigo. Después de la
escena de hoy, temo que don Jerónimo, suspicaz y medroso como buen
avaro, esté alerta toda la noche y ronde la huerta, creyendo que vuelven
a quitarle su hacienda.

---En ese caso---le respondí,---más vale no entrar, porque además del
peligro que trae el caer en manos de ese vestiglo, habrá gran escándalo,
y mañana todos los habitantes de Zaragoza sabrían que el hijo de D. José
de Montoria, el joven destinado a encajarse una mitra en la cabeza, anda
en malos pasos con la hija del tío Candiola.

Pero esto y algo más que le dije era predicar en desierto, y así, sin
atender razones, insistiendo en que yo le siguiera, hizo la señal
amorosa, aguardando con la mayor ansiedad que fuera contestada.
Transcurrió algún tiempo, y al cabo, después de mucho mirar y remirar
desde la acera de enfrente, percibimos luz en la ventana alta. Sentimos
luego descorrer muy quedamente el cerrojo del portalón, y este se abrió
sin rechinar, pues sin duda el amor había tenido la precaución de
engrasar sus viejos goznes. Los dos entramos, topando de manos a boca,
no con la deslumbradora hermosura de una perfumada y voluptuosa
doncella, sino con una avinagrada cara, en la que al punto reconocí a
doña Guedita.

---Vaya unas horas de venir acá---dijo gruñendo,---y viene con otro.
Caballeritos, hagan Vds. el favor de no meter ruido. Anden sobre las
puntas de los pies y cuiden de no tropezar ni con una hoja seca, que el
señor me parece que está despierto.

Esto nos lo dijo en voz tan baja que apenas lo entendimos, y luego
marchó adelante haciendo señas de que la siguiéramos y poniendo el dedo
en los labios para intimarnos un silencio absoluto. La huerta era
pequeña; pronto le dimos fin, tropezando con una escalerilla de piedra
que conducía a la entrada de la casa, y no habíamos subido seis
escalones cuando nos salió al encuentro una esbelta figura, arrebujada
en una manta, capa o cabriolé. Era Mariquilla. Su primer ademán fue
imponernos silencio, y luego miró con inquietud una ventana lateral que
también caía a la huerta. Después mostró sorpresa al ver que Agustín iba
acompañado; pero este supo tranquilizarla, diciendo:

---Es Gabriel, mi amigo, mi mejor, mi único amigo, de quien me has oído
hablar tantas veces.

---Habla más bajo---dijo María.---Mi padre salió hace poco de su cuarto
con una linterna, y rondó toda la casa y la huerta. Me parece que no
duerme aún. La noche está oscura. Ocultémonos en la sombra del ciprés y
hablemos en voz muy baja.

La escalera de piedra conducía a una especie de corredor o balcón con
antepecho de madera. En el extremo de este corredor un ciprés
corpulento, plantado en la huerta, proyectaba gran masa de sombra,
formando allí una especie de refugio contra la claridad de la luna. Las
ramas desnudas del olmo se extendían sin sombrear por otro lado, y
garabateaban con mil rayas el piso del corredor, la pared de la casa y
nuestros cuerpos. Al amparo de la sombra del ciprés sentose Mariquilla
en la única silla que allí había; púsose Montoria en el suelo y junto a
ella, apoyando las manos en sus rodillas, y yo senteme también sobre el
piso, no lejos de la hermosa pareja. Era la noche, como de Enero,
serena, seca y fría; quizás los dos amantes, caldeados en el amoroso
rescoldo de sus corazones, no sentían la baja temperatura; pero yo,
criatura ajena a sus incendios, me envolví en mi capote para
resguardarme de la frialdad de los ladrillos. La tía Guedita había
desaparecido. Mariquilla entabló la conversación abordando desde luego,
el punto difícil.

---Esta mañana te vi en la calle. Cuando sentimos Guedita y yo el gran
ruido de la mucha gente que se agolpaba en nuestra puerta, me asomé a la
ventana y te vi en la acera de enfrente.

---Es verdad---respondió Montoria con turbación.---Allá fui; pero tuve
que marcharme al instante porque se me acababa la licencia.

---¿No viste cómo aquellos bárbaros atropellaron a mi padre?---dijo
Mariquilla conmovida.---Cuando aquel hombre cruel le castigó, miré a
todos lados, esperando que tú saldrías en su defensa; pero ya no te vi
por ninguna parte.

---Lo que te digo, Mariquilla de mi corazón---repuso Agustín,---es que
tuve que marcharme antes\ldots{} Después me dijeron que tu padre había
sido maltratado, y me dio un coraje\ldots{} Quise venir\ldots{}

---¡A buenas horas! Entre tantas, entre tantas personas---añadió la
Candiola llorando,---ni una, ni una sola hizo un gesto para defenderle.
Yo me moría de miedo aquí arriba, viéndole en peligro. Miramos con
ansiedad a la calle. Nada, no había más que enemigos\ldots{} Ni una mano
generosa, ni una voz caritativa\ldots{} Entre todos aquellos hombres,
uno, más cruel que todos arrojó a mi padre en el suelo\ldots{} ¡Oh!
Recordando esto, no sé lo que me pasa. Cuando lo presencié, un gran
terror me tuvo por momentos paralizada. Hasta entonces no conocí yo la
verdadera cólera, aquel fuego interior, aquel impulso repentino que me
hizo correr de aposento en aposento buscando\ldots{} Mi pobre padre
yacía en el suelo y el miserable le pisoteaba como si fuera un reptil
venenoso. Viendo esto, yo sentía la sangre hirviendo en mi cuerpo. Como
te he dicho, corrí por la habitación buscando un arma, un cuchillo, un
hacha, cualquier cosa. No encontré nada\ldots{} Desde lo interior oí los
lamentos de mi padre, y sin esperar más bajé a la calle. Al verme en el
almacén entre tantos hombres, sentí de nuevo invencible terror, y no
podía dar un paso. El mismo que le había maltratado me alargó un puñado
de monedas de oro. No las quise tomar, pero luego se las arrojé a la
cara con fuerza. Me parecía tener en la mano un puñado de rayos, y que
vengaba a mi padre lanzándolos contra aquellos viles. Salí después, miré
otra vez a todos lados buscándote, pero nada vi. Sólo entre la turba
inhumana, mi padre se encontraba sobre el cieno pidiendo misericordia.

---¡Oh! María, Mariquilla de mi corazón---exclamó Agustín con dolor,
besando las manos de la desgraciada hija del avaro,---no hables más de
ese asunto, que me destrozas el alma. Yo no podía defenderle\ldots{}
tuve que marcharme\ldots{} no sabía nada\ldots{} creí que aquella gente
se reunía con otro objeto. Es verdad que tienes razón; pero deja ese
asunto que me lastima, me ofende y me causa inmensa pena.

---Si hubieras salido a la defensa de mi padre, este te hubiera mostrado
gratitud. De la gratitud se pasa al cariño. Habrías entrado en
casa\ldots{}

---Tu padre es incapaz de amar a nadie---respondió Montoria.---No
esperes que consigamos nada por ese camino. Confiemos en llegar al
cumplimiento de nuestro deseo por caminos desconocidos, con la ayuda de
Dios y cuando menos lo parezca. No pensemos en lo ordinario ni en lo que
tenemos delante, porque todo lo que nos rodea está lleno de peligros, de
obstáculos, de imposibilidades; pensemos en algo imprevisto, en algún
medio superior y divino, y llenos de fe en Dios y en el poder de nuestro
amor, aguardemos el milagro que nos ha de unir, porque será un milagro,
María, un prodigio como los que cuentan de otros tiempos y nos
resistimos a creer.

---¡Un milagro!---exclamó María con melancólica estupefacción.---Es
verdad. Tú eres un caballero principal, hijo de personas que jamás
consentirían en verte casado con la hija del Sr.~Candiola. Mi padre es
aborrecido en toda la ciudad. Todos huyen de nosotros, nadie nos visita;
si salgo, me señalan, me miran con insolencia y desprecio. Las muchachas
de mi edad no gustan de alternar conmigo, y los jóvenes del pueblo que
recorren de noche la ciudad cantando músicas amorosas al pie de las
rejas de sus novias, vienen junto a las mías a decir insultos contra mi
padre, llamándome a mí misma con los nombres más feos. ¡Oh! ¡Dios mío!
Comprendo que ha de ser preciso un milagro para que yo sea feliz\ldots{}
Agustín, nos conocemos hace cuatro meses y aún no has querido decirme el
nombre de tus padres. Sin duda no serán tan odiados como el mío. ¿Por
qué lo ocultas? Si fuera preciso que nuestro amor se hiciera público, te
apartarías de las miradas de tus amigos, huyendo con horror de la hija
del tío Candiola.

---¡Oh! No, no digas eso---exclamó Agustín, abrazando las rodillas de
Mariquilla y ocultando el rostro en su regazo.---No digas que me
avergüenzo de quererte, porque al decirlo insultas a Dios. No es verdad.
Hoy nuestro amor permanece en secreto porque es necesario que así pase;
pero cuando sea preciso descubrirlo, lo descubriré arrostrando la cólera
de mi padre. Sí, María, mis padres me maldecirán, arrojándome de su
casa. Pero mi amor será más fuerte que su enojo. Hace pocas noches me
dijiste, mirando ese monumento que desde aquí se descubre: «Cuando esa
torre se ponga derecha, dejaré de quererte.» Yo te juro que la firmeza
de mi amor excede a la inmovilidad, al grandioso equilibrio de esa
torre, que podrá caer al suelo, pero jamás ponerse a plomo sobre la base
que la sustenta. Las obras de los hombres son variables; las de la
naturaleza son inmutables y descansan eternamente sobre su inmortal
asiento. ¿Has visto el Moncayo, esa gran peña que escalonada con otras
muchas se divisa hacia Poniente, mirando desde el arrabal? Pues cuando
el Moncayo se canse de estar en aquel sitio, y se mueva, y venga andando
hasta Zaragoza, y ponga uno de sus pies sobre nuestra ciudad,
reduciéndola a polvo, entonces, sólo entonces dejaré de quererte.

De este modo hiperbólico y con este naturalismo poético expresaba mi
amigo su grande amor, correspondiendo y halagando así la imaginación de
la hermosa Candiola, que propendía con impulso ingénito al mismo
sistema. Callaron ambos un momento, y luego los dos, mejor dicho, los
tres, proferimos una exclamación y miramos a la torre, cuya campana
había lanzado al viento dos toques de alarma. En el mismo instante un
globo de fuego surcó el espacio negro describiendo rápidas oscilaciones.

---¡Una bomba! ¡Es una bomba---exclamó María con pavor, arrojándose en
brazos de su amigo.

La espantosa luz pasó velozmente por encima de nuestras cabezas, por
encima de la huerta y de la casa, iluminando a su paso la torre, los
techos vecinos, hasta el rincón donde nos escondíamos. Luego sintiose el
estallido. La campana empezó a clamar, uniéndose a su grito el de otras
más o menos lejanas, agudas, graves, chillonas, cascadas, y oímos el
tropel de la gente que corría por las inmediatas calles.

---Esa bomba no nos matará---dijo Agustín, tranquilizando a su
novia.---¿Tienes miedo?

---¡Mucho, muchísimo miedo!---respondió esta.---Aunque a veces me parece
que tengo mucho, muchísimo valor. Paso las noches rezando y pidiéndole a
Dios que aparte el fuego de mi casa. Hasta ahora ninguna desgracia nos
ha ocurrido, ni en este ni en el otro sitio. Pero ¡cuántos infelices han
perecido, cuántas casas de personas honradas y que nunca hicieron mal a
nadie han sido destruidas por las llamas! Yo deseo ardientemente ir,
como los demás, a socorrer a los heridos; pero mi padre me lo prohíbe, y
se enfada conmigo siempre que se lo propongo.

Esto decía, cuando en el interior de la casa sentimos ruido vago y
lejano en que se confundía con la voz de la señora Guedita la
desapacible del tío Candiola. Los tres obedeciendo a un mismo
pensamiento nos estrechamos en el rincón y contuvimos el aliento,
temiendo ser sorprendidos. Luego sentimos más cerca la voz del avaro que
decía:

---¿Qué hace Vd. levantada a estas horas, señora Guedita?

---Señor---contestó la vieja, asomándose por una ventana que daba al
corredor,---¿quién puede dormir con este horroroso bombardeo? Si a lo
mejor se nos mete aquí una señora bomba y nos coge en la cama y en paños
menores, y vienen los vecinos a sacar los trastos y a pagar el
fuego\ldots{} ¡Oh, qué falta de pudor! No pienso desnudarme mientras
dure este endemoniado bombardeo.

---Y mi hija, ¿duerme?---preguntó Candiola, que al decir esto se asomaba
por un ventanillo al otro extremo de la huerta.

---Arriba está durmiendo como una marmota---repuso la dueña.---Bien
dicen que para la inocencia no hay peligros. A la niña no le asusta una
bomba más que un cohete.

---Si desde aquí se divisara el punto donde ha caído ese
proyectil\ldots---dijo Candiola alargando su cuerpo fuera de la ventana
para poder extender la vista por sobre los tejados vecinos, más bajos
que el de su casa.---Se ve claridad como de incendio; pero no puedo
decir si es cerca o lejos.

---O yo no entiendo nada de bombas---dijo Guedita desde el corredor,---o
esta ha caído allá por el mercado.

---Así parece. Si cayeran todas en las casas de los que sostienen la
defensa, y se empeñan en no acabar de una vez tantos desastres\ldots{}
Si no me engaño, señora Guedita, el fuego luce hacia la calle de la
Tripería. ¿No están por allá los almacenes de la junta de abastos? ¡Ah!
¡Bendita bomba, que no cayera en la calle de la Hilarza y en la casa del
malvado y miserable ladrón!\ldots{} Señora Guedita, estoy por salir a la
calle a ver si el regalo ha caído en la calle de la Hilarza, en la casa
del orgulloso, del entrometido, del canalla, del asesino D. José de
Montoria. Se lo he pedido con tanto fervor esta noche a la Virgen del
Pilar, a las Santas Masas y a Santo Domingo del Val, que al fin creo que
han oído.

---Sr.~D. Jerónimo---dijo la vieja,---déjese de correrías que el frío de
la noche traspasa, y no vale la pena de coger una pulmonía por ver dónde
paró la bomba, que harto tenemos ya con saber que no se nos ha metido en
casa. Si la que pasó no ha caído en casa de ese bárbaro sayón, otra
caerá mañana, pues los franceses tienen buena mano. Conque acuéstese su
merced, que yo me quedo rondando la casa, por si ocurriese algo.

Candiola, respecto a la salida, varió sin duda de parecer, en vista de
los buenos consejos de la criada, porque cerrando la ventanilla, metiose
dentro, y no se le sintió más en el resto de la noche. Mas no porque
desapareciera rompieron los amantes el silencio, temerosos de ser
escuchados o sorprendidos; y hasta que la vieja no vino a participarnos
que el señor roncaba como un labriego, no se reanudó el diálogo
interrumpido.

---Mi padre desea que las bombas caigan sobre la casa de su
enemigo---dijo María.---Yo no quisiera verlas en ninguna parte, pero si
alguna vez se puede desear mal al prójimo, es en esta ocasión, ¿no es
verdad?

Agustín no contestó nada.

---Tú te marchaste---continuó la joven;---tú no viste cómo aquel hombre,
el más cruel, el más malvado y cobarde de todos los que vinieron, le
arrojó al suelo, ciego de cólera y le pisoteó. Así patearán su alma los
demonios en el infierno, ¿no es verdad?

---Sí---contestó lacónicamente el mozo.

---Esta tarde, después que todo aquello pasó, Guedita y yo curábamos las
contusiones de mi padre. Él estaba tendido sobre su cama, y loco de
desesperación, se retorcía mordiéndose los puños y lamentándose de no
haber tenido más fuerza que el otro. Nosotras procurábamos consolarle;
pero él nos decía que calláramos. Después me echó en cara ¡tal era su
rabia!, que hubiese yo arrojado a la calle el dinero de la harina;
enfadose mucho conmigo, y me dijo que pues no se pudo sacar otra cosa,
los tres mil reales y pico no debían despreciarse; y que yo era una loca
despilfarradora, que lo estaba arruinando. De ningún modo podíamos
calmarle. Cerca ya del anochecer sentimos otra vez ruido en la calle.
Creímos que volvían los mismos y el mismo del mediodía. Mi padre quiso
arrojarse del lecho lleno de furia. Yo tuve al principio mucho miedo;
después me reanimé, considerando que era necesario mostrar valor.
Pensando en ti, dije: «Si él estuviera en casa, nadie nos insultaría.»
Como el rumor de la calle aumentara, lleneme de valor, cerré bien todas
las puertas, y rogando a mi padre que continuase quieto en su cama,
resolví esperar. Mientras Guedita rezaba de rodillas a todos los santos
del cielo, yo registré la casa buscando un arma, y al fin pude hallar un
cuchillo. La vista de esta arma siempre me ha causado horror; pero hoy
la empuñé con decisión. ¡Oh!, estaba fuera de mí, y aún ahora mismo me
causa espanto el pensar en aquello. Frecuentemente me desmayo al mirar
un herido; me asusto y tiemblo sólo de ver una gota de sangre; casi
lloro si castigan a un perro delante de mí, y jamás he tenido fuerzas
para matar una mosca; pero esta tarde, Agustín, esta tarde cuando sentí
ruido en la calle, cuando creí oír de nuevo los golpes en la puerta,
cuando esperaba por momentos ver delante de mí a aquellos
hombres\ldots{} Te juro que si llega a salir verdad lo que temí, si
cuando yo estaba en el cuarto de mi padre, junto a su lecho, llega a
entrar el mismo hombre que le maltrató algunas horas antes, te juro que
allí mismo\ldots{} sin vacilar\ldots{} cierro los ojos y le parto el
corazón.

---¡Calla, por Dios!---dijo Montoria con horror.---Me causas miedo,
María, y al oírte me parece que tus propias manos, estas divinas manos
clavan en mi pecho la hoja fría. No maltratarán otra vez a tu padre. Ya
ves cómo lo de esta noche fue puro miedo. No, no hubieras sido capaz de
lo que dices; tú eres una mujer, y una mujer débil, sensible, tímida,
incapaz de matar a un hombre, como no le mates de amor. El cuchillo se
te hubiera caído de las manos y no habrías manchado tu pureza con la
sangre de un semejante. Esos horrores se quedan para nosotros los
hombres, que nacemos destinados a la lucha, y que a veces nos vemos en
el triste caso de gozar arrancando hombres a la vida. María, no hables
más de ese asunto, no recuerdes a los que te ofendieron; olvídalos,
perdónalos, y sobre todo no mates a nadie, ni aun con el pensamiento.

\hypertarget{xvi}{%
\chapter{XVI}\label{xvi}}

Mientras esto decían, observé el rostro de la Candiola, que en la
oscuridad parecía modelado en pálida cera y tenía el tono pastoso y mate
del marfil. De sus negros ojos, siempre que los alzaba al cielo, partía
un ligero rayo. Sus negras pupilas, sirviendo de espejo a la claridad
del cielo, producían, en el fondo donde nos encontrábamos, dos rápidos
puntos de luz, que aparecían y se borraban, según la movilidad de su
mirada. Y era curioso observar en aquella criatura, toda ella pasión, la
borrascosa crisis que removía y exaltaba su sensibilidad hasta ponerla
en punto de bravura. Aquel abandono voluptuoso, aquel arrullo (pues no
hallo nombre más propio para pintarla), aquel tibio agasajo que había en
la atmósfera junto a ella, no se avenía bien aparentemente con los
alardes de heroísmo en defensa del ultrajado padre; pero una observación
atenta podía descubrir que ambas corrientes afluían de un mismo
manantial.

---Yo admiro tu exaltado cariño filial---prosiguió Agustín.---Ahora, oye
otra cosa. No disculpo a los que maltrataron a tu padre; pero no debes
olvidar que tu padre es el único que no ha dado nada para la guerra. D.
Jerónimo es una persona excelente; pero no tiene en su alma ni una
chispa de patriotismo. Le son indiferentes las desgracias de la ciudad y
hasta parece alegrarse cuando no salimos victoriosos.

La Candiola exhaló algunos suspiros, elevando los ojos al cielo.

---Es verdad---dijo después.---Todos los días y a todas horas le estoy
suplicando que dé algo para la guerra. Nada puedo conseguir, aunque le
pondero la necesidad de los pobres soldados y el mal papel que estamos
haciendo en Zaragoza. Él se enfada cuando me oye, y dice que el que ha
traído la guerra que la pague. En el otro sitio, me alegraba en extremo
cuando tenía noticia de una victoria, y el 4 de Agosto salí yo misma
sola a la calle, no pudiendo resistir la curiosidad. Una noche estaba en
casa de las de Urries, y como celebraran la acción de aquella tarde, que
había sido muy brillante, empecé a alabar yo también lo ocurrido,
mostrándome muy entusiasmada. Entonces una vieja que estaba presente me
dijo en alta voz y con muy mal tono: «Niña loca, en vez de hacer esos
aspavientos, ¿por qué no llevas al hospital de sangre siquiera una
sábana vieja? En casa del Sr.~Candiola, que tiene los sótanos llenos de
dinero, ¿no hay un mal pingajo que dar a los heridos? Tu papaíto es el
único, el único de todos los vecinos de Zaragoza que no ha dado nada
para la guerra.» Rieron todos al oír esto, y yo me quedé corrida, muerta
de vergüenza, sin atreverme a hablar. En un rincón de la sala estuve
hasta el fin de la tertulia, sin que nadie me dirigiera la palabra. Mis
pocas amigas, que tanto me querían, no se acercaban a mí; entre el
tumulto de la reunión, oí a menudo el nombre de mi padre con comentarios
y apodos muy denigrantes. ¡Oh! Se me partía con esto el corazón. Cuando
me retiré para venir a casa, apenas me saludaron fríamente, y los amos
de la casa me despidieron con desabrimiento. Vine aquí, era ya de noche,
me acosté, y no pude dormir ni cesé de llorar hasta por la mañana. La
vergüenza me requemaba la sangre.

---Mariquilla---exclamó Agustín con amor,---la bondad de tus
sentimientos es tan grande, que por ella olvidará Dios las crueldades de
tu padre.

---Después---prosiguió la Candiola,---a los pocos días, el 4 de Agosto,
vinieron los dos heridos que nombró hoy en la reyerta el enemigo de mi
padre. Cuando nos dijeron que la junta destinaba a casa dos heridos para
que los asistiéramos, Guedita y yo nos alegramos mucho, y locas de
contento empezamos a preparar vendas, hilas y camas. Les esperábamos con
tanta ansiedad que a cada instante nos poníamos a la ventana por ver si
venían. Por fin vinieron; mi padre, que había llegado momentos antes de
la calle con muy negro humor, quejándose de que habían muerto muchos de
sus deudores, y que no tenía esperanza de cobrar, recibió muy mal a los
heridos. Yo le abracé llorando y le pedí que les diera alojamiento; pero
no me hizo caso, y ciego de cólera, les arrojó en medio del arroyo,
atrancó la puerta y subió diciendo: «Que los asista quien los ha
parido.» Era ya de noche. Guedita y yo estábamos muertas de desolación.
No sabíamos qué hacer, y desde aquí sentíamos los lamentos de aquellos
dos infelices, que se arrastraban en la calle pidiendo socorro. Mi
padre, encerrándose en su cuarto para hacer cuentas, no se cuidaba ya ni
de ellos ni de nosotras. Pasito a pasito, para que no nos sintiera,
fuimos a la habitación que da a la calle, y por la ventana les echamos
trapos para que se vendaran; pero no los podían coger. Les llamamos, nos
vieron, y alargaban sus manos hacia nosotras. Atamos un cestillo a la
punta de una caña y les dimos algo de comida; pero uno de ellos estaba
exánime y al otro sus dolores no le permitían comer nada. Les animábamos
con palabras tiernas, y pedíamos a Dios por ellos. Por último,
resolvimos bajar por aquí y salir afuera para asistirles, aunque sólo un
momento; pero mi padre nos sorprendió y se puso furioso. ¡Qué noche,
Santa Virgen! Uno de ellos murió en medio de la calle, y el otro se fue
arrastrando a buscar misericordia no sé dónde.

Agustín y yo callamos, meditando en las monstruosas contradicciones de
aquella casa.

---Mariquilla---exclamó al fin mi amigo,---¡qué orgulloso estoy de
quererte! La ciudad no conoce tu corazón de oro, y es preciso que lo
conozca. Yo quiero decir a todo el mundo que te amo, y probar a mis
padres, cuando lo sepan, que he hecho una elección acertada.

---Yo soy como cualquiera---dijo con humildad la Candiola,---y tus
padres no verán en mí sino la hija del que llaman el \emph{judío
mallorquín}. ¡Oh, me mata la vergüenza! Quiero salir de Zaragoza y no
volver más a este pueblo. Mi padre es de Palma, es cierto; pero no
desciende de judíos, sino de cristianos viejos, y mi madre era aragonesa
y de la familia de Rincón. ¿Por qué somos despreciados? ¿Qué hemos
hecho?

Diciendo esto, los labios de Mariquilla se contrajeron con una sonrisa
entre incrédula y desdeñosa. Agustín, atormentado sin duda por dolorosos
pensamientos, permaneció mudo, con la frente apoyada sobre las manos de
su novia. Terribles fantasmas se alzaban con amenazador ademán entre uno
y otro. Con los ojos del alma, él y ella les estaban mirando llenos de
espanto.

Después de un largo rato, Agustín alzó el rostro.

---María, ¿por qué callas? Di algo.

---¿Por qué callas tú, Agustín?

---¿En qué piensas?

---¿En qué piensas tú?

---Pienso---dijo el mancebo,---en que Dios nos protegerá. Cuando
concluya el sitio, nos casaremos. Si tú te vas de Zaragoza, yo iré
contigo a donde tú te vayas. ¿Tu padre te ha hablado alguna vez de
casarte con alguien?

---Nunca.

---No impedirá que te cases conmigo. Yo sé que los míos se opondrán;
pero mi voluntad es irrevocable. No comprendo la vida sin ti, y
perdiéndote no existiría. Eres la suprema necesidad de mi alma, que sin
ti sería como el universo sin luz. Ninguna fuerza humana nos apartará
mientras tú me ames. Esta convicción está tan arraigada dentro de mí,
que si alguna vez pienso que nos hemos de separar en vida para siempre,
se me representa esto como un trastorno en la naturaleza. ¡Yo sin ti!
Esto me parece la mayor de las aberraciones. ¡Yo sin ti! ¡Qué delirio y
qué absurdo! Es como el mar en la cumbre de las montañas y la nieve en
las profundidades del océano vacío, como los ríos corriendo por el cielo
y los astros hechos polvo de fuego en las llanuras de la tierra; como si
los árboles hablaran y el hombre viviera entre los metales y las piedras
preciosas en las entrañas de la tierra. Yo me acobardo a veces, y
tiemblo pensando en las contrariedades que nos abruman; pero la
confianza que ilumina mi espíritu, como la fe de las cosas santas, me
reanima. Si por momentos temo la muerte, después una voz secreta me dice
que no moriré mientras tú vivas. ¿Ves todo este estrago del sitio que
soportamos? ¿Ves cómo llueven bombas, granadas y balas, y cómo caen para
no levantarse más infinitos compañeros míos? Pues pasada la primera
impresión de miedo, nada de esto me hace estremecer, y creo que la
Virgen del Pilar aparta de mí la muerte. Tu sensibilidad te tiene en
comunicación constante con los ángeles del cielo; tú eres un ángel del
cielo, y el amarte, el ser amado por ti, me da un poder divino contra el
cual nada pueden las fuerzas del hombre.

Así habló largo rato Agustín, desbordándose de su llena fantasía los
pensamientos de la amorosa superstición que le dominaba.

---Pues yo---dijo Mariquilla,---también tengo cierta confianza en lo
mismo que has dicho. Temo mucho que te maten; pero no se qué voces me
suenan en el fondo de mi alma, diciéndome que no te matarán. ¿Será
porque he rezado mucho, pidiendo a Dios conserve tu vida en medio de
este horroroso fuego? No lo sé. Por las noches, como me acuesto pensando
en las bombas que han caído, en las que caen, y en las que caerán, sueño
con las batallas, y no ceso de oír el zumbido de los cañones. Deliro
mucho, y Guedita que duerme junto a mí, dice que hablo en sueños,
diciendo mil desatinos. Seguramente diré alguna cosa, porque no ceso de
soñar, y te veo en la muralla y hablo contigo y me respondes. Las balas
no te tocan, y me parece que es por los Padre Nuestros que rezo
despierta y dormida. Hace pocas noches soñé que iba a curar a los
heridos con otras muchachas, y que les poníamos buenos en el acto, casi
resucitándoles con nuestras hilas. También soñé que de vuelta a casa, te
encontré aquí, estabas con tu padre, que era un viejecito muy amable y
risueño, y hablaba con el mío, sentados ambos en el sofá de la sala, y
los dos parecían muy amigos. Después soñé que tu padre me miraba
sonriendo, y empezó a hacerme preguntas.

Otras veces sueño cosas tristes. Cuando despierto, pongo atención, y si
no siento el ruido del bombardeo, digo: «puede que los franceses hayan
levantado el sitio.» Si oigo cañonazos, miro a la imagen de la Virgen
del Pilar que está en mi cuarto, le pregunto con el pensamiento, y me
contesta que no has muerto, sin que yo pueda decir qué signo emplea para
responderme. Paso el día pensando en las murallas, y me pongo en la
ventana para oír lo que dicen los mozos al pasar por la calle. Algunas
veces siento tentaciones de preguntarles si te han visto\ldots{} Llega
la noche, te veo, y me quedo tan contenta. Al día siguiente Guedita y yo
nos ocupamos en preparar alguna cosa de comer a escondidas de mi padre;
si vale la pena, te la guardamos a ti; y si no, se la lleva para los
heridos y enfermos ese frailito que llaman el padre Busto, el cual viene
por las tardes con pretexto de visitar a doña Guedita de quien es
pariente. Nosotras le preguntamos cómo va la cosa, y él nos dice:
«Perfectamente. Las tropas están haciendo grandes proezas, y los
franceses tendrán que retirarse como la otra vez.» Estas noticias de que
todo va bien nos vuelven locas de gozo. El ruido de las bombas nos
entristece después; pero rezando recobramos la tranquilidad. A solas en
nuestro cuarto, de noche, hacemos hilas y vendas, que se lleva también a
escondidas el padre Busto, como si fueran objetos robados, y al sentir
los pasos de mi padre, lo guardamos todo con precipitación y apagamos la
luz, porque si descubre lo que estamos haciendo, se pone furioso.

Contando sus sustos y sus alegrías con divina sencillez, Mariquilla
estaba risueña y algo festiva. El encanto especial de su voz no es
descriptible, y sus palabras semejantes a una vibración de notas
cristalinas dejaban eco armonioso en el alma. Cuando concluyó, el primer
resplandor de la aurora empezaba a alumbrar su semblante.

---Despunta el día Mariquilla---dijo Agustín,---y tenemos que
marcharnos. Hoy vamos a defender las Tenerías; hoy habrá un fuego
horroroso y morirán muchos; pero la Virgen del Pilar nos amparará y
podremos gozar de la victoria. María, Mariquilla, no me tocarán las
balas.

---No te vayas todavía---repuso la hija de Candiola.---Comienza el día;
pero aún no hacéis falta en la muralla.

Sonó la campana de la torre.

---Mira qué pájaros cruzan el espacio anunciando la aurora---dijo
Agustín con amarga ironía.

Una, dos, tres bombas atravesaron el cielo, débilmente aclarado todavía.

---¡Qué miedo!---exclamó María, dejándose abrazar por Montoria.---¿Nos
preservará Dios hoy como nos ha preservado ayer?

---¡A la muralla!---exclamé yo, levantándome a toda prisa.---¿No oyes
que tocan a llamada las campanas y las cajas? ¡A la muralla!

Mariquilla, poseída de un pánico imposible de pintar, lloraba, queriendo
detener a Montoria. Yo, resuelto a partir, pugnaba por llevármele.

Estruendo de tambores y campanas sonaba en la ciudad convocando a las
armas, y si en el instante mismo no acudíamos a las filas, corríamos
riesgo de ser arcabuceados o tenidos por cobardes.

---Me voy, me voy, María---dijo mi amigo con profunda emoción.---¿Temes
al fuego? No; esta casa es sagrada, porque tú la habitas; será respetada
por el fuego enemigo, y la crueldad de tu padre no la castigará Dios en
tu santa cabeza. Adiós.

Apareció bruscamente doña Guedita, diciendo que su amo se estaba
levantando a toda prisa. Entonces la misma María nos empujó hacia lo
bajo de la huerta, ordenándonos que saliéramos al instante. Agustín
estaba traspasado de pena, y en la puerta hizo movimientos de
perplejidad y dio algunos pasos para volver al lado de la infeliz
Candiolilla, que muerta de miedo, derramando lágrimas y con las manos
cruzadas en disposición de orar, nos miraba partir, aún envuelta en la
sombra del ciprés que nos había dado abrigo.

En el momento en que abríamos la puerta oyose un grito en la parte
superior de la casa, y vimos al tío Candiola, que saliendo a medio
vestir, se dirigía hacia nosotros en actitud amenazadora. Quiso Agustín
volver atrás; pero le empujé hacia afuera, y salimos.

---¡Al momento a las filas! ¡A las filas!---exclamé.---Nos echarán de
menos, Agustín. Deja por ahora a tu futuro suegro que se entienda con tu
futura esposa.

Y velozmente corrimos hasta dar en el Coso, donde observamos el
sinnúmero de bombas arrojadas sobre la infeliz ciudad. Todos acudían con
presteza a distintos sitios, cuál a las Tenerías, cuál al Portillo, cuál
a Santa Engracia o a Trinitarios. Al llegar al arco de Cineja,
tropezamos con D. José de Montoria, que seguido de sus amigos, corría
hacia el Almudí. En el mismo instante un terrible estampido, resonando a
nuestra espalda, nos anunció que un proyectil enemigo había caído en
paraje cercano. Agustín, al oír esto, volvió hacia atrás, disponiéndose
a tornar al punto de donde veníamos.

---¿A dónde vas?, ¡porra!---le dijo su padre deteniéndole.---A las
Tenerías, pronto a las Tenerías.

La gente que iba y venía supo al instante el lugar del desastre, y oímos
decir:

---Tres bombas han caído juntas en la casa del tío Candiola.

---Los ángeles del cielo apuntaron sin duda los morteros---exclamó D.
José de Montoria con estrepitosa carcajada.---Veremos cómo se las
compone ese judío mallorquín, si es que ha quedado vivo, para poner en
salvo su dinero.

---Corramos a salvar a esos desgraciados---dijo Agustín con vehemencia.

---A las filas, cobardes---exclamó el padre sujetándole con férrea
mano.---Esa es obra de mujeres. Los hombres a morir en la brecha.

Era preciso acudir a nuestros puestos, y fuimos, mejor dicho, nos
llevaron, nos arrastró la impetuosa oleada de gente que corría a
defender el barrio de las Tenerías

\hypertarget{xvii}{%
\chapter{XVII}\label{xvii}}

Mientras los morteros situados al Mediodía arrojaban bombas en el centro
de la ciudad, los cañones de la línea oriental dispararon con bala rasa
sobre la débil tapia de las Mónicas y las fortificaciones de tierra y
ladrillo del Molino de aceite y de la batería de Palafox. Bien pronto
abrieron tres grandes brechas, y el asalto era inminente. Apoyábanse en
el molino de Goicoechea, que tomaron el día anterior, después de ser
abandonado e incendiado por los nuestros.

Seguras del triunfo, las masas de infantería recorrían el campo,
ordenándose para asaltarnos. Mi batallón ocupaba una casa de la calle de
Pabostre, cuya pared había sido en toda su extensión aspillerada. Muchos
paisanos y compañías de varios regimientos aguardaban en la cortina,
llenos de furor y sin que les arredrara la probabilidad de una muerte
segura, con tal de escarmentar al enemigo en su impetuoso avance.

Pasaron largas horas; los franceses apuraban los recursos de su
artillería por ver si nos aterraban, obligándonos a dejar el barrio;
pero las tapias se desmoronaban, estremecíanse las casas con espantoso
sacudimiento, y aquella gente heroica, que apenas se había desayunado
con un zoquete de pan, gritaba desde la muralla, diciéndoles que se
acercasen. Por fin, contra la brecha del centro y la de la derecha
avanzaron fuertes columnas, sostenidas por otras a retaguardia, y se vio
que la intención de los franceses era apoderarse a todo trance de
aquella línea de pulverizados ladrillos, que defendían algunos
centenares de locos, y tomarla a cualquier precio, arrojando sobre ella
masas de carne y haciendo pasar la columna viva sobre los cadáveres de
la muerta.

No se diga para amenguar el mérito de los nuestros, que el francés
luchaba a pecho descubierto; los defensores también lo hacían; y detrás
de la desbaratada cortina no podía guarecerse una cabeza. Allí era de
ver cómo chocaban las masas de hombres y cómo las bayonetas se cebaban
con saña más propia de fieras que de hombres en los cuerpos enemigos.
Desde las casas hacíamos fuego incesante, viéndolos caer materialmente
en montones, heridos por el plomo y el acero al pie mismo de los
escombros que querían conquistar. Nuevas columnas sustituían a las
anteriores, y en los que llegaban después, a los esfuerzos del valor se
unían ferozmente las brutalidades de la venganza.

Por nuestra parte el número de bajas era enorme: los hombres quedaban
por docenas estrellados contra el suelo en aquella línea que había sido
muralla, y ya no era sino una aglomeración informe de tierra, ladrillos
y cadáveres. Lo natural, lo humano habría sido abandonar unas posiciones
defendidas contra todos los elementos de la fuerza y de la ciencia
militar reunidos; pero allí no se trataba de nada que fuese humano y
natural, sino de extender la potencia defensiva hasta límites infinitos,
desconocidos para el cálculo científico y para el valor ordinario,
desarrollando en sus inconmensurables dimensiones el genio aragonés, que
nunca se sabe a dónde llega.

Siguió pues la resistencia, sustituyendo los vivos a los muertos con
entereza sublime. Morir era un accidente, un detalle trivial, un
tropiezo del cual no debía hacerse caso.

Mientras esto pasaba, otras columnas igualmente poderosas trataban de
apoderarse de la casa de González, que he mencionado arriba; pero desde
las casas inmediatas y desde los cubos de la muralla se les hizo un
fuego tan terrible de fusilería y cañón, que desistieron de su intento.
Iguales ataques tenían lugar, con mejor éxito de parte suya por nuestra
derecha, hacia la huerta de Camporeal y baterías de los Mártires, y la
inmensa fuerza desplegada por los sitiadores a una misma hora y en una
línea de poca extensión no podía menos de producir resultados.

Desde la casa de la calle de Pabostre inmediata al Molino de la ciudad,
hacíamos fuego, como he dicho, contra los que daban el asalto, cuando he
aquí que las baterías de San José, antes ocupadas en demoler la muralla,
enfilaron sus cañones contra aquel viejo edificio, y sentimos que las
paredes retemblaban; que las vigas crujían como cuadernas de un buque
conmovido por las tempestades; que las maderas de los tapiales
estallaban destrozándose en mil astillas; en suma, que la casa se venía
abajo.

---¡Cuerno, recuerno!---exclamó el tío Garcés.---Que se nos viene la
casa encima.

El humo, el polvo, no nos permitía ver lo que pasaba fuera, ni lo que
pasaba dentro.

---¡A la calle, a la calle!---gritó Pirli, arrojándose por una ventana.

---¡Agustín, Agustín!, ¿dónde estás?---grité yo, llamando a mi amigo.

Pero Agustín no parecía. En aquel momento de angustia, y no encontrando
en medio de tal confusión ni puerta para salir, ni escalera para bajar,
corrí a la ventana para arrojarme fuera, y el espectáculo que se ofreció
a mis ojos obligome a retroceder sin aliento ni fuerzas. Mientras los
cañones de la batería de San José intentaban por la derecha sepultarnos
entre los escombros de la casa, y parecían conseguirlo sin esfuerzo, por
delante, y hacia la era de San Agustín, la infantería francesa había
logrado penetrar por las brechas, rematando a los infelices que ya
apenas eran hombres, y acabándoles de matar, pues su agonía desesperada
no puede llamarse vida. De los callejones cercanos se les hacía un fuego
horroroso y los cañones de la calle de Diezma sustituían a los de la
batería vencida. Pero asaltada la brecha, se aseguraban en la muralla.
Era imposible conservar en el ánimo una chispa de energía ante tamaño
desastre.

Huí de la ventana hacia dentro, despavorido, fuera de mí. Un trozo de
pared estalló, reventó, desgajándose en enormes trozos y una ventana
cuadrada tomó la figura de un triángulo isósceles: el techo dejó ver por
una esquina la luz del cielo y los trozos de yeso y las agudas astillas
salpicaron mi cara. Corrí hacia el interior, siguiendo a otros que
decían: ¡por aquí, por aquí!

---¡Agustín, Agustín!---grité de nuevo llamando a mi amigo.

Por fin le vi entre los que corríamos pasando de una habitación a otra,
y subiendo una escalerilla que conducía a un desván.

---¿Estás vivo?---le pregunté.

---No lo sé---me dijo,---ni me importa saberlo.

En el desván rompimos fácilmente un tabique, y pasando a otra pieza,
hallamos una empinada escalera; la bajamos, y nos vimos en una
habitación chica. Unos siguieron adelante, buscando salida a la calle, y
otros detuviéronse allí.

Se ha quedado fijo en mi imaginación, con líneas y colores indelebles,
el interior de aquella mezquina pieza, bañada por la copiosa luz que
entraba por una ventana abierta a la calle. Cubrían las paredes
desiguales estampas de vírgenes y santos. Dos o tres cofres viejos y
forrados de piel de cabra, ocupaban un testero. Veíase en otro ropa de
mujer, colgada de clavos y alcayatas, y una cama altísima de humilde
aspecto, aún con las sábanas revueltas. En la ventana había tres grandes
tiestos de yerbas; y parapetadas tras ellos, dirigiendo por los huecos
la rencorosa visual de su puntería, dos mujeres hacían fuego sobre los
franceses, que ya ocupaban la brecha. Tenían dos fusiles. Una cargaba y
otra disparaba; agachábase la fusilera para enfilar el cañón entre los
tiestos, y suelto el tiro, alzaba la cabeza por sobre las matas para
mirar el campo de batalla.

---Manuela Sancho---exclamé, poniendo la mano sobre el hombro de la
heroica muchacha.---Toda resistencia es inútil. Retirémonos. La casa
inmediata es destruida por las baterías de San José, y en el techo de
esta empiezan a caer las balas. Vámonos.

Pero no hacía caso, y seguía disparando. Al fin la casa, que era débil
como la vecina, y aún menos que esta podía resistir el choque de los
proyectiles, experimentó una fuerte sacudida, cual si temblara la tierra
en que arraigaba sus cimientos. Manuela Sancho arrojó el fusil. Ella y
la mujer que la acompañaba penetraron precipitadamente en una inmediata
alcoba, de cuyo oscuro recinto sentí salir angustiosas lamentaciones. Al
entrar, vimos que las dos muchachas abrazaban a una anciana tullida que,
en su pavor, quería arrojarse del lecho.

---Madre, esto no es nada---le dijo Manuela cubriéndola con lo primero
que encontró a mano.---Vámonos a la calle, que la casa parece que se
quiere caer.

La anciana no hablaba, no podía hablar. Tomáronla en brazos las dos
mozas; mas nosotros la recogimos en los nuestros, encargándoles a ellas
que llevaran nuestros fusiles y la ropa que pudieran salvar. De este
modo pasamos a un patio, que nos dio salida a otra calle, donde aún no
había llegado el fuego.

\hypertarget{xviii}{%
\chapter{XVIII}\label{xviii}}

Los franceses habíanse apoderado también de la batería de los Mártires,
y en aquella misma tarde fueron dueños de las ruinas de Santa Engracia y
del Convento de Trinitarios. ¿Se concibe que continúe la resistencia de
una plaza después de perdido lo más importante de su circuito? No, no se
concibe, ni en las previsiones del arte militar ha entrado nunca que,
apoderado el enemigo de la muralla por la superioridad incontrastable de
su fuerza material, ofrezcan las casas nuevas líneas de fortificaciones,
improvisadas por la iniciativa de cada vecino; no se concibe que, tomada
una casa, sea preciso organizar un verdadero plan de sitio para tomar la
inmediata, empleando la zapa, la mina y ataques parciales a la bayoneta,
desarrollando contra un tabique ingeniosa estratagema; no se concibe que
tomada una acera sea preciso para pasar a la de enfrente poner en
ejecución las teorías de Vauban, y que para saltar un arroyo sea preciso
hacer paralelas, zig-zags y caminos cubiertos.

Los generales franceses se llevaban las manos a la cabeza diciendo:
«Esto no se parece a nada de lo que hemos visto.» En los gloriosos
anales del imperio se encuentran muchos partes como este: «Hemos entrado
en Spandau; mañana estaremos en Berlín.» Lo que aún no se había escrito
era lo siguiente: «Después de dos días y dos noches de combate hemos
tomado la casa número 1 de la calle de Pabostre. Ignoramos cuándo se
podrá tomar el número 2.»

No tuvimos tiempo para reposar. Los dos cañones que enfilaban la calle
de Pabostre, en el ángulo de Puerta Quemada, se habían quedado sin
gente. Unos corrimos a servirlo, y el resto del batallón ocupó varias
casas en la calle de Palomar. Los franceses dejaron de hacer fuego de
cañón contra los edificios que habíamos abandonado, ocupándose
precipitadamente en repararlos como pudieron. Lo que amenazaba ruina lo
demolían, y tapiaban los huecos con vigas, cascajo y sacas de lana.

Como no podían atravesar sin riesgo el espacio intermedio entre los
restos de muralla y sus nuevos alojamientos, comenzaron a abrir una
zanja en zig-zag desde el Molino de la ciudad a la casa que antes
ocupáramos nosotros, la cual, sólo conservaba en buen estado para
alojamiento la planta baja.

Al punto comprendimos que una vez dueños de aquella casa, procurarían,
derribando tabiques, apoderarse de toda la manzana, y para evitarlo la
tropa disponible fue distribuida en guarniciones que ocuparon todos los
edificios donde había peligro. Al mismo tiempo se levantaban barricadas
en las bocacalles, aprovechando los escombros. Nos pusimos a trabajar
con ardor frenético en distintas faenas, entre las cuales la menos
penosa era seguramente la de batirnos. Dentro de las casas arrojábamos
por los balcones todos los muebles; afuera transportábamos heridos o
arrimábamos los muertos al zócalo de los edificios, pues las únicas
honras fúnebres que por entonces podían hacérseles, consistían en
quitarlos de donde estorbaban.

Quisieron también los franceses ganar a Santa Mónica, Convento situado
en la línea de las Tenerías, más al Norte de la calle de Pabostre; pero
sus paredes ofrecían buena resistencia, y no era fácil tomarlo como
aquellas endebles casas, que el estruendo tan sólo de los cañones hacía
estremecer. Los voluntarios de Huesca la defendían con gran arrojo, y
después de repetidos ataques, los sitiadores dejaron la empresa para
otro día. Posesionados tan sólo de algunas casas, en ellas permanecían a
la caída de la tarde como en escondida madriguera, y ¡ay de aquel que la
cabeza asomaba fuera de las ventanas! Las paredes próximas, los tejados,
las bohardillas y tragaluces abiertos en distintas direcciones estaban
llenos de atentos ojos que observaban el menor descuido del soldado
enemigo para soltarle un tiro.

Cuando anocheció empezamos a abrir huecos en los tabiques para
comunicar, todas las casas de una misma manzana. A pesar del incesante
ruido del cañón y la fusilería, en el interior de los edificios pudimos
percibir el golpear de las piquetas enemigas, ocupadas en igual tarea
que nosotros. También ellos establecían comunicaciones. Como aquella
arquitectura era frágil y casi todos los tabiques de tierra, en poco
tiempo abrimos paso entre varias casas.

A eso de las diez de la noche nos hallábamos en una que debía de ser muy
inmediata a la de Manuela Sancho, cuando sentimos que por conductos
desconocidos, por sótanos, pasillos o subterráneas comunicaciones,
llegaba a nuestros oídos el rumor de las voces del enemigo. Una mujer
subió azorada por una escalerilla, diciéndonos que los franceses estaban
abriendo un boquete en la pared de la cuadra, y bajamos al instante;
pero aún no estábamos todos en el patio frío, estrecho y oscuro de la
casa, cuando a boca de jarro se nos disparó un tiro, y un compañero fue
levemente herido en el hombro.

A la escasa claridad percibimos varios bultos que sucesivamente se
internaron en la cuadra, e hicimos fuego, avanzando después con brío
tras ellos.

Al ruido de los tiros acudieron otros compañeros nuestros que habían
quedado arriba, y penetramos denodadamente en la lóbrega pieza. Los
enemigos no se detuvieron en ella, y a todo escape repasaron el agujero
abierto en la pared medianera, buscando refugio en su primitiva morada,
desde la cual nos enviaron algunas balas. No estábamos completamente a
oscuras, porque ellos tenían una hoguera, de cuyas llamas algunos
débiles rayos penetraban por la abertura, difundiendo rojiza claridad
sobre el teatro de aquella lucha.

Yo no había visto nunca cosa semejante, ni jamás presencié combate
alguno entre cuatro negras paredes y a la luz indecisa de una llama
lejana, cuya oscilación proyectaba movibles sombras y espantajos en
nuestro derredor.

Adviértase que la claridad era perjudicial a los franceses, porque a
pesar de guarecerse tras el hueco, nos ofrecían blanco seguro. Nos
tiroteamos un breve rato, y dos compañeros cayeron muertos o mal heridos
sobre el húmedo suelo. A pesar de este desastre, hubo otros que
quisieron llevar adelante aquella aventura, asaltando el agujero e
internándose en la guarida del enemigo; pero aunque este había cesado de
ofendernos, parecía prepararse para atacar mejor. De repente se apagó la
hoguera y quedamos en completa oscuridad. Dimos repetidas vueltas
buscando la salida, y chocábamos unos con otros. Esta situación, junto
con el temor de ser atacados con elementos superiores, o de que
arrojaran en medio de aquel sepulcro granadas de mano, nos obligó a
retirarnos al patio confusamente y en tropel.

Tuvimos tiempo, sin embargo, para buscar a tientas y recoger a los dos
camaradas que habían caído durante la refriega, y luego que salimos,
cerramos la puerta, tabicándola por dentro con piedras, escombros,
vigas, toneles y cuanto en el patio se nos vino a las manos. Al subir,
el que nos mandaba repartió algunos hombres en distintos puntos de la
casa, dejando un par de escuchas en el patio para atender a los golpes
de la zapa enemiga, y a mí me tocó salir fuera con otros, para traer un
poco de comida, que a todos nos hacía muchísima falta.

En la calle nos pareció que de una mansión de tranquilidad pasábamos al
mismo infierno, porque en medio de la noche continuaba el fuego entre
las casas y la muralla. La claridad de la luna permitía correr sin
tropiezo de un punto a otro, y las calles eran a cada instante
atravesadas por escuadrones de tropa y paisanos que iban a donde, según
la voz pública, había verdadero peligro. Muchos, sin entrar en fila y
guiados de su propio instinto, acudían aquí y allí, haciendo fuego desde
el punto que mejor les venía a cuento. Las campanas de todas las
iglesias tocaban a la vez con lúgubre algazara, y a cada paso se
encontraban grupos de mujeres transportando heridos.

Por todas partes, especialmente en el extremo de las calles que
remataban en la muralla de Tenerías, se veían hacinados los cuerpos, y
el herido se confundía con el cadáver, no pudiendo determinarse de qué
boca salían aquellas voces lastimosas que imploraban socorro. Yo no
había visto jamás desolación tan espantosa; y más que el espectáculo de
los desastres causados por el hierro, me impresionó ver en los dinteles
de las casas o arrastrándose por el arroyo en busca de lugar seguro, a
muchos atacados de la epidemia y que se morían por momentos sin tener en
las carnes la más ligera herida. El horroroso frío les hacía dar diente
con diente, e imploraban auxilio con ademanes de desesperación, porque
no podían hablar.

A todas estas, el hambre nos había quitado por completo las fuerzas, y
apenas nos podíamos tener.

---¿Dónde encontraremos algo de comida?---me dijo Agustín.---¿Quién se
va a ocupar de semejante cosa?

---Esto tiene que acabarse pronto de una manera o de
otra---respondí.---O se rinde la ciudad o perecemos todos.

Al fin, hacia las piedras del Coso encontramos una cuadrilla de
Administración que estaba repartiendo raciones, y ávidamente tomamos las
nuestras, llevando a los compañeros todo lo que podíamos cargar. Ellos
lo recibieron con gran algarabía y cierta jovialidad impropia de las
circunstancias; pero el soldado español es y ha sido siempre así.
Mientras comían aquellos mendrugos tan duros como el guijarro, cundió
por el batallón la opinión unánime de que Zaragoza no podía ni debía
rendirse \emph{nunca}.

Era la medianoche, cuando empezó a disminuir el fuego. Los franceses no
conquistaban un palmo de terreno fuera de las casas que ocuparon por la
tarde, aunque tampoco se les pude echar de sus alojamientos. Esta
epopeya se dejaba para los días sucesivos; y cuando los hombres
influyentes de la ciudad: los Montoria, los Cereso, los Sas, los
Salamero y los San Clemente volvían de las Mónicas, teatro aquella noche
de grandes prodigios, manifestaban una confianza enfática y un desprecio
del enemigo, que enardecía el ánimo de cuantos les oían.

---Esta noche se ha hecho poco---decía Montoria.---La gente ha estado
algo floja. Verdad que no había para qué echar el resto, ni debemos
salir de nuestro ten con ten, mientras los franceses nos ataquen con tan
poco brío\ldots{} Veo que hay algunas desgracias\ldots{} poca cosa. Las
monjas han batido bastante aceite con vino, y todo es cuestión de
aplicar unos cuantos parches\ldots{} Si hubiera tiempo, bueno sería
enterrar los muertos de ese montón; pero ya se hará más adelante. La
epidemia crece\ldots{} es preciso dar muchas friegas\ldots{} friegas y
más friegas; es mi sistema. Por ahora, bien pueden pasarse sin caldo; el
caldo es un brebaje repugnante. Yo les daría un trago de aguardiente, y
en poco tiempo podrían tomar el fusil. Con que, señores, la fiesta
parece acabarse por esta noche; descabezaremos un sueño de media hora, y
mañana\ldots{} mañana se me figura que los franceses nos atacarán
formalmente.

Luego encaró con su hijo, que en mi compañía se le acercaba, y continuó
así:

---¡Oh Agustinillo! Ya había preguntado por ti. Pues estaba con cuidado,
porque en acciones como la de hoy suele suceder que muere alguna gente.
¿Estás herido? No, no tienes nada; a ver\ldots{} un simple
rasguño\ldots{} ¡Ah!, ¡chico!, se me figura que no te has portado como
un Montoria. Y Vd., Sr.~de Araceli, ¿ha perdido alguna pierna? Tampoco;
parece que los dos acaban de salir de la fábrica: no les falta ni un
pelo. Malo, malo. Me parece que tenemos aquí un par de gallinas\ldots{}
Ea, a descansar un rato, nada más que un rato. Si se sienten Vds.
atacados de la epidemia, friegas y más friegas\ldots{} es el mejor
sistema\ldots{} Con que, señores, quedamos en que mañana se defenderán
estas casas tabique por tabique. Lo mismo pasa en todo el contorno de la
ciudad; pero en cada alcoba habrá una batalla. Vamos a la capitanía
general, y veremos si Palafox ha acordado lo que pensamos. No hay otro
camino: o entregarles la ciudad, o disputarles cada ladrillo como si
fuera un tesoro. Se aburrirán. Hoy han perdido seis u ocho mil hombres.
Pero vamos a ver al excelentísimo Sr.~D. José\ldots{} Buenas noches,
muchachos, y mañana tratad de sacudir esa cobardía\ldots{}

---Durmamos un poquito---dije a mi amigo, cuando nos quedamos
solos.---Vamos a la casa que estamos guarneciendo, donde me parece que
he visto algunos colchones.

---Yo no duermo---me contestó Montoria, siguiendo por el Coso adelante.

---Ya sé dónde vas. No se nos permitirá alejarnos tanto, Agustín.

Mucha gente, hombres y mujeres, en diversas direcciones, discurrían por
aquella gran vía. De improviso una mujer corrió velozmente hacia
nosotros y abrazó a Agustín sin decirle nada. Profunda emoción ahogaba
la voz en su garganta.

---Mariquilla, Mariquilla de mi corazón---exclamó Montoria, abrazándola
con júbilo.---¿Cómo estás aquí? Iba ahora en busca tuya.

Mariquilla no podía hablar, y sin el sostén de los brazos del amante, su
cuerpo, desmadejado y flojo, hubiera caído al suelo.

---¿Estás enferma? ¿Qué tienes? ¿Por qué lloras? ¿Es cierto que las
bombas han derribado tu casa?

Cierto debía de ser, pues la desgraciada joven mostraba en su desaliñado
aspecto una gran desolación. Su vestido era el que le vimos la noche
anterior. Tenía suelto el cabello y en sus brazos magullados observamos
algunas quemaduras.

---Sí---dijo al fin con apagada voz.---Nuestra casa no existe; no
tenemos nada, lo hemos perdido todo. Esta mañana cuando salistes de
allá, una bomba hundió el techo. Luego cayeron otras dos\ldots{}

---¿Y tu padre?

---Mi padre está allá, y no quiere abandonar las ruinas de la casa. Yo
he estado todo el día buscándote para que nos dieras algún socorro. Me
he metido entre el fuego, he estado en todas las calles del arrabal, he
subido a algunas casas. Creí que habías muerto.

Agustín se sentó en el hueco de una puerta, y abrigando a Mariquilla con
su capote, la sostuvo en sus brazos como se sostiene a un niño. Repuesta
de su desmayo, pudo seguir hablando, y entonces nos dijo que no habían
podido salvar ningún objeto y que apenas tuvieron tiempo para huir. La
infeliz temblaba de frío, y poniéndole mi capote sobre el que ya tenía,
tratamos de llevarla a la casa que guarnecíamos.

---No---dijo.---Quiero volver al lado de mi padre. Está loco de
desesperación y dice mil blasfemias injuriando a Dios y a los santos. No
he podido arrancarle de aquello que fue nuestra casa. Carecemos de
alimento. Los vecinos no han querido darle nada. Si Vds. no quieren
llevarme allá, me iré yo sola.

---No Mariquilla, no, no irás allá---dijo Montoria;---te pondremos en
una de estas casas, donde al menos por esta noche estarás segura, y,
entre tanto Gabriel irá en busca de tu padre, y llevándole algún
alimento, de grado o por fuerza le sacará de allí.

Insistió la Candiola en volver a la calle de Antón Trillo, pero como
apenas tenía fuerzas para moverse, la llevamos en brazos a una casa de
la calle de los Clavos donde estaba Manuela Sancho.

\hypertarget{xix}{%
\chapter{XIX}\label{xix}}

Cesado el fuego de cañón y de fusil, un gran resplandor iluminaba la
ciudad. Era el incendio de la Audiencia que, comenzando cerca de la
media noche, había tomado terribles proporciones y devoraba por sus
cuatro costados aquel hermoso edificio.

Sin atender más que a mi objeto, seguí presuroso hasta la calle de Antón
Trillo. La casa del tío Candiola había estado ardiendo todo el día, y al
fin sofocada la llama entre los escombros de los techos hundidos, de
entre las paredes agrietadas salía negra columna de humo. Los huecos,
perdida su forma, eran unos agujeros irregulares por donde se veía el
cielo, y el ladrillo desmoronado formaba una dentelladura desigual en lo
que fue arquitrabe. Parte del lienzo de pared que daba frente a la
huerta se había venido al suelo, obstruyendo esta en términos que había
desaparecido el antepecho, y la escalerilla de piedra, llegando el
cascajo hasta la misma tapia de la calle. En medio de estas ruinas
subsistía incólume el ciprés, como el pensamiento que permanece vivo al
sucumbir la materia, y alzaba su negra cima como un monumento
conmemorativo.

El portalón estaba destrozado por los hachazos de los que en el primer
momento acudieron a contener el fuego. Cuando penetré en la huerta vi
que hacia la derecha y junto a la reja de una ventana baja había alguna
gente. Aquella parte de la casa era la que se conservaba mejor, pues el
piso bajo no había sufrido casi nada, y el desplome del techo sobre el
principal no había conmovido a este, aunque era de esperar que con el
gran peso se rindiera más o menos pronto.

Acerqueme al grupo, creyendo encontrar a Candiola, y en efecto, allí
estaba sentado junto a la reja, con las manos en cruz, inclinada la
cabeza sobre el pecho y lleno el vestido de jirones y quemaduras. Era
rodeado una pequeña turba de mujeres y chiquillos, que cual abejorros
zumbaban en su alrededor, prodigándole toda clase de insultos y
vejámenes. No me costó gran trabajo ahuyentar tan molesto enjambre, y
aunque no se fueron todos y persistían en husmear por allí, creyendo
encontrar entre las ruinas el oro del rico Candiola, este se vio al fin
libre de los tirones, pedradas, y de las crueles agudezas con que era
mortificado.

---Señor militar---me dijo,---le agradezco a usted que ponga en fuga a
esa vil canalla. Aquí se le quema a uno la casa y nadie le da auxilio.
Ya no hay autoridades en Zaragoza. ¡Qué pueblo, señor, qué pueblo! No
será porque dejemos de pagar gabelas, diezmos y contribuciones.

---Las autoridades no se ocupan más que de las operaciones
militares---le dije;---y son tantas las casas destruidas, que es
imposible acudir a todas.

---¡Maldito sea mil veces---exclamó llevándose la mano a la cabeza
desnuda,-quien nos ha traído estos desastres! Atormentado en el infierno
por mil eternidades no pagaría su culpa. Pero ¿qué demonios busca Vd.
aquí, señor militar? ¿Quiere Vd. dejarme en paz?

---Vengo en busca del Sr.~Candiola---le respondí,---para llevarle a
donde se le pueda socorrer, curando sus quemaduras, y dándole un poco de
alimento.

---¡A mí\ldots! Yo no salgo de mi casa---exclamó con voz lúgubre.---La
junta tendrá que reedificármela. ¿Y a dónde me quiere llevar Vd.?
Ya\ldots{} ya\ldots{} ya estoy en el caso de que me den una limosna. Mis
enemigos han conseguido su objeto, que era ponerme en el caso de pedir
limosna; pero no la pediré, no. Antes me comeré mi propia carne y beberé
mi sangre, que humillarme ante los que me han traído a semejante estado.
¡Ah, miserables! Le quitan a uno su harina para ponerla después en las
cuentas como adquirida a noventa o cien reales. Como que están vendidos
a los franceses, y prolongan la resistencia para redondear sus negocios;
luego les entregan la ciudad y se quedan tan frescos.

---Deje Vd. todas esas consideraciones para otro momento---le dije,---y
sígame ahora, que no está el tiempo para pensar en eso. Su hija de Vd.
ha encontrado donde guarecerse, y a Vd. le daremos asilo en el mismo
lugar.

---Yo no me muevo de aquí. ¿En dónde está mi hija?---preguntó con
pena.---¡Ah! Esa loca no sabe permanecer al lado de su padre en
desgracia. La vergüenza la hace huir de mí. Maldita sea su liviandad y
el momento en que la descubrí. Señor Jesús Nazareno, y tú mi patrono,
Santo Dominguito del Val, decidme: ¿qué he hecho yo para merecer tantas
desgracias en un mismo día? ¿No soy bueno, no hago todo el bien que
puedo, no favorezco a mis semejantes prestándoles dinero con un interés
módico, pongo por caso, la miseria de tres o cuatro reales por peso
fuerte al mes? Y si soy un hombre bueno a carta cabal, ¿a qué llueven
sobre mí tantas desventuras? Y gracias que no pierdo lo poco que a
fuerza de trabajos he reunido, porque está en paraje a donde no pueden
llegar las bombas; pero ¿y la casa, los muebles, y los recibos y lo que
aún queda en el almacén? Maldito sea yo, y cómanme los demonios, si
cuando esto se acabe y cobre los piquillos que por ahí tengo, no me
marcho de Zaragoza para no volver más.

---Nada de eso viene ahora al caso, Sr.~de Candiola. Sígame Vd.

---Sí---dijo con furia,---sí viene al caso. Mi hija se ha envilecido. No
sé cómo no la maté esta mañana. Hasta aquí yo había supuesto a María un
modelo de virtudes y de honestidad; me deleitaba su compañía, y de todos
los buenos negocios destinaba un real para comprarle regalitos. ¡Mal
empleado dinero! Dios mío, tú me castigas por haber despilfarrado un
gran capital en cosas superfluas, cuando a interés compuesto hubiérase
ya triplicado. Yo tenía confianza en mi hija. Esta mañana levanteme al
amanecer; acababa de pedir con fervor a la Virgen del Pilar que me
librara del bombardeo, y tranquilamente abrí la ventana para ver cómo
estaba el día. Póngase Vd. en mi caso, señor militar, y comprenderá mi
asombro y pena al ver dos hombres allí\ldots{} allí, en aquel corredor,
junto al ciprés\ldots{} me parece que les estoy viendo. Uno de ellos
abrazaba a mi hija. Ambos vestían uniforme; no pude verles el rostro
porque aún era escasa la claridad del día\ldots{} Precipitadamente salí
de mi cuarto; pero cuando bajé a la huerta ya los dos estaban en la
calle. Quedose muda mi hija al ver descubierta su liviandad, y leyendo
en mi cara la indignación que tan vil conducta me producía, se arrodilló
delante de mí, pidiéndome perdón. «Infame---le dije ciego de
cólera,---tú no eres hija mía, tú no eres hija de este hombre honrado
que jamás ha hecho mal a nadie. Muchacha loca y sin pudor, no te
conozco, tú no eres mi hija; vete de aquí\ldots{} ¡Dos hombres, dos
hombres en mi casa, de noche, contigo! ¿No has reparado en las canas de
tu anciano padre? ¿No consideras que esos hombres pueden robarme? ¿No
has reparado que la casa está llena de mil objetos de valor, que caben
fácilmente en una faltriquera?\ldots{} ¡Mereces la muerte! Y si no me
engaño, aquellos dos hombres se llevaban alguna cosa. ¡Dos hombres! ¡Dos
novios! ¡Y recibirlos de noche en mi casa, deshonrando a tu padre y
ofendiendo a Dios! ¡Y yo, desde mi cuarto, miraba la luz del tuyo,
creyendo con esto que velabas allí haciendo alguna labor!\ldots{} De
modo, miserable chicuela; de modo hembra despreciable, que mientras tú
estabas en la huerta, en tu cuarto se estaba gastando inútilmente una
vela.»

¡Oh señor militar!, no pude contener mi indignación, y luego que esto le
dije, cogila por un brazo y la arrastré para echarla fuera. En mi cólera
ignoraba lo que hacía. La infeliz me pedía perdón, añadiendo: «Yo le
amo, padre; yo no puedo negar que le amo.» Oyéndola, se redobló mi
furor, y exclamé así: «¡Maldito sea el pan que te he dado en diez y
nueve años! ¡Meter ladrones en mi casa! ¡Maldita sea la hora en que
naciste y malditos los lienzos en que te envolvimos en la noche del 3 de
febrero del año 91! Antes se hundirá el cielo ante mí, y antes me dejará
de su mano la Señora Virgen del Pilar, que volver a ser para ti tu
padre, y tú para mí la Mariquilla a quien tanto he querido.» Apenas dije
esto, señor militar, cuando pareció que todo el firmamento reventaba en
pedazos, cayendo sobre mi casa. ¡Qué espantoso estruendo y qué conmoción
tan horrible! Una bomba cayó en el techo, y en el espacio de cinco
minutos cayeron otras dos. Corrimos adentro; el incendio se propagaba
con voracidad y el hundimiento del techo amenazaba sepultarnos allí.
Quisimos salvar a toda prisa algunos objetos; pero no nos fue posible.
Mi casa, esta casa que compré el año 87, casi de balde, porque fue
embargada a un deudor que me debía cinco mil reales con trece mil y un
pico de intereses, se desmoronaba; se deshacía como un bollo de mazapán,
y por aquí cae una viga, por allí salta un vidrio, por acullá se
desploma una pared. El gato mayaba; doña Guedita me arañó el rostro al
salir de su cuarto; yo me aventuré a entrar en el mío para recoger un
recibito que había dejado sobre la mesa, y estuve a punto de perecer.

Así habló el tío Candiola. Su dolor, además de profunda afección moral,
era como un desorden nervioso, y al instante se comprendía que aquel
organismo estaba completamente perturbado por el terror, el disgusto y
el hambre. Su locuacidad, más que desahogo del alma, era un
desbordamiento impetuoso, y aunque aparentaba hablar conmigo, en
realidad dirigíase a entes invisibles, los cuales, a juzgar por los
gestos de él, también le devolvían alguna palabra. Por esto, sin que yo
le dijera nada, siguió hablando en tono de contestación, y respondiendo
a preguntas que sus ideales interlocutores le hacían.

---Yo he dicho que no me marcharé de aquí mientras no recoja lo mucho
que aún puede salvarse. Pues qué, ¿voy a abandonar mi hacienda? Ya no
hay autoridades en Zaragoza. Si las hubiera, se dispondría que vinieran
aquí cien o doscientos trabajadores a revolver los escombros para sacar
alguna cosa. Pero señor, ¿no hay quien tenga caridad, no hay quien tenga
compasión de este infeliz anciano que nunca ha hecho mal a nadie? ¿Ha de
estar uno sacrificándose toda la vida por los demás para que al llegar
un caso como este no encuentre un brazo amigo que le ayude? No, no
vendrá nadie, y si vienen es por ver si entre las ruinas encuentran
algún dinero\ldots{} ¡Ja, ja, ja!---decía esto riendo como un
demente.---¡Buen chasco se llevan! Siempre he sido hombre precavido, y
ahora, desde que empezó el sitio, puse mis ahorros en lugar tan seguro,
que sólo yo puedo encontrarlo. No, ladrones; no, tramposos; no,
egoístas; no encontraréis un real aunque levantéis todos los escombros y
hagáis menudos pedazos lo que queda de esta casa, aunque piquéis toda la
madera, haciendo con ella palillos de dientes, aunque reduzcáis todo a
polvo, pasándolo luego por un tamiz.

---Entonces, Sr.~de Candiola---le dije tomándole resueltamente por un
brazo para llevarle fuera,---si las peluconas están seguras, ¿a qué
viene el estar aquí de centinela? Vamos fuera.

---¿Cómo se entiende, señor entrometido?---exclamó desasiéndose con
fuerza.---Vaya Vd. noramala, y déjeme en paz. ¿Cómo quiere Vd. que
abandone mi casa, cuando las autoridades de Zaragoza no mandan un
piquete de tropa a custodiarla? Pues qué, ¿cree usted que mi casa no
está llena de objetos de valor? ¿Ni cómo quiere que me marche de aquí
sin sacarlos? ¿No ve Vd. que el piso bajo está seguro? Pues quitando
esta reja se entrará fácilmente, y todo puede sacarse. Si me aparto de
aquí un solo momento, vendrán los rateros, los granujas de la vecindad y
¡ay de mi hacienda, ay del fruto de mi trabajo, ay de los utensilios que
representan cuarenta años de laboriosidad incesante! Mire Vd., señor
militar, en la mesa de mi cuarto hay una palmatoria de cobre que pesa lo
menos tres libras. Es preciso salvarla a toda costa. Si la junta mandara
aquí, como es su deber, una compañía de ingenieros\ldots{}

Pues también hay una vajilla que está en el armario del comedor, y que
debe permanecer intacta. Entrando con cuidado y apuntalando el techo se
la puede salvar. ¡Oh!, sí; es preciso salvar esa desgraciada vajilla. No
es esto sólo, señor militar, señores. En una caja de lata tengo los
recibos: espero salvarlos. También hay un cofre donde guardo dos casacas
antiguas, algunas medias y tres sombreros. Todo esto está aquí abajo y
no ha padecido deterioro. Lo que se pierde irremisiblemente es el ajuar
de mi hija. Sus trajes, sus alfileres, sus pañuelos, sus frascos de agua
de olor podrían valer un dineral, si se vendieran ahora. ¡Cómo se habrá
destrozado todo! ¡Jesús, qué dolor! Verdad es que Dios quiso castigar el
pecado de mi hija, y las bombas se fueron a los frascos de agua de olor.
Pero en mi cuarto quedó sobre la cama mi chupa, en cuyo bolsillo hay
siete reales y diez cuartos. ¡Y no tener yo aquí veinte hombres con
piquetas y azadas\ldots! ¡Dios justo y misericordioso! ¡En qué están
pensando las autoridades de Zaragoza!\ldots{} El candil de dos mecheros
estará intacto. ¡Oh Dios! Es la mejor pieza que ha llevado aceite en el
mundo. Le encontraremos por ahí, levantando con cuidado los escombros
del cuarto de la esquina. Tráiganme una cuadrilla de trabajadores, y
verán qué pronto despacho\ldots{} ¿Cómo quiere que me aparte de aquí?
¡Si me aparto, si me duermo un solo instante, vendrán los
ladrones\ldots{} sí\ldots{} ¡vendrán y se llevarán la palmatoria!

La tenacidad del avaro era tal, que resolví marcharme sin él, dejándole
entregado a su delirante inquietud. Llegó doña Guedita a toda prisa,
trayendo una piqueta y una azada, juntamente con un canastillo en que vi
algunas provisiones.

---Señor---dijo sentándose fatigada y sin aliento,---aquí está la
piqueta y el azadón que me ha dado mi sobrino. Ya no hace falta, porque
no se trabajará más en fortificaciones\ldots{} Aquí están estas pasas
medio podridas y algunos mendrugos de pan.

La dueña comía con avidez. No así Candiola, que despreciando la comida,
cogió la piqueta y resueltamente, como si en su cuerpo hubiera infundido
súbita robustez y energía, empezó a desquiciar la reja. Trabajando con
ardiente actividad, decía:

---Si las autoridades de Zaragoza no quieren favorecerme, doña Guedita,
entre Vd. y yo lo haremos todo. Coja Vd. la azada y prepárese a levantar
el cascajo. Mucho cuidado con las vigas, que todavía humean. Mucho
cuidado con los clavos.

Luego volviéndose a mí, que fijaba la atención en las señas de
inteligencia, hechas por el ama de llaves, me dijo:

---¡Eh! Vaya Vd. noramala. ¿Qué tiene Vd. que hacer en mi casa? ¡Fuera
de aquí! Ya sabemos que viene a ver si puede pescar alguna cosa. Aquí no
hay nada. Todo se ha quemado.

No había, pues, esperanza de llevarle a las Tenerías para tranquilizar a
la pobre Mariquilla, por cuya razón, no pudiendo detenerme más, me
retiré. Amo y criada proseguían con gran ardor su trabajo.

\hypertarget{xx}{%
\chapter{XX}\label{xx}}

Dormí desde las tres al amanecer, y por la mañana oímos misa en el Coso.
En el gran balcón de la casa llamada de las Monas, hacia la entrada de
la calle de las Escuelas Pías ponían todos los domingos un altar y allí
se celebraba el oficio divino pudiéndose ver el sacerdote, por la
situación de aquel edificio, desde cualquier punto del Coso. Semejante
espectáculo era muy conmovedor, sobre todo en el momento de alzar, y
cuando puestos todos de rodillas, se oía un sordo murmullo de extremo a
extremo.

Poco después de terminada la misa, advertí que venía como del mercado un
gran grupo de gente alborotada y gritona. Entre la multitud algunos
frailes pugnaban por apaciguarla; pero ella, sorda a las voces de la
razón, más rugía a cada paso, y en su marcha arrastraba una víctima sin
que fuerza alguna pudiera arrancársela de las manos. Detúvose el pueblo
irritado junto a la subida del Trenque donde estaba la horca, y al poco
rato uno de los dogales de esta suspendió el cuerpo convulso de un
hombre, que se sacudió en el aire hasta quedar exánime. Sobre el madero
apareció bien pronto un cartel que decía: \emph{Por asesino del género
humano, a causa de haber ocultado veinte mil camas.}

Era aquel infeliz un D. Fernando Estallo, guarda almacén de la
Casa-utensilios. Cuando los enfermos y los heridos expiraban en el
arroyo y sobre las frías baldosas de las iglesias, encontrose un gran
depósito de camas, cuya ocultación no pudo justificar el citado Estallo.
Desencadenose impetuosamente sobre él la ira popular y no fue posible
contenerla. Oí decir que aquel hombre era inocente. Muchos lamentaron su
muerte; pero al comenzar el fuego en las trincheras, nadie se acordó más
de él.

Palafox publicó aquel día una proclama, en que trataba de exaltar los
ánimos, y ofrecía el grado de capitán al que se presentara con cien
hombres, amenazando con \emph{pena de horca y confiscación de bienes al
que no acudiese prontamente a los puntos o los desamparase.} Todo esto
era señal del gran apuro de las autoridades.

Memorable fue aquel día por el ataque a Santa Mónica, que defendían los
voluntarios de Huesca. Durante el anterior y gran parte de la noche, los
franceses habían estado bombardeando el edificio. Las baterías de la
huerta estaban inservibles, y fue preciso retirar los cañones, operación
que nuestros valientes llevaron a cabo, sufriendo a descubierto el fuego
enemigo. Este abrió al fin brecha, y penetrando en la huerta, quiso
apoderarse también del edificio, olvidando que había sido rechazado dos
veces en los días anteriores. Pero Lannes contrariado por la
extraordinaria y nunca vista tenacidad de los nuestros, había mandado
reducir a polvo el Convento, lo cual, teniendo morteros y obuses, era
más fácil que conquistarlo. Efectivamente, después de seis horas de
fuego de artillería, una gran parte del muro de Levante cayó al suelo, y
allí era de ver el regocijo de los franceses, que sin pérdida de tiempo
se abalanzaron a asaltar la posición, auxiliados por los fuegos oblicuos
del molino de la ciudad. Viéndoles venir, Villacampa, jefe de los de
Huesca, y Palafox, que había acudido al punto del peligro, trataron de
cerrar la brecha con sacos de lana y unos cajones vacíos que habían
venido con fusiles. Llegando los franceses, asaltaron con furia loca, y
después de un breve choque cuerpo a cuerpo, fueron rechazados. Durante
la noche, siguieron cañoneando el Convento.

Al siguiente día resolvieron dar otro asalto, seguros de que no habría
mortal que defendiese aquel esqueleto de piedra y ladrillo que por
momentos se venía al suelo. Embistiéronlo por la puerta del locutorio;
pero durante la mañana no pudieron conquistar ni un palmo de terreno en
el claustro.

Desplomose al caer de la tarde el techo por la parte oriental del
Convento. El piso tercero, que estaba muy quebrantado no pudo resistir
el peso y cayó sobre el segundo. Este, que era aún más endeble, dejose
ir sobre el principal, y el principal, incapaz por sí solo de resistir
encima todo el edificio, hundiose sobre el claustro, sepultando
centenares de hombres. Parecía natural que los demás se acobardaran con
esta catástrofe; pero no fue así. Los franceses dominaron una parte del
claustro; pero nada más, y para apoderarse de la otra necesitaban
franquearse camino por entre los escombros. Mientras lo hicieron, los de
Huesca, que aún existían, fijaban su alojamiento en la escalera, y
agujereaban el piso del claustro alto, para arrojar granadas de mano
contra los sitiadores.

Entre tanto nuevas tropas francesas logran penetrar por la iglesia,
pasan al techo del Convento, extiéndense por el interior del maderamen
abohardillado, bajan al claustro alto, y atacan a los voluntarios
indomables. Con la algazara de este encuentro, anímanse los de abajo,
redoblan sus esfuerzos, y sacrificando multitud de hombres, consiguen
llegar a la escalera. Los voluntarios se encuentran entre dos fuegos, y
si bien aún pueden retirarse por uno de los dos agujeros practicados en
el claustro alto, casi todos juran morir antes que rendirse. Corren
buscando un lugar estratégico que les permita defenderse con alguna
ventaja, y son cazados a lo largo de las crujías. Cuando sonó el último
tiro fue señal de que había caído el último hombre. Algunos pudieron
salir por un portillo que habían abierto en los más escondidos aposentos
del edificio junto a la ciudad; por allí salió también D. Pedro
Villacampa, comandante del batallón de voluntarios de Huesca, y al
hallarse en la calle, miraba maquinalmente en torno suyo, buscando a sus
muchachos.

Durante esta jornada, nosotros nos hallábamos en las casas inmediatas de
la calle de Palomar, haciendo fuego sobre los franceses que se
destacaban para asaltar el Convento. Antes de concluida la acción,
comprendimos que en las Mónicas ya no había defensa posible, y el mismo
D. José de Montoria que estaba con nosotros lo confesó.

---Los voluntarios de Huesca no se han portado mal---dijo.---Se conoce
que son buenos chicos. Ahora les emplearemos en defender estas casas de
la derecha\ldots{} pero se me figura que no ha quedado ninguno. Allí
sale solo Villacampa. ¿Pues y Mendieta, y Paúl, y Benedicto, y Oliva?
Vamos: veo que todos han quedado en el sitio.

De este modo, el Convento de las Mónicas pasó a poder de Francia.

\hypertarget{xxi}{%
\chapter{XXI}\label{xxi}}

Al llegar a este punto de mi narración ruego al lector que me dispense,
si no puedo consignar concretamente las fechas de lo que refiero. En
aquel período de horrores comprendido desde el 27 de Enero hasta la
mitad del siguiente mes los sucesos se confunden, se amalgaman y se
eslabonan en mi mente de tal modo, que no puedo distinguir días ni
noches, y a veces ignoro si algunos lances de los que recuerdo
ocurrieron a la luz del sol. Me parece que todo aquello pasó en un largo
día, o en una noche sin fin, y que el tiempo no marchaba entonces con
sus divisiones ordinarias. Los acontecimientos, los hombres, las
diversas sensaciones se reúnen en mi memoria formando un cuadro inmenso
donde no hay más líneas divisorias que las que ofrecen los mismos
grupos, el mayor espanto de un momento, la furia inexplicable o el
pánico de otro momento.

Por esta razón no puedo precisar el día en que ocurrió lo que voy a
narrar ahora; pero fue, si no me engaño, al día siguiente de la jornada
de las Mónicas, y según mis conjeturas del 30 de Enero al 2 de Febrero.
Ocupábamos una casa de la calle de Pabostre. Los franceses eran dueños
de la inmediata, y trataban de avanzar por el interior de la manzana
hasta llegar a Puerta Quemada. Nada es comparable a la expedición
laboriosa por dentro de las casas; ninguna clase de guerra, ni las más
sangrientas batallas en campo abierto, ni el sitio de una plaza, ni la
lucha en las barricadas de una calle, pueden compararse a aquellos
choques sucesivos entre el ejército de una alcoba y el ejército de una
sala, entre las tropas que ocupan un piso y las que guarnecen el
superior.

Sintiendo el sordo golpe de las piquetas por diversos puntos, nos
causaba espanto el no saber por qué parte seríamos atacados. Subíamos a
las bohardillas, bajábamos a los sótanos, y pegando el oído a los
tabiques, procurábamos indagar el intento del enemigo según la dirección
de sus golpes. Por último, advertimos que se sacudía con violencia el
tabique de la misma pieza donde nos encontrábamos, y esperamos a pie
firme en la puerta, después de amontonar los muebles formando una
barricada. Los franceses abrieron un agujero, y luego, a culatazos,
hicieron saltar maderos y cascajo, presentándosenos en actitud de querer
echarnos de allí. Éramos veinte. Ellos eran menos, y como no esperaban
ser recibidos de tal manera, retrocedieron volviendo al poco rato en
número tan considerable, que nos hicieron gran daño, obligándonos a
retirarnos, después de dejar tras los muebles cinco compañeros, dos de
los cuales estaban muertos. En el angosto pasillo topamos con una
escalera por donde subimos precipitadamente sin saber a dónde íbamos;
pero luego nos hallamos en un desván, posición admirable para la
defensa. Era estrecha la escalera, y el francés que intentaba pasarla,
moría sin remedio. Así estuvimos un buen rato, prolongando la
resistencia y animándonos unos a otros con vivas y aclamaciones, cuando
el tabique que teníamos a la espalda empezó a estremecerse con fuertes
golpes, y al punto comprendimos que los franceses, abriendo una entrada
por aquel sitio, nos cogerían irremisiblemente entre dos fuegos. Éramos
trece, porque en el desván habían caído dos gravemente heridos.

El tío Garcés que nos mandaba, exclamó furioso:

---¡Recuerno! No nos cogerán esos perros. En el techo hay un tragaluz.
Salgamos por él al tejado. Que seis sigan haciendo fuego\ldots{} al que
quiera subir, partirlo. Que los demás agranden el agujero: fuera miedo y
¡viva la Virgen del Pilar!

Se hizo como él mandaba. Aquello iba a ser una retirada en regla, y
mientras parte de nuestro ejército contenía la marcha invasora del
enemigo, los demás se ocupaban en facilitar el paso. Este hábil plan fue
puesto en ejecución con febril rapidez, y bien pronto el hueco de escape
tenía suficiente anchura para que pasaran tres hombres a la vez, sin que
durante el tiempo empleado en esto ganaran los franceses un solo
peldaño. Velozmente salimos al tejado. Éramos nueve. Tres habían quedado
en el desván y otro fue herido al querer salir, cayendo vivo en poder
del enemigo.

Al encontrarnos arriba saltamos de alegría. Paseamos la vista por los
techos del arrabal, y vimos a lo lejos las baterías francesas. A gatas
avanzamos un buen trecho, explorando el terreno, después de dejar dos
centinelas en el boquete con orden de descerrajar un tiro al que
quisiese escurrirse por él; y no habíamos andado veinte pasos, cuando
oímos gran ruido de voces y risas, que al punto nos parecieron de
franceses. Efectivamente: desde un ancho bohardillón nos miraban riendo
aquellos malditos. No tardaron en hacernos fuego; pero parapetados
nosotros tras las chimeneas y tras los ángulos y recortaduras que allí
ofrecían los tejados, les contestamos a los tiros con tiros y a los
juramentos y exclamaciones con otras mil invectivas que nos inspiraba el
fecundo ingenio del tío Garcés.

Al fin nos retiramos saltando al tejado de la casa cercana. Creímosla en
poder de los nuestros y nos internamos por la ventana de un chiribitil,
considerando fácil el bajar desde allí a la calle, donde unidos y
reforzados con más gente podíamos proseguir aquella aventura al través
de pasillos, escaleras, tejados y desvanes. Pero aún no habíamos puesto
el pie en firme, cuando sentimos en los aposentos que quedaban bajo
nosotros el ruido de repetidas detonaciones.

---Abajo se están batiendo---dijo Garcés,---y de seguro los franceses
que dejamos en la casa de al lado se han pasado a esta, donde se habrán
encontrado con los compañeros. ¡Cuerno, recuerno! Bajemos ahora mismo.
¡Abajo todo el mundo!

Pasando de un desván a otro, vimos una escalera de mano que facilitaba
la entrada a un gran aposento interior, desde cuya puerta se oía vivo
rumor de voces, destacándose principalmente algunas de mujer. El
estruendo de la lucha era mucho más lejano y por consiguiente, procedía
de punto más bajo; franqueando, pues, la escalerilla, nos hallamos en
una gran habitación, materialmente llena de gente, la mayor parte
ancianos, mujeres y niños, que habían buscado refugio en aquel lugar.
Muchos, arrojados sobre jergones, mostraban en su rostro las huellas de
la terrible epidemia, y algún cuerpo inerte sobre el suelo tenía todas
las trazas de haber exhalado el último suspiro pocos momentos antes.

Otros estaban heridos, y se lamentaban sin poder contener la crueldad de
sus dolores; dos o tres viejas lloraban o rezaban. Algunas voces se oían
de rato en rato, diciendo con angustia, «agua, agua.» Desde que bajamos
distinguí en un extremo de la sala al tío Candiola, que ponía
cuidadosamente en un rincón multitud de baratijas, ropas y objetos de
cocina y de loza. Con gesto displicente apartaba a los chicos curiosos
que querían poner sus manos en aquella despreciable quincalla, y lleno
de inquietud, diligente en amontonar y resguardar su tesoro, sin que la
última pieza se le escapase, decía:

---Ya me han quitado dos tazas. Y no me queda duda: alguien de los que
están aquí las ha de tener. No hay seguridad en ninguna parte; no hay
autoridades que le garanticen a uno la posesión de su hacienda. Fuera de
aquí, muchachos mal criados. ¡Oh! Estamos bien\ldots{} ¡Malditas sean
las bombas y quien las inventó! Señores militares, a buena hora llegan
ustedes. ¿No podrían ponerme aquí un par de centinelas para que
guardaran estos objetos preciosos que con gran trabajo logré salvar?

Como es de suponer, mis compañeros se rieron de tan graciosa pretensión.
Ya íbamos a salir, cuando vi a Mariquilla. La infeliz estaba
trasfigurada por el insomnio, el llanto y el terror; pero tanta
desolación en torno suyo y en ella misma, aumentaba la dulce expresión
de su hermoso semblante. Ella me vio, y al punto fue hacia mí con
viveza, mostrando deseo de hablarme.

---¿Y Agustín?---le pregunté.

---Está abajo---repuso con voz temblorosa.---Abajo están dando una
batalla. Las personas que nos habíamos refugiado en esta casa, estábamos
repartidas por los distintos aposentos. Mi padre llegó esta mañana con
doña Guedita. Agustín nos trajo de comer y nos puso en un cuarto donde
había un colchón. De repente sentimos golpes en los tabiques\ldots{}
venían los franceses. Entró la tropa, nos hicieron salir, trajeron los
heridos y los enfermos a esta sala alta\ldots{} aquí nos han encerrado a
todos, y luego, rotas las paredes, los franceses se han encontrado con
los españoles y han empezado a pelear\ldots{} ¡Ay! Agustín está abajo
también\ldots{}

Esto decía, cuando entró Manuela Sancho trayendo dos cántaros de agua
para los heridos. Aquellos desgraciados se arrojaron frenéticamente de
sus lechos, disputándose a golpes un vaso de agua.

---No empujar, no atropellarse, señores---dijo Manuela riendo.---Hay
agua para todos. Vamos ganando. Trabajillo ha costado echarles de la
alcoba, y ahora están disputándose la mitad de la sala, porque la otra
mitad está ya ganada. No nos quitarán tampoco la cocina ni la escalera.
Todo el suelo está lleno de muertos.

Mariquilla se estremeció de horror.

---Tengo sed---me dijo.

Al punto pedí agua a la Sancho; pero como el único vaso que trajera
estaba ocupado en aplacar la sed de los demás, y andaba de boca en boca,
por no esperar, tomé una de las tazas que en su montón tenía el tío
Candiola.

---Eh, señor entrometido---dijo sujetándome la mano,---deje Vd. ahí esa
taza.

---Es para que beba esta señorita---contesté indignado.---¿Tanto valen
estas baratijas, Sr.~Candiola?

El avaro no me contestó, ni se opuso a que diera de beber a su hija; mas
luego que esta calmó su sed, un herido tomó ávidamente de sus manos la
taza, y he aquí que esta empezó a correr también, pasando de boca en
boca. Cuando yo salí para unirme a mis compañeros, D. Jerónimo seguía
con la vista, de muy mal talante, el extraviado objeto que tanto tardaba
en volver a sus manos.

Tenía razón Manuela Sancho al decir que íbamos ganando. Los franceses,
desalojados del piso principal de la casa, habíanse retirado al de la
contigua, donde continuaban defendiéndose. Cuando yo bajé, todo el
interés de la batalla estaba en la cocina, disputada con mucho
encarnizamiento; pero lo demás de la casa nos pertenecía. Muchos
cadáveres de una y otra nación cubrían el ensangrentado suelo; algunos
patriotas y soldados, rabiosos por no poder conquistar aquella cocina
funesta, desde donde se les hacía tanto fuego, lanzáronse dentro de ella
a la bayoneta, y aunque perecieron bastantes, este acto de arrojo
decidió la cuestión, porque tras ellos fueron otros, y por fin todos los
que cabían.

Aterrados los imperiales con tan ruda embestida, buscaron salida
precipitadamente por el laberinto que de pieza en pieza habían abierto.
Persiguiéndolos por pasillos y aposentos, cuya serie inextricable
volvería loco al mejor topógrafo, les rematábamos donde podíamos
alcanzarles, y algunos de ellos se arrojaban desesperadamente a los
patios. De este modo, después de reconquistada aquella casa,
reconquistamos la vecina, obligándolos a contenerse en sus antiguas
posiciones, que eran por aquella parte las dos casas primeras de la
calle de Pabostre.

Después retiramos los muertos y heridos, y tuve el sentimiento de
encontrar entre estos a Agustín de Montoria, aunque no era de gravedad
el balazo recibido en el brazo derecho. Mi batallón quedó aquel día
reducido a la mitad.

Los infelices que se refugiaban en la habitación alta de la casa,
quisieron acomodarse de nuevo en los distintos aposentos; pero esto no
se juzgó conveniente, y fueron obligados a abandonarla, buscando asilo
en lugares más lejanos del peligro.

Cada día, cada hora, cada instante las dificultades crecientes de
nuestra situación militar, se agravaban con el obstáculo que ofrecía
número tan considerable de víctimas, hechas por el fuego y la epidemia.
¡Dichosos mil veces los que eran sepultados en las ruinas de las casas
minadas, como aconteció a los valientes defensores de la calle de Pomar,
junto a Santa Engracia! Lo verdaderamente lamentable estaba allí donde
se hacinaban unos sobre otros sin poder recibir auxilio, multitud de
hombres destrozados por horribles heridas. Había recursos médicos para
la centésima parte de los pacientes. La caridad de las mujeres, la
diligencia de los patriotas, la multiplicación de la actividad en los
hospitales, nada bastaba.

Llegó un día en que cierta impasibilidad, más bien espantosa y cruel
indiferencia se apoderó de los defensores, y nos acostumbramos a ver un
montón de muertos, cual si fuera un montón de sacas de lana; nos
acostumbramos a ver sin lástima largas filas de heridos, arrimados a las
casas, curándose cada cual como mejor podía. A fuerza de padecimientos,
parece que las necesidades de la carne habían desaparecido, y que no
teníamos más vida que la del espíritu. La familiaridad con el peligro
había transfigurado nuestra naturaleza, infundiéndole al parecer un
elemento nuevo, el desprecio absoluto de la materia y total indiferencia
de la vida. Cada uno esperaba morir \emph{dentro de un rato}, sin que
esta idea le conturbara.

Recuerdo que oí contar el ataque dado al Convento de Trinitarios para
arrebatarlo a los franceses; y las hazañas fabulosas, la inconcebible
temeridad de esta empresa, me parecieron un hecho natural y ordinario.

No sé si he dicho que inmediato al Convento de las Mónicas estaba el de
Agustinos observantes, edificio de bastante capacidad, con una iglesia
no pequeña y muy irregular, vastas crujías y un claustro espacioso. Era,
pues, indudable que los franceses, dueños ya de las Mónicas, habrían de
poner gran empeño en poseer también aquel otro monasterio, para
establecerse sólida y definitivamente en el barrio.

---Ya que no tuvimos la suerte de hallarnos en las Mónicas---me dijo
Pirli,---hoy nos daremos el gustazo de defender hasta morir las cuatro
paredes de San Agustín. Como no basta Extremadura para defenderlo, nos
mandan también a nosotros. ¿Y qué hay de grados, amigo Araceli? ¿Con que
es cierto que este par de caballeros que están aquí es un par de
sargentos?

---No sabía nada, amigo Pirli---le respondí, y verdad era que ignoraba
aquel mi ascenso a las alturas jerárquicas del sargentazgo.

---Pues sí: anoche lo acordó el general. El señor de Araceli es sargento
primero y el Sr.~de Pirli sargento segundo. Harto bien lo hemos ganado,
y gracias que nos ha quedado cuerpo en que poner las charreteras.
También me han dicho que a Agustín Montoria le han nombrado teniente por
lo bien que se portó en el ataque dentro de las casas. Ayer tarde al
anochecer, el batallón de las Peñas de San Pedro no tenía más que cuatro
sargentos, un alférez, un capitán y doscientos hombres.

---A ver, amigo Pirli, si hoy nos ganamos un par de ascensos.

---Todo es ganar el ascenso del pellejo---repuso.---Los pocos soldados
que viven del batallón de Huesca, creo que van para generales. Ya tocan
llamada. ¿Tienes qué comer?

---No mucho.

---Manuela Sancho me ha dado cuatro sardinas: las partiré contigo. Si
quieres un par de docenas de garbanzos tostados\ldots{} ¿Te acuerdas tú
del gusto que tiene el vino? Dígolo porque hace días no nos dan una
gota\ldots{} Por ahí corre el rum rum de que esta tarde nos repartirán
un poco cuando acabe la guerra en San Agustín. Ahí tienes tú: sería muy
triste cosa que le mataran a uno antes de saber qué color tiene eso que
van a repartir esta tarde. Si siguieran mi consejo, lo darían antes de
empezar, y así el que muriera, eso se llevaba\ldots{} Pero la junta de
abastos habrá dicho: «hay poco vino; si lo repartimos ahora, apenas
tocarán tres gotas a cada uno. Esperemos a la tarde, y como de los que
defienden a San Agustín será milagro que quede la cuarta parte, les
tocará a trago por barba.»

Y con este criterio siguió discurriendo sobre la escasez de vituallas.
No tuvimos tiempo de entretenernos en esto, porque apenas nos dábamos la
mano con los de Extremadura, que guarnecían el edificio, cuando ved aquí
que una fuerte detonación nos puso en cuidado, y entonces un fraile
apareció diciendo a gritos:

---Hijos míos: han volado la pared medianera del lado de las Mónicas, y
ya les tenemos en casa. Corred a la iglesia; ellos deben de haber
ocupado la sacristía, pero no importa. Si vais a tiempo, seréis dueños
de la nave principal, de las capillas, del coro. ¡Viva la Santa Virgen
del Pilar y el batallón de Extremadura!

Marchamos a la iglesia serenos y confiados.

\hypertarget{xxii}{%
\chapter{XXII}\label{xxii}}

Los buenos padres nos animaban con sus exhortaciones, y alguno de ellos,
confundiéndose con nosotros en lo más apretado de las filas nos decía:

---Hijos míos, no desmayéis. Previendo que llegaría este caso, hemos
conservado un mediano número de víveres en nuestra despensa. También
tenemos vino. Sacudid el polvo a esa canalla. Ánimo, jóvenes queridos.
No temáis el plomo enemigo. Más daño hacéis vosotros con una de vuestras
miradas, que ellos con una descarga de metralla. Adelante, hijos míos.
La Santa Virgen del Pilar es entre vosotros. Cerrad los ojos al peligro,
mirad con serenidad al enemigo y entre las nubes veréis la santa figura
de la madre de Dios. ¡Viva España y Fernando VII!

Llegamos a la iglesia; pero los franceses que habían entrado por la
sacristía, se nos adelantaron, y ya ocupaban el altar mayor. Yo no había
visto jamás una mole churrigueresca, cuajada de esculturas y follajes de
oro, sirviendo de parapeto a la infantería; yo no había visto que
vomitasen fuego los mil nichos, albergue de mil santos de ebanistería;
yo no había visto nunca que los rayos de madera dorada, que fulminan su
llama inmóvil desde los huecos de una nube de cartón poblada de
angelitos, se confundieran con los fogonazos, ni que tras los pies del
Santo Cristo, y tras el nimbo de oro de la Virgen María, el ojo
vengativo del soldado atisbara el blanco de su mortífera puntería.

Baste deciros que el altar mayor de San Agustín era una gran fábrica de
entalle dorado, cual otras que habréis visto en cualquier templo de
España. Este armatoste se extendía desde el piso a la bóveda, y de
machón a machón, representando en sucesivas hileras de nichos como una
serie de jerarquías celestiales. Arriba el Cristo ensangrentado abría
sus brazos sobre la cruz, abajo y encima del altar, un templete
encerraba el símbolo de la Eucaristía. Aunque la mole se apoyaba en el
muro del fondo, había pequeños pasadizos interiores, destinados al
servicio casero de aquella república de santos, y por ellos el lego
sacristán podía subir desde la sacristía a mudar el traje de la Virgen,
a encender las velas del altísimo Crucifijo, o a limpiar el polvo que
los siglos depositaban sobre el antiguo tisú de los vestidos y la madera
bermellonada de los rostros.

Pues bien, los franceses se posesionaron rápidamente del camarín de la
Virgen, de los estrechos tránsitos que he mencionado; y cuando nosotros
llegamos, en cada nicho, detrás de cada santo, y en innumerables
agujeros abiertos a toda prisa, brillaba el cañón de los fusiles.
Igualmente establecidos detrás del ara santa, que a empujones
adelantaron un poco, se preparaban a defender en toda regla la cabecera
de la iglesia.

Nosotros no estábamos enteramente a descubierto, y para resguardarnos
del gran retablo, teníamos los confesonarios, los altares de las
capillas y las tribunas. Los más expuestos éramos los que entramos por
la nave principal; y mientras los más osados avanzaron resueltamente
hacia el fondo, otros tomamos posiciones en el coro bajo, y tras el
facistol, tras las sillas y bancos amontonados contra la reja,
molestando desde allí con certera puntería a la nación francesa,
posesionada del altar mayor.

El tío Garcés, con otros nueve de igual empuje, corrió a posesionarse
del púlpito, otra pesada fábrica churrigueresca, cuyo guarda-polvo,
coronado por una estatua de la fe, casi llegaba al techo. Subieron,
ocupando la cátedra y la escalera, y desde allí con singular acierto
dejaban seco a todo francés que abandonando el presbiterio se adelantaba
a lo bajo de la iglesia. También sufrían ellos bastante, porque les
abrasaban los del altar mayor, deseosos de quitar de enmedio aquel
obstáculo. Al fin se destacaron unos veinte hombres, resueltos a tomar a
todo trance aquel reducto de madera, sin cuya posesión era locura
intentar el paso de la nave. No he visto nada más parecido a una gran
batalla, y así como en ésta la atención de uno y otro ejército se
reconcentra a veces en un punto, el más disputado y apetecido de todos,
y cuya pérdida o conquista decide el éxito de la lucha, así la atención
de todos se dirigió al púlpito, tan bien defendido como bien atacado.

Los veinte tuvieron que resistir el vivísimo fuego que se les hacía
desde el coro, y la explosión de las granadas de mano que los de las
tribunas les arrojaban; pero, a pesar de sus grandes pérdidas, avanzaron
resueltamente a la bayoneta sobre la escalera. No se acobardaron los
diez defensores del fuerte, y defendiéronse a arma blanca con aquella
superioridad infalible que siempre tuvieron en este género de lucha.
Muchos de los nuestros, que antes hacían fuego parapetados tras los
altares y los confesionarios, corrieron a atacar a los franceses por la
espalda, representando de este modo en miniatura la peripecia de una
vasta acción campal; y trabose la contienda cuerpo a cuerpo a
bayonetazos, a tiros y a golpes, según como cada cual cogía a su
contrario.

De la sacristía salieron mayores fuerzas enemigas, y nuestra
retaguardia, que se había mantenido en el coro, salió también. Algunos
que se hallaban en las tribunas de la derecha, saltaron fácilmente al
cornisamento de un gran retablo lateral, y no satisfechos con hacer
fuego desde allí, desplomaron sobre los franceses tres estatuas de
santos que coronaban los ángulos del ático. En tanto el púlpito se
sostenía con firmeza, y en medio de aquel infierno, vi al tío Garcés
ponerse en pie, desafiando el fuego, y accionar como un predicador,
gritando desaforadamente con voz ronca. Si alguna vez viera al demonio
predicando el pecado en la cátedra de una iglesia, invadida por todas
las potencias infernales en espantosa bacanal, no me llamaría la
atención.

Aquello no podía prolongarse mucho tiempo, y Garcés, atravesado por cien
balazos, cayó de improviso lanzando un ronco aullido. Los franceses, que
en gran número llenaban la sacristía, vinieron en columna cerrada, y en
los tres escalones que separan el presbiterio del resto de la iglesia,
nos presentaron un muro infranqueable. La descarga de esta columna
decidió la cuestión del púlpito, y quintados en un instante, dejando
sobre las baldosas gran número de muertos, nos retiramos a las capillas.
Perecieron los primitivos defensores del púlpito, así como los que luego
acudieron a reforzarlos, y al tío Garcés, acribillado a bayonetazos
después de muerto, le arrojaron en su furor los vencedores por encima
del antepecho. Así concluyó aquel gran patriota que no nombra la
historia.

El capitán de nuestra compañía quedó también inerte sobre el pavimento.
Retirándonos desordenadamente a distintos puntos, separados unos de
otros, no sabíamos a quién obedecer; bien es verdad que allí la
iniciativa de cada uno o de cada grupo de dos o tres era la única
organización posible, y nadie pensaba en compañías ni en jerarquías
militares. Había la subordinación de todos al pensamiento común, y un
instinto maravilloso para conocer la estrategia rudimentaria que las
necesidades de la lucha a cada instante nos iba ofreciendo. Este
instintivo golpe de vista nos hizo comprender que estábamos perdidos,
desde que nos metimos en las capillas de la derecha, y era temeridad
persistir en la defensa de la iglesia ante las enormes fuerzas francesas
que la ocupaban.

Algunos opinaron que con los bancos, las imágenes y la madera de un
retablo viejo, que fácilmente podía ser hecho pedazos, debíamos levantar
una barricada en el arco de la capilla y defendernos hasta lo último;
pero dos padres agustinos se opusieron a este esfuerzo inútil, y uno de
ellos nos dijo:

---Hijos míos, no os empeñéis en prolongar la resistencia, que os
llevaría a perder vuestras vidas sin ventaja alguna. Los franceses están
atacando en este instante el edificio por la calle de las Arcadas.
Corred allí a ver si lográis atajar sus pasos; pero no penséis en
defender la iglesia, profanada por esos cafres.

Estas exhortaciones nos obligaron a salir al claustro, y todavía
quedaban en el coro algunos soldados de Extremadura tiroteándose con los
franceses que ya invadían toda la nave.

Los frailes sólo cumplieron a medias su oferta en lo de darnos algún
\emph{gaudeamus}, como recompensa por haberles defendido hasta el último
extremo su iglesia, y fueron repartidos algunos trozos de tasajo y pan
duro; sin que viéramos ni oliéramos el vino en ninguna parte, por más
que alargamos la vista y las narices. Para explicar esto dijeron que los
franceses, ocupando todo lo alto, se habían posesionado del principal
depósito de provisiones, y lamentándose del suceso procuraron
consolarnos con alabanzas de nuestro buen comportamiento.

La falta del vino prometido hízome acordar del gran Pirli, y entonces
caí en la cuenta de que le había visto al principio del lance en una de
las tribunas. Pregunté por él; pero nadie me sabía dar razón de su
paradero.

Ocupaban los franceses la iglesia y también parte de los altos del
Convento. A pesar de nuestra desfavorable posición en el claustro bajo,
estábamos resueltos a seguir resistiendo, y traíamos a la memoria la
heroica conducta de los voluntarios de Huesca, que defendieron las
Mónicas hasta quedar sepultados bajo sus escombros. Estábamos delirantes
y ebrios: nos creíamos ultrajados si no vencíamos, y nos impulsaba a las
luchas desesperadas una fuerza secreta, irresistible, que no me puedo
explicar sino por la fuerte tensión erectiva del espíritu y una
aspiración poderosa hacia lo ideal.

Nos contuvo una orden venida de fuera, y que dictó sin duda, en su buen
sentido práctico el general Saint-March.

---El Convento no se puede sostener---dijeron.---Antes que sacrificar
gente sin provecho alguno para la ciudad, salgan todos a defender los
puntos atacados en la calle de Pabostre y Puerta Quemada, por donde el
enemigo quiere extenderse, conquistando las casas de que se le ha
rechazado varias veces.

Salimos, pues, de San Agustín. Cuando pasábamos por la calle del mismo
nombre, paralela a la de Palomar, vimos que desde la torre de la
iglesia, arrojaban granadas de mano sobre los franceses establecidos en
la plazoleta inmediata a la última de aquellas dos vías. ¿Quién lanzaba
aquellos proyectiles desde la torre? Para decirlo más brevemente y con
más elocuencia, abramos la historia y leamos: «En la torre se habían
situado y pertrechado siete u ocho paisanos con víveres y municiones
para hostigar al enemigo, y subsistieron verificándolo por unos días sin
querer rendirse.»

Allí estaba el insigne Pirli. ¡Oh Pirli! Más feliz que el tío Garcés, tú
ocupas un lugar en la historia.

\hypertarget{xxiii}{%
\chapter{XXIII}\label{xxiii}}

Incorporados al batallón de Extremadura, se nos llevó por la calle de
Palomar hasta la plaza de la Magdalena, desde donde oímos fuerte
estrépito de combate hacia el extremo de la calle de Puerta Quemada.
Como nos habían dicho, el enemigo procuraba extenderse por la calle de
Pabostre para apoderarse de Puerta Quemada, punto importantísimo que le
permitía enfilar con su artillería la calle del mismo nombre hasta la
plaza de la Magdalena; y como la posesión de San Agustín y las Mónicas,
les permitía amenazar aquel punto céntrico por el fácil tránsito de la
calle de Palomar, ya se conceptuaban dueños del arrabal. En efecto, si
los de San Agustín lograban avanzar hasta las ruinas del Seminario, y
los de la calle de Pabostre hasta Puerta Quemada, era imposible disputar
a los franceses el barrio de Tenerías.

Después de una breve espera, nos llevaron a la calle de Pabostre, y como
la lucha era combinada entre el interior de los edificios y la vía
pública, entramos por la calle de los Viejos a la primera manzana. Desde
las ventanas de la casa en que nos situaron no se veía más que humo, y
apenas podíamos hacernos cargo de lo que allí estaba pasando; mas luego
advertí que la calle estaba llena de zanjas y cortaduras de trecho en
trecho, con parapetos de tierra, muebles y escombros. Desde las ventanas
se hacía un fuego horroroso. Recordando una frase del mendigo cojo
\emph{Sursum Corda}, puedo decir que nuestra alma era toda balas. En el
interior de las casas corría la sangre a torrentes. El empuje de la
Francia era terrible; y para que la resistencia no fuese menor, las
campanas convocaban sin cesar al pueblo, los generales dictaban órdenes
crueles para castigar a los rezagados; los frailes reunían gente de los
otros barrios, trayéndola como en traílla, y algunas mujeres heroicas
daban el ejemplo, arrojándose en medio del peligro, fusil en mano.

Día horrendo, cuyo rumor pavoroso retumba sin cesar en los oídos del que
lo presenció, cuyo recuerdo le persigue, pesadilla indeleble de toda la
vida. Quien no vio sus excesos, quien no oyó su vocerío y estruendo,
ignora con que aparato externo se presenta a los sentidos humanos el
ideal del horror. Y no me digáis que habéis visto el cráter de un volcán
en lo más recio de sus erupciones, o una furiosa tempestad en medio del
Océano, cuando la embarcación, lanzada al cielo por una cordillera de
agua, cae después al abismo vertiginoso; no me digáis que habéis visto
eso, pues nada de eso se parece a los volcanes y a las tempestades que
hacen estallar los hombres, cuando sus pasiones les llevan a eclipsar
los desórdenes de la Naturaleza.

Era difícil contenernos, y no pudiendo hacer gran hostilidad desde allí,
bajamos a la calle unos tras otros, sin hacer caso de los jefes que
querían contenernos. El combate tenía sobre todos una atracción
irresistible, y nos llamaba como llama el abismo al que le mira desde el
vértice de elevada cima.

Jamás me he considerado héroe; pero es lo cierto que en aquellos
momentos ni temía la muerte, ni me arredraba el espectáculo de las
catástrofes que a mi lado veía. Verdad es que el heroísmo, como cosa del
momento e hijo directo de la inspiración, no pertenece exclusivamente a
los valerosos, razón por la cual suele encontrarse con frecuencia en las
mujeres y en los cobardes.

Por no parecer prolijo no referiré aquí las peripecias de aquel combate
de la calle de Pabostre. Se parecen mucho a las que antes he contado, y
si en algo se diferenciaron fue por el exceso de la constancia y de la
energía, llevadas a un grado tal que allí acababa lo humano y empezaba
lo divino. Dentro de las casas pasaban escenas como las que en otro
lugar he referido; pero con mayor encarnizamiento, porque el triunfo se
creía más definitivo. La ventaja adquirida en una pieza, perdíanla los
imperiales en otra; la acción trabada en la bohardilla descendía peldaño
por peldaño hasta el sótano, y allí se remataba al arma blanca, con
ventaja siempre para los paisanos. Las voces de mando con que unos y
otros dirigían los movimientos dentro de aquellos laberintos, retumbaban
de pieza en pieza con ecos espantosos.

En la calle usaban ellos artillería y nosotros también. Varias veces
trataron de apoderarse con rápidos golpes de mano de nuestras piezas;
pero perdían mucha gente sin conseguirlo nunca. Acobardados al ver que
el esfuerzo empleado otra vez para ganar una batalla no bastaba entonces
para conquistar dos varas de calle, se negaban a batirse, y sus
oficiales les sacudían a palos la pereza.

Por nuestra parte no era preciso emplear tales medios, y bastaba la
persuasión. Los frailes, sin dejar de prestar auxilio a los moribundos,
atendían a todo, y al advertir debilidad en un punto, volaban a llamar
la atención de los jefes.

En una de las zanjas abiertas en la calle, una mujer, más que ninguna
valerosa, Manuela Sancho, después de hacer fuego de fusil, disparó
varios tiros en la pieza de a 8. Mantúvose ilesa, durante gran parte del
día, animando a todos con sus palabras, y sirviendo de ejemplo a los
hombres; pero serían las tres de la tarde cuando cayó en la zanja,
herida en una pierna, y durante largo tiempo confundiose con los
muertos, porque la hemorragia la puso exánime y con apariencia de
cadáver. Más tarde, advirtiendo que respiraba, la retiramos, y fue
curada, quedando tan bien, que muchos años después tuve el gusto de
verla viva. La Historia no ha olvidado a aquella valiente joven y
además, la calle de Pabostre, cuyas mezquinas casas son más elocuentes
que las páginas de un libro, lleva el nombre de \emph{Manuela Sancho}.

Poco después de las tres, una horrísona explosión conmovió las casas que
los franceses nos habían disputado tan encarnizadamente durante la
mañana, y entre el espeso humo y el polvo, más espeso aún que el humo,
vimos volar en pedazos mil las paredes y el techo, cayendo todo al suelo
con un estruendo de que no puede darse idea. Los franceses empezaban a
emplear la mina para conquistar lo que por ningún otro medio podía
arrancarse de las manos aragonesas. Abrieron galerías, cargaron los
hornillos, y los hombres cruzáronse de brazos, esperando que la pólvora
lo hiciera todo.

Cuando reventó la primera casa, nos mantuvimos serenos en las inmediatas
y en la calle; pero cuando con estallido más fuerte aún vino a tierra la
segunda, iniciose el movimiento de retirada con bastante desorden. Al
considerar que eran sepultados entre las ruinas o lanzados al aire
tantos infelices compañeros que no se habrían dejado vencer por la
fuerza del brazo, nos sentimos débiles para luchar con aquel elemento de
destrucción, y parecíanos que en todas las demás casas y en la calle,
minadas ya también, iban a estallar horribles cráteres que en pedazos
mil nos salpicarían desgarrados en sangrientos girones.

Los jefes nos detenían diciendo:

---Firmes, muchachos. No correr. Eso es para asustaros. Nosotros también
tenemos pólvora en abundancia, y abriremos minas. ¿Creéis que eso les
dará ventaja? Al contrario. Veremos cómo se defienden entre los
escombros.

Palafox se presentó a la entrada de la calle, y su presencia nos contuvo
algún tanto. El mucho ruido impidiome oír lo que nos dijo; pero por sus
gestos comprendí que quería impelernos a marchar sobre las ruinas.

---Ya oís, muchachos; ya oís lo que dice el capitán general---vociferó a
nuestro lado un fraile de los que venían en la comitiva de
Palafox.---Dice que si hacéis un pequeño esfuerzo más, no quedará vivo
un solo francés.

---¡Y tiene razón!---exclamó otro fraile.---No habrá en Zaragoza una
mujer que os mire, si al punto no os arrojáis sobre las ruinas de las
casas y echáis de allí a los franceses.

---Adelante, hijos de la Virgen del Pilar---añadió un tercer
fraile.---Allí hay un grupo de mujeres. ¿Las veis? Pues dicen que si no
vais vosotros, irán ellas. ¿No os da vergüenza vuestra cobardía?

Con esto nos contuvimos un poco. Reventó otra casa a la derecha, y
entonces Palafox se internó en la calle. Sin saber cómo ni por qué, nos
llevaba tras sí. Y ahora es ocasión de hablar de este personaje
eminente, cuyo nombre va unido al de las célebres proezas de Zaragoza.
Debía en gran parte su prestigio a su gran valor; pero también a la
nobleza de su origen, al respeto con que siempre fue mirada allí la
familia de Lazán y a su hermosa y arrogante presencia. Era joven. Había
pertenecido al Cuerpo de Guardias, y se le elogiaba mucho por haber
despreciado los favores de una muy alta señora, tan famosa por su
posición como por sus escándalos. Lo que más que nada hacía simpático al
caudillo zaragozano era su indomable y serena valentía, aquel ardor
juvenil con que acometía lo más peligroso y difícil, por simple afán de
tocar un ideal de gloria.

Si carecía de dotes intelectuales para dirigir obra tan ardua como
aquella, tuvo el acierto de reconocer su incompetencia, y rodeose de
hombres insignes por distintos conceptos. Estos lo hacían todo, y
Palafox quedábase tan sólo con lo teatral. Sobre un pueblo en que tanto
prevalece la imaginación, no podía menos de ejercer subyugador dominio
aquel joven general, de ilustre familia y simpática figura, que se
presentaba en todas partes reanimando a los débiles y distribuyendo
recompensas a los animosos.

Los zaragozanos habían simbolizado en él sus virtudes, su constancia, su
patriotismo ideal con ribetes de místico y su fervor guerrero. Lo que él
disponía, todos lo encontraban bueno y justo. Como aquellos monarcas a
quienes las tradicionales leyes han hecho representación personal de los
principios fundamentales del gobierno, Palafox no podía hacer nada malo:
lo malo era obra de sus consejeros. Y en realidad, el ilustre caudillo
reinaba y no gobernaba. Gobernaban el padre Basilio, O'Neilly,
Saint-March y Butrón, clérigo escolapio el primero, generales insignes
los otros tres.

En los puntos de peligro aparecía siempre Palafox como la expresión
humana del triunfo. Su voz reanimaba a los moribundos, y si la Virgen
del Pilar hubiera hablado, no hubiera hablado por otra boca. Su rostro
expresaba siempre una confianza suprema, y en él la triunfal sonrisa
infundía coraje como en otros el ceño feroz. Vanagloriábase de ser el
impulsor de aquel gran movimiento. Como comprendía por instinto que
parte del éxito era debido, más que a lo que tenía de general a lo que
tenía de actor, siempre se presentaba con todos sus arreos de gala,
entorchados, plumas y veneras, y la atronadora música de los aplausos y
los vivas le halagaban en extremo. Todo esto era preciso, pues ha de
haber siempre algo de mutua adulación entre la hueste y el caudillo para
que el enfático orgullo de la victoria arrastre a todos al heroísmo.

\hypertarget{xxiv}{%
\chapter{XXIV}\label{xxiv}}

Como he dicho, Palafox nos detuvo, y aunque abandonamos casi toda la
calle de Pabostre, nos mantuvimos firmes en Puerta Quemada.

Si encarnizada fue la batalla hasta las tres, hora en que nos
concentramos hacia la plaza de la Magdalena, no lo fue menos desde dicha
ocasión hasta la noche. Los franceses empezaron a hacer trabajos en las
casas arruinadas por los hornillos, y era curioso ver cómo entre las
masas de cascote y vigas se abrían pequeñas plazas de armas, caminos
cubiertos y plataformas para emplazar la artillería. Aquella era una
guerra que cada vez se iba pareciendo menos a las demás guerras
conocidas.

De esta nueva fase de batalla resultó una ventaja, y un inconveniente
para los franceses, porque si la demolición de las casas les permitía
colocar en ellas algunas piezas, en cambio los hombres quedaban a
descubierto. Por nuestra desgracia no supimos aprovecharnos de esto al
presenciar las voladuras. El terror nos hizo ver una centuplicación del
peligro, cuando en realidad lo disminuía, y no queriendo ser menos que
ellos en aquel duelo a fuego, los zaragozanos empezaron a incendiar las
casas de la calle de Pabostre que no podían sostener.

Sitiadores y sitiados, deseosos de rematarse pronto, y no pudiendo
conseguirlo en la laberíntica guerra de las madrigueras, empezaron a
destruirlas unos con la mina otros con el incendio, quedándose a
descubierto como el impaciente gladiador que arroja su escudo.

¡Qué tarde, qué noche! Al llegar aquí me detengo cansado y sin aliento,
y mis recuerdos se nublan, como se nublaron mi pensar y mi sentir en
aquella tarde espantosa. Hubo, pues, un momento, en que no pudiendo
resistir más, mi cuerpo, como el de otros compañeros que habían tenido
la suerte o la desgracia de vivir, se arrastraba sobre el arroyo
tropezando con cadáveres insepultos o medio inhumados entre los
escombros. Mis sentidos, salvajemente lanzados a los extremos del
delirio, no me representaban claramente el lugar donde me encontraba, y
la noción del vivir era un conjunto de vagas confusiones, de dolores
inauditos. No me parecía que fuese de día, porque en algunos puntos
lóbrega oscuridad envolvía la escena; mas tampoco me consideraba en
medio de la noche, porque llamas semejantes a las que suponemos en el
infierno, enrojecían la ciudad por otro lado.

Sólo sé que me arrastraba pisando cuerpos, yertos unos, con movimiento
otros, y que más allá, siempre más allá, creía encontrar un pedazo de
pan y un buche de agua. ¡Qué desfallecimiento tan horrible! ¡Qué hambre!
¡Qué sed! Vi correr a muchos con ágiles movimientos, les oí gritar, vi
proyectadas sus inquietas sombras formando espantajos sobre las paredes
cercanas; iban y venían no sé a dónde ni de dónde. No era yo el único
que agotadas las fuerzas del cuerpo y del espíritu después de tantas
horas de lucha, se había rendido. Otros muchos, que no tenían la acerada
entereza de los cuerpos aragoneses, se arrastraban como yo, y nos
pedíamos unos a otros un poco de agua. Algunos, más felices que los
demás, tuvieron fuerza para registrar entre los cadáveres, y recoger
mendrugos de pan, piltrafas de carne fría y envuelta en tierra, que
devoraban con avidez.

Algo reanimados, seguimos buscando, y pude alcanzar una parte en las
migajas de aquel festín. No sé si estaba yo herido: algunos de los que
hablaban conmigo comunicándome su gran hambre y sed, tenían horribles
golpes, quemaduras y balazos. Por fin encontramos unas mujeres que nos
dieron a beber agua fangosa y tibia. Nos disputamos el vaso de barro, y
luego en las manos de un muerto, descubrimos un pañuelo liado que
contenía dos sardinas secas y algunos bollos de aceite. Alentados por
los repetidos hallazgos, seguimos merodeando, y al fin, lo poco que
logramos comer, y más que nada el agua sucia que bebimos nos devolvió en
parte las fuerzas.

Yo me sentí con algún brío y pude andar, aunque difícilmente. Advertí
que todo mi vestido estaba lleno de sangre, y sintiendo un vivo escozor
en el brazo derecho, juzgueme gravemente herido; pero aquel malestar era
de una contusión insignificante, y las manchas de mis ropas provenían de
haberme arrastrado entre charcos de fango y sangre.

Volví a pensar sin confusiones, volví a ver sin oscuridad, y oí
distintamente los gritos, los pasos precipitados, los cañonazos cercanos
y distantes en diálogo pavoroso. Sus estampidos aquí y allí parecían
preguntas y respuestas.

Los incendios continuaban. Había sobre la ciudad una densa niebla,
formada de polvo y humo, la cual con el resplandor de las llamas,
formaba perspectivas horrorosas que jamás se ven en el mundo; en sueños
sí. Las casas despedazadas con sus huecos abiertos a la claridad como
ojos infernales, las recortaduras angulosas de las ruinas humeantes, las
vigas encendidas, eran espectáculo menos siniestro que el de aquellas
figuras saltonas e incansables, que no cesaban de revolotear allí
delante, allí mismo, casi en medio de las llamas. Eran los paisanos de
Zaragoza que aún se estaban batiendo con los franceses, y les disputaban
ferozmente un palmo de Infierno.

Me encontraba en la calle de Puerta Quemada, y lo que he descrito se
veía en las dos direcciones opuestas del Seminario y de la entrada de la
calle de Pabostre. Di algunos pasos, pero caí otra vez rendido de
fatiga. Un fraile, viéndome cubierto de sangre, se me acercó, y empezó a
hablarme de la otra vida y del premio eterno destinado a los que mueren
por la patria. Díjele que no estaba herido; pero que el hambre, el
cansancio y la sed me habían postrado, y que creía tener los primeros
síntomas de la epidemia. Entonces el buen religioso, en quien al punto
reconocí al padre Mateo del Busto, se sentó a mi lado y dijo exhalando
un hondo suspiro:

---Yo tampoco me puedo tener y creo que me muero.

---¿Está Vuestra Paternidad herido?---le pregunté viendo un lienzo atado
a su brazo derecho.

---Sí, hijo mío; una bala me ha destrozado el brazo y el hombro. Siento
grandísimo dolor; pero es preciso aguantarlo. Más padeció Cristo por
nosotros. Desde que amaneció no he cesado de curar heridos, y encaminar
moribundos al cielo. En diez y seis horas no he descansado un solo
momento, ni comido ni bebido cosa alguna. Una mujer me ató este lienzo
en el brazo derecho, y seguí mi tarea. Creo que no viviré mucho\ldots{}
¡Cuánto muerto, Dios mío! ¿Y estos heridos que nadie recoge\ldots? Pero
¡ay! yo no puedo tenerme en pie, yo me muero. ¿Has visto aquella zanja
que hay al fin de la calle de los Clavos? Pues allí yace sin vida el
desgraciado Coridón. Fue víctima de su arrojo. Pasábamos por allí para
recoger unos heridos, cuando vimos hacia las eras de San Agustín un
grupo de franceses que pasaban de una casa a otra. Coridón, cuya sangre
impetuosa le impele a los actos más heroicos, se lanzó ladrando sobre
ellos. ¡Ay!, ensartándole en una bayoneta, le arrojaron exánime dentro
de la zanja\ldots{} ¡Cuántas víctimas en un solo día, Sr.~de Araceli!
Pues no tiene Vd. poca suerte en haber salido ileso. Pero se morirá Vd.
de la epidemia, que es peor. Hoy he dado la absolución a sesenta
moribundos de la epidemia. A Vd. también se la daré, amigo, porque sé
que no comete pecadillos y que se ha portado valientemente en estos
días\ldots{} ¿Qué tal? ¿Crece el mal? Efectivamente, está Vd. más
amarillo que esos cadáveres que nos rodean. Morir de la epidemia durante
el horroroso cerco, también es morir por la patria. Joven, ánimo: el
cielo se abre para recibirle a Vd. y la virgen del Pilar le agasajará
con su manto de estrellas. La vida no vale nada. ¡Cuánto mejor es morir
honrosamente y ganar con el padecer de un día la eterna gloria! En
nombre de Dios le perdono a Vd. todos sus pecados.

Después de murmurar la oración propia del caso, pronunció,
bendiciéndome, el \emph{ego te absolvo}, y extendiéndose luego cuan
largo era sobre el suelo. Su aspecto era tristísimo, y aunque yo no me
encontraba bien, juzgueme en mejor estado de salud que el buen fraile.
No fue aquella la primera ocasión en que el confesor caía antes que el
moribundo, y el médico antes que el enfermo. Llamé al padre Mateo, y
como no me respondiera sino con lastimeros quejidos, aparteme de allí
para buscar quien fuese en su ayuda. Encontré a varios hombres y
mujeres, y les dije:

---Ahí está el padre fray Mateo del Busto, que no puede moverse.

Pero no me hicieron caso, y siguieron adelante. Muchos heridos me
llamaban a su vez, pidiéndome que les diese auxilio; pero yo tampoco les
hacía caso. Junto al Coso encontré un niño de ocho o diez años, que
marchaba solo y llorando con el mayor desconsuelo. Le detuve, le
pregunté por sus padres, y señaló un punto cercano, donde había gran
número de muertos y heridos.

Más tarde encontré al mismo niño en diversos puntos, siempre solo,
siempre llorando, y nadie se cuidaba de él.

No se oía otra cosa que las preguntas \emph{¿Has visto a mi hermano?
¿Has visto a mi hijo? ¿Has visto a mi padre?} Pero mi hermano, mi hijo y
mi padre no parecían por ninguna parte. Ya nadie se cuidaba de llevar
los enfermos a las iglesias, porque todas o casi todas estaban
atestadas. Los sótanos y cuartos bajos, que antes se consideraron buenos
refugios, ofrecían una atmósfera infesta y mortífera. Llegó el momento
en que donde mejor se encontraban los heridos era en medio de la calle.

Me dirigí hacia el centro del Coso, porque me dijeron que allí se
repartía algo de comer; pero nada alcancé. Iba a volver a las Tenerías,
y al fin frente al Almudí me dieron un poco de comida caliente. Al punto
me sentí mejor, y lo que creía síntomas de epidemia, desapareció poco a
poco, pues mi mal hasta entonces era de los que se curan con pan y vino.
Acordeme al punto del padre Mateo del Busto, y con otros que se me
juntaron fuimos a prestarle auxilio. El desgraciado anciano no se había
movido, y cuando nos acercamos preguntándole cómo se encontraba, nos
contestó así:

---¡Cómo! ¿Ha sonado la campana de maitines? Todavía es temprano.
Déjenme ustedes descansar. Me hallo fatigadísimo, padre González. He
estado durante diez y seis horas cogiendo flores en la huerta\ldots{}
Estoy rendido.

A pesar de sus ruegos le cargamos entre cuatro; pero al poco trecho se
nos quedó muerto en los brazos.

Mis compañeros acudieron al fuego, y yo me disponía a seguirlos, cuando
alcancé a ver un hombre cuyo aspecto llamó mi atención. Era el tío
Candiola que salió de una casa cercana con los vestidos chamuscados y
apretando entre sus manos un ave de corral que cacareaba sintiéndose
prisionera. Le detuve en medio de la calle preguntándole por su hija y
por Agustín, y con gran agitación me dijo:

---¡Mi hija!\ldots{} No sé\ldots{} Allá, allá está\ldots{} ¡Todo, todo
lo he perdido! ¡Los recibos! ¡Se han quemado los recibos!\ldots{} Y
gracias que al salir de la casa tropecé con este pollo, que huía como yo
del horroroso fuego. ¡Ayer valía una gallina cinco duros!\ldots{} Pero
mis recibos, ¡Santa Virgen del Pilar, y tú Santo Dominguito de mi alma!,
¿por qué se han quemado mis recibos?\ldots{} Todavía se pueden
salvar\ldots{} ¿Quiere usted ayudarme? Debajo de una gran viga ha
quedado la caja de lata en que los tenía\ldots{} ¿Dónde hay por ahí
media docena de hombres?\ldots{} ¡Dios mío! Pero esa junta, esa
audiencia, ese capitán general, ¿en qué están pensando?\ldots{}

Y luego siguió, gritando a los que pasaban:

---¡Eh, paisano, amigo, hombre caritativo!\ldots{} ¡a ver si levantamos
la viga que cayó en el rincón!\ldots{} ¡Eh!, buenos amigos, dejen Vds.
ahí en un ladito ese enfermo moribundo que llevan al hospital, y vengan
a ayudarme. ¿No hay un alma piadosa? Parece que los corazones se han
vuelto de bronce\ldots{} Ya no hay sentimientos humanitarios\ldots{}
¡Oh! Zaragozanos sin piedad, ¡ved cómo Dios os está castigando!

Viendo que nadie le amparaba, entró de nuevo en la casa; pero salió al
poco rato gritando con desesperación:

---¡Ya no se puede salvar nada! ¡Todo está ardiendo! Virgen mía del
Pilar, ¿por qué no haces un milagro?, ¿por qué no me concedes el don de
aquellos prodigiosos niños del horno de Babilonia, para que pueda
penetrar dentro del fuego y salvar mis papeles?

\hypertarget{xxv}{%
\chapter{XXV}\label{xxv}}

Luego se sentó sobre un montón de piedras y a ratos se golpeaba el
cráneo, a ratos sin soltar el gallo llevábase la mano al pecho,
exhalando profundos suspiros. Preguntele de nuevo por su hija, con
objeto de saber de Agustín, y me dijo:

---Yo estaba en aquella casa de la calle de Añón, donde nos metimos
ayer. Todos me decían que allí no había seguridad y que mejor estaríamos
en el centro del pueblo; pero a mí no me gusta ir allí donde van todos,
y el lugar que prefiero es el que abandonan los demás. El mundo está
lleno de ladrones y rateros. Conviene, pues, huir del gentío. Nos
acomodamos en un cuarto bajo de aquella casa. Mi hija tenía mucho miedo
al cañoneo, y quería salir afuera. Cuando reventaron las minas en los
edificios cercanos, ella y Guedita salieron despavoridas. Quedeme solo,
pensando en el peligro que corrían mis efectos, y de pronto entraron
unos soldados con teas encendidas diciendo que iban a pegar fuego a la
casa. Aquellos canallas miserables no me dieron tiempo a recoger nada, y
lejos de compadecer mi situación, burláronse de mí. Yo escondí la caja
de los recibos, por temor a que creyéndola llena de dinero, me la
quisieran quitar; pero no me fue posible permanecer allí mucho tiempo.
Me abrasaba con el resplandor de las llamas, y me ahogaba con el humo; a
pesar de todo, insistí en salvar mi caja\ldots{} ¡Cosa imposible! Tuve
que huir. Nada pude traer, ¡Dios poderoso!, nada más que este pobre
animal, que había quedado olvidado por sus dueños en el gallinero. Buen
trabajo me costó el cogerle. ¡Casi se me quemó toda una mano! ¡Oh,
maldito sea el que inventó el fuego! ¡Que pierda uno su fortuna por el
gusto de estos héroes!\ldots{} Yo tengo dos casas en Zaragoza, además de
la que vivía. Una de ellas, la de la calle de la Sombra, se me conserva
ilesa, aunque sin inquilinos. La otra que llaman Casa de los Duendes, a
espaldas de San Francisco, está ocupada por las tropas, y toda me la han
destrozado. ¡Ruinas, nada más que ruinas! ¡Es feliz la ocurrencia de
quemar las casas, sólo por impedir que las conquisten los franceses!

---La guerra exige que se haga así---le respondí,---y esta heroica
ciudad quiere llevar hasta el último extremo su defensa.

---¿Y qué saca Zaragoza con llevar su defensa hasta el último extremo? A
ver, ¿qué van ganando los que han muerto? Hábleles Vd. a ellos de la
gloria, del heroísmo y de todas esas zarandajas. Antes que volver a
vivir en ciudades heroicas, me iré a un desierto. Concedo que haya
alguna resistencia; pero no hasta ese bárbaro extremo. Verdad es que los
edificios valían poco, tal vez menos que esta gran masa de carbón que
ahora resulta. A mí no me vengan con simplezas. Esto lo han ideado los
pájaros gordos, para luego hacer negocio con el carbón.

Esto me hizo reír. No crean mis lectores que exagero, pues tal como lo
cuento, me lo dijo él punto por punto, y pueden dar fe de mi veracidad
los que tuvieron la desdicha de conocerle. Si Candiola hubiera vivido en
Numancia, habría dicho que los numantinos eran negociantes de carbón
disfrazados de héroes.

---¡Estoy perdido, estoy arruinado para siempre!---añadió después,
cruzando las manos en actitud dolorosa.---Esos recibos eran parte de mi
fortuna. Vaya Vd. ahora a reclamar las cantidades sin documento alguno,
y cuando casi todos han muerto, y yacen en putrefacción por esas calles.
No, lo digo y lo repito, no es conforme a la ley de Dios lo que han
hecho esos miserables. Es un pecado mortal, es un delito imperdonable
dejarse matar, cuando se deben piquillos que el acreedor no podrá cobrar
fácilmente. Ya se ve\ldots{} esto de pagar es muy duro, y algunos dicen:
«muramos y nos quedaremos con el dinero\ldots» Pero Dios debiera ser
inexorable con esta canalla heroica, y en castigo de su infamia,
resucitarlos para que se las vieran con el alguacil y el escribano.
¡Dios mío, resucítalos! ¡Santa Virgen del Pilar, Santo Dominguito del
Val, resucítalos!

---Y su hija de Vd.---le pregunté con interés,---¿ha salido ilesa del
fuego?

---No me nombre Vd. a mi hija---replicó con desabrimiento.---Dios ha
castigado en mí su culpa. Ya sé quién es su infame pretendiente. ¿Quién
podía ser sino ese condenado hijo de D. José de Montoria, que estudia
para clérigo? María me lo ha confesado. Ayer estaba curándole la herida
que tiene en el brazo. ¿Hase visto muchacha más desvergonzada? ¡Y esto
lo hacía delante de mí!

Esto decía, cuando doña Guedita, que buscaba afanosamente a su amo,
apareció trayendo en una taza algunas provisiones. Él se las comió con
voracidad, y luego a fuerza de ruegos logramos arrancarle de allí,
conduciéndole al callejón del Órgano donde estaba su hija, guarecida en
un zaguán con otras infelices. Candiola, después de regañarla, se
internó con el ama de llaves.

---¿Dónde está Agustín?---pregunté a Mariquilla.

---Hace un instante estaba aquí; pero vinieron a darle la noticia de la
muerte de un hermano suyo, y se fue. Oí decir, que estaba su familia en
la calle de las Rufas.

---¿Que ha muerto su hermano, el primogénito?

---Así se lo dijeron, y él corrió allí muy afligido.

Sin oír más, yo también corrí a la calle de la Parra para aliviar en lo
posible la tribulación de aquella generosa familia, a quien tanto debía,
y antes de llegar a ella encontré a D. Roque, que con lágrimas en los
ojos se acercó a hablarme.

---Gabriel---me dijo,---Dios ha cargado hoy la mano sobre nuestro buen
amigo.

---¿Ha muerto el hijo mayor, Manuel de Montoria?

---Sí; y no es esa la única desgracia de la familia. Manuel era casado,
como sabes, y tenía un hijo de cuatro años. ¿Ves aquel grupo de mujeres?
Pues allí está la mujer del desgraciado primogénito de Montoria, con su
hijo en brazos, el cual, atacado de la epidemia, agoniza en estos
momentos. ¡Qué horrible situación! Ahí tienes a una de las primeras
familias de Zaragoza, reducida al más triste estado, sin un techo en que
guarecerse, y careciendo hasta de lo más preciso. Toda la noche ha
estado esa infeliz madre en la calle y a la intemperie con el enfermo en
brazos, aguardando por instantes que exhale el último suspiro; y en
realidad, mejor está aquí que en los pestilentes sótanos, donde no se
puede respirar. Gracias a que yo y otros amigos la hemos socorrido en lo
posible\ldots{} ¿pero qué podemos hacer, si apenas hay pan, si se ha
acabado el vino, y no se encuentra un pedazo de carne de vaca, aunque se
dé por él un pedazo de la nuestra?

Principiaba a amanecer. Acerqueme al grupo de mujeres, y vi el lastimoso
espectáculo. Con el ansia de salvarle, la madre y las demás mujeres que
le hacían compañía martirizaban al infeliz niño aplicándole los remedios
que cada cual discurría; pero bastaba ver a la víctima para comprender
la imposibilidad de salvar aquella naturaleza, que la muerte había asido
ya con su mano amarilla.

La voz de D. José de Montoria me obligó a seguir adelante, y en la
esquina de la calle de las Rufas, un segundo grupo completaba el cuadro
horroroso de las desgracias de aquella familia. En el suelo estaba el
cadáver de Manuel de Montoria, joven de treinta años, no menos simpático
y generoso en vida que su padre y hermano. Una bala le había atravesado
el cráneo, y de la pequeña herida exterior en el punto por donde entró
el proyectil, salía un hilo de sangre, que bajando por la sien el
carrillo y el cuello, escurríase entre la piel y la camisa. Fuera de
esto, su cuerpo no parecía el de un difunto.

Cuando yo me acerqué, su madre no se había decidido aún a creer que
estaba muerto, y poniendo la cabeza del cadáver sobre sus rodillas,
quería reanimarle con ardientes palabras. Montoria, de rodillas al
costado derecho, tenía entre sus manos la de su hijo, y sin decir nada,
no le quitaba los ojos. Tan pálido como el muerto, el padre no lloraba.

---Mujer---exclamó al fin.---No pidas a Dios imposibles. Hemos perdido a
nuestro hijo.

---¡No; mi hijo no ha muerto!---gritó la madre con desesperación.---Es
mentira. ¿Para qué me engañan? ¿Cómo es posible que Dios nos quite a
nuestro hijo? ¿Qué hemos hecho para merecer este castigo? ¡Manuel! ¡Tú,
hijo mío! ¿No me respondes? ¿Por qué no te mueves? ¿Por qué no
hablas?\ldots{} Al instante te llevaremos a casa\ldots{} pero ¿dónde
está nuestra casa? Mi hijo se enfría sobre este desnudo suelo. ¡Ved qué
heladas están sus manos y su cara!

---Retírate, mujer---dijo Montoria conteniendo el llanto.---Nosotros
cuidaremos al pobre Manuel.

---¡Señor, Dios mío!---exclamó la madre,---¿qué tiene mi hijo que no
habla, ni se mueve, ni despierta? Parece muerto; pero no está ni puede
estar muerto. Santa Virgen del Pilar, ¿no es verdad que mi hijo no ha
muerto?

---Leocadia---repitió Montoria, secando las primeras lágrimas que
salieron de sus ojos.---Vete de aquí, retírate por Dios. Ten
resignación, porque Dios nos ha dado un fuerte golpe, y nuestro hijo no
vive ya. Ha muerto por la patria\ldots{}

---¡Que ha muerto mi hijo!---exclamó la madre, estrechando el cadáver
entre sus brazos como si se lo quisieran quitar.---No, no, no: ¿qué me
importa a mí la patria? ¡Que me devuelvan a mi hijo! ¡Manuel, niño mío!
No te separes de mi lado, y el que quiera arrancarte de mis brazos,
tendrá que matarme.

---¡Señor, Dios mío! ¡Santa Virgen del Pilar!---dijo D. José de Montoria
con grave acento.---Nunca os ofendí a sabiendas ni deliberadamente. Por
la patria, por la religión y por el rey he dado mis bienes y mis hijos.
¿Por qué antes que llevaros a este mi primogénito, no me quitasteis cien
veces la vida, a mí, miserable viejo que para nada sirvo? Señores que
estáis presentes: no me avergüenzo de llorar delante de Vds. Con el
corazón despedazado, Montoria es el mismo. ¡Dichoso tú mil veces, hijo
mío, que has muerto en el puesto del honor! ¡Desgraciados los que
vivimos después de perderte! Pero Dios lo quiere así, y bajemos la
frente ante el dueño de todas las cosas. Mujer, Dios nos ha dado paz,
felicidad, bienestar y buenos hijos; ahora parece que nos lo quiere
quitar todo. Llenemos el corazón de humildad, y no maldigamos nuestro
sino. Bendita sea la mano que nos hiere, y esperemos tranquilos el
beneficio de la propia muerte.

Doña Leocadia no tenía vida más que para llorar, besando incesantemente
el frío cuerpo de su hijo. D. José, tratando de vencer las irresistibles
manifestaciones de su dolor, se levantó y dijo con voz entera:

---Leocadia, levántate. Es preciso enterrar a nuestro hijo.

---¡Enterrarle!---exclamó la madre.---¡Enterrarle\ldots!

Y no pudo decir más porque se quedó sin sentido.

En el mismo instante oyose un grito desgarrador, no lejos de allí, y una
mujer corrió despavorida hacia nosotros. Era la mujer del desgraciado
Manuel, viuda ya y sin hijo. Varios de los presentes nos abalanzamos a
contenerla para que no presenciase aquella escena, tan horrible como la
que acababa de dejar y la infeliz dama forcejeó con nosotros,
pidiéndonos que la dejásemos ver a su marido.

En tanto D. José, apartándose de allí, llegó a donde yacía el cuerpo de
su nieto: tomole en brazos y lo trajo junto al de Manuel. Las mujeres
exigían todo nuestro cuidado, y mientras doña Leocadia continuaba sin
movimiento ni sentido, abrazada al cadáver, su nuera, poseída de un
dolor febril, corría en busca de imaginarios enemigos, a quienes
anhelaba despedazar. La conteníamos y se nos escapaba de las manos. Reía
a veces con espantosa carcajada, y luego se nos ponía de rodillas
delante, rogándonos que le devolviéramos los dos cuerpos que le habíamos
quitado.

Pasaba la gente, pasaban soldados, frailes, paisanos, y todos veían
aquello con indiferencia porque a cada paso se encontraba un espectáculo
semejante. Los corazones estaban osificados y las almas parecían haber
perdido sus más hermosas facultades, no conservando más que el rudo
heroísmo. Por fin, la pobre mujer cedió a la fatiga, al aniquilamiento
producido por su propia pena, quedándosenos en los brazos como muerta.
Pedimos algún cordial o algún alimento para reanimarla, pero no había
nada, y las demás personas que allí vi, harto trabajo tenían con atender
a los suyos. En tanto D. José, ayudado de su hijo Agustín, que también
trataba de vencer su acerbo dolor, desligó el cadáver de los brazos de
doña Leocadia. El estado de esta infeliz señora era tal que creímos
tener que lamentar otra muerte en aquel día.

Luego Montoria repitió:

---Es preciso que enterremos a mi hijo.

Miró él, miramos todos en derredor, y vimos muchos, muchísimos cadáveres
insepultos. En la calle de las Rufas había bastantes; en la inmediata de
la Imprenta\footnote{Hoy de Flandro.} se había constituido una especie
de depósito. No es exageración lo que voy a decir. Innumerables cuerpos
estaban apilados en la angosta vía, formando como un ancho paredón entre
casa y casa. Aquello no se podía mirar, y el que lo vio fue condenado a
tener ante los ojos durante toda su vida la fúnebre pira hecha con
cuerpos de sus semejantes. Parece mentira, pero es cierto. Un hombre
entró en la calle de la Imprenta y empezó a dar voces. Por un ventanillo
apareció otro hombre, que contestando al primero, dijo: «sube.»
Entonces, aquel, creyendo que era extravío entrar en la casa y subir por
la escalera, trepó por el montón de cuerpos y llegó al piso principal,
una de cuyas ventanas le sirvió de puerta.

En otras muchas calles ocurría lo mismo. ¿Quién pensaba en darles
sepultura? Por cada par de brazos útiles y por cada azada había
cincuenta muertos. De trescientos a cuatrocientos perecían diariamente
sólo de la epidemia. Cada acción encarnizada arrancaba a la vida algunos
miles, y ya Zaragoza empezaba a dejar de ser una ciudad poblada por
criaturas vivas.

Montoria al ver aquello, habló así:

---Mi hijo y mi nieto no pueden tener el privilegio de dormir bajo
tierra. Sus almas están en el cielo, ¿qué importa lo demás?
Acomodémosles ahí en la puerta de la calle de las Rufas\ldots{} Agustín,
hijo mío: más vale que te vayas a las filas. Los jefes pueden echarte de
menos, y creo que hace falta gente en la Magdalena. Ya no tengo más hijo
varón que tú. Si mueres ¿qué me queda? Pero el deber es lo primero, y
antes que cobarde prefiero verte como tu pobre hermano con la sien
traspasada por una bala francesa.

Después poniendo la mano sobre la cabeza de su hijo, que estaba
descubierto y de rodillas junto al cadáver de Manuel, prosiguió así,
elevando los ojos al cielo:

---Señor, si has resuelto también llevarte a mi segundo hijo, llévame a
mí primero. Cuando se acabe el sitio, no deseo tener mas vida. Mi pobre
mujer y yo hemos sido bastante felices, hemos recibido hartos beneficios
para maldecir la mano que nos ha herido; pero para probarnos ¿no ha sido
ya bastante? ¿Ha de perecer también nuestro segundo hijo?\ldots{} Ea,
señores---añadió luego---,despachemos pronto, que quizás hagamos falta
en otra parte.

---Señor D. José---dijo D. Roque llorando,---retírese Vd. también, que
los amigos cumpliremos este triste deber.

---No, yo soy hombre para todo, y Dios me ha dado un alma que no se
dobla ni se rompe.

Y tomó ayudado de otro, el cadáver de Manuel, mientras Agustín y yo
cogimos el del nieto, para ponerlos a entrambos en la entrada del
callejón de las Rufas, donde otras muchas familias habían depositado los
muertos. Montoria luego que soltó el cuerpo, exhaló un suspiro y dejando
caer los brazos, como si el esfuerzo hecho hubiera agotado sus fuerzas,
dijo:

---Es verdad, señores, yo no puedo negar que estoy cansado. Ayer me
encontraba joven; hoy me encuentro muy viejo.

Efectivamente, Montoria estaba viejísimo, y una noche había condensado
en él la vida de diez años.

Sentose sobre una piedra, y puestos los codos en las rodillas, apoyó la
cara entre las manos, en cuya actitud permaneció mucho tiempo, sin que
los presentes turbáramos su dolor. Doña Leocadia, su hija y su nuera,
asistidas por otros individuos de la familia, continuaban en el Coso. D.
Roque, que iba y venía de uno a otro extremo, dijo:

---La señora sigue tan abatida\ldots{} Ahora están todas rezando con
mucha devoción, y no cesan de llorar. Están muy caídas las pobrecitas.
Muchachos, es preciso que deis por la ciudad una vuelta, a ver si se
encuentra algo sustancioso con que alimentarlas.

Montoria se levantó entonces, limpiando las lágrimas que corrían
abundantemente de sus ojos encendidos.

---No ha de faltar, según creo. Amigo D. Roque, busque Vd. algo de
comer, cueste lo que cueste.

---Ayer pedían cinco duros por una gallina en la Tripería---dijo uno que
era criado antiguo de la casa.

---Pero hoy no las hay---indicó D. Roque.---He estado allí hace un
momento.

---Amigos, buscad por ahí, que algo se encontrará. Yo nada necesito para
mí.

Esto decía, cuando sentimos un agradable cacareo de ave de corral.
Miramos todos con alegría hacia la entrada de la calle, y vimos al tío
Candiola, que sosteniendo en su mano izquierda el pollo consabido, le
acariciaba con la derecha el negro plumaje. Antes que se lo pidieran,
llegose a Montoria, y con mucha sorna le dijo:

---Una onza por el pollo.

---¡Qué carestía!---exclamó D. Roque.---¡Si no tiene más que huesos el
pobre animal!

No pude contener la cólera al ver ejemplo tan claro de la repugnante
tacañería y empedernido corazón del tío Candiola. Así es que llegueme a
él y, arrancándole el pollo de las manos, le dije violentamente:

---Ese pollo es robado. Venga acá. ¡Miserable usurero! ¡Si al menos
vendiera lo suyo! ¡Una onza! A cinco duros estaban ayer en el mercado.
¡Cinco duros, canalla, ladrón, cinco duros! Ni un ochavo más.

Candiola empezó a chillar reclamando su pollo, y a punto estuvo de ser
apaleado impíamente; pero D. José de Montoria intervino diciendo:

---Désele lo que quiere. Tome Vd., Sr.~Candiola, la onza que pide por
ese animal.

Diole la onza, que el infame tacaño no tuvo reparo en tomar, y luego
nuestro amigo prosiguió hablando de esta manera:

---Sr.~de Candiola, tenemos que hablar. Ahora caigo en que le ofendí a
Vd\ldots{} Sí\ldots{} hace días, cuando aquello de la harina\ldots{} Es
que a veces no es uno dueño de sí mismo, y se nos sube la sangre a la
cabeza\ldots{} Verdad es que Vd. me provocó, y como se empeñaba en que
le dieran por la harina más de lo que el señor capitán general había
mandado\ldots{} Lo cierto es, amigo D. Jerónimo, que yo me
amosqué\ldots{} ya ve Vd\ldots{} no lo puede uno remediar así de
pronto\ldots{} pues\ldots{} y creo que se me fue la mano; creo que hubo
algo de\ldots{}

---Sr.~Montoria---dijo gruñendo el avaro,---llegará un día en que haya
otra vez autoridades en Zaragoza. Entonces nos veremos las caras.

---¿Va Vd. a meterse entre jueces y escribanos? Malo. Aquello
pasó\ldots{} Fue un arrebato de cólera, una de esas cosas que no se
pueden remediar. Lo que me llama la atención, es que hasta ahora no
había caído en que hice mal, muy mal. No se debe ofender al
prójimo\ldots{}

---Y menos ofenderle después de robarle---dijo D. Jerónimo, mirándonos a
todos y sonriendo con desdén.

---Eso de robar no es cierto---continuó Montoria,---porque yo hice lo
que el capitán general me mandaba. Cierto es lo de la ofensa de palabra
y de obra, y ahora cuando le he visto a Vd. venir con el pollo, he caído
en la cuenta de que hice mal. Mi conciencia me lo dice\ldots{} ¡Ah!
Sr.~Candiola, soy muy desgraciado. Cuando uno es feliz, no conoce sus
faltas. Pero ahora\ldots{} Lo cierto es, D. Jerónimo de Candiola, que en
cuanto le vi venir a Vd., me sentí inclinado a pedirle perdón por
aquellos golpes\ldots{} yo tengo la mano pesada, y\ldots{} Así es que en
un pronto\ldots{} no sé lo que me hago\ldots{} Sí, yo le ruego a Vd. que
me perdone y seamos amigos. Sr.~D. Jerónimo, seamos amigos;
reconciliémonos y no hagamos caso de resentimientos antiguos. El odio
envenena las almas, y el recuerdo de no haber obrado bien nos pone
encima un peso insoportable.

---Después de hecho el daño, todo se arregla con hipócritas
palabrejas---dijo Candiola volviendo la espalda a Montoria, y
escurriéndose fuera del grupo.---Más vale que piense el Sr.~Montoria en
reintegrarme el precio de la harina\ldots{} ¡Perdoncitos a mí\ldots! Ya
no me queda nada que ver.

Dijo esto en voz baja, y alejose lentamente. Montoria, viendo que alguno
de los presentes corría tras él insultándole, añadió:

---Dejadle marchar tranquilo, y tengamos compasión de ese desgraciado.

\hypertarget{xxvi}{%
\chapter{XXVI}\label{xxvi}}

El 3 de Febrero se apoderaron los franceses del Convento de Jerusalén,
que estaba entre Santa Engracia y el Hospital\footnote{Hoy existe
  renovado el Convento de Jerusalén. Su fachada da al salón de la
  Independencia. El Hospital ocupaba el sitio donde está hoy la fonda de
  Europa. El actual palacio de la Diputación provincial se ha construido
  sobre el solar del Convento de San Francisco.}. La acción que precedió
a la conquista de tan importante posición fue tan sangrienta como las de
las Tenerías, y allí murió el distinguido comandante de ingenieros D.
Marcos Simonó. Por la parte del arrabal poco adelantaban los sitiadores,
y en los días 6 y 7 todavía no habían podido dominar la calle de Puerta
Quemada.

Las autoridades comprendían que era difícil prolongar mucho más la
resistencia, y con ofertas de honores y dinero intentaban exaltar a los
patriotas. En una proclama del 2 de Febrero, Palafox, al pedir recursos,
decía: «Doy mis dos relojes y veinte cubiertos de plata, que es lo que
me queda.» En la de 4 de Febrero ofrecía armar caballeros a los doce que
más se distinguieran, para lo cual creaba una Orden militar noble,
llamada de \emph{Infanzones}; y en la del 9 se quejaba de la
indiferencia y \emph{abandono con que algunos vecinos miraban la suerte
de la patria}, y después de suponer que el desaliento era producido por
el \emph{oro francés}, amenazaba con grandes castigos al que se mostrara
cobarde.

Las acciones de los días 3, 4 y 5 no fueron tan encarnizadas como la
última que describí. Franceses y españoles estaban muertos de fatiga.
Las boca-calles que conservamos en la plazuela de la Magdalena,
conteniendo siempre al enemigo en sus dos avances de la calle de Palomar
y de Pabostre, se defendían con cañones. Los restos del seminario
estaban asimismo erizados de artillería, y los franceses, seguros de no
poder echamos de allí por los medios ordinarios, trabajaban sin cesar en
sus minas.

Mi batallón se había fundido en el de Extremadura, pues el resto de uno
y otro no llegaba a tres compañías. Agustín de Montoria era capitán, y
yo fui ascendido a alférez el día 2. No volvimos a prestar servicio en
las Tenerías y lleváronnos a guarnecer a San Francisco, vasto edificio
que ofrecía buenas posiciones para tirotear a los franceses,
establecidos en Jerusalén. Se nos repartían raciones muy escasas, y los
que ya nos contábamos en el número de oficiales comíamos rancho lo mismo
que los soldados. Agustín guardaba su pan, para llevárselo a Mariquilla.

Desde el día 4 empezaron los franceses a minar el terreno para
apoderarse del Hospital y de San Francisco, pues harto sabían que de
otro modo era imposible. Para impedirlo contraminanos, con objeto de
volarles a ellos antes que nos volaran a nosotros, y este trabajo
ardoroso en las entrañas de la tierra a nada del mundo puede compararse.
Parecíanos haber dejado de ser hombres, para convertirnos en otra
especie de seres, insensibles y fríos habitantes de las cavernas, lejos
del sol, del aire puro y de la hermosa luz. Sin cesar labrábamos largas
galerías, como el gusano que se fabrica la casa en lo oscuro de la
tierra y con el molde de su propio cuerpo. Entre los golpes de nuestras
piquetas oíamos, como un sordo eco, el de las piquetas de los franceses,
y después de habernos batido y destrozado en la superficie, nos
buscábamos en la horrible noche de aquellos sepulcros para acabar de
exterminamos.

El Convento de San Francisco tenía por la parte del coro vastas bodegas
subterráneas. Los edificios que ocupaban más abajo los franceses también
las tenían, y rara era la casa que no se alzaba sobre profundos sótanos.
Las galerías abiertas por las azadas de unos y otros juntábanse al fin
en uno de aquellos aposentos: a la luz de nuestros faroles veíamos a los
franceses, como imaginarias figuras de duendes engendradas por la luz
rojiza en las sinuosidades de la mazmorra; ellos nos veían también, y al
punto nos tiroteábamos; pero nosotros íbamos provistos de granadas de
mano, y arrojándolas sobre ellos les poníamos en dispersión
persiguiéndoles luego a arma blanca a lo largo de las galerías. Todo
aquello parecía una pesadilla, una de esas luchas angustiosas que a
veces trabamos contra seres aborrecidos en las profundas concavidades
del sueño: pero era cierto y se repetía a cada instante en diversos
puntos.

En esta penosa tarea nos relevábamos frecuentemente, y en los ratos de
descanso salíamos al Coso, sitio céntrico de reunión y al mismo tiempo
parque, hospital y cementerio general de los sitiados. Una tarde (creo
que la del 5) estábamos en la puerta del Convento varios muchachos del
batallón de Estremadura y de San Pedro y comentábamos las peripecias del
sitio, opinando todos que bien pronto sería imposible la resistencia. El
corrillo se renovaba constantemente. D. José de Montoria se acercó a
nosotros, y saludándonos con semblante triste, sentose en el banquillo
de madera que teníamos junto a la puerta.

---Oiga Vd. lo que se habla por aquí, señor don José---le dije.---La
gente cree que es imposible resistir muchos días más.

---No os desaniméis, muchachos---contestó.---Bien dice el capitán
general en su proclama que corre mucho oro francés por la ciudad.

Un franciscano que venía de auxiliar a algunas docenas de moribundos
tomó la palabra y dijo:

---Es un dolor lo que pasa. No se habla por ahí de otra cosa que de
rendirse. Si parece que esto ya no es Zaragoza. ¡Quién conoció a aquella
gente templada del primer sitio!\ldots{}

---Dice bien su paternidad---afirmó Montoria.---Está uno avergonzado, y
hasta los que tenemos corazón de bronce nos sentimos atacados de esta
flaqueza que cunde más que la epidemia. Y en resumidas cuentas, no sé a
qué viene ahora esa novedad de rendirse cuando nunca lo hemos hecho,
¡porra! Si hay algo después de este mundo como nuestra religión nos
enseña, ¿a qué apurarse por un día más o menos de vida?

---Verdad es, Sr.~D. José---dijo el fraile,---que las provisiones se
acaban por momentos y que donde no hay harina todo es mohína.

---¡Boberías y melindres!, padre Luengo---exclamó Montoria.---Ya\ldots{}
Si esta gente, acostumbrada al regalo de otros tiempos, no puede pasarse
sin carne y pan, no hemos dicho nada. Como si no hubiera otras muchas
cosas que comer\ldots{} Soy partidario de la resistencia a todo trance,
cueste lo que cueste. He experimentado terribles desgracias; la pérdida
de mi primogénito y de mi nieto ha cubierto de luto mi corazón; pero el
honor nacional, llenando toda mi alma, a veces no deja hueco para otro
sentimiento. Un hijo me queda, único consuelo de mi vida y depositario
de mi casa y mi nombre. Lejos de apartarle del peligro le obligo a
persistir en la defensa. Si le pierdo, me moriré de pena; pero que salve
el honor nacional, aunque perezca mi único heredero.

---Y según he oído---dijo el padre Luengo,---el señor D. Agustín ha
hecho prodigios de valor. Está visto que los primeros laureles de esta
campaña pertenecen a los insignes guerreros de la Iglesia.

---No: mi hijo no pertenecerá ya a la Iglesia. Es preciso que renuncie a
ser clérigo, pues yo no puedo quedarme sin sucesión directa.

---Sí: vaya Vd. a hablarle de sucesiones y de casorios. Desde que es
soldado parece que ha cambiado un poco; pero antes sus conversaciones
trataban siempre \emph{de re theologica}, y jamás le oí hablar \emph{de
erotica}. Es un chico que tiene a Santo Tomás en las puntas de los
dedos, y no sabe en qué sitio de la cara llevan los ojos las muchachas.

---Agustín sacrificará por mí su ardiente vocación. Si salimos bien del
sitio y la Virgen del Pilar me lo deja con vida, pienso casarle al
instante con mujer que le iguale en condición y fortuna.

Cuando esto decía, vimos que se nos acercaba sofocada Mariquilla
Candiola, la cual llegándose a mí me preguntó:

---Sr.~de Araceli, ¿ha visto Vd. a mi padre?

---No, señorita doña María---le respondí.---Desde ayer no le he visto.
Puede que esté en las ruinas de su casa, ocupándose en ver si puede
sacar alguna cosa.

---No está---dijo Mariquilla con desaliento.---Le he buscado por todas
partes.

---¿Ha estado Vd. aquí detrás, por junto a San Diego? El Sr.~Candiola
suele ir a visitar su casa llamada de los Duendes por ver si se la han
destrozado.

---Pues voy al momento allá.

Cuando desapareció, dijo Montoria:

---Es esta, a lo que parece, la hija del tío Candiola. A fe que es
bonita, y no parece hija de aquel lobo\ldots{} Dios me perdone el mote.
De aquel buen hombre, quise decir.

---Es guapilla---afirmó el fraile.---Pero se me figura que es una buena
pieza. De la madera del tío Candiola no puede salir un buen santo.

---No se habla mal del prójimo---dijo D. José.

---Candiola no es prójimo. La muchacha desde que se quedaron sin casa,
no abandona la compañía de los soldados.

---Estará entre ellos para asistir a los heridos.

---Puede ser; pero me parece que le gustan más los sanos y robustos. Su
carilla graciosa está diciendo que allí no hay pizca de vergüenza.

---¡Lengua de escorpión!

---Pura verdad---añadió el fraile.---Bien dicen que de tal palo, tal
astilla. ¿No aseguran que su madre la Pepa Rincón fue mujer pública o
poco menos?

---Alegre de cascos tal vez\ldots{}

---¡No está mala alegría! Cuando fue abandonada por su tercer cortejo,
cargó con ella el Sr.~D. Jerónimo.

---Basta de difamación---dijo Montoria,---y aunque se trata de la peor
gente del mundo, dejémosles con su conciencia.

---Yo no daría un maravedí por el alma de todos los Candiolas
reunidos---repuso el fraile.

Pero allí aparece el Sr.~D. Jerónimo, si no me engaño. Nos ha visto y
viene hacia acá.

En efecto, el tío Candiola avanzaba despaciosamente por el Coso, y llegó
a la puerta del Convento.

---Buenas tardes tenga el Sr.~D. Jerónimo---le dijo Montoria.---Quedamos
en que se acabaron los rencorcillos\ldots{}

---Hace un momento ha estado aquí preguntando por Vd. su inocente
hija---le indicó Luengo con malicia.

---¿Dónde está?

---Ha ido a San Diego---dijo un soldado.---Puede que se la roben los
franceses que andan por allí cerca.

---Quizás la respeten al saber que es hija del señor D. Jerónimo---dijo
Luengo.---¿Es cierto, amigo Candiola, lo que se cuenta por ahí?

---¿Qué?

---Que Vd. ha pasado estos días la línea francesa para conferenciar con
la canalla.

---¡Yo! ¡Qué vil calumnia!---exclamó el tacaño.---Eso lo dirán mis
enemigos para perderme. ¿Es usted, Sr.~Montoria, quien ha hecho correr
esas voces?

---Ni por pienso---respondió el patriota.---Pero es cierto que lo oí
decir. Recuerdo que le defendí a usted, asegurando que el Sr.~Candiola
es incapaz de venderse a los franceses.

---¡Mis enemigos, mis enemigos quieren perderme! ¡Qué infamias inventan
contra mí! También quieren que pierda la honra, después de haber perdido
la hacienda. Señores, mi casa de la calle de la Sombra ha perdido parte
del tejado. ¿Hay desolación semejante? La que tengo aquí detrás de San
Francisco y pegada a la huerta de San Diego, se conserva bien; pero está
ocupada por la tropa, y me la destrozan que es un primor.

---El edificio vale bien poco, Sr.~D. Jerónimo---dijo el fraile,---y si
mal no recuerdo, hace diez años que nadie quiere habitarla.

---Como dio la gente en la manía de decir si había duendes o no\ldots{}
Pero dejemos eso. ¿Han visto por aquí a mi hija?

---Esa virginal azucena ha ido hacia San Diego en busca de su simpático
papá.

---Mi hija ha perdido el juicio.

---Algo de eso.

---También tiene de ello la culpa el Sr.~de Montoria. Mis enemigos, mis
pérfidos enemigos no me dejan respirar.

---¡Cómo!---exclamó mi protector.---¿También tengo yo la culpa de que
esa niña haya sacado las malas mañas de su madre?\ldots{} quiero
decir\ldots{} ¡Maldita lengua mía! Su madre fue una señora ejemplar.

---Los insultos del Sr.~Montoria no me llaman la atención y los
desprecio---dijo el avaro con ponzoñosa cólera.---En vez de insultarme
el Sr.~D. José, debiera sujetar a su niño Agustín, libertino y
embaucador, que es quien ha trastornado el seso a mi hija. No, no se la
daré en matrimonio, aunque bebe los vientos por ella. Y quiere
robármela. ¡Buena pieza el tal D. Agustín! No, no la tendrá por esposa.
Vale más, mucho más mi María.

D. José de Montoria, al oír esto, púsose blanco, y dio algunos pasos
hacia el tío Candiola, con intento sin duda de renovar la violenta
escena de la calle de Antón Trillo. Después se contuvo, y con voz
dolorida habló así:

---¡Dios mío! Dame fuerzas para reprimir mis arrebatos de cólera. ¿Es
posible matar la soberbia y ser humilde delante de este hombre? Le pedí
perdón de la ofensa que le hice, humilleme ante él, le ofrecí una mano
de amigo, y sin embargo, se me pone delante para injuriarme otra vez,
para insultarme del modo más horrendo\ldots{} ¡Miserable hombre,
castígame, mátame, bébete toda mi sangre y vende después mis huesos para
hacer botones; pero que tu vil lengua no arroje tanta ignominia sobre mi
hijo querido! ¿Qué has dicho, que ha dicho Vd. de mi Agustín?

---La verdad.

---No sé cómo me contengo. Señores, sean ustedes testigos de mi bondad.
No quiero arrebatarme; no quiero atropellar a nadie; no quiero ofender a
Dios. Yo le perdono a este hombre sus infamias; pero que se quite al
punto de mi presencia, porque viéndole no respondo de mí.

Candiola, amedrentado por estas palabras, entró en el portalón del
Convento. El padre Luengo se llevó a Montoria por el Coso abajo.

Y sucedió que en el mismo instante, entre los soldados que allí estaban
reunidos, empezó a cundir un murmullo rencoroso que indicaba
sentimientos muy hostiles contra el padre de Mariquilla, lo cual,
atendidos los antecedentes de aquel, no tenía nada de particular. Él
quiso huir, viéndose empujado de un lado para otro; mas le detuvieron, y
sin saber cómo, en un rápido movimiento del grupo amenazador, fue
llevado al claustro. Entonces una voz dijo con colérico acento:

---Al pozo; arrojarle al pozo.

Candiola fue asido por varias manos, y magullado, roto y descosido más
de lo que estaba.

---Es de los que andan repartiendo dinero para que la tropa se
rinda---dijo uno.

---Sí, sí---gritaron otros.---Ayer decían que andaba en el Mercado
repartiendo dinero.

---Señores---decía el infeliz con voz ahogada,---yo les juro a Vds. que
jamás he repartido dinero.

Y así era la verdad.

---Anoche dicen que le vieron traspasar la línea y meterse en el campo
francés.

---De donde volvió por la mañana. ¡Al pozo con él!

Otro amigo y yo forcejeamos un rato por salvar a Candiola de una muerte
segura; pero no lo pudimos conseguir sino a fuerza de ruegos y
persuasiones, diciendo:

---Muchachos, no hagamos una barbaridad. ¿Qué daño puede causar este
vejete despreciable?

---Es verdad---añadió él en el colmo de la angustia.---¿Qué mal puedo
hacer yo, que siempre me he ocupado en socorrer a los menesterosos?
Vosotros no me mataréis; sois soldados de las Peñas de San Pedro y de
Extremadura; sois todos guapos chicos. Vosotros incendiasteis aquellas
casas de las Tenerías, donde yo encontré el pollo que me valió una onza.
¿Quién dice que yo me vendo a los franceses? Les odio, no les puedo ver,
y a vosotros os quiero como a mi propio pellejo. Niñitos míos, dejadme
en paz. Todo lo he perdido; que me quede al menos la vida.

Estas lamentaciones, y los ruegos míos y de mi amigo ablandaron un poco
a los soldados, y una vez pasada la primera efervescencia, nos fue fácil
salvar al desgraciado viejo. Al relevarse la gente que estaba en las
posiciones, quedó completamente a salvo; pero ni siquiera nos dio las
gracias cuando, después de librarle de la muerte, le ofrecimos un pedazo
de pan. Poco después, y cuando tuvo alientos para andar, salió a la
calle, donde él y su hija se reunieron.

\hypertarget{xxvii}{%
\chapter{XXVII}\label{xxvii}}

Aquella tarde, casi todo el esfuerzo de los franceses se dirigió contra
el arrabal de la izquierda del Ebro. Asaltaron el monasterio de Jesús, y
bombardearon el templo del Pilar, donde se refugiaba el mayor número de
enfermos y heridos, creyendo que la santidad del lugar les ofrecía allí
más seguridad que en otra parte.

En el centro no se trabajó mucho en aquel día. Toda la atención estaba
reconcentrada en las minas y nuestros esfuerzos se dirigían a probar al
enemigo que antes que consentir en ser volados solos, trataríamos de
volarles a ellos, o volar juntos, por lo menos.

Por la noche ambos ejércitos parecían entregados al reposo. En las
galerías subterráneas no se sentía el rudo golpe de la piqueta. Yo salí
afuera y hacia San Diego encontré a Agustín y a Mariquilla, que hablaban
sosegadamente sentados en el dintel de una puerta de la casa de los
Duendes. Se alegraron mucho de verme, y me senté junto a ellos
participando de los mendrugos que comían.

---No tenemos donde albergarnos---dijo Mariquilla.---Estábamos en un
portal del callejón del Órgano, y nos echaron. ¿Por qué aborrecen tanto
a mi pobre padre? ¿Qué daño les ha hecho? Después nos guarecimos en un
cuartucho de la calle de las Urreas y también nos echaron. Nos sentamos
luego bajo un arco en el Coso, y todos los que allí estaban huyeron de
nosotros. Mi padre está furioso.

---Mariquilla de mi corazón---dijo Agustín,---espero que el sitio se
acabe pronto de un modo o de otro. Quiera Dios que muramos los dos, si
vivos no podemos ser felices. No sé por qué en medio de tantas
desgracias mi corazón está lleno de esperanza; no sé por qué me ocurren
ideas agradables y pienso constantemente en un risueño porvenir. ¿Por
qué no? ¿Todo ha de ser desgracias y calamidades? Las desventuras de mi
familia son infinitas. Mi madre no tiene ni quiere tener consuelo. Nadie
puede apartarla del sitio en que están el cadáver de mi hermano y el de
mi sobrino, y cuando por fuerza la llevamos lejos de allí, la vemos
luego arrastrándose sobre las piedras de la calle para volver. Ella, mi
cuñada y mi hermana ofrecen un espectáculo lastimoso; niéganse a tomar
alimentos, y al rezar, deliran, confundiendo los nombres santos. Esta
tarde al fin hemos conseguido llevarlas a un sitio cubierto donde se las
obliga a mantenerse en reposo y a tomar algún alimento. Mariquilla, ¡a
qué triste estado ha traído Dios a los míos! ¿No hay motivo para esperar
que al fin se apiade de nosotros?

---Sí---repuso la Candiola;---el corazón me dice que hemos pasado las
amarguras de nuestra vida, y que ahora tendremos días tranquilos. El
sitio se acabará pronto, porque según dice mi padre, lo que queda es
cosa de días. Esta mañana fui al Pilar; cuando me arrodillé delante de
la Virgen, pareciome que la Santa Señora me miraba y se reía. Después
salí de la iglesia, y un gozo muy vivo hacía palpitar mi corazón. Miraba
al cielo y las bombas me parecían un juguete; miraba a los heridos, y se
me figuraba que todos ellos se volvían sanos; miraba a las gentes y en
todas creía encontrar la alegría que se desbordaba en mi pecho. Yo no sé
lo que me ha pasado hoy, yo estoy contenta\ldots{} Dios y la Virgen sin
duda se han apiadado de nosotros; y estos latidos de mi corazón, esta
alegre inquietud son avisos de que al fin después de tantas lágrimas
vamos a ser dichosos.

---¡Lo que dices es la verdad!---exclamó Agustín, estrechando a
Mariquilla amorosamente contra su pecho.---Tus presentimientos son
leyes; tu corazón identificado con lo divino no puede engañarnos;
oyéndote me parece que se disipa la atmósfera de penas en que nos
ahogamos, y respiro con delicia los aires de la felicidad. Espero que tu
padre no se opondrá a que te cases conmigo.

---Mi padre es bueno---dijo la Candiola.---Yo creo que si los vecinos de
la ciudad no le mortificaran, él sería más humano. Pero no le pueden
ver. Esta tarde ha sido maltratado otra vez en el claustro de San
Francisco, y cuando se reunió conmigo en el Coso estaba furioso y juraba
que se había de vengar. Yo procuraba aplacarle; pero todo en vano. Nos
echaron de varias partes. Él, cerrando los puños y pronunciando voces
destempladas, amenazaba a los transeúntes. Después echó a correr hacia
aquí; yo pensé que venía a ver si le han destrozado esta casa, que es
nuestra; seguile, volviose él hacia mí como atemorizado al sentir mis
pasos, y me dijo: «tonta, entrometida, ¿quién te manda seguirme?» Yo no
le contesté nada, pero viendo que avanzaba hacia la línea francesa con
ánimo de traspasarla, quise detenerle, y le dije: «Padre ¿a dónde vas?»
Entonces me contestó: «¿No sabes que en el ejército francés está mi
amigo el capitán de suizos D. Carlos Lindener, que servía el año pasado
en Zaragoza? Voy a verle; recordarás que me debe algunas cantidades.»
Hízome quedar aquí y se marchó. Lo que siento es que sus enemigos, si
saben que traspasa la línea y va al campo francés, le llamarán traidor.
No sé si será por el gran cariño que le tengo por lo que me parece
incapaz de semejante acción. Temo que le pase algún mal, y por eso deseo
la conclusión del sitio. ¿No es verdad que concluirá pronto, Agustín?

---Sí, Mariquilla, concluirá pronto, y nos casaremos. Mi padre quiere
que me case.

---¿Quién es tu padre? ¿Cómo se llama? No es tiempo todavía de que me lo
digas?

---Ya lo sabrás. Mi padre es persona principal y muy querido en
Zaragoza. ¿Para qué quieres saber más?

---Ayer quise averiguarlo\ldots{} Somos curiosas. A varias personas
conocidas que hallé en el Coso les pregunté: «¿Saben Vds. quién es ese
señor que ha perdido a su hijo primogénito?» Pero como hay tantos en
este caso, la gente se reía de mí.

---No me lo preguntes. Yo te lo revelaré a su tiempo, y cuando al
decírtelo, pueda darte una buena noticia.

---Agustín, si me caso contigo, quiero que me lleves fuera de Zaragoza
por unos días. Deseo durante corto tiempo ver otras casas, otros
árboles, otro país; deseo vivir algunos días en sitios que no sean
estos, donde tanto he padecido.

Sí, Mariquilla de mi alma---exclamó Montoria con arrebato;---iremos a
donde quieras, lejos de aquí, mañana mismo\ldots{} mañana no, porque no
está levantado el sitio; pasado\ldots{} en fin, cuando Dios
quiera\ldots{}

---Agustín---añadió Mariquilla, con voz débil que indicaba cierta
somnolencia---, quiero que al volver de nuestro viaje, reedifiques la
casa en que he nacido. El ciprés continúa en pie.

Mariquilla inclinando la cabeza, mostraba estar medio vencida por el
sueño.

---¿Deseas dormir, pobrecilla?---le dijo mi amigo tomándola en brazos.

---Hace varias noches que no duermo---respondió la joven cerrando los
ojos.---La inquietud, el pesar, el miedo me han mantenido en vela. Esta
noche el cansancio me rinde, y la tranquilidad que siento me hace
dormir.

---Duerme en mis brazos, María---dijo Agustín,---y que la tranquilidad
que ahora llena tu alma no te abandone cuando despiertes.

Después de un breve rato en que la creímos dormida, Mariquilla mitad
despierta, mitad en sueños, habló así:

---Agustín, no quiero que quites de mi lado a esa buena doña Guedita,
que tanto nos protegía cuando éramos novios\ldots{} Ya ves cómo tenía yo
razón al decirte que mi padre fue al campo francés a cobrar sus
cuentas\ldots{}

Después no habló más y se durmió profundamente. Sentado Agustín en el
suelo, la sostenía sobre sus rodillas y entre sus brazos. Yo abrigué sus
pies con mi capote.

Callábamos Agustín y yo, porque nuestras voces no turbaran el sueño de
la muchacha. Aquel sitio era bastante solitario. Teníamos a la espalda
la casa de los Duendes, inmediata al Convento de San Francisco, y
enfrente el colegio de San Diego, con su huerta circuida por largas
tapias que se alzaban en irregulares y angostos callejones. Por ellos
discurrían los centinelas que se relevaban y los pelotones que iban a
las avanzadas o venían de ellas. La tregua era completa, y aquel reposo
anunciaba grandes luchas para el día siguiente.

De pronto, el silencio me permitió oír sordos golpes debajo de nosotros
en lo profundo del suelo. Al punto comprendí que andaba por allí la
piqueta de los minadores franceses, y comuniqué mi recelo a Agustín, el
cual, prestando atención, me dijo:

---Efectivamente, parece que están minando. Pero ¿a dónde van por aquí?
Las galerías que hicieron desde Jerusalén están todas cortadas por las
nuestras. No pueden dar un paso sin que se les salga al encuentro.

---Es que este ruido indica que están minando por San Diego. Ellos
poseen una parte del edificio. Hasta ahora no han podido llegar a las
bodegas de San Francisco. Si por casualidad han discurrido que es fácil
el paso desde San Diego a San Francisco por los bajos de esta casa, es
probable que este paso sea el que están abriendo ahora.

---Corre al instante al Convento---me dijo,---baja a los subterráneos, y
si sientes ruido, cuenta a Renovales lo que pasa. Si algo ocurre, me
llamas enseguida.

Agustín quedose solo con Mariquilla. Fui a San Francisco, y al bajar a
las bodegas, encontré, con otros patriotas, a un oficial de ingenieros,
el cual, como yo le expusiera mi temor, me dijo:

---Por las galerías abiertas debajo de la calle de Santa Engracia, desde
Jerusalén y el Hospital, no pueden acercarse aquí, porque con nuestra
zapa hemos inutilizado la suya, y unos cuantos hombres podrán
contenerlos. Debajo de este edificio dominamos los subterráneos de la
iglesia, las bodegas y los sótanos que caen hacia el claustro de
Oriente. Hay una parte del Convento que no está minada, y es la de
Poniente y Sur; pero allí no hay sótanos, y hemos creído excusado abrir
galerías, porque no es probable se nos acerquen por esos dos lados.
Poseemos la casa inmediata, y yo he reconocido su parte subterránea, que
está casi pegada a las cuevas de la sala capitular. Si ellos dominaran
la casa de los Duendes, fácil les sería poner hornillos y volar toda la
parte del Sur y de Poniente; pero aquel edificio es nuestro, y desde él
a las posiciones francesas enfrente de San Diego y en Santa Rosa, hay
mucha distancia. No es probable que nos ataquen por ahí, a menos que no
exista alguna comunicación desconocida entre la casa y San Diego o Santa
Rosa, que les permitiera acercársenos sin advertirlo.

Hablando sobre el particular estuvimos hasta la madrugada. Al amanecer,
Agustín entró muy alegre, diciéndome que había conseguido albergar a
Mariquilla en el mismo local donde estaba su familia. Después nos
dispusimos para hacer un esfuerzo aquel día, porque los franceses,
dueños ya del Hospital, mejor dicho, de sus ruinas, amenazaban asaltar a
San Francisco, no por bajo tierra, sino a descubierto y a la luz del
sol.

\hypertarget{xxviii}{%
\chapter{XXVIII}\label{xxviii}}

La posesión de San Francisco iba a decidir la suerte de la ciudad. Aquel
vasto edificio, situado en el centro del Coso, daba una superioridad
incontestable a la nación que lo ocupase. Los franceses lo cañonearon
desde muy temprano, con objeto de abrir brecha para el asalto, y los
zaragozanos llevaron a él lo mejor de su fuerza para defenderlo. Como
escaseaban ya los soldados, multitud de personas graves que hasta
entonces no sirvieran sino de auxiliares, tomaron las armas. Sas,
Cereso, La Casa, Piedrafita, Escobar, Leiva, D. José de Montoria, todos
los grandes patriotas habían acudido también.

En la embocadura de la calle de San Gil y en el arco de Cineja había
varios cañones para contener los ímpetus del enemigo. Yo fui enviado con
otros de Extremadura al servicio de aquellas piezas, porque apenas
quedaban artilleros, y cuando me despedí de Agustín, que permanecía en
San Francisco al frente de la compañía, nos abrazamos creyendo que no
nos volveríamos a ver.

D. José de Montoria, hallándose en la barricada de la Cruz del Coso,
recibió un balazo en la pierna y tuvo que retirarse; pero apoyado en la
pared de una casa inmediata al arco de Cineja, resistió por algún tiempo
el desmayo que le producía la hemorragia, hasta que al fin sintiéndose
desfallecido, me llamó, diciéndome:

---Sr.~de Araceli, se me nublan los ojos\ldots{} No veo nada\ldots{}
¡Maldita sangre, cómo se marcha a toda prisa cuando hace más falta!
¿Quiere Vd. darme la mano?

---Señor---le dije corriendo hacia él y sosteniéndole.---Más vale que se
retire Vd. a su alojamiento.

---No, aquí quiero estar\ldots{} Pero, Sr.~de Araceli, si me quedo sin
sangre\ldots{} ¿Dónde demonios se ha ido esta condenada sangre\ldots?, y
parece que tengo piernas de algodón\ldots{} Me caigo al suelo como un
costal vacío.

Hizo terribles esfuerzos por reanimarse; pero casi llegó a perder el
sentido, más que por la gravedad de la herida, por la pérdida de la
sangre, el ningún alimento, los insomnios y penas de aquellos días.
Aunque él rogaba que le dejáramos allí arrimado a la pared, para no
perder ni un solo detalle de la acción que iba a trabarse, le llevamos a
su albergue, que estaba en el mismo Coso, esquina a la calle del
Refugio. La familia había sido instalada en una habitación alta. La casa
toda estaba llena de heridos, y casi obstruían la puerta los muchos
cadáveres depositados en aquel sitio. En el angosto portal, en las
habitaciones interiores no se podía dar un paso porque la gente que
había ido allí a morirse lo obstruía todo, y no era fácil distinguir los
vivos de los difuntos.

Montoria, cuando le entramos allí, dijo:

---No me llevéis arriba, muchachos, donde está mi familia. Dejadme en
esta pieza baja. Ahí veo un mostrador que me viene de perillas.

Pusímosle donde dijo. La pieza baja era una tienda. Bajo el mostrador
habían expirado aquel día algunos heridos y apestados, y muchos enfermos
se extendían por el infecto suelo, arrojados sobre piezas de tela.

---A ver---continuó,---si hay por ahí algún alma caritativa que me ponga
un poco de estopa en este boquete por donde sale la sangre\ldots{}

Una mujer se adelantó hacia el herido. Era Mariquilla Candiola.

---Dios os lo premie, niña---dijo D. José, al ver que traía hilas y
lienzo para curarle.---Basta por ahora con que me remiende Vd. un poco
esta pierna. Creo que no se ha roto el hueso.

Mientras esto pasaba, unos veinte paisanos invadieron la casa, para
hacer fuego desde las ventanas contra las ruinas del hospital.

---Sr.~de Araceli, ¿se marcha Vd. al fuego? Aguarde Vd. un rato, para
que me lleve, porque me parece que no puedo andar solo. Mande Vd. el
fuego desde la ventana. Buena puntería. No dejar respirar a los del
Hospital\ldots{} A ver, joven, despache Vd. pronto. ¿No tiene Vd. un
cuchillo a mano? Sería bueno cortar ese pedazo de carne que
cuelga\ldots{} ¿Cómo va eso, señor de Araceli? ¿Vamos ganando?

---Vamos bien---le respondí desde la ventana.---Ahora retroceden al
Hospital. San Francisco es un hueso un poco duro de roer.

María en tanto miraba fijamente a Montoria, y seguía curándole con mucho
cuidado y esmero.

---Es Vd. una alhaja, niña---dijo mi amigo.---Parece que no pone las
manos encima de la herida\ldots{} Pero ¿a qué me mira Vd. tanto? ¿Tengo
monos en la cara? A ver\ldots{} ¿Está concluido eso?\ldots{} Trataré de
levantarme\ldots{} Pero si no me puedo tener\ldots{} ¿Qué agua de malva
es esta que tengo en las venas? Porr\ldots{} iba a decirlo\ldots{} que
no pueda corregir la maldita costumbre\ldots{} Sr.~de Araceli, no puedo
con mi alma. ¿Cómo anda la cosa?

---Señor, a las mil maravillas. Nuestros valientes paisanos están
haciendo prodigios.

En esto llegó un oficial herido a que le pusieran un vendaje.

---Todo marcha a pedir de boca---nos dijo.---No tomarán a San Francisco.
Los del hospital han sido rechazados tres veces. Pero lo portentoso,
señores, ha ocurrido por el lado de San Diego. Viendo que los franceses
se apoderaban de la huerta pegada a la casa de los Duendes, cargaron
sobre ellos a la bayoneta los valientes soldados de Orihuela, mandados
por Pino-Hermoso, y no sólo los desalojaron, sino que dieron muerte a
muchos, cogiendo trece prisioneros.

---Quiero ir allá. ¡Viva el batallón de Orihuela! ¡Viva el marqués de
Pino-Hermoso!---exclamó con furor sublime D. José de Montoria.---Sr.~de
Araceli, vamos allá. Lléveme Vd. ¿Hay por ahí un par de muletas?
Señores, las piernas me faltan. Pero andaré con el corazón. Adiós niña,
hermosa curandera\ldots{} Pero ¿por qué me mira Vd. tanto?\ldots{} Me
conoce Vd. y yo creo haber visto esa cara en alguna parte\ldots{}
sí\ldots{} pero no recuerdo dónde.

---Yo también le he visto a Vd. una vez, una vez sola---dijo Mariquilla
con aplomo,---y ojalá no me acordara.

---No olvidaré este beneficio---añadió Montoria.---Parece Vd. una buena
muchacha\ldots{} y muy linda por cierto. Adiós, estoy muy agradecido,
sumamente agradecido\ldots{} Venga un par de muletas, un bastón, que no
puedo andar, Sr.~de Araceli. Deme Vd. el brazo\ldots{} ¿Qué telarañas
son estas que se me ponen ante los ojos?\ldots{} Vamos allá, y echaremos
a los franceses del Hospital.

Disuadiéndole de su temerario propósito de salir, me disponía a marchar
yo solo, cuando se oyó una detonación tan fuerte, que ninguna palabra
del lenguaje tiene energía para expresarla. Parecía que la ciudad entera
era lanzada al aire por la explosión de un inmenso volcán abierto bajo
sus cimientos. Todas las casas temblaron; oscureciose el cielo con
inmensa nube de humo y de polvo, y a lo largo de la calle vimos caer
trozos de pared, miembros despedazados, maderos, tejas, lluvias de
tierra y material de todas clases.

---¡La Santa Virgen del Pilar nos asista!---exclamó Montoria.---Parece
que ha volado el mundo entero.

Los enfermos y heridos gritaban creyendo llegada su última hora, y todos
nos encomendamos mentalmente a Dios.

¿Qué es esto? ¿Existe todavía Zaragoza?---preguntaba uno.

---¿Volamos nosotros también?

---Debe haber sido en el Convento de San Francisco esta terrible
explosión---dije yo.

---Corramos allá---dijo Montoria sacando fuerzas de flaqueza.---Sr.~de
Araceli. ¿No decían que estaban tomadas todas las precauciones para
defender a San Francisco?\ldots{} ¡Pero no hay un par de muletas, por
ahí?

Salimos al Coso, donde al punto nos cercioramos de que una gran parte de
San Francisco había sido volada.

---Mi hijo estaba en el Convento---dijo Montoria pálido como un
difunto.---¡Dios mío, si has determinado que lo pierda también, que
muera por la patria en el puesto del honor!

Acercose a nosotros el locuaz mendigo de quien hice mención en las
primeras páginas de esta relación, el cual trabajosamente andaba con sus
muletas, y parecía en muy mal estado de salud.

\emph{---Sursum Corda}---le dijo el patriota,---dame tus muletas, que
para nada las necesitas.

---Déjeme su merced---repuso el cojo,---llegar a aquel portal y se las
daré. No quiero morirme en medio de la calle.

---¿Te mueres tú?

---¡Así parece! La calentura me abrasa. Estoy herido en el hombro desde
ayer y todavía no me han sacado la bala. Siento que me voy. Tome usía
las muletas.

---¿Vienes de San Francisco?

---No, señor; yo estaba en el arco del Trenque\ldots{} Allí había un
cañón: hemos hecho mucho fuego. Pero San Francisco ha volado por los
aires cuando menos lo creíamos. Toda la parte del Sur y de Poniente vino
al suelo, enterrando mucha gente. Ha sido traición, según dice el
pueblo\ldots{} Adiós, D. José\ldots{} Aquí me quedo\ldots{} Los ojos se
me oscurecen, la lengua se me traba, yo me voy\ldots{} la Señora Virgen
del Pilar me ampare, y aquí tiene usía mis remos.

Con ellos pudo avanzar un poco Montoria hacia el lugar de la catástrofe;
pero tuvimos que doblar la calle de San Gil, porque no se podía seguir
más adelante. Los franceses habían cesado de hostilizar el Convento por
el lado del Hospital; pero asaltándolo por San Diego, ocupaban a toda
prisa las ruinas, que nadie podía disputarles. Conservábase en pie la
iglesia y torre de San Francisco.

---¡Eh, padre Luengo!---dijo Montoria llamando al fraile de este nombre,
que entraba apresuradamente en la calle de San Gil.---¿Qué hay? ¿Dónde
está el Capitán general? ¿Ha perecido entre las ruinas?

---No---repuso el padre deteniéndose.---Está con otros jefes en la
plazuela de San Felipe. Puedo anunciarle a Vd. que su hijo Agustín se ha
salvado, porque era de los que ocupaban la torre.

---¡Bendito sea Dios!---dijo D. José cruzando las manos.

---Toda la parte de Sur y Poniente ha sido destruida---prosiguió
Luengo.---No se sabe cómo han podido minar por aquel sitio. Debieron
poner los hornillos debajo de la sala del capítulo, y por allí no se
habían hecho minas, creyendo que era lugar seguro.

---Además---dijo un paisano armado y que se acercó al grupo,---teníamos
la casa inmediata, y los franceses, posesionados sólo de parte de San
Diego y de Santa Rosa, no podían acercarse allí con facilidad.

---Por eso se cree---indicó un clérigo armado que se nos agregó---que
han encontrado un paso secreto entre Santa Rosa y la Casa de los
Duendes. Apoderados de los sótanos de esta, con una pequeña galería,
pudieron llegar hasta los subterráneos de la sala del Capítulo.

---Ya se sabe todo---dijo un capitán del ejército.---La Casa de los
Duendes tiene un gran sótano que nos era desconocido. Desde este sótano
partía, sin duda, una comunicación con Santa Rosa, a cuyo Convento
perteneció antiguamente dicho edificio y servía de granero y almacén.

---Pues si eso es cierto, si esa comunicación existe---añadió
Luengo,---ya comprendo quién se la ha descubierto a los franceses. Ya
saben Vds. que cuando los enemigos fueron rechazados en la huerta de San
Diego, se hicieron algunos prisioneros. Entre ellos está el tío
Candiola, que varias veces ha visitado estos días el campo francés, y
desde anoche se pasó al enemigo.

---Así tiene que ser---afirmó Montoria,---porque la Casa de los Duendes
pertenece a Candiola. Harto sabe el condenado judío los pasos y
escondrijos de aquel edificio. Señores, vamos a ver al Capitán general.
¿Se cree que aún podrá defenderse el Coso?

---¿Pues no se ha de defender?---dijo el militar.---Lo que ha pasado es
una friolera: algunos muertos más. Aún se intentará reconquistar la
iglesia de San Francisco.

Todos mirábamos a aquel hombre que tan serenamente hablaba de lo
imposible. La concisa sublimidad de su empeño parecía una burla, y sin
embargo, en aquella epopeya de lo increíble, semejantes burlas solían
parar en realidad.

Los que no den crédito a mis palabras, abran la historia y verán que
unas cuantas docenas de hombres extenuados, hambrientos, descalzos,
medio desnudos, algunos de ellos heridos, se sostuvieron todo el día en
la torre; mas no contentos con esto, extendiéronse por el techo de la
iglesia, y abriendo aquí y allí innumerables claraboyas, sin atender al
fuego que se les hacía desde el Hospital, empezaron a arrojar granadas
de mano contra los franceses, obligándoles a abandonar el templo al caer
de la tarde. Toda la noche pasó en tentativas del enemigo para
reconquistarlo; pero no pudieron conseguirlo hasta el día siguiente,
cuando los tiradores del tejado se retiraron, pasando a la casa de
Sástago.

\hypertarget{xxix}{%
\chapter{XXIX}\label{xxix}}

¿Zaragoza se rendirá? La muerte al que esto diga.

Zaragoza no se rinde. La reducirán a polvo: de sus históricas casas no
quedará ladrillo sobre ladrillo; caerán sus cien templos; su suelo
abrirase vomitando llamas; y lanzados al aire los cimientos, caerán las
tejas al fondo de los pozos; pero entre los escombros y entre los
muertos habrá siempre una lengua viva para decir que Zaragoza no se
rinde.

Llegó el momento de la suprema desesperación. Francia ya no combatía:
minaba. Era preciso desbaratar el suelo nacional para conquistarlo.
Medio Coso era suyo, y España destrozada se retiró a la acera de
enfrente. Por las Tenerías, por el arrabal de la izquierda, habían
alcanzado también ventajas, y sus hornillos no descansaban un instante.

Al fin ¡parece mentira!, nos acostumbramos a las voladuras, como antes
nos habíamos acostumbrado al bombardeo. A lo mejor se oía un ruido como
el de mil truenos retumbando a la vez. ¿Qué ha sido? Nada: la
Universidad, la capilla de la Sangre, la casa de Aranda, tal convento o
iglesia que ya no existe. Aquello no era vivir en nuestro pacífico y
callado planeta; era tener por morada las regiones del rayo, mundos
desordenados donde todo es fragor y desquiciamiento. No había sitio
alguno donde estar, porque el suelo ya no era suelo y bajo cada planta
se abría un cráter. Y sin embargo, aquellos hombres seguían
defendiéndose contra la inmensidad abrumadora de un volcán continuo y de
una tempestad incesante. A falta de fortalezas, habían servido los
conventos; a falta de conventos, los palacios; a falta de palacios, las
casas humildes. Todavía había algunas paredes.

Ya no se comía. ¿Para qué, si se esperaba la muerte de un momento a
otro? Centenares, miles de hombres perecían en las voladuras y la
epidemia había tomado carácter fulminante. Tenía uno la suerte de salir
ileso de entre la lluvia de balas, y luego al volver una esquina, el
horroroso frío y la fiebre, apoderándose súbitamente de la naturaleza,
le conducían en poco tiempo a la muerte. Ya no había parientes ni
amigos; menos aún: ya los hombres no se conocían unos a otros, y
ennegrecidos los rostros por la tierra, por el humo, por la sangre,
desencajados y cadavéricos, al juntarse después del combate, se
preguntaban: «¿quién eres tú? ¿Quién es usted?»

Ya las campanas no tocaban a alarma, porque no había campaneros: ya no
se oían pregones, porque no se publicaban proclamas; ya no se decía
misa, porque faltaban sacerdotes; ya no se cantaba la jota, y las voces
iban expirando en las gargantas a medida que iba muriendo gente. De hora
en hora el fúnebre silencio iba conquistando la ciudad. Sólo hablaba el
cañón, y las avanzadas de las dos naciones no se entretenían diciéndose
insultos. Más que de rabia, las almas empezaban a llenarse de tristeza,
y la ciudad moribunda se batía en silencio para que ni un átomo de
fuerza se le perdiera en voces importunas.

La necesidad de la rendición era una idea general; pero nadie la
manifestaba, guardándola en el fondo de su conciencia, como se guarda la
idea de la culpa que se va a cometer. ¡Rendirse! Esto parecía una
imposibilidad, una obra difícil, y perecer era más fácil.

Pasó un día después de la explosión de San Francisco; día horrible que
no parece haber existido en las series del tiempo, sino tan sólo en el
reino engañoso de la imaginación.

Yo había estado en la calle de las Arcadas poco antes de que la mayor
parte de sus casas se hundieran. Corrí después hacia el Coso a cumplir
una comisión que se me encargó y recuerdo que la pesada e infecta
atmósfera de la ciudad me ahogaba, de tal modo que apenas podía andar.
Por el camino encontré el mismo niño que algunos días antes vi llorando
y solo en el barrio de las Tenerías. También entonces iba solo y
llorando, y además el infeliz metía las manos en la boca, como si se
comiese los dedos. A pesar de esto nadie le hacía caso. Yo también pasé
con indiferencia por su lado; pero después una vocecilla dijo algo en mi
conciencia, volví atrás y me le llevé conmigo, dándole algunos pedazos
de pan.

Cumplida mi comisión, corrí a la plazuela de San Felipe, donde después
de lo de las Arcadas, estaban los pocos hombres que aun subsistían de mi
batallón. Era ya de noche, y aunque en el Coso había gran fuego entre
una y otra acera, los míos fueron dejados en reserva para el día
siguiente, porque estaban muertos de cansancio.

Al llegar vi un hombre que envuelto en su capote paseaba de largo a
largo sin hacer caso de nada ni de nadie. Era Agustín Montoria.

---¡Agustín! ¿Eres tú?---le dije acercándome.---¡Qué pálido y demudado
estás! ¿Te han herido?

---Déjame---me contestó agriamente,---no quiero compañías importunas.

---¿Estás loco? ¿Qué te pasa?

---Déjame, te digo---añadió, repeliéndome con fuerza.---Te digo que
quiero estar solo. No quiero ver a nadie.

---Amigo---exclamé, comprendiendo que algún terrible pesar perturbaba el
alma de mi compañero,---si te ocurre algo desagradable dímelo y tomaré
para mí una parte de tu desgracia.

---¿Pues no lo sabes?

---No sé nada. Ya sabes que me mandaron con veinte hombres a la calle de
las Arcadas. Desde ayer, desde la explosión de San Francisco, no nos
hemos visto.

---Es verdad---repuso.---Gabriel, he buscado la muerte en esa barricada
del Coso y la muerte no ha querido venir. Innumerables compañeros míos
cayeron a mi lado y no ha habido una bala para mí. Gabriel, amigo mío
querido, pon el cañón de una de tus pistolas en mi sien y arráncame la
vida. ¿Lo creerás? Hace poco intenté matarme\ldots{} No sé\ldots{}
parece que vino una mano invisible y me apartó el arma de las sienes.
Después, otra mano suave y tibia pasó por mi frente.

---Cálmate, Agustín, y cuéntame lo que tienes.

---¡Lo que tengo! ¿Qué hora es?

---Las nueve.

---Falta una hora---exclamó con nervioso estremecimiento.---¡Sesenta
minutos! Puede ser que los franceses hayan minado esta plazuela de San
Felipe donde estamos, y tal vez dentro de un instante la tierra,
saltando bajo nuestros pies, abra una horrible sima en que todos
quedemos sepultados, todos, la víctima y los verdugos.

---¿Qué víctima es esa?

---¿No lo sabes? El desgraciado Candiola. Está encerrado en la Torre
Nueva.

En la puerta de la Torre Nueva había algunos soldados, y una macilenta
luz alumbraba la entrada.

---En efecto---dije,---sé que ese infame viejo fue cogido prisionero con
algunos franceses en la huerta de San Diego.

---Su crimen es indudable. Enseñó a los enemigos el paso desde Santa
Rosa a la casa de los Duendes, de él solo conocido. Además de que no
faltan pruebas, el infeliz esta tarde ha confesado todo con esperanza de
salvar la vida.

---Le han condenado\ldots{}

---Sí. El consejo de guerra no ha discutido mucho. Candiola será
arcabuceado dentro de una hora por traidor. ¡Allí está! Y aquí me tienes
a mí, Gabriel, aquí me tienes a mí, capitán del batallón de las Peñas de
San Pedro; ¡malditas charreteras!, aquí me tienes con una orden en el
bolsillo en que se me manda ejecutar la sentencia a las diez de la
noche, en este mismo sitio, aquí, en la plazuela de San Felipe, al pie
de la torre. ¿Ves, ves la orden? Está firmada por el general
Saint-March.

Callé, porque no se me ocurría una sola palabra que decir a mi compañero
en aquella terrible ocasión.

---¡Amigo mío, valor!---exclamé al fin.---Es preciso cumplir la orden.

Agustín no me oía. Su actitud era la de un demente y se apartaba de mí
para volver enseguida, balbuciendo palabras de desesperación. Después
mirando a la torre, que majestuosa y esbelta alzábase sobre nuestras
cabezas, exclamó con terror:

---Gabriel, ¿no la ves, no ves la torre? ¿No ves que está derecha,
Gabriel? La torre se ha puesto derecha. ¿No la ves? ¿Pero no la ves?

Miré a la torre, y, como era natural, la torre continuaba inclinada.

---Gabriel---añadió Montoria,---mátame: no quiero vivir. No: yo no le
quitaré a ese hombre la vida. Encárgate tú de esta comisión. Yo, si
vivo, quiero huir; estoy enfermo; me arrancaré estas charreteras y se
las tiraré a la cara al general Saint-March. No, no me digas que la
Torre Nueva sigue inclinada. Pero hombre, ¿no ves que está derecha?
Amigo, tú me engañas; mi corazón está traspasado por un acero candente,
rojo, y la sangre chisporrotea. Me muero de dolor.

Yo procuraba consolarle, cuando una figura blanca penetró en la plaza
por la calle de Torresecas. Al verla temblé de espanto: era Mariquilla.
Agustín no tuvo tiempo de huir, y la desgraciada joven se abrazó a él,
exclamando con ardiente emoción:

---Agustín, Agustín. Gracias a Dios que te encuentro aquí. ¡Cuánto te
quiero! Cuando me dijeron que eras tú el carcelero de mi padre, me volví
loca de alegría, porque tengo la seguridad de que has de salvarle. Esos
caribes del Consejo le han condenado a muerte. ¡A muerte! ¡Morir él, que
no ha hecho mal a nadie! Pero Dios no quiere que el inocente perezca, y
le ha puesto en tus manos para que le dejes escapar.

---Mariquilla, María de mi corazón---dijo Agustín.---Déjame,
vete\ldots{} no te quiero ver\ldots{} Mañana, mañana hablaremos. Yo
también te amo\ldots{} Estoy loco por ti. Húndase Zaragoza; pero no
dejes de quererme. Esperaban que yo matara a tu padre\ldots{}

---Jesús, no digas eso---exclamó la muchacha.---¡Tú!

---No, mil veces no; que castiguen otros su traición.

---No, mentira, mi padre no ha sido traidor. ¿Tú también le acusas?
Nunca lo creí\ldots{} Agustín, es de noche. Desata sus manos, quítale
los grillos que destrozan sus pies; ponle en libertad. Nadie lo puede
ver. Huiremos; nos esconderemos aquí cerca, en las ruinas de nuestra
casa, allí en la sombra del ciprés, en aquel mismo sitio donde tantas
veces hemos visto el pico de la Torre Nueva.

---María\ldots{} espera un poco\ldots---dijo Montoria con suma
agitación.---Eso no puede hacerse así\ldots{} Hay mucha gente en la
plaza. Los soldados están muy irritados contra el preso. Mañana\ldots{}

---¡Mañana! ¿Qué has dicho? ¿Te burlas de mí? Ponle al instante en
libertad, Agustín. Si no lo haces, creeré que he amado al más vil, al
más cobarde y despreciable de los hombres.

---María, Dios nos está oyendo. Dios sabe que te adoro. Por él juro que
no mancharé mis manos con la sangre de ese infeliz; antes romperé mi
espada, pero en nombre de Dios te digo también que no puedo poner en
libertad a tu padre. María, el cielo se nos ha caído encima.

---Agustín, me estás engañando---dijo la joven con angustiosa
perplejidad.---¿Dices que no le pondrás en libertad?

---No, no, no puedo. Si Dios en forma humana viniera a pedirme la
libertad del que ha vendido a nuestros heroicos paisanos, entregándolos
al cuchillo francés, no podría hacerlo. Es un deber supremo al que no
puedo faltar. Las innumerables víctimas inmoladas por la traición; la
ciudad rendida, el honor nacional ultrajado, son recuerdos y
consideraciones que pesan en mi conciencia de un modo formidable.

---Mi padre no puede haber hecho traición---dijo Mariquilla, pasando
súbitamente del dolor a una exaltada y nerviosa cólera.---Son calumnias
de sus enemigos. Mienten los que le llaman traidor, y tú, más cruel y
más inhumano que todos, mientes también. No, no es posible que yo te
haya amado: vergüenza me causa pensarlo. ¿Has dicho que no le pondrás en
libertad? ¿Pues para qué existes, de qué sirves tú? ¿Esperas ganar con
tu crueldad sanguinaria el favor de esos bárbaros inhumanos que han
destruido la ciudad, fingiendo defenderla? ¡Para ti nada vale la vida
del inocente ni la desolación de una huérfana! ¡Miserable y ambicioso
egoísta, te aborrezco más de lo que te he querido! ¿Has pensado que
podrías presentarte delante de mí con las manos manchadas en la sangre
de mi padre? No, él no ha sido traidor. Traidor eres tú y todos los
tuyos. ¡Dios mío! ¿No hay un brazo generoso que me ampare; no hay entre
tantos hombres uno solo que impida este crimen? ¡Una pobre mujer corre
por toda la ciudad buscando un alma caritativa, y no encuentra más que
fieras!

---María---dijo Agustín,---me estás despedazando el alma; me pides lo
imposible, lo que yo no haré ni puedo hacer, aunque en pago me ofrezcas
la bienaventuranza eterna. Todo lo he sacrificado ya, y contaba con que
me aborrecerías. Considera que un hombre se arranca con sus propias
manos el corazón y lo arroja al lodo; pues eso he hecho yo. No puedo
más.

La ardiente exaltación de María Candiola la llevaba de la ira más
intensa a la sensibilidad más patética. Antes mostraba con enérgica
fogosidad su cólera, y después se deshacía en lágrimas amargas,
expresándose así:

---¿Qué he dicho, y qué locuras has dicho tú? ¡Agustín, tú no puedes
negarme lo que te pido! ¡Cuánto te he querido y cuánto te quiero! Desde
que te vi por primera vez en nuestra \emph{torre}, no te has apartado un
solo instante de mi pensamiento. Tú has sido para mí el más amable, el
más generoso, el más discreto, el más valiente de todos los hombres. Te
amé sin saber quién eras; yo ignoraba tu nombre y el de tus padres; pero
te habría amado aunque hubieras sido hijo del verdugo de Zaragoza.
Agustín: tú te has olvidado de mí desde que no nos vemos. ¡Soy yo,
Mariquilla! Siempre he creído y creo que no me quitarás a mi buen padre,
a quien amo tanto como a ti. Él es bueno; no ha hecho mal a nadie, es un
pobre anciano\ldots{} Tiene algunos defectos; pero yo no los veo, yo no
veo en él más que virtudes. No he conocido a mi madre, que murió siendo
yo muy niña; he vivido retirada del mundo; mi padre me ha criado en la
soledad, y en la soledad se ha formado el grande amor que te tengo. Si
no te hubiera conocido a ti, todo el mundo me hubiera sido indiferente
sin él.

Leí claramente en el semblante de Montoria la indecisión. Él miraba con
aterrados ojos tan pronto a la muchacha como a los hombres que estaban
de centinela en la entrada de la torre, y la hija de Candiola, con
admirable instinto, supo aprovechar esta disposición a la debilidad, y
echándole los brazos al cuello, añadió:

---Agustín, ponle en libertad. Nos ocultaremos donde nadie pueda
descubrirnos. Si te dicen algo, si te acusan de haber faltado al deber,
no les hagas caso y vente conmigo. ¡Cuánto te amará mi padre al ver que
le salvas la vida! Entonces ¡qué gran felicidad nos espera, Agustín!
¡Qué bueno eres! Ya lo esperaba yo, y cuando supe que el pobre preso
estaba en tu poder, se me figuró que me abrían las puertas del cielo.

Mi amigo dio algunos pasos y retrocedió después. Había bastantes
militares y gente armada en la plazuela. De repente se nos apareció
delante un hombre con muletas, acompañado de otros paisanos y algunos
oficiales de alta graduación.

---¿Qué pasa aquí?---dijo D. José de Montoria.---Me pareció oír
chillidos de mujer. Agustín, ¿estás llorando? ¿Qué tienes?

---Señor---gritó Mariquilla con terror, volviéndose hacia
Montoria.---Vd. no se opondrá tampoco a que dejen en libertad a mi
padre. ¿No se acuerda Vd. de mí? Ayer estaba Vd. herido y yo le curé.

---Es verdad, niña---dijo gravemente D. José.---Estoy muy agradecido.
Ahora caigo en que es Vd. la hija del Sr.~Candiola.

---Sí señor: ayer, cuando le curaba a Vd., reconocí en su cara la de
aquel hombre que maltrató a mi padre hace muchos días.

---Sí, hija mía, fue un arrebato, un pronto\ldots{} No lo pude
remediar\ldots{} Tengo la sangre muy viva\ldots{} Y usted me
curó\ldots{} Así se portan los buenos cristianos. Pagar las injurias con
beneficios, y hacer bien a los que nos aborrecen es lo que manda Dios.

---Señor---exclamó María toda deshecha en lágrimas,---yo perdono a mis
enemigos: perdone usted también a los suyos. ¿Por qué no han de poner en
libertad a mi padre? Él no ha hecho nada.

---Es un poco difícil lo que Vd. pretende. La traición del Sr.~Candiola
no puede perdonarse. La tropa está furiosa.

---¡Todo es un error! Si Vd. quiere interceder\ldots{} Usted será de los
que mandan.

---¿Yo?\ldots---dijo Montoria.---Ese es un asunto que no me
incumbe\ldots{} Pero serénese Vd., joven\ldots{} De veras que parece Vd.
una buena muchacha. Recuerdo el esmero con que me curaba, y me llega al
alma tanta bondad. Grande ofensa hice a Vd., y de la misma persona a
quien ofendí he recibido un bien inmenso, ¡tal vez la vida! De este modo
nos enseña Dios con un ejemplo que debemos ser humildes y caritativos,
¡porr\ldots!, ¡ya la iba a soltar\ldots! ¡Maldita lengua mía!

¡Señor, qué bueno es Vd.!---exclamó la joven.---¡Yo le creía muy malo!
Vd. me ayudará a salvar a mi padre. Él tampoco se acuerda del ultraje
recibido.

---Oiga Vd.---le dijo Montoria tomándola por un brazo.---Hace poco pedí
perdón al Sr.~D. Jerónimo por aquel vejamen, y lejos de reconciliarse
conmigo, me insultó del modo más grosero. Él y yo no casamos, niña.
Dígame Vd. que me perdona lo de los golpes, y mi conciencia se
descargará de un gran peso.

---¡Pues no le he de perdonar! ¡Oh señor, qué bueno es Vd.! Vd. manda
aquí sin duda. Pues haga poner en libertad a mi padre.

---Eso no es de mi cuenta. El Sr.~Candiola ha cometido un crimen que
espanta. Es imposible perdonarle, imposible: comprendo la aflicción de
Vd\ldots{} De veras lo siento; mayormente al acordarme de su
caridad\ldots{} Ya la protegeré a Vd\ldots{} Veremos.

---Yo no quiero nada para mí---dijo María, ronca ya de tanto
gritar.---Yo no quiero sino que pongan en libertad a un infeliz que nada
ha hecho. Agustín, ¿no mandas aquí? ¿Qué haces?

---Este joven cumplirá con su deber.

---Este joven---repuso la Candiola con furor,---hará lo que yo le mande,
porque me ama. ¿No es verdad que pondrás en libertad a mi padre? Tú me
lo dijiste\ldots{} Señores, ¿qué buscan ustedes aquí? ¿Piensan
impedirlo? Agustín, no les hagas caso y defendámonos.

---¿Qué es esto?---exclamó Montoria con estupefacción.---Agustín, ¿ha
dicho esta muchacha que te disponías a faltar a tu deber? ¿La conoces
tú?

Agustín, dominado por profundo temor, no contestó nada.

---Sí, le pondrá en libertad---exclamó María con desesperación.---Fuera
de aquí, señores. Aquí no tienen Vds. nada que hacer.

---¡Cómo se entiende!---gritó D. José, tomando a su hijo por un
brazo.---Si lo que esta muchacha dice fuera cierto; si yo supiera que mi
hijo faltaba al honor de ese modo, atropellando la lealtad jurada al
principio de autoridad delante de las banderas; si yo supiera que mi
hijo hacía burla de las órdenes cuyo cumplimiento se le ha encargado, yo
mismo le pasaría una cuerda por los codos, llevándole delante del
consejo de guerra para que le dieran su merecido.

---¡Señor, padre mío!---repuso Agustín, pálido como la muerte.---Jamás
he pensado en faltar a mi deber.

---¿Es este tu padre?---dijo María.---Agustín, dile que me amas, y
quizás tenga compasión de mí.

---Esta joven está loca---afirmó D. José.---Desgraciada niña: la
tribulación de Vd. me llega al alma. Yo me encargo de protegerla en su
orfandad\ldots{} Pero serénese Vd.. Sí, la protegeré, siempre que usted
reforme sus costumbres\ldots{} Pobrecilla: Vd. tiene buen
corazón\ldots{} un excelente corazón\ldots{} pero\ldots{} sí\ldots{} me
lo han dicho, un poco levantada de cascos\ldots{} Es lástima que por una
perversa educación se pierda una buena alma\ldots{} Con que ¿será Vd.
buena?\ldots{} Creo que sí\ldots{}

---Agustín, ¿cómo permites que me insulten?---exclamó María con inmenso
dolor.

---No os insulto---añadió el padre.---Es un consejo. ¡Cómo había yo de
insultar a mi bienhechora! Creo que si Vd. se porta bien, le tendremos
gran cariño. Queda Vd. bajo mi protección, desgraciada
huerfanita\ldots{} ¿Para qué toma Vd. en boca a mi hijo? Nada, nada: mas
juicio, y por ahora basta ya de agitación\ldots{} El chico tal vez la
conozca a Vd\ldots{} Sí, me han dicho que durante el sitio no ha
abandonado Vd. la compañía de los soldados\ldots{} Es preciso
enmendarse: yo me encargo\ldots{} No puedo olvidar el beneficio
recibido; además, conozco que su fondo es bueno\ldots{} Esa cara no
miente; tiene Vd. una figura celestial. Pero es preciso renunciar a los
goces mundanos, refrenar el vicio\ldots{} pues\ldots{}

---No---gritó de súbito Agustín, con tan vivo arrebato de ira, que todos
temblamos al verle y oírle.---No, no consiento a nadie, ni aun a mi
padre, que la injurie delante de mí. Yo la amo, y si antes lo he
ocultado, ahora lo digo aquí sin miedo ni vergüenza, para que todo el
mundo lo sepa. Señor, Vd. no sabe lo que está diciendo ni cuánto falta a
lo verdadero, sin duda porque le han engañado. Máteme Vd. si le falto al
respeto; pero no la infame delante de mí, porque oyendo otra vez lo que
he oído, ni la presencia de mi propio padre me reportaría.

Montoria, que no esperaba aquello, miró con asombro a sus amigos.

---Bien, Agustín---exclamó la Candiola.---No hagas caso de esa gente.
Este hombre no es tu padre. Haz lo que te indica tu buen corazón. ¡Fuera
de aquí, señores, fuera de aquí!

---Te engañas, María---repuso el joven.---Yo no he pensado poner en
libertad al preso, ni lo pondré; pero al mismo tiempo digo que no seré
yo quien disponga su muerte. Oficiales hay en mi batallón que cumplirán
la orden. Ya no soy militar: aunque esté delante del enemigo, arrojo mi
espada, y corro a presentarme al capitán general para que disponga de mi
suerte.

Diciendo esto, desenvainó, y doblando la hoja sobre la rodilla,
rompiola, y después de arrojar los dos pedazos en medio del corrillo, se
fue sin decir una palabra más.

---¡Estoy sola! ¡Ya no hay quien me ampare!---exclamó Mariquilla con
abatimiento.

---No hagan Vds. caso de las barrabasadas de mi hijo---dijo
Montoria.---Ya le tomaré yo por mi cuenta. Tal vez la muchacha le haya
interesado\ldots{} pues\ldots{} no tiene nada de particular. Estos
eclesiásticos inexpertos suelen ser así\ldots{} Y Vd. señorita doña
María, procure serenarse\ldots{} Ya nos ocuparemos de Vd. Yo le prometo
que si tiene buena conducta, se le conseguirá que entre en las
Arrepentidas\ldots{} Vamos, llevarla fuera de aquí.

---¡No, no me sacarán de aquí sino a pedazos!---gritó la muchacha en el
colmo de la desolación.---¡Oh! Sr.~D. José de Montoria: Vd. le pidió
perdón a mi padre. Si él no le perdonó, yo le perdono mil veces\ldots{}
Pero\ldots{}

---Yo no puedo hacer lo que Vd. me pide---replicó el patriota con
pena.---El crimen cometido es enorme. Retírese Vd\ldots{} ¡Qué espantoso
dolor! ¡Es preciso tener resignación! Dios le perdonará a Vd. todas sus
culpas, pobre huerfanita\ldots{} Cuente Vd. conmigo, y todo lo que yo
pueda\ldots{} la socorreremos, la auxiliaremos\ldots{} Estoy conmovido,
y no sólo por agradecimiento, sino por lástima\ldots{} Vamos, venga Vd.
conmigo\ldots{} Son las diez menos cuarto.

---Sr.~Montoria---dijo María poniéndose de rodillas delante del patriota
y besándole las manos.---Vd. tiene influencia en la ciudad, y puede
salvar a mi padre. Se ha enfadado Vd. conmigo, porque Agustín dijo que
me quería. No, no le amo; ya no le miraré más. Aunque soy honrada, él es
superior a mí, y no puedo pensar en casarme con él. Sr.~de Montoria, por
el alma de su hijo muerto, hágalo Vd. Mi padre es inocente. No, no es
posible que haya sido traidor. Aunque el Espíritu Santo me lo dijera, no
lo creería. Dicen que no era patriota: mentira, yo digo que mentira.
Dicen que no dio nada para la guerra; pues ahora se dará todo lo que
tenemos. En el sótano de casa hay enterrado mucho dinero. Yo le diré a
Vd. dónde está, y pueden llevárselo todo. Dicen que no ha tomado las
armas. Yo las tomaré ahora: no temo las balas, no me asusta el ruido del
cañón, no me asusto de nada; volaré al sitio de mayor peligro, y allí
donde no puedan resistir los hombres me pondré yo sola ante el fuego. Yo
sacaré con mis manos la tierra de las minas, y haré agujeros para llenar
de pólvora todo el suelo que ocupan los franceses. Dígame Vd. si hay
algún castillo que tomar, o alguna muralla que defender, porque nada
temo, y de todas las personas que aún viven en Zaragoza, yo seré la
última que se rinda.

---Desgraciada muchacha---murmuró el patriota, alzándola del
suelo.---Vámonos, vámonos de aquí.

---Sr.~de Araceli---ordenó el jefe de la fuerza, que era uno de los
presentes---, puesto que el capitán don Agustín Montoria no está en su
puesto, encárguese Vd. del mando de la compañía.

---No, asesinos de mi padre---exclamó María, no ya exasperada, sino
furiosa como una leona.---No mataréis al inocente. Cobardes verdugos,
los traidores sois vosotros, no él. No podéis vencer a vuestros
enemigos, y os gozáis en quitar la vida a un infeliz anciano. Militares,
¿a qué habláis de vuestro honor, si no sabéis lo que es eso? Agustín,
¿dónde estás? Sr.~D. José de Montoria, esto que ahora pasa es una ruin
venganza, tramada por Vd., hombre rencoroso y sin corazón. Mi padre no
ha hecho mal a nadie. Vds. intentaban robarle\ldots{} Bien hacía él en
no querer dar su harina, porque los que se llaman patriotas, son
negociantes que especulan con las desgracias de la ciudad\ldots{} No
puedo arrancar a estos crueles una palabra compasiva. Hombres de bronce,
bárbaros, mi padre es inocente, y si no lo es, bien hizo en vender la
ciudad. Siempre le darían más de lo que Vds. valen\ldots{} ¿Pero no hay
uno, uno tan solo, que se apiade de él y de mí?

---Vamos: retirémosla,---señores; llevarla a cuestas. ¡Infeliz
muchacha!---dijo Montoria.---Esto no puede prolongarse. ¿En dónde se ha
metido mi hijo?

Se la llevaron, y durante un rato oí desde la plazuela sus desgarradores
gritos.

---Buenas noches, Sr.~de Araceli---me dijo Montoria.---Voy a ver si hay
un poco de agua y vino que dar a esa pobre huérfana.

\hypertarget{xxx}{%
\chapter{XXX}\label{xxx}}

Vete lejos de mí, horrible pesadilla. No quiero dormir. Pero el mal
sueño que anhelo desechar vuelve a mortificarme. Quiero borrar de mi
imaginación la lúgubre escena; pero pasa una noche y otra, y la escena
no se borra. Yo, que en tantas ocasiones he afrontado sin pestañear los
mayores peligros, hoy tiemblo: mi cuerpo se estremece y helado sudor
corre por mi frente. La espada teñida en sangre de franceses se cae de
mi mano y cierro los ojos para no ver lo que pasa delante de mí.

En vano te arrojo, imagen funesta. Te expulso y vuelves porque has
echado profunda raíz en mi cerebro. No, yo no soy capaz de quitar a
sangre fría la vida a un semejante, aunque un deber inexorable me lo
ordene. ¿Por qué no temblaba en las trincheras, y ahora tiemblo? Siento
un frío mortal. A la luz de las linternas veo algunas caras siniestras;
una sobre todo, lívida y hosca que expresa un espanto superior a todos
los espantos. ¡Cómo brillan los cañones de los fusiles! Todo está
preparado, y no falta más que una voz, mi voz. Trato de pronunciar la
palabra, y me muerdo la lengua. No, esa palabra no saldrá jamás de mis
labios.

Vete lejos de mí, negra pesadilla. Cierro los ojos, me aprieto los
párpados con fuerza para cerrarlos mejor, y cuanto más los cierro más te
veo, horrendo cuadro. Esperan todos con ansiedad; pero ninguna ansiedad
es comparable a la de mi alma, rebelándose contra la ley que obliga a
determinar el fin de una existencia extraña. El tiempo pasa, y unos ojos
que yo no quisiera haber visto nunca, desaparecen bajo una venda. Yo no
puedo ver tal espectáculo y quisiera que pusieran también un lienzo en
los míos. Los soldados me miran, y yo disimulo mi cobardía, frunciendo
el ceño. Somos estúpidos y vanos hasta en los momentos supremos. Parece
que los circunstantes se burlan de mi perplejidad, y esto me da cierta
energía. Entonces despego mi lengua del paladar y grito: \emph{¡Fuego!}

La maldita pesadilla no se quiere ir, y me atormenta esta noche, como
anoche, y como anteanoche, reproduciéndome lo que no quiero ver. Más
vale no dormir, y prefiero el insomnio. Sacudo el letargo, y aborrezco
despierto la vigilia como antes aborrecía el sueño. Siempre el mismo
zumbido de los cañones. Esas insolentes bocas de bronce no han cesado de
hablar aún. Han pasado días, y Zaragoza no se ha rendido, porque todavía
algunos locos se obstinan en guardar para España aquel montón de polvo y
ceniza. Siguen reventando los edificios, y Francia después de sentar un
pie, gasta ejércitos y quintales de pólvora para conquistar terreno en
que poner el otro. España no se retira mientras tenga una baldosa en que
apoyar la inmensa máquina de su bravura.

Yo estoy exánime y no me puedo mover. Esos hombres que veo pasar por
delante de mí no parecen hombres. Están flacos, macilentos, y sus
rostros serían amarillos, si no les ennegrecieran el polvo y el humo.
Brillan bajo la negra ceja los ojos que ya no saben mirar sino matando.
Se cubren de inmundos harapos, y un pañizuelo ciñe su cabeza como un
cordel. Están tan escuálidos, que parecen los muertos del montón de la
calle de la Imprenta, que se han levantado para relevar a los vivos.
Generales, soldados, paisanos, frailes, mujeres, todos están
confundidos. No hay clases ni sexos. Nadie manda ya, y la ciudad se
defiende en la anarquía.

No sé lo que me pasa. No me digáis que siga contando, porque ya no hay
nada. Ya no hay nada que contar, y lo que veo no parece cosa real,
confundiéndose en mi memoria lo verdadero con lo soñado. Estoy tendido
en un portal de la calle de la Albardería, y tiemblo de frío; mi mano
izquierda está envuelta en un lienzo lleno de sangre y fango. La
calentura me abrasa, y anhelo tener fuerzas para acudir al fuego. No son
cadáveres todos los que hay a mi lado. Alargo la mano, y toco el brazo
de un amigo que vive aún.

---¿Qué ocurre, Sr.~\emph{Sursum Corda?}---le preguntó.

---Los franceses parece que están del lado acá del Coso---me contesta
con voz desfallecida.---Han volado media ciudad. Puede ser que sea
preciso rendirse. El capitán general ha caído enfermo de la epidemia y
está en la calle de Predicadores. Creen que se morirá. Entrarán los
franceses. Me alegro de morirme para no verlos. ¿Qué tal se encuentra
Vd., Sr.~de Araceli?

---Muy mal. Veré si puedo levantarme.

---Yo estoy vivo todavía, a lo que parece. No lo creí. El Señor sea
conmigo. Me iré derecho al cielo. Sr.~de Araceli, ¿se ha muerto Vd. ya?

Me levanto y doy algunos pasos. Apoyándome en las paredes, avanzo un
poco y llego junto a las Escuelas Pías. Algunos militares de alta
graduación acompañan hasta la puerta a un clérigo pequeño y delgado, que
les despide diciendo: «Hemos cumplido con nuestro deber, y la fuerza
humana no alcanza a más.» Era el Padre Basilio.

Un brazo amigo me sostiene y reconozco a don Roque.

---Amigo Gabriel---me dice con aflicción.---La ciudad se rinde hoy
mismo.

---¿Qué ciudad?

---Esta.

Al hablar así, me parece que nada está en su sitio. Los hombres y las
casas, todo corre en veloz fuga. La Torre Nueva saca sus pies de los
cimientos para huir también, y desapareciendo a lo lejos, el capacete de
plomo se le cae de un lado. Ya no resplandecen las llamas de la ciudad.
Columnas de negro humo corren de Levante a Poniente, y el polvo y la
ceniza, levantados por los torbellinos del viento, marchan en la misma
dirección. El cielo no es cielo, sino un toldo de color plomizo, que
tampoco está quieto.

---Todo huye, todo se va de este lugar de desolación---digo a D.
Roque.---Los franceses no encontrarán nada.

---Nada: hoy entran por la puerta del Ángel. Dicen que la capitulación
ha sido honrosa. Mira: ahí vienen los espectros que defendían la plaza.

En efecto, por el Coso desfilan los últimos combatientes, aquel uno por
mil que había resistido a las balas y a la epidemia. Son padres sin
hijos, hermanos sin hermanos, maridos sin mujer. El que no puede
encontrar a los suyos entre los vivos, tampoco es fácil que los
encuentre entre los muertos, porque hay cincuenta y dos mil cadáveres,
casi todos arrojados en las calles, en los portales de las casas, en los
sótanos, en las trincheras. Los franceses, al entrar, se detienen llenos
de espanto ante tan horrible espectáculo, y casi están a punto de
retroceder. Las lágrimas corren de sus ojos y se preguntan si son
hombres o sombras las pocas criaturas con movimiento que discurren ante
su vista.

El soldado voluntario, al entrar en su casa, tropieza con los cuerpos de
su esposa y de sus hijos. La mujer corre a la trinchera, al paredón, a
la barricada, y busca a su marido. Nadie sabe dónde está: los mil
muertos no hablan y no pueden dar razón de si está Fulano entre ellos.
Familias numerosas se encuentran reducidas a cero, y no queda en ellas
uno solo que eche de menos a los demás. Esto ahorra muchas lágrimas, y
la muerte ha herido de un solo golpe al padre y al huérfano, al esposo y
a la viuda, a la víctima y a los ojos que habían de llorarla.

Francia ha puesto al fin el pie dentro de aquella ciudad edificada a
orillas del clásico río que da su nombre a nuestra Península; pero la ha
conquistado sin domarla. Al ver tanto desastre y el aspecto que ofrece
Zaragoza, el ejército imperial, más que vencedor, se considera
sepulturero de aquellos heroicos habitantes. Cincuenta y tres mil vidas
le tocaron a la ciudad aragonesa en el contingente de doscientos
millones de criaturas con que la humanidad pagó las glorias militares
del imperio francés.

Este sacrificio no será estéril, como sacrificio hecho en nombre de una
idea. El imperio francés, cosa vana y de circunstancias, fundado en la
movible fortuna, en la audacia, en el genio militar que siempre es
secundario, cuando abandonando el servicio de la idea, sólo existe en
obsequio de sí propio; el imperio francés, digo; aquella tempestad que
conturbó los primeros años del siglo y cuyos relámpagos, truenos y rayos
aterraron tanto a la Europa, pasó, porque las tempestades pasan, y lo
normal en la vida histórica, como en la naturaleza, es la calma. Todos
le vimos pasar, y presenciamos su agonía en 1815: después vimos su
resurrección algunos años adelante, pero también pasó, derribado el
segundo como el primero por la propia soberbia. Tal vez retoñe por
tercera vez este árbol viejo; pero no dará sombra al mundo durante
siglos, y apenas servirá para que algunos hombres se calienten con el
fuego de su última leña.

Lo que no ha pasado ni pasará es la idea de nacionalidad que España
defendía contra el derecho de conquista y la usurpación. Cuando otros
pueblos sucumbían, ella mantiene su derecho, lo defiende, y sacrificando
su propia sangre y vida, lo consagra, como consagraban los mártires en
el circo la idea cristiana. El resultado es que España, despreciada
injustamente en el Congreso de Viena, desacreditada con razón por sus
continuas guerras civiles, sus malos gobiernos, su desorden, sus
bancarrotas más o menos declaradas, sus inmorales partidos, sus
extravagancias, sus toros y sus pronunciamientos, no ha visto nunca,
después de 1808, puesta en duda la continuación de su nacionalidad; y
aún hoy mismo, cuando parece hemos llegado al último grado del
envilecimiento, con más motivos que Polonia para ser repartida, nadie se
atreve a intentar la conquista de esta casa de locos.

Hombres de poco seso, o sin ninguno en ocasiones, los españoles darán
mil caídas hoy como siempre, tropezando y levantándose, en la lucha de
sus vicios ingénitos, de las cualidades eminentes que aún conservan, y
de las que adquieren lentamente con las ideas que les envía la Europa
central. Grandes subidas y bajadas, grandes asombros y sorpresas,
aparentes muertes y resurrecciones prodigiosas, reserva la Providencia a
esta gente, porque su destino es poder vivir en la agitación como la
salamandra en el fuego; pero su permanencia nacional está y estará
siempre asegurada.

\hypertarget{xxxi}{%
\chapter{XXXI}\label{xxxi}}

Era el 21 de Febrero. Un hombre que no conocí se me acercó y me dijo:

---Ven, Gabriel,---necesito de ti.

---¿Quién es Vd.?---le pregunté.---Yo no le conozco a Vd..

---Soy Agustín Montoria---repuso.---¿Tan desfigurado estoy? Ayer me
dijeron que habías muerto. ¡Qué envidia te tenía! Veo que eres tan
desgraciado como yo, y vives aún. ¿Sabes, amigo mío, lo que acabo de
ver? Acabo de ver el cuerpo de Mariquilla. Está en la calle de Antón
Trillo, a la entrada de la huerta. Ven y la enterraremos.

---Yo más estoy para que me entierren que para enterrar. ¿Quién se ocupa
de eso? ¿De qué ha muerto esa mujer?

---De nada, Gabriel, de nada.

---Es singular muerte: no la entiendo.

---Mariquilla no tiene heridas ni las señales que deja en el rostro la
epidemia. Parece que se ha dormido. Apoya la cara contra el suelo, y
tiene las manos en ademán de taparse fuertemente los oídos.

---Hace bien. Le molesta el ruido de los tiros. Lo mismo me pasa a mí
que todavía los siento.

---Ven conmigo y me ayudarás. Llevo una azada.

Difícilmente llegué a donde mi amigo con otros dos compañeros me
llevaba. Mis ojos no podían fijarse bien en objeto alguno, y sólo vi una
sombra tendida. Agustín y los otros dos levantaron aquel cuerpo
fantasma, vana imagen o desconsoladora realidad que allí existía. Creo
haber distinguido su cara, y al verla, tristísima penumbra se extendió
por mi alma.

---No tiene ni la más ligera herida---decía Agustín,---ni una gota de
sangre mancha sus vestidos. Sus párpados no se han hinchado como los que
mueren de la epidemia. María no ha muerto de nada. ¿La ves, Gabriel?
Parece que esta figura que tengo en brazos no ha vivido nunca; parece
que es una hermosa imagen de cera a quien he amado en sueños
representándomela con vida, con palabra y con movimiento. ¿La ves?
Siento que todos los habitantes de la ciudad estén muertos por esas
calles. Si vivieran, les llamaría para decirles que la he amado. ¿Por
qué lo oculté como un crimen? María, Mariquilla, esposa mía, ¿por qué te
has muerto sin heridas y sin enfermedad? ¿Qué tienes, qué te pasa; qué
te pasó en tu último momento? ¿En dónde estás ahora? ¿Tú piensas? ¿Te
acuerdas de mí y sabes acaso que existo? María, Mariquilla, ¿por qué
tengo yo ahora esto que llaman vida y tú no? ¿En dónde podré oírte,
hablarte y ponerme delante de ti para que me mires? Todo a oscuras está
en torno mío, desde que has cerrado los ojos. ¿Hasta cuándo durará esta
noche de mi alma y esta soledad en que me has dejado? La tierra me es
insoportable. La desesperación se apodera de mi alma, y en vano llamo a
Dios para que la llene toda. Dios no quiere venir, y desde que te has
ido, Mariquilla, el universo está vacío.

Diciendo esto, un vivo rumor de gente llegó a nuestros oídos.

---Son los franceses que toman posesión del Coso---dijo uno.

---Amigos, cavad pronto esa sepultura---exclamó Agustín, dirigiéndose a
los dos compañeros, que abrían un gran hoyo al pie del ciprés.---Si no,
vendrán los franceses y nos la quitarán.

Un hombre avanza por la calle de Antón Trillo, y deteniéndose junto a la
tapia destruida, mira hacia adentro. Le veo y tiemblo. Está
transfigurado, cadavérico, con los ojos hundidos, el paso inseguro, la
mirada sin brillo, el cuerpo encorvado, y me parece que han pasado
veinte años desde que no le veo. Su vestido es de harapos manchados de
sangre y lodo. En otro lugar y ocasión hubiéresele tomado por un mendigo
octogenario que venía a pedir una limosna. Acercose a donde estábamos, y
con voz tan débil que apenas se oía, dijo:

---¿Agustín, hijo mío, qué haces aquí?

---Señor padre---repuso el joven sin inmutarse,---estoy enterrando a
Mariquilla.

---¿Por qué haces eso? ¿Por qué tanta solicitud por una persona extraña?
El cuerpo de tu pobre hermano yace aún sin sepultura entre los demás
patriotas. ¿Por qué te has separado de tu madre y de tu hermana?

---Mi hermana está rodeada de personas amantes y piadosas que cuidarán
de ella, mientras esta no tiene a nadie más que a mí.

D. José de Montoria sombrío y meditabundo entonces cual nunca le vi, no
dijo nada, y empezó a echar tierra en el hoyo, en cuya profundidad ya
habían colocado el cuerpo de la hermosa joven.

---Echa tierra, hijo, echa tierra pronto---exclamó al fin,---pues todo
ha concluido. Han dejado entrar a los franceses en la ciudad cuando
todavía podía defenderse un par de meses más. Esta gente no tiene alma.
Ven conmigo y hablaremos de ti.

---Señor---repuso Agustín con voz entera,---los franceses están en la
ciudad, y las puertas han quedado libres. Son las diez: a las doce
saldré de Zaragoza, para ir al monasterio de Veruela donde pienso morir.

La guarnición, según lo estipulado, debía salir con los honores
militares por la puerta del Portillo. Yo estaba tan enfermo, tan
desfallecido a causa de la herida que recibí en los últimos días, y a
causa del hambre y cansancio, que mis compañeros tuvieron que llevarme
casi a cuestas. Apenas vi a los franceses, cuando con más tristeza que
júbilo se extendieron por lo que había sido ciudad.

Inmensas, espantosas ruínas la formaban. Era la ciudad de la desolación,
de la epopeya digna de que la llorara Jeremías y de que la cantara
Homero.

En la Muela, donde me detuve para reponerme, se me presentó D. Roque, el
cual salió también de la ciudad, temiendo ser perseguido por sospechoso.

---Gabriel---me dijo,---nunca creí que la canalla fuera tan vil, y yo
esperaba que en vista de la heroica defensa de la ciudad, serían más
humanos. Hace unos días vimos dos cuerpos que arrastraba el Ebro en su
corriente. Eran las dos víctimas de esa soldadesca furiosa, que manda
Lannes; eran mosén Santiago Sas, jefe de los valientes escopeteros de la
parroquia de San Pablo, y el padre Basilio Boggiero, maestro, amigo y
consejero de Palafox. Dicen que a ese último le fueron a llamar a media
noche, so color de encomendarle una misión importante, y luego que le
tuvieron entre las traidoras bayonetas, lleváronle al puente, donde le
acribillaron, arrojándole después al río. Lo mismo hicieron con Sas.

---¿Y nuestro protector y amigo D. José de Montoria no ha sido
maltratado?

---Gracias a los esfuerzos del presidente de la Audiencia ha quedado con
vida: pero me lo querían arcabucear\ldots{} nada menos. ¿Has visto
cafres semejantes? A Palafox parece que le llevan preso a Francia,
aunque prometieron respetar su persona. En fin, hijo, es una gente esa,
con la cual no me quisiera encontrar ni en el cielo. ¿Y qué me dices de
la hombrada del mariscalazo Sr.~Lannes? Se necesita frescura para hacer
lo que ha hecho. Pues nada más sino que mandó que le llevaran las
alhajas de la Virgen del Pilar, diciendo que en el templo no estaban
seguras. Luego que vio tal balumba de piedras preciosas, diamantes,
esmeraldas y rubíes, parece que le entraron por el ojo derecho\ldots{}
nada, hijo\ldots{} que se quedó con ellas. Para disimular esta rapiña,
ha hecho como que se las ha regalado la junta\ldots{} De veras te digo,
que siento no ser joven para pelear como tú en contra de ese ladrón de
caminos, y así se lo dije a Montoria cuando me despedí de él. ¡Pobre D.
José, qué triste está! Le doy pocos años de vida: la muerte de su hijo
mayor y la determinación de Agustín de hacerse cura, fraile o cenobita
le tienen muy abatido y en extremo melancólico.

D. Roque se detuvo para acompañarme, y luego partimos juntos. Después de
restablecido continué la campaña de 1809, tomando parte en otras
acciones, conociendo nueva gente y estableciendo amistades frescas o
renovando las antiguas. Más adelante referiré algunas cosas de aquel
año, así como lo que me contó Andresillo Marijuán, con quien tropecé en
Castilla, cuando yo volvía de Talavera y él de Gerona.

\flushright{Marzo-Abril de 1874.}

~

\bigskip
\bigskip
\begin{center}
\textsc{Fin de Zaragoza}
\end{center}

\end{document}
