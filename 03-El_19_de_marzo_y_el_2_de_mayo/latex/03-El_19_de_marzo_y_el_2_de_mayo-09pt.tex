\PassOptionsToPackage{unicode=true}{hyperref} % options for packages loaded elsewhere
\PassOptionsToPackage{hyphens}{url}
%
\documentclass[oneside,9pt,spanish,]{extbook} % cjns1989 - 27112019 - added the oneside option: so that the text jumps left & right when reading on a tablet/ereader
\usepackage{lmodern}
\usepackage{amssymb,amsmath}
\usepackage{ifxetex,ifluatex}
\usepackage{fixltx2e} % provides \textsubscript
\ifnum 0\ifxetex 1\fi\ifluatex 1\fi=0 % if pdftex
  \usepackage[T1]{fontenc}
  \usepackage[utf8]{inputenc}
  \usepackage{textcomp} % provides euro and other symbols
\else % if luatex or xelatex
  \usepackage{unicode-math}
  \defaultfontfeatures{Ligatures=TeX,Scale=MatchLowercase}
%   \setmainfont[]{EBGaramond-Regular}
    \setmainfont[Numbers={OldStyle,Proportional}]{EBGaramond-Regular}      % cjns1989 - 20191129 - old style numbers 
\fi
% use upquote if available, for straight quotes in verbatim environments
\IfFileExists{upquote.sty}{\usepackage{upquote}}{}
% use microtype if available
\IfFileExists{microtype.sty}{%
\usepackage[]{microtype}
\UseMicrotypeSet[protrusion]{basicmath} % disable protrusion for tt fonts
}{}
\usepackage{hyperref}
\hypersetup{
            pdftitle={EL 19 DE MARZO Y EL 2 DE MAYO},
            pdfauthor={Benito Pérez Galdós},
            pdfborder={0 0 0},
            breaklinks=true}
\urlstyle{same}  % don't use monospace font for urls
\usepackage[papersize={4.80 in, 6.40  in},left=.5 in,right=.5 in]{geometry}
\setlength{\emergencystretch}{3em}  % prevent overfull lines
\providecommand{\tightlist}{%
  \setlength{\itemsep}{0pt}\setlength{\parskip}{0pt}}
\setcounter{secnumdepth}{0}

% set default figure placement to htbp
\makeatletter
\def\fps@figure{htbp}
\makeatother

\usepackage{ragged2e}
\usepackage{epigraph}
\renewcommand{\textflush}{flushepinormal}

\usepackage{indentfirst}

\usepackage{fancyhdr}
\pagestyle{fancy}
\fancyhf{}
\fancyhead[R]{\thepage}
\renewcommand{\headrulewidth}{0pt}
\usepackage{quoting}
\usepackage{ragged2e}

\newlength\mylen
\settowidth\mylen{...................}

\usepackage{stackengine}
\usepackage{graphicx}
\def\asterism{\par\vspace{1em}{\centering\scalebox{.9}{%
  \stackon[-0.6pt]{\bfseries*~*}{\bfseries*}}\par}\vspace{.8em}\par}

 \usepackage{titlesec}
 \titleformat{\chapter}[display]
  {\normalfont\bfseries\filcenter}{}{0pt}{\Large}
 \titleformat{\section}[display]
  {\normalfont\bfseries\filcenter}{}{0pt}{\Large}
 \titleformat{\subsection}[display]
  {\normalfont\bfseries\filcenter}{}{0pt}{\Large}

\setcounter{secnumdepth}{1}
\ifnum 0\ifxetex 1\fi\ifluatex 1\fi=0 % if pdftex
  \usepackage[shorthands=off,main=spanish]{babel}
\else
  % load polyglossia as late as possible as it *could* call bidi if RTL lang (e.g. Hebrew or Arabic)
%   \usepackage{polyglossia}
%   \setmainlanguage[]{spanish}
%   \usepackage[french]{babel} % cjns1989 - 1.43 version of polyglossia on this system does not allow disabling the autospacing feature
\fi

\title{EL 19 DE MARZO Y EL 2 DE MAYO}
\author{Benito Pérez Galdós}
\date{}

\begin{document}
\maketitle

\hypertarget{i}{%
\chapter{I}\label{i}}

En Marzo de 1808, y cuando habían transcurrido cuatro meses desde que
empecé a trabajar en el oficio de cajista, ya componía con mediana
destreza, y ganaba tres reales por ciento de líneas en la imprenta del
\emph{Diario de Madrid}. No me parecía muy bien aplicada mi
laboriosidad, ni de gran porvenir la carrera tipográfica; pues aunque
toda ella estriba en el manejo de las letras, más tiene de embrutecedora
que de instructiva. Así es, que sin dejar el trabajo ni aflojar mi
persistente aplicación, buscaba con el pensamiento horizontes más
lejanos y esfera más honrosa que aquella de nuestra limitada, oscura y
sofocante imprenta.

Mi vida al principio era tan triste y tan uniforme como aquel oficio,
que en sus rudimentos esclaviza la inteligencia sin entretenerla; pero
cuando había adquirido alguna práctica en tan fastidiosa manipulación,
mi espíritu aprendió a quedarse libre, mientras las veinte y cinco
letras, escapándose por entre mis dedos, pasaban de la caja al molde.
Bastábame, pues, aquella libertad para soportar con paciencia la
esclavitud del sótano en que trabajábamos, el fastidio de la
composición, y las impertinencias de nuestro regente, un negro y tiznado
cíclope, más propio de una herrería que de una imprenta.

Necesito explicarme mejor. Yo pensaba en la huérfana Inés, y todos los
organismos de mi vida espiritual describían sus amplias órbitas
alrededor de la imagen de mi discreta amiga, como los mundos subalternos
que voltean sin cesar en torno del astro que es base del sistema. Cuando
mis compañeros de trabajo hablaban de sus amores o de sus trapicheos,
yo, necesitando comunicarme con alguien, les contaba todo sin hacerme de
rogar, diciéndoles:

---Mi amiga está en Aranjuez con su reverendo tío, el padre D. Celestino
Santos del Malvar, uno de los mejores latinos que ha echado Dios al
mundo. La infeliz Inés es huérfana y pobre; pero no por eso dejará de
ser mi mujer, con la ayuda de Dios, que hace grandes a los pequeños.
Tiene diez y sei años, es decir, uno menos que yo, y es tan linda, que
avergüenza con su carita a todas las rosas del Real Sitio. Pero, díganme
Vds., señores, ¿qué vale su hermosura comparada con su talento? Inés es
un asombro, es un portento; Inés vale más que todos los sabios, sin que
nadie la haya enseñado nada: todo lo saca de su cabeza, y todo lo
aprendió hace cientos de miles de años.

Cuando no me ocupaba en estas alabanzas, departía mentalmente con ella.
En tanto las letras pasaban por mi mano, trocándose de brutal y muda
materia en elocuente lenguaje escrito. ¡Cuánta animación en aquella masa
caótica! En la caja, cada signo parecía representar los elementos de la
creación, arrojados aquí y allí, antes de empezar la grande obra.
Poníalos yo en movimiento, y de aquellos pedazos de plomo surgían
sílabas, voces, ideas, juicios, frases, oraciones, períodos, párrafos,
capítulos, discursos, la palabra humana en toda su majestad; y después,
cuando el molde había hecho su papel mecánico, mis dedos lo
descomponían, distribuyendo las letras: cada cual se iba a su casilla,
como los simples que el químico guarda después de separados; los
caracteres perdían su sentido, es decir, su alma, y tornando a ser plomo
puro, caían mudos e insignificantes en la caja.

¡Aquellos pensamientos y este mecanismo todas las horas, todos los días,
semana tras semana, mes tras mes! Verdad es que las alegrías, el
inefable gozo de los domingos compensaban todas las tristezas y
angustiosas cavilaciones de los demás días. ¡Ah!, permitid a mi
ancianidad que se extasíe con tales recuerdos; permitid a esta negra
nube que se alboroce y se ilumine traspasada por un rayo de sol. Los
sábados eran para mí de una belleza incomparable: su luz me parecía más
clara, su ambiente más puro; y en tanto ¿quién podía dudar que los
rostros de las gentes eran más alegres, y el aspecto de la ciudad más
alegre también?

Pero la alegría no estaba sino en el alma. El sábado es el precursor del
domingo, y a eso del medio día comenzaban mis preparativos de viaje, de
aquel viaje al cielo, que mi imaginación renueva hoy, sesenta y cinco
años después. Aún me parece que estoy tratando con los trajineros de la
calle Angosta de San Bernardo sobre las condiciones del viaje: me ajusto
al fin y no puedo menos de disertar un buen rato con ellos acerca de las
probabilidades de que tengamos una hermosa noche para la expedición. En
seguida me lavo una, dos, tres, cuatro veces, hasta que desaparezcan de
mi cara y manos las últimas huellas de la aborrecida tinta, y me paseo
por Madrid esperando que llegue la noche. Duermo un poco; si la
inquietud me lo permite, y cuando el reló del Buen Suceso da las doce
campanadas más alegres que han retumbado en mi cerebro, me visto a toda
prisa con mi traje nuevo; corro al lado de aquellos buenos arrieros, que
son sin disputa los mejores hombres de la tierra, subo al carromato, y
ya estoy en viaje.

Con voluble atención observo todos los accidentes del camino, y mis
preguntas marean y enfadan a los conductores. Pasamos el puente de
Toledo, dejamos a derecha mano los caminos de Carabanchel y de Toledo,
el portazgo de las Delicias, el ventorrillo de León; las ventas de
Villaverde van quedando a nuestra espalda; dejamos a la derecha los
caminos de Getafe y de Parla, y en la venta de Pinto descansan un poco
las caballerías. Valdemoro nos ve pasar por su augusto recinto, y la
casa de Postas de Espartinas ofrece nuevo descanso a las perezosas
mulas. Por fin nos amanece bajando la cuesta de la Reina, desde donde la
vista abarca toda la extensión del inmenso valle en que se juntan Tajo y
Jarama; atravesamos el famoso puente largo, entramos más tarde en la
calle larga, y al fin ponemos el pie en la plaza del Real Sitio.

Mis miradas buscan entre los árboles y sobre las techumbres la modesta
torre de la iglesia. Corro allá. El Sr.~D. Celestino está en la misa,
que por ser día festivo es cantada. Desde la puerta oigo la voz del tío
de Inés, que exclama \emph{Gloria in excelsis Deo}. Yo también canto
\emph{gloria} en voz baja y entro en la iglesia. Una alegría solemne y
grave que da idea de la bienaventuranza eterna llena aquel recinto y se
reproduce en mi alma como en un espejo. Los vidrios incoloros permiten
que entre abundante luz y que se desparrame por la bóveda desnuda, sin
más pinturas que las del yeso mate. El altar mayor es todo oro, los
santos y retablos todo polvo; en el primero veo al santo varón, que se
vuelve hacia el pueblo y abre sus brazos; después consume, suenan las
campanillas dentro y las campanas fuera; se arrodillan todos,
golpeándose el pecho pecador. El oficio adelanta y concluye: durante él
he mirado sin cesar los grupos de mujeres sentadas en el suelo, y de
espaldas a mí: entre aquellos centenares de mantillas negras, distingo
la que cubre la hermosa cabeza de Inés: la conocería entre mil.

Inés se levanta cuando todo ha concluido, y sus ojos me buscan entre los
hombres, como los míos la buscan entre las mujeres. Por fin me ve, nos
vemos; pero no nos decimos una palabra. La ofrezco agua bendita, y
salimos. Parece que nuestras primeras palabras al vernos juntos han de
ser arrebatadas y vehementes; pero no decimos cosa alguna que no sea
insignificante. Nos reímos de todo.

La casa está a espalda de la iglesia, y entramos en ella cogidos de las
manos. Hay un patio con un ancho corredor, en cuyos gruesos pilares
retuerce sus brazos negros, ásperos y leñosos una vieja parra, junto a
un jazmín que aguarda la primavera para echar al mundo sus mil flores.
Subimos, y allí nos recibe D. Celestino, cuyo cuerpo no se cubre ya con
la sotana verdinegra de antaño, sino con otra flamante. Comemos juntos,
y luego los tres, Inés y yo delante, él detrás apoyándose en su bastón,
nos vamos a pasear al jardín del Príncipe, si hace buen tiempo y los
pisos están secos. Inés y yo charlamos con los ojos o con las palabras;
pero no quiero referir ahora nuestros poemas. A cada instante el padre
Celestino nos dice que no andemos tan aprisa, porque no puede seguirnos,
y nosotros, que desearíamos volar, detenemos el paso. Por último, nos
sentamos a orillas del río, y en el sitio en que el Tajo y el Jarama,
encontrándose de improviso, y cuando seguramente el uno no tenía
noticias de la existencia del otro, se abrazan y confunden sus aguas en
una sola corriente, haciendo de dos vidas una sola. Tan exacta imagen de
nosotros mismos, no puede menos de ocurrírsele a Inés al mismo tiempo
que a mí.

El día se va acabando, porque aunque a nuestros corazones les parezca lo
contrario, no hay razón ninguna para que se altere el sistema
planetario, dando a aquel día más horas que las que le corresponden.
Viene la tarde, el crepúsculo, la noche y yo me despido para volver a
mis galeras; estoy pensativo, hablo mil desatinos y a veces me parece
que me siento muy alegre, a veces muy triste. Regreso a Madrid por el
mismo camino, y vuelvo a mi posada. Es lunes, día que tiene un semblante
antipático, día de somnolencia, de malestar, de pereza y aburrimiento;
pero necesito volver al trabajo, y la caja me ofrece sus letras de
plomo, que no aguardan más que mis manos para juntarse y hablar; pero mi
mano no conoce en los primeros momentos sino cuatro de aquellos negros
signos que al punto se reúnen para formar este solo nombre: \emph{Inés}.

Siento un golpe en el hombro: es el cíclope o regente que me llama
holgazán, y me pone delante un papelejo manuscrito que debo componer al
instante. Es uno de aquellos interesantes y conmovedores anuncios del
\emph{Diario de Madrid}, que dicen: «\emph{Se necesita un joven de
diecisiete a dieciocho años, que sepa de cuentas, afeitar, algo de
peinar, aunque sólo sea de hombre, y guisar si se ofreciere. El que
tenga estas partes, y además buenos informes, puede dirigirse a la calle
de la Sal, número 5, frente a los peineros, lonja de lanería y pañolería
de D. Mauro Requejo, donde se tratará del salario y demás}.»

Al leer el nombre del tendero, un recuerdo viene a mi mente:---D. Mauro
Requejo ---digo.---Yo he oído este nombre en alguna parte.

\hypertarget{ii}{%
\chapter{II}\label{ii}}

He recordado días tan felices, y ahora me corresponde contar lo que me
pasó en uno de aquellos viajes. No se olvide que he empezado mi
narración en Marzo de 1808, y cuando yo había honrado el Real Sitio con
diez o doce de mis visitas. En el día a que me refiero, llegué cuando la
misa había concluido, y desde el portal de la casa un armonioso son de
flauta me anunció que D. Celestino estaba tan alegre como de costumbre,
señal de que nada desagradable ocurría en la modesta familia. Inés salió
a recibirme, y hechos los primeros cumplidos, me dijo:

---El tío Celestino ha recibido una carta de Madrid, que le ha puesto
muy alegre.

---¿De quién?---pregunté.

---No me lo ha dicho su merced, ni tampoco lo que la carta reza; pero él
está contento y\ldots{} dice que la carta trae muy buenas noticias para
mí.

---Eso es particular---añadí confundido.---¿Quién puede escribir desde
Madrid cartas que a ti te traigan buenas noticias?

---No sé; pero pronto saldremos de dudas---repuso Inés.---El tío me
dijo: «Cuando venga Gabriel y nos sentemos a la mesa, os contaré lo que
dice la carta. Es cosa que interesa a los tres: a ti principalmente,
porque eres la favorecida, a mí porque soy tu tío, y a él porque va a
ser tu novio cuando tenga edad para ello.»

No hablamos más del caso, y entré en el cuarto del buen sacerdote y
humanista. Una cama cubierta de blanquísima colcha pintada de verdes
ramos ocupaba el primer puesto en el reducido local. La mesa de pino con
dos o tres sillas que le servían de simétrica compañía, llenaba el
resto, y aún quedaba espacio para una cómoda estrambótica, con chapas y
remiendos de diversos palos y metales. Completaban tan modesto ajuar un
crucifijo y una virgen vestida de terciopelo, y acribillada de espadas y
rayos, ambas imágenes con sendos ramos de carrasca o de olivo clavados
en varios agujeritos que para el caso tenían las peanas. Los libros, que
eran muchos, no cubrían por el orden de su colocación más que media mesa
y media cómoda, dejando hueco para algunos papeles de música y otros en
que borrajeaba versos latinos el buen cura. Desde la ventana se veía un
huerto no mal cultivado, y a lo lejos las elevadas puntas de aquellos
olmos eminentes que guarnecen como hileras de gigantescos centinelas
todas las avenidas del Real Sitio. Tal era la habitación del padre
Celestino.

Sentámonos los tres, y el tío de Inés me dijo:

---Gabrielillo: tengo que leerte una poesía latina que he compuesto en
loor del serenísimo señor príncipe de la Paz, mi paisano, amigo y aun
creo que pariente. Me ha costado una semanita de trabajo; que componer
versos latinos no es soplar buñuelos. Verás, te la voy a leer, pues
aunque tú no eres hombre de letras, qué sé yo\ldots{} tienes un pícaro
gancho para comprender las cosas\ldots{} Luego pienso enviarla a Sánchez
Barbero, el primero de los poetas españoles desde que hay poesía en
España; y no me hablen a mí de fray Luis de León, de Rioja, de Herrera,
ni de todos esos que compusieron en romance. Fruslerías y juegos de
chicos. Un verso latino de Sánchez Barbero vale más que toda esa jerga
de epístolas, sonetos, silvas, églogas, canciones con que se emboba el
vulgo ignorante\ldots{} Pero vuelvo a lo que decía, y es que antes que
aquel fénix de los modernos ingenios la examine, quiero leértela a ti a
ver qué te parece.

---Pero, Sr.~D. Celestino, si yo no sé ni una palabra en latín, a no ser
\emph{Dominus vobiscum} y \emph{bóbilis bóbilis}.

---Eso no importa. Precisamente los profanos son los que mejor pueden
apreciar la armonía, la rimbombancia, el \emph{ore rotundo}, con que
tales versos deben escribirse ---dijo el clérigo con tenacidad
implacable.

Inés me dirigió una mirada en que me recomendaba, con su habitual
sabiduría, la abnegación y la paciencia para soportar al prójimo
impertinente. Ambos prestamos atención, y D. Celestino nos leyó unos
cuatrocientos versos, que sonaban en mi oído como una serie de
modulaciones sin sentido. Él parecía muy satisfecho, y a cada instante
interrumpía su lectura para decirnos:---¿Qué os parece ese pasajillo?
Inés: a esa figura llamamos \emph{litote}, y a este paloteo de las
palabras para imitar los ruidos del mar tempestuoso de la nación cuando
lo surca la nave del Estado se llama \emph{onomatopeya}, la cual figura
va encajada en otra que es la \emph{alegoría}.

Así nos fue leyendo toda la composición, de la cual figúrense Vds. lo
que entenderíamos. Aún conservo en mi poder la obra de nuestro amigo,
que empieza así:

\small
\newlength\mlena
\settowidth\mlena{Inserere aegrediar: per te Pax alma biformem}
\begin{center}
\parbox{\mlena}{\textit{  Te Godoie, canam pacis: tua munera caelo      \\
                        Inserere aegrediar: per te Pax alma biformem    \\
                        Vincla recusantem conduxit carcere Janum.}      \\
                        \null\dotfill}                                  \\  
\end{center}
\normalsize

Cuatrocientos versos por este estilo nos tragamos Inés y yo, siendo de
notar que ella atendía a la lectura con tanta formalidad como si la
comprendiera, y aun en los pasajes más ruidosos hacía señales de
asentimiento y elogio, para contentar al pobre viejo: ¡tal era su
discreción!

---Puesto que os ha agradado tanto, hijos míos---dijo D. Celestino
guardando su manuscrito,---otro día os leeré parte del poema. Lo dejo
para mejor ocasión, y así se comparte el placer entre varios días,
evitando el empacho que produce la sucesión de manjares demasiado dulces
y apetitosos.

---¿Y piensa Vd. leérsela también al príncipe de la Paz?

---¿Pues para qué la he escrito? A Su Alteza Serenísima le encantan los
versos latinos\ldots{} porque es un gran latino\ldots{} y pienso darle
un buen rato uno de estos días. Y a propósito, ¿qué se dice por Madrid?
Aquí está la gente bastante alarmada. ¿Pasa allá lo mismo?

---Allá no saben qué pensar. Figúrese Vd., la cosa no es para menos.
Temen a los franceses que están entrando en España a más y mejor. Dicen
que el rey no dio permiso para que entrara tanta gente, y parece que
Napoleón se burla de la corte de España, y no hace maldito caso de lo
que trató con ella.

---Es gente de pocos alcances la que tal dice---repuso D.
Celestino.---Ya saben Godoy y Bonaparte lo que se hacen. Aquí todos
quieren saber tanto como los que mandan, de modo que se oyen unos
disparates\ldots{}

---Lo de Portugal ha resultado muy distinto de lo que se creía. Un
general francés se plantó allá, y cuando la familia real se marchó para
América, dijo: «Aquí no manda nadie más que el Emperador, y yo en su
nombre; vengan cuatrocientos milloncitos de reales, vengan los bienes de
los nobles que se han ido al Brasil con la familia real.»

---No juzguemos por las apariencias---dijo D. Celestino;---sabe Dios lo
que habrá en eso.

---En España van a hacer lo mismo---añadí;---y como los Reyes están
llenos de miedo, y el príncipe de la Paz tan aturrullado, que no sabe
qué hacer\ldots{}

---¿Qué estás diciendo, tontuelo? ¿Cómo tratas con tan poco respeto a
ese espejo de los diplomáticos, a esa natilla de los ministros? ¿Que no
sabe lo que se hace?

---Lo dicho, dicho. Napoleón les engaña a todos. En Madrid hay muchos
que se alegran de ver entrar tanta tropa francesa, porque creen que
viene a poner en el trono al príncipe Fernando. ¡Buenos tontos están!

---¡Tontos, mentecatos, imbéciles!---exclamó con enfado el padre
Celestino.

---Lo que fuere sonará. Si vienen con buen fin esos caballeros, ¿por qué
se apoderan por sorpresa de las principales plazas y fortalezas? Primero
se metieron en Pamplona engañando a la guarnición; después se colaron en
Barcelona, donde hay un castillo muy grande que llaman el Monjuich.
Después fueron a otro castillo que hay en Figueras, el cual no es menos
grande, el mayor del mundo, según dice Pacorro Chinitas, y lo cogieron
también, y por último se han metido en San Sebastián. Digan lo que
quieran, esos hombres no vienen como amigos. El ejército español está
trinando: sobre todo, hay que oír a los oficiales que vienen del Norte y
han visto a los franceses en las plazas fuertes\ldots{} le digo a Vd.
que echan chispas. El gobierno del rey Carlos IV está que no le llega la
camisa al cuerpo, y todos conocen la barbaridad que han hecho dejando
entrar a los franceses; pero ya no tiene remedio\ldots{} ¿sabe Vd. lo
que se dice por Madrid?

---¿Qué, hijo mío? Sin duda alguna de esas vulgarísimas aberraciones
propias de entendimientos romos. Ya lo he dicho: nosotros no entendemos
de negocios de Estado; ¿a qué viene el comentar las combinaciones y
planes de esos hombres eminentes, que se desviven por hacernos felices?

---Pues allá dicen que la familia real de España, viéndose cogida en la
red por Bonaparte, ha determinado marcharse a América, y que no tardará
en salir de Aranjuez para Cádiz. Por supuesto, los partidarios del
príncipe Fernando se alegran, y creen que esto les viene de perillas
para que el otro suba al trono.

---¡Necios, mentecatos!---exclamó el tío de Inés, incomodándose de
nuevo.---¡Pensar que había de consentir tal cosa el señor príncipe de la
Paz, mi paisano, amigo y aun creo que pariente!\ldots{} Pero no nos
incomodemos fuera de tiempo, Gabriel, y por cosas que no hemos de
resolver nosotros. Vamos a comer, que ya es hora, y el cuerpo lo pide.

Inés, que se había retirado un momento antes, volvió a decirnos que la
comida estaba pronta. Durante ella, fue cuando el respetable cura nos
comunicó el contenido de la misteriosa carta que había llegado a la casa
por la mañana.

---Hijos míos---dijo cuando los tres habíamos tomado asiento:---Voy a
participaros un suceso feliz, y tú, Inesilla, regocíjate. La fortuna se
te entra por las puertas, y ahora vas a ver cómo Dios no abandona nunca
a los desvalidos y menesterosos. Ya sabes, que tu buena madre, que santa
gloria haya, tenía un primo llamado D. Mauro Requejo, comerciante en
telas, cuya lonja, si no me engaño, cae hacia la calle de Postas,
esquina a la de la Sal.

---D. Mauro Requejo\ldots---dije yo recordando,---justamente: doña Juana
le nombró delante de mí varias veces, y ahora caigo, en que ese
comerciante pone en el \emph{Diario} unos anuncios que me dan bastante
que hacer.

---Le recuerdo---dijo Inés.---Él y su hermana eran los únicos parientes
que tenía mi madre en Madrid. Por cierto que siempre se negó a
favorecernos, aunque lo necesitábamos bastante: dos veces le vi en casa.
¿Creería su merced que fue a consolarnos, a socorrernos? No: fue a que
mi madre le hiciera algunas piezas de ropa, y después de regatear el
precio, no pagó más que la mitad de lo tratado, y decía: «De algo ha de
servir el parentesco.» Él y su hermana no hablaban más que de su
honradez o de lo mucho que habían adelantado en el comercio y nos
echaban en cara nuestra pobreza, prohibiéndonos que fuéramos a su casa,
mientras no nos encontráramos en posición más desahogada.

---Pues digo---afirmé con enfado,---que ese don Mauro y su señora
hermana son dos grandísimos pillos.

---Poco a poco---continuó el cura.---Déjenme acabar. El primo de tu
madre habrá faltado; pero lo que es ahora, sin duda Dios le ha tocado en
el corazón, y se dispone a enmendar sus yerros, favoreciéndote como buen
pariente y hombre caritativo. Ya sabes que es bastante rico, gracias a
su laboriosidad y mucha economía. Pues bien: en la carta que he recibido
esta mañana me dice que quiere recogerte y ampararte en su casa, donde
estarás como una reina; donde no te faltará nada, ni aun aquello de que
gustan tanto las damiselas del día, tal como joyas, trajes bonitos,
perfumes primorosos, guantes y otras fruslerías. En fin, Dios se ha
acordado de ti, sobrinita. ¡Ah!, ¡si vieras qué interés tan grande
demuestra por ti en sus cartas; qué alabanzas tan calurosas hace de tus
méritos; si vieras cómo te pone por esas nubes, cómo lamenta tu
orfandad, y cómo se enternece considerando que eres de su misma sangre,
y que a pesar de esta natural preeminencia careces de lo que a él le
sobra! Te repito que trabajando mucho y ahorrando más, el Sr.~Requejo ha
llegado a ser muy rico. ¡Qué porvenir te espera, Inesilla! El párrafo
más conmovedor de la carta de tus tíos---añadió sacando la epístola---es
este:\emph{¿A quién hemos de dejar lo que tenemos, sino a nuestra
querida sobrinita?}

Inés, confundida ante tan inesperado cambio en los sentimientos y en la
conducta de sus antes cruelísimos parientes, no sabía qué pensar. Me
miró, buscando sin duda en mis ojos algo que la diera luz sobre tan
inexplicable mudanza; mas yo, que algo creía comprender, me guardé muy
bien de dejarlo traslucir ni con palabras ni con gestos.---Estoy
asombrada---dijo la muchacha---y por fuerza para que mis tíos me quieran
tanto ha de haber algún motivo que no comprendemos.

---No hay más sino que Dios les ha abierto los ojos---dijo D. Celestino,
firme en su ingenuo optimismo.---¿Por qué hemos de pensar mal de todas
las cosas? D. Mauro es un hombre honrado; podrá tener sus defectillos;
pero ¿qué valen esos ligeros celajes del alma, cuando está iluminada por
los resplandores de la caridad?

Inés mirándome parecía decirme:

---¿Y tú qué piensas?

Algunos meses antes de aquel suceso, yo hubiera acogido las
proposiciones de D. Mauro Requejo con el imprevisor optimismo, con el
necio entusiasmo que afluían de mi alma juvenil ante los acontecimientos
nuevos e inesperados; pero las contrariedades me habían dado alguna
experiencia; conocía ya los rudimentos de la ciencia del corazón, y el
mío principiaba a reunir ese tesoro de desconfianzas, merced a las
cuales medimos los pasos peligrosos de la vida. Así es que respondí
sencillamente:

---Puesto que ese tu reverendo tío era antes un bribón, no sé por qué
hemos de creerle santo ahora.

---Tú eres un chicuelo sin experiencia---me dijo D. Celestino algo
enojado,---y yo no debiera consultar esto contigo. ¡Si sabré yo
distinguir lo verdadero de lo falso! Y sobre todo, Inés, si él quiere
favorecerte, poniéndote en pie de gente grande, si él quiere gastarse
sus ahorros con su querida sobrina, ¿por qué no lo has de aceptar? Mucho
más podría decirte; pero él mismo en persona te explicará mejor el gran
cariño que te tiene.

---¿Pues qué---preguntó Inés turbada,---vendrá a Aranjuez?

---Sí, chiquilla---repuso el clérigo.---Yo te reservaba esta noticia
para lo último. Hoy mismo tendrás el gusto de ver aquí a tu amado tío y
protector. ¡Ah, Inés! Mucho sentiré separarme de ti; pero servirame de
consuelo la idea de que estás contenta, de que disfrutas mil comodidades
que yo no te puedo dar. Y cuando este viejo incapaz eche un paseíto a
Madrid para visitarte; espero que le recibirás con alegría y sin
orgullo: espero que no te ofuscará la ruin vanidad al considerarte en
posición superior a la mía, porque tío por tío, hermano soy de tu
difunto padre, mientras que el otro\ldots{}

D. Celestino estaba conmovido, y yo también, aunque por distinta causa.

---Sí---continuó el cura.---Hoy tendremos aquí a ese eminente tendero de
la calle de la Sal. Me dice que habiendo comprado unas tierras en
Aranjuez, junto a la laguna de Ontígola, viene hoy aquí con el doble
objeto de conocer su finca y de verte. Él espera que irás a Madrid en su
compañía y en la de su hermana doña Restituta, a quien también tendremos
el gusto de ver esta tarde, pues si han salido, como dice la carta, hoy
de madrugada, por poco que avancen, ya deben estar pasando el puente
largo.

Después de oír esto, todos callamos. Revolviendo en mi cabeza extraños y
no muy alegres pensamientos, dije a Inés:

---Pero ese hombre, ¿es casado?

Ella leyó en mi interior con su intuición incomparable, y me respondió
con viveza:

---Es viudo.

Después volvimos a callar, y sólo D. Celestino, tarareando una antífona,
interrumpía nuestro grave silencio.

\hypertarget{iii}{%
\chapter{III}\label{iii}}

Tristísimo sobre toda ponderación me volví á Madrid, y pasé toda la
semana meditabundo y como alelado, deseando y temiendo que el domingo
siguiente llegase, porque de un lado la curiosidad y de otro el temor
solicitaban mi espíritu. Tan grande era mi sobresalto en la noche del
sábado, que no pegué los ojos, y de madrugada me fui al mesón de la
calle de la Aduana á buscar un acomodo en cualquier galera que partiese
para el Real Sitio. Mi escasez de numerario me puso en peligro de no
poder ir, lo que me desesperaba y afligía extraordinariamente.

Pero con ruegos y razones sutilísimas, unidas al poco dinero que tenía,
logré ablandar el corazón duro de un carromatero, que al fia consintió
en llevarme. Las tres mulas emplearon no sé si un siglo en el viaje. Yo
temía que se me adelantaran los tíos de Inés; pero no fué así. Cuando
llegué, D. Celestino estaba en la misa mayor: entró en la iglesia lo
mismo que los domingos anteriores; pero el templo me pareció triste y
fúnebre. Al salir di agua bendita a Inés, esperamos al buen párroco en
la puerta de la sacristía, y nos fuimos los tres a la casa.
\textbar Cosa singular! No hablamos nada por el camino. Los tres
suspirábamos. Durante la comida traté de animar á los demás con fingido
buen humor; pero no pude conseguirlo. Viendo la tardanza de la anunciada
visita, yo creí que los Requejos no vendrían; pero mi alegría se disipó
cuando estábamos concluyendo de comer. De improviso sentimos ruido de
voces en el patio de la casa; levantámonos, y saliendo yo al corredor,
oí una voz hueca y áspera que decía: «¿Vive aquí el latino y músico D.
Celestino Santos del Malvar, cura de la parroquia?»

D. Mauro Requejo y su hermana doña Restituta, tíos de Inés, habían
llegado.

Entraron en la habitación donde estábamos, y al punto que D. Mauro vio a
su sobrina dirigiose a ella con los brazos abiertos, y al estrecharla en
ellos, exclamó endulzando la voz:

---¡Inés de mi alma, inocente hija de mi prima Juana! Al fin, al fin te
veo. Bendito sea Dios que me ha dado este consuelo. ¡Qué linda eres!
Ven, déjame que te abrace otra vez.

Doña Restituta hizo lo mismo, pero exagerando hasta lo sumo el mohín
lacrimoso de su rostro, así como la apretura de sus abrazos, y luego que
ambos hubieron desahogado sus amantes corazones, saludaron a D.
Celestino, quien no pudo menos de derramar algunas lágrimas al ver tal
explosión de sensibilidad. Por mi parte de buena gana habría
correspondido con bofetones a los abrazos con que estrujaban a Inés
aquellos gansos, cuya descripción no puedo menos de considerar ahora
como indispensable.

D. Mauro Requejo era un hombre izquierdo. Creo que no necesito decir
más. ¿No habéis entendido? Pues lo explicaré mejor. ¿Ha sido la
naturaleza o es la costumbre quien ha dispuesto que una mitad del cuerpo
humano se distinga por su habilidad y la otra mitad por su torpeza? Una
de nuestras manos es inepta para la escritura, y en los trabajos
mecánicos sólo sirve para ayudar a su experta compañera, la derecha.
Esta hace todo lo importante; en el piano ejecuta la melodía, en el
violín lleva el arco, que es la expresión, en la esgrima maneja la
espada, en la náutica el timón, en la pintura el pincel: es la que
abofetea en las disputas; la que hace la señal de la cruz en el rezo y
la que castiga el pecho en la penitencia. Iguales disposiciones tiene el
pie derecho; si algo eminente y extraordinario ha de hacerse en el
baile, es indudable que lo hará el pie derecho; él es también el que
salta en la fuga, el que golpea la tierra con ira en la desesperación,
el que ahuyenta al perro atrevido, el que aplasta al sucio reptil, el
que sirve de ariete para atacar a un despreciable enemigo que no merece
ser herido por delante. Esta superioridad mecánica, muscular y nerviosa
de las extremidades derechas se extiende a todo el organismo: cuando
estamos perplejos sin saber qué dirección tomar, si el cuerpo se
abandona a su instinto, se inclinará hacia la derecha, y los ojos
buscarán la derecha como un oriente desconocido. Al mismo tiempo en el
lado siniestro todo es torpeza, todo subordinación, todo ineptitud:
cuanto hace por sí resulta torcido, y su inferioridad es tan notoria,
que ni aun en desarrollo puede igualar al otro lado. La mitad de todo
hombre es generalmente más pequeña que la otra: para equilibrarlas, sin
duda, se dispuso que el corazón ocupara el costado izquierdo.

Hemos hecho tan fastidiosa digresión para que se comprenda lo que
dijimos de D. Mauro Requejo. Los dos lados de aquel hombre eran dos
lados izquierdos, es decir, que todo él era torpe, inepto, vacilante,
inhábil, pesado, brusco, embarazoso. No sé si me explico. Parecía que le
estorbaban sus propias manos: al verle mirar de un lado para otro,
creeríase que buscaba un rincón donde arrojar aquellos miembros
inútiles, cubiertos con guantes sin medida, que quitaban la sensibilidad
a los oprimidos dedos, hasta el punto de que su dueño no los conocía por
suyos.

Habíase sentado en el borde de la silla y sus piernas pequeñas y
rígidas, no eran los miembros que reposan con compostura: extendíase a
un lado y otro como las dos muletas que un cojo arrima junto a sí. Ya no
le servían para nada, sino para arrastrar de aquí para allí los pesados
pies. Al quitarse el sombrero, dejándolo en el suelo, al limpiarse el
sudor con un luengo pañuelo de cuadros encarnados y azules, parecía el
mozo de cuerda que se descarga de un gran fardo. La buena ropa que
vestía no era adorno de su cuerpo, pues él no estaba vestido con ella,
sino ella puesta en él. En cuanto a los guantes, embruteciéndole las
manos, se las convertían en pies. A cada instante se tocaba los dijes
del reló y los encajes de las chorreras para cerciorarse de que no se le
habían caído; pero como tras la gamuza había desaparecido el tacto,
necesitaba emplear la vista, y esto le hacía semejante a un mono que al
despertar una mañana se encontrase vestido de pies a cabeza.

Su inquietud era extraordinaria, como la de un cuerpo mortificado por
infinito número de picazones, y cada pliegue del traje debía hacer llaga
en sus sensibles carnes. A veces aquella inerte manopla de ante amarillo
rellena de dedos tiesos e insensibles, partía en dirección del sobaco o
de la cintura con la ansiosa rapidez de una mano que va a rascar; pero
se contenía subiendo a acariciar la barba recién afeitada. También movía
con frecuencia el cuello, como si algún bicho extraño agarrado a su
occipucio juguetease en el pescuezo entre el pelo y la solapa. Era el
coleto encebado que irreverentemente se metía entre piel y camisa, o
escarbaba la oreja. La mano de ante amarillo se alzaba también en
aquella dirección; pero también se detenía pasando a frotar la rodilla.

La cara de D. Mauro Requejo era redonda como una muestra de reloj: no
estaba en su sitio la nariz, que se inclinaba del un hemisferio buscando
el carrillo siniestro que por obra y gracia de cierto lobanillo era más
luminoso que su compañero. Los ojos verdosos y bien puestos bajo cejas
negras y un poco achinescadas, tenían el brillo de la astucia, mientras
que su boca, insignificante si no la afearan los dos o tres dientes
carcomidos que alguna vez se asomaban por entre los labios, tenía todos
los repulgos y mohínes que el palurdo marrullero estudia para engañar a
sus semejantes. La risa de D. Mauro Requejo era repentina y sonora: en
la generalidad de las personas este fenómeno fisiológico empieza y acaba
gradualmente, porque acompaña a estados particulares del espíritu, el
cual no funciona, que sepamos, con la rigurosa precisión de una máquina.
Muy al contrario de esto, nuestro personaje tenía, sin duda, en su
organismo un resorte para la risa, de la cual pasaba a la seriedad tan
bruscamente como si un dedo misterioso se quitara de la tecla de lo
alegre para oprimir la de lo grave. Yo creo que él en su interior
pensaba así, «ahora conviene reír;» y reía.

\hypertarget{iv}{%
\chapter{IV}\label{iv}}

Era imposible decir si doña Restituta sería más joven o más vieja que su
hermano: ambos parecían haber pasado bastante más allá de los cuarenta
años, pero si en la edad se asemejaban, no así en la cara ni el gesto,
pues Restituta era una mujer que no se estorbaba a sí misma y que sabía
estarse quieta. Había en ella si no fineza de modales, esa holgada
soltura, propia de quien ha hablado con gente por mucho tiempo.
Comparando a aquellas dos ramas humanas de un mismo tronco, se decía:
«Mauro ha estado toda la vida cargando fardos, y Restituta midiendo y
vendiendo; el uno es un sabandijo de almacén y la otra la bestiezuela
enredadora de la tienda.»

Alta y flaca, con esa tez impasible y uniforme que parece un forro, de
manos largas y feas, a quien el continuo escurrirse por entre telas
había dado cierta flexibilidad; de pelo escaso, y tan lustrosamente
aplastado sobre el casco, que más parecía pintura que cabello; con su
nariz encarnadita y algo granulenta, aunque jamás fue amiga de oler lo
de Arganda; la boca plegada y de rincones caídos, la barba un poco
velluda, y un mirar así entre tarde y noche, como de ojos que miran y no
miran. Restituta Requejo era una persona cuyo aspecto no predisponía a
primera vista ni en contra ni en favor. Oyéndola hablar, tratándola, se
advertía en ella no sé qué de escurridizo, que se escapaba a la
observación, y se caía en la cuenta de que era preciso tratarla por
mucho tiempo para poder hacer presa con dedos muy diestros en la piel
húmeda de su carácter, que para esconderse poseía la presteza del saurio
y la flexibilidad del ofidio. Pero dejemos estas consideraciones para su
lugar, y por ahora, conténtense Vds. con oír hablar a los tíos de Inés.

---\emph{Éste} estaba tan impaciente por venir---dijo Restituta,
señalando a su hermano---que con la prisa nos fue imposible traer alguna
cosita como hubiéramos deseado.

D. Celestino les dio las gracias con su amable sonrisa.

---Tenía tanta impaciencia por venir a ver esas tierras---dijo D.
Mauro,---que\ldots{} y al mismo tiempo el alma se me arrancaba en
cuajarones al pensar en mi querida sobrinita, huérfana y
abandonada\ldots{} porque las tierras, Sr.~D. Celestino, no son ningún
muladar, Sr.~D. Celestino, y me han costado obra de trescientos cuarenta
y ocho reales, trece maravedíes, sin contar las diligencias ni el por
qué de la escritura. Sí señor; ya está pagado todo, peseta sobre peseta.

---Todo pagado---indicó doña Restituta mirando uno tras otro a los tres
que estábamos presentes.---A \emph{éste} no le gusta deber nada.

---¡Quiten para allá! Antes me dejo ahorcar que deber un
maravedí,---exclamó D. Mauro, llevando la manopla a la garganta,
oprimida por el corbatín.

---En casa no ha habido nunca trampas---añadió la hermana.

---A eso deben Vds. el haber adelantado tanto---dijo D. Celestino.

---La suerte\ldots{} eso sí: hemos tenido suerte---dijo
Requejo.---Luego, \emph{ésta} es tan trabajadora, tan ahorrativa, tan
hormiguita\ldots{}

---Pero todo se debe a tu honradez---añadió Restituta.---Sí, créanlo
Vds., a su honradez. \emph{Éste} tiene tal fama entre los comerciantes,
que le entregarían los tesoros del rey.

---En fin\ldots{} algo se ha hecho, gracias a Dios y a nuestro trabajo.
Si fuera a hacer caso de \emph{ésta}, compraría tierras y más tierras. A
\emph{ésta} no le gustan sino las fincas.

---Y con razón: si \emph{éste} me hiciera caso---dijo la hermana,
mirando otra vez sucesivamente a los circunstantes,---todas nuestras
ganancias se emplearían en tierras de labor.

---Como yo soy así tan\ldots{} pues---indicó Requejo.

---Sin soberbia, Sr.~D. Celestino---dijo Restituta,---bueno es aparentar
que se tiene lo que se tiene.

---Y me hace comprar vestidos, sombreros, alhajas---indicó D.
Mauro.---Qué sé yo la tremolina de cosas que ha entrado en casa. Ello,
como se puede\ldots{} Vea Vd. esta cadena---añadió mostrando a D.
Celestino una que traía al cuello;---vea Vd. también este alfiler.
¿Cuánto cree Vd. que me han costado? La friolerita de mil reales\ldots{}
Ps: yo no quería; pero \emph{ésta} se empeñó, y como se puede\ldots{}

---Son hermosas piezas.

---Y bien te dije que te quedaras también con la \emph{tumbaga} de la
esmeralda, que ya recordarás la daban en poco más de nada. Es una
lástima que la haya tomado el duque de Altamira.

Al decir esto nos miraban, y nosotros les contestábamos con señales de
asentimiento, pero sin palabras, porque ni a Inés ni a mí se nos
ocurrían.

---Pero, ¿cómo está ahí mi sobrina tan calladita?---dijo Requejo
riéndose de improviso, y quedándose muy serio un instante después.

Inés se sonrojó y no dijo nada, porque en efecto no tenía nada que
decir.

---¡Ay, no puede negar la pinta! ¡Cómo se parece a su madre, a la pobre
Juana, mi prima querida!---exclamó Requejo llevándose la manopla a la
boca para tapar un bostezo.---¡Y qué pronto se murió la pobrecita!

---Ya que pasó a mejor vida aquella santa y ejemplar mujer---dijo
Restituta,---no la nombremos, porque así se renueva nuestro dolor y el
de esa pobre muchacha, aunque ella es niña, y los niños se consuelan más
fácilmente.

Inés no dijo nada tampoco; pero el color encendido de su rostro se trocó
en intensa palidez. Creyó conveniente el cura variar la conversación, y
dijo:

---¿Y ha visto Vd. esas tierras de la laguna de Ontígola?

---Todavía no---respondió Requejo;---pero me han dicho que son
magníficas. Ps\ldots{} para mí, poca cosa. \emph{Ésta} se empeñó en que
me quedara con ellas y al fin me decidí. Allá en el país tenemos muchas
más, que hemos ido comprando poco a poco.

---En su país de Vd. hacia el Bierzo, si no me engaño.

---Más acá del Bierzo, en Santiagomillas, que es tierra de Maragatería.
De allí \emph{semos} todos, y allí está todavía el solar de los
Requejos.

---Familia hidalga, según creo---afirmó el cura.

---Ello\ldots{} no deja de tener uno su \emph{motu propio}---contestó D.
Mauro;---y según nos decía un sabio escribano de mi pueblo, nuestros
ascendientes tenían un gran quejigar, de donde les vino el nombre de
Requejo.

---Así debe de ser; los más ilustres apellidos traen su origen de alguna
yerba o legumbre. Y si no, ahí están en la Roma antigua los
\emph{Léntulos}, los \emph{Fabios} y los \emph{Pisones} que se llamaban
así porque alguno de sus mayores cultivó las lentejas, las habas y los
guisantes. En cuanto a mí, creo que este nombre de \emph{Malvar} me
viene de que algún abuelo mío se pintaba solo para el cultivo de las
malvas.

---Pues yo creo---dijo D. Mauro volviendo a reír,---que eso de que la
nobleza viene de las guerras y de las hazañas de algunos caballeros es
pura mentira. Que no me vengan a mí con bolas: yo no creo que haya
habido nunca esas heroicidades. No hay más sino que los reyes hicieron
duque a uno porque tenía un huerto de coles, y a otro marqués porque
sabía escoger melones. De todos modos, nuestra familia no viene de
ningún cardo borriquero.

---Y venga de donde viniere---dijo doña Restituta,---lo principal es lo
principal. Lo que es en nuestra casa, Sr.~D. Celestino, no falta nada en
gracia de Dios, y aunque por fuera no gastamos lujo, ni nos gusta andar
en carroza, ni figurar, lo que es la gallina en el puchero todos los
días\ldots{} eso sí: este y yo no nos podemos pasar sin ciertas
comodidades.

---Lo que es por mí---interrumpió Requejo,---con cualquier cosa me
sustento. Teniendo un pedazo de pan, otro de tocino, y agua de la fuente
del Berro, vamos viviendo; pero \emph{ésta} se empeña en poner las cosas
en buen pie. Todos los días ha de traer libra y media de carne de vaca,
y jamón rancio a morrillo, y abadejo del mejor todos los viernes, y para
cenar una perdiz por barba, y los domingos tres capones, y por Navidad y
por el día de San Mauro, que es el 15 de Enero, o por San Restituto, que
es el 10 de Junio, andan los pavos por casa como si esta fuese la era
del Mico. El mayordomo de los duques de Medina de Rioseco, que suele ir
a casa a pedirnos dinero prestado, se queda estupefacto de ver tanta
abundancia y dice que no ha visto despensa como la nuestra.

---Eso sí---dijo Restituta,---no nos duele gastar en el plato, ni en
buena ropa para vestir, ni en buen cisco de retama para la lumbre.
Vivimos tranquilos y felices: nuestra única pena ha consistido hasta
ahora en no tener una persona querida a quien dejar lo que poseemos,
cuando Dios se sirva llamarnos a su santa gloria; porque los parientes
que nos quedan en Santiagomillas son unos pícaros que nos dan mucho que
hacer.

Al oír esto, D. Mauro movió el resorte de risa, y miró a Inés, diciendo:

---Pero aquí nos depara Dios a nuestra querida sobrinita, a esta rosa
temprana, a esta señoritica que parece un ángel: ¡ay!, si no puede negar
la pinta, si es \emph{éntica} a su madre\ldots{}

---Por Dios, Mauro---exclamó Restituta,---no traigas a la memoria a
aquella santa mujer, porque yo estoy todavía tan impresionada con su
muerte, que si la recuerdo, se me vienen las lágrimas a los ojos.

---Todo sea por Dios, y hágase su santa voluntad---dijo Requejo tocando
el resorte de la seriedad.---Lo que digo es que cuanto tengo y pueda
tener será para esta palomita torcaz, pues todo se lo merece ella con su
cara de princesa.

---Ya, ya\ldots---indicó Restituta guiñando el ojo,---que no tendrá
pretendientes en gracia de Dios. Marquesitos y condesitos conozco yo que
no suspirarán poco debajo de nuestras balcones cuando sepan que
guardamos en casa tal primor.

---Pelambrones, hija, pelambrones sin un cuarto---añadió
Requejo.---Cuando la niña haya de tomar estado, ya le buscaremos un
joven de una de las principales familias de España, que sea digno de
llevarse esta joya.

---Eso por de contado. Casas hay muy ricas, donde no es todo apariencia,
y mayorazgos conozco que en cuanto la vean y sepan la riqueza que ha de
heredar de sus tíos, beberán los vientos por conseguir su mano. A fe mía
que nuestra casa no es ningún guiñapo, y cuando pongamos en la sala las
cortinas de sarga verde con ramos amarillos, y aquellos pájaros color de
pensamiento que parecen vivos, no estará de mal ver para recibir en ella
a todos los señores del Consejo Real. Pues poco tono se va a dar la
niñita en su gran casa.

D. Celestino viendo que su sobrina no contestaba nada a tan patéticas
demostraciones de afecto, creyó conveniente hablar así:

---Ella les agradece a Vds. con toda el alma los beneficios que va a
recibir.

---Ya estoy contento, Sr.~D. Celestino---dijo Requejo.---Una cosa me
faltaba y ya la tengo. Inés será mi heredera, Inés se casará con una
persona que la merezca, y que traiga también buenas peluconas: ella será
feliz y nosotros también.

---No hables mucho de eso, porque lloro---dijo doña Restituta.---¡Qué
gusto es tener quien la acompañe a una en la soledad, y quien comparta
las comodidades que Dios y nuestro trabajo nos han proporcionado! ¡Ay!,
Inesita: eres tan linda, que me recuerdas mi mocedad cuando iba a jugar
a la huerta del convento de las madres Recoletas de Sahagún, donde me
crié. Me parece que si ahora te separaran de mí, no tendría fuerzas para
vivir.

Diciendo esto abrazó a Inés, y pareciome que el forro de su cara, es
decir, la piel se teñía de un leve rosicler.

---Como Inés está impaciente por irse con nosotros---dijo
Requejo,---esta misma tarde nos la llevaremos.---¡Cómo!, ¡esta tarde!,
¡yo!---exclamó ella vivamente.

---Hija mía---dijo Restituta,---no conviene disimular el cariño que nos
tienes. Somos tus tíos, y de veras te digo que no debes agradecernos lo
que hacemos por ti, pues obligación nuestra es.

---Tal vez ponga reparos a ir con Vds. así\ldots{} tan pronto dijo con
timidez D. Celestino,---pero no dudo que comprenda pronto las ventajas
de su nueva posición, y se decida\ldots{}

---¡Que no quiere venir!---exclamó Requejo con asombro.---Con que
nuestra sobrina no nos quiere\ldots{} ¡Jesús! ¡Mayor desgracia!

---Sí\ldots{} les quiere a Vds.---añadió el cura tratando de conciliar
la repugnancia que notaba en el semblante de Inés con el deseo de los
Requejos.

---Hermano, no sabes lo que te dices---afirmó Restituta.---Nuestra
sobrina es un dechado de modestia, de ingenuidad y de sencillez. Quieres
que se ponga ahora a hacer aspavientos en medio de la sala, saltando y
brincando de gusto porque nos la llevamos. Eso no estaría bien. Por el
contrario---prosiguió la hermana de D. Mauro,---se está muy calladita, y
como muchacha honesta y bien criada\ldots{} ¡ya se ve!, como hija de
aquella santa mujer\ldots{} disimula su alborozo y se está así mano
sobre mano, bendiciendo mentalmente a Dios por la suerte que le depara.

---Entonces, Sr.~D. Celestino---dijo Requejo,---nosotros nos vamos ahora
a ver esas tierras de Ontígola que están ahí hacia la parte de Titulcia,
y por la tarde cuando volvamos, Inés estará preparada para venirse con
nosotros a Madrid.

---No tengo inconveniente, si ella está conforme---repuso el clérigo,
mirando a su sobrina.

Mas no dieron tiempo a que esta expresara su opinión sobre aquel viaje,
porque los Requejos se levantaron para marcharse, diciendo que un coche
de dos mulas les esperaba en el paradero del Rincón. Abrazaron por turno
dos o tres veces a su sobrina, hicieron ridículas cortesías a D.
Celestino, y sin dignarse mirarme, lo cual me honró mucho, salieron,
dejando al clérigo muy complacido, a Inés absorta, y a mí furioso.

\hypertarget{v}{%
\chapter{V}\label{v}}

Al punto se trató de resolver en consejo de familia lo que debía
hacerse; pero deseando yo conferenciar con el buen cura para decirle lo
que Inés no debía oír, rogué a esta que nos dejase solos y hablamos así:

---¿Será Vd. capaz, Sr.~D. Celestino, de consentir que Inés vaya a vivir
con ese ganso de D. Mauro, y la lechuza de su hermana?---Hijo---me
contestó,---Requejo es muy rico, Requejo puede dar a Inesilla las
comodidades que yo no tengo, Requejo puede hacerla su heredera cuando
estire la zanca.

---¿Y Vd. lo cree? Parece mentira que tenga Vd. más de sesenta años.
Pues yo digo y repito que ese endiablado D. Mauro me parece un farsante
hipocritón. Yo en lugar de Vd., les mandaría a paseo.

---Yo soy pobre, hijo mío; ellos son ricos, Inés se irá con ellos. En
caso de que la traten mal la recogeremos otra vez.

---No la tratarán mal, no---dije muy sofocado.---Lo que yo temo es otra
cosa, y eso no lo he de consentir.

---A ver, muchacho.

---Usted sabe como yo lo que hay sobre el particular; Vd. sabe que Inés
no es hija de doña Juana; Vd. sabe que Inés nació del vientre de una
gran señora de la corte, cuyo nombre no conocemos, Vd. sabe todo esto, y
¿cómo sabiéndolo no comprende la intención de los Requejos?

---¿Qué intención?

---Los Requejos despreciaron siempre a doña Juana; los Requejos no le
dieron nunca ni tanto así; los Requejos ni siquiera la visitaron en su
enfermedad, y ahora, Sr.~D. Celestino de mi alma, los Requejos lloran
recordando a la difunta, los Requejos echan la baba mirando a su
sobrinita, y no puede ser otra cosa sino que los Requejos han
descubierto quiénes son los padres de Inés, los Requejos han comprendido
que la muchacha es un tesoro, y ¡ay!, no me queda duda de que el Requejo
mayor, ese poste vestido trae entre ceja y ceja el proyecto de casarse
con Inés, obligándola a ello en cuanto la pille en su casa.

---Sosiégate, muchacho, y óyeme. Puede muy bien suceder que la intención
de los Requejos sea la que dices, y puede muy bien que sea la que ellos
han manifestado. Como yo me inclino siempre a creer lo bueno, no dudo de
la sinceridad de D. Mauro, hasta que los hechos me prueben lo contrario.
¿Qué sabes tú si de la mañana a la noche verás a Inés hecha una
damisela, con carroza y pajes, llena de diamantes como avellanas, y
viviendo en uno de esos caserones que hay en Madrid más grandes que
conventos?

---¡Bah, bah! Eso es como cuando yo quería ser príncipe, generalísimo y
secretario del despacho. A los diez y seis años se pueden decir tales
cosas; pero no a los sesenta.

---Viviendo conmigo, Inés ha de estar condenada a perpetua estrechez.
¿No vale más que se la lleven los parientes de su madre, que parecen
personas muy caritativas? En todo caso, Gabriel, si la muchacha no
estuviera contenta allí, tiempo tenemos de recogerla, porque a mí, como
tío carnal, me corresponde la tutela.---¿Y por qué la deja Vd. marchar?

---Porque los Requejos son ricos\ldots{} ¿lo comprenderás al
fin?\ldots{} porque Inés en casa de esa gente puede estar como una
princesa, y casarse al fin con un comerciante muy rico de la calle de
Postas o Platerías.

---Alto allá, señor mío---exclamé muy amostazado,---¿qué es eso de
casarse Inés? Inés, Dios mediante, no se casará más que conmigo. Sí
¡vaya Vd. a hablarle de comerciantes y de usías!

---Es verdad, no me acordaba, hijito---dijo el cura con algo de
mofa.---¡Casarse a los diez y seis años! ¿El matrimonio es algún juego?
Y además: hazme el favor de decirme qué ganas tú en la imprenta donde
trabajas.

---Sobre tres reales diarios.

---Es decir, noventa y tres reales los meses de treinta y uno. Algo es,
pero no basta, chiquillo. Ya ves tú: cuando Inés esté en su sala con
cortinas verdes de ramos amarillos y se siente en aquellas mesas donde
hay siete pavos por Navidad, y todas las noches cena de perdiz por
barba\ldots{} ya ves tú, no sé cómo podrá arrimarse a ella un
pretendiente con noventa y tres reales al mes, en los que traen treinta
y uno.

---Eso ella es quien lo ha de decir---repuse con la mayor zozobra;---y
si ella me quiere así, veremos si todos los Requejos del mundo lo pueden
impedir. En resumidas cuentas, Sr.~D. Celestino, ¿Vd. está decidido a
que Inés se vaya esta tarde con don Mauro!

---Decidido, hijo, es para mí un caso de conciencia.

---¿Y quién le dice a Vd. que con noventa y tres reales al mes no se
puede mantener una familia? Pues a mí me da la gana de casarme, sí
señor.

---¡Casarse a los diez y seis años! Uno y otro debéis esperar a tener
los treinta y cinco cumplidos. La vida se pasa pronto: no te apures.
Para entonces podréis casaros. Sois a propósito el uno para el otro.
Casar y compadrar, cada uno con su igual. Veremos si de aquí allá te
luce más el oficio.

---¿Y no puedo yo buscar un destinillo?

---Eso es como cuando se te puso en la cabeza que te iba a caer un
principado o un ducado.

---No: un destinillo de estos que se dan a cualquier pelón, en la
contaduría de acá o en la de allá.

---¿Pero crees tú que un empleo es cosa fácil de conseguir?

---¿Por qué no?---respondí enfáticamente.---¿Pues para qué son los
destinos sino para darlos a todos los españoles que necesitan de ellos?

---Hijo, las antesalas están llenas de pretendientes. Ya recordarás que
a pesar de ser paisano y amigo del príncipe de la Paz, estuve catorce
años haciendo memoriales.---Y al fin\ldots{} pero hoy visita Vd. a S. A.
y le trata; de modo que si le pidiera para mí una placita no creo que se
la negara.

---¡Ah!---exclamó D. Celestino con satisfacción.---El día que visité a
S. A. fue para mí el más lisonjero de mi vida, porque oí de sus augustos
labios las palabras más cariñosas. Si vieras con cuánto agasajo me
trató; ¡y qué amabilidad, qué dulzura, qué llaneza sin dejar por eso de
ser príncipe en todos sus gestos y palabras! Cuando entré, yo estaba
todo turbado y confuso, y la lengua se me quedó pegada al paladar.
Mandome S. A. que me sentara, y me preguntó si yo era de Villanueva de
la Serena. ¿Ves qué bondad? Contestele que había nacido en los Santos de
Maimona, villa que está en el camino real como vamos de Badajoz a Fuente
de Cantos. Luego me preguntó por la cosecha de este año, y le respondí
que según mis noticias, el centeno y cebada eran malos, pero que la
bellota venía muy bien. Ya comprenderás por esto el interés que se toma
por la agricultura. En seguida me dijo si estaba contento en mi
parroquia, a lo cual contesté afirmativamente, añadiendo que me tenía
edificada la piedad de mis feligreses; al decir esto no pude contener
las lágrimas. Bien claro se ve que al príncipe le interesa mucho cuanto
se refiere a la religión. Hablele después de que entretenía mis ocios
con la poesía latina, y notifiquele haber compuesto un poema en
hexámetros, dedicado a él. Enterado de esto, dijo que \emph{bueno}, en
lo cual se demuestra palmariamente su desmedida afición a las letras
humanas; y por fin, a los diez minutos de conferencia, me rogó
afectuosamente que me retirara, porque tenía que despachar asuntos
urgentísimos. Esto prueba que es hombre trabajador, y que las mejores
horas del día las consagra puntualmente a la administración. Te aseguro
que salí de allí conmovido.

---¿Y no vuelve Vd.?

---¡Pues no he de volver! Supliqué a S. A. que me fijara día para
llevarle el poema latino, y mañana tendré el honor de poner de nuevo los
pies en el palacio de mi ilustre paisano.

---Pues yo iré con Vd. Sr.~D. Celestino---dije con mucha
determinación.---Iremos juntos y Vd. le pedirá un destino para mí.

---¡Estás loco!---exclamó el sacerdote con asombro.---No me creo capaz
de semejante irreverencia.

---Pues se lo pediré yo---dije más resuelto cada vez a entrar en la
administración.

---Modera esos arrebatos, joven sin experiencia. ¿Cómo quieres que te
presente sin más ni más al príncipe de la Paz? ¿Qué puedo decir de ti,
cuáles son tus méritos? ¿Conoces acaso por el forro los versos latinos?
¿Has saludado siquiera el \emph{Divitias alius fulvo sibi congerat
auro}, el \emph{Passer, delitiæ meæ puellæ}, o el \emph{Cynthia prima
suis me cepis ocellis?} ¿Estás loco, piensas que los destinos están ahí
para los mocosos a quienes se les antoja pedirlos?

---Vd. le dice que soy un joven pariente suyo, y yo me encargo de lo
demás.

---¿Pariente mío? Eso sería una mentira, y yo no miento.

Así disputamos un buen rato, y al fin, entre ruegos y razones logré
convencer al padre Celestino para que me llevara a presencia del
serenísimo señor Godoy. Mi tenaz proyecto se explica por el estado de
desesperación en que me puso la visita de los Requejos, y su propósito
de cargar con la pobre Inés. La viva antipatía que ambos hermanos me
inspiraron desde que tuve la desdicha de poner los ojos sobre ellos,
engendró en mi espíritu terribles presentimientos. Se me representaba la
pobre huérfana en dolorosa esclavitud bajo aquel par de trasgos,
condenada a perecer de tristeza si Dios no me deparaba medios para
sacarla de allí. ¿Cómo podía yo conseguirlo, siendo como era, más pobre
que las ratas? Pensando en esto, vino a mi mente una idea salvadora, la
que desde aquellos tiempos principiaba a ser norte de la mitad, de la
mayor parte de los españoles, es decir, de todos aquellos que no eran
mayorazgos ni se sentían inclinados al claustro; la idea de adquirir una
plaza en la administración. ¡Ay!, aunque había entonces menos destinos,
no eran escasos los pretendientes. España había gastado en la guerra con
Inglaterra, la espantosa suma de siete mil millones de reales. Quien
esto derrochó en una calaverada, ¿no podía darme a mí cinco mil para que
me casara? Por supuesto, el pretender casarse entonces a los diez y
siete años, era una calaverada peor que la de gastar siete mil millones
en una guerra. Aquella idea echó raíces en mi cerebro con mucha
presteza. A la media hora de mi conferencia con D. Celestino, ya se me
figuraba estar desempeñando ante la mesa forrada de bayeta verde, las
funciones que el Estado tuviera a bien encomendarme para su prosperidad
y salvación. Atrevido era el proyecto de pedir yo mismo al poderoso
ministro lo que me hacía falta: pero la gravedad de las circunstancias,
y el loco deseo de adquirir una posición que me permitiera disputar la
posesión de Inés a la temerosa pareja de los Requejos, disminuía los
obstáculos ante mis ojos, dándome aliento para las empresas más
difíciles.

La huérfana no disimuló al hablar conmigo la repugnancia que le
inspiraban sus tíos: tal vez hubiera yo logrado impedir el secuestro;
pero D. Celestino repitió que era para él caso de conciencia, y con esto
Inés no se atrevió a formular sus quejas, ¡tan grande era entonces la
subordinación a la autoridad de los mayores! La escrupulosidad del buen
sacerdote no impidió, sin embargo, que yo hablara mil pestes de los dos
hermanos, criticando sus fachas y vestidos, y comentando a mi manera
aquello de los siete pavos y capones, con la añadidura de las perdices
por barba en la hora de la cena. También me reí con implacable saña de
los tratamientos que se daban hermano y hermana, pues, según el lector
observaría, se llamaban simplemente \emph{éste} y \emph{ésta}. D.
Celestino me dijo al oírme, que tratase con más miramientos a dos
personas respetables que habían sabido labrar pingüe fortuna con su
trabajo y honradez, y entre tanto Inés preparaba de muy mala gana su
equipaje para marchar a la corte.

No tardó la casa del cura en verse honrada de nuevo con las personas de
los Requejos, que llegaron a eso de las cuatro, haciendo mil
ponderaciones de las tierras adquiridas cerca de Ontígola; y su contento
al ver que Inés se disponía a seguirles, fue extraordinario.

---No te des prisa, pimpollita---decía D. Mauro,---que todavía hay
tiempo de sobra.

---Su impaciencia por emprender el viaje---añadió doña Restituta,
plegando de un modo indefinible el forro cutáneo de su cara,---es tan
viva, que la pobrecilla quisiera tener alitas para salir más pronto de
aquí.

---Eso no---dijo D. Celestino algo amoscado;---que su tío no le ha dado
malos tratos, para que así se impaciente por abandonarle.

Inés se arrojó llorando a los brazos del cura, y ambos derramaron muchas
lágrimas. Por mi parte, tenía interés en que los Requejos no conocieran
que un antiguo y cordial amor me unía a Inés, así es que disimulé mi
sofocación, y acechándola fuera, cuando salió en busca de un objeto
olvidado, le dije:

---Prendita, no me digas una palabra, ni me mires, ni me saludes. Yo me
quedo aquí, pero descuida; pronto nos hemos de ver allá.

Llegó por fin la hora de la partida; el coche se acercó a la puerta de
la casa. Inés entró en él muy llorosa y los Requejos tomaron asiento a
un lado y otro, pues aun en aquella situación temían que se les
escapara. Jamás he visto mujer ninguna que se asemejara a un cernícalo
como en aquel momento doña Restituta. El coche partió, y al poco rato
nuestros ojos le vieron perderse entre la arboleda. Don Celestino, que
hacía esfuerzos por aparentar gran serenidad, no pudo conservarla, y
haciendo pucheros como un niño, sacó su largo pañuelo y se lo llevó a
los ojos.

---¡Ay, Gabriel! ¡Se la llevaron!

Mi emoción también era intensísima, y no pude contestarle nada.

\hypertarget{vi}{%
\chapter{VI}\label{vi}}

Al día siguiente me llevó D. Celestino al palacio del Príncipe de la
Paz. Era el 15 de Marzo, si no me falla la memoria.

Aunque no tenía ropa para mudarme en tan solemne ocasión, como la que
llevaba a Aranjuez era la mejorcita, con una camisa limpia que me prestó
el cura, quedé en disposición, según él mismo me dijo, de presentarme
aunque fuera a Napoleón Bonaparte. Por el camino, y mientras hacíamos
tiempo hasta que llegara la hora de las audiencias, D. Celestino sacaba
del bolsillo interior de su sotana el poema latino para leerlo en alta
voz, porque,---Quizás el señor Príncipe---decía---me mande leer algún
trozo, y conviene hacerlo con entonación clásica y ritmo seguro,
mayormente si hay delante algún embajador o general extranjero.

Después, guardando el manuscrito, añadió con cierta zozobra:

---¿Sabes que el sacristán de la parroquia, ese condenado
Santurrias\ldots{} ya le conoces\ldots{} me ha puesto esta mañana la
cabeza como un farol? Dice que el señor Príncipe de la Paz no dura dos
días más al frente de la nación, y que le van a cortar la cabeza. Esto
no merece más que desprecio, Gabrielillo; pero me da rabia de oír tratar
así a persona tan respetable. Pues, ¿qué crees tú? he descubierto que
ese pícaro Santurrias es jacobino, y se junta mucho con los cocheros del
infante D. Antonio Pascual, los cuales son gente muy alborotada.

---¿Y qué dice ese reverendo sacristán?

---Mil necedades; figúrate tú. Como si a personas de estudios y que
tienen en la uña del dedo a todos los clásicos latinos, se les pudiera
hacer tragar ciertas bolas. Dice que el señor príncipe de la Paz,
temiendo que Napoleón viene a destronar a nuestros queridos reyes, tiene
el propósito de que éstos marchen a Andalucía para embarcarse y dar la
vela a las Américas.

---Pues anoche---dije yo,---cuando fui al mesón a decir a los arrieros
que no me aguardaran, oí decir lo mismito a unos que estaban allí, y por
cierto que hablaban de su amigo y paisano de Vd. con más desprecio que
si fuera un bodegonero del Rastro.

---No saben lo que se pescan, hijo---me dijo el cura.---Pero o yo me
engaño mucho o los partidarios del príncipe de Asturias andan metiendo
cizaña por ahí. Ello es que en Aranjuez hay mucha gente extraña
y\ldots{} quiera Dios. Ya me dijo esta mañana Santurrias que su mayor
gusto será tocar las campanas a vuelo si el pueblo se amotina para pedir
alguna cosa; pero ya le he dicho---y al hablar así D. Celestino se paró,
y con su dedo índice hacía demostraciones de la mayor energía,---ya le
he dicho que si toca las campanas de la Iglesia sin mi permiso, lo
pondré en conocimiento del señor Patriarca para lo que este tenga a bien
resolver.

Con esta conversación llegó la hora, y nosotros al palacio de S. A.
Atravesamos por entre varios guardias que custodiaban la puerta, porque
ha de saberse que el generalísimo tenía su guardia de a pie y de a
caballo, lo mismo que el rey, y mejor equipada, según observaban los
curiosos.

Nadie nos puso obstáculo en el portal ni en la escalera; pero al llegar
a un gran vestíbulo en cuyo pavimento taconeaban con estrépito las botas
de otra porción de guardias, uno de estos nos detuvo, preguntando a D.
Celestino con cierta impertinencia que a dónde íbamos.

---Su Alteza---dijo el clérigo muy turbado,---tuvo el honor de
señalarme\ldots{} digo\ldots{} yo tuve el honor de que él señalara el
día de hoy y la presente hora para recibirme.

---Su Alteza está en palacio. Ignoramos cuándo vendrá---dijo el guardia
dando media vuelta.

D. Celestino me consultó con sus ojos y también iba a consultarme con
sus autorizados labios, cuando se sintió ruido en el portal.

---¡Ahí está! Su Alteza ha llegado---dijeron los guardias, tomando
apresuradamente sus armas y sombreros para hacer los honores. Pero el
Príncipe subió a sus habitaciones particulares por la escalera excusada,
que al efecto existía en su palacio.

---Quizás Su Alteza no reciba hoy---dijo a don Celestino el guardia, que
poco antes nos había detenido.---Sin embargo, pueden Vds. esperar si
gustan, y él avisará si da audiencia o no.

Dicho esto, nos hizo pasar a una habitación contigua y muy grande donde
vimos a otras muchas personas, que desde por la mañana habían acudido en
solicitud del favor de una entrevista con S. A. Entre aquella gente
había algunas damas muy distinguidas, militares, señores a la antigua,
vestidos con históricas casacas y cubiertos con antiquísimas pelucas, y
también algunas personas humildes.

Los pretendientes allí reunidos se miraban con recelo y mal humor,
porque a todo el que hace antesala molesta mucho el verse acompañado,
considerando sin duda que si el tiempo y la benevolencia del ministro se
reparten entre muchos, no puede tocarles gran cosa. Un ujier se acercó a
nosotros y preguntó a D. Celestino quiénes éramos, a lo cual repuso el
buen eclesiástico:

---Nosotros somos curas de la parroquia de\ldots{} quiero decir, soy
cura de la parroquia y este joven\ldots{} este joven gana noventa y tres
reales en los meses de treinta y uno; y venimos a\ldots{} pero yo no
pienso pedirle nada al señor Príncipe, porque este picarón (señalando a
mí) no se morderá la lengua para decirle lo que desea.

Cuando el ujier se alejó, dije a mi acompañante que tuviera cuidado de
no equivocarse tan a menudo: que no anunciara anticipadamente nuestra
comisión pedigüeña, y que no había necesidad de ir pregonando lo que yo
ganaba, a lo que me respondió que él como persona nueva en antesalas y
palacios, se turbaba a la primera ocasión, diciendo mil desatinos. Uno
de los señores que aguardaban se nos acercó, y reconociendo al cura, se
saludaron ambos muy cortésmente, diciendo el desconocido:

---Sr.~D. Celestino, ¿qué bueno por aquí?

---Vengo a visitar a S. A. Ya sabe Vd. que somos paisanos y amigos. Mi
padre y su abuelo hicieron un viaje juntos desde Trujillo a la Vera de
Plasencia, y un tío de mi madre tenía en Miajadas una dehesa donde los
Godoyes iban a cazar alguna vez. Somos amigos, y le estoy muy
reconocido, porque a la munificencia de S. A. debo el beneficio que
disfruto, el cual me fue concedido en cuanto S. A. tuvo conocimiento de
mi necesidad; así es que desde mi primer memorial hasta el día en que
tomé posesión, sólo transcurrieron catorce años.

---Se conoce que el Príncipe quiso servirle a usted---dijo nuestro
interlocutor---No a todos se les despacha tan pronto. Hace veintidós
años que yo pretendí que se me repusiera en mi antigua plaza de la
colecturía del Noveno y del Excusado, y esta es la hora, Sr.~D.
Celestino. A pesar de todo, yo no me desanimo, y menos ahora, porque
tengo por seguro que la semana que viene\ldots{}

---No todos son tan afortunados como yo---dijo el optimista D.
Celestino.---Verdad es que como paisano y amigo de S. A. estoy en
situación muy favorable. De mi pueblo a Badajoz, cuna de D. Manuel
Godoy, no hay más que trece leguas y media por buen camino, y estoy
cansado de ver la casa en que nació este faro de las Españas. Así es que
en cuanto supo mi necesidad\ldots{}

---Pero diga Vd.---preguntó bajando la voz el señor de la \emph{semana
que viene}---¿tenemos viaje de los reyes a Andalucía o no tenemos viaje?

---¿Pero Vd. cree tales paparruchas?---dijo don Celestino.---Esa voz la
ha corrido Santurrias, el sacristán de mi iglesia. Ya le dicho que si
tocaba las campanas sin mi permiso\ldots{}

---Todo el mundo lo asegura. Ya sabe Vd. que ha venido mucha tropa de
Madrid, y por las calles del pueblo se ve gente de malos modos.

---¿Pero qué objeto puede tener ese viaje?

---Amigo: ya Napoleón tiene en España la friolera de cien mil hombres.
Ha nombrado general en jefe a Murat, el cual dicen que salió ya de
Aranda para Somosierra. Y a todas estas ¿hay alguien que sepa a qué
viene esa gente? ¿Vienen a echar a toda la familia real? ¿Vienen
simplemente de paso para Portugal?

---¿Quién se asusta de semejante cosa?---dijo D. Celestino.---Pongamos
por caso que vengan con mala intención. ¿Qué son cien mil hombres? Con
dos o tres regimientos de los nuestros se podrá dar buena cuenta de
ellos, y ahí nos las den todas. Como Su Alteza se calce las
espuelas\ldots{} Eso del viaje es pura invención de los desocupados y de
los enemigos de Su Alteza, que le insultan porque no les ha dado
destinos. Como si los destinos se pudieran dar a todo el que los
pretende.

No siguió esta conversación, porque el ujier se acercó a nosotros,
haciéndonos señas de que le siguiéramos. Su Alteza nos mandaba pasar.
Cuando los demás pretendientes vieron que se daba la preferencia a los
que habían llegado los últimos, un murmullo de descontento resonó en la
sala. Nosotros la atravesamos muy orgullosos de aquella predilección y
mientras D. Celestino saludaba a un lado y otro con su bondad de
costumbre, yo dirigí a los más cercanos una mirada de desprecio, que
equivalía al convencimiento de mi próximo ingreso en la administración
de ambos mundos.

Pasamos de aquella sala a otras, todas ricamente alhajadas. ¡Qué bellos
tapices, qué lindos cuadros, qué hermosas estatuas de mármol y bronce,
qué vasos tan elegantes, qué candelabros tan vistosos, qué muebles tan
finos, qué cortinajes tan espléndidos, qué alfombras tan muelles! No
pude detenerme en la contemplación de tan bonitos objetos porque el
ujier nos llevaba a toda prisa, y yo me sentía atacado de una cortedad
tal, que se disipó mi anterior envalentonamiento, y empecé a comprender
que me faltarían ideas y saliva para expresar ante el príncipe mi
pensamiento. Por fin llegamos al despacho de Godoy, y al entrar vi a
este en pie, inclinado junto a una mesa y revisando algunos papeles.
Aguardamos un buen rato a que se dignase mirarnos y al fin nos miró.

Godoy no era un hombre hermoso, como generalmente se cree; pero sí
extremadamente simpático. Lo primero en que se fijaba el observador era
en su nariz, la cual, un poco grande y respingada, le daba cierta
expresión de franqueza y comunicatividad. Aparentaba tener sobre
cuarenta años: su cabeza rectamente conformada y airosa, sus ojos vivos,
sus finos modales, y la gallardía de su cuerpo, que más bien era pequeño
que grande, le hacían agradable a la vista. Tenía sin duda la figura de
un señor noble y generoso; tal vez su corazón se inclinaba también a lo
grande; pero en su cabeza estaba el desvanecimiento, la torpeza, los
extravíos y falsas ideas de los hombres y las cosas de su tiempo.

Nos miró, como he dicho, y al punto D. Celestino, que temblaba como un
chiquillo de diez años, hizo una profunda cortesía, a la cual siguió
otra hecha por mi persona. A mi acompañante se le cayó el sombrero;
recogiolo, dio algunos pasos, y con voz tartamuda dijo así:

---Ya que Vuestra Alteza tiene el honor de\ldots{} no\ldots{}
digo\ldots{} ya que yo tengo el honor de ser recibido por Vuestra Alteza
serenísima\ldots{} decía que me felicito de que la salud de Vuestra
Alteza sea buena, para que por mil años sigamos haciendo el bien de la
nación\ldots{}

El príncipe parecía muy preocupado, y no contestó al saludo sino con una
ligera inclinación de cabeza. Después pareció recordar, y dijo:

---Es Vd. el señor chantre de la catedral de Astorga, que viene
a\ldots{}

---Permítame Vuestra Alteza---interrumpió D. Celestino,---que ponga en
su conocimiento cómo soy el cura de la parroquia castrense de Aranjuez.

---¡Ah!---exclamó el príncipe,---ya recuerdo\ldots{} el otro día\ldots{}
se le dio a Vd. el curato por recomendación de la señora condesa de X
(Amaranta). Es usted natural de Villanueva de la Serena.

---No señor: soy de los Santos de Maimona. ¿No recuerda Vuestra Alteza
esa villa? En el camino de Fuente de Cantos. Allí se cogen unas sandías
que pesan muchas arrobas, y también hay muchos melones\ldots{} Pues,
como decía a Vuestra Alteza, hoy venía con dos objetos: con el de tener
el honor de presentarme a Vuestra Alteza, para que este chico lea un
poema latino que ha compuesto\ldots{} no, quiero decir\ldots{}

D. Celestino se atragantó, mientras que el Príncipe, asombrado de mi
precocidad en el estudio de los clásicos, me miraba con ojos benévolos.

---No---dijo el cura entrando de nuevo en posesión de su lengua.---El
poema ha sido compuesto por mí, y, accediendo a los deseos de V. A. voy
a comenzar su lectura.

El Príncipe adelantó la mano con ese instintivo movimiento que parece
apartar un objeto invisible. Pero D. Celestino no comprendió que su
protector rechazaba por medio de un movimiento físico la amenazadora
lectura del poema, y firme en su propósito, desenvainó el manuscrito
homicida. En el mismo instante Godoy, que atendía poco a nosotros, y
parecía estar pensando cosas muy graves, volviose bruscamente hacia la
mesa y empezó a hojear de nuevo los papeles.

D. Celestino me miró y yo miré a D. Celestino.

Así transcurrió un minuto al cabo del cual el Príncipe dirigiose hacia
nosotros y dijo señalando unas sillas:

---Siéntense Vds.

Después siguió en su investigación de papeles. Sentados en nuestros
asientos el cura y yo nos hablábamos en voz baja.---Para exponerle tu
pretensión---me dijo el tío de Inés,---debes esperar a que yo lea mi
poema, en lo cual con la pausa conveniente no tardaré más que hora y
media. El admirable efecto que le ha de producir la audición de los
versos clásicos a que es tan aficionado, le predispondrá en tu favor, y
no dudo que te concederá cuanto le pidas.

Después de otro rato de espera, un oficial entró para dar un despacho al
Príncipe. Este le abrió al punto, y después que lo hubo leído con mucha
ansiedad, dejolo sobre la mesa y se dirigió hacia don Celestino.

---Dispénseme Vd.---dijo,---mi distracción. Hoy es día para mí de
ocupaciones graves e inesperadas. No pensaba recibir a nadie en
audiencia, y si le mandé entrar a Vd. fue porque sabía no es de los que
vienen a pedirme destinos.

D. Celestino se inclinó en señal de asentimiento, y yo dije para mí:
«Lucidos hemos quedado.» Después dirigiose S. A. a mí, y me dijo:

---En cuanto al poema latino que este joven ha compuesto, ya tengo
noticias de que es una obra notable. Persista Vd. en su aplicación a los
buenos estudios y será un hombre de provecho. No puedo hoy tener el
gusto de conocer el poema; pero ya me habían hablado de Vd. con grandes
encomios y desde luego formé propósito de que se le diera a Vd. una
plaza en la oficina de Interpretación de Lenguas, donde su precocidad
sería de gran provecho. Sírvase usted dejarme su nombre\ldots{}

D. Celestino iba a contestar rectificando el error; pero su turbación se
lo impidió. Antes que mi compañero pudiera decir una palabra, levanteme
yo, y extendiendo mi nombre sobre un papel que en la mesa encontré,
ofrecilo respetuosamente al Príncipe, que concluyó así:

---Ruego a Vds. que tengan la bondad de retirarse, pues mis ocupaciones
no me permiten prolongar esta audiencia.

Hicimos nuevas cortesías, D. Celestino balbuceó las fórmulas pomposas
propias del caso, y salimos del despacho del Príncipe. Al pasar por la
sala donde esperaban con impaciencia los demás pretendientes, el ujier
lanzó esta terrorífica exclamación:---«¡No hay audiencia!»

Al encontrarse en la calle, el buen cura, recobrando la serenidad de su
espíritu y la soltura de su lengua, me dijo con cierto enojo:

---¿Por qué no le dijiste tú que el poema no era tuyo sino mío?

No pude menos de soltar la risa, viéndole picado en su amor propio, y
considerando el extraño resultado de nuestra visita al príncipe de la
Paz.

\hypertarget{vii}{%
\chapter{VII}\label{vii}}

---Pues, Gabrielillo---me dijo D. Celestino cuando entrábamos en la
casa,---cierto es que hay demasiada gente en el pueblo. Se ven por ahí
muchas caras extrañas, y también parece que es mayor el número de
soldados. ¿Ves aquel grupo que hay junto a la esquina? Parecen
trajineros de la Mancha\ldots{} y entre ellos se ven algunos uniformes
de caballería. Por este lado vienen otros que parecen estar
bebidos\ldots{} ¿oyes los gritos? Entrémonos, hijo mío, no nos digan
alguna palabrota. Aborrezco el vulgo.

En efecto, por las calles del Real Sitio, y por la plaza de San Antonio
discurrían más o menos tumultuosamente varios grupos, cuyo aspecto no
tenía nada de tranquilizador. Asomábase a las ventanas el vecindario
todo, para observar a los transeúntes, y era opinión general, que nunca
se había visto en Aranjuez tanta gente. Entramos en la casa, subimos al
cuarto de D. Celestino, y cuando este sacudía el polvo de su manteo y
alisaba con la manga las rebeldes felpas del sombrero de teja, la puerta
se entreabrió, y una cara enjuta, arrugada y morena, con ojos vivarachos
y tunantes, una cara de esas que son viejas y parecen jóvenes, o al
contrario, cara a la cual daba peculiar carácter toda la boca necesaria
para contener dos filas de descomunales dientes, apareció en el hueco.
Era Gorito Santurrias, sacristán de la parroquia.

---¿Se puede entrar, señor cura?---preguntó, sonriendo, con aquella
jovialidad mixta de bufón y de demonio que era su rasgo sobresaliente.

---A tiempo viene el Sr.~Santurrias---dijo el cura frunciendo el
ceño,---porque tengo que prevenirle\ldots{} Sepa Vd. que estoy
incomodado, sí señor; y pues los sagrados cánones me autorizan para
imponerle castigo\ldots{} allá veremos\ldots{} y digo y repito que la
gente que se ve por ahí no viene a lo que Vd. me indicó esta mañana.
Pues no faltaba más.

---Señor cura---contestó irrespetuosamente Santurrias,---esta noche me
desollará las manos la cuerda de la campana grande. Es preciso tocar,
tocar para reunir la gente.

---¡Ay de Santurrias si suenan las campanas sin mi permiso!\ldots{} Pero
¿qué quiere esa gentuza? ¿Qué pretende?

---Eso lo veremos luego.

---Ande Vd. con Barrabás, diablo de siete colas. ¿Pero a qué viene esa
gente a Aranjuez?---repitió D. Celestino dirigiéndose a mí.---Gabriel,
se nos olvidó advertir al señor príncipe de la Paz lo que pasa, y
aconsejarle que no esté desprevenido. ¡Cuánto nos hubiese agradecido Su
Alteza nuestro solícito interés!---Ya se lo dirán de misas---murmuró
burlonamente Santurrias.---Lo que quiere esa gente es impedir que nos
lleven para las Indias a nuestros idolatrados Reyes.

---¡Ja, ja!---exclamó el sacerdote poniéndose amarillo.---Ya salimos con
la muletilla. Como si uno no tuviera autoridad para desmentir tales
rumores; como si uno no fuera amigo de personas que le enteran de lo que
pasa; como si uno no estuviera al tanto de todo.

Diciendo esto, D. Celestino no quitaba de mí los ojos, buscando sin duda
una discreta conformidad con sus afirmaciones. En tanto Santurrias, que
era uno de los sacristanes más tunos y desvergonzados que he visto en mi
vida, no cesaba de burlarse de su superior jerárquico, bien
contradiciéndole en cuanto decía, bien cantando con diabólica música una
irreverente ensaladilla compuesta de trozos de sainete mezclados con
versículos latinos del Oficio ordinario.

---¡Ay señor cura, señor cura!---dijo.---Si veremos correr a su
paternidad por el camino de Madrid con los hábitos arremangados. ¡Ja,
ja, ja!

\small
\newlength\mlenb
\settowidth\mlenb{  Préstame tu moquero}
\begin{center}
\parbox{\mlenb}{  Préstame tu moquero         \\
                si está más limpio,           \\
                para echar los tostones       \\
                que me has pedido.}           \\
\end{center}
\normalsize

\small
\newlength\mlenc
\settowidth\mlenc{Asperges me, Domine, hissopo, et mundabor.}
\begin{center}
\parbox{\mlenc}{\textit{Asperges me, Domine, hissopo, et mundabor.}}
\end{center}
\normalsize

---Mi dignidad---repuso el clérigo cada vez más amostazado,---no me
permite rebajarme hasta disputar con el Sr.~de Santurrias. Si yo no le
tratara de igual, como acostumbro, no se habría relajado la disciplina
eclesiástica; pero en lo sucesivo he de ser enérgico, sí señor,
enérgico, y si Santurrias se alegra de que esa plebe indigna vocifere
contra el príncipe de la Paz, sepa que yo mando en mi iglesia, y\ldots{}
no digo más. Parece que soy blando de genio; pero Celestino Santos del
Malvar sabe enfadarse, y cuando se enfada\ldots{}

---Cuando llegue la hora del jaleo, señor cura, su paternidad nos sacará
aquellas botellitas que tiene guardadas en el armario, para que nos
refresquemos---dijo Santurrias descosiéndose de risa otra vez.

---Borracho; así está la santa Iglesia en tus pícaras manos---repuso el
clérigo---Gabriel, ¿querrás creer que hace dos días tuve que coger la
escoba y ponerme a barrer la capilla del Santo Sagrario, que estaba con
media vara de basura? Desde que llegué aquí, me dijeron que este hombre
acostumbraba visitar la taberna del tío Malayerba: yo me propuse
corregirlo con piadosas exhortaciones, pero ¡el diablo le lleve!, hay
días, chiquillo, que hasta el vino del santo sacrificio desaparece de
las vinajeras. ¡Y esto se permite tener opinión, y disputar conmigo,
asegurando que si cae o no cae el dignísimo, el eminentísimo, ¡óigalo
Vd. bien, el incomparabilísimo príncipe de la Paz! ---Pues, y nada más.
¡Como que no le van a arrastrar por las calles de Aranjuez, como al
gigantón de Pascua florida!\ldots{}

---¡Qué abominaciones salen por esa boca, Dios de Israel!

Santurrias tan pronto ahuecaba la voz para cantar gravemente un trozo de
la misa o del oficio de difuntos, como la atiplaba entonando con
grotescos gestos una seguidilla. Luego imitaba el son de las campanas, y
hasta llegó en su irrespetuoso desparpajo, a remedar la voz gangosa de
mi amigo, el cual todo turbado variaba de color a cada instante, sin
poder sobreponerse a las zumbas de su miserable subalterno.

---Pero en resumen---dijo al fin,---¿qué es lo que mi señor sacristán
espera?

¿Cuenta, sin duda, con ordenarse de menores para que le hagan cardenal
subdiácono?

---Allá veremos, Sr.~D. Celestino---contestó el bufón.---Esta noche o
mañana veremos lo que hace Santurrias. No tema nada mi curita; que ya le
pondremos en salvo.

\small
\newlength\mlend
\settowidth\mlend{Tuba mirum spargens sonum }
\begin{center}
\parbox{\mlend}{\textit{Tuba mirum spargens sonum           \\
                        per sepulchra rigionum              \\
                        coget omnes ante thronum.}}         \\
\end{center}
\normalsize

\small
\newlength\mlene
\settowidth\mlene{que aunque tengan las caras de plata}
\begin{center}
\parbox{\mlene}{Esta sí que es tira, tirana:                \\
                ojo alerta, cuidado, señores,               \\
                que aunque tengan las caras de plata        \\
                muchas tienen las manos de cobre.}          \\
\end{center}
\normalsize

---Eso es, mezcle Vd. los cantos divinos con los mundanos. Me gusta.
Pero se me acaba la paciencia, señor rapa-velas. ¡Oh Gabriel!, estoy
sofocadísimo. Yo bien sé que no hay nada; que no ocurre nada: bien sé
que de ese monigote no hay que hacer caso. Sabe Dios cuántos cuartillos
de lo de Yepes tendrá en el bendito estómago; pero conviene
averiguar\ldots{} Mira hijito, sal tú por ahí, entérate bien, y tráeme
noticias de lo que se dice en el pueblo. Puede que esos tunantes tengan
el propósito aleve\ldots{} Si así fuese, haz lo que te digo; que aquí
quedo yo esperándote; y en cuanto descabece un sueñecito, iré a prevenir
al Príncipe, para que se ande con cuidado\ldots{} Pues no me lo
agradecerá poco el buen señor.

No sólo por obedecerle sino también por satisfacer mi curiosidad, salí
de la casa y recorrí las calles del pueblo. El gentío aumentaba en todas
partes, y especialmente en la plaza de San Antonio. No era preciso
molestar a nadie con preguntas para saber que el generoso pueblo,
enojado con la noticia verdadera o falsa de que los Reyes iban a partir
para Andalucía, parecía dispuesto a impedir el viaje, que se consideraba
como una combinación infernal fraguada por Godoy de acuerdo con
Bonaparte.

En todos los grupos se hablaba del generalísimo, como es de suponer, y
en verdad digo que no hubiera querido encontrarme en el pellejo de aquel
señor a quien poco antes había visto tan fastuoso y espléndido; pero
sabido es que la fortuna suele ser la más traidora de las diosas con
aquellos mismos que favoreció demasiado, y no hay que fiarse mucho de
esta ruin cortesana. Decía, pues, que a los vasallos del buen Carlos no
les parecía muy bien el viaje, y aunque hasta entonces no se les había
hablado del derecho a influir en los destinos de esta nuestra bondadosa
madre España, ello es, que guiados, sin duda, por su instinto y buen
ingenio aquellos benditos, se disponían a probar que para algo
respiraban doce millones de seres humanos el aire de la Península.

Más de dos horas estuve paseándome por las calles. Como a cada instante
llegaba gente de la corte traté de encontrar alguna persona conocida;
pero no hallé ningún amigo. Ya me retiraba a la casa del cura, cercana
la noche, cuando de un grupo se apartó un joven de más edad que yo y
llegándose a mí con aparatosa oficiosidad, me saludó llamándome por mi
nombre y pidiéndome informes acerca de mi importantísima salud. Al
pronto no le conocí; mas cuando cambiamos algunas palabras, caí en la
cuenta de que era un señor pinche de las reales cocinas, con quien yo
había trabado conocimiento cinco meses antes en el palacio del Escorial.

---¿No te acuerdas de quién te daba de cenar todas las noches?---me
dijo.---¿No te acuerdas del que te contestaba a tus mil
preguntas?---¡Ah!, sí---repuse,---ya reconozco al Sr.~Lopito; has
engordado sin duda.

---La buena vida, amigo---dijo con petulancia, terciando airosamente la
capa en que se envolvía.---Ya no estoy en las cocinas; he pasado a la
montería del señor infante D. Antonio Pascual, donde no hay mucho que
hacer y se divierte uno. Velay; ahora nos han mandado que nos quitemos
las libreas, y paseemos por el pueblo\ldots{} en fin, esto no se puede
decir.

---Pues yo por nada serviría en palacio. Tres días fui paje de la señora
condesa Amaranta, y quedé harto.

---Quita allá; en ninguna parte se vive como en palacio, porque después
que le dan a uno buena cama, buen plato y buena ropa, cuando llega una
ocasión como esta no falta un dobloncito en el bolsillo\ldots{} pero
esto no es para dicho aquí entre tanta gente, y allí está la taberna del
tío Malayerba, que parece llamarnos, para que refrescando en ella nos
contemos nuestras vidas.

Lopito era un chicuelo de esos que prematuramente se quieren hacer pasar
por hombres, pues también entonces existía esta casta, no conociendo
para tal objeto otros medios que beber a porrillo y dar de puñetazos en
las mesas, desvergonzarse con todo el mundo, mirar con aire matachín, y
contar de sí propios inverosímiles aventuras. Pero con estas cualidades
y otras muchas, el ex-pinche no dejaba de ser simpático, sin duda porque
unía a su vanidosa desenvoltura la generosidad y el rumbo, que acompañan
por lo regular a los pocos años. Convidome a cenar en la taberna,
charlamos luego hasta las nueve y nos separamos tan amigotes, cual si
hubiéramos aprendido a leer en la misma cartilla.

Al día siguiente, como no era posible volverme a Madrid, a causa de que
los trajineros pedían fabulosos precios por el viaje, nos reunimos otra
vez. Lopito estaba tan desocupado como yo, y entre la taberna del tío
Malayerba y los jardines del Príncipe nos pasamos la mayor parte del
día, conferenciando sobre cuanto nos ocurría, y especialmente acerca de
acontecimientos públicos, asunto en que él se daba extraordinaria
importancia. Al principio se mostraba algo reservado en esta cuestión;
pero por último, no pudiendo resistir dentro de su alma el sofocante
peso de un secreto, se franqueó conmigo generosamente.

---Si quieres---me dijo,---puedes ganarte algunos cuartos. Yo te llevaré
a casa del Sr.~Pedro Collado; criado de S. A. el príncipe Fernando, y
verás cómo te dan soldada. ¿Ves esos paletos manchegos que andan por
ahí? Pues todos cobran ocho, diez o doce reales diarios, con viaje
pagado y vino a discreción.

---¿Y por qué es eso, Lopito? Yo creí que esa gente gritaba y chillaba
porque así era su gusto. ¿De modo que todo eso de \emph{vivan nuestros
Reyes} y lo de \emph{muera el choricero} es porque corre el dinero?

---No: te diré. Los españoles todos aborrecen a ese hombre; mas para que
dejen sus casas y tierras y sus caballerías por venir aquí a gritar, es
preciso que alguien les dé el jornal que pierden en un día como este.
Todos los que servimos al infante D. Antonio Pascual y los criados del
príncipe de Asturias hemos estado por ahí buscando gente. De Madrid
hemos traído medio barrio de Maravillas, y en los pueblos de Ocaña,
Titulcia, Villatobas, Corral de Almaguer, Villamejor y Romeral, creo que
no han quedado más que las mujeres y los viejos, pues hasta un racimo de
chiquillos trajo el Sr.~Collado.

---Pero tonto---dije yo, creyendo presentar un argumento
decisivo,---¿qué importa que toda esa gente chille a las puertas de
palacio pidiendo lo que no les han de dar? ¿Pues no tiene ahí S. M. sus
reales tropas para hacerse respetar? Porque o somos o no somos. Si con
un puñado de gente gritona traída de los pueblos y de las Vistillas de
Madrid se puede obligar al rey a que haga una cosa, no sé para qué se
toma ese señor el trabajo de llevar corona en la cabeza.

---Dices bien, Gabrielillo, y si el condenado generalísimo estuviera
seguro de que la tropa le sostenía, ya podían volverse a sus casas todos
esos caballeros, que han venido a darle una serenata; pero tú no sabes
de la misa la media. También han repartido dinero a la tropa---añadió
bajando la voz;---y como el príncipe de Asturias tiene no sé cuántas
arcas llenas de onzas de oro que le ha ido dando su padre para
juguetes\ldots{} ya ves\ldots{} S. A. hará lo que le dé la gana, porque
le ayudan todos los señores de la grandeza, muchos obispos, muchos
generales, y hasta los mismos ministros que ahora tiene el Rey.

---Eso sí que es una grandísima picardía---exclamé con ira.---Son
ministros del Rey, son compañeros del otro, a quien sin duda deben los
zapatos con que se calzan, y al mismo tiempo le hacen la mamola al niño
Fernando, porque ven que el pueblo le quiere, y dicen: «Por fas o nefas,
por la mano derecha o por la izquierda, no ha de tardar en sentarse en
el trono.»

Con este diálogo llegamos a la taberna, y allí nos sentamos, pidiendo
Lopito para sí aguardiente de Chinchón, y yo tintillo de Arganda. No
estábamos solos en aquella academia de buenas costumbres, porque cerca
de la mesa en que nosotros perfeccionábamos nuestra naturaleza física y
moral, se veían hasta dos docenas de caballeros, en cuyas fisonomías
reconocí a algunos famosos Hércules y Teseos de Lavapiés, de aquellos
que invocó con épico acento el poeta al decir:

\small
\newlength\mlenf
\settowidth\mlenf{  Grandes, invencibles héroes,}
\begin{center}
\parbox{\mlenf}{  Grandes, invencibles héroes,               \\
                que en los ejércitos diestros                \\
                de borrachera, rapiña,                       \\
                gatería y vituperio,                         \\
                fatigáis las faltriqueras,                   \\
                  venid á escuchar el modo                   \\
                de vengar nuestro desprecio.                 \\
                Envidiable Pelachón;                         \\
                Marrajo temido y fiero;                      \\
                inimitable Zancudo,                          \\
                y demás que sois modelo                      \\
                de virtudes, venid todos...}                 \\
\end{center}
\normalsize

Entre estos hombres vi otros de figura extraña, y tan astrosos y con
tanto andrajo cubiertos, que daba lástima verlos.

---Estos---me dijo Lopito satisfaciendo mi curiosidad,---son lo
mejorcito de Zocodover de Toledo, donde ejercitan su destreza en el
aligeramiento de bolsillos y alivio de caminantes.

También entraron en las tabernas muchos soldados de caballería, y al
poco rato se había entablado conversación tan viva que no era posible
entender ni una palabra, si palabras pueden llamarse las vociferaciones
y juramentos de aquella gente. Unos sostenían que la familia real
partiría aquella misma tarde, y otros que el Rey no había pensado en tal
viaje. Pronto se disiparon las dudas, porque corrió la voz de que S. M.
dirigía la voz a sus súbditos por medio de una proclama que al punto se
fijó en todos los sitios públicos. En ella, después de llamar
\emph{vasallos} a los españoles, decía el buen Carlos IV, que la noticia
del viaje era invención de la malicia, que no había que temer nada de
los franceses, nuestros queridos amigos y aliados, y que él era muy
dichoso en el seno de su familia y de su pueblo, al cual conceptuaba
asimismo como empachado de prosperidad y bienaventuranza al amparo de
paternales instituciones.

La mayor parte de los héroes de Zocodover y las Vistillas, no parecían
inclinados a dar crédito a la regia palabra, antes bien se burlaban de
cuantos acudían a leerla, añadiendo:---No se nos engañará. A mí con
esas\ldots{} Aspacito, Sr.~D. Carlos, que ya lo arreglaremos.

Cuando fui a casa encontré a D. Celestino loco de alegría: paseaba con
la sotana suelta por su habitación, y aunque no estaba presente ni aun
en sombra el pícaro sacristán, mi amigo profería con desaforado acento
estas palabras:

---¿Lo ves, malvado Santurrias? ¿Lo ves, tunante, borracho, mal acólito,
que no sabes más que juntar gotas de aceite y mocos de vela para
venderlo en pelotillas? ¿Ves cómo yo tenía razón? ¿Ves cómo los Reyes no
han pensado nunca en semejante viaje? Sí, que ahí están esos señores en
el trono para darte gusto a ti, pérfido sacristán, escurridor de
lámparas y ganzúa de cepillos. ¿No bastaba que lo dijera yo, que soy
amigo de Su Alteza Serenísima, y tengo estudios para comprender lo que
conviene al interés de la nación? Véngase Vd. ahora con bromitas,
amenáceme con tocar las campanas sin mi permiso. ¡Ah!, agradézcame el
muy tunante que no me cale ahora mismo el manteo y teja para ir en
persona a contarle a Su Alteza qué clase de pajarraco es usted, con lo
cual, dicho se está que el señor Patriarca me lo pondría de patitas en
la calle. Pero no, Sr.~Santurrias; soy un hombre generoso y no iré; no
quiero quitarle el pan a un viudo con cuatro hijos. Pero véngase Vd.
ahora con bromitas diciendo que mi paisano acá y allá; y que le van a
arrastrar, y repita aquello de «¡Viva Fernando, \emph{Kirie eleyson!}
¡Muera Godoy, \emph{Christe eleyson!}» con que me despierta todos los
días.

A este punto llegaba, cuando advirtió que yo estaba delante, y echándome
los brazos al cuello, me dijo:

---Al fin hemos salido de dudas. Todo era invención de Santurrias. ¿Qué
hay por el pueblo? Estará la gente contentísima ¿no? Ahora cuando salga
el señor príncipe de la Paz a paseo supongo que le victorearán\ldots{}
¡Ay!, qué susto me he llevado, hijito. De veras creí que íbamos a tener
un motín. ¡Un motín! ¿Sabes tú lo que es eso? En mi vida he visto tal
cosa y sírvase Dios llevarme a su seno, antes que lo vea. Un motín no es
ni más ni menos que salirse todos a la calle gritando viva esto o muera
lo otro, y romper alguna vidriera y hasta si se ofrece golpear a algún
desgraciado. ¡Qué horror! Gracias a Dios no tendremos ahora nada de
esto, y sin duda la prudencia y tino de aquel hombre\ldots{} ¿Sabes que
estuve en su palacio a prevenirle de lo que pasaba y no me
recibió?\ldots{}

---Lo creo. En estos días no tendrá Su Alteza humor para recibir, porque
como dijo el otro, no está la Magdalena para tafetanes.

---Tal vez él tenga noticias de las picardías de Santurrias y de los
otros perdidos con quien se junta en la taberna del tío
Malayerba---continuó el cura---¿Pero en dónde está ese endemoniado
sacristán? No parece por aquí porque sabe que le he de poner más
colorado que un pimiento riojano.

No había acabado de decirlo, cuando entreabriéndose la puerta, dejó ver
los dientes, la plegada y siempre risueña boca, la exprimida cara y
arrugada frente del sacristán.

---Venga acá---exclamó D. Celestino con alborozo;---venga el
sapientísimo Sr. Santurrias, presunto cardenal metropolitano; venga acá
para que nos ilustre con su saber, para que nos aconseje con su
prudencia. ¿Puede decirnos cuándo es el viaje? Porque yo tengo para mí
que la proclama de S. M. es una tiñería; y qué crédito merece el Rey de
las Españas, de las Indias de Jerusalén, de Rodas, etc., cuando habla el
Excmo. Sr.~D. Gregorio de las Santurrias, sacristán que fue de monjas
Bernardas, y hoy de mi parroquia. A ver, ¿nos sacará de dudas su
señoría?

---Mañana, mañana, mañanita, señor cura---contestó el
sacristán.---Dígame su paternidad: ¿saca o no las botellicas?

Y luego, sin desconcertarse ante la ironía de su superior, sino por el
contrario burlándose de los graves gestos con que se le interpelaba,
empezó a entonar los singulares cantos de su repertorio, haciendo mil
grotescos visajes y moviendo los brazos, ya en ademán de repicar, ya
aparentando recorrer el teclado de un órgano, ya en fin, con la postura
propia de tocar la guitarra, sin dejar de cantar en la forma siguiente:

\emph{---Domine, ne in furore tuo arguas me\ldots{}}

\small
\newlength\mleng
\settowidth\mleng{que no hay cosa en el mundo}
\begin{center}
\parbox{\mleng}{   Es la corte la mapa                      \\
                de ambas Castillas,                         \\
                y la flor de la corte                       \\
                las Maravillas.                             \\
                  Anda moreno,                              \\
                que no hay cosa en el mundo                 \\
                como tu pelo.}                              \\
\end{center}
\normalsize

\emph{De profundis clamavi ad te, Domine Domine exaudi vocem
meam\ldots{}}

\emph{Don, dilondón, don, don.}

\hypertarget{viii}{%
\chapter{VIII}\label{viii}}

Al día siguiente no hallé tampoco quien me llevase a Madrid; pero
deseando vivamente saber de Inés y curioso por oír de sus propios labios
si era verdad o mentira la bienaventuranza que le habían ofrecido los
Requejos, determiné marcharme a pie, lo cual, si no era muy cómodo, era
más barato: don Celestino y yo hablábamos de esto, cuando Lopito entró a
buscarme.

---Esta noche---me dijo al bajar la escalera,---tendremos fiesta. No lo
digas ni a tu camisa, Gabrielillo. Pues verás\ldots{} aquel papelote que
escribió ayer el Rey es una farsa. Bien decía yo que D. Carlitos, con su
carita de pascua, nos está engañando.

---¿De modo que hay viaje?

---Tan cierto como ahora es día. Pero como no queremos que se vayan,
porque esto es enjuague de Napoleón con Godoy para luego repartirse a
España entre los dos; como no queremos que se vayan, el viaje se prepara
ocultamente para esta noche. Si fuera verdad que no pensaban salir, ¿por
qué no se ha retirado la tropa? ¿Por qué ha venido más tropa y más
tropa, y más tropa? ¿Ves? Ahora está entrando un batallón por la calle
de la Reina.

Confieso que a mí no me importaba gran cosa que saliese un batallón o
entraran ciento, ni tampoco me ponía en cuidado el que mi Sr.~D. Carlos
se marchara a Andalucía o a donde mejor le conviniese. Así se lo
manifesté a mi amigo; pero hallándose el alma de Lopito inundada de
generoso entusiasmo, \emph{por el bien del reino}, me hizo ver que mi
indiferencia era censurable y hasta criminal. Largas horas pasamos
discurriendo por el pueblo y matando el tiempo con amenas
conversaciones. Él se empeñó en llevarme a la taberna, y a la taberna
fuimos. La concurrencia era la misma, aunque el panorama de caras había
variado, viéndose entre ellas la de Santurrias, que no era la menos
animada. También estaba allí muy macilento y meditabundo, con los
agujereados codos sobre la mesa, el poeta calagurritano que tres años
antes capitaneaba la turba de silbantes en el estreno de \emph{El sí de
las niñas}, y con él libaba el néctar de Esquivias en el mismo vaso otro
de los dioses menores del Olimpo Comellesco, el famoso Cuarta y Media,
calderero y poeta. ¡Pobres hijos de Apolo!

El pinche me dijo que todos aquellos personajes habían venido de Madrid
traídos por los confeccionadores de la conjuración, y añadió:

---Esto para que se vea que también toman parte los hombres que se
llaman \emph{científicos}.

No puedo menos de decir que toda aquella gente me repugnaba, y en cuanto
a sus intenciones y propósitos, todo me parecía absurdo sin explicarme
por qué.

---Estúpidos---decía para mí,---¿pensáis que semejante gatería es capaz
de quitar y poner reyes a su antojo?

Pero en la noche de aquel mismo día fue cuando pude medir en toda su
inexplorada profundidad el abismo de ignorancia y fanatismo de aquel
puñado de revolucionarios. No hallando otro alivio a mi aburrimiento que
la asistencia a la taberna en compañía de Lopito, en cuanto cerró la
noche procuré tranquilizar a D. Celestino y me fui allá. Lopito, que me
aguardaba con impaciencia, me dijo al verme a su lado:---Me alegro de
que hayas venido, pues con eso no perderás lo mejor. Aquí está reunida
toda la gente, y después\ldots{} después veremos.

La taberna del tío Malayerba estaba llena de bote en bote, y también
disfrutaba el honor de una desmesurada concurrencia, un patio interior
destinado de ordinario a paradero y taller de carretería. No puedo
haceros formar idea de la variedad de trajes que allí vi, pues creo que
había cuantos han cortado la historia, la costumbre y el hambre con su
triple tijera. Veíanse muchos hombres envueltos en mantas, con sombrero
manchego y abarcas de cuero; otros tantos cuyas cabezas negras y
redondas adornaba un pingajo enrollado, última gradación de turbante
oriental; otros muchos calzados con la silenciosa alpargata, ese pie de
gato que tan bien cuadra al ladrón; muchos con chalecos botonados de
moneditas, se ceñían la faja morada, que parece el último jirón de la
bandera de las comunidades; y entre esta mezcolanza de paños pardos,
sombreros negros y mantas amarillas, se destacaban multitud de capas
encarnadas cubriendo cuerpos famosos de las Vistillas, del Ave María,
del Carnero, de la Paloma, del Águila, del Humilladero, de la
Arganzuela, de Mira el Río, de los Cojos, del Oso, del Tribulete, de
Ministriles, de los Tres Peces, y otros célebres \emph{faubourgs}
(permítasenos la palabrota) donde siempre germinó al beso del sol de
Castilla la flor de la granujería. En cuanto a la variedad de las voces
nada puedo decir, porque todos hablaban a un tiempo. Pero al fin de
aquella reunión, como en todas las de igual naturaleza, resonó una voz
para dominar a las demás. La multitud sabe a veces callar para oír, sin
duda porque se marea con sus propios gritos. Algunos de los presentes
dijeron: «que hable Pujitos,» y al instante Pujitos, cediendo a los
reiterados ruegos de sus \emph{amigos políticos} (dispensadme este
anacronismo), salió al mostrador de la taberna, rompiendo tres vasos y
dos botellas, que sin duda le cargarían en cuenta al heredero de la
corona de dos mundos.

Pujitos era lo que en los sainetes de D. Ramón de la Cruz se señala con
la denominación de \emph{majo decente}, es decir, un majo que lo era más
por afición que por clase, personaje sublimado por el oficio de obra
prima, el de carpintero o el de platero, y que no necesitaba vender
hierro viejo en el Rastro, ni acarrear aguas de las fuentes suburbanas,
ni cortar carne en las plazuelas, ni degollar reses en el matadero, ni
vender aguardiente en \emph{Las Américas}, ni machacar cacao en Santa
Cruz, ni vender torrados en la verbena de San Antonio, ni lavar tripas
allá por el portillo de Gilimón, ni freír buñuelos en la esquina del
hospital de la V.O.T., ni menos se degradaba viviendo holgadamente a
expensas de ninguna mondonguera, o castañera, o de alguna de las muchas
Venus salidas de la jabonosa espuma del Manzanares. Pujitos estaba con
un pie en la clase media; era un artesano honrado, un hábil maestro de
obra prima; pero tan hecho desde su tierna y bulliciosa infancia a las
trapisondas y jaleos manolescos, que ni en el traje ni en las costumbres
se le distinguía de los famosos Tres Pelos, el Ronquito, Majoma, y otras
notabilidades de las que frecuentemente salían a visitar las cortes y
sitios reales de Ceuta, Melilla, etc.

Pujitos era español, y como es fácil comprender, tenía su poco de
imaginación, pues alguno de los granos de sal pródigamente esparcidos
por mano divina sobre esta tierra, había de caer en su cerebro. No sabía
leer, y tenía ese don particular, también español neto, que consiste en
asimilarse fácilmente lo que se oye; pero exagerando o trastornando de
tal manera las ideas, que las repudiaría el mismo que por primera vez
las echó al mundo. Pujitos era además bullanguero; era de esos que en
todas épocas se distinguen, por creer que los gritos públicos sirven de
alguna cosa; gustaba de hablar cuando le oían más de cuatro personas, y
tenía todos los marcados instintos del personaje de club; pero como
entonces no había tales clubs, ni milicias nacionales, fue preciso que
pasaran catorce años para que Pujitos entrara con distinto nombre en el
uso pleno de sus extraordinarias facultades. Setenta años más tarde,
Pujitos hubiera sido un zapatero suscrito a dos o tres periódicos,
teniente de un batallón de voluntarios, vicepresidente de algún círculo
propagandista, elector diestro y activo, vocal de una comisión para la
compra de armas, inventor de algún figurín de uniforme; hubiera hablado
quizás del derecho al \emph{trabajo} y del \emph{colectivismo}, y en vez
de empezar sus discursos así: «\emph{Jeñores: denque los güenos
españoles}\ldots» los comenzaría de este otro modo: «\emph{Ciudadanos: a
raíz de la revolución}\ldots»

Pero entonces no se había hablado de los derechos del hombre, y lo poco
que de la soberanía nacional dijeron algunos, no llegó a las tapiadas
orejas de aquel personaje; ni entonces había asociaciones de obreros, ni
derecho al trabajo, ni batallones de milicias, ni gorros encarnados; ni
había periódicos, ni más discursos que los de la Academia, por cuyas
razones Pujitos no era más que Pujitos.

De pie sobre el mostrador, con la capa terciada, el sombrero echado
sobre la ceja derecha, aquel personaje, hombre pequeño de cuerpo, si
bien de alma grande, morenito, con sus ojuelos abrillantados por los
vapores que le subían del estómago, habló de esta manera:

---Jeñores: denque los güenos españoles golvimos en sí, y vimos quese
menistro de los dimonios tenía vendío el reino a Napolión, risolvimos ir
en ca el palacio de su sacarreal majestad pa icirle cómo estemos cansaos
de que nos gobierne como nos está gobernando, y que naa más sino que nos
han de poner al Príncipe de Asturias, para que el puebro contento diga,
«el \emph{Kirie eleyson} cantando, ¡Viva el príncipe Fernando!»
(\emph{Fuertes gritos y patadas}.) Ansina se ha de hacer, que ínterin
quel otro se guarda el dinero de la Nación, el puebro no come, y Madrid
no quiere al menistro, con que, ¡juera el menistro!, que aquí semos toos
españoles, y si quieren verlo, úrgennos un tantico y verán dó tenemos
las manos. (\emph{Señales de asentimiento.}) Pos sigo iciendo que
esombre nos ha robao, nos ha perdío, y esta noche nos ha de dar cuenta
de too, y hamos de ecirle al Rey que le mande a presillo y que nos ponga
al príncipe Fernando, a quien por esta (y besó la cruz), juro que le
efenderemos contra too el que venga, manque tenga enjércitos y más
enjércitos. Jeñores: astamos ya hasta el gañote, y ahora no hay naa más
sino dejarse de pedricar y coger las armas pacabar con Godoy, y digamos
toos con el ángel:

\small
\newlength\mlenh
\settowidth\mlenh{¡Viva el príncipe Fernando!}
\begin{center}
\parbox{\mlenh}{El \textit{Kirie eleyson} cantando,         \\
                ¡Viva el príncipe Fernando!}                \\
\end{center}
\normalsize

Un alarido, un colosal balido resonó en la taberna, y el orador bajó de
su escabel, rompiendo otro vaso. Mientras limpia el sudor de su frente
coronada con los laureles oratorios, la moza de la taberna se acerca a
escanciarle vino. ¿Es Hebe, la gallarda copera de los dioses, que vierte
el néctar de Chipre en el vaso de oro del joven de los rubios cabellos,
al regresar de la diurna carrera? No: es Mariminguilla, la ninfa de
Perales de Tajuña, a quien trajo desde las riberas de aquel florido río
el Sr.~Malayerba, dándole el cargo de escanciadora mayor, que desempeña
entre pellizcos y requiebros.

Lopito, que tiene con ella alguna aventura pendiente, la llama, la
pellizca también, dícele mil niñerías\ldots{} pero a todas estas la
multitud que ocupa la taberna se levanta obedeciendo a la orden de un
hombre que allí se presentó de improviso. Salieron todos, y yo no
queriendo perder el final de una función que parecía ser divertida, les
seguí.

---Silencio todo el mundo---dijo una voz, perteneciente, según
comprendí, a persona resuelta a hacerse obedecer; y la turba se puso en
marcha con cierto orden. La noche era oscurísima; pero serena.

---¿A dónde vamos, Lopito?---pregunté a mi compañero.

---A donde nos lleven---me contestó por lo bajo.---¿A que no sabes quién
es ese que nos manda?

---¿Quién? ¿Aquel palurdo que va delante con montera, garrote, chaqueta
de paño pardo y polainas; que se para a ratos, mira por las boca calles
y se vuelve hacia acá para mandar que callen?

---Sí; pues ese es el señor conde de Montijo. Con que figúrate,
chiquillo, si no podemos decir aquel refrán de\ldots{} cuando los santos
hablan, será porque Dios les habrá dado licencia.

\hypertarget{ix}{%
\chapter{IX}\label{ix}}

El grupo recorrió algunas calles, y uniose a otro más numeroso que
encontramos al cuarto de hora de haber salido. Lopito, señalándome las
tapias que se veían en el fondo del largo callejón, me dijo:

---Aquellas son las cocheras y la huerta del Príncipe de la Paz.

Pasamos de largo y vimos de lejos las dos cúpulas del palacio. Cerca del
mercado se nos unieron otras muchas personas que, según Lopito, eran
cocheros, palafreneros, pinches, mozos de cuadra y lacayos del infante
D. Antonio y del príncipe de Asturias.

---Pero ¿qué vamos a hacer aquí?---pregunté a mi amigo.---¿Vamos a
impedir que los Reyes salgan del pueblo, o vamos simplemente a tomar el
fresco?

---Eso lo hemos de ver pronto---me contestó.---Yo, si he de decirte la
verdad, no sé lo que se ha de hacer, porque Salvador el cochero no me ha
dicho más sino que vaya donde van los demás y grite lo que los demás
griten. Ves, ahí frente tenemos el palacio: no hay luces en las ventanas
ni se oye ruido alguno, como no sea el de las ranas que cantan en los
charcos del río. La voz del que nos mandaba dijo «alto,» y no dimos un
paso más.

---Es raro---dije a Lopito muy quedamente,---que no hayamos encontrado
centinelas que nos detengan; ni siquiera una ronda de tropa que nos
pregunte a dónde vamos a estas horas.

---¡Necio!---me contestó.---¡Si sabrá la tropa lo que se pesca! ¿Pues
qué hacen ellos si no estarse quietecitos en sus cuarteles esperando a
que les digan: caballeros, esto se acabó?

Dime por convencido y callé. Durante un rato bastante largo no se oyó
más que el sordo murmullo de diálogos sostenidos en voz baja, algunos
sordos ronquidos, sofocadas toses, y a lo lejos el canto de las
discutidoras ranas y el rumor de leves movimientos del aire, sacudiendo
las ramas de los olmos, que empezaban a reverdecer. La noche era
tranquila, triste, impregnada de ese perfume extraño que emiten las
primeras germinaciones de la primavera: el cielo estaba tachonado de
estrellas, a cuya pálida claridad se dibujaban los espesos y negras
arboledas, la silueta cortada del Real Palacio, y más allá la figura del
Anteo de mármol levantado del suelo por Hércules en el grupo de la
fuente monumental que limita el llamado \emph{Parterre}. El sitio y la
hora eran más propios para la meditación que para la asonada.

De improviso aquel silencio profundo y aquella oscuridad intensa se
interrumpieron por el relámpago de un fogonazo y el estrépito de un tiro
que no sé de dónde partió. La turba de que yo formaba parte lanzó mil
gritos, desparramándose en todas direcciones. Parecía que reventaba una
mina, pues no a otra cosa puedo comparar la erupción de aquel rencor
contenido. Todos corrían, yo corría también. Lucieron antorchas y
linternas, se alzaron al aire nudosos garrotes: muchas escopetas se
dispararon, oyose un son vivísimo de cornetas militares, y multitud de
piedras, despedidas por manos muy diestras, fueron a despedazar,
produciendo horribles chasquidos, los cristales de una gran casa. Era la
del Príncipe de la Paz.

La historia dice que el tumulto empezó porque la turba se empeñó en
conocer a una dama encubierta que, acompañada de dos guardias de honor,
salía en coche de casa del generalísimo. Aseguran algunos que en una de
las ventanas del palacio se vio una luz, considerada como señal para
empezar la gresca.

Del tiro y toque de corneta no tengo duda, porque los oí perfectamente.
En cuanto a la luz, yo no la vi, pero creo haber oído decir a Lopito que
él la vio, aunque no estoy muy seguro de ello. Poco importa que
apareciera o no: lo primero es, si no cierto, muy verosímil, porque el
centro de la conjuración estaba en el alcázar, y los principales
conspiradores eran, como todo el mundo sabe, el príncipe de Asturias, su
tío, su hermano, sus amigos y adláteres, muchos gentiles hombres, altos
funcionarios de la casa del Rey y algunos ministros.

Los alborotadores se multiplicaban a cada momento, pues nuevas oleadas
de gente engrosaban la masa principal, sin que un soldado se presentase
a contener al paisanaje. No tardó en caer al suelo destrozada por
repetidos golpes y hachazos la puerta del palacio del Príncipe de la
Paz, cuyo nombre pronunciaba el irritado vulgo entre horribles
juramentos y amenazas.

La turba siempre es valiente en presencia de estos ídolos indefensos,
para quienes ha sonado la hora de la caída. Tienen estos en contra suya
la fatalidad de verse abandonados de improviso por los amigos tibios,
por los servidores asalariados y hasta por los que todo lo deben al
infeliz que cae, de modo que a las manos del odio justo o injusto, se
unen para rematar la víctima las manos de la ingratitud, el más canalla
de todos los vicios. Sintiendo el auxilio de la ingratitud, la turba se
envalentona, se cree omnipotente e inspirada por un astro divino, y
después se atribuye orgullosamente la victoria. La verdad es que todas
las caídas repentinas, así como las elevaciones de la misma clase,
tienen un manubrio interior, manejado por manos más expertas que las del
vulgo.

Cuando la puerta de la casa se abrió, precipitose la turba en lo
interior, bramando de coraje. Su salvaje resoplido me causaba terror e
indignación, mayormente cuando consideré que iba a saciar su sed de
venganza en la persona de un hombre indefenso. Era aquella la primera
vez que veía al pueblo haciendo justicia por sí mismo, y desde entonces
le aborrezco como juez.

A los gritos de «¡Muera Godoy!» se mezclaban preguntas de feroz
impaciencia; «¿Le han cogido?» «¿Le han matado?» Todos querían entrar;
pero no era posible, porque la casa estaba ya atestada de gente. Desde
fuera y al través de los balcones de par en par abiertos, se veía el
resplandor de las hachas: siniestros gritos y ruidos de muebles o vasos
que se quebraban bajo las garras de la fiera, salían de la casa a
mezclarse con el concierto exterior. En un instante se encendió una gran
hoguera que iluminó la calle: las campanas de todas las iglesias y
conventos del pueblo tocaban sin cesar; pero no podía definirse si
aquellos tañidos eran toques de alarma o repiques de triunfo.

Lopito, que bailaba como un demonio adolescente junto a la hoguera, se
acercó a mí y me dijo:

---Gabriel, ¿no te entusiasmas?¿Qué haces ahí tan friote? Ven, subamos
al palacio. Alguna vez ha de ser para nosotros. ¿No dicen que todo lo ha
robado a la nación?

Casi arrastrado por mi joven amigo entré en el palacio y subí a las
habitaciones altas, abriéndonos paso por entre los energúmenos que
bajaban y subían. Recorrí todas las salas por las cuales había
transitado dos días antes, llegué al mismo despacho del príncipe, y vi
la mesa donde escribí mi nombre. La multitud subía y bajaba, abría
alacenas, rompía tapices, volcaba sofás y sillones, creyendo encontrar
tras alguno de estos muebles al objeto de su ira; violentaba las puertas
a puñetazos; hacía trizas a puntapiés los biombos pintados; desahogaba
su indignación en inocentes vasos de China; esparcía lujosos uniformes
por el suelo, desgarraba ropas, miraba con estúpido asombro su espantosa
faz en los espejos, y después los rompía; llevaba a la boca los restos
de cena que existían aún calientes en la mesa del comedor; se arrojaba
sobre los finos muebles para quebrarlos, escupía en los cuadros de Goya,
golpeaba todo por el simple placer de descargar sus puños en alguna
parte; tenía la voluptuosidad de la destrucción, el brutal instinto tan
propio de los niños por la edad como de los que lo son por la
ignorancia; rompía con fruición los objetos de arte, como rompe el rapaz
en su despecho la cartilla que no entiende; y en esta tarea de
exterminio la terrible fiera empleaba a la vez y en espantosa coalición
todas sus herramientas, las manos, las patas, las garras, las uñas y los
dientes, repartiendo puñetazos, patadas, coces, rasguños, dentelladas,
testarazos y mordiscos. La rabia del monstruo aumentó cuando corrieron
de boca en boca estas frases: «No está ese perro.» «El endino se ha
escapao.» Efectivamente; el Príncipe no parecía por ninguna parte, de lo
cual me alegré.

Cuando la turba no puede saciar su hambre de destrucción en el objeto
humano de su rencor, suele darse el gustazo de tomar venganza en los
cuerpos inocentes de los muebles que a aquel pertenecieron. Así ha
ocurrido en todos los motines de nuestro repertorio, y así ocurrió en
aquel, más que ninguno famoso, por las diversas causas que lo
ocasionaron. Convencidos, pues, los conjurados de que no habrían a las
manos ni un pelo del Príncipe de la Paz, concibieron el heroico
pensamiento de quemar todas las preciosidades del palacio recién
saqueado.

Con gozo sin igual, con la embriaguez del triunfo y la conciencia de su
fuerza irresistible, comenzaron los nuevos huéspedes del palacio a
arrojar por los balcones sillas, sofás, tapices, vasos, cuadros,
candelabros, espejos, ropas, papeles, vajillas y otros mil perversos
cómplices de la infame política de Godoy. La fiera cumplía este cometido
con cierto orden, sin dejar de decir:

«¡Muera ese tunante, ladrón!» y «¡Viva el Rey, viva el Príncipe de
Asturias!»

Pero antes de que empezara esta operación, y cuando los exploradores se
convencieron de que el Príncipe había huido, la Princesa de la Paz, que
estaba hasta entonces oculta, se presentó pidiendo socorro, e implorando
la compasión de la multitud. El miedo hacía temblar a la infeliz señora,
lo mismo que a su hija, niña de corta edad que con ambos puños en los
ojos lloraba sin consuelo. No sé si los ruegos de la madre y de la hija
ablandaron a los amotinados, o si las personas de categoría que dirigían
la fiesta determinaron poner en salvo con todo miramiento y
consideración a la infeliz princesa; lo cierto fue, que lejos de
maltratarla de obra o de palabra, sacáronla de la casa, y puesta en una
berlina fue llevada en \emph{ca el palacio} de los reyes, como decía
Pujitos, quien sin que nadie se lo ordenara, se encargó de tan
caballeresca comisión.

Ustedes comprenderán que todo lo que fuese figurar en primer término
agradaba a Pujitos, así es que si se reunía un pelotón para marchar a
cualquier parte, allí estaba él para mandarlo, complaciéndose en decir:
«Marchen, media güelta a lizquielda,» con tanta marcialidad como un
coronel de guardias walonas. No me cansaré de repetirlo: Pujitos tenía
en su cráneo entre un lobanillo y un chichón, la protuberancia (¿cómo lo
diré\ldots?) la protuberancia de la \emph{tenientividad}. Como Napoleón
el genio de la guerra, poseía él el instinto de la milicia nacional; mas
los hados no quisieron que llegase a mandar ninguna compañía de aquella
honrada fuerza, porque antes de 1820 la Parca cruel lo arrebató de este
mundo, privando a nuestro planeta de tan grande y simpática figura.

Cuando los infatigables trabajadores del motín comenzaron a arrojar por
ventanas y balcones los muebles del palacio, Lopito, que llevaba a
cuestas una maravillosa obra de porcelana, producto de los talleres de
la Moncloa, se llegó a mí y díjome:

---Gabrielillo, cuidado cómo coges nada. El \emph{tío Pedro}, que está
allí observando lo que hacemos, tiene en la mano una pistola, y dice que
levantará la tapa de los sesos al que robe cualquier chuchería. No es el
único gran caballero que anda entre nosotros. ¿Ves aquel hombre vestido
de majo que está dando de patadas a un retrato de cuerpo entero? Pues es
un gentilombre del cuarto del Príncipe. ¿Ves?, ya pasó el pie del otro
lado de la tela. Tremendo agujero le han hecho. ¡Al fuego, al fuego!

La hoguera, alimentada con tanto combustible, subía a enorme altura, y
las llamas oscilantes iluminaban de un modo pavoroso la calle toda, y
también el interior del palacio. Parecíamos los cíclopes de una inmensa
fragua; y digo parecíamos, porque yo también, temiendo que mi falta de
entusiasmo fuera sospechosa y me proporcionase algún porrazo, puse manos
a la obra, y cogiendo una armadura milanesa, en cuyo peto y casco se
veían batallas microscópicas trabajadas por finísimo cincel, di con ella
en la calle y en la hoguera. Ni por un momento cesaban los gritos de
«muera Godoy;» y sin duda querían matarle a voces ya que de otra manera
les fue imposible conseguirlo. Pero es de advertir que entre nosotros es
muy común el intento de arreglar las más difíciles cuestiones mandando
vivir o morir a quien se nos antoja, y somos tan dados a los gritos que
repetidas veces hemos creído hacer con ellos alguna cosa.

Yo no sé si los asaltadores de la casa del Príncipe de la Paz creían
estar quemando algo más que muebles muy finos y primorosas obras de
arte; pero por lo que en boca de alguno de aquellos héroes oí, se me
figura que estaban convencidos de que hacían un gran papel político; de
que con la llama de los espinos y de los brezos, sin cesar alimentada
por ébanos tallados y bordadas telas, estaban cauterizando las más feas
llagas de la doliente España. ¡Ay! He presenciado después la misma
escena repetida cada pocos años ya por esta idea, ya por la otra, y he
dicho: «Algunas veces puede conseguirlo la espada en manos de un hombre
de genio; pero el fuego en manos del vulgo, jamás.»

Tras la armadura cogí un reló de bronce, y al llevarlo sobre mí sentía
el palpitar de su máquina. El pobrecillo andaba, vivía; aquel artificio
que tanto se parece a un ser animado, aquella obra de los hombres que
parece obra de Dios, y que ha sido inventada por la ciencia y adornada
por las artes para uno de los más útiles empleos de la vida, iba a
perecer a manos del hombre mismo, sin haber cometido más crimen que el
de marcar las horas\ldots{} ¿Pero a qué vienen estas consideraciones
hechas ante la hoguera del rencor? Aunque me daba lástima del relojito,
y lo estrechaba contra mi pecho escuchando su latido que iba a
extinguirse, arrojelo al fin, y las mil piezas de su máquina ingeniosa
repercutieron sobre el suelo. Al reló siguieron cuantas baratijas
encontré a mano, entre ellas guantes perfumados, un estuche de marfil,
pequeñas estatuas de alabastro y después unos mapas del Asia, libros
lujosamente encuadernados que sin duda los muy necios se creían libres
de la Inquisición, unas pantuflas, cuatro casacas con galones de plata y
oro y el pupitre en que dos días antes se había extendido mi
recomendación. ¡Fortuna, vil prostituta, por qué te invocan los hombres!

\hypertarget{x}{%
\chapter{X}\label{x}}

Cuando revolvía uno de los armarios, aparecieron varias cruces; pero
algunos de los presentes, ni aun me permitieron tocarlas, y pusiéronlas
todas en una bandeja de plata, para entregarlas, según decían, al Rey en
persona. Lo más singular de la determinación de aquellos cortesanos
tiznados con el hollín de la demagogia, era que disputaban sobre quién
debía llevarlas, pues ninguno quería ceder a los demás semejante honor.
Uno de ellos venció al fin; y no quisiera equivocarme, pero me pareció
reconocer al señor de Mañara.

Con el crecer de la llama parecía que cobraban nuevos bríos los
quemadores, si bien puede atribuirse este fenómeno a que algunos zaques
dieron vuelta a la redonda, humedeciendo los secos paladares, y
alegrando los ánimos que un trabajo tan penoso como patriótico, había
comenzado a abatir. Creí oír la voz de Pujitos obligado nuevamente por
sus \emph{amigos políticos} a tomar la palabra; pero no, era Santurrias,
que teniendo en la izquierda la bota y en la derecha mano un leño
encendido, pronunciaba sentidas frases en loor del pueblo y del Rey,
ambos en buen amor y compaña, para bien del \emph{reino}; y añadía que
el endino Príncipe de la Paz estaba bien castigado, puesto que eran ya
cenizas todos los muebles que robó al \emph{reino}, y que de \emph{aquí
palante}, es decir, en lo sucesivo, no habría más \emph{menistros}
pillos y \emph{lairones}.

Las hogueras, cuando ya no había nada que echarles, se aplacaron: el
populacho, mientras el tío Malayerba tuvo vino, y Pujitos y Santurrias
elocuencia, seguía ardiendo y chisporroteando. Algunos quisieron
trasladar el teatro de sus ingeniosas proezas a las puertas de palacio,
no siendo extraños los dos oradores a un proyecto que ensanchaba la
esfera de sus triunfos; pero debió oponerse a esto el tío Pedro y
compañeros de polaina, mayormente cuando tenían la seguridad de que el
motín de las calles no era más que una sucursal de la gran asonada que
en los mismos momentos estallaba en palacio y en la cámara del rey
Carlos IV.

Era ya la madrugada cuando quise retirarme, sin que lograra detenerme
Lopito, que decía:

---Aún falta lo mejor. ¿Qué te parece, Gabrielillo, lo que hemos hecho?
Pues entavía hemos de hacer mucho más. Ya habrá visto el Rey si se puede
o no se puede. Pónganos otra vez menistros malos y verá cómo en menos
que canta un gallo los despabilamos. Lo que es Lopito\ldots{} je,
je\ldots{} ya habrán visto que tiene malas moscas\ldots{} y como yo
hubiera encontrado a Godoy en cualquiera parte de la casa, le juro que
no sale vivo de mis manos.

Diciendo esto, el valiente pinche sacó una navajilla con la cual le vi
describir heroicas curvas en el aire.

---Y si llegamos a ir a palacio---prosiguió alzando el arma
homicida,---yo, yo mesmito soy el que me presento al Rey y a la Reina
para decirles que si no nos ponen al príncipe Fernando en el trono, lo
pondremos nosotros. Lo que es al Rey no le haré nada, porque es el Rey;
pero a la Reina, manque se ponga de rodillas delante, no la perdono.
Dijo y guardó el arma. A todas estas llegó una compañía de guardias para
custodiar la casa después de saqueada: fácil era comprender la
inteligente dirección del motín de que había sido brutal instrumento un
pueblo sencillo. Este no hubiera podido dar un paso más allá de la línea
que se le marcara sin sentir encima la fuerte mano de la autoridad.

No necesito decir que cuando se montó la guardia, el predestinado
Pujitos quiso formar parte de ella, aunque no era militar, y su genio
organizador se entretuvo en reunir en pelotón hasta una docena de
hombres, con los cuales se ocupó en patrullar por las inmediaciones de
la casa, mandándoles marchar a compás y supliendo él mismo con su voz la
falta de tambor.

Al fin me marché, no sólo porque tenía sueño, sino porque cuanto había
visto y oído me repugnaba con exceso. Llegué a la casa del cura, y no
puedo haceros formar idea del estado de agitación y fiebre en que le
encontré. Envuelta en un pañuelo la cabeza, puesta la sotana vieja y con
un antiguo gabán de paño burdo echado sobre los hombros y sus anchos
pantuflos en los pies, estaba mi buen eclesiástico recorriendo de largo
a largo los corredores y pasillos de su casa. Su aspecto era semejante
al de los que sufren un terrible dolor de muelas; a cada instante se
llevaba las manos a las orejas, como para resguardarlas del ruido que
hacían aún las campanas de la iglesia vecina; de vez en cuando golpeaba
el suelo con fuerte patada, y a lo mejor daba media vuelta, cambiando de
dirección en su calenturiento paseo. Entretanto, no cesaba de hablar un
solo momento. ¿Con quién? ¿Con las paredes, con la luna, con la parra,
que enredándose en los maderos del corredor extendía sus flacos y secos
brazos para coger alguna cosa? Cuando me vio, hablome sin aguardar a que
llegase a su lado.

---Estoy loco, Gabrielillo, ¿qué pasa, qué ocurre? ¿Oyes las campanas de
la parroquia? Por los mártires de Alcalá juro\ldots{} no, jurar no, que
es pecado\ldots{} prometo que Santurrias me las ha de pagar todas
juntas. ¿Pero has visto cómo se burla de mí ese condenado? No es él el
que toca, que si fuera\ldots{} Mira, estaba yo descabezando el primer
sueño cuando me hizo saltar de la cama el ruido de las campanas. ¡Dios
mío, qué algazara! Plin, plan, plin, plan\ldots{} parecía que el cielo
se venía abajo. Lleno de indignación subí a la torre, pero Santurrias no
estaba, y en su lugar sus cuatro hijos tocaban las campanas. Tal era mi
cólera, que resolví mostrar la mayor energía y les dije: «Pillos,
granujas, váyanse de aquí noramala;» pero ellos se rieron de mí y
siguieron tocando\ldots{} plin, plan, plin, plan\ldots{} ¡Si hubieras
visto a los cuatro condenados muchachos, con qué alegría, con qué
frenesí tiraban de las cuerdas!\ldots{} ¡Malditos sean!\ldots{} Pues uno
de ellos, el mayor, es listillo y muy mono\ldots{} y ayuda a misa como
un zarapico. Pero me dio tal enfado, que les mandé salir de la torre.
¿Tú me obedeciste?, pues ellos tampoco; el más chico me dijo:
«\emph{Pare Gorio jue a matal a Godoy y nos puso a que tocálamos fuelte,
fuelte}.» Desde las once hasta ahora no han cesado ni un momento. ¿Pero
dime, qué ocurre en el pueblo? He visto el resplandor de una llamarada,
he sentido gritos. La tía Gila fue por orden mía a ver lo que pasaba, y
volvió horrorizada, diciendo que estaban quemando todo el Palacio Real
de punta a punta, y los jardines, y el Tajo y la cascada. Cuéntame,
hijito, que estoy sin sosiego.

Contele lo que había pasado en casa del Príncipe su amigo.

---Pero a estas horas habrán salido las tropas para castigar a esa vil
plebe,---me dijo.

---¡Quia! ¡Si entre la multitud había muchos soldados! La tropa debe de
estar sobornada.

---Pero a estas horas el Príncipe ha de estar tomando sus disposiciones
para arreglarlo todo\ldots{} porque él no es hombre que se anda con
chiquitas, y si les sienta la mano\ldots{} Cuánto deploro no haber
podido advertirle ayer lo que se preparaba. Ya ves, hubiéramos podido
evitar ese tumulto. ¡Miserable de mí!\ldots{} Yo, yo tengo la culpa de
lo que está pasando. Si no fuera por este genio corto que Dios me ha
dado\ldots---El Príncipe ha huido, y debe estar a estas horas muy lejos
de Aranjuez.

---¡Que ha huido! No puede ser, no puede ser---exclamó con cierta
enajenación---Gabriel: ¿para qué mientes? ¿O eres tú también de los que
creen las majaderías y simplezas de Santurrias?

A este punto llegábamos de nuestro coloquio, cuando sentimos una voz
ronca y desapacible que gritaba en el portal.

---¡Ah!---dijo el cura,---me parece que siento a Santurrias. Ahora va a
ser ella: no intercedas por él\ldots{} estoy decidido\ldots{} ahora sí
que es preciso ser enérgico.

La voz se acercaba. Era efectivamente el sacristán, que cantaba así,
subiendo por la escalera:

\small
\newlength\mleni
\settowidth\mleni{por veinticinco pares }
\begin{center}
\parbox{\mleni}{  Vale una seguidilla                       \\
                de las manchegas,                           \\
                por veinticinco pares                       \\
                de las boleras.}                            \\
\end{center}
\normalsize

\emph{Solvet sæclum in favilla, teste David cum Sibylla.}

---Váyase Vd., Sr.~Santurrias---exclamó el cura.---No le quiero ver a
Vd., no quiero oír sus necedades.

El sacristán, que hasta entonces no nos había visto, se paró ante
nosotros, y lanzando una carcajada de estupidez, habló así, con lengua
estropajosa:

\small
\newlength\mlenj
\settowidth\mlenj{¡Viva el príncipe Fernando!}
\begin{center}
\parbox{\mlenj}{El \textit{Kirie eleyson} cantando,         \\
                ¡Viva el príncipe Fernando!}                \\
\end{center}
\normalsize

Luego dio fuertes golpes en el suelo con un garrote medio quemado que en
la mano traía, y acto continuo empezó a marchar militarmente por el
corredor, imitando con la boca el ruido del tambor.

---¡Está borracho!---dijo el cura.---Pero miserable, ¿no ves que el vino
se te sale por los ojos?

Santurrias, apoyado en su palo para no caer al suelo, alargó su cuello,
fijó en nosotros los encandilados ojos, arrugose su cara más aún que de
ordinario, y dijo:

---Señor paterniá: el Príncipe ha juío\ldots{} ¡Viva el Rey! ¡Muera el
Choricero! ¡Muera ese pillo lairón!\ldots{} ¡\emph{O salutaris
hooo\ldots{} stia!} Si me bían dejao, le hago porvo con este
palo\ldots{} Prrun, prrun\ldots{} ¡marchen! Media güelta\ldots{} ¡Viva
el comendante Pujitos!

---¡Oh espectáculo lastimoso!---dijo D. Celestino.---Está como una cuba.
Ya no le aguanto más\ldots{} a la calle, a la calle mañana mismo. Se lo
diré al señor patriarca\ldots{} Pero no; ahora me acuerdo de que es un
viudo con cuatro hijos.

A todas estas las campanas seguían tocando con igual furia, prueba
evidente de que el entusiasmo de los cuatro muchachos no había
disminuido.

Santurrias se agarró al antepecho del corredor para no caer. Después de
haber dicho mil herejías, que a D. Celestino le pusieron el cabello de
puntas, dijo que nos iba a contar lo que había hecho.

---Calla de una vez, deshonra de la santa Iglesia, borracho, hereje,
blasfemo---le dijo D. Celestino empujándole.---Yo te aseguro que si no
fueras un viudo con cuatro hijos\ldots{}

---Pos, pos\ldots---balbuceó Santurrias:---lo que hamos hecho se
llama\ldots{} ¡rigolución!\ldots{} Que si vamos a palacio, que si no
vamos. Yo quería ir pa pedí la aldicación.

---¡Cómo!---exclamó el cura con espanto.---¿Ha abdicado S. M. el rey
Carlos IV?

---Nones\ldots{} entavía nones\ldots{}

\small
\newlength\mlenk
\settowidth\mlenk{Quando judex est venturus.}
\begin{center}
\parbox{\mlenk}{\textit{Quantus tremor es futurus                   \\
                        Quando judex est venturus.}}                \\
\end{center}
\normalsize

\small
\newlength\mlenl
\settowidth\mlenl{que merece la moza}
\begin{center}
\parbox{\mlenl}{  Viva quien baila,                         \\ 
                que merece la moza                          \\
                mejor de España.}                           \\
\end{center}
\normalsize

¡Muera Godoy!\ldots{} marchen\ldots{} señor cura: ya el menistro no es
menistro, polque el Rey\ldots{}

---Creo que el Rey---dije yo para sacar de su ansiedad al buen
anciano,---ha firmado ya la destitución del Príncipe de la Paz. Según
allí se dijo, los ministros que estaban en palacio se lo pedían así.

---Eso\ldots{} eso\ldots{} juimos a palacio---continuó Santurrias, que
no pudiendo sostenerse ya, había caído al suelo,---y salió un gentilón
con un papé escrito y leyó\ldots{} y decía\ldots{} decía:
«\emph{Queriendo mandal por mi mesma mesmedá en el enjército y la
marina, he venido en ex\ldots{} ex\ldots{} ex\ldots{}}.»

---En exonerar---dijo el cura dirigiendo sus ojos al cielo. Santurrias
murmuró algunas palabras más entre latinas y castellanas, y calló al
fin. Un fuerte ronquido anunció el aplanamiento de aquel elevado
espíritu, conturbado por el vino de la conjuración.

Observé que D. Celestino enjugaba una lágrima con la punta del mismo
pañuelo que tenía arrollado en la cabeza. Amanecía, y una turba de
pájaros procedentes de los árboles cercanos, pasaron por sobre el patio
cantando un himno de paz. Las primeras luces de la mañana iluminaron la
casa, y el cura se retiró a su cuarto, diciendo:

---Dentro de un rato diré la misa y la aplicaré por la salvación de mi
amigo el Príncipe de la Paz\ldots{} ¡Ay!, si yo le hubiera avisado con
tiempo\ldots{} Pero ¿no oyes? ¡Esas condenadas campanas me tienen loco!

En efecto, los cuatro muchachos seguían tocando.

\hypertarget{xi}{%
\chapter{XI}\label{xi}}

Pasé todo aquel día durmiendo. Al caer de la tarde salí para observar el
aspecto del pueblo, y en la taberna encontré a Lopito, que hacía con su
navajita mil rúbricas en el aire, para que le viera Mariminguilla.
Después, guardando el arma, me dijo:

---Le he caído en gracia a la muchacha, y si el tío Malayerba no me la
deja sacar de aquí, ya sabrá quién es Lopito. ¡Qué bien me porté anoche,
Gabriel! Todos están entusiasmados conmigo, y para cuando tengamos al
Príncipe en el trono, ya me han prometido darme una plaza de ocho mil
reales en la contaduría del Consejo de Hacienda.

---Chico, si tienes buena letra\ldots{}

---Ni buena ni mala, porque no sé escribir; pero eso será lo de menos.
Me ha dicho Juan el cochero que ahora van a quitar de las oficinas a
todos los que puso el Príncipe de la Paz, y como son cientos de miles,
quedarán muchas plazas vacantes. Conque a toos nos han de poner\ldots{}
porque, chico, esto de la montería me cansa, y para algo más que para
cuidar perros y machos de perdiz, me parece que nos echaron nuestras
madres al mundo.

---Pero ¿ponen al Príncipe de Asturias, o no le ponen?

---Nos lo pondrán; y si no, ¿para qué vienen ahí las tropas de Napoleón?
¡Qué bueno estuvo lo de anoche! Dicen que el Rey temblaba como un
chiquillo, y quería venir a calmarnos; pero parece que los ministrillos
no le dejaron. La Reina decía que nos debían matar a todos para que no
pasara aquí otra como la de Francia, donde le cortaron la cabeza a los
reyes con un instrumento que llaman la \emph{tía Guillotina}. Así me lo
contó esta mañana Pujitos, que sabe de toas estas cosas, y lo ha leído
en un papel que tiene. Nosotros queremos al Rey, porque es el Rey, y
esta mañana, cuando salió al balcón, gritamos mucho y le aclamamos. Él
se llevaba la mano a los ojos para secarse las lágrimas; pero la
condenada Reina estaba allí como un palo, y no nos saludó. Pujitos que
lo sabe todo, dice que es porque está afligida con lo que hemos hecho en
casa del Choricero, y asegura que ella lo tiene escondido en su camarín.

---Puede ser.

---Pues yo me he lucido---continuó Lopito alzando la voz para que lo
oyera Mariminguilla.---Esta mañana cuando prendieron a D. Diego Godoy,
hermano del ministro, íbamos toos gritando detrás, y yo le tiré una
piedra, que si le llega a dar en metá la cara, lo deja en el sitio.

---¿Y qué había hecho ese señor?

---¿Te parece poco ser hermano de ese pillastrón? Era coronel de
guardias, pero sus mismos soldados le quitaron las insignias y ahora me
lo van a llevar a un castillo.

Aquella noche oí un nuevo discurso de Pujitos; pero haré a mis lectores
el señalado favor de no copiarlo aquí. El poeta calagurritano que antes
mencioné, jefe de la conspiración literaria fraguada contra El sí de las
niñas, se arrimó a nosotros, acompañado de Cuarta y Media, y entre uno y
otro nos descerrajaron la cabeza con media docena de sonetos y otros
proyectiles fundidos en sus cerebros. Pero después que nos molieron a
sonetazos, Lopito trabó cierta pendencia con el poeta, porque a este se
le antojó requebrar a Mariminguilla, llamándola \emph{ninfa} de no sé
qué aguas o poéticos charcos. La navaja de Lopito salió a relucir, y si
el poeta no hubiera sido el más cobarde de los cabalgantes del Pegaso,
habría corrido mezclada en espantoso río la sangre de un futuro empleado
de Hacienda, y la de un pretérito émulo del viejo Homero.

Nada más ocurrió en aquella noche, digno de ser transmitido a la
posteridad; pero a la mañana siguiente se esparció con la rapidez del
rayo por todo el pueblo la voz de que el Príncipe de la Paz había sido
encontrado en su propia casa. La taberna del tío Malayerba se vació en
dos minutos, y de todas partes cundió en gran masa la gente para verle
salir.

Era cierto: Godoy se había refugiado en un desván donde le encerró uno
de sus sirvientes, el cual, preso después, no pudo acudir a sacarle. A
las treinta y seis horas de encierro, el Príncipe, prefiriendo sin duda
la muerte a la angustia, hambre y sed que le devoraban, bajó de su
escondite, presentándose a los guardias que custodiaban su morada.
Estos, lejos de amparar al que un día antes era su jefe, alborotaron el
vecindario, y la misma turbamulta de la noche del 17 acudió con heroico
entusiasmo a apoderarse de él.---¡Ya pareció, ya le cogimos, ya es
nuestro!---exclamaban muchas voces.

Fuimos todos allá, y en la puerta del palacio el agolpado gentío formaba
una muralla. Los feroces gritos, los aullidos de cólera componían
espantoso y discorde concierto. Sorprendiome oír entre tanta algarabía
las voces de algunas mujeres chillonas, que deshonraban a su sexo
pidiendo venganza. Lopito no cabía en sí de satisfacción, y la navajilla
fue blandida sobre nuestras cabezas, como si quisiera partir el
firmamento en dos pedazos.

Empujábamos todos, pugnando cada cual por acercarse, y codazo por aquí,
codazo por allí, Lopito y yo pudimos aproximarnos bastante a la puerta.

El poeta y Cuarta y Media estaban en primera fila. El segundo de estos
personajes se volvió a mí, y me dijo con gozo:

---Creo que no saldrá vivo de manos del pueblo.

---¿Y a Vd. qué le ha hecho ese caballero?---le pregunté.

---¡Oh!---me contestó.---Ese hombre es un infame, un pícaro que se ha
hecho rico a costa del reino. Yo le aborrezco, le detesto: yo soy una
víctima de sus picardías. Ha de saber Vd. que la tienda de calderería
que tengo me la puso él, por ser yo hijo de la que le lavaba la
ropa\ldots{} Al año de tener la tienda me arruiné, y él me dio unos
cuartos para seguir adelante; pero como le pidiese un destino donde con
descanso y sin trabajar me ganase la vida, tuvo la poca vergüenza de
contestarme que yo no debía ser empleado sino calderero, y añadió que yo
era un animal. Vea Vd., ¡decir que yo soy un animal!

No quise oírle más, y me volví de otro lado. La turba chillaba: no he
podido olvidar nunca aquellos gritos que relaciono siempre con la voz de
los seres más innobles de la creación; y mientras aquel gatazo de mil
voces mayaba, extendía determinadamente su garra con la decisión
irrevocable y parecida al valor que resulta de la superioridad física,
con la fuerte entereza que da el sentirse gato en presencia del ratón.

La tropa contenía al pueblo, anheloso de entrar, y algunos jinetes de la
guardia se colocaron a derecha e izquierda de la puerta. No lejos de
allí, Pujitos, que tenía, como hemos dicho, el instinto, el genio de la
reglamentación del desorden, mandaba a la turba que se pusiese en fila,
y decía alzando su garrote:

---Señores: a un laíto\ldots{} de dos en dos. Formen en batallón, y no
rempujen.

De pronto un clamor inmenso, compuesto de declamaciones groseras, de
torpes dichos, de gritos rencorosos resonó en la calle. En la puerta
había aparecido un hombre de mediana estatura, con el pelo en desorden,
el rostro blanco como el mármol, los ojos hundidos y amoratados, los
brazos caídos, en mangas de camisa y con un capote echado sobre los
hombros. Era el ministro de ayer, el jefe de los ejércitos de mar y
tierra, el árbitro del gobierno, el opulento Príncipe y prócer, señor de
inmensos estados, el amigo íntimo de los Reyes, el dispensador de
gracias, el dueño de España y de los españoles, pues de aquella y de
estos disponía como de hacienda propia; el coloso de la fortuna, el que
de nada se convirtió en todo, y de pobre en millonario, el guardia que a
los veinticinco años subió desde las cuadras de su regimiento al trono
de los Reyes, el conde de Eboramonte y duque de Sueca y duque de la
Alcudia, y Príncipe de la Paz, y Alteza Serenísima que en un día, en un
instante, en un soplo había caído desde la cumbre de su grandeza y poder
al charco de la miseria y de la nulidad más espantosas.

Cuando apareció, mil puños cerrados se extendieron hacia él: los
caballos tuvieron que recular, y los jinetes que hacer uso de sus
sables, para que el cuerpo del Príncipe no desapareciera, arista
devorada por aquel gran fuego del odio humano. El favorito dirigió al
pueblo una mirada que imploraba conmiseración; pero el pueblo que en
tales momentos es siempre una fiera, más se irritaba cuanto más le veía;
sin duda el mayor placer de esa bestia que se llama vulgo, consiste en
ver descender hasta su nivel a los que por mucho tiempo vio a mayor
altura. El piquete de guardias de a caballo trató de conducir al
Príncipe al cuartel, para lo cual fue preciso que él se colocase entre
dos caballos, apoyando sus brazos en los arzones, y siguiendo el paso de
aquellos, que si al principio era lento, después fue muy acelerado con
objeto de terminar pronto tan fatal viacrucis. Entre tanto la multitud
pugnaba por apartar los caballos; por aquí se alargaba un brazo, por
allí una pierna; los garrotes se blandían bajo la barriga de los
corceles, y las piedras llovían por encima. Tanto menudeaban estas, que
los jinetes empezaron a amoscarse y repartieron algunos linternazos.

Lopito, ebrio de gozo me dijo:

---He sido más listo que todos, porque me escurrí por entre las patas de
los caballos, y le pinché con mi navaja. Mírala: entavía tiene sangre.

Cuarta y Media vociferaba diciendo:

---Es una iniquidad lo que hacen con nosotros. Esos guardias debían ser
fusilados. ¿Por qué no nos dejan acercar?

Pujitos, que en su petulancia no carecía de generosidad, fue el único de
los por mí conocidos, en quien advertí señales de compasión.

Hubo momentos angustiosos en que la turba se arremolinaba estrechándose,
y parecía próxima a devorar al prisionero y a los jinetes que le
custodiaban; pero estos sabían abrirse paso, y aclarándose el grupo
volvía a aparecer la cara del mártir, asido con convulsas manos a los
arzones, cerrados sus ojos, la frente herida y cubierta de sangre, las
piernas flojas y trémulas, llevado casi en volandas y casi arrastrando,
con la respiración jadeante, la boca espumosa, las ropas desgarradas.
Parecíame mentira que fuese aquel el mismo hombre que dos días antes me
recibió en su palacio, el mismo a quien vi asediado por los
pretendientes, agitado y receloso sin duda, pero seguro aún de su poder,
y muy ajeno a aquella tan repentina y traidora y alevosa mudanza del
destino\ldots{} ¡Y los chicos más desarrapados se aventuraban entre los
pies de las cabalgaduras para golpearle, y las mujeres le arrojaban el
fango de las calles, menos repugnante que las exclamaciones de los
hombres\ldots{} y estos no disparaban sus escopetas por temor de herir a
los soldados! No creo que haya ocurrido jamás caída tan degradante. Sin
duda está escrito que la caída sea tan ignominiosa como la elevación.

Los favoritos que dejaron su cabeza sobre el tajo de un cadalso, fueron
sin disputa menos mártires que D. Manuel Godoy, llevado en vergonzosa
procesión entre feroces risas y torpes dicharachos, sin morir, porque no
matan los arañazos y pellizcos.

\hypertarget{xii}{%
\chapter{XII}\label{xii}}

Al fin entró en el cuartel la comitiva, y el populacho, azuzado sin
cesar por los lacayos palaciegos, tuvo el sentimiento de no poder
mostrar su heroísmo con el éxito que deseara. Alguno de los más celosos
entre tan bravos campeones salió malherido a consecuencia de que todas
las piedras lanzadas contra el ministro no seguían la dirección dada por
la mano que las tiraba. Digo esto, porque en el momento en que
Santurrias se encaramaba sobre los hombros de dos palurdos para poder
asestar un golpe certero al infeliz mártir, recibió una peladilla de
arroyo sobre la ceja derecha con tanta fuerza, que el benemérito
sacristán cayó al suelo sin sentido. Al punto los que más cerca
estábamos, Lopito y yo, corrimos en su ayuda, y en unión de otras dos
personas caritativas, llevamos aquel talego a su casa, pues Santurrias
vivía pared por medio con mi buen amigo D. Celestino del Malvar. Luego
que este vio entrar a su subalterno tan mal parado, cruzó las manos y
dijo:

---Castigo de Dios ha sido, por las muchas blasfemias de este hombre y
su abominable complicidad con los enemigos del Estado. No es esta
ocasión de demostrar cólera, sino blandura: aquí estoy yo para curarle y
asistirle, pues prójimo es, aunque un grandísimo bribón. Dejadle ahí
sobre una estera, que yo prepararé las bizmas y el ungüento, con lo cual
quedará como nuevo. Ánimo, amigo Santurrias, ¿estáis encandilado
todavía? ¿Queréis que saque una de aquellas botellas que tanto deseáis?
Tía Gila---añadió dando una llave a la mujer que le servía,---abra Vd.
la alacena y saque al punto una de las que dicen \emph{La Nava, seco},
para ver si con la perspectiva de ella se reanima un tantico este
hombre. Y vosotros, chiquillos---prosiguió dirigiéndose a los cuatro
hijos de Santurrias que exhalaban plañideros hipidos en torno al
desmayado cuerpo de su padre,---no lloréis, que esto no es más que un
rasguño alcanzado por este buen hombre en alguna disputa. No lloréis,
que vuestro padre vive y estará sano dentro de una hora\ldots{} Y si
muriese, yo os prometo que no quedaréis huérfanos, porque aquí me tenéis
a mí, que os he de amparar como un padre. Vamos, chiquillos, aquí no
servís más que de estorbo. Idos a jugar\ldots{} Vaya, para que os
quitéis de en medio, os permito que toquéis un poquito las campanas,
picarones\ldots{} id a la torre; pero no toquéis fuerte, tocad a sermón
o a completas.

Como se levanta la bandada de pájaros, sorprendida por el cazador, así
volaron fuera del cuarto los cuatro muchachos, y un instante después
todas las viejas del pueblo salían a sus puertas y balcones diciéndose
unas a otras:---Señora doña Blasa, esta tarde tenemos sermón y
completas. Buena falta hace, a ver si se acaban pronto estas herejías.

Santurrias, que había perdido mucha sangre, recobró algo tarde el
completo uso de sus eminentes facultades, y al abrir a la luz del día
sus ojos, permaneció como atontado por un buen rato, hasta que fue
devuelta a su lengua el don de la facundia.

---¡Que lo ahorquen!---dijo.---Que nos lo den; que lo echen hacia ca, y
nosotros le enjusticiaremos. Despachemos primero a los guardias de a
caballo y dimpués a él\ldots{} No arrempujar, señores. Darle onde le
duela. Pincha tú por bajo, Agustinillo, que yo con esta almendra le echo
la puntería en metá la nariz. ¡Mil demonios! ¿Quién tira
piedras?\ldots{} ¡Muerto soy!

---No, yerba ruin: vivo estás---dijo D. Celestino aplicándole una venda
a la herida.---Mira esto que he puesto delante. Es una botella de
aquellas que deseabas, borracho: tuya será cuando te pongas bueno, si
prometes no decir disparates.

Después nos preguntó que en qué refriega había acontecido tan funesto
percance, y Lopito y yo, cada cual con distinta manera y estilo, le
contamos lo que había sucedido, el encuentro del Príncipe, su prisión, y
su suplicio por las calles del pueblo.

---Corro allá, voy al instante---exclamó fuera de sí D. Celestino.---Es
mi bienhechor, mi amigo, mi paisano y aun creo que pariente. ¿Cómo he de
desampararle en su desventura?

Quisimos disuadirle de tan peligroso intento; pero él no reparaba en
obstáculos ni menos en el riesgo que corría, haciendo pública
ostentación de sus sentimientos humanitarios en favor del desgraciado
valido. Nada le convencía, y después que dejó a Santurrias muy bien
vendado, y ya algo repuesto de su malestar, tomó el manteo, vistiose a
toda prisa y fue en dirección del cuartel.

---No se exponga Vd.---le decía yo por el camino.---Mire que son unos
bárbaros, y en cuanto Vd. demuestre que es amigo del Príncipe, no
respetarán ni sus canas, ni su traje.

---¡Que me maten!---contestó.---Quiero ver al Príncipe\ldots{} Cuando me
acuerdo de lo que me quería ese buen señor\ldots{} ¡Ah! Gabrielillo: lo
que está pasando es espantoso y clama al cielo. Pase que algunos estén
descontentos de su gobierno; pase que le tengan otros por mal ministro,
aunque yo creo que es el mejor que hemos tenido desde hace mucho tiempo;
se puede perdonar que sus enemigos le quieran derribar y le insulten; se
comprende que dichos enemigos en un momento de coraje le prendan, le
arrastren, le ahorquen; pero hijo, que esto lo hagan los mismos a
quienes ha favorecido tanto, los que sacó de la miseria, los que de
furrieles trocó él en capitanes, y de covachuelos en ministros, los que
han vivido a su arrimo, y han comido sobre sus manteles, y le han
adulado en verso y en prosa\ldots{} ¡ah!, esto no tiene perdón de Dios,
y menos si se considera que se han valido para esto de los mismos
lacayos, cocineros y criados de los infantes\ldots{} Hijo mío, me parece
que veo la corona de España paseada por los patanes y los majos en la
punta de sus innobles garrotes.

Llegamos al cuartel, cuya puerta estaba bloqueada por el populacho, D.
Celestino se abrió paso difícilmente. Algunos preguntaron con
sorna:---«¿Adónde va el padrito?» y él, dando codazos a diestra y
siniestra, repetía:---«Quiero ver a ese desgraciado, mi amigo y
bienhechor.»

Muy mal recibidas fueron estas palabras; pero al fin más que la exaltada
pasión pudo el tradicional respeto que al pueblo español infundían los
sacerdotes.

---Hijos míos---les decía:---sed caritativos; no seáis crueles ni aun
con vuestros enemigos.

La turba se amansó, y D. Celestino pudo abrirse calle por entre dos
filas de garrotes, navajas, escopetas, sables y puños vigorosos, que se
apartaban para darle paso. Yo estaba muy asustado viéndole entre aquella
gente, y mi viva inquietud no se calmó hasta que le consideré sano y
salvo dentro del cuartel.

Y los cuatro hijos de Santurrias seguían tocando a sermón y completas, y
la iglesia se llenaba de viejas, que al tomar agua bendita se saludaban
diciendo: ---«Creo que aún no ha concluido todo, y que tendremos esta
tarde otra jaranita.» Y el segundo acólito, creyendo que la cosa iba de
veras, encendió el altar y preparó las ropas, y abrió los libros santos.
Y dieron las tres, las tres y media, las cuatro, las cuatro y media y el
cura no aparecía, y las viejas se impacientaban, y el segundo acólito se
volvía loco, y los cuatro hijos de Santurrias seguían tocando.

Y yo fui también a la iglesia, y sentado en un banco reflexioné
detenidamente sobre la inestabilidad de las glorias humanas, hasta que
al fin, observando que la impaciencia de las viejas llegaba a su último
extremo y que empezaban a entablar diálogos pintorescos para matar el
fastidio, salí en busca de mi amigo. Encontrele muy a punto en el
momento en que regresaba del cuartel. Su rostro era cadavérico: su habla
trémula.

---¡Ah Gabriel!---me dijo.---Vengo traspasado de dolor. Allí sobre unas
fétidas pajas, cubierto de sangre y pidiendo a voces la muerte, está el
que ayer gobernaba dos mundos. Ni un alma compasiva se acerca a darle
consuelo. Ayer cien mil soldados le obedecían, y hoy hasta los furrieles
se ríen de su miseria. No creí que todo se pudiera perder tan pronto;
pero ¡ay, hijo!, el hombre es así. Gusta mucho de las caídas, y el día
en que un poderoso de la tierra viene al suelo siempre es un día feliz.

---Sosiéguese Vd.---le dije.---Vd. no recordará que mandó tocar a sermón
y a completas. La iglesia está llena de gente. No hay más remedio sino
subir al púlpito.

---Hablé con él---prosiguió sin hacerme caso.---El corazón se me parte
recordándolo. Desde anteanoche hasta esta mañana estuvo en un desván,
envuelto en un saco de esteras, muerto de hambre y de sed. La horrorosa
calentura le devoraba de tal modo, que prefirió la muerte. Por eso salió
el infeliz. ¡Pobre amigo mío! Yo le dije: «Señor si cada uno de los que
han recibido un beneficio de vuestra alteza, le hubiera echado una gota
de agua en la boca, su sed se habría apagado.» Él me miró con expresión
de agradecimiento, y no dijo nada, pero a mí se me caían las lágrimas.
Todo esto ha sido obra del Príncipe de Asturias y de sus amigos. Bien
claro se ve. Cuando el Príncipe fue de orden de su padre a calmar al
pueblo para que no despedazara al infeliz prisionero, los amotinados le
aclamaban y obedecían. Y esto no ha de parar aquí. Ellos quieren la
abdicación del Rey, y viendo que esto no es fácil de conseguir, tratan
de irritar más al populacho para que D. Carlos coja miedo y suelte la
corona. Ahora pusieron en la puerta del cuartel un coche de colleras,
con lo cual ese bestia de pueblo creyó que el preso iba a ser puesto en
salvo de orden del Rey. ¡Qué fácilmente se engaña a esos desgraciados!
El ardid salió bien, porque la turba destrozó el carruaje, y después ha
corrido hacia palacio dando vivas a Fernando VII.

---Ya me lo explicará Vd. detenidamente---repuse.---Ahora prepárese Vd.
para ir a la iglesia, donde le aguarda una multitud de respetables
señoras.

---¿Qué dices? Si no hay sermón esta tarde\ldots{}

---Vd. mandó a los cuatro muchachos que tocaran a\ldots{}

---¡Es verdad, qué inadvertencia!---dijo muy confundido.---Y están allí
esas buenas señoras, doña Robustiana, doña Gumersinda, doña Nicolasa la
del escribano. ¡Oh! ¿Qué dirá Nicolasa si no predico?

---Es preciso que Vd. haga un esfuerzo.

---Si no tengo ideas, si no sé qué decir. No puedo apartar mi mente del
espectáculo que he visto. ¡Ah! ¡Cuánto me quería! ¡Si vieras cómo me
apretó la mano! Yo lloraba a moco y baba. Si a él se lo debo todo. Él
fue mi amparo, él me dio este beneficio a los catorce años de haberlo
solicitado, enseguida, como quien dice. Y lo mejor es que sin
merecimientos por parte mía\ldots{} No, no puedo predicar\ldots{} estoy
atontado\ldots{} Esos endiablados muchachos todavía no cesan de tocar a
sermón\ldots{} ¡Oh! tendré que hacer un esfuerzo.

D. Celestino, comprendiendo la necesidad de no desairar a sus
feligresas, entró en su iglesia y oró un poco, recogiendo su espíritu.
Después subió al púlpito y predicó un sermón sobre la ingratitud.

Todas las viejas lloraron.

\hypertarget{xiii}{%
\chapter{XIII}\label{xiii}}

Ya era de noche cuando me avisaron que a las diez salía un coche para
Madrid. Resolví partir, y por hacer tiempo hasta que llegase la hora de
la marcha, fui a la taberna. Como en los días anteriores, el gentío era
inmenso, los trajes pintorescos y variados, las voces animadas (aunque
ya enronquecidas por el patriotismo), los gestos elocuentes, las patadas
clásicas, los pellizcos propinados a Mariminguilla infinitos, el vino
más aguado que el día anterior, pues por algo disfruta Aranjuez el
beneficio de dos copiosos ríos.

Lopito y Cuarta y Media me convidaron a beber con demostraciones de
entusiasmo, y el primero de aquellos consecuentes hombres políticos, me
dijo:

---Hoy sí que nos hemos lucido Gabrielillo. Aquí me está diciendo el
Sr.~Cuarta y Media que esta noche ponen al Príncipe de Asturias, de modo
que hemos de ir a darle vivas al balcón.

Pujitos distrajo mi atención, hablándome de que pensaba organizar una
compañía de buenos españoles que desfilaran por delante del palacio en
marcial formación como la tropa, con objeto de hacer ver a los Reyes que
el pueblo sabe dar media vuelta a la izquierda lo mismo que el ejército.
¡Qué predestinación! ¡Qué genio! ¡Qué mirada al porvenir! Yo contesté a
Pujitos, excusándome de formar parte de tan brillante ejército, por
serme indispensable marchar del Sitio aquella misma noche.

Había oscurecido. Mariminguilla colgó el candil de cuatro mecheros para
la completa aunque pálida iluminación de la escena, y aún me encontraba
yo allí, cuando llegó la feliz, la anhelada noticia. Algunos entraron
diciéndolo, y no se les dio crédito: otros salieron a averiguarlo y
tornaron al poco rato confirmando tan fausto suceso; y por fin un grupo,
el más bullicioso, el más maleante, el más entrometido de todos los
grupos de aquellos días, la comparsa de los cocineros vestidos de
patanes manchegos, y de pinches convertidos en majos, entró anunciando
con patadas, manoplazos, berridos y coces, que la corona de España había
pasado de las sienes del padre a las del hijo. No dejaban de tener razón
al entusiasmarse aquellos angelitos, porque en apariencia ellos lo
habían hecho todo.

Comunicada por tan brillante pléyade la noticia, no podía menos de ser
cierta, y en prueba de que los \emph{patres conscripti} la creyeron,
allí estaban los mil cascos de los vasos rotos en el momento en que se
convencieron del cambio de monarca. También Mariminguilla tenía en sus
brazos señales evidentes del alborozo Fernandista, pues se redoblaron
los pellizcos. La multitud, espoleada por Pujitos, partió a los
alrededores de palacio a pedir que saliese el nuevo Rey para
victorearle, y la taberna quedó desocupada en dos minutos. Pueblo y
soldados, mujeres y chiquillos, todos se unieron al alegre escuadrón: su
paso era marcha y baile y carrera a un mismo tiempo, y su alarido de
gozo me habría aterrado, si hubiese yo sido el príncipe en cuyo loor
entonaban himno tan discorde las gargantas humedecidas por el
fraudulento vino del tío Malayerba.

No quise ver ni oír más aquello, y fui a despedirme del incomparable D.
Celestino, a quien hallé en el cuarto de Santurrias, ocupado aún en
bizmarle y curar sus heridas. Luego que puso fin a esta operación, se
ocupó en acostar a los cuatro muchachos campaneros, los cuales,
fatigados de la batahola de aquel día, yacían medio dormidos sobre el
suelo. Era preciso desnudarles como a cuerpos muertos, y al mismo tiempo
hacerles comer las sopas de ajo que la tía Gila había traído en una gran
cazuela. D. Celestino, teniendo sobre sus rodillas al más pequeño de
aquellos diablillos, le acercaba la cuchara a la boca, esforzándose en
introducirla por entre los apretados dientes. Después, procurando
despabilarle decía:

---Vamos ahora a rezar todos el Padre Nuestro. Si vieras,
Gabrielillo---añadió dirigiéndose a mí,---¡cómo me han mortificado estos
cuatro enemigos! Uno me ponía rabos de papel en la sotana; otro tendía
una cuerda desde la cama a la mesa para que al pasar me enredara las
piernas y cayese al suelo; otro calentó la llave de la alacena y me
abrasé los dedos cuando fui a abrir; y por último, con mi sombrero
hicieron un muñeco que decían era el Príncipe de la Paz, y después de
arrastrarle por el patio, iban a meterle en el fogón para quemarlo.
Afortunadamente, la tía Gila acudió a tiempo. ¡Pero qué han de hacer, si
ya no hay autoridad, ni se obedece a los superiores! Me parece que ahora
van a venir tiempos muy calamitosos. Si cada vez que se les antoje
quitar a un ministro salen gritando los cocheros de los príncipes con
unas cuantas docenas de labriegos y soldados de la guarnición, de
antemano seducidos, vamos a estar con el alma en un hilo. Gabriel, aquí
para entre los dos, ¿no es indecoroso y humillante, e indigno que un
Príncipe de Asturias arranque la corona de las sienes de su padre,
amedrentándole con los ladridos de torpes lacayos, de ignorantes
patanes, de bárbaros chisperos y de una soldadesca estúpida y sobornada?
¡Ay! Si yo no fuera un hombre corto de genio, y lo hubiera tenido para
decirle al Príncipe de la Paz lo que se fraguaba; si él, siguiendo mis
consejos hubiera puesto a la sombra a tres o cuatro pícaros como
Santurrias y otros\ldots{} Porque, créelo hijo, este borrachón es, según
me han dicho, el que ha embaucado a medio pueblo para hacerle tomar
parte en el alboroto\ldots{} por supuesto, que ha corrido dinero de
largo. Yo de buena gana castigaría a este hombre execrable a este
pérfido sacristán; ¿pero cómo he de dejar sin pan a un viudo con cuatro
hijos? Ya ves: se me parte el corazón al considerar que estos angelitos
andarán por las calles pidiendo una limosna\ldots{} Lo que antes te he
dicho es cierto\ldots{} El vulgo, esa turba que pide las cosas sin saber
lo que pide, y grita viva esto y lo otro, sin haber estudiado la
cartilla, es una calamidad de las naciones, y yo a ser rey, haría
siempre lo contrario de lo que el vulgo quiere. La mejor cosa hecha por
el vulgo resulta mala. Por eso repito yo siempre con el gran latino:
\emph{Odi profanum vulgus et arceo\ldots{} et arceo}, y lo
aparto\ldots{} \emph{et arceo}, y lo echo lejos de mí\ldots{} et arceo,
y no quiero nada con él.

Concluida esta filípica, me abrazó deseándome mil felicidades, y
haciéndome jurar que le enteraría puntualmente de la situación de Inés.
Salí al fin de su casa y del pueblo, y cuando el coche que me conducía
pasó por la plaza de San Antonio, sentí la algazara del pueblo agolpado
delante de palacio. Sus gritos formaban un clamor estrepitoso que hacía
enmudecer de estupor a las ranas de los estanques y asustaba a los
grillos, pues unas y otros desconocían aquella monstruosidad sonora que
tan de improviso les había quitado la palabra.

El pueblo victoreaba al nuevo Rey: el plan concebido en las antecámaras
de palacio había sido puesto en ejecución con el éxito más lisonjero.
Todo estaba hecho, y los cortesanos que desde los balcones contemplaban
con desprecio el entusiasmo de la fiera, tan brutal en su odio como en
su alegría, no cabían en sí de satisfacción, creyendo haber realizado un
gran prodigio. En su ignorancia y necedad no se les alcanzaba que habían
envilecido el trono, haciendo creer a Napoleón que una nación donde
príncipes y reyes jugaban la corona a cara y cruz sobre la capa rota del
populacho, no podía ser inexpugnable.

Hasta que nuestro coche no se internó mucho por la calle Larga no
dejamos de oír los gritos. Aquel fue el primer motín que he presenciado
en mi vida, y a pesar de mis pocos años entonces, tengo la satisfacción
de no haber simpatizado con él. Después he visto muchos, casi todos
puestos en ejecución con los mismos elementos que aquel famosísimo,
primera página del libro de nuestros trastornos contemporáneos; y es
preciso confesar que sin estos divertimientos periódicos, que cuestan
mucha sangre y no poco dinero, la historia moderna de la heroica España
sería esencialmente fastidiosa.

Pasan años y más años: las revoluciones se suceden, hechas en comandita
por los grandes hombres, y por el vulgo, sin que todo lo demás que
existe en medio de estas dos extremidades se tome el trabajo de hacer
sentir su existencia. Así lo digo yo hoy, a los ochenta y dos años de mi
edad, a varios amigos que nos reunimos en el café de Pombo, y oigo con
satisfacción que ellos piensan lo mismo que yo, don Antero, progresista
blindado, cuenta la picardía de O'Donnell el 56; D. Buenaventura
Luchana, progresista fósil, hace depender todos los males de España de
la caída de Espartero el 43; D. Aniceto Burguillos, que fue de la
Guardia Real en tiempo de María Cristina, se lamenta de la caída del
Estatuto. Reúnense junto a nuestra mesa algunos jóvenes estudiantes,
varios capitanes y tenientes de infantería, y no pocos parásitos de esos
que pueblan los cafés, probándonos que son tan pesados de pretendientes
como de cesantes. Todos nos ruegan que les contemos algo de las
felicidades pasadas para edificación de la edad presente, y sin hacerse
de rogar cuenta D. Antero la del 56, D. Buenaventura se conmueve un poco
y relata la del 43, D. Aniceto da doce puñetazos sobre la mesa, mientras
narra la del 36, y yo mojando un terroncito de azúcar y chupándomelo
después, les digo con este tonillo zumbón que no puedo remediar: «Vds.
han visto muchas cosas buenas; ustedes han visto la de los grandes
militares, la de los grandes civiles y la de los sargentos; pero no han
visto la de los lacayos y cocheros, que fue la primera, la primerita y
sin disputa la más salada de todas.»

\hypertarget{xiv}{%
\chapter{XIV}\label{xiv}}

Me siento fatigado; pero es preciso seguir contando. Vds. están
impacientes por saber de Inés: lo conozco, y justo es que no la
olvidemos.

Llegué, pues, a Madrid muy temprano, y después de haber acomodado mi
equipaje en la casa que tenía el honor de albergarme (calle de San José,
número 12, frente al Parque de Monteleón), me arreglé y salí a la calle
resuelto a visitar a Inés en casa de sus tíos. Mas por el camino
ocurriome que no debía presentarme en casa de tales señores sin
informarme primero de su verdadera condición y carácter. Por fortuna, yo
conocía un maestro guarnicionero instalado en la calle de la Zapatería
de Viejo, muy contigua a la de la Sal, y resolví dirigirme a él para
pedir informes del Sr.~Requejo.

Cuando entré por la calle de Postas, mi emoción era violentísima, y
cuando vi la casa en que moraba Inés, me flaqueaban las piernas, porque
toda la vida se me fue de improviso al corazón. La tienda de los
Requejos estaba en la calle de la Sal, esquina a la de Postas, con dos
puertas, una en cada calle. En la muestra, verde, se leía: \emph{Mauro
Requexo}, inscripción pintada con letras amarillas; y de ambos lados de
la entrada, así como del andrajoso toldo, pendían piezas de tela, fajas
de lana, medias de lo mismo, pañuelos de diversos tamaños y colores.
Como la puerta no tenía vidrieras, dirigí con disimulo una mirada al
interior, y vi varias mujeres a quienes mostraba telas un hombre
amarillo y flaco, que era de seguro el mancebo de la lonja. En el fondo
de la tienda había un San Antonio, patrón sin duda de aquel comercio,
con dos velas apagadas, y a la derecha mano del mostrador una como
balaustrada de madera, algo semejante a una reja, detrás de la cual
estaba un hombre en mangas de camisa, y que parecía hacer cuentas en un
libro. Era Requejo: visto al través de los barrotes, parecía un oso en
su jaula.

Aparteme de la puerta, y alzando la vista observé otra muestra colocada
en la ventana del entresuelo, la cual decía: \emph{Préstamos sobre
alhajas}. En la ventanilla donde campeaba tan consolador llamamiento, no
había flores, ni jaulas de pájaros, sino una multitud de capas, que
respiraban higiénicamente el aire matutino por entre los agujeros de sus
remiendos y apolilladuras. Tras los vidrios pendía una mugrienta
cortineja. Observé que una mano apartó la cortina; vi la mano, luego un
brazo y después una cara. ¡Dios mío! Era Inés. Yo la vi y ella me vio.
Pareciome que sus ojos expresaban no sé si terror o alegría. Aquel rayo
de luz duró un segundo. Cayó la cortinilla y ya no la vi más.

Esto avivó en mí el deseo de entrar. ¿Cómo podían encontrarse en aquella
vivienda las comodidades, los lujos, las riquezas que ponderaban los
Requejos en su visita inolvidable? Para salir de dudas, doblé la
esquina, y molí a preguntas al guarnicionero.

---Ese Requejo---me dijo,---es el bicho de peores trazas que ha venido
al mundo. Está rico; pero ya se ve\ldots{} en casa donde no se come, ¿no
ha de haber dinero? Porque has de saber que en el barrio corre la voz de
que él se alimenta con las carnes de su hermana, y su hermana con las
del mancebo, que por eso está como una vela. ¡Y cuidado si tienen dinero
esas dos ratas!\ldots{} Con la tienda y la casa de préstamos, se han
puesto las botas. Verdad que por las prendas de vestir no dan más que la
cuarta parte de su valor, con interés de dos pesetas en duro por cada
mes. Cuando toman sábanas finas y vajillas dan una onza, con interés de
cuatro duros al mes. En la tienda dan al fiado a los vendedores que van
por los pueblos; pero les cobran cuatro pesetas y media por cada duro
que venden. Dicen que cuando doña Restituta entra en la iglesia, roba
los cabos de vela para alumbrarse de noche, y cuando va a la plaza, que
es cada tercer día, compra una cabeza de carnero y sebo del mismo
animal, con lo cual pringa la olla, y con esto y legumbres van viviendo.
Una vez al año van a la botillería, y allí piden dos cafés. Beben un
poquito, y lo demás lo echa ella disimuladamente en un cantarillo que
deja escondido bajo las faldas, cuyo café traen a casa, y echándole agua
lo alargan hasta ocho días. Lo mismo hacen con el chocolate. D. Mauro es
vanidoso y gastaría algo más si su hermana no le tuviera en un puño,
como quien dice. Ella tiene las llaves de todo, y no sale nunca de casa,
por miedo a que les roben; y la casa es bocado apetitoso para los
ladrones, porque se dice que en el sótano está la caja del dinero.

Estas noticias confirmaron la opinión que acerca de los tíos de Inés
había yo formado. La primera pena que sentí al oír el panegírico de los
dos personajes, consistió en la certidumbre de que me sería muy difícil
introducirme y menos trabar amistad con sus dueños. En esto pensaba
tristemente, cuando vino a mi memoria un anuncio que varias veces había
compuesto en la imprenta del \emph{Diario}, el cual decía: «\emph{Se
necesita un mozo de diez y siete a diez y ocho años, que sepa de
cuentas, afeitar, algo de peinar, aunque sólo sea de hombre, y guisar si
se ofreciere. El que tenga estas partes y además buenos informes,
diríjase a la calle de la Sal, esquina a la de Postas, frente a los
peineros, lonja de lencería y pañolería de don Mauro Requexo, donde se
tratará del salario y demás.}.»

Corrí a la imprenta del \emph{Diario} a ver si aún se insertaba aquel
anuncio, y tuve el gusto de saber que los Requejos no habían encontrado
quien les sirviera. Abandoné mi profesión de cajista, y sin consultarlo
con nadie, pues nadie me hubiera comprendido, presenteme en la casa de
la calle de la Sal, declarándome poseedor de las cualidades consignadas
en el anuncio.

Mi único temor consistía en que los Requejos recordasen haberme visto en
Aranjuez, con lo cual recelarían de tomarme a su servicio; pero Dios,
que sin duda protegía mi buena obra, permitió que ni uno ni otro me
reconocieran, y si doña Restituta me miró al pronto con cierta expresión
sospechosa y como diciendo «yo he visto esta cara en alguna parte,» fue
sin duda un fugaz pensamiento que no la decidió a poner obstáculos a mi
admisión.

Cuando entré en la tienda, la primera persona a quien expuse mis
pretensiones fue D. Mauro, el cual dejando un rancio librote donde
escribía torcidos números, se rascó los codos y me dijo:

---Veremos si sirves para el caso. De un mes acá han venido más de
cincuenta; pero piden mucho dinero. Como ahora quieren todos ser
señoritos\ldots{}

Llamada por su hermano, presentose doña Restituta, y entonces fue cuando
me miró como más arriba he dicho.

---¿Tú sabes---me preguntó la tía de Inés,---lo que damos aquí al mozo?
Pues damos la \emph{mantención} y doce reales al mes. En otras partes
dan mucho menos, sí señor, pues en casa de Cobos, después de matarles de
hambre, danles ocho reales y gracias. Con que muchacho, ¿te quedas?

Yo fingí que me parecía poco, hasta intenté regatear para que no se
descubriera mi propósito, y al fin dije, que hallándome sin acomodo,
aceptaba lo que me ofrecían. En cuanto a los informes que me exigieron,
fácil me fue conseguir la merced de una recomendación del regente del
\emph{Diario}.

---Doce reales al mes y la \emph{mantención}---repitió doña Restituta,
creyendo sin duda, vista mi conformidad, que había ofrecido
demasiado.---La \emph{mantención}, sí, que es lo principal.

¡Ay! El lector no conoce aún todo el sarcasmo que allí encerraba la
palabra \emph{mantención}.

---Por supuesto---dijo Requejo,---que aquí se viene a trabajar. Veremos
si sabes tú de todos los menesteres que se necesitan. Y aquí hay que
andar derechito, sí señor; porque sino\ldots{} Mírame a mí: yo era un
\emph{jambrera} lo mismo que tú, y en fin\ldots{} con mi honradez y
mi\ldots{}

---La economía es lo principal---añadió la hermana.---Gabriel, coge la
escoba y barre todo el almacén interior. Después irás a llevar estos
fardos a la posada de la calle del Carnero; luego copiarás las cuentas;
más tarde lavarás la loza de la cocina antes de mondar las patatas, y
así te quedará tiempo para apalear las capas, encender el fuego y
soplarlo, devanar el hilo de la costura, poner los números a las
papeletas, aviar la lamparilla, limpiar el polvo, dar lustre a los
zapatos de mi hermano y todo lo demás que se vaya ofreciendo.

\hypertarget{xv}{%
\chapter{XV}\label{xv}}

Al punto empecé las indicadas operaciones, cuidando de poner en ellas
todo el celo posible para contentar a mis generosos patronos. Debo ante
todo dar a conocer la casa en que me encontraba. La tienda, sin dejar de
ser pequeñísima, era lo más espacioso y claro de aquella triste morada,
uno de los muchos escondrijos en que realizaba sus operaciones el
comercio del Madrid antiguo. La trastienda era almacén y al mismo tiempo
comedor, y los fardos de pañuelos y lanas servían de aparador a la
cacharrería, cuyo brillo se empañaba diariamente con repetidas capas de
polvo. Todos los artículos del comercio estaban allí reunidos y
hacinados con cierto orden. Los Requejos vendían telas de lana y
algodones, a saber: pañuelos del Bearne, género muy común entonces,
percales ingleses, que desafiaban en la frontera portuguesa las aduanas
del bloqueo continental; artículos de lana de las fábricas de Béjar y
Segovia, algunas sederías de Talavera y Toledo; y por último, viendo D.
Mauro que sus negocios iban siempre a pedir de boca, se metió en los
mares de la perfumería, artículo eminentemente lucrativo. Así es, que
además de los géneros citados, había en la trastienda multitud de cajas
que encerraban polvos finos, pomadas y aguas de olor en su variedad
infinita, \emph{verbi gratia}: de lima, tomillo, bergamota, macuba,
clavel, almizcle, lavanda, del Carmen, del cachirulo y otras muchas.
Como el local donde se guardaban todos estos géneros servía de comedor,
ya pueden Vds. figurarse la repugnante mezcolanza de olores,
desprendidos de sustancias tan diversas, como son una pieza de lana
teñida con rubia, un frasco de vinagrillo del príncipe y una cazuela de
migas; pero los Requejos estaban hechos de antiguo a esta repugnante
asociación de olores inarmónicos.

De la trastienda se subía al entresuelo por una escalera que presumo fue
construida por algún sapientísimo maestro de gimnasia, pues no pueden
ustedes figurarse las contorsiones, los dobleces, las planchas, las mil
torturas a que tenía que someterse para subirla el frágil barro de
nuestro cuerpo. Sólo la escurridiza doña Restituta pasaba por aquellos
aéreos escollos sin tropiezo alguno. Subía y bajaba con singular
ligereza; y como por un don especial a ella sola concedido, no se le
sentía el andar; siempre que la veía deslizarse por aquella problemática
escalera, sus pasos no me parecían pasos, sino los ondulantes y
resbaladizos arqueos de una culebra.

Cuando, franqueada la escalera, se llegaba al entresuelo, era preciso
hacer un cálculo matemático para saber qué dirección debía tomarse, pues
el viajero se encontraba en el centro de un pasillo tan oscuro, que ni
en pleno día entraba por él una vergonzante luz. Tentando aquí y allí se
hallaba la puerta de la sala, con ventana a la calle de Postas, y por
cierto que allí no vi ninguna cortina verde con ramos amarillos, sino un
descolorido papel, que en mil jirones se desternillaba de risa sobre las
paredes. Un mostrador negro y muy semejante a las mesillas en que piden
limosna para los ajusticiados los hermanos de la Paz y Caridad, indicaba
que allí estaba el cadalso de la miseria y el altar de la usura.
Efectivamente, un tintero de pluma de ganso, cortada de ocho meses,
servía para extender las papeletas, algunas de las cuales esperaban
sobre la mesa la anhelada víctima. Una cómoda y varios cofres,
resguardados con barrotes, eran Bastilla de las alhajas y Argel de las
ropas finas. Las capas, sábanas y vestidos, estaban en una habitación
inmediata que además tenía la preeminencia de proteger el casto sueño
del amo de la casa.

Además de esta sala había otra con ventana a la calle de la Sal, cuya
elegante pieza no desmerecía de la anterior en lujo ni en exquisitos
muebles, pues su sillería de paja adornada con vistosos festones, y tan
aéreas que cada pieza parecía dispuesta a caer por su lado, no hubieran
hallado compradores en el Rastro. En esta sala estaba el taller. ¿El
taller de qué? Los Requejos tenían tres industrias: la venta, los
préstamos, y la confección de camisas, que en los días a que me refiero
eran cortadas por doña Restituta y cosidas por Inés. Allí estaba Inés
desde las cinco de la mañana hasta las once de la noche, trabajando sin
cesar en beneficio de la sórdida tacañería de sus tíos. Una orden
expresa de doña Restituta le impedía salir de aquel cuarto: no bajaba a
la trastienda sino a la hora de comer; no se le permitía asomarse a la
ventana; no se le permitía cantar ni leer un libro; no se le permitía
distraerse de su obra perenne, ni mencionar a su tío, ni recordar a su
madre, ni hablar de cosa alguna que no fuera la honradez de los
Requejos, y la longanimidad de los Requejos.

Pero sigamos la descripción de la casa. En una habitación interior,
mejor dicho en una caverna, estaba el dormitorio de la tía y la sobrina,
y en el fondo del pasillo y junto a la cocina se abría mi cuarto, el
cual era una vasta pieza como de tres varas de largo por dos de ancho,
con una espaciosísima abertura no menos chica que la palma de mi mano,
por esta claraboya entraban, procedentes del patio medianero, algunos
intrusos rayos de luz, que se marchaban al cuarto de hora después de
pasearse como unos caballeros por la pared de enfrente. Mis muebles eran
un mullido jergón de hoja de maíz, y un cajón vacío que me servía de
pupitre, mesa, silla, cómoda y sofá. Semejante ajuar era para mí en
realidad más que suficiente; y en cuanto a la densa y providencial
lobreguez que envolvía la casa como nube perpetua, me parecía hecha de
encargo para mi objeto.

El entresuelo se comunicaba con la escalera general de la casa, la cual
partía majestuosamente desde la misma puerta de la calle, y en su
grandioso arranque de tres cuartas tenía espacio suficiente para que
fuera matemáticamente imposible que una persona subiese mientras otra se
ocupaba fatigosamente en la tarea de bajar. Por ese túnel ascendente
tenían que introducirse los que iban a empeñar alguna cosa, siendo en
cierto modo simbólico aquel tránsito, y expresión arquitectónica muy
exacta de las angustias del alma miserable en los momentos críticos de
la vida. Bien podía llamarse la escalera de los suspiros.

No debo pasar en silencio que en la casa de los Requejos había cierto
aseo, aunque bien considerado el problema, aquella limpieza era la
limpieza propia de todos los sitios donde no existe nada, \emph{exempli
gratia}, la limpieza de la mesa donde no se come, de la cocina donde no
se guisa, del pasillo donde no se corre, de la sala donde no entran
visitas, la diafanidad del vaso donde no entra más que agua.

Allí no había perros ni gatos, ni animal alguno, si se exceptúan los
ratones, para cuya persecución D. Mauro tenía un gato de hierro, es
decir, una ratonera. Los infelices que caían en ella eran tan flacos,
que bien se conocía estaban alimentados con perfumes. Un perro hubiera
comido mucho: un jilguero habría necesitado más rentas que un obispo:
una codorniz hubiera echado la casa por la ventana: las flores cuestan
caras, y además el agua\ldots{} La fauna y la flora fueron por estas
razones proscritas, y para admirar las obras del Ser Supremo, los
Requejos se recreaban en sí mismos.

Me falta ahora hacerme cargo de otro ser que habitaba la casa durante el
día: me refiero al mancebo.

El cual era un hombre cuajado, quiero decir, que parecía haberse
detenido en un punto de su existencia, renunciando a las
transformaciones progresivas del cuerpo y del alma. Juan de Dios ofrecía
el aspecto de los treinta años, aunque frisaba en los cuarenta. Su cara
amarilla tenía gran semejanza con la de doña Restituta, pero jamás se
notaron en ella las contracciones, los enrojecimientos repentinos,
propios de aquella señora. Era en sus modales lento y acompasado; su
movilidad tenía límites fijos como la de una máquina, y si el método
puede llegar a establecerse de un modo perfecto en los actos del
organismo humano, Juan de Dios había realizado este prodigio. Llegar,
abrir la tienda, barrerla, cortar las plumas, colgar las piezas de tela
en la puerta, recibir al comprador, decirle los precios, regatear
siempre con las mismas palabras, medir y cortar el género, cobrarlo,
contar por las noches el dinero, apartando el oro, la plata y el cobre:
tales eran sus funciones, y tales habían sido por espacio de veinte
años.

Juan de Dios comía en casa de los Requejos, que le trataban como un
hermano. Servíales él con fidelidad incomparable, y si en algo nacido
tenían ellos confianza, era en su mancebo. Cinco años antes de mi
entrada en la casa, la organizadora y genial cabeza de D. Mauro Requejo
concibió un proyecto gigantesco, semejante a esos que de siglo en siglo
transforman la faz del humano linaje. D. Mauro, después de hacer la
cuenta del día, se rascó los codos, diose un golpe en la serena frente,
puso los ojos en blanco, riose con estupidez, y llamando aparte a su
hermana, le dijo:

---¿Sabes lo que estoy pensando? Pues pienso que tú debes casarte con
Juan de Dios.

Es fama que doña Restituta arqueó las cejas, llevose un dedo a la barba,
inclinó hacia el suelo la luminosa mirada y pensó.---Pues sí---continuó
Requejo;---Juan de Dios es trabajador, es ahorrativo, entiende del
comercio, y en cuanto a honradez, creo que, no siendo nosotros, no habrá
en el mundo quien le iguale. Yo no pienso volver a casarme; y si hemos
de tener herederos, no sé cómo nos las vamos a componer.

El mancebo fue enterado del proyecto, y desde entonces se trabó entre
ambos prometidos una comunicación amorosa, de la cual no hablo a mis
lectores porque no puedo figurarme cómo sería, aunque cavilo en ello.
Debieron ellos sin duda, tratar de aquel asunto, como si el matrimonio
no fuera la unión de dos cuerpos. Restituta pensaría en casarse, y Juan
de Dios pensaría en casarse, ambos sin pena ni alegría, de tal modo que
pasados cinco años hablaban del asunto con indiferencia, y dándolo como
cosa cercana. Parecía que no les importaba el rápido paso de los años, y
aquellos seres encerrados en una tienda, sin duda medían la vida por
varas, no considerando que alguna vez llegarían al fin de la pieza.
Ambos novios eran de esos que se aprestan a casarse y se casan al fin,
sin que los hombres, ni Dios, ni el demonio sepan nunca por qué.

\hypertarget{xvi}{%
\chapter{XVI}\label{xvi}}

Por las noches, después de cenar, rezábamos el rosario, que llevaba el
amo de la casa con voz becerrona; y concluida la oración al patrono
bendito, permanecían en la trastienda en plácida tertulia que sólo
duraba hora y media, y a la cual solía concurrir algún antiguo amigo o
vecino cercano. La noche de mi inauguración no se alteró tan santa
costumbre. D. Mauro, su hermana, Juan de Dios, Inés y yo, decíamos el
último \emph{ora pro nobis}, cuando sonó la campanilla del entresuelo y
mandáronme que abriese.

---Es el vecino Lobo---dijo mi ama.

Figúrense mis lectores cuál sería mi confusión cuando al abrir la puerta
encaré con la espantable fisonomía del licenciado de los espejuelos
verdes que había querido prenderme cinco meses antes en el Escorial. El
temor de que me conociera diome gran turbación; pero tuve la suerte de
que el ilustre leguleyo no parara mientes en mi persona. No sé si he
dicho que en mí se estaba verificando la trasformación propia de la
edad, y que un repentino desarrollo había engrosado mi cuerpo y
redondeado mi cara, donde ya me apuntaba ligero bozo. Esta fue la causa
de que el licenciado Lobo no me reconociera, como yo temía.

---Señores---dijo Lobo sentándose en un cajón de medias,---hoy es día de
universal enhorabuena. Ya tenemos a nuestro Rey en el trono. ¿No han
salido ustedes? Pues está Madrid que parece un ascua de oro. ¡Qué
luminarias, qué banderas, qué gentío por esas calles de Dios!

---Nosotros no salimos a ver luminarias---contestó Requejo,---que harto
tenemos que hacer en casa. Ay, Sr.~de Lobo ¡qué trabajo! Aquí no hay
haraganes; y se gana el pan de cada día como Dios manda.

---Loado sea Dios---añadió el leguleyo,---y vivan los hombres ricos como
D. Mauro Requejo, que a fuerza de inteligencia\ldots{}

---La honradez, nada más que la honradez---dijo Requejo rascándose los
codos.

---¡Viva el comercio!---exclamó Lobo;---lo que es la pluma, Sr.~D.
Mauro, no da ni para zapatos. Ahí estoy yo hace veinte y dos años en mi
placita del Consejo y Cámara de Castilla, y Dios sabe que hasta hoy no
he salido de pobre. Mucho romper de zapatos para andar en las
actuaciones y nada más. Lo que hay es que ahora espero que me den una de
las escribanías de Cámara, que harto la merece este cuerpo que se ha de
comer la tierra.

---Como Vd. ha servido al favorito\ldots{}

---No\ldots{} diré a Vd.; yo no me he andado en dibujos, y serví al
gobierno anterior con buena fe y lealtad. Pero amigo, es preciso hacer
algo por este perro garbanzo que tanto cuesta. En cuanto vi que el
generalísimo estaba ya en manos de la Paz y Caridad, he hecho un
memorial al de Asturias, y escrito ocho cartas a D. Juan Escóiquiz para
ver si me cae la escribanía de Cámara. Yo les perseguí cuando la famosa
causa; pero ellos no se acuerdan de eso, y por si se acuerdan ya he
redactado una retractación en forma donde digo que me obligaron a hacer
aquellas actuaciones poniéndome una pistola en el pecho.

---No he visto \emph{jormiguita} como el Sr.~de Lobo.

---¡Y qué entusiasmado está el pueblo español con su nuevo
Rey!---continuó el curial.---Da ganas de llorar, señora doña Restituta.
Ahora salí a llevar a mi Angustias con las niñas a la novena del señor
San José, y después que rezamos el rosario en San Felipe, fuimos a dar
una vuelta por las calles. ¡Ay qué risa! Parece que están quemando la
casa de Godoy, la de su madre y su hermano D. Diego, lo cual está muy
retebién hecho, porque entre los tres han robado tanto que no se ve una
peseta por ningún lado. Después que nos entretuvimos un poco volvimos
allá; ellas se han quedado en el 13 en casa de Corchuelo, y yo me he
venido aquí a charlar un poquito. Pero me había olvidado\ldots{}
Inesita, ¿cómo va? ¿Y Vd., Sr.~D. Juan de Dios?

Inés contestó brevemente al saludo.---Está un poco holgazana---dijo
Restituta mirando con desdén a la huérfana.---Hoy no ha cosido más que
camisa y media, lo cual es un asco.

---Pues me parece bastante.

---¡Ay!, Sr.~de Lobo, no diga Vd. que es bastante. Mi abuela según me
contaba mi madre, echaba en un día la friolera de dos camisas. Pero esta
chica está acostumbrada a la holgazanería; ya se ve\ldots{} su madre no
hacía más que arrastrar el guarda pies por las calles, y la niñita me
andaba todo el día de ceca en meca, aquí te pongo aquí te dejo.

---Pues es preciso trabajar---dijo Requejo,---porque, chiquilla, el
garbanzo y el tocino y el pan y las patatas no caen del cielo, y el que
viene a esta casa a sacar el vientre de mal año no se puede estar mano
sobre mano. Y si no, aprendan todos de mí que me he ganado lo que tengo
ochavo por ochavo, y cuando era mozo, fardo por la mañana, fardo por la
noche, fardo a todas horas, y siempre tan gordo y tan guapote.

---Ella es habilidosilla---afirmó Restituta,---y sabe coser; sólo que le
falta voluntad. No es ya ninguna chiquilla, que tiene sus quince años
cumplidos y ya puede comprender las cosas. A su edad yo gobernaba la
casa de mis padres. Verdad es que como yo había pocas, y me llamaban el
lucero de Santiagomillas.

---Pues yo creo que Inesita es una muchacha que no tiene pero---declaró
benévolamente Lobo.---Y tan calladita, tan modesta, que no se puede
menos de quererla.

---Ya le dije cuando entró aquí---continuó Restituta,---que los tiempos
están muy malos, que no se gana nada, que se vende poco y en lo de
arriba no cae más que miseria. Ella comprenderá que nos hemos echado
encima una carga muy pesada al recogerla, porque\ldots{} ¡si viera Vd.
Sr.~de Lobo, qué miseria había en aquella casa del cura de Aranjuez,
donde estaba mi sobrina! ¡Ay, partía el corazón!

---Pues es preciso que trabaje---dijo D. Mauro.---Mi sobrina es una
muchacha muy buena, y ya he dicho a Vd. cuánto la quiero. Como que al
fin y al cabo para ella ha de ser cuanto hay en esta casa.

---Ya le he dicho---prosiguió Restituta,---que mañana tiene que lavar
toda la ropa de la casa, porque ya que ella está aquí, ¿para qué se ha
de gastar en lavandera? Por supuesto que no ha de dejar la costura; y si
pasa mañana de las veinte varas la echaré en el pañuelo unas gotitas de
agua de bergamota, de la de los frascos averiados. Lo bueno que tiene
esta muchacha, Sr.~de Lobo, es que nunca da malas contestaciones. Verdad
que no le faltan luces y harto conoce lo que nos debe, pues ha
encontrado en nosotros su santo Ángel de la guarda. ¡Ah, no puede usted
figurarse la miseria que había en aquella casa del cura de
Aranjuez!\ldots{}

---Le conozco, sí---dijo Lobo enseñando con feroz sonrisa sus dientes
verdes.---Es un pobre hombre que hacía versos latinos al príncipe de la
Paz. Ya se lo dirán de misas. Está probado que ese D. Celestino con su
capita de hombre de bien era el confidente del favorito, y el que le
llevaba la correspondencia con Napoleón, para repartirse a España.

---¡Jesús, qué iniquidad! Bien decía yo que aquel hombre tenía cara de
malo.

---Pero ya le daremos cordelejo---continuó Lobo.---Como la parroquia de
Aranjuez la pretende un primo mío, ya se la tenemos armada a D.
Celestino, y entre yo y un compañero pensamos escribir ocho resmas de
papel sellado para probar que el señor curita es reo de lesa nación.

Mientras esto hablaban yo hacía esfuerzos por contener mi indignación.
Inés, aterrada por la verbosidad de sus tíos, no se atrevía a decir una
palabra. Lo mismo hacía Juan de Dios; pero por un fenómeno singular, las
facciones heladas y quietas del mancebo, indicaban aquella noche que lo
que oía no le era indiferente.

---Así lo haremos---contestó Lobo frotándose las manos.---¿Pero qué hace
ahí tan callado el señor don Juan de Dios? ¡Ay, Restituta, qué marido
tan mudo va Vd. a tener! Y lo que es por palabra de más o por palabra de
menos no armarán Vds. camorra. ¿Y para cuándo dejan Vds. la boda?
Animarse señores, y anímese Vd. también, Sr.~D. Mauro de mis entrañas,
porque mire Vd. que la niñita lo merece. Nada: el mes que entra a la
vicaría. Restituta con mi señor Juan, y Vd. con su querida sobrinita
Inés, que si no me engaño, le ha rezado ya algún padre nuestro a San
Antonio para que esto se realice.

Todas las miradas se dirigieron hacia Inés. Don Mauro estiró los brazos
en cruz, luego cerrando los puños, levantolos hacia arriba como si
quisiera coger el techo, descoyuntose las quijadas, cayeron luego ambas
manos sobre la mesa con estruendosa pesadez, y habló así:

---Yo se lo he dicho ya, y por cierto que la niñita no tuvo a bien
contestarme.

---¿Pues qué quiere decir el silencio en esos casos? ¿Cómo quiere Vd.
que una niña bien criada diga: «Me quiero casar, sí señor, venga
marido?» Al contrario, es ley que hasta el último momento hagan mil
ascos al matrimonio, diciendo que les da vergüenza.

---Ya te dije, hermano---indicó doña Restituta,---que aunque ese es el
destino de la muchacha, si se porta bien y trabaja, no conviene tratar
todavía de tal asunto. Ya sabes lo que son las muchachas, y si les entra
el entusiasmo y el aquel del casorio, no hay quien las aguante. Ella
bien sé yo que se chupará los dedos; pero haces mal en manifestarle tan
pronto tu generosidad, porque puede echarse a perder, pensando todos los
días en el amorcito, en la palabrilla, en el regalito. ¡Ah, bien sabe
ella lo que se hace, la picarona!

Bien sabe que un hombre como tú no lo catan las muchachas de Madrid
todos los días.

---¿Y por qué no he de decírselo desde luego?---contestó Requejo riendo,
es decir, moviendo la tecla de la risa en su brutal organismo.---Mi
sobrina me gusta; y aunque conocemos todos a una porción de señoras muy
principales que me pretenden y se beben los cuatro vientos por mí, yo
dije: «Vale más que todo se quede en casa.» ¿Por qué no se le ha de
decir de una vez que quiero casarme con ella? Bien sé que del alegrón se
estará ocho noches sin dormir y se trastornará toda, y no dará una
puntada; y si fuera por ella, mañana mismo\ldots{} pero váyase lo uno
por lo otro. Pues digo: ¡si ella viera el collar y los pendientes de oro
que tengo apalabrados con el platero del arco de Manguiteros\ldots!

---Dale\ldots{} dale\ldots---dijo Restituta.---¿A qué viene hablar de
esas cosas? ¿A qué sacar de quicio a la muchacha, trastornándole el
seso? Nada: no hay collar ni pendientes. ¿Ni cómo quieres que la niña
lave la ropa ni cosa las camisas, cuando le dicen que va a ser, como si
dijéramos, princesa?

---Nada, nada\ldots{} yo la quiero y la estimo---afirmó Requejo.---¿Por
qué la hemos de privar de ese gusto? Que lo sepa\ldots{} y digo más,
señora hermana; y es que, aunque a mí no me gusta la holgazanería,
porque ya ven Vds., yo desde la edad de catorce años\ldots{} quiero
decir, que aunque no me gusta la holgazanería, lo que es por estos días
y de aquí a que nos casemos, si Inés quiere trabajar que trabaje, y si
no que no trabaje.

D. Mauro volvió a reír, y alargando el brazo hacia Inés le tocó la
barba. Estremeciose la muchacha como al contacto de un animal asqueroso,
y rechazó bruscamente la caricia de su impertinente tío.

---¿Qué es eso, niña? ¿Qué modales son esos?---dijo D. Mauro frunciendo
el ceño---Después que me caso contigo\ldots{}

---¿Conmigo?---exclamó la huérfana sin poder disimular su horror.

---Contigo, sí.

---Déjala, Mauro; ya sabes que es un poco mal criada. Niña, no se
contesta de ese modo.

---¿Pues no tiene también su orgullo la pazpuerca?

---Yo no me caso con Vd., yo no quiero casarme---dijo enérgicamente Inés
recobrando su aplomo, una vez dicha la primera palabra.

---¿Que no?---preguntó Restituta con un chillido de rabia.---Pues,
indinota, mocosa, ¿cuándo has podido tú soñar con tener semejante
marido, un Mauro Requejo, un hombre como mi hermano? ¡Y eso después que
te hemos sacado de la miseria!\ldots{}

---A mí me han sacado Vds. del bienestar y de la felicidad para traerme
a esta miseria, a esta mortificación en que vivo---dijo la huérfana
llorando.---Pero mi tío vendrá por mí, y me marcharé para no volver aquí
ni verles más. ¡Casarme yo con semejante hombre! Prefiero la muerte.

¡Oh!, al oírla me la hubiera comido. Inés estaba sublime. Yo lloraba.

Cuando los Requejos oyeron en boca de su víctima tan absoluta negativa,
se encendió de un modo espantoso la ira de sus protervas almas.
Restituta se quedó lívida, y levantose D. Mauro balbuciendo palabrotas
soeces.

---¿Cómo es eso? ¡Venir a comer mi pan, venir aquí a lavarse la sarna,
venir aquí después de haber andado por los caminos pidiendo
limosna\ldots{} y portarse de esa manera!\ldots{} ¿Pero eres tú una
Requejo, o de qué endiablada casta eres?\ldots{} Cuidado con la señorita
\emph{Panza en trote}. Niñita, ¿sabes tú quién soy yo? ¿Sabes que tengo
cinco dedos en la mano\ldots{} sabes que me llamo Mauro Requejo\ldots{}
sabes que de mí no se ríe ninguna piojosa\ldots{} sabes que a mí no me
pican pulgas de tu laya?\ldots{} Tengamos la fiesta en paz\ldots{} y ten
por sabido que has de hacer lo que yo mando, y nada más.

Diciendo esto, agarró con su mano de hierro el brazo de la muchacha y la
sacudió con mucha fuerza. Quiso poner más alto aún el principio de
autoridad, y lanzó a Inés contra la pared, avanzando sobre ella en
actitud rabiosa. Cuando tal vi pareciome que se me nublaban los ojos, y
sentí saltar mi sangre toda del corazón a la cabeza. Yo estaba en pie
junto a la mesa, y al alcance de mi mano había un cuchillo de punta
afilada. El lector comprenderá aquella situación terrible, y no es
posible que vitupere mi conducta, si es que tales hechos, hijos de la
ciega cólera y la impremeditación, pueden llamarse conducta. ¿Quién al
ver una huérfana inocente e indefensa, maltratada por el más necio y
soez de los hombres, hubiera podido permanecer en calma? Durante aquella
escena de un segundo, alargué la mano hasta tocar la empuñadura del
cuchillo, y con rápida mirada observé el cuerpo deforme de D. Mauro
Requejo; pero afortunadamente para mí y para todos, este, sin duda
aterrado ante la debilidad de la víctima, se contuvo, y no se atrevió a
tocarla. En un movimiento insignificante, en un paso atrás, en una
mirada, en una idea que pasa y huye estriba la perdición de personas
honradas, y un grano de arena hace tropezar nuestro pie, precipitándonos
en el abismo del crimen. Por aquella vez Dios apartó del camino de mi
vida el cadalso o el presidio.

El licenciado Lobo y el mancebo contribuyeron a calmar la enconada
soberbia de su amigo. En el semblante del segundo noté una alteración
vivísima, y su piel amarilla se encendió con inusitado enrojecimiento,
que yo no sabía si atribuir a la indignación o a la vergüenza. Doña
Restituta, queriendo poner fin a una escena que no podía tener buenas
consecuencias, cortó la cuestión, diciendo:

---No te acalores, hermano. Yo la haré entrar en razón. Ya sabes que es
un poco mal criada. Vamos arriba, niña, y ajustaremos cuentas.

Esta fue la orden de retirada. Juan de Dios salió de la tienda para irse
a su casa, y doña Restituta e Inés subieron seguidas por mí, pues
también se me dio la orden de que me acostara. Entraron las dos mujeres
en su cuarto y yo en el mío; mas no pudiendo dominar mi inquietud, y
recelando que en el dormitorio vecino se repetiría entre tía y sobrina
la violenta escena de la trastienda, luego que pasó un rato, salí muy
quedamente de mi escondrijo, y desliceme por el pasillo, conteniendo la
respiración para que no ser sentido. Puesto cerca de la puerta del
dormitorio, sentí la voz de doña Restituta que decía: «No llores,
duérmete. Mi hermano es una persona muy amable; sólo que de
pronto\ldots{} Si él te quiere mucho, niñita\ldots» Esta afabilidad de
la culebra me sorprendió; mas al punto comprendí que debía ser puro
artificio.

También llegaban confusamente a mí las voces de D. Mauro y de Lobo, que
habían quedado en la trastienda. Avancé un poco más hasta llegar a la
escalera, y echándome en tierra apliqué el oído.

---Cuando yo le doy a Vd. mi palabra de que es así---decía el
leguleyo,---Inesita fue abandonada y recogida por doña Juana. Su madre,
que es una de las principales señoras de la corte, desea encontrarla y
protegerla. Yo poseo los papeles con que se puede identificar la
personalidad de la muchacha. De modo que si Vd. se casa con ella\ldots{}
Amiguito, la señora condesa tiene los mejores olivares de Jaén, las
mejores yeguadas de Córdoba, los mejores prados del Jarama, y más de
treinta mil fanegadas de pan en tierra de Olmedo y de D. Benito, sin
herederos directos que se lo disputen a esa barbilinda que hace poco
estaba haciendo pucheros aquí mismo.

---Pero ya Vd. la ha visto---dijo D. Mauro midiendo con grandes zancadas
el piso de la trastienda.---La muchacha es un puerco-espín. Le hago una
caricia y me da una manotada; le digo que la quiero y me escupe la cara.

---Amigo D. Mauro---repuso el licenciado,---el sistema que Vds. siguen
no es el más a propósito para hacerse querer de la niña. Vds. debían
traerla en palmitas, y la están maltratando haciéndola trabajar hasta
que reviente. ¿A quién se le ocurre que una princesita como esta friegue
los platos y lave la ropa? Por este camino aborrecerá a mi señor don
Mauro como si fuera el demonio.

---Pues me parece---dijo mi amo dándose un golpe en la majestuosa
cerviz,---que el señor licenciado tiene muchísima razón. Eso mismo dije
yo a mi hermana; pero como Restituta es tan ambiciosa, que se dejaría
desollar por un ochavo, ha dado en sacarle el cuero a la muchacha. ¿No
somos ricos Sr.~Lobo? Pues si somos ricos ¿a qué viene el descajillarse
por un maravedí? Pero con mi hermana no hay quien pueda. ¿Le parece a
Vd.? Aquí vivimos como en el hospicio: mi padre se llama hogaza y yo me
muero de hambre, como dijo el otro. Pues digo que ha de ser lo que yo
mando, y mi hermana que se case con Juan de Dios y se lleve lo
suyo\ldots{} Y nada más. Inesita no trabajará más, porque si se me
muere\ldots{}

---Además---dijo Lobo;---procure Vd. ser amable con ella. Cuide algo más
de lo exterior, y no se le presente con esa facha de mozo de cordel,
porque las niñas son niñas, Sr.~D. Mauro, y no se entra en el templo del
amor sino por la puerta del buen parecer.

---Eso está muy bien parlado. Si fuera por mí\ldots{} Yo quiero vestirme
bien, pero esa langostilla de Restituta no me deja, y dice que no me he
de poner el traje bonito más que el día de \emph{San Corpus Christi}.
Nada, nada; aquí mando yo; me pondré guapote, porque yo\ldots{} a Dios
gracias, no soy de esos que necesitan afeites y menjurjes para parecer
bien, y cuanto me cae encima está que ni pintado. Trataré a Inesita como
ella se merece, y Dios por delante. Antes de un mes la llevo a la
parroquia.---Ese es el mejor sistema, Sr.~D. Mauro. Con las amenazas,
con el encierro, con las privaciones, con el trabajo excesivo no
conseguirán Vds. sino que la muchacha les odie, y se enamorisque del
primer pelafustán que pase por la calle.

Así hablaron el comerciante y el leguleyo. Despidiéronse después, y el
segundo salió a la calle por la tienda. Retireme a toda prisa; pero
aunque no hice ruido, doña Restituta, con su sutilísimo órgano auditivo
debió sentir no sé si mi aliento o el ligero rumor de un ladrillo roto
que se movió bajo mis pisadas. Esto produjo cierta alarma en su
vigilante espíritu, y saliendo al encuentro de su hermano que subía, le
dijo:

---Me parece que he sentido ruido. ¿Tendremos ladroncitos? Anoche
hicieron un robo en la calle Imperial, metiéndose por los tejados.

Registraron toda la casa, mientras yo, metido entre mis sábanas, fingía
dormir como un talego. Al fin convencidos de que no había ladrones se
acostaron. Mucho más tarde advertí que doña Restituta registraba la casa
segunda vez, hasta que todo quedó en silencio. Cerca ya de la madrugada
oí ruido de monedas. Era doña Restituta contando su dinero. Después la
sentí salir de su cuarto, bajar a la trastienda y de allí al sótano,
donde estuvo más de una hora.

\hypertarget{xvii}{%
\chapter{XVII}\label{xvii}}

Al siguiente día D. Mauro se desvivió obsequiando a su sobrina; pero tan
ramplonamente lo hacía, que cada una de sus finezas era una gansada y
cada movimiento una coz.

---Restituta---decía,---no quiero que trabaje la muchacha. ¿Óyeslo,
hermana? Inés es mi sobrinita, y todo es para ella. Si hace falta coser,
aquí tengo yo mi dinero para pagar costureras. Sácame el vestido nuevo,
que me lo quiero poner todos los días, y quiero estar en la tienda con
él\ldots{} y no me pongas más olla con cabezas de carnero, sino que
quiero carne de vaca para mí y para este angelito de mi sobrina\ldots{}
y lo que es el collar que tengo apalabrado lo compro hoy mismo\ldots{} y
aquí no manda nadie más que yo\ldots{} y voy a traer un fortepiano para
que Inés aprenda a tocar\ldots{} y la voy a llevar en coche a la
Florida\ldots{} y si entra mañana el nuevo Rey, como dicen, hemos de ir
todos a verle, y yo con mi vestido nuevo y mi sobrinita agarrada del
brazo ¿no verdá, prenda?

Restituta quiso protestar contra estos despilfarros, pero amoscose su
hermano, y no hubo más remedio que obedecer, aunque a regañadientes.
Merced a la enérgica resolución del amo de la casa, viose la trastienda
honrada con inusitados y allí nunca vistos platos, aunque doña
Restituta, firme en su adhesión al antiguo régimen, no probó de ninguno.

---Hermana---le decía D. Mauro,---ya estoy de miserias hasta aquí. Nada,
no más trabajar. ¿Ves esta gallina, Inesilla? Pues te la tienes que
comer toda sin dejar ni una tripa, que para eso la he comprado con mi
dinero. Y aquí te tengo un guardapiés de raso verde con eses de
terciopelo amarillo que te has de poner mañana si vamos a ver entrar al
Rey\ldots{} Y también te pondrás unos zapatos azules y unas mediecitas
encarnadas con rayas negras, y también le tengo echado el ojo a una
escofieta que lo menos tiene catorce varas de cinta de varios
colores\ldots{} Conque a ponerse guapa\ldots{} porque lo mando yo.

---Buenas cosas le estás enseñando a la niña---dijo doña Restituta
dirigiendo oblicuamente los ojos a las prendas indicadas, que acababan
de traer a la tienda.

En efecto, señores, la generosidad de D. Mauro era tan bestial como su
tacañería y salvajismo; así es que su empeño en que Inés se vistiera con
tan chabacano y ridículo traje, fue uno de los mayores tormentos que
padeció la huérfana durante su encierro.

---Esta tarde---continuó el tío,---voy a traer dos ciegos para que
toquen, y puedas bailar cuanto quieras, Inesilla. Yo quiero que bailes
lo menos tres horas seguidas, y así has de hacerlo, porque yo lo
mando\ldots{} y aquellos pendientes de a cuarta que están arriba, y son
nuestros, porque no han venido a desempeñarlos, te los pondrás en tus
lindas orejitas.

---Sí, para ella estaban---dijo con avinagrado gesto Restituta.---¡Dos
pendientes de filigrana de oro, largos como badajos de campana, y que
pertenecieron a una camarista de la reina doña Isabel de Farnesio!
Hermano, tengamos la fiesta en paz.

---Aquí no manda nadie más que yo---manifestó Requejo haciendo retemblar
de un puñetazo el cajón que servía de mesa.

Como es de suponer, Inés se resistió a ponerse los vestidos de sainete
comprados por D. Mauro, lo cual puso de mal humor al buen comerciante,
quien no tuvo sosiego durante todo aquel día, y se quitó y puso
repetidas veces el traje nuevo, jurando que en su casa nadie mandaba más
que él.

Al lector habrá sorprendido una circunstancia, y es que en tres días que
llevaba yo de permanencia en la funesta casa, no pudiese ni una vez tan
sólo hablar con Inés. La suspicacia del ama era tan atroz y tan
previsora, que siempre que bajaba del entresuelo a la trastienda, como
no fuera en la hora tristísima de la comida, la dejaba encerrada,
guardando la llave en su profundo bolsillo. Esto me desesperaba,
quitándome toda esperanza de salvar a la pobre huérfana, hasta que un
día, resuelto a comunicarme con ella, aceché la ocasión en que doña
Restituta estaba desplumando a unos infelices en el despacho de los
préstamos, y acercándome a la puerta del encierro, la llamé muy
quedamente. Sentí el roce de su vestido, y su voz me preguntó:

---Gabriel, ¿eres tú?

---Sí, Inesilla de mi corazón. Hablemos un poquito, pero no alces la
voz. Haré mucho ruido con la escoba para que no nos oigan.

---¿Cómo has venido aquí? Di, Gabrielillo, ¿me sacarás tú?

---Reina, aunque aquí hubiera cien mil Requejos y ochocientas mil
Restitutas, te sacaría. No llores ni te apures. Pero di, picarona, ¿me
quieres ahora menos que antes?

---No, Gabriel---me contestó.---Te quiero más, mucho más.

Hice mucho ruido, y di mil besos a la puerta.

---Toca con tus dedos en la puerta para que yo te sienta.

Inés dio algunos golpecitos en la madera, y después me interrogó:

---¿Tardarás mucho en sacarme? Escribe a mi tío para que venga por mí.

---Tu tío no conseguiría nada de estos cafres. Espera y confía en mí.
Chiquilla, hazme el favor de besar la puerta. Inés besó la puerta.

---Yo te sacaré de esta casa, prenda mía, o no soy Gabriel---le
dije.---Haz por no disgustarles. Si te quieren sacar de paseo no te
resistas. ¿Oyes bien? Déjame a mí lo demás. Adiós, que viene la culebra.

---Adiós, Gabriel. Estoy contenta.

Ambos besamos la barrera que nos separaba, y el diálogo acabó, porque
consumado en el despacho de los préstamos el asesinato pecuniario,
salieron las víctimas, y tras ellas, doña Restituta, radiante de
ferocidad avariciosa. En su cara se conocía que había hecho un buen
negocio.

\hypertarget{xviii}{%
\chapter{XVIII}\label{xviii}}

Aquella noche vino a la tertulia de la trastienda, además del Sr.~de
Lobo, doña Ambrosia de los Linos, tendera de la calle del Príncipe, a
quien mis lectores, si no me engaño, tienen el honor de conocer, pues
algo me parece que figuró en los sucesos que conté anteriormente. Su
difunto esposo había sido compañero de D. Mauro en el cargamento y
arrastre de fardos y mercancías, y desde entonces entre ambas familias
quedó establecida cordial amistad. Reconociome doña Ambrosia, mas no
dijo nada que pudiese desfavorecerme en el concepto de mis nuevos amos,
y cuando se hubo sentado, operación no muy fácil, dados su volumen y la
estrechez de los asientos, soltó la sin hueso en estos términos:

---¿Cómo es eso Restituta, cómo es eso D. Mauro, con que no han ido Vds.
a ver la entrada de los franceses? Pues hijos, les aseguro que era cosa
de ver. ¡Qué majos son, válgame el santo Ángel de la Guarda!\ldots{}
¡Pues digo, si da gloria ver tan buenos mozos\ldots{} y son tantos que
parece que no caben en Madrid! Si viera Vd., D. Mauro, unos que andan
vestidos al modo de moros, con calzones como los maragatos, pero hasta
el tobillo, y unos turbantes en la cabeza con un plumacho muy largo. Si
vieras, Restituta, qué bigotazos, qué sables, qué morriones peludos, y
qué entorchados y cruces! Te digo que se me cae la baba\ldots{} Pues a
esos de los turbantes creo que los llaman los \emph{zamacucos}.

También vienen unos que son, según me dijo D. Lino Paniagua, los
\emph{tragones de la Guardia imperial}, y llevan unas corazas como
espejos. Detrás de todos venía el general que los manda, y dicen está
casado con la hermana de Napoleón\ldots{} es ese que llaman el gran
duque de \emph{Murraz} o no sé qué. Es el mozo más guapo que he visto; y
cómo se sonreía el picarón mirando a los balcones de la calle de
Fuencarral. Yo estaba en casa de las primas, y creo que se fijó en mí.
¡Ay hija, qué ojazos! Me puse más encarnada\ldots{} Por ahí andan
pidiendo alojamiento. A mí no me ha tocado ninguno y lo siento: porque
la verdad, hija, esos señores me gustan.

---Gracias a Dios que tenemos rey---dijo D. Mauro.---Y Vd., doña
Ambrosia, ¿ha vendido mucho estos días? Porque lo que es de aquí no ha
salido ni una hilacha.

---En mi casa ni un botón---contestó la tendera.---¡Ay, hijito mío!
Ahora, cuando ese saladísimo rey que tenemos arregle las cosas, hay
esperanzas de hacer algo. ¡Qué tiempos, Restituta, qué tiempos! Pero no
saben Vds. lo mejor, ¿no saben Vds. la gran noticia?

---¿Qué?

---Que mañana hará su entrada triunfal en Madrid el nuevo rey de España,
Sr.~D. Fernando el Sétimo.

---Ya lo sabe hoy todo Madrid.

---Pues no nos quedaremos sin ir a verle; óyelo tú, Restituta, óyelo tú,
Inés---dijo Requejo.---Mañana no se trabaja.

---Yo, primero me aspan que dejar de ir a verlo---afirmó doña
Ambrosia.---Los primos han salido esta noche al camino de Aranjuez para
esperarle. ¡Ay qué alegría, Sr.~D. Mauro! ¡Si viviera mi esposo para
verlo! Él que me decía: «mientras duren este rey y esta reina de tres al
cuarto, no tendremos un gobierno ilustrado.» Mañana va a ser un día de
alegría. Yo tengo un balcón en la calle de Alcalá, y ya hemos encargado
al valenciano media decena de ramos de flores para apedrear a S. M.
cuando pase.

---Nada, lo dicho, dicho---exclamó D. Mauro,---si esta no quiere ir que
se quede en la tienda. Inés me coserá la manga del casaquín que se me
rompió ayer cuando me lo quité\ldots{} Veremos qué tal sabe hacer
Gabriel el coleto\ldots{} Por supuesto, Inesilla, si quieres coger uno
de esos frascos de agua de clavel que tienes a mano derecha, puedes
hacerlo. Todo es para ti.

Así siguió la conversación sin ningún incidente notable en lo sucesivo,
por lo cual la omito, pues supongo al lector poco interesado en conocer
la historia de la enfermedad que padeció el esposo de doña Ambrosia,
trágico acontecimiento que ella refirió. Los únicos personajes siempre
mudos en aquellas tertulias, además de un servidor de ustedes, eran Inés
y el Sr.~Juan de Dios, este último por ser hombre de pocas palabras,
como he dicho.

Llegó el día 24 de marzo, y la cabeza de D. Mauro peinada por mí, salió
a competir con el sol en brillo y hermosura. Doña Restituta, que no pudo
resistir a las súplicas de su hermano, frotose con una toalla el
apergaminado forro de su cara hasta sacarse lustre, y después se puso el
mismo clásico traje con que por primera vez se presentó a mis ojos en
Aranjuez. Por más que D. Mauro atronó la casa, no pudo conseguir que
Inés se disfrazara con el guardapiés verde, las medias encarnadas, las
azules botas y la escofieta, que su vanidoso tío compró para adornar
dignamente a la que consideraba como futura esposa. Negose la muchacha
ser objeto de una fiesta pública, y al fin para decidirla a salir, la
permitieron vestirse con su ropa de luto. Luego que los tres estuvieron
apercibidos, encargaron a Juan de Dios el cuidado de la casa, y don
Mauro me dijo gravemente:

---Gabriel, hoy es día de descanso. Vente con nosotros: con eso me
enderezarás el rabo del coleto si se me tuerce, y me ayudarás a ponerme
los guantes cuando pase S. M., pues hasta ese momento no quiero meter
mis manos en tal Inquisición. ¿Qué te parece? ¿Voy bien? Tira de ese
faldón que está arrugado. Mira, chiquillo, haz el favor de meter
bonitamente tu mano por entre la casaca y la chupa hacia la espalda, y
rascarme en esa paletilla derecha, que no parece sino que se ha juntado
ahí un regimiento de pulgas\ldots{} Así\ldots{} así\ldots{} basta ya.

Dicho esto, y rascado el asno, tomé mi gorra y salimos. ¡Ay Dios mío,
cómo estaba esa Puerta del Sol, y esa calle Mayor y esa calle de Alcalá!
Mis lectores, cualquiera que sea su edad, habrán visto alguna de las
solemnes entradas con que nos obsequia cada pocos años la historia
contemporánea, de modo que para hacerles formar una idea de aquel
gentío, de aquella algazara y de aquel júbilo, me bastará decirles que
lo del 24 de Marzo de 1808, no se diferenció de lo visto en años
posteriores, sino en la exageración del delirio.

De los balcones de las casas nobles pendían las ricas colgaduras de
damasco con su ancho escudo y brillantes flecos, prendas vinculadas que
hasta hace poco han lucido, ya marchitas y mermadas como el patrimonio
de sus dueños, en alguna fiesta del Corpus. Las demás casas se
engalanaban con lo que el entusiasmo de sus inquilinos había encontrado
a mano, siendo considerable la cantidad de piezas de musolineta que un
pueblo loco lanzó al aire de balcón a balcón en aquel memorable día. La
multitud infinita de abanicos con que resguardaban del sol su cara los
millares de damas asomadas a los balcones, ofrecía un aspecto
sorprendente, y cuando la vista recorría panorama tan encantador,
causábale cierto desvanecimiento el incesante ondular de los que se
movían dando aire a sus dueñas. Aquel parlante dije español en tan
inmenso número reproducido, presentando alternativamente al sol una de
sus caras, ya blanca, ya azul, ya roja, y adornado con lentejuelas de
plata y oro, remedaba el aleteo de millares de pájaros pugnando por
levantar el vuelo. Era un día de Marzo de esos que parecen días de
Junio, privilegio de la corte de las Españas, que suele abrasarse en
Febrero y helarse en Mayo. La naturaleza sonreía como la nación.

El abigarrado gentío que poblaba las calles se componía de todas las
clases de la sociedad, abundando principalmente la manolería y
chispería, hombres y mujeres, viejos y muchachos. Los ancianos inválidos
y gotosos habían dejado el lecho, y sostenidos por sus nietos abríanse
paso. Las viejas santurronas que durante tantos años olvidaran todo
camino que no fuera el de sus casas a la cercana iglesia, acudían
también llevadas de la devoción al nuevo Rey, y felicitándose unas a
otras aturdían a los demás con el cotorreo de sus bocas sin dientes. Los
niños no habían asistido a la escuela, ni los jornaleros al trabajo, ni
los frailes al coro, ni los empleados a la covachuela, ni los mendigos a
las puertas de las iglesias, ni las cigarreras a la fábrica, ni los
profesores de las Vistillas dieron clase, ni hubo tertulia en las
boticas, ni meriendas en la pradera del Corregidor, ni jaleo en el
Rastro, ni colisión de carreteros en la calle de Toledo.

La muchedumbre, obligada por su colosal corpulencia a estarse quieta, se
arremolinaba y estremecía como un monstruo atado. Agrietábase a veces
aquella gran masa, pero el surco abierto era invadido por la corriente:
de pronto crecía la aglomeración en un punto y se aclaraba en otro. El
empuje era tremendo, y el retroceso tan peligroso, que había riesgo de
ser hollado por las mil patas de la bestia. El zumbido con que aquel
enjambre manifestaba sus impresiones, trastornaba el cerebro más fuerte:
exclamaciones de alegría, diálogos entusiastas seguidos de abrazos
generosos, gritos de dolor a consecuencia de los callos aplastados, o de
indignación por cada sombrero que perdía su hechura, se unían a las
donosidades de las majas, que arrojaban cáscaras de naranja sobre los
petimetres, y a los lamentos de los mendigos haraposos y mutilados que
escurriéndose entre la multitud, aun allí imploraban la caridad
enseñando una pierna leprosa o una mano deforme.

Nosotros tuvimos que quedarnos en la Puerta del Sol. Una de las
oscilaciones del gentío nos llevó hacia la acera que hoy une las calles
de Espoz y Mina y Carretas; otra oscilación nos arrastró hacia la
Inclusa, que estaba entre las calles del Carmen y de Preciados; y por
último, un nuevo sacudimiento, haciéndonos pasar por ante Mariblanca,
nos encaminó hacia el Buen Suceso, a cuya verja nos agarramos D. Mauro y
yo, para no ser nuevamente arrastrados a merced de aquel oleaje. Yo me
alegraba de que esto sucediera, por si en alguna evolución quedábamos
Inés y yo apartados de los Requejos; pero buen cuidado tenía D. Mauro de
no separarse de la muchacha, y antes le hubiera roto el brazo que
soltarla; tal era la fuerza con que su mano lagartijera tenía
aprisionados los olivares de Jaén y las yeguadas de Córdoba.

Situados donde he dicho, aguardamos la aparición de aquel sol hespérico,
de aquel iris de paz, de aquel príncipe Fernando, que este pueblo, a ser
pagano, hubiera puesto en la jerarquía de sus dioses más queridos. En
rededor nuestro zumbaban algunas viejas.

---¡Ay, mi señora doña Gumersinda!---decía una estantigua.---Dios y mi
patrono San Serapio, ese bendito fraile de la Merced que es abogado
contra los dolores de coyunturas, han querido que yo no mordiera la
tierra sin ver este día.

---¡Ay, mi señora doña María Facunda!---contestaba otra.---Desde que
entró en Madrid al venir de Nápoles el Sr.~D. Carlos III, a quien vi
desde este mismo sitio, no ha habido en Madrid una alegría semejante.
¿Pero Vd. no llora?

---¿Pues no me ve Vd., señora doña Gumersinda? Bendito sea el Señor, que
nos ha permitido ver este día. Al menos se morirá una con la alegría de
que España sea feliz con ese gran Rey que Dios nos ha dado. Pues pocos
rosarios he rezado yo para que esto sucediera. Al fin la Virgen nos ha
oído, y si nosotras no nos estuviéramos en la iglesia rogando día y
noche, ya podía la nación esperar sentada su felicidad.

---¿Pero Vd. no ha visto al príncipe, señora doña María Facunda? Si es
el más rozagante, el más lindo mozo que hay en toda España y sus Indias.
Yo lo vi el día de la jura, y me parece que le tengo delante.---No le he
visto. Ya sabe Vd. señora doña Gumersinda, que desde que reñí con aquel
oficial de walonas que me quería tanto, allá cuando echaron a los
jesuitas, no he vuelto a mirar a la cara a ningún hombre.

---¡Pero oiga Vd., dicen que viene, ya está cerca!

En efecto; se oían las exclamaciones del gentío apelmazado en la calle
de Alcalá, y muchos gritaban: ¡Ya viene por la Cibeles! ¡Ya viene por el
Carmen Descalzo! ¡Ya viene por las Baronesas! ¡Ya viene por los
Cartujos!

Una voz conocida me hizo volver la cara. Pacorro Chinitas, el famoso
amolador, cuyas opiniones no habrán olvidado Vds., estaba detrás de mí
disputando acaloradamente con una mujer del pueblo, gruesa, garbosa, de
ojos vivos, lengua expedita y expeditísimas manos.

---¡Que en todas partes has de meter camorra, condenada mujer!---decía
Chinitas---Vete callando que ya se me sube la mostaza a la nariz.

---No me da gana de callar---contestó la Primorosa, cruzándose en la
cintura las puntas del pañuelo que le cubría los hombros.---¿Pues qué,
estamos en misa? Si ese señorito del tupé no se nos quita
delante\ldots{}

Un petimetre, que olía a jazmín, volvió la compungida cara pidiendo mil
perdones a la emperatriz del Rastro.---¡Eh, tío
\emph{cata-caldos}!---continuó la Primorosa, tirando por los faldones al
currutaco.---¡Quítese de ahí que me estorba!

---Mujer, deja en paz a ese caballero. Mira que la armo.

---¡Sopa sin sal, endino!---exclamó la manola mostrando sus dedos
cuajados de anillos con piedras falsas.---¡Pos pa qué quiero estas cinco
manos de almirez! ¡Enriten a la Primorosa y verán lo güeno! ¡Eh\ldots{}
señor marqués del Barrilete! ---añadió dirigiéndose a D. Mauro,---que me
está Vd. metiendo por los ojos el rabo de su peluquín.

---Mujer---insistió Chinitas,---que donde quiera que vamos me has de
avergonzar\ldots{}

El petimetre se volvió hacia nosotros y dijo, infestándonos con los
perfumes de su ropa:

---No se puede estar donde hay gente ordinaria.

---¿Qué es eso de gente ordinaria?---exclamó la Primorosa atropellando a
los que tenía al lado para abalanzarse hacia el almibarado
joven.---Ya\ldots{} a mí con esas. Pero si es el Sr.~D. Narciso Pluma.
Eh, Nicolasa, Bastiana, Polonia; mira al Sr.~de Pluma, al que la otra
noche le emprestamos dos reales pa osequiar a las madasmas que llevó a
tu casa\ldots{} Señor marquesito de la olla vacía, menos facha y más
comenencia con las señoras, porque yo soy muy reseñorona y muy
requete-usía, y sé dar pa el pelo, y vivan los farolones de Madrid. A
este punto llegaba, cuando un rumor cercano indicó que el príncipe
estaba cerca. La Primorosa, con las majas que la seguían, trató de
atravesar el gentío dando codazos y manotadas a derecha e izquierda.

---Ea, desepártense toos, que viene el sol del mundo. A un lao, a un
laíto señores. Bastiana, Nicolasa, quitaros las flores del pelo, y
vengan acá, que yo se las daré al lucero de las Españas. Míralo allá,
viene a caballo por la Aduana.

A fuerza de empujones la Primorosa logró, ¡cosa inaudita! despejar en
torno suyo un breve espacio, donde campeaba sin obstáculo. Pero
queriendo avanzar más aún, halló insuperable barrera en la persona de un
\emph{majo decente}, que con la capa en cuadril y el sombrero sobre la
ceja, rechazaba varonilmente a cuantos intentaban adelantar hacia el
centro de la carrera.

---¡Cómo!---dijo la maja con centelleante ira.---¿Que no se pasa? ¿Y
quién lo \emph{ice}? Tú, Pujitos. Anda y qué güeno me sabe.

---No se pasa---dijo Pujitos, que se esforzaba en poner a la multitud en
fondo, en filas, en compañías, en batallones y en brigadas.---Póngase ca
una en su puesto, y no ladrar. Orden, señores\ldots{} toos en fila.
Primorosa, las mujeres a sus casas, y aquí denguna me levante el
chillío.

---Pujitos de mi corazón---dijo la Primorosa con terrible ironía,
clavando ambas manos en la cintura.---Si te requiero, si he venido por
verte, si aquí vengo a pedirte de rodillas que me dejes pasar, y traigo
un irgumento pa tu cara de peine viejo. ¿Quieres verlo?\ldots{} Pues
toma.

Aún no lo había dicho, cuando rápida, fuerte y destructora como un
ariete romano, la mano derecha de la maja voló en dirección de la cara
de Pujitos, y el carrillo de este resonó con tremendo chasquido. Una
risotada general fue el himno con que los circunstantes celebraron la
desgracia de Pujitos, el cual, vacilando primero, y desplomado después,
fue a caer sobre un fraile, rompiéndole la escofieta a doña María
Facunda, y la escusabaraja a doña Gumersinda. La multitud hizo un
movimiento: el oleaje corrió de un lado a otro, y Pujitos desapareció
ante nuestra vista como un cuerpo que cae al mar.

La causa de aquel movimiento de la muchedumbre fue una nueva irrupción
de carne humana en aquel recinto estrecho donde ya había tanta. Un
destacamento de la guardia Imperial, con Murat a la cabeza, apareció por
la calle del Arenal. Figuraos un pie que se empeña en entrar en una bota
donde ya hay otro pie. El gran duque de Berg, petulante y vanidoso, se
obstinó en presentarse con sus tropas en la carrera por donde había de
pasar el Rey, lo cual no tenía nada de culpable; pero lo hizo tan
inoportunamente, y sus mamelucos y dragones vejaron de tal modo al
pueblo madrileño, que algunos historiadores hacen datar desde aquella
hora la general antipatía de que los franceses fueron objeto. La
multitud es un río, cuyo nivel no puede subir cuando recibe el caudal de
otro río, y tiene que acomodarse juntando carne con carne y hueso con
hueso, hasta que desaparece la personalidad humana en el informe
conjunto. Esto pasó cuando los franceses penetraron en la estrecha
plaza, y una tempestad de silbidos, reconvenciones e insultos fue la
primera manifestación del pueblo español contra los invasores. Entre
tanto el desconcierto crecía, la sofocación iba en aumento. D. Mauro
bramó como un toro, doña Restituta lanzó un gemido desde el fondo de su
angosto pecho\ldots{} pero la multitud olvidó sus penas, porque ya
estaba cerca, ya venía, ya le veíamos en su caballo blanco, que apenas
podía dar un paso; ya embocaba en la Puerta del Sol, ya se agitaban los
abanicos; llovían ramos de flores; alzábase de la superficie de aquel
inquieto mar un rumor espantoso, cruzaban el aire como pájaros
desbandados millares de gorras, y los brazos convulsos sobresalían de
las cabezas descubiertas; los pañuelos no eran bastante expresivos, y
las capas eran desplegadas como banderas de triunfo.

Entonces la masa de gente que estaba en torno mío avanzó con
irresistible empuje. D. Mauro y Restituta clavaron las uñas en las
mangas del vestido de Inés, que se les escapaba; pero un jirón de tela
se quedó en sus manos e Inés en mis brazos. Miré a la derecha, y vi
entre una aglomeración de cabezas el coleto de D. Mauro y el moño de
doña Restituta, que huían llevados como despojos de naufragio sobre la
espuma de aquel mar alborotado. Estábamos solos.

Inés y yo nos abrazamos y el gentío comprimiéndose después, estrechaba a
Inés contra mí, como si de nuestros dos cuerpos hubiera querido hacer
uno solo.

\hypertarget{xix}{%
\chapter{XIX}\label{xix}}

---Estamos solos, Inés---le dije.---Ahora podremos hablarnos y vernos.

En efecto, estábamos solos. Yo no veía ni Rey ni pueblo, ni guardia
Imperial, ni balcones, ni quitasoles, ni abanicos, ni capas, ni gorras,
ni flores, ni nada: yo no veía más que a Inés, e Inés no veía más que a
mí. Aprisionados entre un pueblo inmenso, nos creíamos en un desierto.
Olvidamos que existía un Rey recién coronado, y una nación alegre, y una
ciudad feliz, y una multitud ebria, y no pensamos más que en nosotros
mismos. No oíamos nada: el clamor de la gente, los vivas, los mueras,
las felicitaciones, aquella borrachera de entusiasmo no producía en
nuestros oídos más impresión que el vuelo de un insignificante insecto.

---Gracias a Dios que nos han dejado solos---dijo Inés estrechándose más
contra mí.

---¡Inés de mi corazón!---dije yo,---cuánto deseaba hablarte. ¡Cuántas
cosas tengo que decirte! Tus tíos se han ido y no volverán, y si vuelven
no estaremos aquí. Somos libres; oye lo que voy a decirte. Estamos fuera
de esa maldita casa, Inés mía, y serás feliz y rica y poderosa y tendrás
todo lo que es tuyo.

---Yo no tengo nada---me contestó.

---Sí: tú no sabes un cuento que yo te voy a contar, un cuento que sé y
que me hace feliz y desgraciado al mismo tiempo.

---¿Qué estás diciendo, loquillo?

---Que tú no eres lo que pareces. Yo te devolveré a tus padres, que son
muy ricos.

---¿Padres? ¿Acaso yo tengo padres?

---Sí: tú no eres hija de doña Juana. Pero esto te lo explicaré en otra
ocasión. ¡Ah!, amiga mía: estoy alegre y estoy triste, porque deseo que
seas feliz, y rica y señora y poderosa y duquesa y princesa; pero al
mismo tiempo considero que cuando llegues al puesto que te corresponde
no me has de querer.

---No entiendo una palabra de lo que dices.

---Ya veremos. Tú no me querrás. ¿Cómo has de querer a un desgraciado
como yo, sin padres, sin fortuna, sin educación? Te avergonzarás de mí,
que soy un criado, un infeliz de las calles\ldots{} pero ¡ay!, no temas,
que yo te llevaré a donde debes estar, y te pondré en tu verdadero
puesto, y serás lo que debes ser. Yo no quiero nada para mí. Dime: ¿me
dejarás que sea tu criado y que viva en tu casa lo mismo que vivo ahora
mismo en la de tus condenados tíos?

---De veras te digo que pareces un loco, Gabriel. Esto me recuerda
cuando tú decías que ibas a ser ministro, generalísimo y príncipe. Yo no
tengo esas ideas.

---No es lo mismo, niñita. Aquello era una necedad mía, y esto es
cierto. Ya no volveremos a casa de los Requejos. Huiremos por la calle
de Alcalá cuando se despeje, buscando refugio en Aranjuez, hasta tanto
que yo te lleve a donde debo llevarte. Aunque sé que no lo has de
cumplir, júrame que me querrás siempre.

---Yo no necesito jurarlo. Prométeme tú no decir disparates---dijo ella,
mientras la presión de la embriagada multitud estrechaba su cabeza
contra mi pecho.

---No son disparates. Pronto te convencerás de ello; ¿pero me querrás
siempre como me quieres ahora? ¿No te avergonzarás de mí, no me
despreciarás? ¿Seré siempre para ti lo mismo que soy ahora, tu único
amigo, tu salvación y tu amparo?

---Siempre, siempre. Al pronunciar estas palabras, Inés sintió que la
cogían un pie.

Miró ella, miré yo, y vimos que clavaba en el pie sus flacos dedos una
mano correspondiente a un brazo negro, que extendiéndose entre las
piernas de los circunstantes, estaba unido al cuerpo de Restituta, quien
estiraba el otro brazo hasta tocar la mano que pertenecía a una de las
extremidades de don Mauro Requejo, el cual D. Mauro Requejo, colocado
como a dos varas de nosotros, pugnaba por abrirse paso entre piernas de
hombre y faldas de mujer, recibiendo aquí una pisada, allá una coz.
Sucedió, que encontrándose los dos hermanos tan separados de nosotros,
perdían el tino buscándonos, y mientras ella se encaramaba anhelando
divisar por algún lado nuestras cabezas, él a causa de su corpulencia
alcanzó a distinguir mi gorro.

Forcejeaban hasta alcanzarnos, cuando doña Restituta cayó al suelo;
diole D. Mauro la mano, y ella alargó la otra para asir el pie de Inés,
temiendo que en un nuevo vaivén o sacudimiento se le escapara. Nuestro
proyecto de fuga quedó frustrado, y ambos Requejos hicieron presa en los
olivares de Jaén, asiéndoles cada uno por un brazo para estar más
seguros.

---¡Pobrecita mía!---dijo D. Mauro.---Creímos que te nos perdías. Si no
es por ti, Gabriel, se nos pierde. A causa del revolcón quedaron ambos
hermanos tan lastimosamente magullados, que daba compasión verles. Del
casaquín de mi amo se habían hecho dos, sin intervención de ningún
sastre, y su hermana veía con ojos furibundos los flotantes jirones de
su vestido negro, rasgado de arriba abajo.

---¿Ves?---decía Restituta a su hermano al regresar a la casa.---¿Ves lo
que sacamos de ir a donde nadie nos llama? Has perdido un guante\ldots{}
¡lástima de guante, que costó un dineral en el Rastro! ¿Pues y la
casaca? Ya tengo costura para tres días\ldots{} ¡Sí, que está barata la
seda!\ldots{} Y tú, niña, ¿has perdido algo? ¡Ay! ¿Dónde está mi
pañuelo? ¿Pues y mi pañuelo? ¡Lo he perdido!\ldots{} ¡Dios me
favorezca!\ldots{} ¡Jesús mil veces! ¡Y yo que le eché tres gotas de
agua de bergamota!

\hypertarget{xx}{%
\chapter{XX}\label{xx}}

Transcurrieron muchos días desde aquel, famoso por la entrada de nuestro
soberano, sin que se alterara con ningún accidente la uniformidad de la
casa de los Requejos.

Largo tiempo estuve sin poder hablar con Inés, aunque vivíamos tan cerca
el uno del otro; pero el encierro en que la guardaba Restituta era cada
vez más inaccesible, y la vigilancia llegó a ser un acecho implacable.
D. Mauro estaba furioso algunas veces, otras triste, y sin duda en su
rudeza no dejaba de comprender que era incapaz de hacerse amar por Inés.
Su cólera no podía menos de derivarse de la conciencia de su brutalidad.
Si no hubiera mediado el ambicioso interés, que era su alma, quizás D.
Mauro habría sido naturalmente afable y hasta cariñoso con la que pasaba
por su sobrina; pero la falta de educación, de delicadeza, de modales y
de sentido común le perdía, haciéndole no sólo aborrecible sino
espantoso a los ojos de la misma a quien deseaba interesar.

Las dificultades para sacar a Inés del poder de los Requejos aumentaban
de día en día con la suspicaz vigilancia de Restituta; pero esto no me
desanimaba, y firme en mi honrado propósito, procuré por todos los
medios posibles conquistar la benevolencia de los dos hermanos,
fingiendo en mí gustos e inclinaciones iguales a las suyas. Yo aspiraba
a una empresa más difícil que las doce de Hércules; aspiraba a
conquistar el inexpugnable castillo de su confianza, donde jamás entrara
persona alguna.

Para llegar a este fin, principié fingiéndome mezquino y avaro, cual si
me consumiera, como a ellos la mísera pasión del ahorro en su último
delirio. Un día después de haber barrido los pasillos y cuartos, me
ocupaba en reunir el polvo y la tierra, recogiendo y guardando aquellos
ingredientes en un gran cucurucho. Como esta operación la hacía yo de
modo que doña Restituta me observase, preguntome un día cuál era mi
objeto, y le contesté:

---Pues qué, señora, ¿se ha de desperdiciar esta sustancia alimenticia?

---¿Cómo? ¿El polvo y la basura de los ladrillos, con las telarañas de
los techos y el lodo de los zapatos forman una sustancia alimenticia?

---Ya lo creo; y me asombra que Vd. no sepa que hay en Madrid un
jardinero francés que compra todo esto para criar unas endemoniadas
yerbas farmacéuticas, que han inventado ahora.

---¿Qué me dices, Gabriel? Pues yo no sabía nada.

---Pues cuando yo estaba en la casa del señor duque de Torregorda, la
señora duquesa vendía esto todas las semanas, y por un paquete así, le
daban sus cuatro cuartos como cuatro soles.

Ella se regocijaba tanto con esto, que cuando yo, después de arrojar a
un muladar el paquete, volvía entregándole los cuatro cuartos de mi
fingida venta, me decía:

---Eres un chico de disposición, Gabriel: no he conocido otro como tú.

También fingía vender los cráneos de carnero que allí se consumían con
frecuencia, los huesos de toda clase de frutas, los pedazos de papel,
los cascos de vidrio, y hasta los pezones de los higos pasados,
diciéndole que un boticario los compraba para hacer cierta droga
venenosa. Cuando llegó el 20 de Abril, y me dieron los diez reales de mi
salario, dije a doña Restituta:

---Señora, ¿para qué quiero yo todo ese dineral? Puesto que tengo todas
mis necesidades satisfechas y no me falta nada, guárdemelo, y si algún
día salgo de esta bendita casa (lo que ojalá no suceda nunca), me lo
entregará junto. Guardadito quiero que esté como oro en paño, y primero
me dejaré cortar las orejas que consentir en el gasto de un maravedí.

---¡Ay, Gabriel!---me contestó rebosando satisfacción,---no he visto
nunca un chico como tú. Bien es verdad que no en vano se pisa esta casa,
donde reinan el orden y la economía. Eres un rapaz de provecho; si
sigues trabajando, a vuelta de diez años tendrás reunidos sesenta duros,
y si siempre persistes en tan buenas ideas, llegarás al fin de tu
vida\ldots{} (pongamos que vives sesenta años más\ldots) con un capital
de 360 duros que tendrás guardaditos y los enterrarás antes de morirte,
para que ningún heredero holgazán se divierta con tu dinero.

Con estas y otras artimañas me hacía querer de mis amos, hasta el punto
de que confiaban mucho en mí; pero a pesar de todo no logré nunca
adquirir la confianza suprema, que consistía para mí en ser encargado de
la custodia de Inés, mientras ellos estaban fuera. ¡Ay!, cuando alguna
vez permitían los hados que doña Restituta se ahuyentara del hogar
doméstico, siempre era depositario de todas las llaves, el impasible, el
mecánico, el glacial mancebo.

Pero he hablado poco de este personaje, cuando en realidad debiera
ocuparnos mucho, y urge dar de él completa idea. Juan de Dios era sin
género de duda un excéntrico, pues también en aquella época había
excéntricos. Un hombre que no habla, que ignora lo que es risa, que no
da un paso más de los necesarios para trasladarse al punto donde están
la pieza de tela que ha de vender, la vara con que la ha de medir, y la
hortera en que ha de guardar el dinero; un hombre que en todas las
ocasiones de la vida parece una máquina cubierta con la humana piel para
remedar mejor nuestra libre, móvil e impresionable naturaleza, ha de
llevar dentro de sí algo ignorado y excepcional. Sin embargo, al poco
tiempo de conocer yo a Juan de Dios, ocurrió algún percance en el
misterioso engranaje de las piezas de aquel mueble animado.

Por aquellos días D. Mauro y doña Restituta habíanse comunicado con
asombro su extrañeza por las frecuentes distracciones de Juan de Dios.
Juan de Dios que en veinte años no se equivocara nunca midiendo o
contando, contaba y medía como un mancebillo recién venido de la
Alcarria. Aún había algo más alarmante. Juan de Dios se paseaba por la
tienda sin hacer nada, lo cual era tan extraordinario como el choque de
un planeta con otro; Juan de Dios preguntaba al parroquiano si quería
\emph{poplín, cotepalis, organdís, madapolanes o muselinetas}, y en vez
de traer lo pedido, daba media vuelta, rascándose la cabeza, iba a la
trastienda, y salía después a preguntar de nuevo, porque se le había
olvidado. Al mismo tiempo Juan de Dios estaba más amarillo y más flaco,
lo cual parecía imposible al que en sus buenos tiempos le hubiese
conocido, y su mirada, siempre mortecina y tristona como la llama de un
candil que se apaga, indicaba últimamente una resignación, un dolor que
no son susceptibles de descripción ni pintura.

Un día salieron los amos, encargándole como de costumbre, la custodia de
la casa. Inés, encerrada en su aposento, habló conmigo como Tisbe al
través del muro, y en mi desesperación, no pudiendo ni verla, ni sacarla
de allí, discurrí que convenía explorar el corazón del mancebo, por si
era posible ablandarle, para que protegiera nuestra fuga. Bajé a la
tienda, y después que hablamos un poco de cosas indiferentes, dije a
Juan de Dios:

---¿No es un dolor, Sr.~D. Juan, que esa muchacha se muera de tristeza
en ese cuartucho? ¿Por qué no la dejan suelta por la casa? ¿Acaso es
alguna fiera?

Advertí en el semblante del mancebo, un como estremecimiento o
vislumbre, después pareció que la poca sangre de su cuerpo se le
agolpaba en la frente, y me habló así:

---Gabriel, tienes razón. ¿Por qué la encierran así siendo tan buena y
tan humilde?\ldots{} Ya estará libre\ldots---dijo Juan de Dios, como
hablando consigo mismo.

Estas palabras despertaron mucho mi curiosidad, y resolví hacerle hablar
sobre el asunto, fingiendo poco interés por la muchacha.

---Verdad es---dije,---que como está tan mal criada\ldots{}

---¡Mal criada!---exclamó el dependiente con viveza.---Tú sí que eres un
mal criado y un bruto. Cuando la veo tan dulce, tan modesta, tan guapa,
me da lástima que\ldots{} Aquí la tratan de un modo que da
compasión\ldots{}

---Pero los amos son muy buenos con ella; la han comprado un vestido, y
D. Mauro quiere que sea su mujer.

Al oírlo Juan de Dios, se inmutó de tal modo, que le tuve miedo.

---¡Casarse con ella!---exclamó.---No, no; eso no puede ser.

---Bien es verdad, que si la muchacha no quiere, ¿por qué han de
obligarla?

---Es verdad. No; no la obligarán.

Comprendí que convenía variar de táctica, demostrando mucho interés por
la prisionera.

---Pues si ella no quiere---dije,---será una obra de caridad sacarla de
aquí.

---¿Tú crees lo mismo?---me preguntó con ansiedad.

---Sí. Me da tanta lástima de la pobrecita, que si en mí consistiera, ya
le hubiera abierto las puertas para que volara como un pajarito.

---Gabriel---me dijo Juan de Dios solemnemente, poniendo su mano sobre
mi brazo---si tú fueras un chico prudente y discreto, yo te confiaría un
proyectillo\ldots{}

No había más remedio que fingir gran indignación contra los Requejos, y
así lo hice, diciendo:

---¡Pues no he de serlo! A mí puede Vd. confiarme lo que quiera, sobre
todo si se refiere a esa niña, porque la tengo compasión, y si mi amo se
empeña en maltratarla, no lo podré aguantar, y el mejor día\ldots{}

---Nuestros patronos son muy crueles---dijo él con la gravedad de quien
revela importante secreto.

---¿Qué dice Vd., crueles? Bárbaros y tacaños, que serían capaces de
vender a Cristo por dos cuartos.

El semblante de Juan de Dios expresó cierto entusiasmo. Después de
vacilar un momento entre la seriedad y una sonrisa, se apretó el corazón
con ambas manos, y me dijo:

---Gabriel, yo estoy enamorado, yo estoy loco.

---¿De quién? ¿Por quién?

---No me lo preguntes, y adivínalo. A ti solo te lo digo: quiero que me
ayudes. Veo que tienes buenos sentimientos, y que aborreces a los
carceleros de Inés. Pero tú no te has fijado bien en ella. ¿No te admira
su resignación, no te admira su modestia? Y sobre todo, Gabriel, ¿has
visto alguna vez mujer más linda? Dime, ¿te ha mirado alguna vez y no te
has vuelto loco?

Juan de Dios lo parecía al decir estas palabras.

---Inés es una gran personita---respondí.---Hace usted bien en quererla,
y mucho mejor en sacarla de aquí. ¿Pero no dicen que se casa Vd. con
doña Restituta?

---¿Yo?, estás loco\ldots{} Antes de ahora he sido tan estúpido que
llegué a creerme capaz de semejante desgracia. Pero ahora\ldots{} ¿Has
conocido mujer más repugnante que esa?

---No, no hay otra que la iguale en toda la tierra. Pero hablemos de
Inés, que es lo que a Vd. le interesa.

---Sí, hablemos. ¡Ay! No sabes qué desahogo siento al confiarte este
secreto. Yo necesitaba decírselo a alguien para no desesperarme. Desde
que Inés entró en esta casa, yo experimenté una sensación desconocida.
Yo había dicho muchas veces: «tanto como oigo hablar del amor, y yo no
sé lo que es\ldots» Pero ya sé lo que es\ldots{} ¡Ay!, he pasado toda mi
vida trabajando como una bestia. Hace veinte años tuve algo con una
mujer que vivía en mi casa; pero aquello no pasó de tres días. Yo nací
en Francia de padres españoles, me crié en un convento y cuando salí de
él a los veinte años, estaba muy persuadido de que las mujeres todas
eran el demonio, pues así me lo decían los padres del convento de
Guetaria. Así es que cuando pasaba alguna cerca de mí, yo bajaba los
ojos, cuidando de no mirarla. Siempre he sido melancólico y\ldots{} no
sé por qué me han disgustado las mujeres\ldots{} Nunca voy a bailes ni a
tertulias, y con tan uniforme vida me he vuelto tan tristón que me
aburro de mí mismo. Los domingos echo un paseo allá por los
Melancólicos, y esto un año y otro, hasta que ahora\ldots{} te contaré
punto por punto. Cuando llegó Inés aquí, me pareció que no era como las
mujeres que yo he visto siempre; quedeme asombrado contemplándola, y
hasta se me figuró que la había visto en alguna parte; ¿dónde?, ¡qué sé
yo!, sin duda dentro de mí mismo. Todo aquel día pensé en ella, y al día
siguiente, que era domingo, me fui después de oír misa, a mi paseo de
los Melancólicos. Allí di mil vueltas figurándome que hablaba con ella,
y fueron tantas las cosas que le dije, que de seguro no cabrían en este
libro grande. Pasó algún tiempo: Inés no me había mirado nunca, hasta
que una noche\ldots{} estábamos comiendo, yo fui a coger un plato, y
como me temblaba la mano, le dejé caer al suelo y se rompió. Restituta
se puso a dar gritos, y D. Mauro me dijo no sé qué barbaridades.
Entonces Inés alzó los ojos y me miró. Cuando esto decía, Juan de Dios
mostraba la incomparable satisfacción del amante que ha recibido favor
muy lisonjero de su dama.

---Pues ánimo---le dije:---la muchacha es linda y buena. Sáquela Vd. de
aquí.

---¡Que si la saco! ¿Pues no la he de sacar?---exclamó con
decisión.---Resuelto estoy a ello. Pero necesito hablarla, Gabriel;
necesito decirle lo que siento por ella. ¿Me corresponderá, crees tú que
me corresponderá?

---Pero tonto, si quiere Vd. hablarla, ¿qué más tiene que ir a su cuarto
y entrar? ¿Los amos no le dejan las llaves?

---Varias veces he intentado hablar con ella; he subido la escalera, he
llegado junto a la puerta y al fin me he vuelto sin valor para decirle:
«Inés, ¿oye usted una palabra?»

---Pues de esa manera no consigue usted nada---le contesté.---¡Ah! Vea
Vd. lo que me ocurre en este instante. Yo me pinto solo para esas
comisiones. Me da Vd. la llave, abro, entro y le digo que Vd. la quiere
y discurre el modo de sacarla de aquí. ¿Qué le parece mi invención?

---Te equivocas si crees que tengo la llave de su cuarto. Todas me las
dejan menos esa.

---Entonces todo está perdido.

---No, porque voy a que un cerrajero me haga una por un modelo de cera,
enteramente igual. Por de pronto, ya que te ofreces a servirme, mira lo
que he pensado. Aquí tengo un ramito de violetas que he comprado esta
mañana. Se lo llevas, arrojándolo dentro por el tragaluz que está sobre
la puerta, y le dices: «esto le manda a Vd. una persona que la ama,»
pero sin mentarle quién es. Luego, otro día que los amos salgan, le
llevas una carta que estoy escribiendo en mi casa, y que tiene ya ocho
pliegos de papel, con una letra como el sol. ¿Lo harás así?

---Todo lo que Vd. me mande.

---¡Ay, Gabriel! Desde que ella está en esta casa, me he vuelto todo del
revés. Pero di: ¿crees tú que Inés me querrá; lo crees tú? ¡Ay!, yo de
veras te digo que por verme amado de ella por todo el día de hoy,
consentiría mañana en perder la vida. Te juro que si supiera de cierto
que no me puede querer, moriría. Si Inés me ama, seré tan feliz
que\ldots{} no sé lo que me pasará. Y tiene que ser, tiene que amarme;
yo me la llevaré a una parte del mundo donde no haya gente, y allí,
solitos los dos, ¿no es verdad que tendrá que quererme? Estoy ahora
averiguando por qué camino se va a una de esas islas desiertas, que
según dicen, hay no sé dónde\ldots{} La sacaré de aquí, Gabriel; nos
iremos ella y yo, si quiere bien, y si no también. Cuando llegue el
caso, me creo capaz de todo; de matar al que quiera impedírmelo, de
vencer cuantas dificultades se me opongan, de echarme a cuestas toda la
tierra y beberme todo el mar, si es preciso para mi fin\ldots{} Gabriel,
¿llevarás a Inés el ramo de violetas? Yo tengo miedo de ir\ldots{}
Cuando le hable una vez se me quitará esta turbación\ldots{} ¿No es
verdad?\ldots{} ¿Crees tú que ella me amará?

La pasión de Juan de Dios tenía cierta ferocidad. Junto con la timidez
más ingenua, el corazón de aquel hombre abrigaba una determinación
impetuosa y una energía suficiente para llevar adelante el más difícil
propósito. El secreto confiado causome tanto asombro como miedo, porque
si bien el amor del mancebo podía ser un gran auxilio para la evasión de
Inés, también podía ser obstáculo. Pensando en esto me separé de él,
para llevar las violetas, sacadas de un cajón donde guardaba sus plumas:
subí y púsome al habla con mi desgraciada amiga.

---Inés---le dije, arrojando el ramillete por el tragaluz,---toma esas
flores que he comprado para ti.

---Gracias---me contestó.

---Niñita mía---continué,---mételas en tu seno, para que la bruja de tu
tía no las descubra. ¿Las has guardado ya?

---En eso estoy---repuso la dulce voz dentro del cuarto.---Vaya, ya
están.

---Mira Inesilla, pon la mano sobre tu corazón y júrame que no has de
querer a nadie, a nadie más que a mí; ni a D. Mauro, ni a Juan
de\ldots{} quiero decir\ldots{} a nadie.---¿Qué estás ahí hablando?

---Júramelo. Pronto estarás libre, paloma. Pero cuando seas señora, rica
y condesa, y tengas palacio y lacayos y tierras, ¿me olvidarás?
¿Despreciarás al pobre Gabriel? Júrame que no me despreciarás.

La prisionera rió en su cárcel.

---Vaya, adiós. Ponte frente al agujero de la llave para verte; ¡qué
guapa estás! Adiós; me parece que ahí están tus simpáticos tíos. Sí: ya
siento la voz del buitre de D. Mauro. Adiós.

\hypertarget{xxi}{%
\chapter{XXI}\label{xxi}}

Aquella noche nos favorecieron doña Ambrosia de los Linos y el
licenciado Lobo. La primera se quejó de no haber vendido ni una vara de
cinta en toda la semana.

---Porque---decía,---la gente anda tan azorada con lo que pasa, que
nadie compra, y el dinero que hay se guarda por temor de que de la noche
a la mañana nos quedemos todos en camisa.

---Pues aquí nada se ha hecho tampoco---dijo Requejo,---y si ahora no
trajera yo entre ceja y ceja un proyecto para quedarme con la contrata
del abastecimiento de las tropas francesas, puede que tuviéramos que
pedir limosna.---¿Y Vd. va a dar de comer a esa gente?---preguntó con
inquietud doña Ambrosia.---¿Por qué no les echa Vd. veneno para que
revienten todos?

---¿Pero no era Vd.---preguntó Lobo,---tan amiga del francés, y decía
que si Murat la miró o no la miró?\ldots{} Vamos, señora doña Ambrosia,
¿ha habido algo con ese caballero?

---¡Ay! Le juro a Vd. por mi salvación que no he vuelto a ver a ese
señor, ni ganas. ¡Demonios de franceses! ¿Pues no salen ahora con que
vuelve a ser Rey mi Sr.~D. Carlos IV, y que el príncipe se queda otra
vez príncipe? Y todo porque así se le antoja al emperadorcillo.

---¡Bah!---dijo Lobo.---Pues ¿a qué ha ido a Burgos nuestro Rey, si no a
que le reconozca Napoleón?

---No ha ido a Burgos, sino a Vitoria, y puede ser que a estas horas me
le tengan en Francia cargado de cadenas. Si lo que quieren es quitarle
la corona. Buen chasco nos hemos llevado, pues cuando creímos que el
Sr.~de Bonaparte venía a arreglarlo todo, resulta que lo echa a perder.
Parece mentira: deseábamos tanto que vinieran esos señores, y ahora si
se los llevara Patillas con dos mil pares de los suyos, nos daríamos con
un canto en los pechos.

---No: que se estén aquí los franceses mil años es lo que yo
deseo---dijo Requejo.---Como me quede con la contrata ¡ay mi señora doña
Ambrosia!, puede ser que el que está dentro de esta camisa salga de
pobre.

---Quite Vd. allá. ¿Ni para qué queremos aquí franceses, ni
\emph{zamacucos}, ni \emph{tragones}, ni nada de toda esa canalla que no
viene aquí más que a comer? Pues ¿qué cree Vd.?, muertos de hambre están
ellos en su tierra, y harto saben los muy pillastres dónde lo hay. Si es
lo que yo he dicho siempre. Dicen que si Napoleón tiene esta intención o
la otra. Lo que tiene es hambre, mucha hambre.

---Yo creo que tenemos franceses por mucho tiempo---afirmó el
licenciado,---porque ahora\ldots{} Luego que nuestro Rey sea reconocido,
vienen acá juntos para marchar después sobre Portugal.

---¡Qué majadería!---exclamó la señora de los Linos.---Aquí nos están
haciendo la gran jugarreta. Esta mañana estuvo en casa a tomarme medida
de unos zapatos, el maestro de obra prima, ese que llaman Pujitos.
Díjome que en el Rastro y en las Vistillas todos están muy alarmados, y
que cuando ven un francés le silban y le arrojan cáscaras de fruta;
díjome también que él está furioso, y que así como fue uno de los
principales para derribar a Godoy, será también ahora el primero en
alzarles el gallo a los franceses\ldots{} ¡Ah!, lo que es Pujitos mete
miedo, y es persona que ha de hacer lo que dice.

---Si me quedo con la contrata, Dios quiera que no se levanten contra
los franceses---dijo Requejo.---Si hay levantamiento---afirmó
Restituta,---y mueren unos cuantos cientos de docenas, esos menos serán
a comer. Siempre son algunas bocas menos, y la contrata no disminuirá
por eso.

---Has pensado como una doctora---dijo D. Mauro.---¿Pero y si se van?

---Se irán cuando nos hayan molido bastante---añadió doña
Ambrosia.---Pues no tienen poca facha esos señores. Van por las calles
dando unos taconazos y metiendo con sus espuelas, sables, carteras,
chacós y demás ferretería, más ruido que una matraca\ldots{} ¡Y cómo
miran a la gente!\ldots{} Parece que se quieren comer los niños
crudos\ldots{} por supuesto que ya les verá Vd. correr el día en que el
español diga: «por ahí me pica, y me quiero rascar.»

---Eso es música---dijo Lobo.---Deje Vd. que vuelvan a Madrid el Rey y
el Emperador, y verá cómo todo se arregla. D. Juan de Escóiquiz, que es
amigo mío, y el primer diplomático de toda la Europa, me dijo antes de
irse, que son unos bobos los que creen que Napoleón intenta destronar al
rey de acá. Descuiden Vds. que como haya dificultades, mi canónigo las
arreglará todas, que para eso le dio el Señor aquel talentazo que
asusta.

---Napoleón no viene acá sino con la espada en la mano---continuó doña
Ambrosia---El padre Salmón de la orden de la Merced, que estuvo esta
mañana en casa (y por cierto que se llevó media docena de huevos como
puños), me dijo que a él no se le escapa nada, y que tendremos guerra
con los franceses. Napoleón nos está engañando como a unos dominguillos.
Ya ve Vd. hace quince días se dijo que venía, y en palacio enseñaban las
botas y el sombrero que había mandado por delante. D. Lino Paniagua que
vio aquellas prendas y las tuvo en su mano, me dijo que las botas eran
grandísimas y casi tan altas como este cuarto. En cuanto al sombrero,
dice que era tan grasiento, que un cochero simón no se le pondría, lo
cual prueba que este emperador es un grandísimo gorrino, con perdón sea
dicho.

---Veinte mil franceses tenemos aquí---dijo don Mauro con expresión
meditabunda---¡Mucho pan, mucho tocino, muchas patatas, mucho pimentón,
mucha sal, mucha berza, han de entrar por veinte y cinco mil bocas! Y
dicen que traen hambre atrasada.

---Por supuesto, hermano---dijo Restituta,---el dinerito por adelantado.

D. Mauro tomó un papel, y con profunda abstracción hizo cuentas.

---¿Y de lo que sobre en el almacén no se podrá traer lo necesario para
el gasto de la casa?---preguntó la digna hermana.---Porque están unos
tiempos ¡ay!, señora doña Ambrosia: no se gana nada\ldots{}

---Vaya, vaya---dijo doña Ambrosia.---Poco, mal y bien quejado. Más
dinero tienen Vds. que las arcas del Tesoro. Y a propósito, Restituta,
¿cuándo se casa Vd.?

---¡Jesús! ¿Quién piensa ahora en eso? No corre prisa.

---No pensará lo mismo Juan de Dios. ¿Y usted, Inesita, cuándo se
decide?

---Ya está decidida---dijo vivamente Restituta.---La pícara harto
disimula su satisfacción. \emph{Este} la tiene muy mimosa.

---Esto está muy bien: una niña bien criada debe hacer ascos al
matrimonio hasta que llegue el momento crítico. Pero hija, con la
conversación se me ha ido el tiempo: son las diez\ldots{} Adiós, adiós.

Fuese doña Ambrosia, desfiló al poco rato Lobo, y habiendo subido a
acostarse las dos mujeres, quedaron solos en la trastienda el patrono y
el mancebo haciendo las cuentas de la contrata.

Yo me acosté y dormí profundamente; pero a eso de la media noche, y
cuando recogido también el amo, reinaban en la casa el sosiego y la
tranquilidad me desvelaron unos agudos gritos, que al punto reconocí
como procedentes de la exprimida laringe de Restituta.

---Sin duda hay ladrones en la casa---dije levantándome.

Restituta llamaba angustiosamente a su hermano, el cual salió con una
tranca, diciendo:

---¡Dónde están esos pícaros, dónde están para que sepan si soy hombre
que se deja quitar el fruto de su honradez!

---No son ladrones---dijo Restituta con voz temblorosa a causa de la
ira;---no son ladrones, sino otra cosa peor.

---¿Pues qué son, con mil pares de diablos?

---Es que\ldots---continuó la hermana, dirigiéndose al amo y a mí, que
también había acudido con un palo.---Inesilla\ldots{} bien decía yo que
esa muchacha nos daría que sentir\ldots{} es una loca, una mujerzuela,
una trapisondista, una perdida de las calles.

---A ver\ldots{} ¿qué ha hecho?

---Pues yo velaba, ella dormía, y de repente empezó a hablar en sueños.
¡Ay, no sé cómo no la estrangulé! Primero pronunció algunas palabras que
no pude entender, después dijo así: «Juro que te querré siempre; juro
que te querré cuando sea condesa, cuando sea princesa, cuando sea rica,
cuando sea gran señora. Pero yo no quiero ser nada de eso sin ti.»
Estuvo callada un rato, y después siguió diciendo: «¡Cómo no he de
quererte! Tú me arrancarás del poder de estas dos fieras\ldots{} ¡Ay!,
adiós: siento la voz del buitre de mi tío. Adiós\ldots» Después la
condenada niña, como si le parecieran poco estos insultos, llevose las
palmas de las manos a su boquirrita, y se dio muchos besos. ¿Qué te
parece, hermano? ¡No sé cómo no la ahogué! Sin poderme contener,
arrojeme sobre ella; despertose despavorida, y al incorporarse se le
cayó del pecho este ramo de violetas.

Al decir esto, Restituta mostraba en su trémula mano la terrible prueba
del delito. Quedose don Mauro aturrullado y confuso, y luego tomando el
ramo y mordiéndolo con rabia lo arrojó al suelo, donde fue pisoteado
\emph{alterno pede} por ambos furiosos hermanos.

---¡Con que dice que soy un buitre!---exclamó él echando chispas.---¡Un
buitre! ¡Llamar buitre a un caballero como yo! ¡Bonito modo de pagar el
pan que le doy! Ya le enseñaré los dientes a esa chiquilla. Pero ese
ramo, ¿quién le ha dado ese ramo?

---Pero Mauro\ldots{}

---Pero Restituta\ldots{}

Y más se confundían los dos cuanto más se irritaban, y crecía su cólera
a medida que aumentaba su aturdimiento, hasta que Requejo, recogiendo
sus luminosas ideas en rápida meditación, dijo:

---Tiene amores con algún mozalbete de las calles. ¿Habrá entrado aquí?
Esto es para volverse loco. Gabriel, Gabriel, ven acá.

Al punto comprendí que estaba en peligro de hacerme sospechoso a mis
feroces amos, y como en este caso me arrojarían de la casa,
imposibilitando de un modo absoluto la realización de mi proyecto, hallé
prudente el desorientarles con una invención ingeniosa, que apartara de
mí toda sospecha.---Señor---dije a mi amo,---estaba esperando a que su
merced acabara de hablar, para decirle alguna cosa que contribuya a
descubrir esta picardía. Pues anoche cuando salí en busca del cuarterón
de higos pasados, me pareció que vi en la calle a un señorito, el cual
señorito miraba a estos balcones\ldots{} y después, creyendo él que yo
no le veía, arrojó una cosa\ldots{}

---¡Eso, eso fue\ldots{} el ramo!---exclamó Requejo.

---Anoche mismo---continué,---pensaba decírselo a su merced; pero como
estaba ahí esa señora, y después se quedaron Vd. y D. Juan de Dios
haciendo números\ldots{}

---¿Y ella se asomó al balcón?---preguntó Restituta.

---Eso no lo puedo asegurar, porque hacía oscuro y no vi bien. Pero
encárguenme mis amos que esté ojo alerta, y no se me escapará nada. A fe
que si Vds. me dieran la comisión de vigilar a la niña cuando salen de
casa, la niña no se reiría de nosotros.

---¡Esto no se puede aguantar!---exclamó fieramente D. Mauro.---Vaya,
acuéstense todos, que mañana le leeré yo la cartilla a la señorita.

Retireme a mi cuarto, y desde mi cama oía al espantoso Requejo, hablando
con su hermana.

---Nada, nada, esta semana me casaré con ella. Si no quiere de grado
será por fuerza\ldots{} Estoy furioso, estoy bramando. Mañana sabrá ella
si soy yo Mauro Requejo, o quién soy. La encerraremos en el sótano, sin
darle de comer. ¿Acaso vale ella el mendrugo de pan con que le matamos
el hambre? Le diremos que no probará bocado, ni beberá gota hasta que no
consienta en ser mi mujer\ldots{} La encerraremos en el sótano, sí
señor, en el sótano. Y si no quiere, palos y más palos. A fe que tengo
yo buena mano de almirez\ldots{} ¡Llamarme buitre esa rapazuela de las
calles!\ldots{} Estoy furioso\ldots{} me la comería\ldots{} Sí: que yo
iba a dejarla escapar con el mozalbete del ramo\ldots{} Se casará, sí,
se casará, y si no, de aquí no sale, sino difunta\ldots{} ¡Buen genio
tengo yo!\ldots{} Malas brujas me chupen, sino la caso conmigo
mismo\ldots{} Y si no quiere por blandas será por duras, la amarraré a
un poste, la azotaré, la abriré en canal con el cuchillo de abrir las
latas de pomada.

Requejo en aquel instante parecía un demonio escapado del infierno; y la
primera luz de la aurora, entrando difícilmente en la oscura casa, le
encontró despierto aún y vociferando como un insensato.

\hypertarget{xxii}{%
\chapter{XXII}\label{xxii}}

Dicho y hecho: desde la mañana del día siguiente, D. Mauro pareció
dispuesto a llevar adelante su bestial propósito, el de precipitar el
martirio de Inés, \emph{casándola consigo mismo}, como él decía en su
bárbaro lenguaje. La táctica de amabilidad y de astuta dulzura,
recomendada por el licenciado Lobo, se consideró inútil, siendo
sustituida por un sistema de terror, que ponía en fecundo ejercicio las
facultades todas de doña Restituta. Antes de partir a la reunión donde
D. Mauro y otros dos comerciantes debían ponerse de acuerdo para la
subasta del abastecimiento, mi amo tuvo el gusto de plantear por sí
mismo el nuevo sistema. Dispuso que Inés no saldría de su cuarto ni para
comer, que los vidrios y maderas de la ventanilla que daba a la calle de
la Sal, se cerraran, asegurándolas por dentro con fuertísimos clavos, y
que se colocara un centinela de vista dentro de la misma pieza, cuya
misión a nadie podía corresponder más propiamente que a Restituta.

Ya no era posible, pues, ni ver a Inés, ni hablarla, ni prevenirla,
porque todo indicaba que aquella tenaz vigilancia no concluiría sino
cuando los Requejos vieran satisfecho su ardiente anhelo de casar a la
muchacha consigo mismos. Por último, llegaron las vejaciones ejercidas
contra Inés hasta el extremo de notificarle enérgicamente que no vería
la luz del sol sino para ir a casa del señor vicario a tomar los dichos.
La situación de Inés era por lo tanto insostenible y tan crítica, que me
decidí a intentar resueltamente y sin esperar más tiempo, su anhelada
libertad. Para hacer algo de provecho, era indispensable aprovechar un
día en que ambas fieras, macho y hembra, salieran a la calle a cualquier
negocio, pues pensar en la fuga mientras nuestros carceleros estuviesen
en la casa, era pensar en lo excusado. D. Mauro, ocupado en su contrata,
salía con frecuencia; pero Restituta, imperturbable como esfinge
faraónica, no se movía de la casa, ni del cuarto, ni de la silla. Para
vencer tan formidable dificultad, discurrí a fuerza de cavilaciones el
siguiente medio.

Mi seductora ama tenía la costumbre, harto lucrativa, de asistir a todas
las almonedas que se anunciaban en el \emph{Diario}, y hacíalo con la
benemérita intención de pescar muebles, colchones, ropas, adornos de
sala y otros objetos, que adquiridos por poco precio, vendía después en
dos o tres prenderías de la calle de Tudescos, que eran de su exclusiva
pertenencia, aunque no lo pareciese. Hacia el 15 de Abril tuvo noticia
de un ajuar completo de ricos muebles puestos en almoneda en una casa de
la plazuela de Afligidos. Habíales ella visto y examinado, y aunque le
parecieron de perlas, no los tomó porque la dueña, que era viuda de un
consejero de Indias, no se resignaba a entregar su única fortuna casi de
balde. Regatearon: Restituta ofreció una cantidad alzada; mas no fue
posible la avenencia, y volviose aquella a su casa sin aflojar los
cordones de la bolsa, aunque harto se le conocía su desconsuelo por
haber dejado escapar negocio de tal importancia. Pues bien, sobre
aquella almoneda, sobre aquel regateo, sobre este desconsuelo, fundé yo
el edificio de la invención que debía quitarme de delante a mi señora
doña Restituta por unas cuantas horas.

Era un domingo, día 1º de Mayo. Salí por la mañana, y dirigiéndome a mi
antigua casa, buscáronme allí una mujer que se encargó de llevar a doña
Restituta el recado que puntualmente le di. Estaba el ama, a las cuatro
de la tarde, sentada en el cuarto de la costura, cuando se presentó mi
comisionada en la casa, diciendo que la señora de la plazuela de
Afligidos consentía en dar los muebles a la señora de la calle de la
Sal, por el precio que esta había tenido el honor de ofrecer.

Dio un salto en su asiento Restituta, y al punto su acalorada
imaginación ilusionose con las pingües ganancias que iba a realizar. Se
vistió con aquella ligereza viperina que le era propia, y después de
cerrar el balcón y la puerta de la habitación de Inés, tuvo la
condescendencia incomparable de entregarme la llave de la puerta que
conducía a la escalerilla principal: encargó a Juan de Dios el mayor
cuidado, y salió.

Cuando la vi salir, respiré con indecible desahogo. Pareciome que huía
para siempre, llevada en alas de vengadores demonios.

Ya no podía perder un instante, y dije a mi amiga desde fuera.

---Inesilla, prepárate. Recoge toda tu ropa, y aguarda un momento.

La única contrariedad consistía ya en que Juan de Dios descubriese mi
intriga, oponiéndose a nuestra fuga; pero yo contaba con la facilidad
que ha existido siempre para cegar por completo a quien ya tiene ante
los ojos la venda del amor. Bajé a la tienda, y ya desde el primer
momento advertí que la fortuna no me era muy favorable, porque Juan de
Dios estaba en conversación con dos militares franceses, y no era
aquella ocasión a propósito para que me diera la llave falsificada que
hacía falta.

Diré brevemente por qué estaban allí los dos franceses. Un oficial de
administración militar fue en busca de mi amo para hablarle de no sé qué
particularidades relativas al contrato de abastecimiento: acompañábale
otro que me parecía teniente de la guardia imperial, el cual, entablada
conversación con Juan de Dios, habló en incorrecto español y dijo que
era del país vasco-francés. Como el hortera había nacido y criádose en
el mismo país, al punto se las echaron los dos de compatriotas, y hubo
apretones de manos. El extranjero era un mozo alto y rubio, de modales
corteses y simpática figura. ---¿No recuerda Vd. la familia Sajous, en
Bayona?---dijo a Juan de Dios.

---¿Pues no la he de recordar? Mi padre, D. Blas Arroiz, estuvo de
escribiente en casa de Mr.~Hipólito Sajous, en Bayona, y después en casa
de otro Sajous en Saint-Sever---repuso Juan de Dios.

---El de Saint-Sever es mi padre---añadió el francés;---pero yo nací en
Puyoo, donde aquel tiene una fábrica de tejidos. Me acuerdo de haber
oído hablar en mi niñez de un administrador guipuzcoano que falleció en
nuestra casa.

A este tenor continuaron hablando un cuarto de hora, hasta que al fin,
después de mutuas felicitaciones y ofrecimientos, despidiose el francés,
prometiendo volver a visitarnos. Yo estaba tan impaciente, que necesité
disimular mi agitación para que no se me conociera en el semblante lo
que traía entre manos. Sin perder tiempo, porque perderlo era perderme,
dije a Juan de Dios:

---Vamos, amigo; este es el momento de entregar a la niña la carta
amorosa que Vd. tiene escrita.

---Sí, chiquillo, aquí está---repuso mostrándome la epístola, que era un
monumento caligráfico.---¿Qué te parece este trabajo? ¿Has visto alguna
vez letra como esta? Repara bien esa M y esa H mayúsculas. ¡Qué rasgos
tan finos! Y esas letras con que pongo su nombre, ¿qué te parecen? Tres
días de tarea eché en ese nombre divino, que como el de Jesús,

\small
\newlength\mlenm
\settowidth\mlenm{más que con la miel y azúcar,}
\begin{center}
\parbox{\mlenm}{endulza el alma y la lengua                 \\
                más que con la miel y azúcar,               \\
                con sólo sus cinco letras.}                 \\
\end{center}
\normalsize

Este no tiene más que cuatro; pero ¡qué perfiles!, y toda la carta está
lo mismo. No tiene más que once pliegos; pero me parece que es bastante.
Como es la primera que le escribo, no debo marearla mucho: ¿no te
parece?

---Me parece bien. Dos palabritas bien dichas, y basta por ahora. Pero
lo que importa es llevársela cuanto antes, pues la espera con
impaciencia.

---¿Cómo que la espera? ¿Pues acaso tú le has dicho algo?

---No\ldots{} verá Vd\ldots{} Ella debe haberlo adivinado. Cuando la di
el ramo díjele que se lo mandaba una persona de la casa que la quería
mucho y tenía pensado sacarla de aquí: ella lo besó.

---¡Lo besó!---exclamó el mancebo, tan conmovido, que algunas lágrimas
asomaron a sus ojos.---¡Lo besó! Es decir, se lo llevó a sus divinos
labios. ¡Ah!, Gabriel, ¿crees tú que me corresponderá?

---No lo creo, sino que lo afirmo---respondí enérgicamente.---Pero venga
la carta. Pues no se va a poner poco contenta. Ahora caigo en que me
debe usted dar la llave que encargó al cerrajero, para que yo entre y le
dé la carta en propia mano, porque no está bien visto que una cosa de
tanta importancia se arroje así\ldots{} pues.

---No: la llave no te la daré---contestó,---porque no necesitas entrar.
Quiero que esté sola, para que se entregue a sus anchas al placer de la
lectura. ¿Con que dices que lo recibió bien?

---Pero la llave, la llave\ldots{} ¿No me da Vd. la llave!

---No: la llave no te la doy. Déjala encerrada, que no faltará quien la
saque pronto. ¡Ay!, si me atreviera a ir yo mismo, y a hablarla\ldots{}
Pero no. En la carta le digo mi amor y mis proyectos; le digo que la
sacaré pronto de esta espantosa esclavitud, y que será mi mujer, mi
mujercita, pues nos casaremos en tierras lejanas\ldots{} ¿Sabes tú por
dónde se va a alguna de esas islas desiertas que nos cuentan\ldots?
Iremos; porque has de saber, Gabrielillo, que yo soy rico. Yo he
guardado mis ganancias desde hace veinte años. Lo malo es que todo lo
tengo en poder de los Requejos\ldots{} pero ya, ya tomaré yo lo que me
pertenezca. Entre esta noche y mañana he de poner por obra mi plan. ¿Ves
esta carta que tengo aquí para mi amo?, pues de esto depende todo.
Cuando él lea esta carta\ldots{} pero esto es un secreto\ldots{} punto
en boca.

---¿De modo que no me da Vd. la llave?

---No.~¿Para qué? No quiero que la veas, no quiero que la hables, cuando
yo no la hablo ni la veo. Al considerar que si entras en su cuarto te ha
de mirar, siento unos celos\ldots{} ¡Ay!, yo me muero, Gabriel; yo no
duermo, ni como, ni bebo. Si no tuviera qué hacer me estaría día y noche
paseando por los Melancólicos. Esta es mi única delicia, pensar en ella,
representármela en la imaginación y entablar con ella unos diálogos que
no tienen fin. A cada instante la abrazo y la beso a mis anchas, le
pongo una flor en la cabeza, la llevo en mis brazos cuando está cansada,
la arrullo, le canto para que se duerma y la visto por la mañana cuando
despierta.

---Así es Vd. feliz---repuse;---pero si me diera usted la llave le
contaría todo eso.

---No; yo se lo diré mañana, esta noche quizás---dijo Juan de Dios con
exaltación.---¿Pues qué crees tú que soy capaz de consentir un día más
los martirios que padece? Gabriel: a ti te puedo confiar mis planes.
¡Esta noche, esta noche quedará Inés en libertad! ¿Tú sabes por dónde se
va a alguna isla desierta?\ldots{} Anda lleva la carta, se la arrojas
por el tragaluz; ¿entiendes? Pobrecita: qué dirá cuando vea que hay
quien se interesa por ella, quien la adora, y está dispuesto a
sacrificar vida, hacienda y honor\ldots{} Así se lo he dicho esta mañana
al Santísimo Sacramento y a la Virgen María. Todos los días voy a misa y
ruego por ella a Dios y a los Santos. Esta mañana cuando el cura alzaba
el cáliz, le miré y dije: «Santísimo Sacramento de mi alma, yo amo a
Inés. Si quieres que no la ame más que a ti, dámela. Nunca te he pedido
nada. Con ella seré bueno, sin ella seré\ldots{} lo que el demonio
quiera.» Anda, Gabriel; llévale de una vez la esquelita.

A este punto llegábamos, cuando entró D. Mauro con dos amigos. Diole
Juan de Dios la carta de que antes me había hablado con tanto misterio,
y cuando la hubo leído lanzó grandes exclamaciones de coraje, que a
todos los presentes nos infundieron miedo. Al instante hizo salir a Juan
de Dios con una comisión apremiante, y yo me retiré. Aunque el maniático
no había querido entregar la llave, comprendí que no debía retroceder en
mi empresa, y resuelto a todo, pensé en descerrajar la puerta de la
prisión de Inés. Favorecía este proyecto la circunstancia de estar
Requejo en coloquio muy acalorado con sus dos amigos, y además ignorante
de la ausencia de su hermana.

Pedí auxilio a Dios mentalmente, y después de advertir a Inés para que
estuviese preparada y me ayudase por dentro, cogí un pequeño barrote de
hierro en figura de escoplo, que había en la sala de los empeños, y
comencé la delicada obra. El miedo de hacer ruido me obligaba a emplear
poca fuerza, y la cerradura no cedía. Canté en alta voz para ahogar todo
rumor, y al fin ayudado por Inés, que empujaba desde dentro, logré
desquiciar una de las hojas, que tuvimos buen cuidado de sostener para
que no viniese al suelo.

---Estás libre Inés, vámonos. Huyamos sin tardanza---exclamé con
locura.---Si nos detenemos un instante estamos perdidos.

Nos dirigimos a la puerta que conducía a la escalera exterior. Abrila
yo, y salimos. Ya oscurecía. Un hombre bajaba de los pisos superiores, y
se juntó a nosotros en la meseta. Advertí que nos miraba con sorpresa:
observele yo a mi vez, y no pude menos de temblar reconociendo al
licenciado Lobo, el cual extendiendo sus brazos como para detenernos,
preguntó:

---¿Adónde van Vds.?

---¿Y a Vd. qué le importa?---dije con rabia viendo delante de mí
obstáculo tan terrible.

Después, considerando que contra semejante cernícalo más convenía la
astucia que la fuerza, añadí:

---Doña Restituta nos ha mandado salir en busca suya. Ha ido en casa de
una amiga\ldots{}

---Tú eres un picarón redomado---me contestó.---¿A dónde vas con esa
muchacha? Tunantes: ¡os fugáis de esta santa casa! Ya os arreglaré yo.
Adentro pronto, si no queréis ir conmigo a la cárcel de Villa.

Mi desesperación no tuvo límites, y ahora celebro no haber tenido en
aquel momento un puñal en mi mano, porque de seguro le hubiera partido
el corazón al leguleño trapisondista.

---¡Ah!, pícaro ladrón, ya te conozco, ya sé quién
eres---continuó.---Esta noche precisamente pensaba venir a ajustarte las
cuentas\ldots{} No te había conocido, bribonzuelo; pero ya sé qué clase
de pájaro eres\ldots{} Ya tenía ganas de cogerte entre mis uñas.

Y efectivamente me tenía tan cogido, que no sé cómo no me desolló el
brazo.

Inés lloraba. Lobo la asió también por un brazo y empujándonos hacia
dentro, nos dijo:

---¡Qué a tiempo llegué, pimpollitos míos!

Hice un esfuerzo desesperado para desprenderme de sus garras y me
desprendí. Él entonces alzó el grito, exclamando:

---¡Que se me escapa ese tuno\ldots{} ladrones\ldots{} acudan acá!

Subió precipitadamente D. Mauro, reuniose en el portal alguna gente, y
acertando a llegar Restituta, poco después me encontraba entre ambos
Requejos como Cristo entre los dos ladrones. Inés desmayada, era
sostenida por el escribano.

\hypertarget{xxiii}{%
\chapter{XXIII}\label{xxiii}}

---Pero si apenas puedo creerlo---exclamaba mi ama.---¡Con que la
señorita huía con Gabriel! Tunante, ladroncillo, y cómo nos engañaba con
su carita de Pascua. Ven acá---añadió dándome golpes.---¿A dónde ibas
con Inesilla, monstruo? ¿Qué te han dado por entregarla, ladrón de
doncellas? A la cárcel, a presidio pronto, si es que no le desollamos
vivo. Pero di, ¿robabas a Inés?

---¡Sí, vieja bruja!---respondí con furia.---¡Me iba con ella!

---Pues ahora vas a ir por el balcón a la calle---dijo D. Mauro,
clavando en mi cuerpo su poderosa zarpa.

Francamente, señores, creí que había llegado mi último instante entre
aquellos tres bárbaros, que, cada cual según su estilo peculiar, me
mortificaban a porfía. De todos los golpes y vejaciones que allí recibí,
les aseguro a Vds. que nada me dolía tanto como los pellizcos de doña
Restituta, cuyos dedos, imitando los furiosos picotazos de un ave de
rapiña, se cebaban allí donde encontraban más carne.

---Y sin duda fuiste tú quien mandó a aquella maldita mujer, para
sacarme de la casa, pues en la plazuela de Afligidos no hay ya rastros
de almoneda. Este chico merece la horca, sí, Sr.~de Lobo, la horca.

---¡Y la muy andrajosa de mi sobrina se marchaba tan contenta!---dijo
Requejo, encerrando de nuevo a Inés en el miserable cuartucho.

---Si tenemos metido el infierno dentro de la casa---añadió
Restituta.---La horca, sí señor, la horca, Sr.~de Lobo. No tiene Vd.
pizca de caridad si no se lo dice al señor alcalde de casa y corte.
¡Pero cómo nos engañaba este dragoncillo! Si esto es para morirse uno de
rabia.

El leguleyo tomó entonces la autorizada palabra, y extendiendo sobre mi
cabeza sus brazos en la actitud propia de esa tutelar justicia que
ampara hasta a los criminales, dijo:

---Moderen Vds. su justa cólera y óiganme un instante. Ya les he dicho
que ahora nos ocupamos celosísimamente de hacer un benemérito expurgo
descubriendo y desenmascarando a todas las indignas personas que fueron
protegidas por el príncipe de la Paz; ese monstruo, señora, ese vil
mercader, ese infame favorito\ldots{} ¡gracias a Dios que está caído y
podemos insultarle sin miedo! Pues como decía, para que la nación se vea
libre de pícaros, a todos los que con él sirvieron, les quitamos ahora
sus destinos, si no pagan sus crímenes en la cárcel o en el destierro.
Si vieran Vds., amigos míos, cómo me estoy luciendo en estas pesquisas;
si oyeran ustedes los elogios que he merecido de los principales
servidores de la real persona\ldots{}

---Pero ¿a qué viene tanta palabrería---dijo impaciente Requejo,---ni
qué tiene eso que ver?\ldots{}

---Tiene que ver\ldots---prosiguió el hombre de la justicia,---porque
¿qué dirán mis señores D. Mauro y doña Restituta al saber que ese
tramposo y embaucador chicuelo aquí presente, recibió favores del
Príncipe, y es el mismo Gabrielillo que desde hace quince días estamos
buscando con los hígados en la boca mi compañero y yo?

Los Requejos macho y hembra se miraron con espanto.

---Pues oigan Vds. y tiemblen de indignación---prosiguió el
leguleyo.---El día antes de su caída, el Sr.~Godoy envió a la secretaría
de Estado un volante mandando que se diese a este joven una plaza en las
oficinas de la interpretación de lenguas. ¿Qué tal, señores? ¿Y por
qué?, dirán Vds. Porque este joven parece que sabe latín, y compuso un
poema en versos latinos; y algunos de esos alcahuetones que lo leyeron,
fueron con el cuento al Príncipe, diciéndole que mi niño era un portento
de sabiduría. ¡Mentiras y más mentiras! Ya se ve; cuando en la
secretaría de Estado recibieron el volante, se escandalizaron, porque ya
había caído el príncipe de la Paz, y aquellos eminentes repúblicos,
después de poner en la calle a Moratín, esperaron a que se presentara
este prodigio, si no para colocarlo, para verle al menos. Pero yo ando
tras el objeto de que coloquen allí a un primo mío que sabe tres
lenguas, el valenciano, el gallego y el castellano; así es que al punto
mi compañero y yo pusimos una \emph{diligencia en busca} para tener
antecedentes de esta buena pieza, y hemos conseguido probar: que en
Aranjuez vivía con el curita D. Celestino; otrosí que todos los días
iban ambos a casa de Godoy; otrosí, que el chico le escribía las cartas
y las traía a Madrid los domingos al embajador de Francia; otrosí, que
se disfrazaba para entrar en cierta taberna a oír lo que se decía, y
otras muchas bribonadas de que en el supradicho protocolo tengo hecha
detallada mención.

---¡Jesús, Dios nos ampare! Al santo patrono de la tienda debemos el
haber descubierto a tiempo lo que teníamos en casa---dijo Restituta.

---Por supuesto, que lo del latín era pura farsa.

---Pues no hay que andarse con chiquitas---dijo mi amo,---sino
entregarle a la justicia.

---Eso corre de mi cuenta---repuso Lobo.---Veremos qué responde a los
cargos que se le hacen en la sumaria como cómplice del cura castrense de
Aranjuez. A éste no le hemos podido coger, y según las noticias que hoy
recibí, ha desaparecido del Real Sitio. Es seguro que ha venido a
Madrid, y lo que es aquí no se nos escapa.

---¡Cuidado con el sabandijo que tenía yo en mi casa!---vociferó D.
Mauro, amenazando segunda vez poner fin a mis días.---Sr.~de Lobo,
quítemelo, quítemelo Vd. de entre las manos, porque acabo con él. Estoy
furioso. ¡Qué día, señor San Antonio de mi alma! ¡Qué día!

---Yo me encargaré del mocito---dijo Lobo.---Lo único que les pido, es
que me lo guarden hasta mañana.---¿Hasta mañana?

---Este bandolero no puede quedar en la casa hasta mañana; no
señor,---objetó mi ama.

---¿No hay lugar seguro donde encerrarle?

---¡Oh!, pierda Vd. cuidado; que si lo guardamos en el sótano, estará
como en un sepulcro---dijo Requejo.---Dificililla es la salida, y puedo
irme tranquilo.

---¿Pero te vas, hermano? ¿A dónde vas de noche?

---¿A dónde he de ir? ¡Mil pares de demonios! ¿A dónde he de ir sino a
Navalcarnero? ¿No saben ustedes lo que me pasa? ¿No les he contado?

---Nada nos ha dicho. Verdad es que con esta trapisonda de la
sobrinita\ldots{}

---Pues acabo de recibir una carta en que se me notifica que mi almacén
de Navalcarnero ha sido robado. ¿Ves, hermana? ¡Esto es para volverse
loco! Sí\ldots{} me escribe D. Roque notificándome el robo, y diciéndome
que acuda allí esta noche misma, si no quiero perderlo todo.

---¿Y va Vd.?

---Ahora mismo voy a buscar coche. Conque vean ustedes qué desastre.
¡Ay, Restituta! Bien te dije que no dejaras de encender la vela al santo
patrono. ¿Ves? Esto es un castigo.

---En el cielo no gustan de despilfarros. ¿Vas allá? ¿Pero me dejas en
la casa a este ladronzuelo?

---En el sótano, en el sótano: hasta mañana, hasta que mi Sr.~de Lobo
disponga de él. ¿No puede hacerse cuenta de que le dejamos en la
sepultura?

Sólo Dios puede sacarle.

---¿Pero me quedo sola? ¡Ánimas benditas!

---Juan de Dios vendrá a eso de las diez. Ya le he dicho que se quedará
en casa esta noche.

La conferencia terminó aquí, y sin más palabras, me encerraron en el
sótano, a cuyo subterráneo aposentamiento daba entrada una gran
compuerta por bajo el piso de la trastienda. Yo estaba medio aletargado
por la rabia y el despecho de aquella situación terrible. Sentí que me
impulsaban escalera abajo. D. Mauro cerró el escotillón, riendo con ese
gozo felino que da la conciencia de la propia crueldad, y me encontré
entre densas tinieblas. Mi amo había dicho bien al asegurar que allí
estaba como en un sepulcro. Sólo Dios podía sacarme.

Para que se comprenda si ellos tenían confianza en la seguridad de mi
cárcel, baste decir que allí tenían parte de su fortuna en un arca de
hierro. Cuando me encerraban en compañía de su dinero, ¿tendrían mis
amos la convicción de que era imposible la salida?

Hallábame en una de esas construcciones abovedadas con rosca de
ladrillo, que sirven de fundamento a casi todas las casas de Madrid
antiguas y modernas. Faltos de espacio superficial, los madrileños han
buscado la extensión hasta el cielo y hacia el abismo, de modo que cada
albergue es una torre colocada sobre un pozo. La de mis amos no tenía en
su sótano luces a la calle; la oscuridad era absoluta y el silencio
también, excepto cuando pasaba algún coche. Extendiendo mis brazos a
derecha, a izquierda y hacia arriba, tocaba ásperos ladrillos
endurecidos por un siglo, no tan húmedos como los que describen los
novelistas, cuando el hilo de sus relatos les lleva a alguna mazmorra
donde ocurren maravillosas. Como he dicho, ni un ruido lejano, ni un
rayo de luz turbaban la paz de aquel antro donde era posible llegar al
convencimiento de no existir, existiendo. Todo un arsenal de
herramientas no habría bastado a proporcionarme escapatoria, y pensar en
la fuga, habría sido pensar en lo absurdo. No tenía más consuelo que la
resignación, y me resigné. Estar allí dentro en plena soledad, en plena
lobreguez, en pleno silencio, era como cuando cerramos los ojos
encarcelándonos voluntariamente dentro de esa otra bóveda de nuestro
pensamiento. Acosteme en el suelo rendido de fatiga y medité. Mi prisión
no me parecía otra cosa que una prolongación de mi cerebro.

Quise pensar en varias cosas, pero no pude pensar más que en Dios.
Reconociéndome absolutamente incapaz para vencer la desgracia, comprendí
que la voluntad suprema había arrojado sobre mí tan gran pesadumbre de
males, y cruzándome de brazos, incliné la cabeza esperando que la misma
voluntad suprema me descargase de ella. Como esta esperanza me infundió
pronto una fe que hasta entonces en pocas ocasiones había tenido, creí
firmemente que Dios me sacaría de allí, y con esta creencia empecé a
adquirir un reposo moral y físico, precursor de cierto desvanecimiento
parecido al sueño. El de la desgracia se diferencia mucho al sueño de
todos los días, así es que el mío fue conforme al angustioso estado de
mi alma, un sueño de esos en que se representa el malestar real que
experimentamos, en proporciones informes, estrambóticas, monstruosas.
Percibía vagamente figuras y formas de esas que no pertenecen al mundo
visible, ni a la humanidad, ni a la fauna ni a la flora, ni al cielo ni
a la tierra, sino a cierta misteriosa geología, a yacimientos que
contradicen todas las leyes de la estática y la dinámica; percibía una
fantástica y continuada concatenación de colores geométricos que se
enredaban en mi cuerpo como culebras, y en aquella transmutación de lo
físico y lo moral, se verificaba el fenómeno de que un color me dolía, y
un objeto semejante a una espada, a un cangrejo o a un arpa pronunciaba
palabras incomprensibles. ¿Quién no ha desvariado alguna vez con estos
sueños de lo absurdo? Las ideas se mezclan con las visiones, y estas son
aquellas y aquellas estas. En aquel laberinto, en aquella aberración, mi
pensamiento formulaba sin cesar un silogismo azul, verde, ahora con
picos, después con curvas, más tarde irradiado, luego concéntrico, en
seguida poligonal y dorado, y al fin pequeño como un punto, para luego
ser grande como el universo. El interminable silogismo era: «La justicia
triunfa siempre: los Requejos son unos pillos; Inés y yo somos personas
honradas. Luego nosotros triunfaremos.»

Así pasé mucho tiempo en poder de estos demonios del sueño, cuando
percibí una claridad que no irradiaba de los focos de mi imaginación.
¿Estaba dormido o despierto? Híceme esta pregunta, y al punto contesté
que no sabía. La claridad aumentaba, y un chirrido metálico produjo en
mí cierto estremecimiento. Me moví, miré y vi las paredes del sótano, la
bóveda de ladrillo y multitud de cajas llenas y vacías; a mi izquierda,
una puerta que comunicaba con otro departamento subterráneo; a mi
derecha, una escalera, por la cual descendía la claridad que llamaba mi
atención. Estaba indudablemente despierto, y así lo reconocí. Miré a la
escalera, y vi dos pies que se trasladaban lentamente de peldaño a
peldaño. La luz de una linterna me deslumbró; pero en el foco de la
repentina claridad distinguí una cara amarilla. Era la de Juan de Dios;
era Juan de Dios en persona.

Cuando me vio, su espanto fue tan grande, que la linterna con que se
alumbraba estuvo a punto de caer de sus manos. Temblando y mudo, me
miraba como se mira una aparición diabólica o imagen evocada por la
brujería.

Figuraos la impresión del que entra en un sepulcro no creyendo, como es
natural, encontrar nada vivo, y encuentra un hombre que se mueve y no
parece pertenecer al mundo de los muertos.

\hypertarget{xxiv}{%
\chapter{XXIV}\label{xxiv}}

Santiguose Juan de Dios, y ya parecía dispuesto a huir como se huye de
las apariciones de ultratumba, cuando le hablé para disipar su miedo.

---Juan de Dios, soy yo. ¿No sabía usted que estaba aquí?

---Gabriel, si lo veo y no lo creo. ¡Jesús, María y José! ¿Cómo has
entrado aquí dentro?

---¿No sabe usted que me encerró don Mauro, al sorprenderme en el
momento de arrojar la carta a la señorita Inés? Acababa usted de salir.

---¡No había vuelto hasta ahora! ¡Y te encerraron aquí! ¡Qué casualidad!
Estoy absorto. Pero dime, ¿la carta\ldots?

---Ella la tiene. No hay cuidado por eso. Después de habérsela dado, me
entró tentación de hablar con ella. Toqué a la puerta, ¡ay!, este fue el
crítico momento en que se apareció doña Restituta. Puede usted figurarse
lo demás. Gracias a Dios que viene una buena alma para ponerme en
libertad. Dios le ha enviado a Vd.

---Óyeme, Gabrielillo---añadió con más sosiego.---Ya te dije que mi
fortunilla la tengo depositada en poder de los Requejos. Si se la pido
de improviso estoy seguro de que no me la han de dar. Por consiguiente,
yo la tomo. Mira lo que hay allí.

Señaló al fondo del sótano contiguo, y vi un arca de hierro. Juan de
Dios prosiguió de este modo.

---Yo tengo mi conciencia tranquila. No cojo más que lo mío, y antes
moriría que tomar un ochavo más. Eso bien lo sabe el Santísimo
Sacramento, que ya me conoce. Pero si en esta parte estoy
tranquilo\ldots{} ¡ay!, ya le he dicho al Santísimo Sacramento que estoy
loco de amor y que me perdone los dos grandes pecados que he cometido
hoy:

---¿Y qué pecados son esos?

---Trabajo me cuesta el decirlo; pero allá van para empezar desde ahora
a purgarlos con la vergüenza que me causan. Los dos pecados son: haber
escrito una carta falsa a D. Mauro para obligarle a ir a Navalcarnero, y
haber hecho construir por un molde de cera la llave con que he entrado
aquí, y la de la caja. La carta estaba perfectamente falsificada; las
llaves no valen menos.

---¿Con que eso va a toda prisa? ¿Y nuestra chicuela?---Esta noche me la
llevo. ¡Ah!, ya habrá leído la carta. La habrá leído, sabrá que la
quiero poner en libertad, y su inquietud, su agonía, su zozobra entre la
esperanza y el temor serán inmensas. Dentro de un rato será mía. ¿Cuento
contigo?

---Para lo que Vd. quiera. Pues no faltaba más---dije discurriendo cuál
sería el mejor modo de burlar a un mismo tiempo a doña Restituta y a su
prometido esposo.

---¡Ay!, tiemblo todo al pensar que pronto he de sacarla del poder de
estas fieras---dijo Juan de Dios.---La pobrecita me estará esperando ya.
¿Qué te parece? ¡Ah!, he preguntado a varias personas por una isla
desierta, y nadie me ha dado razón. ¿Esas que llaman las Canarias son
desiertas? ¿Sabes tú a dónde caen? Creo que allá por el gran golfo, o
como si dijéramos, entre la China y el Moro. ¿Por dónde se va?

---De eso sí que no sé palotada---contesté tratando de dejar a un lado
la geografía.---Pero vamos a ver: ¿cómo piensa Vd. engañar a doña
Restituta?

---Eso no me inquieta. La amarraremos tapándole la boca, pero sin
hacerle daño, porque es una buena mujer como no sea para criar
sobrinas\ldots{} y ya ves. Hace veinte años que como el pan de esta
casa. Si no fuera por esta terrible sofocación que me ha entrado\ldots{}
Gabriel yo me vuelvo loco; lo que no te sabré decir es si me vuelvo loco
de alegría o de pena.

---¿Le parece a Vd.---dije, afectando oficiosidad,---que suba pasito a
pasito a ver si doña Restituta duerme o vela?

---Bien pensado. Mejor es que te estés en la trastienda de centinela, y
en caso de que sientas ruido en el entresuelo me avisas al instante. Yo
despacharé eso fácilmente.

No esperé a que me lo repitiera y subí. No, Gabriel no subía, volaba. Mi
resolución, prontamente tomada, llevome sin vacilar al cuarto donde
dormía Inés y velaba su feroz tía. Cuando esta sintió mis pasos, cuando
oyó que alguien se acercaba, cuando llegué al cuarto, y me puso ante su
vista, su terror no tuvo límites. Como no comprendía la posibilidad
material de mi evasión, y era además mujer supersticiosa, no creyó sino
que yo era el diablo en persona, o al menos hombre protegido por todos
los diablos del infierno. Quedose muda de terror; quiso hablar y no
pudo; quiso gritar y lanzó un aullido congojoso, cual si la apretaran el
cuello. No queriendo yo perder un instante, me arrojé a sus plantas,
exclamando con sofocante precipitación:

---Señora, ama mía, ama de mi corazón: óigame su merced, soy inocente.
Perdóneme su merced. Quise revelarles a Vds. todo; pero aquellos hombres
no me dejaron. Yo no intenté robar a Inés, quise sacarla de aquí para
impedir que la robara su amante. ¿No sabe Vd. quién es? ¡Juan de Dios,
Juan de Dios! ¡Ah!, ¡señora!, ¡y dudaba Vd. de mi fidelidad! Restituta
pasó del terror a la sorpresa, al asombro, al anonadamiento, a la
estupidez.

---Juan de Dios!---exclamó.---¡Juan de Dios! Mi\ldots{} No, no puede
ser\ldots{} tú eres el demonio; Jesús, María y José. Por la señal de la
santa cruz\ldots{}

---¿Qué cruz ni cruz? ¿Quiere Vd. la prueba? Pues tome Vd. esa carta que
el caballerito me dio para su novia---dije, entregándole la carta del
mancebo.

Restituta la tomó en sus manos, frías como el mármol y temblorosas,
recorrió muy deprisa sus once pliegos, examinó la firma y díjome
después:

---¿Estoy soñando? Tú\ldots{} eres Gabriel\ldots¡Oh!, yo estoy
loca\ldots{} Ese miserable, a quien hemos dado de comer\ldots{}

---¿Aún lo duda Vd.?---dije.---Pues en este momento Juan de Dios está en
el sótano abriendo el arca del dinero.

No me es posible hacer formar idea del salto que dio Restituta. Creo que
hasta la silla saltó también arrastrada por el espantoso sacudimiento de
los nervios de la hermana del Sr.~D. Mauro.

---Venga Vd. y lo verá con sus propios ojos---exclamé tomándola de la
mano e impeliéndola hacia afuera.

Restituta me siguió, porque la curiosidad, la rabia, el mismo terror, la
impulsaban tras mí. Tropezó mil veces. Su cuerpo temblaba, y con
frecuencia llevábase las manos a los desgreñados pelos para arrancarse
algunos, o para echarlos todos hacia atrás. El extravío de sus ojos a
nada es comparable, y a mí mismo, que ya creía tenerla vencida, me
causaba miedo.

Llegamos a la boca del escotillón, y allí, mientras hería nuestros ojos
la tenue claridad que del sótano salía, oímos claramente ruido de
monedas. Juan de Dios contaba sus ahorros de veinte años. Cuando el
tímpano de Restituta fue afectado de aquel vibrante sonido, un
estremecimiento nervioso como el producido en la organización humana por
la descarga de poderosas pilas eléctricas, sacudió sus miembros,
precipitándose ciegamente por la escalera, exclamó:

---¡Malvado! ¡Así nos pagas el pan de veinte años!

Aún no habían llegado los resbaladizos pies de mi ama al quinto peldaño,
cuando la pesada puerta del escotillón cayó, lanzada por mis manos. No
había llave con qué cerrar, porque Juan de Dios la había quitado; pero
al instante puse sobre la puerta una caja de latas de pomada, y luego
dos, y luego cuatro, y después un fardo de tela, y otro y otro encima.
En diez minutos puse sobre la entrada de la que había sido mi prisión un
peso tal, que cuatro hombres fuertes no hubieran podido levantarlo desde
abajo.

Concluido esto subí. Inés, despavorida y aterrada, no sabía a qué santo
encomendarse.---¡Ya eres libre, Inés!---exclamé con la mayor
alegría.---Vístete, vámonos pronto. No perder un momento: puede venir el
amo.

Vistiose tan precipitadamente, que la vi medio desnuda. Pero ni ella con
el gran azoramiento de la prisa cayó en la cuenta de que me estaba
mostrando su lindo cuerpo, ni yo me cuidaba más que de ayudarla a
vestir, poniéndole enaguas, medias, zapatos, ligas. Al fin salimos de la
casa y huimos a toda prisa de la calle de la Sal por temor de encontrar
al licenciado Lobo o a mi amo. Hasta que no nos vimos en la Puerta del
Sol, no tomamos aliento, y sintiéndome yo sin fuerzas, nos sentamos en
un escalón junto a Mariblanca. Profundo silencio reinaba en la plaza:
Madrid dormía sosegado y tranquilo. Paseé mi vista en derredor y no vi
más que dos perros que se disputaban un hueso: el chorro de la fuente
alegraba nuestras almas, con su parlero rumor.

---Ya estás libre, condesilla---dije reclinándome sobre el pecho de
Inés.---Bendito sea Dios que nos ha sacado de allí. No te olvidaré
nunca, horrenda noche de amargura; no te olvidaré nunca, risueña mañana
de este día feliz. Estamos en lunes, día 2 del mes de Mayo.

Un rato permanecí en aquella actitud, porque estaba rendido de
cansancio. El día se acercaba y se sentían los primeros y vagos rumores,
desperezos de la indolente ciudad que despierta. Por Oriente hacia el
fin de la calle de Alcalá se veía el resplandor de la aurora, y cuando
nos retirábamos, Inés y yo nos detuvimos un instante a contemplar el
cielo que por aquella parte se teñía de un vivo color de sangre.

\hypertarget{xxv}{%
\chapter{XXV}\label{xxv}}

Al entrar en mi casa, donde yo pensaba descansar un rato con Inés, antes
de emprender la fuga, encontramos al buen D. Celestino que habiendo
llegado la noche anterior, creyó conveniente albergarse en mi humilde
posada antes que en otra cualquiera de las de la corte. Ya le había yo
informado por escrito de la verdadera situación de las cosas en casa de
los Requejos, así es que desde luego guardose de poner los pies en la
famosa tienda. Él y nosotros nos alegramos mucho de vernos juntos, y
apenas teníamos tiempo para preguntarnos nuestras mutuas desgracias,
pues ya habrán comprendido Vds. que las del bondadoso sacerdote no eran
menores que las nuestras.

---Pero hijos míos---nos dijo,---Dios nos ha de proteger. ¿Cómo es
posible que los malvados triunfen fácilmente de los rectos de corazón?
Vosotros huís de la maldad de aquellos dos hermanos, y yo también huyo,
yo también vengo aquí ocultando mi nombre honrado, porque me persiguen
como a un criminal.

Al decir esto, el buen anciano derramó algunas lágrimas y nosotros para
consolarle, le animábamos presentándole el espectáculo de nuestra
alegría, y contábamos entre risas y chistes las extravagancias y
tacañerías de los tíos de Inés.

---Dios nos ayudará---continuó el cura.---Veamos ahora cómo salimos de
Madrid. ¡Oh qué persecución tan horrorosa! Me acusan de que fui amigo
del príncipe de la Paz. Ya lo creo que fui amigo de S. A. No sólo amigo,
sino aun creo que pariente. No puedes figurarte los líos que me han
armado, Gabrielillo\ldots{} y también te acusan a ti\ldots{} ¡Has visto
qué pícaros!\ldots{} Que si escribíamos cartas\ldots{} que si tú las
llevabas\ldots{} Verdad es que yo fui varias veces al palacio de S. A.
para aconsejarle lo que me parecía conveniente para el bien de la
nación; pero nunca le dije nada, porque con esta mi cortedad de
genio\ldots{} En resumen, hijo, sabiendo que me iban a prender, me puse
en camino callandito, y pienso presentarme al señor Patriarca, para que
disponga de mí. Pero oíd lo mejor. ¿Creeréis que ese tunante de
Santurrias es quien más sañudamente me ha perseguido, dando testimonios
falsos de mi conducta? Nada, nada; es cierto lo que yo dije en aquel
sermón: ¿te acuerdas, Gabriel? Dije que la ingratitud es el más feo
monstruo que existe sobre la tierra. \emph{Vilissima et turpissima
hydra}. ¡Quién lo había de pensar!

---Ahora pensemos, señor cura, cómo nos las vamos a componer para salir
de este laberinto. ¿A dónde vamos? ¿Qué recursos tenemos?

---Hijo mío, Dios no ha de desampararnos. Confiemos en él, y entre tanto
oye un proyecto que esta madrugada me ha ocurrido. Hace ocho días estaba
en Aranjuez la señora marquesa de ***, persona discreta, muy temerosa de
Dios, y de tan buen corazón, que remedia cuantas necesidades llegan a su
noticia. Visitome ella varias veces, la visité yo también, y según me
decía, mi trato le era sumamente agradable. Esto lo diría por urbanidad.
Me preguntaba mucho por Inés, mostrando grandísimos deseos de conocerla,
y cuando por última vez la vi, suplicome encarecidamente que si alguna
vez pasaba a la corte, no dejase de acudir a su casa, en compañía de mi
sobrina. Esto me lo repitió muchas veces, y su empeño por ver a la
sobrinilla, me ha llamado mucho la atención.

---También a mí---repuse.---Conozco a la señora marquesa, en cuyo
palacio representé cierto papel de traidor, de que no quisiera
acordarme. Era en la misma casa donde Vds. vivían.

---Pero la señora marquesa no vive ahora allí, pues durante la primavera
se traslada a la casa de su hermano, allá por la cuesta de la Vega, en
un palacio que tiene muy amenos jardines, y espacioso horizonte hacia la
parte del Manzanares. Allí encontraremos hoy a esa insigne señora, honor
de la hispana grandeza. ¿Por qué no acudir a ella? Me ha dicho infinitas
veces que desea servirme, tanto a mí como a mi sobrina, y que espera con
ansia el momento en que yo quiera usar de su poder y valimiento para
cualquier asunto.

---En esa señora nos manda Dios un comisionado para salir de este
apuro---dije yo sintiéndome con mayores ánimos.---Le contaremos lo que
nos pasa, comprenderá con cuánta injusticia se nos persigue, y cuando
vea a Inés\ldots{} ¡Ay!, se me figura que el empeño de la marquesa en
ver a Inés no es simple curiosidad. En fin: visitarémosla hoy mismo y
Dios dirá.

---Temo salir a la calle.

---Yo también; pero es preciso salir, no es cosa de que andemos por los
tejados. Si quiere usted iré yo ahora mismo a casa de la señora
marquesa, que ya me conoce, y diciéndole que voy de parte de Vd. le
pintaré la situación en que nos encontramos, hablándole también de
Inesilla, que es sin duda lo que le interesa más.

---Me parece bien; ¿y si te ven?

---Iré por calles extraviadas, y en caso de apuro, no me faltan piernas
con que perderme de vista. Yo estaba dominado por vivísima excitación, y
cuando adoptaba un plan, cada segundo que transcurría sin ponerlo por
obra, parecíame un siglo. No me era posible entregarme al reposo sin dar
aquel paso en un camino que me parecía conducir a lugar seguro en
nuestro desgraciado aislamiento. Inés no podía descansar tampoco, y su
espíritu, no repuesto del azoramiento y zozobra de la madrugada
anterior, era impresionado fuertemente por cuanto veía. Asomábase a la
ventana que caía hacia la calle de San José, frente al parque de
artillería, y como la vivienda era piso principal bajando del cielo, se
veía el gran patio interior de aquel establecimiento de guerra, con los
cañones y demás pertrechos, puestos en ordenadas filas a un lado y otro.

---Esto que ves es el parque de artillería, niña---le dijo D.
Celestino.---¿Ves?, en aquellos grandes edificios se alojan los
artilleros. Mira, salen algunos con un carro para ir a casa del
abastecedor en busca de las provisiones.

---¿Y esas montañitas tan bonitas, formadas por cosas negras y redondas,
iguales todas y puestas con mucho orden?---preguntó la muchacha, sin dar
tregua a su admiración.

---Esas son balas, chicuela---repuso el clérigo.---Los hombres han
inventado esos juguetes para matarse unos a otros.

---Esas balas se meten en los cañones que están allí junto---dije yo,
queriendo mostrar mi erudición,---y poniendo también pólvora y un
cartucho se dispara y es muy bonito. Hace un ruido, chiquilla, que se
vuelve uno loco. ¡Si vieras cómo me lucí en el combate de Trafalgar! ¡Si
tú me hubieras visto!\ldots{} Lo menos maté mil ingleses.

---Quiten para allá---exclamó con miedo D. Celestino.---Sólo de pensar
que eso se dispara me pongo a temblar.

Y se retiraron de la ventana. Yo aconsejé a Inés que descansara, y salí
a la calle después que D. Celestino, echándome algunas bendiciones, rezó
un \emph{pater noster} por mi seguridad y buena suerte en la comisión
que iba a desempeñar.

Alejándome todo lo posible del centro de la villa, llegué a la plazuela
de Palacio, donde me detuvo un obstáculo casi insuperable; un gran
gentío, que bajando de las calles del Viento, de Rebeque, del Factor, de
Noblejas y de las plazuelas de San Gil y del Tufo, invadía toda la calle
Nueva y parte de la plazuela de la Armería. Pensando que sería probable
encontrar entre tanta gente al licenciado Lobo, procuré abrirme paso
hasta rebasar tan molesta compañía; pero esto era punto menos que
imposible, porque me encontraba envuelto, arrastrado por aquel inmenso
oleaje humano, contra el cual era difícil luchar.

Yo estaba tan preocupado con mis propios asuntos, que durante algún
tiempo no discurrí sobre la causa de aquella tan grande y ruidosa
reunión de gente, ni sobre lo que pedía, porque indudablemente pedía o
manifestaba desear alguna cosa. Después de recibir algunos porrazos y
tropezar repetidas veces, me detuve arrimado al muro de Palacio, y
pregunté a los que me rodeaban:

---¿Pero qué quiere toda esa gente?

---Es que se van, se los llevan---me dijo un chispero,---y eso no lo
hemos de consentir.

El lector comprenderá que no me importaba gran cosa que se fueran o
dejaran de irse los que lo tuvieran por conveniente, así es que intenté
seguir mi camino. Poco había adelantado, cuando me sentí cogido por un
brazo. Estremecime de terror creyendo que estaba nuevamente en las
garras del licenciado; pero no se asusten Vds.: era Pacorro Chinitas.

---¿Con que parece que se los llevan?---me dijo.

---¿A los infantes? Eso dicen; pero te aseguro, Chinitas que eso me
tiene sin cuidado.

---Pues a mí no. Hasta aquí llegó la cosa, hasta aquí aguantamos, y de
aquí no ha de pasar. Tú eres un chiquillo y no piensas más que en jugar,
y por eso no te importa.

---Francamente, Chinitas, yo tengo que ocuparme demasiado de lo que a mí
me pasa.

---Tú no eres español---me dijo el amolador con gravedad.---Sí que lo
soy---repuse.

---Pues entonces no tienes corazón, ni eres hombre para nada.

---Sí que soy hombre y tengo corazón para lo que sea preciso.

---Pues entonces, ¿qué haces ahí como un marmolillo? ¿No tienes armas?
Coge una piedra y rómpele la cabeza al primer francés que se te ponga
por delante.

---Han pasado sin duda cosas que yo no sé, porque he estado muchos días
sin salir a la calle.

---No, no ha pasado nada todavía, pero pasará. ¡Ah! Gabrielillo, lo que
yo te decía ha salido cierto. Todos se han equivocado, menos el
amolador. Todos se han ido y nos han dejado solos con los franceses. Ya
no tenemos Rey, ni más gobierno que esos cuatro carcamales de la Junta.

Yo me encogí de hombros, no comprendiendo por qué estábamos sin Rey y
sin más gobierno que los cuatro carcamales de la Junta.

---Gabriel---me dijo mi amigo después de un rato---¿te gusta que te
manden los franceses, y que con su lengua que no entiendes, te digan
«haz esto o haz lo otro,» y que se entren en tu casa, y que te hagan ser
soldado de Napoleón, y que España no sea España, vamos al decir, que
nosotros no seamos como nos da la gana de ser, sino como el Emperador
quiera que seamos?

---¿Qué me ha de gustar? Pero eso es pura fantasía tuya. ¿Los franceses
son los que nos mandan? ¡Quia! Nuestro Rey, cualquiera que sea, no lo
consentiría.

---No tenemos Rey.

---¿Pero no habrá en la familia otro que se ponga la corona?

---Se llevan todos los infantes.

---Pero habrá grandes de España y señores de muchas campanillas, y
generales y ministros que les digan a los ministros: «Señores, hasta
aquí llegó. Ni un paso más.»

---Los señores de muchas campanillas se han ido a Bayona, y allí andan a
la greña por saber si obedecen al padre o al hijo.

---Pero aquí tenemos tropas que no consentirán\ldots{}

---El Rey les ha mandado que sean amigos de los franceses y que les
dejen hacer.

---Pero son españoles, y tal vez no obedezcan esa barbaridad; porque
dime: si los franceses nos quieren mandar, ¿es posible que un español de
los que vistan uniforme lo consienta?

---El soldado español no puede ver al francés pero son uno por cada
veinte. Poquito a poquito se han ido entrando, entrando, y ahora,
Gabriel, esta baldosa en que ponemos los pies es tierra del emperador
Napoleón.

---¡Oh, Chinitas! Me haces temblar de cólera. Eso no se puede aguantar,
no señor. Si las cosas van como dices, tú y todos los demás españoles
que tengan vergüenza cogerán un arma, y entonces\ldots{}

---No tenemos armas.

---Entonces, Chinitas, ¿qué remedio hay? Yo creo que si todos, todos,
todos dicen: «vamos a ellos,» los franceses tendrán que retirarse.

---Napoleón ha vencido a todas las naciones.

---Pues entonces echémonos a llorar y metámonos en nuestras casas.

---¿Llorar?---exclamó el amolador cerrando los puños.---Si todos
pensaran como yo\ldots{} No se puede decir lo que sucederá, pero\ldots{}
Mira: yo soy hombre de paz, pero cuando veo que estos condenados
franceses se van metiendo callandito en España diciendo que somos
amigos: cuando veo que se llevan engañado al Rey; cuando les veo por
esas calles echando facha y bebiéndose el mundo de un sorbo; cuando
pienso que ellos están muy creídos de que nos han metido en un puño por
los siglos de los siglos, me dan ganas\ldots{} no de llorar, sino de
matar, pongo el caso, pues\ldots{} quiero decir que si un francés pasa y
me toca con su codo en el pelo de la ropa, levanto la mano\ldots{} mejor
dicho\ldots{} abro la boca y me lo como. Y cuidado, que un francés me
enseñó el oficio que tengo. El francés me gusta; pero allá en su tierra.

\hypertarget{xxvi}{%
\chapter{XXVI}\label{xxvi}}

Durante nuestra conversación advertí que la multitud aumentaba,
apretándose más. Componíanla personas de ambos sexos y de todas las
clases de la sociedad, espontáneamente venidas por uno de esos
llamamientos morales, íntimos, misteriosos, informulados, que no parten
de ninguna voz oficial, y resuenan de improviso en los oídos de un
pueblo entero, hablándole el balbuciente lenguaje de la inspiración. La
campana de ese arrebato glorioso no suena sino cuando son muchos los
corazones dispuestos a palpitar en concordancia con su anhelante ritmo,
y raras veces presenta la historia ejemplos como aquel, porque el
sentimiento patrio no hace milagros sino cuando es una condensación
colosal, una unidad sin discrepancias de ningún género, y por lo tanto
una fuerza irresistible y superior a cuantos obstáculos pueden oponerle
los recursos materiales, el genio militar y la muchedumbre de enemigos.
El más poderoso genio de la guerra es la conciencia nacional, y la
disciplina que da más cohesión el patriotismo.

Estas reflexiones se me ocurren ahora recordando aquellos sucesos.
Entonces, y en la famosa mañana de que me ocupo, no estaba mi ánimo para
consideraciones de tal índole, mucho menos en presencia de un conflicto
popular que de minuto en minuto tomaba proporciones graves. La ansiedad
crecía por momentos: en los semblantes había más que ira, aquella
tristeza profunda que precede a las grandes resoluciones, y mientras
algunas mujeres proferían gritos lastimosos, oí a muchos hombres
discutiendo en voz baja planes de no sé qué inverosímil lucha.

El primer movimiento hostil del pueblo reunido fue rodear a un oficial
francés que a la sazón atravesó por la plaza de la Armería. Bien pronto
se unió a aquél otro oficial español que acudía como en auxilio del
primero. Contra ambos se dirigió el furor de hombres y mujeres, siendo
estas las que con más denuedo les hostilizaban; pero al poco rato una
pequeña fuerza francesa puso fin a aquel incidente. Como avanzaba la
mañana, no quise ya perder más tiempo, y traté de seguir mi camino; mas
no había pasado aún el arco de la Armería, cuando sentí un ruido que me
pareció cureñas en acelerado rodar por calles inmediatas.

---¡Que viene la artillería!---clamaron algunos.

Pero lejos de determinar la presencia de los artilleros una dispersión
general, casi toda la multitud corría hacia la calle Nueva\footnote{Hoy
  de Bailén}. La curiosidad pudo en mí más que el deseo de llegar pronto
al fin de mi viaje, y corrí allá también; pero una detonación espantosa
heló la sangre en mis venas; y vi caer no lejos de mí algunas personas,
heridas por la metralla. Aquel fue uno de los cuadros más terribles que
he presenciado en mi vida. La ira estalló en boca del pueblo de un modo
tan formidable, que causaba tanto espanto como la artillería enemiga.
Ataque tan imprevisto y tan rudo había aterrado a muchos que huían con
pavor, y al mismo tiempo acaloraba la ira de otros, que parecían
dispuestos a arrojarse sobre los artilleros; mas en aquel choque entre
los fugitivos y los sorprendidos, entre los que rugían como fieras y los
que se lamentaban heridos o moribundos bajo las pisadas de la multitud,
predominó al fin el movimiento de dispersión, y corrieron todos hacia la
calle Mayor. No se oían más voces que «armas, armas, armas.» Los que no
vociferaban en las calles, vociferaban en los balcones, y si un momento
antes la mitad de los madrileños eran simplemente curiosos, después de
la aparición de la artillería todos fueron actores. Cada cual corría a
su casa, a la ajena o a la más cercana en busca de un arma, y no
encontrándola, echaba mano de cualquier herramienta. Todo servía con tal
que sirviera para matar.

El resultado era asombroso. Yo no sé de dónde salía tanta gente armada.
Cualquiera habría creído en la existencia de una conjuración
silenciosamente preparada; pero el arsenal de aquella guerra imprevista
y sin plan, movida por la inspiración de cada uno, estaba en las
cocinas, en los bodegones, en los almacenes al por menor, en las salas y
tiendas de armas, en las posadas y en las herrerías.

La calle Mayor y las contiguas ofrecían el aspecto de un hervidero de
rabia imposible de describir por medio del lenguaje. El que no lo vio,
renuncie a tener idea de semejante levantamiento. Después me dijeron que
entre 9 y 11 todas las calles de Madrid presentaban el mismo aspecto;
habíase propagado la insurrección como se propaga la llama en el bosque
seco azotado por impetuosos vientos.

En el Pretil de los Consejos, por San Justo y por la plazuela de la
Villa, la irrupción de gente armada viniendo de los barrios bajos era
considerable; mas por donde vi aparecer después mayor número de hombres
y mujeres, y hasta enjambres de chicos y algunos viejos fue por la plaza
Mayor y los portales llamados de Bringas. Hacia la esquina de la calle
de Milaneses, frente a la Cava de San Miguel, presencié el primer choque
del pueblo con los invasores, porque habiendo aparecido como una
veintena de franceses que acudían a incorporarse a sus regimientos,
fueron atacados de improviso por una cuadrilla de mujeres ayudadas por
media docena de hombres. Aquella lucha no se parecía a ninguna peripecia
de los combates ordinarios, pues consistía en reunirse súbitamente
envolviéndose y atacándose sin reparar en el número ni en la fuerza del
contrario. Los extranjeros se defendían con su certera puntería y sus
buenas armas: pero no contaban con la multitud de brazos que les ceñían
por detrás y por delante, como rejos de un inmenso pulpo; ni con el
incansable pinchar de millares de herramientas, esgrimidas contra ellos
con un desorden y una multiplicidad semejante al de un ametrallamiento a
mano; ni con la espantosa centuplicación de pequeñas fuerzas que sin
matar imposibilitaban la defensa. Algunas veces esta superioridad de los
madrileños era tan grande, que no podía menos de ser generosa; pues
cuando los enemigos aparecían en número escaso, se abría para ellos un
portal o tienda donde quedaban a salvo, y muchos de los que se alojaban
en las casas de aquella calle debieron la vida a la tenacidad con que
sus patronos les impidieron la salida.

No se salvaron tres de a caballo que corrían a todo escape hacia la
Puerta del Sol. Se les hicieron varios disparos; pero irritados ellos
cargaron sobre un grupo apostado en la esquina del callejón de la
Chamberga, y bien pronto viéronse envueltos por el paisanaje. De un
fuerte sablazo, el más audaz de los tres abrió la cabeza a una infeliz
maja en el instante en que daba a su marido el fusil recién cargado, y
la imprecación de la furiosa mujer al caer herida al suelo, espoleó el
coraje de los hombres. La lucha se trabó entonces cuerpo a cuerpo y a
arma blanca.

Entretanto yo corrí hacia la Puerta del Sol buscando lugar más seguro, y
en los portales de Pretineros encontré a Chinitas. La Primorosa salió
del grupo cercano exclamando con frenesí:

---¡Han matado a Bastiana! Más de veinte hombres hay aquí y denguno vale
un rial. Canallas; ¿para qué os ponéis bragas si tenéis almas de
pitiminí?

---Mujer---dijo Chinitas cargando su escopeta,---quítate de en medio.
Las mujeres aquí no sirven más que de estorbo.

---Cobardón, calzonazos, corazón de albondiguilla---dijo la Primorosa
pugnando por arrancar el arma a su marido.---Con el aire que hago
moviéndome, mato yo más franceses que tú con un cañón de a ocho.

Entonces uno de los de a caballo se lanzó al galope hacia nosotros
blandiendo su sable.

---¡Menegilda!, ¿tienes navaja?---exclamó la esposa de Chinitas con
desesperación.

---Tengo tres, la de cortar, la de picar y el cuchillo grande.

---¡Aquí estamos, espanta-cuervos!---gritó la maja tomando de manos de
su amiga un cuchillo carnicero cuya sola vista causaba espanto.

El coracero clavó las espuelas a su corcel y despreciando los tiros se
arrojó sobre el grupo. Yo vi las patas del corpulento animal sobre los
hombros de la Primorosa; pero ésta, agachándose más ligera que el rayo,
hundió su cuchillo en el pecho del caballo. Con la violenta caída, el
jinete quedó indefenso, y mientras la cabalgadura expiraba con horrible
pataleo, lanzando ardientes resoplidos, el soldado proseguía el combate
ayudado por otros cuatro que a la sazón llegaron.

Chinitas, herido en la frente y con una oreja menos, se había retirado
como a unas diez varas más allá, y cargaba un fusil en el callejón del
Triunfo, mientras la Primorosa le envolvía un pañuelo en la cabeza,
diciéndole:

---Si te moverás al fin. No parece sino que tienes en cada pata las
pesas del reló de Buen Suceso.

El amolador se volvió hacia mí y me dijo:

---Gabrielillo, ¿qué haces con ese fusil? ¿Lo tienes en la mano para
escarbarte los dientes?

En efecto, yo tenía en mis manos un fusil sin que hasta aquel instante
me hubiese dado cuenta de ello. ¿Me lo habían dado? ¿Lo tomé yo? Lo más
probable es que lo recogí maquinalmente, hallándose cercano al lugar de
la lucha, y cuando caía sin duda de manos de algún combatiente herido;
pero mi turbación y estupor eran tan grandes ante aquella escena, que ni
aun acertaba a hacerme cargo de lo que tenía entre las manos.

---¿Pa qué está aquí esa lombriz?---dijo la Primorosa encarándose
conmigo y dándome en el hombro una fuerte manotada.---Descosío: coge ese
fusil con más garbo. ¿Tienes en la mano un cirio de procesión?

---Vamos: aquí no hay nada que hacer---afirmó Chinitas, encaminándose
con sus compañeros hacia la Puerta del Sol.

Echeme el fusil al hombro y les seguí. La Primorosa seguía burlándose de
mi poca aptitud para el manejo de las armas de fuego.

---¿Se acabaron los franceses?---dijo una maja mirando a todos
lados.---¿Se han acabado?

---No hemos dejado uno pa simiente de rábanos---contestó la
Primorosa.---¡Viva España y el Rey Fernando!

En efecto, no se veía ningún francés en toda la calle Mayor; pero no
distábamos mucho de las gradas de San Felipe, cuando sentimos ruido de
tambores, después ruido de cornetas, después pisadas de caballos,
después estruendo de cureñas rodando con precipitación. El drama no
había empezado todavía realmente. Nos detuvimos, y advertí que los
paisanos se miraban unos a otros, consultándose mudamente sobre la
importancia de las fuerzas ya cercanas. Aquellos infelices madrileños
habían sostenido una lucha terrible con los soldados que encontraron al
paso, y no contaban con las formidables divisiones y cuerpos de ejército
que se acampaban en las cercanías de Madrid. No habían medido los
alcances y las consecuencias de su calaverada, ni aunque los midieran,
habrían retrocedido en aquel movimiento impremeditado y sublime que les
impulsó a rechazar fuerzas tan superiores. Había llegado el momento de
que los paisanos de la calle Mayor pudieran contar el número de armas
que apuntaban a sus pechos, porque por la calle de la Montera apareció
un cuerpo de ejército, por la de Carretas otro, y por la Carrera de San
Jerónimo el tercero, que era el más formidable.

---¿Son muchos?---preguntó la Primorosa.

---Muchísimos, y también vienen por esta calle. Allá por Platerías se
siente ruido de tambores.

Frente a nosotros y a nuestra espalda teníamos a los infantes, a los
jinetes y a los artilleros de Austerlitz. Viéndoles, la Primorosa reía;
pero yo\ldots{} no puedo menos de confesarlo\ldots{} yo temblaba.

\hypertarget{xxvii}{%
\chapter{XXVII}\label{xxvii}}

Llegar los cuerpos de ejército a la Puerta del Sol y comenzar el ataque,
fueron sucesos ocurridos en un mismo instante. Yo creo que los
franceses, a pesar de su superioridad numérica y material, estaban más
aturdidos que los españoles; así es que en vez de comenzar poniendo en
juego la caballería, hicieron uso de la metralla desde los primeros
momentos.

La lucha, mejor dicho, la carnicería era espantosa en la Puerta del Sol.
Cuando cesó el fuego y comenzaron a funcionar los caballos, la guardia
polaca llamada \emph{noble}, y los famosos mamelucos cayeron a sablazos
sobre el pueblo, siendo los ocupadores de la calle Mayor los que
alcanzamos la peor parte, porque por uno y otro flanco nos atacaban los
feroces jinetes. El peligro no me impedía observar quién estaba en torno
mío, y así puedo decir que sostenían mi valor vacilante además de la
Primorosa, un señor grave y bien vestido que parecía aristócrata, y dos
honradísimos tenderos de la misma calle, a quienes yo de antiguo
conocía.

Teníamos a mano izquierda el callejón de la Duda; como sitio estratégico
que nos sirviera de parapeto y de camino para la fuga, y desde allí el
señor noble y yo, dirigíamos nuestros tiros a los primeros mamelucos que
aparecieron en la calle. Debo advertir, que los tiradores formábamos una
especie de retaguardia o reserva, porque los verdaderos y más aguerridos
combatientes, eran los que luchaban a arma blanca entre la caballería.
También de los balcones salían muchos tiros de pistola y gran número de
armas arrojadizas, como tiestos, ladrillos, pucheros, pesas de reloj
etc.

---Ven acá, Judas Iscariote---exclamó la Primorosa, dirigiendo los puños
hacia un mameluco que hacía estragos en el portal de la c asa de
Oñate.---¡Y no hay quien te meta una libra de pólvora en el cuerpo! ¡Eh,
so estantigua!, ¿pa qué le sirve ese chisme? Y tú, Piltrafilla, echa
fuego por ese fusil, o te saco los ojos.

Las imprecaciones de nuestra generala nos obligaban a disparar tiro tras
tiro. Pero aquel fuego mal dirigido no nos valía gran cosa, porque los
mamelucos habían conseguido despejar a golpes gran parte de la calle, y
adelantaban de minuto en minuto.

---A ellos, muchachos---exclamó la maja, adelantándose al encuentro de
una pareja de jinetes, cuyos caballos venían hacia nosotros.

Ustedes no pueden figurarse cómo eran aquellos combates parciales.
Mientras desde las ventanas y desde la calle se les hacía fuego, los
manolos les atacaban navaja en mano, y las mujeres clavaban sus dedos en
la cabeza del caballo, o saltaban, asiendo por los brazos al jinete.
Este recibía auxilio, y al instante acudían dos, tres, diez, veinte, que
eran atacados de la misma manera, y se formaba una confusión, una
mescolanza horrible y sangrienta que no se puede pintar. Los caballos
vencían al fin y avanzaban al galope, y cuando la multitud encontrándose
libre se extendía hacia la Puerta del Sol, una lluvia de metralla le
cerraba el paso.

Perdí de vista a la Primorosa en uno de aquellos espantosos choques;
pero al poco rato la vi reaparecer lamentándose de haber perdido su
cuchillo, y me arrancó el fusil de las manos con tanta fuerza, que no
pude impedirlo. Quedé desarmado en el mismo momento en que una fuerte
embestida de los franceses nos hizo recular a la acera de San Felipe el
Real. El anciano noble fue herido junto a mí: quise sostenerle; pero
deslizándose de mis manos, cayó exclamando: «¡Muera Napoleón! ¡Viva
España!»

Aquel instante fue terrible, porque nos acuchillaron sin piedad; pero
quiso mi buena estrella, que siendo yo de los más cercanos a la pared,
tuviera delante de mí una muralla de carne humana que me defendía del
plomo y del hierro. En cambio era tan fuertemente comprimido contra la
pared, que casi llegué a creer que moría aplastado. Aquella masa de
gente se replegó por la calle Mayor, y como el violento retroceso nos
obligara a invadir una casa de las que hoy deben tener la numeración
desde el 21 al 25, entramos decididos a continuar la lucha desde los
balcones. No achaquen Vds. a petulancia el que diga \emph{nosotros},
pues yo, aunque al principio me vi comprendido entre los sublevados como
al acaso y sin ninguna iniciativa de mi parte, después el ardor de la
refriega, el odio contra los franceses que se comunicaba de corazón a
corazón de un modo pasmoso, me indujeron a obrar enérgicamente en pro de
los míos. Yo creo que en aquella ocasión memorable hubiérame puesto al
nivel de algunos que me rodeaban, si el recuerdo de Inés y la
consideración de que corría algún peligro no aflojaran mi valor a cada
instante.

Invadiendo la casa, la ocupamos desde el piso bajo a las buhardillas:
por todas las ventanas se hacía fuego arrojando al mismo tiempo cuanto
la diligente valentía de sus moradores encontraba a mano. En el piso
segundo un padre anciano, sosteniendo a sus dos hijas que medio
desmayadas se abrazaban a sus rodillas, nos decía: «Haced fuego; coged
lo que os convenga. Aquí tenéis pistolas; aquí tenéis mi escopeta de
caza. Arrojad mis muebles por el balcón, y perezcamos todos y húndase mi
casa si bajo sus escombros ha de quedar sepultada esa canalla. ¡Viva
Femando! ¡Viva España! ¡Muera Napoleón!»

Estas palabras reanimaban a las dos doncellas, y la menor nos conducía a
una habitación contigua, desde donde podíamos dirigir mejor el fuego.
Pero nos escaseó la pólvora, nos faltó al fin, y al cuarto de hora de
nuestra entrada ya los mamelucos daban violentos golpes en la puerta.

---Quemad las puertas y arrojadlas ardiendo a la calle---nos dijo el
anciano---Ánimo, hijas mías. No lloréis. En este día el llanto es
indigno aun en las mujeres. ¡Viva España! ¿Vosotras sabéis lo que es
España? Pues es nuestra tierra, nuestros hijos, los sepulcros de
nuestros padres, nuestras casas, nuestros reyes, nuestros ejércitos,
nuestra riqueza, nuestra historia, nuestra grandeza, nuestro nombre,
nuestra religión. Pues todo esto nos quieren quitar. ¡Muera Napoleón!

Entretanto los franceses asaltaban la casa, mientras otros de los suyos
cometían las mayores atrocidades en la de Oñate.

---Ya entran, nos cogen y estamos perdidos---exclamamos con terror,
sintiendo que los mamelucos se encarnizaban en los defensores del piso
bajo.

---Subid a la buhardilla---nos dijo el anciano con frenesí,---y saliendo
al tejado, echad por el cañón de la escalera todas las tejas que podáis
levantar. ¿Subirán los caballos de estos monstruos hasta el techo?

Las dos muchachas, medio muertas de terror, se enlazaban a los brazos de
su padre, rogándole que huyese.

---¡Huir!---exclamaba el viejo.---No, mil veces no. Enseñemos a esos
bandoleros cómo se defiende el hogar sagrado. Traedme fuego, fuego, y
apresarán nuestras cenizas, no nuestras personas.

Los mamelucos subían. Estábamos perdidos. Yo me acordé de la pobre Inés,
y me sentí más cobarde que nunca. Pero algunos de los nuestros habíanse
en tanto internado en la casa, y con fuerte palanca rompían el tabique
de una de las habitaciones más escondidas. Al ruido, acudí allá
velozmente, con la esperanza de encontrar escapatoria, y en efecto vi
que habían abierto en la medianería un gran agujero, por donde podía
pasarse a la casa inmediata. Nos hablaron de la otra parte,
ofreciéndonos socorro, y nos apresuramos a pasar; pero antes de que
estuviéramos del opuesto lado sentimos, a los mamelucos y otros soldados
franceses vociferando en las habitaciones principales: oyose un tiro;
después una de las muchachas lanzó un grito espantoso y desgarrador. Lo
que allí debió ocurrir no es para contado.

Cuando pasamos a la casa contigua, con ánimo de tomar inmediatamente la
calle, nos vimos en una habitación pequeña y algo oscura, donde
distinguí dos hombres, que nos miraban con espanto. Yo me aterré también
en su presencia, porque eran el uno el licenciado Lobo, y el otro Juan
de Dios.

Habíamos pasado a una casa de la calle de Postas, a la misma casa en
cuyo cuarto entresuelo había yo vivido hasta el día anterior al servicio
de los Requejos. Estábamos en el piso segundo, vivienda del leguleyo
trapisondista. El terror de este era tan grande que al vernos dijo:

---¿Están ahí los franceses? ¿Vienen ya? Huyamos.

Juan de Dios estaba también tan pálido y alterado, que era difícil
reconocerle.

---¡Gabriel!---exclamó al verme.---¡Ah!, tunante; ¿qué has hecho de
Inés?

---¡Los franceses, los franceses!---exclamó Lobo saliendo a toda prisa
de la habitación y bajando la escalera de cuatro en cuatro
peldaños.---¡Huyamos!

La esposa del licenciado y sus tres hijas, trémulas de miedo, corrían de
aquí para allí, recogiendo algunos objetos para salir a la calle. No era
ocasión de disputar con Juan de Dios, ni de darnos explicaciones sobre
los sucesos de la madrugada anterior, así es que salimos a todo escape,
temiendo que los mamelucos invadieran aquella casa.

El mancebo no se separaba de mí, mientras que Lobo, harto ocupado de su
propia seguridad, se cuidaba de mi presencia tanto como si yo no
existiera.

---¿A dónde vamos?---preguntó una de las niñas al salir.---¿A la calle
de San Pedro la Nueva, en casa de la primita?

---¿Estáis locas? ¿Frente al parque de Monteleón?

---Allí se están batiendo---dijo Juan de Dios.---Se ha empeñado un
combate terrible, porque la artillería española no quiere soltar el
parque.

---¡Dios mío! ¡Corro allá!---exclamé sin poderme contener.

---¡Perro!---gritó Juan de Dios, asiéndome por un brazo.---¿Allí la
tienes guardada?

---Sí, allí está---contesté sin vacilar.---Corramos.

Juan de Dios y yo partimos como dos insensatos en dirección a mi casa.

\hypertarget{xxviii}{%
\chapter{XXVIII}\label{xxviii}}

En nuestra carrera no reparábamos en los mil peligros que a cada paso
ofrecían las calles y plazas de Madrid, y andábamos sin cesar, tomando
las vías más apartadas del centro, con tantas vueltas y rodeos, que
empleamos cerca de dos horas para llegar a la puerta de Fuencarral por
los pozos de nieve. Por un largo rato, ni yo hablaba a mi acompañante,
ni él a mí tampoco, hasta que al fin Juan de Dios, con voz entrecortada
por el fatigoso aliento, me dijo:

---¿Pero tú sacaste a Inés para entregármela después, o eres un tunante
ladrón digno de ser fusilado por los franceses?

---Sr.~Juan de Dios---repuse apretando más el paso.---No es ocasión de
disputar, y vamos más a prisa, porque si los franceses llegan a meterse
en mi casa\ldots{}

---¡Cuánto se asustará la pobrecita! Pero di, ¿por qué la sacaste, por
qué me encontré encerrado en el sótano con aquella maldita mujer\ldots?
¡Oh!, me falta el aliento; pero no nos detengamos\ldots{} ¿Inés no se
asustó al verse en tu poder? ¿No te preguntó por mí, no te rogó que me
llevases a su lado? ¡Qué confusión! ¿Qué es lo que ha pasado? ¿Quién
eres tú? ¿Eres un infame o un hombre de bien? Ya me darás cuenta y razón
de todo. ¡Ay!, cuando me encontré en el sótano con Restituta\ldots{}
¿Ves este rasguño que tengo en la mano?\ldots{} Yo me quedé azorado y
mudo de espanto cuando la vi. ¡Qué desdicha! Creo que fue castigo de
Dios por los pecadillos de que te hablé\ldots{} Ella me insultaba
llamándome ladrón, y a mí un sudor se me iba y otro se me venía. Luego
que tratamos de salir\ldots{} La compuerta cerrada\ldots{} ella parecía
una gata rabiosa. ¿Ves este arañazo que tengo en la cara\ldots?
Descansemos un rato, porque me ahogo. ¿No llegamos nunca a tu casa? ¿Y
mi Inés está allí? Pero tunante, modera un poco el paso y dime: ¿Inés me
espera? ¿Te mandó en busca mía? ¿Sabe que a mí me debe su libertad?
Gabriel, te juro que tengo la cabeza como una jaula de grillos, y que no
sé qué pensar. Cuando vi entrar a Restituta\ldots{} ¿Creerás que no
puedo apartar de mi memoria su repugnante imagen? Lo que dije\ldots{}
aquellos dos pecadillos\ldots{} Pero en cuanto Inés esté a mi lado, me
confesaré\ldots{} El Santísimo Sacramento sabe que mi intención es
buena, y que el inmenso, el loco amor que me domina es causa de
todo\ldots{} ¿Pero no hablas? ¿Estás mudo? ¿Inés me espera? Dímelo
francamente y no me hagas padecer. ¿Está contenta, está triste? ¿Ella
quiso desde luego salir contigo para esperarme fuera?\ldots{} ¡Mil
demonios! ¿Cuándo llegamos a tu casa? Me aguarda, ¿no es verdad? Ahora
le hablaré cara a cara por primera vez. ¿Sabes que me da
vergüenza?\ldots{} Pero ella quizás me dirá primero algunas palabras,
dándome pie para que después siga yo hablando como un cotorro. ¿Estás tú
seguro de que leyó mi carta? Pues si la leyó, ya está al corriente de mi
ardiente amor, y en cuanto me vea se arrojará llorando en mis brazos,
dándome gracias por su salvación. ¿No lo crees tú así? ¿Pero por qué
callas? ¿Te has quedado sin lengua? ¿Qué le has dicho tú, qué te ha
dicho ella? ¿No te habló de aquel pasaje de la carta en que le decía que
mi amor es tan casto como el de los ángeles del cielo?\ldots{} Me faltó
decirle que mi corazón es el altar en que la adoro con tanto fervor como
al Dios que hizo el mundo para todos y para nosotros una isla desierta
llena de flores y pajaritos muy lindos que canten día y noche\ldots{}
¡Ah, Gabriel! ¿Sabes que soy rico? Cogí lo mío, aunque la condenada me
clavó las uñas para arrebatármelo. ¡Cuánto luchamos! ¡Espantosa noche!
Por fin, ya muy avanzado el día, llega D. Mauro y abre el sótano para
sacarte\ldots{} Salimos Restituta y yo; ella está medio muerta. Su
hermano, al vernos\ldots{} ¡Jesús, cómo se pone! Después de insultarnos,
nos dice que tenemos que casarnos el mismo día. Luego, al saber que Inés
se ha fugado contigo, brama como un león, arráncase los cabellos, y
después de amenazar con la muerte a su hermana y a mí, enciende las dos
velas al santo patrono. Yo salgo de la casa sin contestar a nada, y como
ya empiezan los tiros, me refugio en la del licenciado Lobo\ldots{}
Todos están allí llenos de terror\ldots{} los franceses, los
franceses\ldots{} ¡ban, bun!, golpean un tabique, acudimos: se abre un
agujero y apareces tú\ldots{} ¿Pero llegaremos al fin? ¡Qué impaciente
estará la pobrecita! Cuando me vea entrar, ella romperá a hablar, ¿no lo
crees tú? Si no\ldots{} yo estoy seguro de que me quedaré como una
estatua. Si se me quitara esta vergüenza\ldots{}

Yo no contestaba a ninguna de las atropelladas e inconexas razones de
Juan de Dios, pues más que la verbosidad de aquel desgraciado, ocupaba
mi mente la idea de los peligros que corrían Inés y su tío en mi casa.
Nuestra marcha era sumamente fatigosa, pues algunas veces después de
recorrer toda una calle, teníamos que volver atrás huyendo de los
mamelucos: otras veces nos detenía algún grupo compuesto en su mayor
parte de mujeres y ancianos que con lamentos y gritos rodeaban un
cadáver, víctima reciente de los invasores; más adelante veíamos
desfilar precipitadamente pelotones de granaderos que hacían retroceder
a todo el mundo; luego el espectáculo de una lucha parcial, tan
encarnizada como las anteriores, era lo que de improviso nos estorbaba
el paso.

En la calle de Fuencarral el gentío era grande, y todos corrían hacia
arriba, como en dirección al parque. Oíanse fuertes descargas, que
aterraron a mi acompañante, y cuando embocamos a la calle de la Palma
por la casa de Aranda, los gritos de los héroes llegaban hasta nuestros
oídos.

Era entre doce y una. Dando un gran rodeo pudimos al fin entrar en la
calle de San José, y desde lejos distinguí las altas ventanas de mi casa
entre el denso humo de la pólvora.

---No podemos subir a nuestra casa---dije a Juan de Dios,---a menos que
no nos metamos en medio del fuego.

---¡En medio del fuego! ¡Qué horror! No: no expongamos la vida. Veo que
también hacen fuego desde algún balcón. Escondámonos, Gabriel.

---No avancemos. Parece que cesa el fuego.

---Tienes razón. Ya no se oyen sino pocos tiros, y me parece que oigo
decir: «victoria, victoria.»

---Sí, y el paisanaje se despliega, y vienen algunos hacia acá. ¡Ah! ¿No
son franceses aquellos que corren hacia la calle de la Palma? Sí: ¿no ve
Vd. los sombreros de piel?

---Vamos allá. ¡Qué algazara! Parece que están contentos. Mira cómo
agitan las gorras aquellos que están en el balcón.

---Inés, allí está Inés, en el balcón de arriba, arriba\ldots{} Allí
está: mira hacia el parque, parece que tiene miedo y se retira. También
sale a curiosear don Celestino. Corramos y ahora nos será fácil entrar
en la casa.

Después de una empeñada refriega, el combate había cesado en el parque
con la derrota y retirada del primer destacamento francés que fue a
atacarlo. Pero si el crédulo paisanaje se entregó a la alegría creyendo
que aquel triunfo era decisivo; los jefes militares conocieron que
serían bien pronto atacados con más fuerzas, y se preparaban para la
resistencia.

Pacorro Chinitas, que había sido uno de los que primero acudieron a
aquel sitio, se llegó a mí ponderándome la victoria alcanzada con las
cuatro piezas que Daoíz había echado a la calle; pero bien pronto él y
los demás se convencieron de que los franceses no habían retrocedido
sino para volver pronto con numerosa artillería. Así fue en efecto, y
cuando subíamos la escalera de mi casa, sentí el alarmante rumor de la
tropa cercana.

El mancebo tropezaba a cada peldaño, circunstancia que cualquiera
hubiera atribuido al miedo, y yo atribuí a la emoción. Cuando llegamos a
presencia de Inés y D. Celestino, estos se alegraron en extremo de verme
sano, y ella me señaló una imagen de la Virgen, ante la cual habían
encendido dos velas. Juan de Dios permaneció un rato en el umbral, medio
cuerpo fuera y dentro el otro medio, con el sombrero en la mano, el
rostro pálido y contraído, la actitud embarazosa, sin atreverse a hablar
ni tampoco a retirarse, mientras que Inés, enteramente ocupada de mi
vuelta, no ponía en él la menor atención.

---Aquí, Gabriel---me dijo el clérigo,---hemos presenciado escenas de
grande heroísmo. Los franceses han sido rechazados. Por lo visto, Madrid
entero se levanta contra ellos.

Al decir esto, una detonación terrible hizo estremecer la casa.

---¡Vuelven los franceses! Ese disparo ha sido de los nuestros, que
siguen decididos a no entregarse. Dios y su santa Madre, y los cuatro
patriarcas y los cuatro doctores nos asistan.

Juan de Dios continuaba en la puerta, sin que mis dos amigos, hondamente
afectados por el próximo peligro hicieran caso de su presencia.

---Va a empezar otra vez---exclamó Inés huyendo de la ventana después de
cerrarla.---Yo creí que se había concluido. ¡Cuántos tiros! ¡Qué gritos!
¿Pues y los cañones? Yo creí que el mundo se hacía pedazos; y puesta de
rodillas no cesaba de rezar. Si vieras, Gabriel\ldots{} Primero sentimos
que unos soldados daban recios golpes en la puerta del parque. Después
vinieron muchos hombres y algunas mujeres pidiendo armas. Dentro del
patio un español con uniforme verde disputó un instante con otro de
uniforme azul, y luego se abrazaron, abriendo enseguida las puertas.
¡Ay! ¡Qué voces, qué gritos! Mi tío se echó a llorar y dijo también
«¡viva España!» tres veces, aunque yo le suplicaba que callase para no
dar que hablar a la vecindad. Al momento empezaron los tiros de fusil, y
al cabo de un rato los de cañón, que salieron empujados por dos o tres
mujeres\ldots{} El del uniforme azul mandaba el fuego, y otro del mismo
traje, pero que se distinguía del primero por su mayor estatura, estaba
dentro disponiendo cómo se habían de sacar la pólvora y las
balas\ldots{} Yo me estremecía al sentir los cañonazos; y si a veces me
ocultaba en la alcoba, poniéndome a rezar, otras podía tanto la
curiosidad, que sin pensar en el peligro me asomaba a la ventana para
ver todo\ldots{} ¡Qué espanto! Humo, mucho humo, brazos levantados,
algunos hombres tendidos en el suelo y cubiertos de sangre y por todos
lados el resplandor de esos grandes cuchillos que llevan en los fusiles.

Una segunda detonación seguida del estruendo de la fusilería, nos dejó
paralizados de estupor. Inés miró a la Virgen, y el cura encarándose
solemnemente con la santa imagen, dirigiole así la palabra:

---Señora: proteged a vuestros queridos españoles, de quienes fuisteis
reina y ahora sois capitana. Dadles valor contra tantos y tan fieros
enemigos, y haced subir al cielo a los que mueran en defensa de su
patria querida.

Quise abrir la ventana; pero Inés se opuso a ello muy acongojada. Juan
de Dios, que al fin traspasó el umbral, se había sentado tímidamente en
el borde de una silla puesta junto a la misma puerta, donde Inés le
reconoció al fin, mejor dicho, advirtió su presencia, y antes que
formulara una pregunta, le dije yo:

---Es el Sr.~Juan de Dios, que ha venido a acompañarme.

---Yo\ldots{} yo\ldots---balbució el mancebo en el momento en que la
gritería de la calle apenas permitía oírle.---Gabriel habrá enterado a
Vd\ldots{}

---El miedo le quita a Vd. el habla---dijo Inés.---Yo también tengo
mucho miedo. Pero Vd. tiembla, Vd. está malo\ldots{}

En efecto, Juan de Dios parecía desmayarse, y alargaba sus brazos hacia
la muchacha, que absorta y confundida no sabía si acercarse a darle
auxilio o si huir con recelo de visitante tan importuno. Yo estaba an
excitado, que sin parar mientes en lo que junto a mí ocurría, ni atender
al pavor de mi amiga, abrí resueltamente la ventana. Desde allí pude ver
los movimientos de los combatientes, claramente percibidos, cual si
tuviera delante un plano de campaña con figuras movibles. Funcionaban
cuatro piezas: he oído hablar de cinco, dos de a 8 y tres de a 4; pero
yo creo que una de ellas no hizo fuego, o sólo trabajó hacia el fin de
la lucha. Los artilleros me parece que no pasaban de veinte; tampoco
eran muchos los de infantería mandados por Ruiz; pero el número de
paisanos no era escaso ni faltaban algunas heroicas amazonas de las que
poco antes vi en la Puerta del Sol. Un oficial de uniforme azul mandaba
las dos piezas colocadas frente a la calle de San Pedro la
Nueva\footnote{Hoy del Dos de Mayo.}. Por cuenta del otro del mismo
uniforme y graduación corrían las que enfilaban la calle de San Miguel y
de San José\footnote{Hoy de Daoíz y Velarde.}, apuntando una de ellas
hacia la de San Bernardo, pues por allí se esperaban nuevas fuerzas
francesas en auxilio de las que invadían la Palma Alta y sitios
inmediatos a la iglesia de Maravillas. La lucha estaba reconcentrada
entonces en la pequeña calle de San Pedro la Nueva, por donde atacaron
los granaderos imperiales en número considerable. Para contrarrestar su
empuje los nuestros disparaban las piezas con la mayor rapidez posible,
empleándose en ello lo mismo los artilleros que los paisanos; y
auxiliaba a los cañones la valerosa fusilería que tras las tapias del
parque, en la puerta, y en la calle, hacía mortífero e incesante fuego.

\hypertarget{xxix}{%
\chapter{XXIX}\label{xxix}}

Cuando los franceses trataban de tomar las piezas a la bayoneta, sin
cesar el fuego por nuestra parte, eran recibidos por los paisanos con
una batería de navajas, que causaban pánico y desaliento entre los
héroes de las Pirámides y de Jena, al paso que el arma blanca en manos
de estos aguerridos soldados, no hacía gran estrago moral en la gente
española, por ser esta de muy antiguo aficionada a con ella, de modo que
al verse heridos, antes les enfurecía que les desmayaba. Desde mi
ventana abierta a la calle de San José, no se veía la inmediata de San
Pedro la Nueva, aunque la casa hacía esquina a las dos, así es que yo,
teniendo siempre a los españoles bajo mis ojos, no distinguía a los
franceses, sino cuando intentaban caer sobre las piezas, desafiando la
metralla, el plomo, el acero y hasta las implacables manos de los
defensores del parque. Esto pasó una vez, y cuando lo vi pareciome que
todo iba a concluir por el sencillo procedimiento de destrozarse
simultáneamente unos a otros; pero nuestro valiente paisanaje, sublimado
por su propio arrojo y el ejemplo, y la pericia, y la inverosímil
constancia de los dos oficiales de artillería, rechazaba las bayonetas
enemigas, mientras sus navajas, hacían estragos, rematando la obra de
los fusiles.

Cayeron algunos, muchos artilleros, y buen número de paisanos; pero esto
no desalentaba a los madrileños. Al paso que uno de los oficiales de
artillería hacía uso de su sable con fuerte puño sin desatender el cañón
cuya cureña servía de escudo a los paisanos más resueltos, el otro,
acaudillando un pequeño grupo, se arrojaba sobre la avanzada francesa,
destrozándola antes de que tuviera tiempo de reponerse. Eran aquellos
los dos oficiales oscuros y sin historia, que en un día, en una hora,
haciéndose, por inspiración de sus almas generosas, instrumento de la
conciencia nacional, se anticiparon a la declaración de guerra por las
juntas y descargaron los primeros golpes de la lucha que empezó a abatir
el más grande poder que se ha señoreado del mundo. Así sus ignorados
nombres alcanzaron la inmortalidad.

El estruendo de aquella colisión, los gritos de unos y otros, la heroica
embriaguez de los nuestros y también de los franceses, pues estos
evocaban entre sí sus grandes glorias para salir bien de aquel empeño,
formaban un conjunto terrible, ante el cual no existía el miedo, ni
tampoco era posible resignarse a ser inmóvil espectador. Causaba rabia y
al mismo tiempo cierto júbilo inexplicable lo desigual de las fuerzas, y
el espectáculo de la superioridad adquirida por los débiles a fuerza de
constancia. A pesar de que nuestras bajas eran inmensas, todo parecía
anunciar una segunda victoria. Así lo comprendían sin duda los
franceses, retirados hacia el fondo de la calle de San Pedro la Nueva; y
viendo que para meter en un puño a los veinte artilleros ayudados de
paisanos y mujeres, era necesaria más tropa con refuerzos de todas
armas, trajeron más gente, trajeron un ejército completo; y la división
de San Bernardino, mandada por Lefranc apareció hacia las Salesas Nuevas
con varias piezas de artillería. Los imperiales daban al parque cercado
de mezquinas tapias las proporciones de una fortaleza, y a la abigarrada
pandilla las proporciones de un pueblo.

Hubo un momento de silencio, durante el cual no oí más voces que las de
algunas mujeres, entre las cuales reconocí la de la Primorosa,
enronquecida por la fatiga y el perpetuo gritar. Cuando en aquel breve
respiro me aparté de la ventana, vi a Juan de Dios completamente
desvanecido. Inés estaba a su lado, presentándole un vaso de agua.

---Este buen hombre---dijo la muchacha,---ha perdido el tino. ¡Tan
grande es su pavor! Verdad que la cosa no es para menos. Yo estoy
muerta. ¿Se ha acabado, Gabriel? Ya no se oyen tiros. ¿Ha concluido
todo? ¿Quién ha vencido?

Un cañonazo resonó estremeciendo la casa. A Inés cayósele el vaso de las
manos, y en el mismo instante entró D. Celestino, que observaba la lucha
desde otra habitación de la casa.

---Es la artillería francesa---exclamó.---Ahora es ella. Traen más de
doce cañones. ¡Jesús, María y José nos amparen! Van a hacer polvo a
nuestros valientes paisanos. ¡Señor de justicia! ¡Virgen María, santa
patrona de España!

Juan de Dios abrió sus ojos buscando a Inés con una mirada calmosa y
apagada como la de un enfermo. Ella, en tanto, puesta de rodillas ante
la imagen, derramaba abundantes lágrimas.

---Los franceses son innumerables---continuó el cura.---Vienen cientos
de miles. En cambio los nuestros, son menos cada vez. Muchos han muerto
ya. ¿Podrán resistir los que quedan? ¡Oh! Gabriel, y usted, caballero,
quien quiera que sea, aunque presumo será español: ¿están Vds. en paz
con su conciencia, mientras nuestros hermanos pelean abajo por la patria
y por el Rey? Hijos míos, ánimo: los franceses van a atacar por tercera
vez. ¿No veis cómo se aperciben los nuestros para recibirlos con tanto
brío como antes? ¿No oís los gritos de los que han sobrevivido al último
combate? ¿No oís las voces de esa noble juventud? Gabriel, Vd.,
caballero, cualquiera que sea, ¿habéis visto a las mujeres? ¿Darán
lección de valor esas heroicas hembras a los varones que huyen de la
honrosa lucha?

Al decir esto, el buen sacerdote, con una alteración que hasta entonces
jamás había advertido en él, se asomaba al balcón, retrocedía con
espanto, volvía los ojos a la imagen de la Virgen, luego a nosotros, y
tan pronto hablaba consigo mismo como con los demás.

---Si yo tuviera quince años, Gabriel---continuó,---si yo tuviera tu
edad\ldots{} Francamente, hijos míos, yo tengo muchísimo miedo. En mi
vida había visto una guerra, ni oído jamás el estruendo de los
mortíferos cañones; pero lo que es ahora cogería un fusil, sí señores,
lo cogería\ldots{} ¿No veis que va escaseando la gente? ¿No veis cómo
los barre la metralla?\ldots{} Mirad aquellas mujeres que con sus brazos
despedazados empujan uno de nuestros cañones hasta embocarle en esta
calle. Mirad aquel montón de cadáveres del cual sale una mano increpando
con terrible gesto a los enemigos. Parece que hasta los muertos hablan,
lanzando de sus bocas exclamaciones furiosas\ldots{} ¡Oh!, yo tiemblo,
sostenedme; no, dejadme tomar un fusil, lo tomaré yo. Gabriel,
caballero, y tú también, Inés; vamos todos a la calle, a la calle. ¿Oís?
Aquí llegan las vociferaciones de los franceses. Su artillería avanza.
¡Ah!, perros: todavía somos suficientes, aunque pocos. ¿Queréis a
España, queréis este suelo? ¿Queréis nuestras casas, nuestras iglesias,
nuestros reyes, nuestros santos? Pues ahí está, ahí está dentro de esos
cañones lo que queréis. Acercaos\ldots{} ¡Ah! Aquellos hombres que
hacían fuego desde la tapia han perecido todos. No importa. Cada muerto
no significa más sino que un fusil cambia de mano, porque antes de que
pierda el calor de los dedos heridos que lo sueltan, otros lo
agarran\ldots{} Mirad: el oficial que los manda parece contrariado, mira
hacia el interior del parque y se lleva la mano a la cabeza con ademán
de desesperación. Es que les faltan balas, les falta metralla. Pero
ahora sale el otro con una cesta de piedras\ldots{} sí\ldots{} son
piedras de chispa. Cargan con ellas, hacen fuego\ldots{} ¡Oh!, que
vengan, que vengan ahora. ¡Miserables! España tiene todavía piedras en
sus calles para acabar con vosotros\ldots{} Pero ¡ay!, los franceses
parece que están cerca. Mueren muchos de los nuestros. Desde los
balcones se hace mucho fuego; mas esto no basta. Si yo tuviera veinte
años\ldots{} Si yo tuviera veinte años, tendría el valor que ahora me
falta, y me lanzaría en medio del combate, y a palos, sí señores, a
palos, acabaría con todos esos franceses. Ahora mismo, con mis sesenta
años\ldots{} Gabriel, ¿sabes tú lo que es el deber? ¿Sabes tú lo que es
el honor? Pues para que lo sepas, oye: Yo que soy un viejo inútil, yo
que nunca he visto un combate, yo que jamás he disparado un tiro, yo que
en mi vida he peleado con nadie, yo que no puedo ver matar un pollo, yo
que nunca he tenido valor para matar un gusanito, yo que siempre he
tenido miedo a todo, yo que ahora tiemblo como una liebre y a cada tiro
que oigo parece que entrego el alma al Señor, voy a bajar al instante a
la calle, no con armas, porque armas no me corresponden, sino para
alentar a esos valientes, diciéndoles en castellano aquello de
\emph{¡Dulce et decorum est pro patria mori!}

Estas palabras, dichas con un entusiasmo que el anciano no había
manifestado ante mí sino muy pocas veces, y siempre desde el púlpito, me
enardeció de tal modo que me avergoncé de reconocerme cobarde espectador
de aquella heroica lucha sin disparar un tiro, ni lanzar una piedra en
defensa de los míos. A no contenerme la presencia de Inés, ni un
instante habría yo permanecido en aquella situación. Después cuando vi
al buen anciano precipitarse fuera de la casa, dichas sus últimas
palabras, miedo y amor se oscurecieron en mí ante una grande, una
repentina iluminación de entusiasmo, de esas que rarísimas veces, pero
con fuerza poderosa, nos arrastran a las grandes acciones.

Inés hizo un movimiento como para detenerme pero sin duda su admirable
buen sentido comprendió cuánto habría desmerecido a mis propios ojos
cediendo a los reclamos de la debilidad, y se contuvo ahogando todo
sentimiento. Juan de Dios, que al volver de su desmayo era completamente
extraño a la situación que nos encontrábamos, y no parecía tener ojos ni
oídos más que para espectáculos y voces de su propia alma, se adelantó
hacia Inés con ademán embarazoso, y le dijo:

---Pero Gabriel la habrá enterado a Vd. de todo. ¿La he ofendido a Vd.
en algo? Bien habrá comprendido Vd\ldots{}

---Este caballero---dijo Inés,---está muerto de miedo, y no se moverá de
aquí. ¿Quiere Vd. esconderse en la cocina?

---¡Miedo! ¡Que yo tengo miedo!---exclamó el mancebo con un repentino
arrebato que le puso encendido como la grana.---¿A dónde vas, Gabriel?

---A la calle---respondí saliendo.---A pelear por España. Yo no tengo
miedo.

---Ni yo, ni yo tampoco---afirmó resuelta, furiosamente Juan de Dios
corriendo detrás de mí.

\hypertarget{xxx}{%
\chapter{XXX}\label{xxx}}

Llegué a la calle en momentos muy críticos. Las dos piezas de la calle
de San Pedro habían perdido gran parte de su gente, y los cadáveres
obstruían el suelo. La colocada hacia Poniente había de resistir el
fuego de la de los franceses, sin más garantía de superioridad que el
heroísmo de D. Pedro Velarde y el auxilio de los tiros de fusil. Al dar
los primeros pasos encontré uno, y me situé junto a la entrada del
parque, desde donde podía hacer fuego hacia la calle Ancha, resguardado
por el machón de la puerta. Allí se me presentó una cara conocida,
aunque horriblemente desfigurada, en la persona de Pacorro Chinitas, que
incorporándose entre un montón de tierra y el cuerpo de otro infeliz ya
moribundo, hablome así con voz desfallecida:

---Gabriel, yo me acabo; yo no sirvo ya para nada.

---Ánimo, Chinitas---dije devolviéndole el fusil que caía de sus
manos---levántate.

---¿Levantarme? Ya no tengo piernas. ¿Traes tú pólvora? Dame acá: yo te
cargaré el fusil\ldots{} Pero me caigo redondo. ¿Ves esta sangre? Pues
es toda mía y de este compañero que ahora se va\ldots{} Ya
expiró\ldots{} Adiós, Juancho: tú al menos no verás a los franceses en
el parque.

Hice fuego repetidas veces, al principio muy torpemente, y después con
algún acierto, procurando siempre dirigir los tiros a algún francés
claramente destacado de los demás. Entre tanto, y sin cesar en mi faena,
oí la voz del amolador que apagándose por grados decía:

«Adiós, Madrid, ya me encandilo\ldots{} Gabriel, apunta a la cabeza.
Juancho que ya estás tieso, allá voy yo también: Dios sea conmigo y me
perdone. Nos quitan el parque; pero de cada gota de esta sangre saldrá
un hombre con su fusil, hoy, mañana y al otro día. Gabriel, no cargues
tan fuerte, que revienta. Ponte más adentro. Si no tienes navaja,
búscala, porque vendrán a la bayoneta. Toma la mía. Allí está junto a la
pierna que perdí\ldots{} ¡Ay!, ya no veo más que un cielo negro. ¡Qué
humo tan negro! ¿De dónde viene ese humo? Gabriel, cuando esto se acabe,
¿me darás un poco de agua? ¡Qué ruido tan atroz!\ldots{} ¿Por qué no
traen agua? ¡Agua, Señor Dios Poderoso! ¡Ah!, ya veo el agua; ahí está.
La traen unos angelitos; es un chorro, una fuente, un río\ldots»

Cuando me aparté de allí, Chinitas ya no existía. La debilidad de
nuestro centro de combate me obligó a unirme a él, como lo hicieron los
demás. Apenas quedaban artilleros, y dos mujeres servían la pieza
principal, apuntaban hacia la calle Ancha. Era una de ellas la
Primorosa, a quien vi soplando fuertemente la mecha, próxima a
extinguirse.

---Mi general---decía a Daoíz.---Mientras su merced y yo estemos aquí,
no se perderán las Españas ni sus Indias\ldots{} Allá va el
petardo\ldots{} Venga ahora acá el \emph{destupidor}. Cómo rempuja pa
tras este animal cuando suelta el tiro. ¡Ah! ¿Ya estás aquí,
Tripita?---gritó al verme.---Toca este instrumento y verás lo bueno.

El combate llegaba a un extremo de desesperación; y la artillería
enemiga avanzó hacia nosotros. Animados por Daoíz, los heroicos paisanos
pudieron rechazar por última vez la infantería francesa que se destacaba
en pequeños pelotones de la fuerza enemiga.

---¡Ea!---gritó la Primorosa cuando recomenzó el fuego de
cañón.---Atrás, que yo gasto malas bromas. ¿Vio Vd. cómo se fueron,
señor general? Sólo con mirarles yo con estos recelestiales ojos, les
hice volver pa tras. Van muertos de miedo. ¡Viva España y muera
Napoleón!\ldots{} Chinitas, ¿no está por ahí Chinitas? Ven acá, cobarde,
calzonazos.

Y cuando los franceses, replegando su infantería, volvieron a
cañonearnos, ella, después de ayudar a cargar la pieza, prosiguió
gritando desesperadamente:

---Renacuajos, volved acá. Ea, otro paseíto. Sus mercedes quieren
conquistarme a mí, ¿no verdá? Pues aquí me tenéis. Vengan acá: soy la
reina, sí señores, soy la emperadora del Rastro, y yo acostumbro a fumar
en este cigarro de bronce, porque no las gasto menos. ¿Quieren ustedes
una chupadita? Pos allá va. Desapártense pa que no les salpique la
saliva; si no\ldots{}

La heroica mujer calló de improviso, porque la otra maja que cerca de
ella estaba, cayó tan violentamente herida por un casco de metralla, que
de su despedazada cabeza saltaron salpicándonos repugnantes pedazos. La
esposa de Chinitas, que también estaba herida, miró el cuerpo expirante
de su amiga. Debo consignar aquí un hecho trascendental; la Primorosa se
puso repentinamente pálida, y repentinamente seria. Tuvo miedo.

Llegó el instante crítico y terrible. Durante él sentí una mano que se
apoyaba en mi brazo. Al volver los ojos vi un brazo azul con charreteras
de capitán. Pertenecía a D. Luis Daoíz, que herido en la pierna, hacía
esfuerzos por no caer al suelo y se apoyaba en lo que encontró más
cerca. Yo extendí mi brazo alrededor de su cintura, y él, cerrando los
puños, elevándolos convulsamente al cielo, apretando los dientes y
mordiendo después el pomo de su sable, lanzó una imprecación, una
blasfemia, que habría hecho desplomar el firmamento, si lo de arriba
obedeciera a las voces de abajo.

En seguida se habló de capitulación y cesaron los fuegos. El jefe de las
fuerzas francesas acercose a nosotros, y en vez de tratar decorosamente
de las condiciones de la rendición, habló a Daoíz de la manera más
destemplada y en términos amenazadores y groseros. Nuestro inmortal
artillero pronunció entonces aquellas célebres palabras: \emph{Si
fuérais capaz de hablar con vuestro sable, no me trataríais así}.

El francés, sin atender a lo que le decía, llamó a los suyos, y en el
mismo instante\ldots{} Ya no hay narración posible, porque todo acabó.
Los franceses se arrojaron sobre nosotros con empuje formidable. El
primero que cayó fue Daoíz, traspasado el pecho a bayonetazos.
Retrocedimos precipitadamente hacia el interior del parque todos los que
pudimos, y como aun en aquel trance espantoso quisiera contenernos D.
Pedro Velarde, le mató de un pistoletazo por la espalda un oficial
enemigo. Muchos fueron implacablemente pasados a cuchillo; pero algunos
y yo pudimos escapar, saltando velozmente por entre escombros, hasta
alcanzar las tapias de la parte más honda, y allí nos dispersamos,
huyendo cada cual por donde encontró mejor camino, mientras los
franceses, bramando de ira, indicaban con sus alaridos al aterrado
vecindario que Monteleón había quedado por Bonaparte.

Difícilmente salvamos la vida, y no fuimos muchos los que pudimos dar
con nuestros fatigados cuerpos en la huerta de las Salesas Nuevas o en
el quemadero. Los franceses no se cuidaban de perseguirnos, o por creer
que bastaba con rematar a los más próximos, o porque se sentían con
tanto cansancio como nosotros. Por fortuna, yo no estaba herido sino muy
levemente en la cabeza, y pude ponerme a cubierto en breve tiempo: al
poco rato ya no pensaba más que en volver a mi casa, donde suponía a
Inés en penosa angustia por mi ausencia. Cuando traté de regresar hallé
cerrada la puerta de Santo Domingo; y tuve que andar mucho trecho
buscando el portillo de San Joaquín. Por el camino me dijeron que los
franceses, después de dejar una pequeña guarnición en el parque, se
habían retirado.

Dirigime con esta noticia tranquilamente a casa, y al llegar a la calle
de San José, encontré aquel sitio inundado de gente del pueblo,
especialmente de mujeres, que reconocían los cadáveres. La Primorosa
había recogido el cuerpo de Chinitas. Yo vi llevar el cuerpo, vivo aún,
de Daoíz en hombros de cuatro paisanos, y seguido de apiñado gentío. D.
Pedro Velarde oí que había sido completamente desnudado por los
franceses, y en aquellos instantes sus deudos y amigos estaban
amortajándole para darle sepultura en San Marcos. Los imperiales se
ocupaban en encerrar de nuevo las piezas, y retiraban silenciosamente
sus heridos al interior del parque: por último, vi una pequeña fuerza de
caballería polaca, estacionada hacia la calle de San Miguel.

Ya estaba cerca de mi casa, cuando un hombre cruzó a lo lejos la calle,
con tan marcado ademán de locura, que no pude menos de fijar en él mi
atención. Era Juan de Dios, y andaba con pie inseguro de aquí para allí
como demente o borracho, sin sombrero, el pelo en desorden sobre la
cara, las ropas destrozadas y la mano derecha envuelta en un pañuelo
manchado de sangre.

---¡Se la han llevado!---exclamó al verme, agitando sus brazos con
desesperación.

---¿A quién?---pregunté, adivinando mi nueva desgracia.

---¡A Inés!\ldots{} Se la han llevado los franceses; se han llevado
también a aquel infeliz sacerdote.

La sorpresa y la angustia de tan tremenda nueva me dejaron por un
instante como sin vida.

\hypertarget{xxxi}{%
\chapter{XXXI}\label{xxxi}}

---Una vez que tomaron el parque---continuó Juan de Dios,---entraron en
esa casa de la esquina y en otra de la calle de San Pedro para prender a
todos los que les habían hecho fuego, y sacaron hasta dos docenas de
infelices. ¡Ay, Gabriel, qué consternación! Yo entraba en la taberna
para echarme un poco de agua en la mano\ldots{} porque sabrás que una
bala me llevó los dos dedos\ldots{} entraba en la taberna y vi que
sacaban a Inés. La pobrecita lloraba como un niño y volvía la vista a
todos lados, sin duda buscándome con sus ojos. Acerqueme, y hablando en
francés, rogué al sargento que la soltase; pero me dieron tan fuerte
golpe que casi perdí el sentido. ¡Si vieras cómo lloraba el pobre ángel,
y cómo miraba a todos lados, buscándome sin duda!\ldots{} Yo me vuelvo
loco, Gabriel. El buen eclesiástico subía la escalera cuando lo
cogieron, y dicen que llevaba un cuchillo en la mano. Todos los de la
casa están presos. Los franceses dijeron que desde allí les habían
tirado una cazuela de agua hirviendo. Gabriel, si no ponen en libertad a
Inés, yo me muero, yo me mato, yo les diré a los franceses que me maten.

Al oír esta relación, el vivo dolor arrancó al principio ardientes
lágrimas a mis ojos; pero después fue tanta mi indignación, que
prorrumpí en exclamaciones terribles y recorrí la calle gritando como un
insensato. Aún dudé; subí a mi casa, encontrela desierta; supe de boca
de algunos vecinos consternados la verdad, tal como Juan de Dios me la
había dicho, y ciego de ira, con el alma llena de presentimientos
siniestros, y de inexplicables angustias, marché hacia el centro de
Madrid, sin saber a dónde me encaminaba, y sin que me fuera posible
discurrir cuál partido sería más conveniente en tales circunstancias. ¿A
quién pedir auxilio, si yo a mi vez era también injustamente perseguido?
A ratos me alentaba la esperanza de que los franceses pusieran en
libertad a mis dos amigos. La inocencia de uno y otro, especialmente de
ella, era para mí tan obvia, que sin género de duda había de ser
reconocida por los invasores. Juan de Dios me seguía, y lloraba como una
mujer.

---Por ahí van diciendo---me indicó,---que los prisioneros han sido
llevados a la casa de Correos. Vamos allá, Gabriel, y veremos si
conseguimos algo.

Fuimos al instante a la Puerta del Sol, y en todo su recinto no oíamos
sino quejas y lamentos, por el hermano, el padre, el hijo o el amigo,
bárbaramente aprisionados sin motivo. Se decía que en la casa de Correos
funcionaba un tribunal militar; pero después corrió la voz de que los
individuos de la junta habían hecho un convenio con Murat, para que todo
se arreglara, olvidando el conflicto pasado y perdonándose
respectivamente las imprudencias cometidas. Esto nos alborozó a todos
los presentes, aunque no nos parecía muy tranquilizador ver a la entrada
de las principales calles una pieza de artillería con mecha encendida.
Dieron las cuatro de la tarde, y no se desvanecía nuestra duda, ni de
las puertas de la fatal casa de Correos salía otra gente que algún
oficial de órdenes que a toda prisa partía hacia el Retiro o la Montaña.
Nuestra ansiedad crecía; profunda zozobra invadía los ánimos, y todos se
dispersaban tratando de buscar noticias verídicas en fuentes
autorizadas.

De pronto oigo decir que alguien va por las calles leyendo un bando.
Corremos todos hacia la del Arenal, pero no nos es posible enterarnos de
lo que leen. Preguntamos y nadie nos responde, porque nadie oye.
Retrocedemos pidiendo informes, y nadie nos los da. Volvemos a mirar la
casa de Correos tras cuyas paredes están los que nos son queridos, y
media compañía de granaderos con algunos mamelucos dispersan al padre,
al hermano, al hijo, al amante, amenazándoles con la muerte. Nos vamos
al fin por las calles, cada cual discurriendo qué influencias pondrá en
juego para salvar a los suyos.

Juan de Dios y yo nos dirigimos hacia los Caños del Peral, y al poco
rato vimos un pelotón de franceses que conducían maniatados y en traílla
como a salteadores, a dos ancianos y a un joven de buen porte. Después
de esta fatídica procesión, vimos hacia la calle de los Tintes otra no
menos lúgubre, en que iban una señora joven, un sacerdote, dos
caballeros y un hombre del pueblo en traje como de vendedor de plazuela.
La tercera la encontramos en la calle de Quebrantapiernas, y se componía
de más de veinte personas, pertenecientes a distintas clases de la
sociedad. Aquellos infelices iban mudos y resignados guardando el odio
en sus corazones, y ya no se oían voces patrióticas en las calles de la
ciudad vencida y aherrojada, porque los invasores dominábanla toda
piedra por piedra, y no había esquina donde no asomase la boca de un
cañón, ni callejuela por la cual no desfilaran pelotones de fusileros,
ni plaza donde no apareciesen, fúnebremente estacionados, fuertes
piquetes de mamelucos, dragones o caballería polaca.

Repetidas veces vimos que detenían a personas pacíficas y las
registraban, llevándoselas presas por si acertaban a guardar acaso algún
arma, aunque fuera navaja para usos comunes. Yo llevaba en el bolsillo
la de Chinitas, y ni aun se me ocurrió tirarla, ¡tales eran mi
aturdimiento y abstracción! Pero tuvimos la suerte de que no nos
registraran. Últimamente y a medida que anochecía, apenas encontrábamos
gente por las calles. No íbamos, no, a la ventura por aquellos desiertos
lugares, pues yo tenía un proyecto que al fin comuniqué a mi
acompañante; pensaba dirigirme a casa de la marquesa, con viva esperanza
de conseguir de ella poderoso auxilio en mi tribulación. Juan de Dios me
contestó que él por su parte había pensado dirigirse a un amigo que a su
vez lo era del Sr.~O'Farril, individuo de la Junta. Dicho esto,
convinimos en separarnos, prometiendo acudir de nuevo a la Puerta del
Sol una hora después.

Fui a casa de la Marquesa, y el portero me dijo que Su Excelencia había
partido dos días antes para Andalucía. También pregunté por Amaranta;
mas tuve el disgusto de saber que Su Excelencia la señora condesa estaba
en camino de Andalucía. Desesperado regresé al centro de Madrid,
elevando mis pensamientos a Dios, como el más eficaz amparador de la
inocencia, y traté de penetrar en la casa de Correos. Al poco rato de
estar allí procurándolo inútilmente, vi salir a Juan de Dios tan pálido
y alterado que temblé adivinando nuevas desdichas.

---¿No está?---pregunté.---¿Los han puesto en libertad?

---No---dijo secando el sudor de su frente.---Todos los presos que
estaban aquí han sido entregados a los franceses. Se los han llevado al
Buen Suceso, al Retiro, no sé a dónde\ldots{} ¿Pero no conoces el bando?
Los que sean encontrados con armas, \emph{serán arcabuceados}\ldots{}
Los que se junten en grupo de más de ocho personas, serán
arcabuceados\ldots{} Los que hagan daño a un francés, \emph{serán
arcabuceados}\ldots{} Los que parezcan agentes de Inglaterra,
\emph{serán arcabuceados}.

---¿Pero dónde está Inés?---exclamé con exaltación.---¿Dónde está? Si
esos verdugos son capaces de sacrificar a una niña inocente, y a un
pobre anciano, la tierra se abrirá para tragárselos, las piedras se
levantarán solas del suelo para volar contra ellos, el cielo se
desplomará sobre sus cabezas, se encenderá el aire, y el agua que beban
se les tornará veneno; y si esto no sucede, es que no hay Dios ni puede
haberlo. Vamos, amigo: hagamos esta buena obra. ¿Dice Vd. que están en
el Retiro?

---O aquí en el Buen Suceso, o en la Moncloa. Gabriel, yo salvaré a Inés
de la muerte, o me pondré delante de los fusiles de esa canalla para que
me quiten también la vida. Quiero irme al cielo con ella; si supiera que
sus dulces ojos no me habían de mirar más en la tierra, ahora mismo
dejaría de existir. Gabriel, todo lo que tengo es tuyo si me ayudas a
buscarla; que después que ella y yo nos juntemos, y nos casemos, y nos
vayamos al lugar desierto que he pensado, para nada necesitamos dinero.
Yo tengo esperanza; ¿y tú?

---Yo también---respondí, pensando en Dios.

---Pues, hijo, marcha tú al Retiro, que yo entraré en el Buen Suceso,
por la parte del hospital, que allí conozco a uno de los enfermeros.
También conozco a dos oficiales franceses. ¿Podrán hacer algo por ella?
Vamos: las diez. ¡Ay! ¿No oíste una descarga?

---Sí, hacia abajo; hacia el Prado: se me ha helado la sangre en las
venas. Corre allá. Adiós, y buena suerte. Si no nos encontramos después
aquí, en mi casa.

Dicho esto, nos separamos a toda prisa, y yo corrí por la Carrera de San
Jerónimo. La noche era oscura, fría y solitaria. En mi camino encontré
tan sólo algunos hombres que corrían despavoridos, y a cada paso
lamentos dolorosísimos llegaban a mis oídos. A lo lejos distinguí las
pisadas de las patrullas francesas y de rato en rato un resplandor
lejano seguido de estruendosa detonación.

\hypertarget{xxxii}{%
\chapter{XXXII}\label{xxxii}}

Cómo se presentaba en mi alma atribulada aquel espectáculo en la negra
noche, aquellos ruidos pavorosos, no es cosa que puedo yo referir, ni
palabras de ninguna lengua alcanzan a manifestar angustia tan grande.
Llegaba junto al Espíritu Santo, cuando sentí muy cercana ya una
descarga de fusilería. Allá abajo en la esquina del palacio de
Medinaceli la rápida luz del fogonazo, había iluminado un grupo, mejor
dicho, un montón de personas, en distintas actitudes colocadas, y con
diversos trajes vestidos. Tras de la detonación, oyéronse quejidos de
dolor, imprecaciones que se apagaban al fin en el silencio de la noche.
Después algunas voces hablando en lengua extranjera, dialogaban entre
sí; se oían las pisadas de los verdugos, cuya marcha en dirección al
fondo del Prado era indicada por los movimientos de unos farolillos de
agonizante luz. A cada rato circulaban pequeños tropeles, con gentes
maniatadas, y hacia el Retiro se percibía resplandor muy vivo, como de
la hoguera de un vivac.

Acerqueme al palacio de Medinaceli por la parte del Prado, y allí vi
algunas personas que acudían a reconocer los infelices últimamente
arcabuceados. Reconocilos yo también uno por uno, y observé que pequeña
parte de ellos estaban vivos, aunque ferozmente heridos; y arrastrábanse
estos pidiendo socorro, o clamaban en voz desgarradora suplicando que se
les rematase.

Entre todas aquellas víctimas no había más que una mujer, que no tenía
semejanza con Inés, ni encontré tampoco sacerdote alguno. Sin prestar
oídos a las voces de socorro, ni reparar tampoco en el peligro que cerca
de allí se corría, me dirigí hacia el Retiro.

En la puerta que se abría al primer patio me detuvieron los centinelas.
Un oficial se acercó a la entrada.

---Señor---exclamé juntando las manos y expresando de la manera más
espontánea el vivo dolor que me dominaba,---busco a dos personas de mi
familia que han sido traídas aquí por equivocación. Son inocentes: Inés
no arrojó a la calle ningún caldero de agua hirviendo, ni el pobre
clérigo ha matado a ningún francés. Yo lo aseguro, señor oficial, y el
que dijese lo contrario es un vil mentiroso.

El oficial, que no entendía, hizo un movimiento para echarme hacia
fuera; pero yo, sin reparar en consideraciones de ninguna clase, me
arrodillé delante de él, y con fuertes gritos proseguí suplicando de
esta manera:

---Señor oficial, ¿será Vd. tan inhumano que mande fusilar a dos
personas inofensivas, a una muchacha de diez y seis años y a un infeliz
viejo de sesenta! No puede ser. Déjeme Vd. entrar; yo le diré cuáles
son, y Vd. les mandará poner en libertad. Los pobrecitos no han hecho
nada. Fusílenme a mí, que disparé muchos tiros contra Vds. en la acción
del parque; pero dejen en libertad a la muchacha y al sacerdote. Yo
entraré, les sacaremos\ldots{} Mañana, mañana probaré yo, como esta es
noche, que son inocentes, y si no resultasen tan inocentes como los
ángeles del cielo, fusíleme Vd. a mí cien veces. Señor oficial, Vd. es
bueno, Vd. no puede ser un verdugo. Esas cruces que tiene en el pecho
las habrá adquirido honrosamente en las grandes batallas que dicen ha
ganado el ejército de Napoleón. Un hombre como usted no puede
deshonrarse asesinando a mujeres inocentes. Yo no lo creo, aunque me lo
digan. Señor oficial, si quieren Vds. vengarse de lo de esta mañana
maten a todos los hombres de Madrid, mátenme a mí también; pero no a
Inés. ¿Vd. no tiene hermanitas jóvenes y lindas? Si Vd. las viera
amarradas a un palo, a la luz de una linterna, delante de cuatro
soldados con los fusiles en la cara, ¿estaría tan sereno como ahora
está? Déjeme entrar: yo le diré quiénes son los que busco, y entre los
dos haremos esta buena obra que Dios le tendrá en cuenta cuando se
muera. El corazón me dice que están aquí\ldots{} entremos, por Dios y
por la Virgen. Vd. está aquí en tierra extranjera, y lejos, muy lejos de
los suyos. Cuando recibe cartas de su madre o de sus hermanitas, ¿no le
rebosa el corazón de alegría, no quiere verlas, no quiere volver allá?
Si le dijesen que ahora las estaban poniendo un farol en el pecho para
fusilarlas\ldots{}

El estrépito de otra descarga me hizo enmudecer, y la voz expiró en mi
garganta por falta de aliento. Estuve a punto de caer sin sentido; pero
haciendo un heroico esfuerzo, volví a suplicar al oficial con voz ronca
y ademán desesperado, pretendiendo que me dejase entrar a ver si algunos
de los recién inmolados eran los que yo buscaba. Sin duda mi ruego,
expresado ardientemente y con profundísima verdad, conmovió al joven
oficial, más por la angustia de mis ademanes que por el sentido de las
palabras, extranjeras para él, y apartándose a un lado me indicó que
entrara. Hícelo rápidamente, y recorrí como un insensato el primer patio
y el segundo. En este, que era el de la Pelota, no había más que
franceses; pero en aquel yacían por el suelo las víctimas aún
palpitantes, y no lejos de ellas las que esperaban la muerte. Vi que las
ataban codo con codo, obligándolas a ponerse de rodillas, unos de
espalda, otros de frente. Los más extendían los brazos agitándolos al
mismo tiempo que lanzaban imprecaciones y retos a los verdugos; algunos
escondían con horror la cara en el pecho del vecino; otros lloraban;
otros pedían la muerte, y vi uno que rompiendo con fuertes sacudidas las
ligaduras, se abalanzó hacia los granaderos. Ninguna fórmula de juicio,
ni tampoco preparación espiritual, precedían a esta abominación: los
granaderos hacían fuego una o dos veces, y los sacrificados se revolvían
en charcos de sangre con espantosa agonía.

Algunos acababan en el acto; pero los más padecían largo martirio antes
de expirar, y hubo muchos que heridos por las balas en las extremidades
y desangrados, sobrevivieron después de pasar por muertos hasta la
mañana del día 3, en que los mismos franceses, reconociendo su mala
puntería, les mandaron al hospital. Estos casos no fueron raros, y yo sé
de dos o tres a quienes cupo la suerte de vivir después de pasar por los
horrores de una ejecución sangrienta. Un maestro herrero, comprendido en
una de las traíllas del Retiro, dio señales de vida al día siguiente, y
al borde mismo del hoyo en que se le preparaba sepultura: lo mismo
aconteció a un tendero de la calle de Carretas, y hasta hace poco tiempo
ha existido uno que era entonces empleado en la imprenta de Sancha, y
fue fusilado torpemente dos veces, una en la Soledad, donde se hizo la
primera matanza, después en el patio del Buen Suceso, desde cuyo sitio
pudo escapar, arrastrándose entre cadáveres y regueros de sangre hasta
el hospital cercano, donde le dieron auxilio. Los franceses, aunque a
quema-ropa, disparaban mal, y algunos de ellos, preciso es confesarlo,
con marcada repugnancia, pues sin duda conocían el envilecimiento en que
habían repentinamente caído las águilas imperiales.

Casi sin esperar a que se consumara la sentencia de los que cayeron ante
mí, les examiné a todos. Las linternas, puestas delante de cada grupo,
alumbraban con siniestra luz la escena. Ni entre los inmolados ni entre
los que aguardaban el sacrificio, vi a Inés ni a D. Celestino, aunque a
veces me parecía reconocerles en cualquier bulto que se movía implorando
compasión o murmurando una plegaria.

Recuerdo que en aquel examen una mano helada cogió la mía, y al
inclinarme vi un hombre desconocido que dijo algunas palabras y expiró.
Repetidas veces pisé los pies y las manos de varios desgraciados; pero
en trances tan terribles, parece que se extingue todo sentimiento
compasivo hacia los extraños, y buscando con anhelo a los nuestros,
somos impasibles para las desgracias ajenas.

Algunos franceses me obligaron a alejar de aquel sitio; y por las
palabras que oí me juzgué en peligro de ser también comprendido en la
traílla pero a mí no me importaba la muerte, ni en tal situación hubiera
dejado de mirar a un punto donde creyera distinguir el semblante de mis
dos amigos, aunque me arcabucearan cien veces. Corrí hacia otro extremo
del patio, donde sonaban lamentos y mucha bulla de gente, cuando un
anciano se acercó a mí tomándome por el brazo.

---¿A quién busca Vd.?---le dije.

---¡Mi hijo, mi único hijo!---me contestó.---¿Dónde está? ¿Eres tú mi
hijo? ¿Eres tú mi Juan? ¿Te han fusilado? ¿Has salido de aquel montón de
muertos?

Comprendí por su mirada y por sus palabras que aquel hombre estaba loco,
y seguí adelante. Otro se llegó a mí y preguntome a su vez que a quién
buscaba. Contele brevemente la historia, y me dijo:

---Los que fueron presos en el barrio de Maravillas, no han venido aquí
ni a la casa de Correos. Están en la Moncloa. Primero los llevaron a San
Bernardino, y a estas horas\ldots{} Vamos allá. Yo tengo un
salvo-conducto de un oficial francés, y podemos salir.

Salimos en efecto, y en el Prado aquel hombre corrió desaladamente y le
perdí de vista. Yo también corrí cuanto me era posible, pues mis
fuerzas, a tan terribles pruebas sometidas por tanto tiempo,
desfallecían ya. No puedo decir qué calles pasé, porque ni miraba a mi
alrededor, ni tenía entonces más ojos que los del alma para ver siempre
dentro de mí mismo el espectáculo de aquella gran tragedia. Sólo sé que
corrí sin cesar; sólo sé que ninguna voz, ninguna queja que sonasen
cerca de mí me conmovían ni me interesaban; sólo sé que mientras más
corría, mayores eran mi debilidad y extenuación, y que al fin, no sé en
qué calle, me detuve apoyándome en la pared cercana, porque mi cuerpo se
caía al suelo y no me era posible dar un paso más. Limpié el sudor de mi
frente; parecíame que se había acabado el aire y que el suelo se
marchaba también bajo mis pies, que las casas se hundían sobre mi
cabeza. Recuerdo haber hecho esfuerzos para seguir; pero no me fue
posible, y por un espacio de tiempo que no puedo apreciar, sólo
tinieblas me rodearon, acompañadas de absoluto silencio.

\hypertarget{xxxiii}{%
\chapter{XXXIII}\label{xxxiii}}

Durante mi desvanecimiento, hijo de la extenuación, traje a la memoria
las arboledas de Aranjuez, con sus millares de pájaros charlatanes,
aquellas tardes sonrosadas, aquellos paseos por los bordes del Jarama y
el espectáculo de la unión de este con el Tajo. Me acordé de la casa del
cura y parecíame ver la parra del patio y los tiestos de la huerta, y
oír los chillidos de la tía Gila, riñendo formalmente con las gallinas
porque sin su permiso se habían salido del corral. Se me representaba el
sonido de las campanas de la iglesia, tocadas por los cuatro muchachos o
por el ingrato padre. La imagen de Inés completaba todas estas imágenes,
y en mi delirio no me parecía que estaba la desgraciada muchacha junto a
mí ni tampoco delante, sino dentro de mi propia persona, como formando
parte del ser a quien reconocía como yo mismo. Nada estorbaba nuestra
felicidad, ni nos cuidábamos de lo porvenir, porque abandonada a su
propio ímpetu la corriente de nuestras almas, se habían juntado al fin
Tajo y Jarama, y mezcladas ambas corrientes cristalinas, cavaban en el
ancho cauce de una sola y fácil existencia.

Sacome de aquel estado soñoliento un fuerte golpe que me dieron en el
cuerpo, y no tardé en verme rodeado de algunas personas, una de las
cuales dijo examinándome de cerca: «Está borracho.»

Creí reconocer la voz del licenciado Lobo, aunque a decir verdad, aún
hoy no puedo asegurar que fuera él quien tal cosa dijo. Lo que sí afirmo
es que uno de los que me miraban era Juan de Dios.

---¡Eres tú, Gabriel!---me dijo.---¿Cómo estás por los suelos? Bonito
modo de buscar a la muchacha. No está en el Retiro, ni en el Buen
Suceso. El señor licenciado me ayuda en mis pesquisas, y estamos seguros
de encontrarla, y aun de salvarla.

Estas palabras las oí confusamente, y después me quedé solo, o mejor
dicho, acompañado de algunos chicuelos que me empujaban de acá para allá
jugando conmigo. No tardé en recobrar con el completo uso de mis
facultades, la idea perfecta de la terrible situación, sólo olvidada
durante un rato de marasmo físico y de turbación mental. Oí
distintamente las dos en un reloj cercano, y observé el sitio en que me
encontraba, el cual no era otro que la plazuela del Barranco, inmediata
a los Caños del Peral. Contemplar mental y retrospectivamente cuanto
había pasado, medir con el pensamiento la distancia que me separaba de
la Montaña y correr hacia allá todo pasó en el mismo instante. Sentíame
ágil; la desesperación aligeraba tanto mis pasos, que en poco tiempo
llegué al fin de mi viaje; y en la portalada que daba a la huerta del
Príncipe Pío vi tanta gente curiosa que era difícil acercarse. Yo lo
hice a pesar de los obstáculos, y habría sido preciso matarme para
hacerme retroceder. Las mujeres allí reunidas daban cuenta de los
desgraciados que habían visto penetrar para no salir más. Desde luego
quise introducirme, e intenté conmover a los centinelas con ruegos, con
llantos, con razones, hasta con amenazas. Pero mis esfuerzos eran
inútiles y cuanto más clamaba, más enérgicamente me impelían hacia
fuera. Después de forcejear un rato, la desesperación y la rabia me
sugirieron estas palabras que dirigí al centinela:

---Déjeme entrar. Vengo a que me fusilen.

El centinela me miró con lástima, y apartome con la culata de su fusil.

---¡Tienes lástima de mí---continué,---y no la tienes de los que busco!
No, no tengas lástima. Yo quiero entrar. Quiero ser arcabuceado con
ellos. Fui nuevamente rechazado: pero de tal modo me dominaba el deseo
de entrar, y tan terriblemente pesaba sobre mi espíritu aquella
horrorosa incertidumbre, que la vida me parecía precio mezquino para
comprar el ingreso de la funesta puerta, tras la cual agonizaban o se
disponían a la muerte mis dos amigos.

Desde fuera escuchaba un sordo murmullo, concierto lúgubre a mi parecer,
de plegarias dolorosas y de violentas imprecaciones. Yo tan pronto me
apartaba de la puerta como volvía a ella, a suplicar de nuevo, y la
angustia me sugería razones incontestables para cualquiera, menos para
los franceses. A veces golpeaba la pared con mi cabeza, a veces
clavábame las uñas en mi propio cuerpo hasta hacerme sangre; medía con
la vista la altura de la tapia, aspirando a franquearla de un vuelo; iba
y venía sin cesar insultando a los afligidos circunstantes y miraba el
negro cielo, por entre cuyos turbios y apelmazados celajes creía
distinguir danzando en veloz carrera una turba de mofadores demonios.

Volvía a suplicar al centinela, diciéndole:

---¿Por qué no me fusiláis? ¿Por qué no entro, para que me maten con mis
amigos? ¡Ah! ¡Asesinos de Madrid! ¿Sabéis para qué quiero yo a vuestro
Emperador? Para esto.

Y escupía con rabia a los pies de los soldados, que sin duda me tenían
por loco. Luego, concibiendo una idea que me parecía salvadora, registré
ávidamente mis bolsillos como si en ellos encerrase un tesoro, y sacando
la navaja de Chinitas que aún conservaba, exclamé con febril alegría:

---¡Ah! ¿No veis lo que tengo aquí? Una navaja, un cuchillo aún manchado
de sangre. Con él he matado muchos franceses, y mataría al mismo
Napoleón I. ¿No prendéis a todo el que lleva armas? Pues aquí estoy.
Torpes; habéis cogido a tantos inocentes y a mí me dejáis suelto por las
calles\ldots{} ¿No me andabais buscando? Pues aquí estoy. Ved, ved el
cuchillo; aún gotea sangre.

Tan convincentes razones me valieron el ser aprehendido, y al fin
penetré en la huerta. Apenas había dado algunos pasos hacia las personas
que confusamente distinguía delante de mí, cuando un vivo gozo inundó mi
alma. Inés y D. Celestino estaban allí, ¡pero de qué manera! En el
momento de mi entrada a ambos los ataban, como eslabones de la cadena
humana que iba a ser entregada al suplicio. Me arrojé en sus brazos, y
por un momento, estrechados con inmenso amor, los tres no fuimos más que
uno solo. Inés empezó después a llorar amargamente; mas el clérigo
conservaba su semblante sereno.

---Desde que le has visto, Inés, has perdido la serenidad---dijo
gravemente.---Ya no estamos en la tierra. Dios aguarda a sus queridos
mártires, y la palma que merecemos nos obliga a rechazar todo
sentimiento que sea de este mundo.

---¡Inés!---exclamé con el dolor más vivo que he sentido en toda mi
vida.---¡Inés! Después de verte en esta situación, ¿qué puedo hacer sino
morir?

Y luego volviéndome a los franceses ebrio de coraje, y sintiéndome con
un valor inmenso, extraordinario, sobrehumano, exclamé:

---Canallas, cobardes verdugos, ¿creéis que tengo miedo a la muerte?
Haced fuego de una vez y acabad con nosotros.

Mi furor no irritaba a los franceses, que hacían los preparativos del
sacrificio con frialdad horripilante. Lleváronme a presencia de uno, el
cual después de decirme algunas palabras, me envió ante otro que al fin
decidió de mi suerte. Al poco rato me vi puesto en fila junto al
clérigo, cuya mano estrechó la mía.

---¿Cuándo te cogieron? ¿Te encontraron alguna arma, desgraciado?---me
dijo.---Pero no es esta ocasión de mostrar odio, sino resignación. Vamos
a entrar en nueva y más gloriosa vida. Dios ha querido que nuestra
existencia acabe en este día, y nos ha dado el laurel de mártires por la
patria, que todos no tienen la dicha de alcanzar. Gabriel, eleva tu
mente al cielo. Tú estás libre de todo pecado, y yo te absuelvo. Hijo
mío, este trance es terrible; pero tras él viene la bienaventuranza
eterna. Sigue el ejemplo de Inés. Y tú, hija mía, la más inocente de
todas las víctimas inmoladas en este día, implora por nosotros, si como
creo llegas la primera al goce de la eterna dicha.

Pero yo no atendía a las razones de mi amigo, sino que me empeñaba en
hablar con Inés, en distraerla de su devoto recogimiento, en pretender
que dirigiera a mí las palabras que a Dios sin duda dirigía, en
obligarla a alzar los ojos y mirarme, pues sin esto, yo me sentía
incapaz de contrición.

Un oficial francés nos pasó una especie de revista, examinándonos uno a
uno.

---¿Para qué prolongáis nuestro martirio?---exclamé sin poderme contener
al ver sobre mí la impertinente mirada del francés.---Todos somos
españoles; todos hemos luchado contra vosotros; por cada vida que
ahoguéis en sangre, renacerán otras mil que al fin acabarán con
vosotros, y ninguno de los que estáis aquí verá la casa en que nació.

---Gabriel, modérate y perdónalos como les perdono yo---me dijo el
cura.---¿Qué te importa esa gente? ¿Para qué les afeas su pasado, si
harto lo verán en el turbio espejo de su conciencia? ¿Qué importa morir?
Hijo mío, destruirán nuestros cuerpos, pero no nuestra alma inmortal,
que Dios ha de recibir en su seno. Perdónalos; haz lo que yo, que pienso
pedir a Dios por los enemigos del príncipe de la Paz, mi amigo y hasta
pariente; por Santurrias, por el licenciado Lobo, por los tíos de
Inesilla, y hasta por los franceses que nos quieren quitar nuestra
patria. Mi conciencia está más serena que ese cielo que tenemos sobre
nuestras cabezas y por cuyo lejano horizonte aparece ya la aurora del
nuevo día. Lo mismo están nuestras almas, Gabriel, y en ellas despuntan
ya los primeros resplandores del día sin fin.

---Ya amanece---dije mirando a Oriente.---Inés: no bajes los ojos, por
Dios, y mírame; estréchate más contra nosotros.

---Procura serenar tu conciencia, hijo mío---continuó el clérigo.---La
mía está serena. No, no he manchado mis manos con sangre porque soy
sacerdote; me encontraron con un cuchillo, pero no era mío. Yo cumplí mi
deber, que era arengar a aquellos valientes, y si ahora me soltaran
acudiría de pueblo en pueblo repitiendo aquello de \emph{Dulce et
decorum est} del gran latino. Únicamente me arrepiento de no haber
advertido a tiempo al señor Príncipe. ¡Ah!, si él hubiera puesto en la
cárcel a aquellos perdidos\ldots{} tal vez no habría caído, tal vez no
habría sido rey Fernando VII, tal vez no habrían venido los
franceses\ldots{} tal vez\ldots{} Pero Dios lo ha querido así\ldots{}
Verdad es que si yo hubiera vencido la cortedad de mi genio\ldots{} si
yo hubiera prevenido a Su Alteza, que me quería tanto\ldots{} ¡Ah!, no
nos ocupemos ya más que de morir y perdonar. ¡Ah, Gabriel! Haz lo que
yo, y verás con cuánta tranquilidad recibes la muerte. ¿Ves a Inés? ¿No
parece su cara la de un ángel celeste? ¿No la ves cómo está tranquila en
su recogimiento, y digna y circunspecta sin afectación; no la ves cómo
mira a los franceses sin odio, y suspira dulcemente, animándonos con su
mirada!

---¡Inés!---exclamé yo sin poder adquirir nunca la serenidad que D.
Celestino me pedía.---Tú no debes morir, tú no morirás. Señor oficial,
fusiladnos a todos, fusilad al mundo entero, pero poned en libertad a
esta infeliz muchacha que nada ha hecho. Así como digo y repito, y juro
que he matado yo más de cincuenta franceses, digo y repito, y juro que
Inés no arrojó a la calle ningún caldero de agua hirviendo, como han
dicho.

El francés miró a Inés, y viéndola tan humilde, tan resignada, tan
bella, tan dulcemente triste en su disposición para la muerte, no pudo
menos de mostrarse algo compasivo. D. Celestino viendo aquella
inclinación favorable, se echó a llorar y dijo también: «todos nosotros
hemos pecado; pero Inés es inocente.» Las lágrimas del anciano
produjeron en mí trastorno tan vivo, que de improviso a la tirantez
colérica de mi irritado ánimo sucedió una como tranquila aunque
penosísima expansión, un reblandecimiento, si así puede decirse, de mi
endurecido dolor.

---Inés es inocente---exclamé de nuevo.---¿No ven ustedes su semblante,
señores oficiales? ¡Ah!, ustedes son unos caballeros muy decentes y muy
honrados, y no pueden cometer la villanía de asesinar a esta niña.

---Nosotros no valemos para nada---dijo el clérigo con voz
balbuciente.---Mátennos en buen hora, porque somos hombres y el que más
y el que menos\ldots{} Pero ella\ldots{} señores militares\ldots{} Me
parece que son ustedes unas personas muy finas\ldots{} pues\ldots{} ¡Ah!
Inés es inocente. No tienen Vds. conciencia; ¿no tienen en su corazón
una voz que les dice que esa jovencita es inocente?

El oficial pareció más inclinado a la compasión, pareció hasta
conmovido. Acercándose, miró a Inés con interés.

Mas la muchacha se abrazó a nosotros en el momento en que los granaderos
formaron la horrenda fila. Yo miraba todo aquello con ojos absortos y
sentíame nuevamente aletargado, con algo como enajenación o delirio en
mi cabeza.

Vi que se acercó otro oficial con una linterna, seguido de dos hombres,
uno de los cuales nos examinó ansiosamente, y al llegar a Inés, parose y
dijo: «Esta.»

Era Juan de Dios, acompañado del licenciado Lobo y de aquel mismo
oficial francés que varias veces le visitó en nuestra tienda.

Lo que entonces pasó se me representa siempre en formas vagas como las
que pasea la mentirosa fiebre ante nuestros ojos cuando estamos
enfermos.

\hypertarget{xxxiv}{%
\chapter{XXXIV}\label{xxxiv}}

El oficial recienvenido y el que antes nos custodiaba hablaron un
instante con precipitación. El segundo dirigiose en seguida a desatar a
Inés para entregarla a su amigo. ¡Momento inexplicable! Inés no quería
separarse de nosotros, y abrazándonos, se aferraba a la muerte con sus
manos ya libres. Un violento, un irresistible egoísmo que hundía sus
poderosas raíces hasta lo más profundo de mi ser, se apoderó de mí. No
sé qué íntima fuerza desarrollada de súbito me permitió romper la
ligadura de un brazo y pude asir fuertemente a Inés, mientras con
angustiosa impaciencia miraba los fusiles del pelotón de granaderos.

¡Instante terrible cuyo recuerdo hiela la sangre en las venas y paraliza
el corazón, simulando la muerte! Aunque la muchacha quería compartir
nuestra suerte, la tardía compasión de nuestros asesinos nos la quitaba.
Ella, durante la breve lucha, dijo algo que he olvidado. Yo también
pronuncié palabras de que hoy no puedo darme cuenta. Pero nos la
quitaron: recuerdo la extraña sensación que experimenté al perder el
calor de sus manos y de su cara. Yo estaba como loco. Pero la vi
claramente cuando se la llevaron, cuando desapareció de entre las filas,
arrastrada, sostenida, cargada por Juan de Dios.

Y al ver esto sentí un estruendo horroroso, después un zumbido dentro de
la cabeza y un hervidero en todo el cuerpo; después un calor intenso,
seguido de penetrante frío; después una sensación inexplicable, como si
algo rozara por toda mi epidermis; después un vapor dentro del pecho,
que subía invadiendo mi cabeza; después una debilidad incomprensible que
me hacía el efecto de quedarme sin piernas; después una palpitación
vivísima en el corazón; después un súbito detenimiento en el latido de
esta víscera; después la pérdida de toda sensación en el cuerpo, y en el
busto, y en el cuello, y en la boca; después la inconsciencia de tener
cabeza, la absoluta reconcentración de todo yo en mi pensamiento;
después unas como ondulaciones concéntricas en mi cerebro, parecidas a
las que forma una piedra cayendo al mar; después un chisporroteo colosal
que difundía por espacios mayores que cielo y tierra juntos la imagen de
Inés en doscientos mil millones de luces; después oscuridad profunda,
misteriosamente asociada a un agudísimo dolor en las sienes; después un
vago reposo, una extinción rápida, un olvido creciente e invasor, y por
último nada, absolutamente nada.

\flushright{Madrid, Julio de 1873.}

~

\bigskip
\bigskip
\begin{center}
\textsc{Fin de el 19 de marzo y el 2 de mayo}
\end{center}

\end{document}
