\PassOptionsToPackage{unicode=true}{hyperref} % options for packages loaded elsewhere
\PassOptionsToPackage{hyphens}{url}
%
\documentclass[oneside,8pt,spanish,]{extbook} % cjns1989 - 27112019 - added the oneside option: so that the text jumps left & right when reading on a tablet/ereader
\usepackage{lmodern}
\usepackage{amssymb,amsmath}
\usepackage{ifxetex,ifluatex}
\usepackage{fixltx2e} % provides \textsubscript
\ifnum 0\ifxetex 1\fi\ifluatex 1\fi=0 % if pdftex
  \usepackage[T1]{fontenc}
  \usepackage[utf8]{inputenc}
  \usepackage{textcomp} % provides euro and other symbols
\else % if luatex or xelatex
  \usepackage{unicode-math}
  \defaultfontfeatures{Ligatures=TeX,Scale=MatchLowercase}
%   \setmainfont[]{EBGaramond-Regular}
    \setmainfont[Numbers={OldStyle,Proportional}]{EBGaramond-Regular}      % cjns1989 - 20191129 - old style numbers 
\fi
% use upquote if available, for straight quotes in verbatim environments
\IfFileExists{upquote.sty}{\usepackage{upquote}}{}
% use microtype if available
\IfFileExists{microtype.sty}{%
\usepackage[]{microtype}
\UseMicrotypeSet[protrusion]{basicmath} % disable protrusion for tt fonts
}{}
\usepackage{hyperref}
\hypersetup{
            pdftitle={LA CORTE DE CARLOS IV},
            pdfauthor={Benito Pérez Galdós},
            pdfborder={0 0 0},
            breaklinks=true}
\urlstyle{same}  % don't use monospace font for urls
\usepackage[papersize={4.80 in, 6.40  in},left=.5 in,right=.5 in]{geometry}
\setlength{\emergencystretch}{3em}  % prevent overfull lines
\providecommand{\tightlist}{%
  \setlength{\itemsep}{0pt}\setlength{\parskip}{0pt}}
\setcounter{secnumdepth}{0}

% set default figure placement to htbp
\makeatletter
\def\fps@figure{htbp}
\makeatother

\usepackage{ragged2e}
\usepackage{epigraph}
\renewcommand{\textflush}{flushepinormal}

\usepackage{indentfirst}

\usepackage{fancyhdr}
\pagestyle{fancy}
\fancyhf{}
\fancyhead[R]{\thepage}
\renewcommand{\headrulewidth}{0pt}
\usepackage{quoting}
\usepackage{ragged2e}

\newlength\mylen
\settowidth\mylen{...................}

\usepackage{stackengine}
\usepackage{graphicx}
\def\asterism{\par\vspace{1em}{\centering\scalebox{.9}{%
  \stackon[-0.6pt]{\bfseries*~*}{\bfseries*}}\par}\vspace{.8em}\par}

 \usepackage{titlesec}
 \titleformat{\chapter}[display]
  {\normalfont\bfseries\filcenter}{}{0pt}{\Large}
 \titleformat{\section}[display]
  {\normalfont\bfseries\filcenter}{}{0pt}{\Large}
 \titleformat{\subsection}[display]
  {\normalfont\bfseries\filcenter}{}{0pt}{\Large}

\setcounter{secnumdepth}{1}
\ifnum 0\ifxetex 1\fi\ifluatex 1\fi=0 % if pdftex
  \usepackage[shorthands=off,main=spanish]{babel}
\else
  % load polyglossia as late as possible as it *could* call bidi if RTL lang (e.g. Hebrew or Arabic)
%   \usepackage{polyglossia}
%   \setmainlanguage[]{spanish}
%   \usepackage[french]{babel} % cjns1989 - 1.43 version of polyglossia on this system does not allow disabling the autospacing feature
\fi

\title{LA CORTE DE CARLOS IV}
\author{Benito Pérez Galdós}
\date{}

\begin{document}
\maketitle

\hypertarget{i}{%
\chapter{I}\label{i}}

Sin oficio ni beneficio, sin parientes ni habientes, vagaba por Madrid
un servidor de ustedes, maldiciendo la hora menguada en que dejó su
ciudad natal por esta inhospitalaria Corte, cuando acudió a las páginas
del \emph{Diario} para buscar ocupación honrosa. La imprenta fue mano de
santo para la desnudez, hambre, soledad y abatimiento del pobre Gabriel,
pues a los tres días de haber entregado a la publicidad en letras de
molde las altas cualidades con que se creía favorecido por la Naturaleza
le tomó a su servicio una cómica del teatro del Príncipe, llamada Pepita
González o la González. Esto pasaba a fines de 1805; pero lo que voy a
contar ocurrió dos años después, en 1807, y cuando yo tenía, si mis
cuentas son exactas, diez y seis años, lindando ya con los diez y siete.

Después os hablaré de mi ama. Ante todo debo decir que mi trabajo, si no
escaso, era divertido y muy propio para adquirir conocimiento del mundo
en poco tiempo. Enumeraré las ocupaciones diurnas y nocturnas en que
empleaba con todo el celo posible mis facultades morales y físicas. El
servicio de la histrionisa me imponía los siguientes deberes:

Ayudar al peinado de mi ama, que se verificaba entre doce y una, bajo
los auspicios del maestro Richiardini, artista de Nápoles, a cuyas
divinas manos se encomendaban las principales testas de la Corte.

Ir a la calle del Desengaño en busca del \emph{Blanco de perla}, del
\emph{Elixir de Circasia}, de la \emph{Pomada a la Sultana}, o de los
\emph{Polvos a la Marechala}, drogas muy ponderadas que vendía un
monsieur Gastan, el cual recibiera el secreto de confeccionarlas del
propio alquimista de María Antonieta.

Ir a la calle de la Reina, número 21, cuarto bajo, donde existía un
taller de estampación para pintar telas, pues en aquel tiempo los
vestidos de seda, generalmente de color claro, se pintaban según la
moda, y cuando ésta pasaba, se volvía a pintar con distintos ramos y
dibujos, realizando así una alianza feliz entre la moda y la economía,
para enseñanza de los venideros tiempos.

Llevar por las tardes una olla con restos de puchero, mendrugos de pan y
otros despojos de comida a don Luciano Francisco Comella, autor de
comedias muy celebradas, el cual se moría de hambre en una casa de la
calle de la Berenjena, en compañía de su hija, que era jorobada y le
ayudaba en los trabajos dramáticos.

Limpiar con polvos la corona y el cetro que sacaba mi ama haciendo de
reina de Mongolia en la representación de la comedia titulada
\emph{Perderlo todo en un día por un ciego y loco amor}, y falso
\emph{Czar de Moscovia}.

Ayudarla en el estudio de sus papeles, especialmente en el de la comedia
\emph{Los inquilinos de sir John, o la familia de la India, Juanito y
Coleta}, para lo cual era preciso que yo recitase la parte de \emph{Lord
Lulleswing}, a fin de que ella comprendiese bien el de \emph{Milady
Pankoff}.

Ir en busca de la litera que había de conducirla al teatro y cargarla
también cuando era preciso.

Concurrir a la cazuela del teatro de la Cruz, para silbar
despiadadamente \emph{El sí de las niñas}, comedia que mi ama aborrecía,
tanto por lo menos, como a las demás del mismo autor.

Pasearme por la plazuela de Santa Ana, fingiendo que miraba las tiendas,
pero prestando disimulada y perspicua atención a lo que se decía en los
corrillos allí formados por cómicos o saltarines, y cuidando de pescar
al vuelo lo que charlaban los de la Cruz en contra de los del Príncipe.

Ir en busca de un billete de balcón para la plaza de toros, bien al
despacho, bien a la casa del banderillero Espinilla, que le tenía
reservado para mi ama, cual obsequio de una amistad tan fina como
antigua.

Acompañarla al teatro, donde me era forzoso tener el cetro y la corona
cuando ella entraba después de la segunda escena del segundo acto, en
\emph{El falso Czar de Moscovia}, para salir luego convertida en reina,
confundiendo a Osloff y a los magnates, que la tenían por buñolera de
esquina.

Ir a avisar puntualmente a los \emph{mosqueteros} para indicarles los
pasajes que debían aplaudir fuertemente en la comedia y en la tonadilla,
indicándoles también la función que preparaban \emph{los de allá} para
que se apercibieran con patriótico celo a la lucha.

Ir todos los días a casa de Isidoro Máiquez con el aparente encargo de
preguntarle cualquier cosa referente a vestidos de teatro; pero con el
fin real de averiguar si estaba en su casa cierta y determinada persona,
cuyo nombre me callo por ahora.

Representar un papel insignificante, como de paje que entra con una
carta, diciendo simplemente: «¡\emph{Tomad}!» o de \emph{hombre del
pueblo primero}, que exclama al presentarse la multitud ante el rey:
«\emph{Señor, justicia},» o: «\emph{A tus reales plantas, coronado
apéndice del sol}.» (Esta clase de ocupación me hacía dichoso por una
noche.)

Y por este estilo otras mil tareas, ejercicios y empleos que no cito,
porque acabaría tarde, molestando a mis lectores más de lo conveniente.
En el transcurso de esta puntual historia irán saliendo mis proezas, y
con ellas los diversos y complejos servicios que presté. Por ahora voy a
dar a conocer a mi ama, la sin par Pepita González, sin omitir nada que
pueda dar perfecta idea del mundo en que vivía.

Mi ama era una muchacha más graciosa que bella, si bien aquella primera
calidad resplandecía en su persona de un modo tan sobresaliente que la
presentaba como perfecta sin serlo. Todo lo que en lo físico se llama
hermosura y cuanto en lo moral lleva el nombre de expresión, encanto,
coquetería, monería, etc., estaba reconcentrado en sus ojos negros,
capaces por sí solos de decir con una mirada más que dijo Ovidio en su
poema sobre el arte que nunca se aprende y que siempre se sabe. Ante los
ojos de mi ama dejaba de ser una hipérbole aquello de \emph{combustibles
áspides y flamígeros ópticos disparos}, que Cañizares Añorbe aplicaban a
las miradas de sus heroínas.

Generalmente de los individuos que conocimos en nuestra niñez recordamos
o los accidentes más marcados de su persona, o algún otro, que a pesar
de ser muy insignificante, queda sin embargo grabado de un modo
indeleble en nuestra memoria. Esto me pasa a mí con el recuerdo de la
González. Cuando la traigo al pensamiento, se me representan
clarísimamente dos cosas, a saber: sus ojos incomparables y el taconeo
de sus zapatos, \emph{abreviadas cárceles de sus lindos pedestales},
como dirían Valladares o Moncín.

No sé si esto bastará para que ustedes se formen idea de mujer tan
agraciada. Yo, al recordarla, veo yo aquellos grandes ojos negros, cuyas
miradas resucitaban un muerto, y oigo el \emph{tip-tap} de su ligero
paso. Esto basta para hacerla resucitar en el recinto oscuro de mi
imaginación, y, no hay duda, es ella misma. Ahora caigo en que no había
vestido, ni mantilla, ni lazo, ni garambaina que no le sentase a
maravilla; caigo también en que sus movimientos tenían una gracia
especial, un cierto no sé qué, un encanto indefinible, que podrá
expresarse cuando el lenguaje tenga la riqueza suficiente para poder
designar con una misma palabra la malicia y el recato, la modestia y la
provocación. Esta rarísima antítesis consiste en que nada hay más
hipócrita que ciertas formas de compostura o en que la malignidad ha
descubierto que el mejor medio de vencer a la modestia es imitarla.

Pero sea lo que quiera, lo cierto es que la González electrizaba al
público con el airoso meneo de su cuerpo, su hermosa voz, su patética
declamación en las obras sentimentales, y su inagotable sal en las
cómicas. Igual triunfo tenía siempre que era vista en la calle por la
turba de sus admiradores y mosqueteros, cuando iba a los toros en calesa
o simón, o al salir del teatro en silla de mano. Desde que veían asomar
por la ventanilla el risueño semblante, guarnecido por los encajes de la
blanca mantilla, la aclamaban con voces y palmadas diciendo: «Ahí va
toda la gracia del mundo, viva la sal de España,» u otras frases del
mismo género. Estas ovaciones callejeras, les dejaban a ellos muy
satisfechos, y también a ella, es decir a nosotros, porque los criados
se apropian siempre los triunfos de sus amos.

Pepita era sumamente sensible, y según mi parecer, de sentimientos muy
vivos y arrebatados, aunque por efecto de cierto disimulo tan
sistemático en ella, que parecía segunda naturaleza, todos la tenían por
fría. Doy fe además de que era muy caritativa, gustando de aliviar todas
las miserias de que tenía noticia. Los pobres asediaban su casa,
especialmente los sábados, y una de mis más trabajosas ocupaciones
consistía en repartirles ochavos y mendrugos, cuando no se los llevaba
todos el señor de Comella, que se comía los codos de hambre, sin dejar
de ser el \emph{asombro de los siglos}, y el primer dramático del mundo.
La González vivía en una casa sin más compañía que la de su abuela, la
octogenaria doña Dominguita y dos criados de distinto sexo que la
servíamos.

Y después de haber dicho lo bueno, ¿se me permitirá decir lo malo,
respecto al carácter y costumbres de Pepa González? No, no lo digo.
Téngase en cuenta, en disculpa de la muchacha ojinegra, que se había
criado en el teatro, pues su madre fue \emph{parte de por medio} en los
ilustres escenarios de la Cruz y los Caños, mientras su padre tocaba el
contrabajo en los Sitios y en la Real Capilla. De esta infeliz y mal
avenida coyunda nació Pepita, y excuso decir que desde la niñez comenzó
a aprender el oficio, con tal precocidad, que a los doce años se
presentó por primera vez en escena, desempeñando un papel en la comedia
de don Antonio Frumento \emph{Sastre, rey y reo a un tiempo, o el sastre
de Astracán}. Conocida, pues, la escuela, los hábitos poco austeros de
aquella alegre gente, a quien el general desprecio autorizaba en cierto
modo para ser peor que los demás, ¿no sería locura exigir de mi ama una
rigidez de principios, que habrían sido suficientes, en las
circunstancias de su vida, para asegurarle la canonización?

Réstame darla a conocer como actriz. En este punto debo decir tan sólo
que en aquel tiempo me parecía excelente: ignoro el efecto que su
declamación produciría en mí, si hoy la viera aparecer en el escenario
de cualquiera de nuestros teatros. Cuando mi ama estaba en la plenitud
de sus triunfos, no tenía rivales temibles con quienes luchar. María del
Rosario Fernández, conocida por la Tirana, había muerto el año 1803.
Rita Luna, no menos famosa que aquélla, se había retirado de la escena
en 1806; María Fernández, denominada la Caramba, también había
desaparecido. La Prado, Josefa Virg, María Ribera, María García y otras
de aquel tiempo, no poseían extraordinarias cualidades: de modo que si
mi ama no sobresalía de un modo notorio sobre las demás, tampoco su
estrella se oscurecía ante el brillo de ningún astro enemigo. El único
que entonces atraía la atención general y los aplausos de Madrid entero
era Máiquez, y ninguna actriz podía considerarle como rival, no
existiendo generalmente el antagonismo y la emulación sino entre los
dioses de un mismo sexo.

Pepa González estaba afiliada al bando de los anti-Moratinistas, no sólo
porque en el círculo por ella frecuentado abundaban los enemigos del
insigne poeta, sino también porque personalmente tenía no sé qué motivos
de irreconciliable inquina contra él. Aquí tengo que resignarme a
apuntar una observación que por cierto favorece bien poco a mi ama; pero
como para mí la verdad es lo primero, ahí va mi parecer, mal que pese a
los manes de Pepita González. Mi observación es que la actriz del
Príncipe no se distinguía por su buen gusto literario, ni en la elección
de obras dramáticas, ni tampoco al escoger los libros que daban alimento
a su abundante lectura. Verdad es que la pobrecilla no había leído a
Luzán, ni a Mortiano, ni tenía noticia de la sátira de Jorge Pitillas,
ni mortal alguno se había tomado el trabajo de explicarle a Batteux ni a
Blair, pues cuantos se acercaron a ella, tuvieron siempre más presente a
Ovidio que a Aristóteles y a Boccaccio más que a Despreaux.

Por consiguiente, mi señora formaba bajo las banderas de don Eleuterio
Crispín de Andorra, con perdón sea dicho de cejijuntos Aristarcos. Y es
que ella no veía más allá, ni hubiera comprendido toda la jerigonza de
las reglas, aunque se las predicaran frailes descalzos. Es preciso
advertir que el abate Cladera, de quien parece ser fidelísimo retrato el
célebre don Hermógenes, fue amigote del padre de nuestra heroína, y sin
duda aquel gracioso pedantón echó en su entendimiento durante la niñez,
la semilla de los principios, que en otra cabeza dieron por fruto
\emph{El gran cerco de Viena}.

Ello es que mi ama gustaba de las obras de Comella, aunque últimamente,
visto el descrédito en que había caído este dios del teatro, al
despeñarse en la miseria desde la cumbre de su popularidad, no se
atrevía a confesarlo delante de literatos y gente ilustrada. Como tuve
ocasión de observar, atendiendo a sus conversaciones y poniendo atención
a sus preferencias literarias, le gustaban aquellas comedias en que
había mucho jaleo de entradas y salidas, revista de tropas, niños
hambrientos que piden la teta, decoración de \emph{gran plaza con arco
triunfal a la entrada}, personajes muy barbudos, tales como irlandeses,
moscovitas o escandinavos, y un estilo mediante el cual podía decir la
dama en cierta situación de apuro: \emph{Estatua viva soy de
hielo\ldots{} o Rencor, finjamos\ldots{} encono, no disimulemos\ldots{}
cautela, favorecedme}.

Recuerdo que varias veces la oí lamentarse de que el nuevo gusto hubiera
alejado de la escena diálogos concertados como el siguiente, que
pertenece si mal no recuerdo a la comedia \emph{La mayor piedad de
Leopoldo el Grande}:

\normalsize
\small
\begin{center}
\begin{tabular}{ l l } 
  \textsc{Margarita.}     & Vamos, amor...                                        \\[.075em] 
  \textsc{Nadasti.}       & \qquad \qquad \qquad Odio...                          \\[.075em]
  \textsc{Zrin.}          & \qquad \qquad \qquad \qquad \qquad Duda...            \\[.075em]
  \textsc{Carlos.}        & Horror...                                             \\[.075em]
  \textsc{Alburquerque.}  & \qquad \qquad Confusión...                            \\[.075em]
  \textsc{Ulrica.}        & \qquad \qquad \qquad \qquad \qquad Martirio...        \\[.075em]
  \textsc{Los seis.}      & Vamos a esperar que el tiempo                         \\[.075em]
                          & diga lo que tú no has dicho.                          \\[.075em]
\end{tabular}
\end{center}
\normalsize

Como este género de literatura iba cayendo en desuso, rara vez tenía mi
ama el gusto de ver en la escena a \emph{Pedro el Grande en el sitio de
Pultowa}, mandando a sus soldados que comieran caballos crudos y sin
sal; y prometiendo él por su parte almorzar piedras antes que rendir la
plaza. Debo advertir que esta preferencia más consistía en una tenaz
obstinación contra los Moratinistas que en falta de luces para
comprender la superioridad de la nueva escuela, y en que mi ama, rancia
e intransigente española por los cuatro costados, creía que las reglas y
el buen gusto eran malísimas cosas sólo por ser extranjeras, y que para
dar muestras de españolismo bastaba abrazarse, como a un lábaro santo, a
los despropósitos de nuestros poetas calagurritanos. En cuanto a
Calderón y a Lope de Vega, ella los tenía por admirables, sólo porque
eran despreciados por los clásicos.

De buena gana me extendería aquí haciendo algunas observaciones sobre
los partidos literarios de entonces y sobre los conocimientos del pueblo
en general y de los que se disputaban su favor con tanto
encarnizamiento; pero temo ser pesado y apartarme de mi principal
objeto, que no es discutir con pluma académica sobre cosas, tal vez
mejor conocidas por el lector que por mí. Quédese en el tintero lo que
no es del caso, y volvamos, una vez que dejo consignado el gusto de mi
ama, que hoy afearía a cualquier marquesa, artista o virtuosa de lo que
llaman el gran mundo; pero que entonces no era bastante a oscurecer
ninguna de las gracias de su persona.

Ya la conocen ustedes. Pues bien; voy a contar lo que me he
propuesto\ldots{} pero ¡por vida de\ldots! ahora caigo en que no debo
seguir adelante sin dar a conocer el papel que, por mi desgracia,
desempeñé en el ruidoso estreno de \emph{El sí de las niñas}, siendo
causa de que la tirantez de relaciones entre mi ama y Moratín se
aumentara hasta llegar a una solemne ruptura.

\hypertarget{ii}{%
\chapter{II}\label{ii}}

El hecho es anterior a los sucesos que me propongo narrar aquí; pero no
importa. \emph{El sí de las niñas} se estrenó en enero de 1806. Mi ama
trabajaba en los \emph{Caños del Peral}, porque el Príncipe, incendiado
algunos años antes, no estaba aún reedificado. La comedia de Moratín
leída varias veces por éste en las reuniones del Príncipe de la Paz y de
Tineo, se anunciaba como un acontecimiento literario que había de
rematar gloriosamente su reputación. Los enemigos en letras que eran
muchos, y los envidiosos, que eran más, hacían correr rumores
alarmantes, diciendo que la tal obra era un comedión más soporífero que
\emph{La mojigata}, más vulgar que \emph{El barón} y más anti-español
que \emph{El café}. Aún faltaban muchos días para el estreno, y ya
corrían de mano en mano sátiras y diatribas, que no llegaron a
imprimirse. Hasta se tocaron registros de pasmoso efecto entonces,
cuales eran excitar la suspicacia de la censura eclesiástica, para que
no se permitiera la representación; pero de todo triunfó el mérito de
nuestro primer dramático, y \emph{El sí de las niñas} fue representado
el 24 de enero.

Yo formé parte, no sin alborozo, porque mis pocos años me autorizaban a
ello, de la tremenda conjuración fraguada en el vestuario de los Caños
del Peral, y en otros oscuros conciliábulos, donde míseramente vivían,
entre \emph{cendales arachneos}, algunos de los más afamados dramaturgos
del siglo precedente. Capitaneaba la conjuración un poeta, de cuya
persona y estilo pueden ustedes formarse idea si recuerdan al omnímodo
escritor a quien Mercurio escoge entre la gárrula multitud para
presentarlo a Apolo. No recuerdo su nombre, aunque sí su figura, que era
la de un despreciable y mezquino ser constituido moral y físicamente
como por limosna de la maternal Naturaleza. Consumido su espíritu por la
envidia, y su cuerpo por la miseria, ganaba en fealdad y repulsión de
año en año; y como su numen ramplón, probado en todos los géneros, desde
el heroico al didascálico, no daba ya sino frutos a que hacían ascos los
mismos sectarios de la escuela, estaba al fin consagrado a componer
groseras diatribas y torpes críticas contra los enemigos de aquéllos a
cuya sombra vivía sin más trabajo que el de la adulación.

Este hijo de Apolo nos condujo en imponente procesión a la cazuela de la
Cruz, donde debíamos manifestar con estudiadas señales de desagrado los
errores de la escuela clásica. Mucho trabajo nos costó entrar en el
coliseo, pues aquella tarde la concurrencia era extraordinaria; pero al
fin, gracias a que habíamos acudido temprano, ocupamos los mejores
asientos de la región paradisíaca, donde se concertaban todos los
discordes ruidos de la pasión literaria, y todos los malos olores de un
público que no brillaba por su cultura.

Ustedes creerán que el aspecto interior de los teatros de aquel tiempo
se parece algo al de nuestros modernos coliseos. ¡Qué error tan grande!
En el elevado recinto donde el poeta había fijado los reales de su
tumultuoso batallón, existía un compartimiento que separaba los dos
sexos, y de seguro el sabio legislador que tal cosa ordenó en los
pasados siglos se frotaría con satisfacción las manos y daríase un golpe
en la augusta frente, creyendo adelantar gran paso en la senda de la
armonía entre hombres y mujeres. Por el contrario, la separación avivaba
en hembras y varones el natural anhelo de entablar conversación, y lo
que la proximidad hubiera permitido en voz baja, la pérfida distancia lo
autorizaba en destempladas voces. Así es que entre uno y otro hemisferio
se cruzaban palabras cariñosas, o burlonas o soeces, observaciones que
hacían desternillar de risa a todo el ilustre concurso, preguntas que se
contestaban con juramentos, y agudezas cuya malicia consistía en ser
dichas a gritos. Frecuentemente de las palabras se pasaba a las obras, y
algunas andanadas de castañas, avellanas, o cáscaras de naranjas,
cruzaban de polo a polo, arrojadas por diestra mano, ejercicio que si
interrumpía la función, en cambio regocijaba mucho a entrambas partes.

Sin embargo, bueno es advertir que este mismo público, a quien afeaban
tan groseras exterioridades, solía dar muestras de gran instinto
artístico, llorando con Rita Luna en el drama de Kotzebue
\emph{Misantropía y arrepentimiento}, o participando del sublime horror
expresado por Isidoro en la tragedia \emph{Orestes}. Verdad es también
que ningún público del mundo ha excedido a aquél en donaire, para
burlarse de los autores malos y de los poetas que no eran de su agrado.
Igualmente dispuesto a la risa que al sentimiento, obedecía como un
débil niño a las sugestiones de la escena. Si alguien no pudo jamás
tenerle propicio, culpa suya fue.

Mirando el teatro desde arriba parecía el más triste recinto que puede
suponerse. Las macilentas luces de aceite que encendía un mozo saltando
de banco en banco apenas le iluminaban a medias, y tan débilmente, que
ni con anteojos se descubrían bien las descoloridas figuras del ahumado
techo, donde hacía cabriolas un señor Apolo con lira y borceguíes
encarnados. Era de ver la operación de encender la lámpara central, que,
una vez consumada tan delicada maniobra, subía lentamente por máquina,
entre las exclamaciones de la gente de arriba, que no dejaba pasar tan
buena ocasión de manifestarse de un modo ruidoso.

Abajo también había compartimiento, y consistía en una fuerte viga,
llamada \emph{degolladero}, que separaba las lunetas del patio
propiamente dicho. Los palcos o aposentos eran unos cuchitriles
estrechos y oscuros donde se acomodaban como podían las personas de pro;
y como era costumbre que las damas colgasen en los antepechos sus chales
y abrigos, el conjunto de las galerías tenía un aspecto tal, que parecía
decoración hecha ex profeso para representar las calles de Postas o de
Mesón de Paños.

El reglamento de teatros, publicado en 1803, tendía a corregir muchos de
estos abusos; pero como nadie se cuidaba de hacerlo cumplir, sólo la
costumbre y el progreso de la cultura reformó hábitos tan feos. Recuerdo
que hasta mucho después de la época a que me refiero, las gentes
conservaban el sombrero puesto, aunque el reglamento decía
terminantemente en uno de sus artículos: «En los aposentos de todos los
pisos, y sin excepción de alguno, no se permitirá sombrero puesto,
gorro, ni red al pelo; pero sí capa o capote para su comodidad.»

Mientras aguardábamos a que se alzase el telón, el poeta me hacía
minucioso relato del infinito número de obras que había compuesto entre
dramáticas, cómicas, elegíacas, epigramáticas, venatorias, bucólicas y
del género sentimental y mixto. Me contó el argumento de tres o cuatro
tragedias que no esperaban más que la protección de un Mecenas para
pasar de las musas al teatro, y como si mis culpas no estuvieran aún
bastante purgadas con oír los argumentos, me espetó algunos sonetos, que
si no eran exactamente iguales a aquel famosísimo

\small
\newlength\mlenb
\settowidth\mlenb{Reverberante numen que del Istro }
\begin{center}
\parbox{\mlenb}{Reverberante numen que del Istro              \\ 
                al Marañón sublimas con tu Zurda,}            \\ 
\end{center}
\normalsize

\justifying{\noindent le eran tan semejantes como una calabaza a otra.}

Cuando la representación iba a empezar, el poeta dirigió su mirada de
gerifalte a los abismos del patio para ver si habían puntualmente
acudido otros no menos importantes caudillos de la manifestación
fraguada contra \emph{El sí de las niñas}. Todos estaban en sus puestos,
con puntual celo por la causa nacional. No faltaba ninguno; allí estaba
el vidriero de la calle de la Sartén, uno de los más ilustres capitanes
de la mosquetería; allí el vendedor de libros de la Costanilla de los
Ángeles, hombre perito en las letras humanas; allí \emph{Cuarta y
Media}, cuyo fuerte pulmón hizo acallar él solo a todos los admiradores
de \emph{La mojigata}; allí el hojalatero de las Tres Cruces, esforzado
adalid, que traía bajo la ancha capa algún reluciente y ruidoso caldero
para sorprender al auditorio con sinfonías no anunciadas en el programa;
allí el incomparable Roque Pamplinas, barbero, veterinario y sangrador,
que con los dedos en la boca, desafiaba a todos los flautistas de Grecia
y Roma; allí, en fin, lo más granado y florido que jamás midió sus armas
en palenques literarios. Mi poeta quedó satisfecho después de pasar
revista a su ejército, y luego dirigimos todos nuestra atención al
escenario, porque la comedia había empezado.

---¡Qué principio!---dijo oyendo el primer diálogo entre don Diego y
Simón.---¡Bonito modo de empezar una comedia! La escena es una posada.
¿Qué puede pasar de interés en una posada? En todas mis comedias, que
son muchas, aunque ninguna se ha representado, se abre la acción con un
\emph{jardín corintiano}, \emph{fuentes monumentales a derecha e
izquierda}, \emph{templo de Juno en el fondo}, o con \emph{gran plaza},
\emph{donde están formados tres regimientos}; \emph{en el fondo la
ciudad de Varsovia, a la cual se va por un puente}\ldots{} etc. Y oiga
usted las simplezas que dice ese vejete. Que se va a casar con una niña
que han educado las monjas de Guadalajara. ¿Esto tiene algo de
particular? ¿No es acaso lo mismo que estamos viendo todos los días?

Con estas observaciones, el endiablado poeta no me dejaba oír la
función, y yo, aunque a todas sus censuras contestaba con monosílabos de
la más humilde aquiescencia, hubiera deseado que callara con mil
demonios. Pero era preciso oírle; y cuando aparecieron doña Irene y doña
Paquita, mi amigo y jefe no pudo contener su enfado, viendo que atraían
la atención dos personas, de las cuales una era exactamente igual a su
patrona, y la otra no era ninguna princesa, ni senescala, ni canonesa,
ni landgraviata, ni archidapífera de país ruso o mongol.

---¡Qué asuntos tan comunes! ¡Qué bajeza de ideas!---exclamaba de modo
que le pudieran oír todos los circunstantes.---¿Y para esto se escriben
comedias? ¿Pero no oye usted que esa señora está diciendo las mismas
necedades que diría doña Mariquita o doña Gumersinda, o la tía
Candungas? Que si tuvo un pariente obispo, que si las monjas educaron a
la niña sin artificios ni embelecos; que la muy piojosa se casó a los 19
con don Epitafio; que parió veintidós hijos\ldots{} así reventara la
maldita vieja.

---Pero oigamos---dije yo, sin poder aguantar las importunidades del
caudillo,---y luego nos burlaremos de Moratín.

---Es que no puedo sufrir tales despropósitos---continúo.---No se viene
al teatro para ver lo que a todas horas se ve en las calles y en casa de
cada \emph{quisque}. Si esa señora en vez de hablar de sus partos,
entrase echando pestes contra un general enemigo porque le mató en la
guerra sus veintiún hijos, dejándole sólo el veintidós, que está aún en
la mamada, y lo trae para que no se lo coman los sitiados, que se mueren
de hambre, la acción tendría interés, y ya estaría el público con las
manos desolladas de tanto palmoteo\ldots{} Amigo Gabriel, es preciso
protestar con fuerza. Golpeemos el suelo con los pies y los bastones,
demostrando nuestro cansancio e impaciencia. Ahora bostecemos abriendo
la boca hasta que se disloquen las quijadas, y volvamos la cara hacia
atrás, para que todos los circunstantes, que ya nos tienen por
literatos, vean que nos aburrimos de tan sandia y fastidiosa obra.

Dicho y hecho; comenzamos a golpear el suelo, y luego bostezamos en
coro, diciéndonos unos a otros: ¡\emph{Qué fastidio}!\ldots{} ¡\emph{Qué
cosa tan pesada}!\ldots{} ¡\emph{Mal empleado dinero}!\ldots{} y otras
frases por el mismo estilo, que no dejaban de hacer su efecto: los del
patio imitaron puntualísimamente nuestra patriótica actitud. Bien pronto
un general murmullo de impaciencia resonó en el ámbito del teatro. Pero
si había enemigos, no faltaban amigos, desparramados por lunetas y
aposentos, y aquéllos no tardaron en protestar contra nuestra
manifestación, ya aplaudiendo ya mandándonos callar con amenazas y
juramentos, hasta que una voz fortísima, gritando desde el fondo del
patio: ¡\emph{Afuera los chorizos}!, provocó ruidosa salva de aplausos,
y nos impuso silencio.

El poetastro no cabía en su pellejo de indignación. Siguió haciendo
observaciones, conforme avanzaba la pieza, y decía:

---Ya, ya sé lo que va a resultar aquí. Ahora resulta que doña Paquita
no quiere al viejo, sino a un militarito, que aún no ha salido, y que es
sobrino del cabronazo de don Diego. Bonito enredo\ldots{} Parece mentira
que esto se aplauda en una nación culta. Yo condenaba a Moratín a
galeras, obligándole a no escribir más vulgaridades en toda su vida. ¿Te
parece, Gabrielito, que esto es comedia? Si no hay enredo, ni trama, ni
sorpresa, ni confusiones, ni engaños, ni \emph{quid pro quo}, ni aquello
de disfrazarse un personaje para hacer creer que es otro, ni tampoco
aquello de que salen dos insultándose como enemigos, para después
percatarse de que son padre e hijo\ldots{} Si ese don Diego cogiera a su
sobrino y matándolo bonitamente en la cueva, preparara un festín e
hiciera servir a su novia un plato de carne de la víctima, bien
condimentado con especias y hoja de laurel, entonces la cosa tendría
alguna malicia\ldots{} ¿Y la niña por qué disimula? ¿No sería más
dramático que se negase a casarse con el viejo, que le insultara
llamándole tirano, o le amenazara con arrojarse al Danubio, o al Don, si
osaba tocar su virginidad?\ldots{} Estos poetas nuevos no saben inventar
argumentos bonitos, sino estas majaderías con que engañan a los bobos,
diciéndoles que son conformes a las reglas. Ánimo, compañeros,
prepararse todo el mundo. Pronunciemos frases coléricas y finjamos
disputar en corro, diciendo unos que esta obra es peor que La
\emph{mojigata}, y otros que aquélla era peor que ésta. El que sepa
silbar con los dedos, hágalo \emph{ad libitum}, y patadas a discreción.
Apostrofar a doña Irene cuando se retire de la escena, llamándola cada
cual como le ocurra.

Dicho y hecho: conforme a las terminantes órdenes de nuestro jefe,
armamos una espantosa grita al finalizar el acto primero. Como los
amigos del autor protestaron contra nosotros, exclamamos ¡\emph{Afuera
la polaquería}! y enardecidos los dos bandos por el calor de la porfía,
se cruzaron más duros apóstrofes, entre el discorde gritar de la cazuela
y el patio. El acto segundo no pasó más felizmente que el primero; y por
mi parte, ponía gran atención al diálogo, porque la verdad era, con
perdón sea dicho del poeta mi amigo, que la comedia me parecía muy
buena, sin que yo acertara a explicarme entonces en qué consistían sus
bellezas.

La obstinación de aquella doña Irene, empeñada en que su hija debía
casarse con don Diego porque así cuadraba a su interés, y la torpeza con
que cerraba los ojos a la evidencia, creyendo que el consentimiento de
su hija era sincero, sin más garantía que la educación de las monjas; el
buen sentido del don Diego, que no las tenía todas consigo respecto a la
muchacha, y desconfiaba de su remilgada sumisión; la apasionada
cortesanía de don Carlos, la travesura de Calamocha, todos los
incidentes de la obra, lo mismo los fundamentales que los accesorios, me
cautivaban, y al mismo tiempo descubría vagamente en el centro de
aquella trama un pensamiento, una intención moral, a cuyo desarrollo
estaban sujetos todos los movimientos pasionales de los personajes. Sin
embargo, me cuidaba mucho de guardar para mí estos raciocinios, que
hubieran significado alevosa traición a la ilustre hueste de silbantes,
y fiel a mis banderas no cesaba de repetir con grandes aspavientos:
«¡Qué cosa tan mala!\ldots{} ¡Parece mentira que esto se
escriba!\ldots{} Ahí sale otra vez la viejecilla\ldots{} Bien por el
viejo ñoño\ldots{} ¡Qué aburrimiento! ¡Miren la gracia!» etc., etc.

El segundo acto pasó, como el primero, entre las manifestaciones de uno
y otro lado; pero me parece que los amigos del poeta llevaban ventaja
sobre nosotros. Fácil era comprender que la comedia gustaba al público
imparcial y que su buen éxito era seguro, a pesar de las indignas
cábalas, en las cuales tenía yo también parte. El tercer acto fue sin
disputa el mejor de los tres: yo le oí con religio so respeto, y
luchando con las impertinencias de mi amigo el poeta, que en lo mejor de
la pieza creyó oportuno desembuchar lo más escogido de sus disparates.

Hay en el dicho acto tres escenas de una belleza incomparable. Una es
aquélla en que doña Paquita descubre ante el buen don Diego las luchas
entre su corazón y el deber impuesto por una hipócrita conformidad con
superiores voluntades: otra es aquélla en que intervienen don Carlos y
don Diego, y se desata, merced a nobles explicaciones, el nudo de la
fábula; y la tercera es la que sostienen del modo más gracioso don Diego
y doña Irene, aquél deseando dar por terminado el asunto del matrimonio,
y ésta interrumpiéndole a cada paso con sus importunas observaciones.

No pude disimular el gusto que me causó esta escena, que me parecía el
colmo de la naturalidad, de la gracia y del interés cómico; pero el
poeta me llamó al orden injuriándome por mi deserción del campo
\emph{chorizo}.

---Perdone usted---le dije,---me he equivocado. Pero ¿no cree usted que
esa escena no está del todo mal?

---¡Cómo se conoce que eres novato, y en la vida has compuesto un verso!
¿Qué tiene esa escena de extraordinario, ni de patético, ni de
historiográfico\ldots?

---Es que la naturalidad\ldots{} Parece que ha visto uno en el mundo lo
que el poeta pone en escena.

---Cascaciruelas: pues por eso mismo es tan malo. ¿Has visto que en
\emph{Federico II}, en \emph{Catalina de Rusia}, en \emph{La esclava de
Negroponto} y otras obras admirables, pase jamás nada que remotamente se
parezca a las cosas de la vida? ¿Allí no es todo extraño, singular,
excepcional, maravilloso y sorprendente? Pues por eso es tan bueno. Los
poetas de hoy no aciertan a imitar a los de mi tiempo, y así está el
arte por los mismos suelos.

---Pues yo, con perdón de usted---dije,---creo que\ldots{} la obra es
malísima, convengo; y cuando usted lo dice, bien sabido se tendrá por
qué. Pero me parece laudable la intención del autor que se ha propuesto
aquí, según creo, censurar los vicios de la educación que dan a las
niñas del día, encerrándolas en los conventos, y enseñándolas a
disimular y a mentir\ldots{} Ya lo ha dicho don Diego: las juzgan
honestas, cuando les han enseñado el arte de callar, sofocando sus
inclinaciones, y las madres se quedan muy contentas cuando las
pobrecillas se prestan a pronunciar un sí perjuro, que después las hace
desgraciadas.

---¿Y quién le mete al autor en esas filosofías?---dijo el
pedante.---¿Qué tiene que ver la moral con el teatro? En \emph{El mágico
de Astracán}, en \emph{A España, dieron blasón las Asturias y León}, y
\emph{Triunfos de don Pelayo}, comedias que admira el mundo, ¿has visto
acaso algún pasaje en que se hable del modo de educar a las niñas?

---Yo he oído o leído en alguna parte que el teatro sirve de
entretenimiento y de enseñanza.

---¡Patarata! Además, el señor Moratín se va a encontrar con la horma de
su zapato por meterse a criticar la educación que dan las señoras
monjas. Ya tendrá que habérselas con los reverendos obispos y la santa
Inquisición ante cuyo tribunal se ha pensado delatar \emph{El sí}, y se
delatará, sí, señor.

---Vea usted el final---dije atendiendo a la tierna escena en que don
Diego casa a los dos amantes, bendiciéndoles con cariño de un padre.

---¡Qué desenlace tan desabrido! Al menos lerdo se le ocurre que don
Diego debe casarse con doña Irene.

---¡Hombre! ¿Don Diego con doña Irene? Si él es una persona discreta y
seria, ¿cómo va a casarse con esa vieja fastidiosa?

---¿Qué entiendes tú de eso, chiquillo?---exclamó amostazado el
pedante.---Digo que lo natural es que don Diego se case con doña Irene,
don Carlos con Paquita, y Rita con Simón. Así quedaría regular el fin, y
mucho mejor si resultara que la niña era hija natural de don Diego y don
Carlos hijo espurio de doña Irene, que le tuvo de algún rey disfrazado,
comandante del Cáucaso o bailío condenado a muerte. De este modo tendría
mucho interés el final, mayormente si uno salía diciendo: ¡\emph{Padre
mío}!, y otro: ¡\emph{Madre mía}!, con lo cual después de abrazarse, se
casaban para dar al mundo numerosa y masculina sucesión.

---Vamos, que ya se acaba. Parece que el público está satisfecho---dije
yo.

---Pues apretar ahora, muchachos. Manos a la boca. La comedia es pésima,
inaguantable.

La consigna fue prontamente obedecida. Yo mismo, obligado por la
disciplina, me introduje los dedos en la boca y\ldots{} ¡Sombra de
Moratín! ¡Perdón mil veces!\ldots{} No lo quiero decir: que comprenda el
lector mi ignominia y me juzgue.

Pero nuestra mala estrella quiso que la mayor parte del público
estuviese bien dispuesta en favor de la comedia. Los silbidos provocaron
una tempestad de aplausos, no sólo entre la gente de los aposentos y
lunetas, sino entre los de la cazuela y tertulia.

El justiciero pueblo que nos rodeaba, y que en su buen instinto
artístico comprendía el mérito de la obra, protestó contra nuestra
indigna cruzada, y algunos de los más ardientes de la falange se vieron
aporreados de improviso. Lo que tengo más presente es la mala aventura
que ocurrió al alumno de Apolo en aquella breve batalla por él
provocada. Usaba un sombrero tripico, de dimensiones harto mayores que
las proporcionadas a su cabeza, y en el momento en que se volvía para
contestar a las injurias de cierto individuo, una mano vigorosa, cayendo
a plomo sobre aquella prenda hiperbólica, se la hundió hasta que las
puntas descansaron sobre los hombros. En esta actitud estuvo el infeliz
manoteando un rato, incapaz para sacar a luz su cabeza del tenebroso
recinto en que había quedado sepultada.

Por fin, los amigos le sacamos con gran esfuerzo el sombrero, y él
echando espumarajos por la boca, juró tomar venganza tan sangrienta como
pronta; pero no pasó de aquí su furor, porque todos los circunstantes se
reían de él, y a ninguno se dirigió para vengarse. Le sacamos a la
calle, donde se serenó algún tanto, y nos separamos, prometiendo
juntarnos otra vez al día siguiente en el mismo sitio.

Tal fue el estreno de \emph{El sí de las niñas}. Aunque la primera tarde
fuimos derrotados, aún había esperanzas de hundir la obra en la segunda
o tercera representación. Se sabía que el ministro Caballero la
desaprobaba, jurando castigar a su autor, y esto daba esperanza al
partido de los silbantes, que ya veían a Moratín en poder del Santo
Oficio, con coroza de sapos, sambenito y soga al cuello. Pero la segunda
tarde vinieron de un golpe a tierra las ilusiones de los más ardientes
anti-Moratinistas, porque la presencia del Príncipe de la Paz impuso
silencio a las chicharras, y nadie osó formular demostraciones de
desagrado. Desde entonces, el autor de \emph{El sí}, a quien se dijo que
la conspiración había sido fraguada en el cuarto de mi ama, interrumpió
la tibia amistad que con ésta le unía. La González pagó este desvío con
un cordial aborrecimiento.

\hypertarget{iii}{%
\chapter{III}\label{iii}}

Contado este suceso, muy anterior a los que son objeto del presente
libro, empezaré mi narración, la cual irá al compás de ciertos hechos
ocurridos en el Otoño de 1807, año que en la mente de los madrileños
quedó marcado con el recuerdo de la famosa conspiración de El Escorial.

No quiero escribir una palabra más, sin daros a conocer a una persona
que desde aquellos días ocupó lugar privilegiado en mi corazón, siendo a
la vez como se verá por este relato, lección viva de mi existencia, pues
la enseñanza que de su conocimiento me provino contribuyó de un modo
poderoso a formar mi carácter.

Todas las ropas de teatro y de calle que usaba mi ama, eran
confeccionadas por una costurera de la calle de Cañizares, excelente y
honradísima mujer, joven aún, aunque desmejorada por el trabajo,
discreta y afable, en tales términos que por entre la corteza de su
malestar presente parecían distinguirse nacimiento y condición muy
superiores. Esto no era más que apariencia, pero a la citada persona le
pasaba lo contrario de lo que a otros pasa, y es que son nobles sin
parecerlo. Doña Juana, que éste era el nombre de aquella santa mujer,
tenía una hija llamada Inés, de quince años de edad, la cual le ayudaba
en sus tareas, con más solicitud de la que podía esperarse de su
delicado organismo y edad temprana.

Enaltecía a esta muchacha, además de las gracias de su persona, un buen
sentido, cual no he visto jamás en criaturas de su mismo sexo ni aun del
nuestro, amaestrado ya por los años. Inés tenía el don especialísimo de
poner todas las cosas en su verdadero lugar, viéndolas con luz singular
y muy clara, concedida a su privilegiado entendimiento, sin duda para
suplir con ella la inferioridad que le negó la fortuna. No he visto en
mi larga vida otra muchacha que a aquella se asemejase, y estoy seguro
de que a muchos parecerá este tipo invención mía, pues no comprenderán
que haya existido, entre las infinitas hijas de Eva, una tan diferente
de las demás. Pero créanlo bajo mi palabra honrada.

Si ustedes hubieran conocido a Inés, y notado la imperturbable serenidad
de su semblante, imagen del espíritu más tranquilo, más equilibrado, más
claro, más dueño de sí mismo que ha animado el corporal barro, no
pondrían en duda lo que digo. Todo en ella era sencillez, hasta su
hermosura, no a propósito para despertar mundano entusiasmo amoroso,
sino semejante a una de esas figuras simbólicas, que no están
materialmente representadas en ninguna parte; pero que vemos con los
ojos del alma, cuando las ideas agitándose en nuestra mente, pugnan por
vestirse de formas visibles en la oscura región del cerebro.

Su lenguaje era también la misma sencillez; jamás decía cosa alguna que
no me sorprendiese como la más clara y expresiva verdad. Sus razones
trayéndome al sentido equitativo y templado de todas las cosas, daban a
mi entendimiento un descanso, un aplomo, de que carecía obrando por sí
mismo. Puedo decir comparando mi espíritu con el de Inés, y escudriñando
la radical diferencia entre uno y otro, que el de ella tenía un centro y
el mío no. El mío divagaba llevado y traído por impresiones diversas,
por sentimientos contradictorios y repentinos: mis facultades eran como
meteoros errantes que tan pronto brillan como se oscurecen, tan pronto
marchan como chocan, según la influencia recibida de superiores cuerpos;
mientras las suyas eran un completo y armónico sistema planetario,
atraído, puesto en movimiento y calentado por el gran sol de su pura
conciencia.

Alguien se burlará de estas indicaciones psicológicas, que yo quisiera
fuesen tan exactas como las concibe mi oscura inteligencia: alguien
encontrará digna de risa la presentación de semejante heroína, y harán
mil aspavientos al ver que he querido hacer una irrisoria
\emph{Beatrice} con los materiales de una modistilla; pero estas burlas
no me importan, y sigo.

Desde que conocí a Inés, la amé del modo más extraño que pueden ustedes
imaginar: una viva inclinación arrastraba mi corazón hacia ella: pero
esta inclinación era como el culto que tributamos a una superioridad
indiscutible, como la fe que nos ocupa sublimando lo más noble de
nuestro ser; pero dejando libre una parte de él para las pasiones del
mundo. Así es, que sin dejar de ser Inés para mí la primera de todas las
mujeres, yo creía poder amar a otras con amor apropiado a las
circunstancias de cada momento de la vida. Yo he observado que los que
se consagran a un ideal, casi nunca lo hacen por entero, dejan una parte
de sí mismos para el mundo, a que están unidos, aunque sólo sea por el
suelo que pisan. Hago esta observación fastidiosa por si contribuye a
esclarecer el peculiar estado de mi alma ante tan noble criatura. ¡Y era
una modista; una modistilla! Reíd si os place.

El tercer individuo de aquella honesta familia era el padre Celestino
Santos del Malvar, hermano del difunto esposo de doña Juana, tío por lo
tanto de Inés, clérigo desde su mocedad, varón simplísimo y benévolo,
pero el más desgraciado de su clase, pues no tenía rentas, ni
capellanía, ni beneficio alguno. Su modestia, su buena fe y su candor
inagotable fueron sin duda parte a tenerle en la miseria por tanto
tiempo; y él, aunque era un gran latino, jamás pudo conseguir colocación
alguna. Pasaba la vida escribiendo memoriales al Príncipe de la Paz, de
quien era paisano y fue allá en la niñez amigo; mas ni el Príncipe ni
nadie le hacía caso.

Cuando Godoy subió al Ministerio prometiole una canonjía o ración, y en
la época de este relato hacía catorce años que don Celestino del Malvar
estaba esperando lo prometido: mas sin que la tardanza del favor hiciese
desmayar su ingenua confianza. Siempre que se le preguntaba, respondía:
«La semana que viene recibiré el nombramiento: así me lo ha dicho el
oficial de la Secretaría.» De este modo pasaron catorce años, y la
\emph{semana que viene} no venía nunca.

Siempre que yo iba a aquella casa con recados de mi ama, me detenía todo
el tiempo posible, y a ella acudía también en mis ratos de ocio, gozando
mucho en contemplar la apacible existencia de una familia, cuyos tres
individuos tan honda simpatía habían despertado en mi corazón. Doña
Juana y su hija siempre cosiendo, cosiendo con eterna aguja una tela sin
fin: de esto vivían los tres, pues el padre Celestino, tocando la
flauta, haciendo versos latinos, o consumiendo tinta y papel en
larguísimos memoriales, no ganaba más caudal que el de sus esperanzas,
siempre colocadas a interés compuesto.

Nuestras conversaciones eran siempre entretenidas y amenas. Yo les
contaba mi breve historia, y les hacía reír dándoles a conocer los pocos
proyectos que imaginaba para lo porvenir. Nos reíamos discretamente y
sin saña de la buena fe de don Celestino, y éste después de salir a
informarse de su asunto, volvía lleno de júbilo, dejaba sobre una silla
el sombrero de teja y el manteo, y restregándose las manos, decía al
sentarse junto a nosotros:

---Ahora sí que va de veras. La semana que entra, sin falta. Me han
dicho que ocurrieron ciertas dilacioncillas; pero ya están vencidas, a
Dios gracias. La semana que viene, sin falta.

Cierto día le dije:

---Usted, don Celestino, no ha conseguido ya lo que deseaba, porque es
hombre encogido y no se lanza\ldots{} pues\ldots{} no se lanza.

---¿Qué es eso de lanzarse, chiquillo?---me preguntó.

---Pues\ldots{} a mí me han dicho que hoy conviene pedir veinte para que
den cinco. Además, váyase el mérito con mil demonios: lo que conviene es
tener desvergüenza para meterse en todas partes, buscar la amistad de
personas poderosas; en fin, hacer lo que los demás han hecho para subir
a esos puestos en que son la admiración del mundo.

---¡Ah, Gabriel!---dijo doña Juana.---Tú eres un ambiciosillo a quien
alguien ha trastornado el juicio. Lo que menos crees tú es que te has de
ver por ensalmo en la corte, cubierto de galones y mandando y
disponiendo desde la secretaría del despacho.

---Justo y cabal, señora mía---dije yo riendo y atento a lo que
expresaba el semblante de Inés, con quien repetidas veces había hablado
del mismo asunto.---Aunque estoy en el mundo sin padre ni madre, ni
perro que me ladre, yo creo que bien puedo esperar lo que otros han
tenido sin ser más sabios que yo. De menos hizo Dios a Cañete a quien
hizo de un puñete.

---Tú tienes disposición, Gabriel---dijo gravemente don Celestino;---y
mucho será que de un día para otro no te veamos convertido en personaje.
Entonces no te dignarás hablarnos, ni vendrás a casa; pero hijo, es
preciso que aprendas los clásicos latinos, sin lo cual no hallarás
abierta ninguna de las puertas de la fortuna; y además te aconsejo que
aprendas a tañer la flauta, porque la música es suavizadora de las
costumbres, endulza los ánimos más agrios, y predispone a la
benevolencia para con los que la manejan bien. Y si no, aquí me tienes a
mí, que de seguro nada habría conseguido si de antiguo no cultivara mi
entendimiento en aquellas dos divinísimas artes.

---No echaré en saco roto la advertencia---repuse,---pues todos sabemos
a qué debe su encumbramiento el hombre más poderoso que hay hoy en
España después del Rey.

---¡Calumnias!---exclamó irritado el sacerdote.---Mi paisano, amigo y
mecenas, el señor Príncipe de la Paz, debe su elevación a su gran
mérito, y a su sabiduría y tacto político, y no a supuestas habilidades
en la guitarra y las castañuelas, como dice el estólido vulgo.

---Sea lo que quiera---añadí yo,---lo cierto es que ese hombre, de
humildísimo guardia ha subido a cuanto hay que subir. Bien claro está.

---Pues no dudes que tú harás otro tanto---dijo con ironía doña
Juana.---De hombres se hacen los obispos, como dijo el otro.

---Verdad es---repuse siguiendo la broma,---y juro que he de hacer a don
Celestino arzobispo de Toledo.

---Alto allá---dijo el clérigo seriamente.---No aceptaré yo un cargo
para el que me reconozco sin méritos. Bastante tendré yo con una
capellanía de Reyes Nuevos o el arcedianato de Talavera.

Así siguió entre veras y burlas la conversación, hasta que saliendo de
la salita doña Juana y el buen presbítero, nos dejaron solos a Inés y a
mí.

---Cómo se ríen de mis proyectos, niñita mía---le dije.---Pero tú
comprenderás que un muchacho como yo no debe contentarse con servir a
cómicos por toda su vida. A ver: de todo lo que yo puedo ser, Dios
mediante, ¿qué te gustaría más? Escoge: ¿te gustaría que fuese capitán
general, príncipe coronado, con vasallos y ejército, señor de muchas
tierras, primer ministro que quite y ponga los empleados a su antojo,
obispo\ldots? No, obispo no, porque entonces no podría casarme contigo,
para hacerte llevar en carroza de doce caballos\ldots{}

Inés soltó la risa como quien oye un cuento de esos cuyo chiste consiste
en la magnitud de lo absurdo.

---Ríete de mí, pero contesta: ¿qué quieres más?

---Lo que quiero---dijo con dulce voz y suspendiendo la costura,---es
verte general, primer ministro, gran duque, emperador o arzobispo; pero
de tal modo que cuando te acuestes por la noche en tu colchoncito de
plumas puedas decir: «Hoy no he hecho mal a nadie ni nadie ha muerto por
mi causa.»

---Pero reinita---dije yo interesándome más cada vez en aquel
coloquio,---si llego a ser eso que tú dices, (pues bien podría suceder)
¿qué importa que mueran por mí o por el bien del Estado tres o cuatro
prójimos que nada significan en el mundo?

---Bueno---repuso ella,---pero que los maten otros. Si tú llegas a ser
eso que has dicho, y para mantenerte en un puesto que no mereces,
necesitas sacrificar a muchos desgraciados, buen provecho te haga.

---¡Qué escrupulosa eres, Inesilla!---dije.---Si te hiciera caso, mi
vida se encerraría entre cuatro paredes. ¿Qué es eso de sacrificar
desgraciados? Yo voy a mi negocio, y los demás\ldots{} como yo no he de
matar a nadie. Y sobre todo, si hago daño a alguno serán tantos los que
reciban beneficios de mi mano, que todo quedará compensado y mi
conciencia en santa paz. Veo que tú no te entusiasmas como yo, ni
piensas lo que yo pienso. ¿Quieres que te sea franco? Pues oye. A mí se
me ha metido en la cabeza que cuando tenga más años, he de ocupar una
posición\ldots{} qué sé yo\ldots{} me mareo pensando en esto. No te
puedo decir ni cómo he de llegar a ella, ni quién me dará la mano para
subir de un salto tantos escalones; pero ello es que yo cavilo en esto,
y me figuro que ya me estoy viendo elevado a la más alta dignidad por
una dama poderosa que me haga su secretario, o por un joven que me crea
listo para ayudarle en sus asuntos\ldots; no te enfades, chiquilla, que
cuando tales cosas se ocurren y uno tiene la cabeza llena a todas horas
de los mismos pensamientos, al fin tiene que salir cierto, como éste es
día.

Inés no se enfadaba, sino que reía. Después marcando con su aguja el
compás gramatical de su discurso me dijo:

---Pues mira: si tú hubieras nacido en cuna de príncipes, no te digo que
no. Pero has de saber que si tú, que eres un pobrecillo hijo de
pescadores y no tienes más ciencia que leer mal y escribir peor, llegas
a ser hombre ilustre y poderoso, no porque saques talento y sabiduría,
sino porque a una señora caprichosa o a un vejete rico se le ocurra
protegerte, como a otros muchos de quienes cuentan maravillas; has de
saber, digo, que tan fácilmente como subas volverás a caer, y hasta los
sapos se reirán de ti.

---Eso será lo que Dios quiera---respondí.---Caeremos o no: pues aunque
ignorantes, no nos faltará nuestra gramática parda.

---¡Qué necio eres! Mira: a mí me han dicho\ldots{} no, nadie me lo ha
dicho: pero lo sé\ldots{} que en el mundo al fin y al cabo, pasa siempre
lo que debe pasar.

---Reinita---dije,---en eso te equivocas, porque nosotros deberíamos ser
ricos, y no lo somos.

---Todos creerán lo mismo, hijito, y es preciso que alguno esté
equivocado. Pues bien: todas las cosas del mundo concluyen siempre como
deben concluir. No sé si me explico.

---Sí: te entiendo.

---A mí me han dicho\ldots{} no, no me lo han dicho: lo sé desde hace
mil años\ldots{} yo sé que en el mundo todo lo que pasa es según la
ley\ldots{} porque, chiquillo, las cosas no pasan porque a ellas les da
la gana, sino porque así está dispuesto. Las aves vuelan y los gusanos
se arrastran, y las piedras se están quietas, y el sol alumbra, y las
flores huelen, y los ríos corren hacia abajo y el humo hacia arriba,
porque así es su regla\ldots{} ¿me entiendes?

---Lo que es eso todos lo sabemos---respondí menospreciando la ciencia
de Inesilla.

---Bien, muchacho---continuó la profesora:---¿crees tú que una tortuga
puede volar, aunque esté meneando toda la vida sus torpes patas?

---No, seguramente.

---Pues tú pensando en ser hombre ilustre y poderoso, sin ser noble, ni
rico, ni sabio, eres como una tortuga que se empeñara en subir volando
al pico más alto de Guadarrama.

---Pero, reina y emperatriz---dije yo,---si no pienso subir solo, sino
que pienso encontrar, como otros que yo me sé, una personita que me suba
en un periquete. Hazme el favor de decirme cuál era la sabiduría y
riqueza \emph{del otro}, cuando le hicieron duque y generalísimo.

---Pero, señor duquillo---contestó ella jovialmente,---si esa personita
le sube a usted será como si un águila o buitre cogiera por su concha a
la tortuga para llevársela por los aires. Sí, te levantará: pero cuando
estés arriba, el pájaro, que no va a estarse toda la vida con tanto peso
en las patas, te dirá: «Ahora, niño mío, mantente solo.» Tú moverás las
patucas, pero como no tienes alas, pataplús, caerás en el suelo
haciéndote mil pedazos.

---¡Qué tonta eres!---dije con petulancia.---Eso pasa en las cosas que
se ven y se tocan; pero, chica, lo que se piensa y lo que se siente es
otro mundo aparte. ¿Qué tiene que ver una cosa con otra?

---Estás lucido, sí---repuso Inés.---Todo debe ser así mismamente.
Cuando tú quieres a una persona o cuando la aborreces, no es porque se
te antoje. ¡Ay!, chico: el corazón tiene también\ldots{} pues\ldots{} su
ley, y todo lo que pensamos con nuestra cabecita, va según lo que debe
ser y está mandado.

---¿Pero di, chiquilla, de dónde sabes tú todo eso?---le pregunté.

---¿Pero esto es saber?---respondió con naturalidad.---Pues esto lo
sabes tú y todos. De veras te digo que se me ocurrió cuando estabas
hablando, y que jamás había pensado en tales cosas.

---¡Picarona! Cuando menos, tienes escondido un rimero de libros, con
los cuales te vas a hacer doctora por Salamanca.

---No, hijito; no he leído más libros, fuera de los de devoción, que
\emph{Don Quijote de la Mancha}. ¿Ves? A ti te va a pasar algo de lo de
aquel buen señor: sólo que aquél tenía alas para volar, ¡pobrecillo!, lo
que le faltaba era aire en que moverlas.

Inesilla no dijo más. Yo callé también, porque a pesar de mi petulancia,
no pude menos de comprender, que las palabras de mi amiga encerraban
profundo sentido. ¡Y la que así hablaba era una modistilla! \emph{Ridete
cives}.

---Lo que yo sé---dije al fin, sintiendo en mí un vivo arrebato de
afecto,---es que te quiero, que te amo, que te adoro, que me subyugas y
dominas como a un papanatas, que eres una divinidad, y que juro no hacer
cosa alguna sin consultarte. Adiós, reinita: mañana te diré lo que se me
ocurra esta noche. Quién sabe, quién sabe, si llegaremos a ser\ldots{}
¿Por qué no? Es preciso estar dispuesto, porque la escalera de los
honores es penosa, y si uno se rompe la crisma, como dices\ldots{}

---Siempre quedará la del cielo---me dijo inclinando otra vez la cabeza
sobre la costura.

---Tienes cosas que me hacen estremecer. Adiós, Inesilla, luz y
pensamiento mío.

Dicho esto, me despedí de ella y salí. Al abandonar la casa la sentí
cantar, y su armoniosa voz se mezclaba en extraña disonancia con los
ecos de la flauta que tañía en lo interior de la morada el buen don
Celestino. Siempre que salía de allí, mi espíritu experimentaba un
reposo, una estabilidad, no sé cómo expresarlo, una frescura, que luego
destruía el trato con personas de diversa condición. De esto hablaré en
seguida; mas ante todo me cumple manifestar que Inesilla tenía razón al
burlarse de mis locos proyectos. Es el caso que como a todas horas oía
hablar de personajes nulos, a quienes el cortesano elevó a honrosas
alturas sin mérito alguno, se me antojó que la Providencia me reservaba,
como en compensación de mi orfandad y pobreza, una de aquellas
repentinas y escandalosas mudanzas que por entonces ocurrían en nuestra
España; y de tal modo se encajó en mi cerebro semejante idea, que llegó
a ser artículo de fe. Me hallaba por más señas en la edad en que somos
tontos. No todos poseen el don de saber las cosas \emph{desde hace mil
años}, como Inesilla.

Ahora verán ustedes la serie de circunstancias que llevaron mi necia
credulidad al último extremo. Para esto tengo que dar a cono cer a otras
personas, a quienes espero recibirá el lector con gusto. Hablemos, pues,
de teatros.

\hypertarget{iv}{%
\chapter{IV}\label{iv}}

El del Príncipe estaba ya reconstruido en 1807 por Villanueva, y la
compañía de Máiquez trabajaba en él, alternando con la de ópera,
dirigida por el célebre Manuel García; mi ama y la de Prado eran las dos
damas principales de la compañía de Máiquez. Los galanes secundarios
valían poco, porque el gran Isidoro, en quien el orgullo era igual al
talento, no consentía que nadie despuntara en la escena, donde tenía el
pedestal de su inmensa gloria y no se tomó el trabajo de instruir a los
demás en los secretos de su arte, temiendo que pudieran llegar a
aventajarle. Así es que alrededor del célebre histrión todo era mediano.
La Prado, mujer de Máiquez, y mi ama alternaban en los papeles de
primera dama, desempeñando aquélla el de Clitemnestra, en el
\emph{Orestes}, el de Estrella en \emph{Sancho Ortiz de las Roelas} y
otros. La segunda se distinguía en el de doña Blanca, de \emph{García
del Castañar}, y en el de Edelmira (Desdémona), del \emph{Otello}.

La compañía de ópera era muy buena. Además de Manuel García, que era un
gran maestro, cantaban su mujer Manuela Morales, un italiano llamado
Cristiani, y la Briones. De esta mujer, que era concubina de Manuel
García, nació el año siguiente el portento de las virtuosas, la reina de
las cantantes de ópera, Mariquita Felicidad García, conocida en su
tiempo por la \emph{Malibrán}.

Figúrense ustedes, señores míos, si estaría yo divertido con
representación o música por tarde y noche, asistiendo gratis, aunque por
dentro y en sitios donde se pierde parte de la ilusión, a las funciones
más bonitas y más aplaudidas que se celebraban en Madrid; rozándome con
guapísimas actrices, y familiarizado con los hombres que hacían reír o
llorar a la corte entera.

Y no piensen ustedes que sólo alternaba con los cómicos; gente que
entonces no era considerada como la nata de la sociedad; también me veía
frecuentemente en medio de personajes muy ilustres, de los que
menudeaban en los vestuarios; no faltando en tales sitios alguna dama
tan hermosa como linajuda de las que no desdeñaban de ensuciar su
guardapiés con el polvo de los escenarios.

Precisamente voy a contar ahora cómo mi ama tenía relaciones de íntima
amistad con dos señoras de la corte, cuyos títulos nobiliarios, de los
más ilustres y sonoros que desde remoto tiempo han exornado nuestra
historia, me propongo callar por temor a que pudieran enojarse las
familias que todavía los llevan. Estos títulos, que recuerdo muy bien,
no serán escritos en este papel; y para designar a las dos hermosas
mujeres emplearé nombres convencionales.

Recuerdo haber visto por aquel tiempo en la fábrica de Santa Bárbara un
hermoso tapiz en que estaban representadas dos lindas pastoras. Habiendo
preguntado quiénes eran aquellas simpáticas chicas, me dijeron: «Estas
son las dos hijas de Artemidoro: Lesbia y Amaranta.» He aquí dos nombres
que vienen de molde para mi objeto, amado lector. Haz cuenta que siempre
que diga \emph{Lesbia}, quiero significar a la duquesa de X, y cuando
ponga \emph{Amaranta}, a la condesa de X. Con este sistema quedan a
salvo todos los títulos nobiliarios de aquellas dos diosas de mi tiempo.
En cuanto a su hermosura, todo lo que mi descolorida pluma pueda
expresar será poco para describirlas, porque eran encantadoras,
especialmente la condesa de\ldots{} digo, Amaranta. Ambas tenían gusto
muy refinado por las artes, protegían a los pintores, aplaudían y
obsequiaban a los cómicos, ponían bajo su patrocinio las primeras
representaciones de la obra de algún poeta desvalido, coleccionaban
tapices, vasos y cajas de tabaco, introducían y propagaban las más
vistosas modas de la despótica París, se hacían llevar en litera a la
Florida, merendaban con Goya en el Canal, y recordaban con tristeza la
trágica muerte de Pepe Hillo, acontecida en 1803.

Nada tiene de extraño, pues, que su misma vida, la tumultuosa ansiedad
de novedades y fuertes impresiones que las dominaba, fuesen parte a
lanzarlas en un dédalo de aventuras, tales como las que voy a contar.
Las pobrecillas no sabían otra cosa, y puesto que habían perdido cuanto
la rancia educación española pudo haberles dado, sin adquirir nada que
llenase este vacío, no debemos culparlas acerbamente. Alguno quizás las
culpe, y con razón aunque por otras cosas; pero ¡ay!, eran\ldots{}
lindísimas.

Una tarde mi ama salió de muy mal humor del teatro. Isidoro la había
reprendido no sé por qué, y aquí debo advertir que el sublime actor
trataba a sus subalternos como si fueran chiquillos de escuela. Al
llegar Pepita a su casa me dijo:

---Prepara todo, que vendrán a cenar las señoras Lesbia y Amaranta.

El preparar todo, consistía en azotar un poco los muebles de la sala
para limpiar el polvo, o mejor dicho, para que el polvo variara de
sitio; en echar aceite en los velones; en comprar la prima para la
guitarra si le faltaba; en llamar a don Higinio para que afinase el
clave; limpiar las cornucopias; ir por nueva remesa de pomada a la
\emph{Marechala}, etc., etc. En cuanto a la cena, venía hecha de una
repostería. Di cumplimiento a estos encargos, y pedí nuevas órdenes;
pero mi ama estaba de muy mal humor, y sin hacer caso de lo que le
decía, me preguntó:

---¿No te dijo si venía esta noche?

---¿Quién?---pregunté.

---Isidoro.

---No, señora, no me ha dicho nada.

---Como hablaba contigo al concluir la representación\ldots{}

---Fue para decirme, que si volvía a enredar entre bastidores, mientras
él representaba, me mandaría desollar vivo.

---¡Qué genio! Le convidé para venir y no me contestó.

Después de esto no dijo más, y con ademán triste y sombrío se encerró en
su cuarto con la criada para cambiar de vestido. Seguí preparando todo,
y al poco rato apareció mi ama.

---¿Qué hora es?---preguntó.

---Las nueve acaban de dar en el reloj de la Trinidad.

---Me parece que siento ruido en el portal---dijo con mucha ansiedad.

---La señora se equivoca.

---¿De modo, que él no te dijo terminantemente si venía o no venía?

---¿Quién, Isidoro? No señora; nada me dijo.

---Como tiene ese genio tan\ldots{} ya ves que incomodado estaba esta
tarde. Sin embargo, yo creo que vendrá. Le convidé ayer, y aunque no me
dijo una palabra\ldots{} él es así.

Al decir esto, mostraba en su semblante una inquietud, una agitación,
una zozobra, que eran señales de las más vivas emociones de su alma. ¿A
qué tanto interés por la asistencia de Isidoro, persona a quien
diariamente veía en el teatro?

Después examinó la sala, por ver si faltaba algo, y se sentó aguardando
la llegada de sus convidados. Al fin sentimos abrir la puerta de la
calle, y pasos de hombre sonaron en la escalera.

---Es él---dijo mi ama, levantándose de un salto y andando con cierto
atolondramiento por la habitación.

Yo corrí a abrir, y un instante después el gran actor entró en la sala.

Isidoro era un hombre de treinta y ocho años, de alta estatura, actitud
indolente, semblante pálido, y con tal expresión en éste y en la mirada,
que observado una vez, su imagen no se borraba nunca de la memoria.
Aquella noche traía un traje verde oscuro, con pantalón de ante y botas
polonesas, prendas todas de irreprensible elegancia que usaba con más
propiedad que ninguno. Su vestir era un modo de ser propio y personal;
él constituía por sí una especie de moda, y no se podía decir que se
sometiera; cual dócil lechuguino, al uso común. En otros infringir las
reglas habría sido ridículo; pero en él infringirlas era lo mismo que
modificarlas o crearlas de nuevo.

Ya os lo daré a conocer más adelante como actor. Por ahora podréis
conocer algunos rasgos de su carácter como hombre. Al entrar se arrojó
sobre un sillón sin saludar a mi ama más que con una de esas fórmulas
familiares e indiferentes que se emplean entre personas acostumbradas a
verse con frecuencia. Por un buen rato permaneció sin decir nada,
tarareando un aria con la vista fija en las paredes y el techo, y sin
dejar de golpearse la bota con el bastón.

Salí de la sala a traer no sé qué cosa, y al volver oí a Isidoro que
decía:

---¡Qué mal has representado esta tarde, Pepilla!

Observé que mi ama, turbada como una chicuela ante el fiero maestro de
escuela, no supo contestar más que con trémulas frases a aquella brusca
reprensión.

---Sí---continuó Isidoro;---de algún tiempo a esta parte estás
desconocida. Esta tarde todos los amigos se han quejado de ti y te han
llamado fría, torpe\ldots{} Te equivocabas a cada instante, y parecías
tan distraída que era preciso que yo te llamara la atención para que
salieras de tu embobamiento.

Efectivamente, según oí entre bastidores aquella tarde, mi ama había
estado muy infeliz en su papel de Blanca, en \emph{García del Castañar}.
Todos los amigos estaban admirados, considerando la perfección con que
la actriz había desempeñado en otras ocasiones papel tan difícil.

---Pues no sé---respondió mi ama con voz conmovida.---Yo creo que he
representado esta tarde lo mismo que las demás.

---En algunas escenas sí; pero en las que dijiste conmigo, estuviste
deplorable. Parece que habías olvidado el papel, o que trabajabas de
mala gana. En la escena de nuestra salida recitaste tu soneto como una
cómica de la legua que representa en Barajas o en Cacabelos. Al decirme:

\small
\newlength\mlenc
\settowidth\mlenc{que en los fragantes vasos el sol bebe...}
\begin{center}
\parbox{\mlenc}{    No quieren más las flores al rocío          \\ 
                que en los fragantes vasos el sol bebe...}      \\ 
\end{center}
\normalsize

\justifying{\noindent tu voz temblaba como la de quien sale por primera vez a las tablas...
me diste la mano y la tenías ardiendo, como si estuvieras con calentura... te
equivocabas a cada momento, y parecías no hacer maldito caso de que yo estaba
en la escena.}

---¡Oh, no\ldots{} pero te diré! El mismo miedo de hacerlo mal. Temía
que te enfadaras, y como nos reprendes con tanta violencia cuando nos
equivocamos\ldots{}

---Pues es preciso que te enmiendes, si quieres seguir en mi compañía.
¿Estás enferma?

---No.

---¿Estás enamorada?

---¡Oh, no, tampoco!---contestó la actriz con turbación.

---Apuesto a que por atender demasiado a alguna persona de las lunetas,
no acertabas con los versos de la comedia.

---No, Isidoro; te equivocas---dijo mi ama afectando buen humor.

---Lo raro es que en las escenas que siguieron, sobre todo en la de don
Mendo, hiciste perfectamente tu papel; pero luego en el tercer acto
cuando te tocó otra vez declamar conmigo, vuelta a las andadas.

---¿Dije mal el parlamento del bosque?

---No: al contrario; recitaste con buena entonación los versos

\small
\newlength\mlend
\settowidth\mlend{Llorad, ojos, llorad mi desventura.}
\begin{center}
\parbox{\mlend}{    ¿Dónde voy sin aliento,                   \\ 
                cansada, sin amparo, sin intento,             \\ 
                entre aquesta espesura?                       \\ 
                Llorad, ojos, llorad mi desventura.}          \\ 
\end{center}
\normalsize

\justifying{\noindent En la escena con la reina también estuviste muy feliz, lo mismo que
en el diálogo con don Mendo. Con qué elocuente tono exclamaste
«¡tengo esposo!» y después aquello de}

\small
\newlength\mlene
\settowidth\mlene{Llorad, ojos, llorad mi desventura.}
\begin{center}
\parbox{\mlene}{                                    Sí harán, \\
                porque bien o mal nacido,                     \\
                el más indigno marido                         \\
                excede al mejor galán.}                       \\
\end{center}
\normalsize

\justifying{\noindent Pero desde que salí yo y me viste...}

---Es lo que digo. El temor de hacerlo mal y disgustarte\ldots{}

---Pues me has disgustado de veras. Cuando decías: «Esposo mío, García,»
te hubiera dado un pescozón en medio de la escena y delante del público.
Marmota, ¿no te he dicho mil veces cómo deben pronunciarse esas
palabras? ¿No has comprendido todavía la situación? Blanca teme que su
marido sospecha una falta. El contento que experimenta al verle, y el
temor de que García dude de su inocencia, deben mezclarse en aquella
frase. Tú, en vez de expresar estos sentimientos, te dirigiste a mí como
una modistilla enamorada, que se encuentra de manos a boca con su
querido hortera. Luego cuando me suplicabas que te matara, lo hiciste
sin lo que llamamos nosotros decoro trágico. Parecía que realmente
deseabas recibir la muerte de mi mano, y hasta te pusiste de hinojos
ante mí, cuando te tengo dicho terminantemente que no hagas tal cosa,
sino en los pasajes en que te lo ordene. En las décimas

\small
\newlength\mlenf
\settowidth\mlenf{Llorad, ojos, llorad mi desventura.}
\begin{center}
\parbox{\mlenf}{García, guárdete el cielo,}                   \\
\end{center}
\normalsize

\justifying{ \noindent te equivocaste más de veinte veces, y cuando yo dije:}

\small
\newlength\mleng
\settowidth\mleng{Llorad, ojos, llorad mi desventura.}
\begin{center}
\parbox{\mleng}{¡Ay, querida esposa mía,                      \\
                qué dos contrarios extremos!}                 \\
\end{center}
\normalsize

\justifying{ \noindent te arrojaste en mis brazos, cuando aún no era llegada la ocasión,
y yo, preocupado por el agravio recibido, no podía entregarme a
halagos amorosos. Echaste a perder el final, Pepilla, desluciste la
comedia y me desluciste a mí.}

---Yo no puedo deslucirte nunca.

---Pues ya ves cómo no fui aplaudido esta tarde como las anteriores; y
de esto tienes tú la culpa, sí, tú misma, por tus torpezas y tus
tonterías. No haces caso de mis lecciones, no te esfuerzas por
complacerme, y por último, me pondrás en el caso de quitarte el partido
en mi compañía, poniéndote de parte de por medio o racionera, si no me
obligas con tus descuidos a echarte del teatro.

---¡Ay Isidoro!---dijo mi ama.---Yo procuro siempre hacerlo lo mejor
posible para que no te enfades ni me riñas; pero tanto miedo tengo a que
me reprendas que en la escena tiemblo desde que te veo aparecer.
¿Querrás creer una cosa? Pues cuando estamos representando juntos, hasta
temo hacerlo demasiado bien porque si me aplauden mucho, me parece que
tomo para mí una parte del triunfo que a ti sólo corresponde, y creo que
has de enfadarte si no te aplauden a ti solo. Este temor, unido al que
me causas cuando me amenazas por señas o me corriges con enojo me hace
temblar y balbucir, y a veces no sé lo que me digo. Pero descuida que ya
me enmendaré: no tendrás que echarme de tu teatro.

No oí lo que siguió a estas palabras, porque salí con un velón que
exhalaba mal olor; al volver noté que la conversación había variado.
Isidoro permanecía en el sillón con indolencia y mostrando un gran
aburrimiento.

---¿Pero no vienen tus convidados?---preguntó.

---Es temprano. Veo que te fastidias en mi compañía---contestó mi ama.

---No; pero la reunión hasta ahora no tiene nada de divertida.

Isidoro sacó un cigarro y fumó. Debo advertir que el ilustre actor no
gastaba tabaco por las narices, como todos los grandes hombres de su
tiempo, Talleyrand, Metternich, Rossini, Moratín y el mismo Napoleón,
que si no miente la historia por abreviar la operación de sacar y
destapar la tabaquera, llevaba derramado el aromático polvo en el
bolsillo del chaleco, forrado interiormente de hules; y mientras
disponía los escuadrones de Jena, o durante las conferencias de Tilsitt,
no cesaba de meter en el susodicho bolsillo los dedos pulgar e índice
para llevarlos a la nariz cada minuto. Por esta singular costumbre dicen
que el chaleco amarillo y las solapas que cubrían el primer corazón del
siglo, eran una de las cosas más sucias que se han señoreado de la
Europa entera.

Farinelli también se atarugaba las narices entre un aria y un oratorio,
y de ciertos papeles viejos que hemos visto, se desprende que el mejor
regalo que podía hacer una dama enamorada, o un noble entusiasta, a
cualquier músico, pintor o virtuoso italiano, era un par de arrobas de
tabaco.

El abate Pico de la Mirandola, Rafael Mengs, el tenor Montagnana, la
soprano Pariggi, el violinista Alaí y otras notabilidades del teatro del
Buen Retiro, consumieron lo mejor que venía de América en los regios
galeones.

Perdóneseme la digresión, y conste que Isidoro no usaba tabaco en polvo.

\hypertarget{v}{%
\chapter{V}\label{v}}

Las diez serían cuando solemnemente entraron las dos damas de que antes
hice mención. ¡Lesbia, Amaranta! ¿Quién podrá olvidaros si alguna vez os
vio? Excusado es decir que iban de incógnito, y en coche, no en litera
donde fácil hubiera sido conocerlas al indiscreto vulgo. Las pobrecillas
gustaban mucho de aquellas reuniones de confianza, donde hallaban
desahogo sus almas comprimidas por la etiqueta.

Ha de saberse que en las reuniones clásicas de familia o de palacio, en
las reuniones donde reinaba con despótico imperio la ley castiza, no
ocurría cosa alguna que no fuese encaminada a producir entre los
asistentes un decoroso aburrimiento. No se hablaba, ni mucho menos se
reía. Las damas ocupaban el estrado, los caballeros el resto de la sala,
y las conversaciones eran tan sosas como los refrescos. Si alguien
tocaba el clave o la guitarra, la tertulia se animaba un poco; pero
pronto volvía a reinar el más soporífero decoro. Se bailaba un minueto;
entonces los amantes podían saborear las platónicas e ideales delicias
que resultaban de tocarse las yemas de los dedos, y después de muchas
cortesías hechas con música, volvía a reinar el decoro, que era una
deidad parecida al silencio.

Nada tiene de particular que algunas damas de imaginación buscaran en
reuniones menos austeras pasatiempos más acordes con su naturaleza, y
aquí traigo a la memoria \emph{El sí de las niñas}, que censurando la
hipocresía en la educación, es una general censura de la hipocresía en
todas las fases de nuestras antiguas costumbres. Todo anunciaba en
aquellos días una fuerte tendencia a adoptar usos un poco más libres,
relaciones más francas entre ambos sexos, sin dejar de ser honradas,
vida en fin, que se fundara antes en la confianza del bien, que en el
recelo del mal, y que no pusiera por fundamentos de la sociedad la
suspicacia y la probabilidad del pecado. La verdad es que había mucha
hipocresía entonces: porque las cosas no se hicieran en público, no
dejaban de hacerse, y siendo menos libres las costumbres, no por eso
eran mejores.

Lesbia y Amaranta entraron haciendo cortesías y gestos encantadores, que
revelaban la alegría de sus corazones. Las acompañaba el tío de
Amaranta, viejo marqués diplomático: pero antes de decir quién era éste,
voy a referiros cómo eran ellas.

La duquesa de X (Lesbia), era una hermosura delicada y casi infantil, de
esas que, semejantes a ciertas flores con que poéticamente son
comparadas, parece que han de ajarse al impulso del viento, al influjo
de un fuerte sol, o perecer deshechas si una débil tempestad las agita.
Las que se desataron en el corazón de Lesbia no hicieron estrago alguno,
al menos hasta entonces, en su belleza.

Parecía haber salido el día antes del poder de las buenas madres de
Chamartín de la Rosa y que aún no sabía hablar sino de los bollos del
convento, de las hormigas de la regla de San Benito y de los cariños de
la madre Circuncisión. ¡Pero cómo desmentía esta creencia en cuanto
comenzaba a hablar la muy picarona! En su lenguaje tomaba mucha parte la
risa, con tanta franqueza y tan discreta desenvoltura, que nadie estaba
triste en su presencia. Era rubia, y no muy alta, aunque sí esbelta y
ligera como un pajarito. Todo en ella respiraba felicidad y satisfacción
de sí misma; era una naturaleza tan voluntariosa como alegre, a quien
ningún extraño albedrío podía sujetar. Los que tal intentaron
principiarían por enojarla, y enojarla era echarla a perder destruyendo
la mitad de sus encantos.

Entre las cualidades que hacían agradable el trato de Lesbia descollaba
su habilidad en el arte de la declamación. Era una cómica consumada, y
según conocí después, su talento sin igual para la escena no se reducía
a los estrechos lienzos pintados de los teatros caseros, sino que tomaba
más ancho vuelo, desplegándose en todos los actos de la vida. Siempre
que se daba alguna función extraordinaria en cualquiera de las
principales casas de la corte, ella hacía la mejor parte, y a la sazón
Máiquez le enseñaba el papel de Edelmira en la tragedia \emph{Otello},
que debía ponerse en escena en el teatro doméstico de cierta marquesa.
Isidoro y mi ama estaban también designados para cooperar en aquella
representación, anunciada como muy espléndida.

Lesbia era casada. Tres años antes, y cuando apenas tenía diez y nueve,
contrajo matrimonio con un señor duque que se pasaba el tiempo cazando
como un Nemrod en sus vastas dehesas: venía alguna vez a Madrid hecho un
zafiote para pedir perdón a su mujer por las largas ausencias, y jurarle
que tenía el propósito de no disgustarla más, viviendo lejos de ella.
Sin que nadie me lo diga, afirmo que Lesbia se quejaría con su dulce
vocecita; pero cuidando de no esforzar su queja en términos que pudieran
decidir al duque a cambiar de vida.

Amaranta era un tipo enteramente contrario al de Lesbia. Ésta agradaba;
pero Amaranta entusiasmaba. La apacible y graciosa hermosura de la
primera hacía pasajeramente felices a cuantos la miraban. La belleza
ideal y grandiosa de la segunda causaba un sentimiento extraño, parecido
a la tristeza. Pensando en esto después, he creído que la singular
estupefacción que experimentamos ante uno de estos raros portentos de la
hermosura humana, consiste o en creencia de nuestra inferioridad o en la
poca esperanza de poseer el afecto de una persona, que a causa de sus
muchas perfecciones, será solicitada por sinnúmero de golosos.

Entre las mujeres que he visto en mi vida, no recuerdo otra que poseyera
atracción tan seductora en su semblante, así es que no he podido
olvidarla nunca, y siempre que pienso en las cosas acabadas y
superiores, cuya existencia depende exclusivamente de la Naturaleza, veo
su cara y su actitud como intachables prototipos que me sirven para mis
comparaciones. Amaranta parecía tener treinta años. La gloria de haber
producido a aquella mujer te pertenece en primer término a ti,
Andalucía, y después a ti, Tarifa, fin de España, rincón de Europa donde
se han refugiado todas las gracias del tipo español, huyendo de
extranjera invasión.

Con lo dicho podrán ustedes formar idea cómo era la incomparable condesa
de X, \emph{alias} Amaranta, y excuso descender a pormenores que ustedes
podrán representarse fácilmente, tales como su arrogante estatura, la
blancura de su tez, el fino corte de todas las líneas de su cara, la
expresión de sus dulces y patéticos ojos, la negrura de sus cabellos y
otras muchas indefinidas perfecciones que no escribo, porque no sé cómo
expresarlas; calidades que se comprenden, se sienten y se admiran por el
inteligente lector, pero cuyo análisis no debe éste exigirnos, si no
quiere que el encanto de esas mil sutiles maravillas se disipe entre los
dedos de esta alquimia del estilo, que a veces afea cuanto toca.

No conservo cabal memoria de sus vestidos. Al acordarme de Amaranta, me
parece que los encajes negros de una voluminosa mantilla, prendida entre
los dientes de la más fastuosa peineta, dejan ver por entre sus mil
recortes e intersticios el brillo de un raso carmesí, que en los hombros
y en las bocamangas vuelve a perderse entre la negra espuma de otros
encajes, bolillos y alamares. La basquiña del mismo raso carmesí y tan
estrecha y ceñida como el uso del tiempo exigía, permite adivinar la
hermosa estatua que cubre; y de las rodillas abajo el mismo follaje
negro y la cuajada y espesa pasamanería terminan el traje, dejando ver
los zapatos, cuyas respingadas puntas aparecen o se ocultan como
encantadores animalitos que juegan bajo la falda. Este accidente hasta
llega a ser un lenguaje cuando Amaranta, atenta a la conversación,
aumenta con el encanto de su palabra los demás encantos, y añade a todas
las elocuencias de su persona la elocuencia de su abanico.

Esto en cuanto a la condesa. Refiriéndome a Lesbia, si quiero acordarme
de su vestido, todo me parece azul. Figúrensela ustedes con mantilla
blanca y guarda-pies azul bordado de encajes negros; y si no es cierto
que estuviera así, tampoco es inverosímil que pudiera estarlo.

Antes de la noche a que me refiero, había visto hasta tres veces a las
dos lindas mujeres en casa de mi ama. Desde luego comprendí que una y
otra eran personas muy metidas en los enredos de la corte, aunque en las
clandestinas tertulias de mi casa poco dejaban traslucir. Algunas veces,
sin embargo, disputaban las dos en tales términos y con tan mal
disimulado ensañamiento que me pareció no existía entre ellas la mejor
armonía. También mentaban de vez en cuando los negocios públicos, y a
tal o cual persona de la real familia: pero en tales casos siempre daba
el tema el señor marqués y tío de Amaranta, personaje que no podía estar
en sosiego, si no realzaba a todas horas su personalidad, sacando a
relucir a tontas y a locas los negocios diplomáticos en que se creía muy
experto.

La noche a que corresponde mi narración, había asistido también el
celebérrimo tío, de quien ante todo diré que parecía cosido a las faldas
de su sobrina, pues la acompañaba a todas partes, sirviéndole de
rodrigón en la iglesia, de caballero en el paseo y de pareja en los
bailes. No sé si he dicho que Amaranta era viuda. Si antes lo dije, dese
por repetido.

El marqués (callemos el título por las mismas razones que nos movieron a
disfrazar el de las damas) era un viejo de mas de sesenta años, que
había ejercido varios cargos diplomáticos. Elevado por Floridablanca,
sostenido por Aranda, y derribado al fin por Godoy, conservó rencorosa
pasión contra este ministro, y por esta causa todas sus disertaciones,
que eran interminables, giraban sobre el capitalísimo tema de la caída
del favorito. Su carácter era vano, aparatoso y hueco, como de hombre
que habiéndose formado de sí mismo elevado concepto, se cree destinado a
desempeñar los más altos papeles. Por su grandilocuencia, que no era
inferior a la flojedad efectiva de su ánimo, servía como objeto de
agudísimas burlas entre sus amigos, y en todos los círculos que
frecuentaba, se divertían oyéndole decir: \emph{¿Qué hará la
Rusia?\ldots{}} \emph{¿Secundará el Austria tan atroz proyecto?}
\emph{¡Un gran desastre nos amaga!\ldots{}} \emph{¡Ay de las potencias
del Mediodía!..}. y otras igualmente misteriosas, con que se proponía
darse importancia, cuidando siempre en su estudiada reserva de decir las
cosas a medias, y de no dar noticias claras de nada, para que los
oyentes, llenos de dudas y oscuridades, le rogasen con insistencia que
fuese más explícito.

He dado estos detalles para que se comprenda qué clase de espantajos
había entonces para regocijo de aquella generación. En cuanto a mí,
siempre me han hecho gracia estos tipos de la vanidad humana, que son
sin disputa los que más divierten y los que más enseñan.

Como hombre poco dispuesto a transigir con las \emph{novedades
peligrosas}, y enemigo del jacobinismo, el marqués se esforzaba en
conseguir que su persona fuese espejo fiel de sus elevados pensamientos,
así es que miraba con desdén los trajes de moda, y tenía gusto en
sorprender al público elegante de la corte y villa con vestidos
anticuados de aquéllos que sólo se veían ya en la veneranda persona de
algún buen consejero de Indias. Así es que si usó hasta 1798 la casaca
de tontillo y la chupa mandil, en 1807 todavía no se había decidido a
adoptar el frac solapado y el chaleco ombliguero, que los poetas
satíricos de entonces calificaban de moda \emph{anglo-gala}.

Me falta añadir que el marqués, con su anti-jacobinismo y su peluca
empolvada, digna de figurar en las Juntas de Coblentza, había sido
hombre de costumbres bastante disipadas. En la época de mi relación la
edad le había corregido un poco, y todas sus calaveradas no pasaban de
una benévola complicidad en todos los caprichos de su sobrina. No
vacilaba en acompañarla a sus excursiones y meriendas en la pradera del
Canal o en la Florida, con gente de categoría muy inferior a la suya.
Tampoco ponía reparos en ser su pareja en las orgías celebradas en casa
de la González o la Prado, pues tío y sobrina gustaban mucho de aquella
familiaridad con cómicos y otra gente de parecida laya. Excusado es
decir que tales excursiones eran secretas y tenían por único objeto el
esparcir y alegrar el espíritu abatido por la etiqueta. ¡Pobre gente!
Aquellos nobles que buscaban la compañía del pueblo para disfrutar
pasajeramente de alguna libertad en las costumbres estaban consumando,
sin saberlo, la revolución que tanto temían, pues antes de que vinieran
los franceses y los volterianos y los doceañistas, ya ellos estaban
echando las bases de la futura igualdad.

\hypertarget{vi}{%
\chapter{VI}\label{vi}}

Lesbia, dando golpecitos con su abanico en el hombro de Isidoro, decía:

---Estoy muy enfadada con usted señor Máiquez, sí señor, muy enfadada.

---¿Porque he representado mal esta tarde?---contestó el
actor.---Pepilla tiene la culpa.

---No es eso---continuó la dama,---y me las pagará usted todas juntas.

Al oír esto, Isidoro inclinó la cabeza. Lesbia acercó su rostro, y habló
tan bajo, que ni yo ni los demás entendimos una palabra; pero por la
sonrisa de Máiquez se adivinaba que la dama le decía cosas muy dulces.
Después continuaron hablando en voz baja, y el uno atendía a las
palabras del otro con tal interés, daban tanta fuerza y energía al
lenguaje de los ojos, se ponían serios o joviales, tristes o alborozados
con transición tan ansiosa y brusca, que al más listo se le alcanzaba la
injerencia del travieso amor en las relaciones de aquellos dos
personajes.

Para que todo se sepa de una vez, diré que el diplomático no miraba con
malos ojos a la González; mas ésta no podía contestar a sus tiernas
insinuaciones, porque harto tenía que hacer atendiendo al íntimo diálogo
que sostenían Lesbia e Isidoro. A mi ama un color se le iba y otro se le
venía, de pura zozobra: a veces parecía encendida en violenta ira; a
veces dominada por punzante dolor, pugnaba por distraerles, injiriendo
en su conversación conceptos extraños, y al fin, no pudiendo contenerse,
dijo con muy mal humor.

---¿No concluirá tan larga confesión? Si siguen ustedes así, entonaremos
el \emph{Yo, pecador}.

---¿Y a ti qué te importa?---dijo Máiquez con semblante sañudo y con
aquel despótico tono que usaba con los desdichados subalternos de su
compañía.

Mi ama se quedó perpleja, y en un buen rato no dijo una palabra.

---Tienen que contarse muchas cosas---dijo Amaranta con malicia.---Lo
mismo sucedió el otro día en casa. Pero estas cosas pasan, señor
Máiquez. El placer es breve y fugaz. Conviene aprovechar las dulzuras de
la vida hasta que el horrible hastío las amargue.

Lesbia miró a su amiga\ldots{} Mejor dicho, ambas se miraron de un modo
que no indicaba la existencia de una apacible concordia entre una y
otra.

El secreto entre Isidoro y la dama continuaba cada vez más íntimo, más
ardoroso, más impaciente. Parecía que el tiempo se les abreviaba entre
palabra y palabra, no permitiéndoles decirlo todo. Amaranta se aburría,
el Marqués dirigía con ojos y boca inútiles flechas al enajenado corazón
de mi ama, y ésta cada vez más inquieta, mostrando en su semblante ya la
interna rabia de los celos, ya la dolorosa conformidad del martirio, no
procuraba entablar conversación, ni parecía cuidarse de sus convidados.
Pero al fin el marqués, comprendiendo que aquélla era ocasión propicia
para hablar, aunque fuera ante mujeres, de su tema favorito que eran los
asuntos públicos, rompió el grave silencio y dijo:

---La verdad es que estamos aquí divirtiéndonos, y a estas horas tal vez
se preparan cosas que mañana nos dejarán a todos asombrados y lelos.

Hallándose mi ama, como he dicho, absorta entre el despecho y la
resignación, se dejó dominar del primero, que la inducía a trabar otro
diálogo íntimo con el diplomático, y dijo con viveza:

---¿Pues qué pasa?

---Ahí es nada\ldots{} Parece mentira que estén ustedes con tanta
calma---contestó el marqués retardando el dar las noticias.

---Dejemos esas cuestiones que no son de este lugar---dijo la sobrina
con hastío.

---¡Oh, oh, oh!---exclamó con grandes aspavientos el diplomático.---¡Por
qué no han de serlo! Yo sé que Pepa desea vivamente saber lo que pasa, y
saberlo de mis autorizados labios: ¿no?

---Sí, muchísimo; quiero que usted me cuente todo---dijo mi ama.---Esas
cosas me encantan. Estoy de un humor\ldots{} divertidísimo: hablemos,
hablemos, señor marqués.

---Pepa, usted me electriza---dijo el marqués clavando en ella con amor
sus turbios y amortiguados ojos.---Tanto es así, que yo, a pesar de
haberme distinguido siempre, durante mi carrera diplomática, por mi gran
reserva, seré con usted franco, revelándole hasta los más profundos
secretos de que depende la suerte de las naciones.

---¡Oh!, me encantan los diplomáticos---dijo mi ama con cierta agitación
febril.---Hábleme usted, cuénteme todo lo que sepa. Quiero estar
hablando con usted toda la noche. Es usted, señor marqués, la persona de
conversación más dulce, más amena, más divertida que he tratado en mi
vida.

---Nada te dirá, Pepa, sino lo que todo el mundo sabe---indicó
Amaranta,---y es que a estas horas las tropas de Napoleón deben de estar
entrando en España.

---¡Oh, qué cosa más linda!---dijo mi ama.---Hable usted, señor marqués.

---Sobrina, ¿acabarás de apurarme la paciencia?---exclamó el marqués,
dando importancia extraordinaria al asunto.---No se trata de que entren
o no entren esas tropas, se trata de que van a Portugal a apoderarse de
aquel reino para repartirlo\ldots{}

---¿Para repartirlo?---dijo la González con su calenturienta
jovialidad.---Bien; me alegro. Que se lo repartan.

---Lindísima Pepa, esas cosas no pueden decidirse tan de ligero---dijo
el marqués gravemente.---¡Oh, usted aprenderá conmigo a tener juicio!

---Es cierto---añadió Amaranta---que se ha acordado dividir a Portugal
en tres pedazos: el del Norte se dará a los reyes de Etruria; el centro
quedará para Francia y la provincia de Algarbes y Alentejo, servirá para
hacer un pequeño reino, cuya corona se pondrá el señor Godoy en su
cabeza.

---¡Patrañas, sobrina, patrañas!---dijo el marqués.---Eso es lo que dio
tanto que hablar el año pasado; pero ¿quién se acuerda ya de semejante
combinación? Tú no estás al tanto de lo que pasa\ldots{} Por supuesto,
no necesito repetir que es preciso guardar absoluto secreto sobre lo que
voy a decir.

---¡Ah!, descuide usted---repuso mi ama.---En cuanto a mí, estoy
encantada de esta conversación.

---El año pasado Godoy trató de ese asunto, por medio de Izquierdo, su
representante reservado, con Napoleón. Parece que la cosa estaba
arreglada. Pero de repente el emperador pareció desistir, y entonces don
Manuel, ofendido en su amor propio y viendo defraudadas sus esperanzas,
quiso mostrarse fuerte contra Napoleón, publicó la famosa proclama de
octubre del año pasado, y envió un mensajero secreto a Inglaterra, para
tratar de adherirse a la coalición de las potencias del Norte contra
Francia. Esto lo tengo yo muy sabido\ldots{} porque ¿qué secreto puede
escaparse a mi penetración y consumada experiencia de estos arduos
negocios? Bien\ldots{} así las cosas, venció Napoleón a los prusianos en
Jena, y ya tenemos a nuestro don Manuel asustadico y hecho un lego
motilón, temiendo la venganza del que había sido gravemente ofendido con
la publicación de la proclama, considerada aquí y en Francia como una
declaración de guerra. Envió a Izquierdo a Alemania, para implorar
perdón, y al fin le fue concedido; pero no se volvió a hablar más del
reparto de Portugal, ni de la soberanía de los Algarbes. He aquí,
señoras, la pura verdad. Yo, por mis antecedentes y mis conocimientos,
estoy al tanto de todos estos asuntos, pues al paso que los atisbo y
escudriño aquí, no falta algún diplomático extranjero que me los
comunique con toda reserva. Hoy no se habla ya del reparto de Portugal,
señora sobrinita. Lo que ocurre es mucho más grave, y\ldots{} pero no,
no somos dueños de comunicar a nadie ciertas cosas. Callaré hasta que el
gran cataclismo se haga público\ldots{} ¿Aprueba usted mi discreción,
querida Pepa? ¿Conviene usted conmigo en que la reserva es hermana
gemela de la diplomacia?

---¡Oh, la diplomacia!---exclamó mi ama con afectación.---Es cosa que me
tiene enamorada. ¡La pérfida Albión! ¡Los tratados! ¡Bonaparte! ¡La
coalición! ¡Oh, qué asuntos tan divinos! Confieso que hasta aquí me han
aburrido mucho; pero ahora\ldots{} esta noche, rabio por conocerlos, y
esta conversación, señor marqués, me tiene embelesada.

---Es verdad---dijo el diplomático relamiéndose de satisfacción,---qué
pocas personas tratan de estas materias con tanta delicadeza, con tanta
prudencia, digámoslo de una vez, con tanta gracia como yo. Cuando estaba
en Viena por el año 84 todas las damas de la corte me rodeaban, y si
vieran ustedes cómo pasaban el rato oyéndome\ldots{}

---Lo comprendo: lo mismo me pasa a mí esta noche---dijo mi ama sin
cesar en su extraña exaltación.---Por piedad, hábleme usted del Austria,
de la Turquía, de la China, del protocolo y de la guerra; sobre todo de
la guerra.

---Dejemos a un lado, por esta noche tan fastidiosa
conversación---indicó Amaranta.---No creo que usted, querido tío, sea de
la ridícula opinión que se supone que Godoy intenta, con el auxilio de
Bonaparte, mandar a América a la Real familia, quedándose él de rey de
España.

---Sobrina, por todos los santos, no me incites a hablar; no me hagas
olvidar el gran principio de que la discreción es hermana gemela de la
diplomacia.

---Es absurdo también---continuó la sobrina---suponer que Napoleón haya
mandado sus tropas a España para poner la corona al príncipe Fernando.
El heredero de un trono no puede solicitar el favor de un soberano
extranjero para ningún fin contrario a los de sus reales padres.

---Vamos, vamos, señoras, asuntos tan graves no pueden tratarse de
ligero. Si yo me decidiera a hablar, se quedarían ustedes espantadas, y
no podríamos cenar.

A esta sazón ya había venido la cena, y yo comenzaba a servirla. Isidoro
y Lesbia, requeridos por mi ama para que se acercaran a la mesa, dieron
tregua al arrobamiento y tomaron parte por un rato en la conversación
general.

---¿Pero, qué están ustedes hablando?---dijo Lesbia.---¿Hemos venido
aquí para ocuparnos de lo que no nos importa? ¡Bonito tema!

---¿Pues de qué quiere usted que se hable, desgraciada?

---De otras cosas\ldots{} vamos; de bailes, de toros, de comedias, de
versos, de vestidos\ldots{}

---¡Qué sosada!---indicó mi ama con desdén.---Además, ustedes pueden
tratar de lo que gusten, y nosotras hablaremos de lo que más nos
convenga.

---Ya veo por qué anda Pepa tan distraída---dijo Máiquez burlándose de
mi ama.---Se ha dedicado a estudiar la política y la diplomacia,
carreras más propias de su ingenio que la del teatro.

Mi ama intentó contestar a esta mofa, pero las palabras expiraron en sus
labios y se puso muy encendida.

---Aquí venimos a divertirnos---añadió Lesbia.

---¡Oh, frívola y vana juventud!---exclamó el marqués después de beberse
un gran vaso de vino.---No piensa más que en divertirse, cuando la
Europa entera\ldots{}

---Dale con la Europa entera.

---Pepa es la única que comprende la gravedad de las circunstancias.
Usted, encantadora actriz, será de las pocas que, como yo, no se
sorprendan del cataclismo.

---¿Querrá usted explicarnos de una vez lo que va a pasar?

---¡Por Dios y todos los santos!---exclamó el diplomático afectando
cierta compunción suplicante.---Yo ruego a ustedes que no me obliguen
con sus apremiantes excitaciones a decir lo que no debe salir de mis
labios. Aunque tengo confianza en mi propia prudencia, temo mucho que si
ustedes siguen hostigándome, se me escape alguna frase, alguna
palabra\ldots{} Callen ustedes por Dios, que la amistad tiene en mí
fuerza irresistible, y no quiero verme obligado por ella a olvidar mis
honrosos antecedentes.

---Pues callaremos: no deseamos saber nada, señor marqués---dijo
Máiquez, comprendiendo que el mejor medio para mortificar al buen viejo
consistía en no preguntarle cosa alguna.

Hubo un momento de silencio. El marqués, contrariado en su locuacidad,
no cesaba de engullir, entablando relaciones oficiosas con un capón, e
impetrando para este fin los buenos oficios de una ensalada de escarola,
que le ayudaba en sus negociaciones. Mientras tanto se deshacía en
obsequios con mi ama, y sus turbios ojos, reanimados no sé si por el
vino o por el amor, brillaban entre los arrugados párpados y bajo las
espesas cenicientas cejas que contraía siempre en virtud de la costumbre
de leer la vieja escritura de los \emph{memorandums}. La González no
decía tampoco una palabra, y sólo ponía su reconcentrada atención,
aunque sin mirarlos, en los dos amantes, mientras que Amaranta, agitada
sin duda por pensamientos muy diferentes, no miraba a Isidoro ni a
Lesbia, ni a mi ama, ni a su tío, sino\ldots{} ¿tendré valor para
decirlo?, me miraba a mí. Pero esto merece capítulo aparte, y pongo
punto final en éste para descansar un poco.

\hypertarget{vii}{%
\chapter{VII}\label{vii}}

Sí, ¿lo creerán ustedes?, me miraba, ¡y de qué modo! Yo no podía
explicarme la causa que motivaba aquella tenaz curiosidad, y si he decir
verdad como hombre honrado, aún no he salido de dudas. Yo servía a la
mesa, como es de suponer, y no pueden ustedes figurarse cuál fue mi
turbación cuando advertí que aquella hermosa dama, objeto por parte mía
de la más fervorosa admiración, fijaba en mí los ojos más perfectos,
que, según creo, se han abierto a la luz desde que hay luz en el mundo.
Un color se me iba y otro se me venía; a veces mi sangre toda corría
precipitadamente hacia mi semblante poniéndome encendido, y a veces se
recogía por entero en mi palpitante corazón, dejándome más pálido que un
difunto. Ignoro el número de fuentes que rompí aquella noche, pues las
manos me temblaban, y creo que serví de un modo lamentable, trocando el
orden de los platos, y dando sal cuando me pedían azúcar.

Yo decía para mí: «¿Qué es esto? ¿Tendré algo en la cara? ¿Por qué me
mirará tanto esa mujer?\ldots» Al salir fuera, iba a la cocina, me
miraba a toda prisa en un espejillo roto que allí tenía; mas no
encontraba en mi semblante nada que de notar fuese. Volví a la sala, y
otra vez Amaranta me clavaba los ojos. Por un instante llegué a
creer\ldots{} ¡pero quiá!, me reía yo mismo de tan loca presunción. Cómo
era posible que una dama tan hermosa y principal sintiera\ldots{} ¡Ay!,
recuerdo haber dicho, aunque al revés, lo que después escribió en un
célebre verso cierto poeta moderno. Pero todo debía de ser un sueño de
mi infantil soberbia. ¿Cómo podía la estrella del cielo mirar al gusano
de la tierra, sino para recrearse, comparando, en su propia magnitud y
belleza?

Pero debo añadir otra circunstancia, y es que cuando mi ama me reprendía
por las muchas torpezas que cometí en el servicio de la mesa, Amaranta
acompañaba sus miradas de una dulce sonrisa, que parecía implorar
indulgencia por mis faltas. Yo estaba perplejo, y un violento fluido que
parecía súbito acrecentamiento de vida corría por mis nervios,
produciéndome una actividad devoradora a la cual seguía un vago
aturdimiento.

Después de largo rato la conversación, anudándose de nuevo, fue general.
El marqués, viendo que no se le preguntaba nada, estaba en gran
desasosiego, y a los rostros de todos dirigía con inquietud sus ojos
buscando una víctima de su conversación; pero nadie parecía dispuesto a
escucharle, con lo cual, lleno de enojo, tomó la palabra para decir que
si continuaban apremiándole para que hablara, se vería en el caso de no
poner por segunda vez a prueba su discreción concurriendo a tertulias
donde no reinaba el más profundo respeto hacia los secretos de la
diplomacia.

---Pero si no le hemos dicho a usted una palabra---indicó Lesbia,
riendo.

Isidoro, conociendo que el marqués era enemigo de Godoy, dijo con mucha
sorna:

---No se puede negar que el Príncipe de la Paz, como hombre de gran
talento, burlará las intrigas de sus enemigos. Napoleón le apoya, y no
digo yo la coronita de los Algarbes, sino la de Portugal entero o quizás
otra mejor, recibirá de manos de su majestad imperial. Conozco a
Napoleón, le he tratado en París, y sé que gusta de los hombres
arrojados como Godoy. Verá usted, verá usted, señor marqués, todavía le
hemos de ver a usted llamado a los consejos del nuevo rey, y tal vez
representándole como plenipotenciario en alguna de las Cortes de Europa.

El marqués se limpió la boca con la servilleta, echose hacia atrás,
sopló con fuerza, desahogando la satisfacción que le producía el verse
interpelado de aquel modo, fijó la vista en un vaso, como buscando
misterioso punto de apoyo para una sutil meditación, y dijo con mucha
pausa:

---Mis enemigos, que son muchos, han hecho correr por toda Europa la
especie de que yo llevaba correspondencia secreta con el Príncipe de
Talleyrand, con el Príncipe Borghese, con el Príncipe Piombino, con el
gran duque de Aremberg, y con Luciano Bonaparte, en connivencia con
Godoy, para estipular las bases de un tratado en virtud del cual España
cedería las provincias catalanas a Francia a cambio de Portugal y el
reino de Nápoles\ldots{} pasando Milán a la reina de Etruria, y el reino
de Westfalia a un Infante de España. Yo sé que esto se ha dicho---añadió
alzando la voz y dando un fuerte puñetazo en la mesa.---¡Yo sé que esto
se ha dicho; ha llegado a mis oídos, sí, señor! Los calumniadores lo
hicieron creer a los soberanos de Austria y Prusia; se me interpeló
sobre el caso, Rusia no titubeó en hacerse eco de la calumnia, y fue
preciso que yo empleara todo mi valimiento y tacto para disipar las
densas nubes que se habían acumulado en el horizonte de mi reputación.

Al decir esto el marqués empleaba el mismo tono que habría usado ante un
Consejo de los principales políticos de Europa. Después de sonarse con
estrépito, prosiguió así:

---Afortunadamente soy bien conocido, y al fin\ldots{} tengo la
satisfacción de haber sido objeto de las más satisfactorias frases por
parte de los soberanos citados. ¡Ah!\ldots{} ya sé yo el objeto que guió
a los calumniadores y el sitio de donde partió la calumnia. En casa de
Godoy se inventó esa trama abominable con objeto de ver si, autorizada
con mi nombre, podía esa combinación correr con alguna fortuna por
Europa. Pero tan inicuos planes, quedaron sin éxito, como era de
suponer, y la Europa entera convencida de que el Príncipe de la Paz y yo
no podemos obrar de concierto en negocio alguno de interés general para
las grandes potencias.

---¿De modo---dijo Isidoro,---que usted no es, como dicen, amigo secreto
de Godoy?

El diplomático frunció el ceño, sonrió con desdén, llevó un polvo a la
nariz, y continuó así:

---¿Qué incongruentes especies no inventará la calumnia? ¿Qué torpes
ardides no imaginarán la astucia y la doblez contra la prudencia y la
rectitud? Mil veces me han hecho esos cargos, y mil veces los he
rebatido. Pero es fuerza que repita ahora lo que en otras ocasiones he
dicho. Había hecho propósito solemne de no ocuparme más de este asunto;
pero la terquedad de mis amigos, y la obcecación del público me obligan
a ello. Hablaré claro: si en el calor de mi defensa hago revelaciones
que puedan sonar mal en ciertos oídos cúlpese a los que me han
provocado, no a mí, que todo debo posponerlo al brillo de mi inmaculada
reputación.

Lesbia, Isidoro y mi ama hacían esfuerzos para contener la risa, al ver
el énfasis con que nuestro hombre defendía, contra imaginarias
acusaciones una personalidad de que nadie se ocupaba sino él. Amaranta
parecía meditabunda, mas sus reflexiones no le impedían fijar alguna vez
en mí sus incomparables ojos.

---En el año de 1792---prosiguió el viejo,---cayó del ministerio el
conde de Floridablanca, que se había propuesto poner coto a los estragos
de la revolución francesa. ¡Ah! El vulgo no conoció la mano oculta que
había arrojado de la secretaría del Estado a aquel hombre insigne,
envejecido en servicio del Rey. ¿Pero cómo podía ocultarse a los hombres
perspicaces la máquina interior de aquel cambio de ministerio? Un joven
de 25 años a quien los Reyes miraban con particular afecto y que tenía
frecuente entrada en palacio, y hasta participación en los consejos,
influyó en el cambio de ministerio, y en la elevación del señor conde de
Aranda. ¿Tuve yo participación en aquel suceso? No, mil veces no;
hallábame a la sazón agregado a la embajada española, cerca del
emperador Leopoldo, y no pude de ningún modo influir para que
desempeñara el ministerio mi amigo el conde de Aranda. Pero ¡ay!, éste
duró poco en el poder, porque nuevas maquinaciones le derribaron, y en
noviembre del mismo año, España y el mundo todo vieron con sorpresa que
era elevado a la primera dignidad política aquel mismo joven de 25 años,
ya colmado de honores inmerecidos, tales como el ducado de la Alcudia y
la grandeza de España de primera clase, la gran cruz de Carlos III, la
cruz de Santiago, los cargos de ayudante general del cuerpo de guardias,
mariscal de campo de los reales ejércitos, gentil-hombre de cámara de Su
Majestad con ejercicio, sargento mayor del real cuerpo de guardias de
Corps, consejero de Estado, superintendente general de correos y
caminos, etc., etc. Empuñó Godoy las riendas del Estado en tiempos muy
críticos: todos los hombres de previsión, comprendíamos la proximidad de
grandes males, e hicimos lo posible por conjurarlos. El torpe duque de
la Alcudia declaró la guerra a Francia, contra la opinión de Aranda, y
de todos cuantos teníamos alguna experiencia en los negocios. ¿Se nos
hizo caso? No.~¿Se oyeron nuestros consejos? No.~Pues veamos ahora lo
que ocurría después de hecha la paz con Francia.

El Rey continuaba acumulando en la persona de su favorito toda clase de
distinciones y honores, y por fin le enlazó con una princesa de la
familia real. Tanto favor dispensado a un hombre nulo y que en los más
indignos hechos buscaba ocasión de medro, produjo la animadversión y el
descontento de todos los españoles. La caída de un favorito, que había
desconcertado el Erario público, y desmoralizado la justicia vendiendo
los destinos, era segura. Y aquí debo decir, aunque por un momento falte
a las leyes de mi sistemática reserva, que yo nada influí para que
entraran en los ministerios de Hacienda y Gracia y Justicia Saavedra y
Jovellanos. Ruego a ustedes que no revelen este secreto, que hoy por
primera vez sale de mis labios.

---Seremos tan callados como guardacantones, señor marqués---dijo
Isidoro.

---Pero la cosa no tenía remedio---continuó el diplomático dirigiendo
sus ojos a todos los lados de la sala, como si le oyera gran número de
personas.---Jovellanos y Saavedra no podían concertarse en el gobierno
con quien ha sido siempre la misma torpeza y la corrupción en persona.
La república francesa trabajaba en contra del favorito; Jovellanos y
Saavedra se empeñaron en desprenderse de tan peligroso compañero, y al
fin el rey, cediendo a tantas sugestiones, y a la voz popular, dio a
Godoy su retiro en marzo de 1798. Yo declaro aquí de una vez para
siempre que no tuve participación en su caída, como han dado en suponer.
Y ésta sería ocasión de decir algo que sé, y que siempre he callado;
pero\ldots{} no, no fío bastante en la prudencia de los que me escuchan,
y prefiero guardar silencio sobre un punto delicado que nadie conoce.
Conste tan sólo que no contribuí a la caída de Godoy en 1798.

---Pero la desgracia del señor don Manuel duró poco---dijo
Isidoro,---porque el ministerio Jovellanos-Saavedra fue de poca
duración, y el de Caballero y Urquijo, que le sucedió, tampoco tuvo
larga vida.

---Efectivamente, a eso iba---continuó el marqués.---Los Reyes no podían
pasarse sin su amigo. Ocupó éste nuevamente la secretaría de Estado, y
queriendo acreditarse de guerrero, ideó la famosa expedición contra
Portugal, para obligar a este pequeño reino a romper sus relaciones con
Inglaterra. Ya desde entonces nuestro ministro no pensaba más que en
secundar los planes de Bonaparte del modo menos ventajoso para España.
Él mismo mandó aquel ejército, que se puso en planta a costa de grandes
sacrificios; y cuando los pobres portugueses abandonaron a Olivenza sin
que pudiera entablarse una lucha formal, el favorito celebró sus soñadas
victorias con un festejo teatral que dio a aquella guerra el nombre de
\emph{Guerra de las Naranjas}. Ustedes saben que los Reyes habían
acudido a la frontera. El favorito mandó construir unas angarillas que
adornó con flores y ramajes, y sobre esta máquina hizo poner a la reina,
que fue tan chabacanamente llevada en procesión ante las tropas, para
recibir de manos del generalísimo un ramo de naranjas, cogido en Elvas
por nuestros soldados. No añadiré una palabra más, ni recordaré los
punzantes chistes que circularon en aquella ocasión de boca en boca. Que
cada cual se entienda con su conciencia, y que todos tengan bastante
energía para defender sus propios actos, como defiendo yo los míos en
este momento. Ahora paso a otra cuestión.

Y aunque necesite repetirlo mil veces, diré también que no tuve parte
alguna en las negociaciones del tratado de San Ildefonso, ni en la
alianza de nuestra marina con la francesa, origen del desastre de
Trafalgar. Pero sobre ese tratado sé cosas curiosísimas que me confió el
general Duroc y que no puedo revelar a ustedes por más empeño que
muestren en conocerlas. No\ldots{} no me pidan ustedes que revele lo que
sé; no pongan a prueba mi discreción; hay secretos que no pueden
confiarse en el seno de la amistad más íntima. Yo debo callar y callaré.
Si los dijese, cuán pronto confundiría al Príncipe de la Paz y a los que
me suponen cómplice de sus infames tratos con Bonaparte. Mi único afán
ha consistido en destruir sus combinaciones, y aquí en confianza puedo
decir que repetidas veces lo he conseguido. Por eso se empeña en
desacreditarme a los ojos de Europa, en malquistarme con los hombres de
Estado, que han depositado en mí su confianza; por eso suena mi nombre
unido a todas las combinaciones que fragua Izquierdo en París. Pero
¡ah!, gracias a mi destreza podré anonadar a los calumniadores, salvando
mi buen nombre. Ojalá pudiera asimismo salvar a nuestros Reyes y a
nuestro país del descrédito a que los conduce ciegamente un hombre
abominable, que se ha elevado por las causas que todos sabemos y sigue
dirigiendo la nave del Estado valido de su torpe arrogancia e insolente
travesura.

Dijo, y llevándose a la nariz con diplomático aplomo el polvo de rapé se
sonó con más estruendo que el de una batería, miró a todos por encima
del pañuelo, y luego pronunció vagas frases que anunciaban la agitación
de su grande espíritu. Oyéndole y viéndole, parecía que sobre el mantel
de la mesa que yo había servido iban a resolverse las más arduas
cuestiones europeas, repartiendo pueblos y arreglando naciones como en
el tapete de Campo-Formio, de Presburgo o de Luneville.

---Estamos ya convencidos, señor marqués---dijo Lesbia,---de que usted
no ha tenido ni tiene parte alguna en los desastres ocasionados por el
Príncipe de la Paz; pero no nos ha dicho cuáles son los cataclismos que
nos amenazan.

---Ni una palabra más, no diré ni una palabra más---dijo el marqués
alzando la voz.---Cesen, pues, las preguntas. Todo es inútil, señoras
mías. Soy inflexible e implacable: todos los esfuerzos, todas las
astucias de la curiosidad no conseguirán arrancarme una revelación. He
suplicado a ustedes que no me preguntasen nada, y ahora, no ruego, sino
mando que me dejen en paz, renunciando a corromper y sobornar mi
experimentada prudencia con los halagos de la amistad.

Oyendo al diplomático, yo recordaba a cierto mentiroso que conocí en
Cádiz, llamado don José María Malespina. Ambos eran portentos de
vanidad; pero el de Cádiz mentía desvergonzadamente y sin atadero,
mientras que el de Madrid, sin alterar nunca los sucesos reales, se
suponía hombre de importancia, y su prurito consistía en defenderse de
ataques imaginarios y en negarse a revelar secretos que no sabía. Esto
prueba la inmensa variedad que el Creador ha puesto en la fauna moral,
así como en la física.

Isidoro y Lesbia, retirándose de la mesa, habían vuelto a formar la tela
de araña de sus comunicaciones amorosas. Mi ama había variado en sus
disposiciones favorables con el marqués. En vano le prometió franquearse
con ella, revelándole lo que ningún ser humano había oído hasta entonces
de sus labios; pero sin duda a la González no debió de halagar mucho la
promesa de conocer los planes de todas las potencias europeas, porque no
tuvo para su solícito cortejante palabra ni frase alguna que no fuesen
el mismo acíbar.

Amaranta, cuya reconcentración mental se desvanecía poco a poco, clavó
en mí sus ojos de una manera que parecía indicar vivo deseo de entablar
conversación conmigo. En efecto, contra todas las prescripciones del
decoro, en cierta ocasión en que yo recogía los platos vacíos que tenía
delante, se sonrió de un modo celestial, atravesándome el corazón con
estas palabras:

---¿Estás contento con tu ama?

No puedo asegurarlo terminantemente; pero creo que sin mirarla,
contesté:---Sí, señora.

---¿Y no desearías cambiar de ama? ¿No deseas encontrar colocación en
otra parte?

Tampoco aseguro que sea cierto, pero me parece que respondí:---Según con
quien fuera.

---Tú pareces un chico de disposición---añadió con una sonrisa que
parecía abrir el cielo ante mis ojos.

A esto sí estoy seguro de no haber contestado una palabra. Después de
una breve pausa, en que mi corazón parecía querer echárseme fuera del
pecho, tuve un arranque de osadía, que hoy mismo me causa asombro, y
dije:

---¿Es que quiere usía tomarme a su servicio?

Al oírme, Amaranta prorrumpió en graciosa carcajada, y yo me quedé
perplejo, creyendo haber dicho alguna inconveniencia. Al punto salí de
la sala con mi carga de platos: en la cocina procuré calmar mi
turbación, tratando de explicarme los sentimientos de Amaranta respecto
a mí, y después de mil dudas, dije:

---Mañana mismo le contaré todo a Inés, y veremos lo que ella piensa.

\hypertarget{viii}{%
\chapter{VIII}\label{viii}}

Cuando regresé a la sala, la escena continuaba la misma, pero la llegada
de un nuevo personaje iba a variarla por completo. Oímos ruido de
alegres voces y como preludios de guitarra en el portal, y después entró
un joven a quien diferentes veces había yo visto en el teatro.
Acompañábanle otros; pero se despidieron en la puerta, y él subió solo,
mas haciendo tanto ruido, que no parecía sino que un ejército se nos
metía en la casa. Me acuerdo bien de que aquel joven vestía el traje
popular; esto es, un rico marsellés, gorra peluda de forma semejante a
la de los sombreros tripicos, pero mucho más pequeña, y capa de grana
con forros de felpa manchada. Al verle con esta facha, no crean ustedes
que era algún manolo de Lavapiés o chispero de Maravillas, pues los
arreos con que le he presentado cubrían la persona de uno de los
principales caballeros de la corte; sólo que éste, como otros muchos de
su época, gustaba de buscar pasatiempo entre la gente de baja estofa, y
concurría a los salones de \emph{Polonia la Aguardentera}, \emph{Juliana
la Naranjera}, y otras célebres majas de que se hablaba mucho entonces.
En sus nocturnas correrías usaba siempre aquel traje, que en honor de la
verdad, le caía a las mil maravillas.

Pertenecía aquel joven a la guardia real, y sus conocimientos no
traspasaban más allá de la ciencia heráldica, en que era muy experto,
del arte del toreo y la equitación. Su constante oficio era la
galantería arriba y abajo, en los estrados y en los bailes de candil.
Parecían escritos expresamente para él los famosos versos:

\small
\newlength\mlenh
\settowidth\mlenh{¿Ves, Arnesto, aquel majo en siete varas}
\begin{center}
\parbox{\mlenh}{¿Ves, Arnesto, aquel majo en siete varas       \\
                de pardomonte envuelto...}                     \\
\end{center}
\normalsize

---¡Oh, don Juan!---exclamó Amaranta al verle entrar.

---Bienvenido sea el señor de Mañara.

Animose la reunión como por encanto con la entrada de aquel joven, cuyo
carácter jovial y bullanguero se manifestó desde el primer momento.
Advertí que el rostro de Amaranta adquiría de súbito extraordinaria
viveza y malicia.

---Señor de Mañara---dijo con gran desenfado,---llega usted a tiempo.
Lesbia le echaba a usted de menos.

Lesbia miró a su amiga de un modo terrible, mientras Isidoro parecía
dominado por violenta cólera.

---Aquí, don Juan, siéntese usted a mi lado---indicó mi ama con alegría,
señalando a Mañara la silla que tenía a la izquierda.

---No creí encontrar a usted aquí, señora duquesa---dijo el petimetre
dirigiéndose a Lesbia.---He venido, sin embargo, impulsado por la voz de
mi corazón; ya veo que el corazón no se equivoca siempre.

Lesbia estaba bastante turbada, mas no era mujer a quien arredraban las
situaciones críticas; así es que entre ella y Mañara hubo un verdadero
tiroteo de dichos agudos, risas y epigramas. Máiquez estaba cada vez más
intranquilo.

---Ésta es noche de suerte para mí---dijo don Juan sacando un bolsillo
de seda.---He estado en casa de la Primorosa, y allí he ganado cerca de
dos mil reales.

Diciendo esto, vació el oro sobre la mesa.

---¿Había allí mucha gente?---preguntó Amaranta.

---Mucha; mas la marquesita no pudo ir porque estaba con dolor de
muelas. ¡Ah!, nos hemos divertido.

---Para usted---dijo Amaranta con verdadero ensañamiento en su
malicia,---no hay diversión allí donde no está Lesbia.

Ésta volvió a dirigir a su amiga colérica mirada.

---Por eso he venido.

---¿Quiere usted seguir probando fortuna?---dijo mi ama.---La baraja,
Gabriel; trae la baraja.

Hice lo que se me mandaba, y los oros, las espadas, los bastos y las
copas se entremezclaron bajo los dedos del petimetre, que barajaba con
toda la rapidez que da la experiencia.

---Sea usted banquero.

---Bien: ahí va.

Cayeron las primeras cartas: todos los personajes sacaron su dinero;
fijáronse ansiosas miradas en los terribles signos, y comenzó el juego.

Por un momento no se oyeron más que estas breves y elocuentes frases:
«¡Tres duros al caballo!\ldots{} Yo no abandono a mi siete de
espadas\ldots{} Bien por el rey\ldots{} Gané\ldots, perdí\ldots{} Diez a
mí\ldots{} Maldita sota!»

---Mala suerte tiene usted esta noche, Máiquez---dijo Mañara, recogiendo
el dinero del actor, que ni una vez apuntaba sin perder cuanto ponía.

---¡Y yo, qué buena!---dijo mi ama recogiendo sus monedas, que ascendían
ya a una respetable cantidad.

---¡Oh, Pepa; para usted es toda la suerte!---exclamó el
banquero.---Pero dice el refrán: «Afortunado en el juego, desgraciado en
amores.»

---En cambio usted---dijo Amaranta,---puede decir que es afortunado en
ambos juegos. ¿Verdad, Lesbia?

Y luego, dirigiéndose a Isidoro, que perdía mucho, añadió:

---Para usted, pobre Máiquez, sí que no se ha hecho aquel refrán; porque
usted es desgraciado en todo. ¿Verdad, Lesbia?

El rostro de ésta se encendió súbitamente. Me pareció que la vi
dispuesta a contestar con violencia a su amiga; pero se contuvo y la
tempestad quedó conjurada por algún tiempo. El marqués perdía siempre,
pero no paró de jugar mientras tuvo una peseta en su bolsillo. No así
Máiquez, que una vez desvalijado, recibió un préstamo del banquero, y
así siguió el juego hasta más de la una, hora en que comenzaron a hablar
de retirarse.

---Debo a usted treinta y siete duros---dijo Máiquez.

---Y por fin---preguntó el petimetre,---¿cuál es la función escogida
para representarse en casa de la señora marquesa?

---Ya está acordado que sea \emph{Otello}.

---¡Oh!, me parece bien, amigo Isidoro. Me entusiasma usted en el papel
de celoso ---dijo Mañara.

---¿Querría usted hacer el de Loredano?---preguntó el actor.

---No; es papel muy desairado. Además, no sirvo para el teatro.

---Yo le enseñaré a usted.

---Gracias. ¿Ya ha enseñado usted a Lesbia su papel?

---Lo sabe perfectamente.

---Cuánto deseo que llegue esa noche---dijo Amaranta.---Pero diga usted,
Isidoro, si le ocurriera a usted un lance como el de \emph{Otello}, si
se viera engañado por la mujer que ama, ¿sentiría usted aquel terrible
furor, sería capaz de matar a su Edelmira?

Esta flecha iba dirigida a Lesbia.

---¡Quiá!---exclamó Mañara.---Eso no pasa nunca sino en el teatro.

---No mataría a Edelmira; pero sí a Loredano---repuso Máiquez con
firmeza, clavando su enérgica mirada en el petimetre.

Hubo un momento de silencio, durante el cual pude advertir perfectamente
las señales de la más reconcentrada rabia en el rostro de Lesbia.

---Pepa, no me has obsequiado esta noche---dijo Mañara.---Verdad es que
he cenado; pero son las dos, hija mía.

Serví de beber al joven, y habiéndome retirado, oí desde fuera el
siguiente diálogo. Mañara, alzando una copa llena hasta los bordes,
dijo:

---Señores: brindo por nuestro querido Príncipe de Asturias: brindo
porque la santa causa que representa tenga dentro de pocos días el éxito
más completo: brindo por la caída del favorito y el destronamiento de
los Reyes Padres.

---Muy bien---exclamó Lesbia aplaudiendo.

---Creo que estoy entre amigos---continuó el joven.---Creo que un fiel
súbdito del nuevo Rey puede manifestar aquí sin recelo, alegría y
esperanza.

---¡Qué horror! Está usted loco. Prudencia, joven---dijo el diplomático
escandalizado.---¿Cómo se atreve usted a revelar\ldots?

---Cuidado---dijo Lesbia con mucha viveza,---cuidado señor Mañara, está
delante una confidenta de Su Majestad la Reina.

---¿Quién?

---Amaranta.

---Tú también lo eres, y según dicen posees los secretos más graves.

---No tanto como tú, hija mía---dijo Lesbia sintiendo reponerse su
osadía;---tú, que, según se asegura, eres hoy depositaria de todas las
confianzas de nuestra amada soberana. Esto es una gran honra para ti.

---Seguramente---repuso Amaranta, dominando su cólera.---Sigo al lado de
mi bienhechora. La ingratitud es vicio muy feo, y no he querido imitar
el ejemplo de las que insultan a quien les ha favorecido. ¡Ah!, es muy
cómodo hablar de las faltas ajenas para que no se fije la vista en las
propias.

Lesbia, después de un momento de vacilación, iba a contestar. El diálogo
tomaba alguna gravedad, y de seguro se habrían oído cosas bastante
duras, si el diplomático, interviniendo con su tacto de costumbre, no
hubiera dicho:

---Señoras, por Dios\ldots{} ¿qué es esto? ¿No son ustedes íntimas
amigas? ¿Una diferencia de opinión puede turbar el cielo purísimo de la
amistad? Dense las manos, y bebamos todos el último vaso a la salud de
Lesbia y Amaranta enlazadas en dulce y amorosa fraternidad.

---Estoy conforme; ésta es mi mano---dijo Amaranta alargando la suya con
gravedad.

---Ya hablaremos de esto---añadió Lesbia estrechando con desabrimiento
las manos de la otra dama.---Por ahora seremos amigas.

---Bien: ya hablaremos de esto.

En aquel momento entré yo y la expresión del semblante de una y otra no
me pareció indicar predisposiciones a la concordia. Con aquel
desagradable incidente, que por fortuna no tomó proporciones, tuvo fin
la tertulia, y la aparente reconciliación fue señal de partida.
Levantáronse todos, y mientras el diplomático y Mañara se despedían de
mi ama, Amaranta se llegó a mí con disimulo, acercó su boca a mi oído, y
me dijo con una vocecita que parecía resonar dentro de mi cerebro:

---Tengo que hablarte.

Dejome aturdido; pero mi sorpresa subió de punto un poco después, cuando
acompañé a la comitiva por la calle, precediéndoles con un farol, según
costumbre, porque en aquel tiempo el alumbrado público, si en alguna
calle existía, era digno émulo de la oscuridad más profunda. Llegamos a
la calle de Cañizares, a una suntuosa casa, que era la misma en cuyo
sotabanco vivía Inés, aunque se subía por distinta escalera. En el patio
de aquella casa, que era la del marqués diplomático, por mejor dicho, de
su hermana, esperaban las literas que debían conducir a las dos damas a
sus respectivas mansiones. Antes de entrar en la litera, Amaranta me
llamó aparte, y díjome que al día siguiente fuese a buscarla a aquella
misma casa, preguntando por una tal Dolores, que luego supe era doncella
o confidenta suya, cuyo mandato me alegró mucho, porque en él vi el
fundamento de mi fortuna.

Volví a casa apresuradamente, y encontré a mi ama muy agitada, paseando
con precipitación en la estrecha sala, y departiendo consigo misma, como
si no tuviera el juicio muy sano.

---¿Observaste---me dijo---si Isidoro y Mañara disputaban por la calle?

---No reparé, señora---le respondí.---¿Pues qué motivo tienen esos dos
caballeros para enemistarse?

---¡Ah!, no sabes cuán alegre estoy, Gabriel; estoy satisfecha---me dijo
la González con extraviados ojos y tan febril inquietud, que me impuso
miedo.

---¿Por qué, señora?---pregunté.---Ya es hora de descansar, y usted
parece necesitar descanso.

---No, tonto, yo no duermo esta noche---dijo.---¿No sabes que yo no
puedo dormir? ¡Ah, cuánto gozo considerando su desesperación!

---No entiendo a usted.

---Tú no entiendes de esto, chiquillo; vete a acostar\ldots{} Pero no,
no, ven acá y escucha. ¿Verdad que parece castigo de Dios? El muy simple
no conoce la víbora que tiene entre sus brazos.

---Creo que se refiere usted a Isidoro.

---Justo. Ya sabes que está enamorado de Lesbia. Está loco, como nunca
lo ha estado. ¡Ah! Con todo su orgullo, ¡qué vilmente se arrastra a los
pies de esa mujer! Él, acostumbrado a dominar, es dominado ahora, y su
impetuoso amor servirá de diversión y chacota en el teatro y fuera de
él.

---Pero me parece que el señor Máiquez es correspondido.

---Lo fue; pero los favores de Lesbia pasan pronto. ¡Oh! Bien merecido
le está. Lesbia es la misma inconstancia.

---No lo hubiera creído en una persona tan simpática y tan linda.

---Con esa carita angelical, con su sonrisa inalterable y su aire de
ingenuidad, Lesbia es un monstruo de liviandad y coquetería.

---Tal vez ese señor Mañara\ldots{}

---Eso no tiene duda. Mañara es hoy el favorecido, y si habla con
Isidoro es para divertirse a su costa, jugando con el corazón de ese
desgraciado. Sí, el corazón de Isidoro está hoy como un ovillo de
algodón entre las patas de una gata traviesa. ¿Pero no es verdad que le
está bien merecido?\ldots{} ¡Oh, rabio de placer!

---Por eso la señora Amaranta no cesaba de decir aquellas
cosas\ldots---indiqué, deseando que mi ama esclareciera mis dudas sobre
muchos sucesos y palabras de aquella noche.

---¡Ah! Lesbia y Amaranta, aunque vienen juntas aquí, se aborrecen, se
detestan, y quisieran destruirse una a otra. Antes se llevaban muy bien;
mas de algún tiempo a esta parte, yo creo que algo ocurrido en palacio
es la causa de esta inquina que ha empezado hace poco y será una guerra
a muerte.

---Bien se conoce que no se llevan bien.

---En palacio, según me han dicho, arden pasiones encarnizadas
implacables. Amaranta es muy amiga de los Reyes Padres, mientras que
Lesbia parece que es de las damas que más intrigan en el bando de los
amigos del Príncipe de Asturias. Tan irritadas están hoy la una contra
la otra, que ya no saben disimular el odio que se profesan.

---¿Y es Amaranta mujer de tan mala condición como su amiga?---pregunté,
deseando inquirir noticias de la que ya consideraba como mi protectora.

---Todo lo contrario---repuso.---Amaranta es una gran señora, tan
discreta como hermosa, y de conducta intachable. Gusta de proteger a los
desvalidos: su sensible y tierno corazón es inagotable para los
menesterosos que necesitan de su ayuda; y como es poderosísima en la
corte, porque su valimiento casi excede al de los mismos Reyes, el que
tenga la dicha de caer en gracia, ya se puede considerar puesto en los
cuernos de la luna.

---Ya me lo parecía a mí---dije muy contento por tan lisonjeras
noticias.

---Espero que Amaranta---prosiguió mi ama con la misma calenturienta
agitación,---me ayudará en mi venganza.

---¿Contra quién?---pregunté alarmado.

---Creo que se ha aplazado la función de la marquesa---continuó sin
atender a mi pregunta.---Nadie quiere hacer el desairado papel de
Pésaro, y esto será ocasión de un lamentable retraso. ¿Querrás
desempeñarlo tú, Gabriel?

---¡Yo, señora!\ldots{} no sirvo para el caso.

---Quedose luego muy meditabunda, con el ceño fruncido y los ojos fijos
en el suelo, y por fin volvió a su primer tema.

---Estoy satisfecha---dijo con esa hilaridad dolorosa, que indica las
grandes crisis de la pasión.---Lesbia le es infiel, Lesbia le engaña,
Lesbia le pone en ridículo, Lesbia le castiga\ldots{} ¡Oh, Dios mío! Veo
que hay justicia en la tierra.

Después, serenándose un poco, me mandó retirar, y cuando me hallé fuera,
dejándola con su doncella, la sentí llorar con lágrimas francas y
abundantes, que debían templar la irritación de su espíritu y poner
calma en su excitado cerebro. A los consuelos y ruegos de su criada para
que se retirase a descansar, no respondía más que esto:

---¿Para qué me acuesto, si sé que no he dormir en toda la noche?

Retireme a mi cuarto, que era un estrecho dormitorio donde jamás
entraban ni en pleno día importunas luces. Me acosté bastante afligido
al considerar la triste pasión de mi ama; pero estos pensamientos se
enlazaron con otros relativos a mi propio estado, los cuales, lejos de
ser tristes, alborozaban mi alma; y acompañado por la imagen de Amaranta
que iluminaba mi mezquino asilo como un rayo de luna, me dormí
profundamente pensando en la fábula de Diana y Endimión, que conocía por
una de las estampas de la sala.

\hypertarget{ix}{%
\chapter{IX}\label{ix}}

Al despertar en la mañana siguiente, acudieron en tropel a mi
pensamiento todas las ideas y las imágenes que me habían agitado la
noche anterior. La inclinación hacia mi persona que suponía en Amaranta
me trastornaba el juicio, como verá el amigo lector, si le cuento los
disparates que dije y las locuras que imaginé en las reflexiones y
monólogos de aquella mañana.

---No veo la hora---decía para mí---de presentarme a esa señora. No me
queda duda de que le he caído en gracia, lo cual no es extraño, pues
algunas personas me han dicho que no tengo mal ver. Como dice doña
Juana, de hombres se hacen obispos, y quién sabe si a la vuelta de una
media docena de añitos, me encuentro hecho en dos palotadas duque, conde
o almirante, como otros que yo me sé y que deben lo que son a haber
caído en gracia a esta o la otra persona. Hablemos claro, Gabriel. ¿No
estás oyendo mentar todos los días a cierto personaje que antes era un
pobre pelambrón, y ahora es todo cuanto puede ser un hombre? ¿Y todo por
qué? Por la inclinación de una elevada señora. ¿Y quién dice que lo que
puede pasar a un hombre no le pueda suceder a otro? Verdad es que el tal
personaje es un gallardo mozo; pero yo bien sabido me tengo que no soy
saco de paja, pues muchas personas me han dicho que les gusto, y que no
puede negarse que tengo unos ojillos picarescos, capaces de trastornar a
todo el sexo femenino. Ánimo, señor Gabrielito. Mi ama ha dicho que
Amaranta es la mujer más poderosa de toda la corte, y quién sabe si será
de sangre real. ¡Oh, divina Amaranta! ¿Qué haré para merecerte? Por
supuesto, que si llego a verme desempeñando esos elevados cargos, juro
por Dios y mi salvación, que he de ser el hombre más formal que jamás
haya gobernado en el mundo: a buen seguro que nadie me acuse, como
acusan al otro, de haber hecho tantas picardías. Lo que es eso\ldots{}
yo tendré las cosas bien arregladitas, y en mi persona no gastaré sino
lo muy preciso. Lo primero que voy a disponer es que no haya pobres, que
España no vuelva a unirse con Francia, y que en todas las plazuelas de
España se fije el precio de los comestibles, para que los pobres compren
todo muy barato. Veremos si sé yo mandar o no sé\ldots{} ¡y que tengo un
geniecillo! Como no hagan lo que mando, nada, nada\ldots{} no me andaré
con chiquitas. Al que no obedezca, cortarle la cabeza y se acabó\ldots{}
así andarán todos derechos como un huso. Y lo dicho dicho. Nada con los
franceses. Napoleón que se entienda solo; nosotros haremos lo que nos dé
la gana, y que no me busquen el genio, porque yo tengo muy malas
moscas\ldots{} ¡Oh!, si esto sucediera, cómo se había de alegrar la
pobre Inés: entonces sí que no repetiría lo de la tortuga y del águila.
Se me figura que Inés es algo corta de alcances; sin embargo, es tan
buena que la amaré siempre\ldots; pero debo amar a Amaranta\ldots; pero
¿cómo puedo dejar de amar a Inés?\ldots{} Pero es preciso que adore
sobre todas las cosas a Amaranta\ldots; pero Inés es tan sencilla, tan
buena, tan\ldots; pero Amaranta me subyuga, me fascina, me vuelve
loco\ldots; pero Inés\ldots{} pero Amaranta. \dotfill

Esto decía yo, despeñado como corcel salvaje, por los derrumbaderos de
mi fantasía; y ya habrá observado el lector que, al suponerme amado por
una mujer poderosa, mis primeras ideas versaron sobre mi
engrandecimiento personal, y el ansia de adquirir honores y destinos. En
esto he reconocido después la sangre española. Siempre hemos sido los
mismos.

Levanteme, cogí el cesto para ir a la compra, y cuando recorría los
puestos de la plazuela regateando las patatas y las coles, consideré
cuán inconveniente y deshonroso era que se ocupase en tan bajos
menesteres un joven destinado a ser dentro de algún tiempo generalísimo
de los ejércitos de mar y tierra, gran almirante, mi nistro, y quién
sabe si rey de algún reinito chico que le caería por chiripa en los
repartos europeos.

Dejando aparte por ahora lo que se refiere a mi persona, voy a dar una
idea de la opinión pública en aquellos días, con motivo de los sucesos
políticos. En la plazuela advertí que se hablaba del asunto, y por las
calles las personas se paraban preguntándose noticias, y regalándose
mutuamente las mentiras de que cada cual era forjador o inocente
vehículo. Yo hablé del caso con varias personas conocidas, y voy a
copiar imparcialmente el parecer de algunas, pues siendo las más de
diversa condición y capacidad, el conjunto de sus observaciones puede
ofrecer exactamente una muestra del pensamiento público.

Un hortera de ultramarinos, que era nuestro abastecedor y hombre muy
aficionado a mover la sin hueso, me pareció más alegre que de ordinario
y en extremo jovial con sus parroquianos.

---¿Qué nuevas corren por ahí?---le pregunté.

---¡Oh!, grandes nuevas. Los franceses han entrado en España. Yo estoy
contentísimo.

Luego, bajando la voz, dijo con semblante risueño:

---¡Van a conquistar a Portugal! Es para volverse loco de alegría.

---Hombre, no lo entiendo.

---¡Ah! Gabrielillo: tú como eres un pobre chico, no entiendes estas
cosas. Ven acá, mentecato. Si conquistan a Portugal, ¿para qué ha de ser
sino para regalárselo a España?

---¿Y un reino se conquista y se regala como si fuera una libra de
nísperos, señor de Cuacos?

---Pues es claro. Napoleón es un hombre que me gusta. Quiere mucho a
España, y se desvive por hacernos felices.

---Vaya con el hombre. ¿Y nos quiere por nuestra linda cara o porque le
conviene, para sacarnos dinero, barcos, tropas, y cuanto le da la
gana?---dije yo, cada vez más resuelto a romper con Francia, cuando
fuese ministro.

---Nos quiere porque sí, y sobre todo ahora va a quitar de en medio al
señor Godoy, que ya nos tiene hasta el tragadero.

---¿Querrá usted decirme qué es lo que ha hecho ese caballero para que
todos le quieran tan mal?

---¡Bicoca!, ahí es nada lo del ojo. ¿No sabes que es un embustero,
atrevido, lascivo, tramposo y enredador? Ya sabemos todos a qué debe su
fortuna, y la verdad es que la culpa no la tiene él, sino quien lo
consiente. Ya sabes tú que vende los destinos, ¡y de qué manera! Los que
tienen mujer guapa o hija doncella son los que consiguen de Su Alteza
cuanto solicitan. Pues ahora trata de que se vayan a América los
príncipes para quedarse él de rey de España\ldots{} Pero no echó muy
bien las cuentas, y a lo mejor se presenta Napoleón para desbaratar sus
planes\ldots{} Sabe Dios lo que ocurrirá dentro de algunos días: yo creo
que Napoleón, como amigo y admirador que es de nuestro gran Príncipe de
Asturias, nos lo va a poner en el trono, sí señor\ldots{} y el Rey
Carlos, con la buena pieza de su mujer, se irá a donde mejor le
convenga.

No hablemos más del asunto. Entré luego en la tienda de doña Ambrosia, a
comprar un poco de seda que me había encargado la doncella, y vi tras el
mostrador a la grave tendera, acariciando su gato, sin dejar por eso de
atender a la conversación entablada entre don Anatolio, el papelista de
la acera de enfrente, y el abate don Lino Paniagua, que estaba
escogiendo unas cintas verdes y azules.

---No le quede a usted duda, señora doña Ambrosia---decía el
papelista;---de esta vez nos veremos libres del \emph{choricero}.

---No puede ser menos---contestó la tendera---sino que alguna buena alma
ha ido a Francia y le ha contado a ese bendito emperador todas las
picardías que aquí hace Godoy, por lo cual éste ha mandado un ejército
entero para quitarle de en medio.

---Pues con perdón de ustedes---dijo el abate Paniagua alzando la
vista,---yo que frecuento la sociedad de etiqueta, puedo asegurar que
las intenciones de Napoleón son muy distintas de lo que se cree
vulgarmente. Napoleón no manda sus tropas contra Godoy, sino para Godoy;
porque han de saber ustedes que en un tratado secreto (y esto lo digo
con reserva) se ha convenido echar de Portugal a los Braganzas, y
repartirse aquel reino entre tres personas, de las cuales una será el
Príncipe de la Paz.

---Eso se dijo hace tiempo---observó con desdén don Anatolio;---pero
ahora no se trata de tal reparto. La verdad pura y neta es que Napoleón
viene a quitar el Portugal a los ingleses, lo cual está muy requetebién
hecho; sí señor.

---Pues a mí me han dicho---añadió doña Ambrosia,---que lo que quiere
Godoy es mandar al Príncipe a América con sus hermanos, para quedarse él
solito de rey de España. Eso no lo habíamos de consentir. ¿Verdá usté
don Anatolio? Miren qué ideas de hombre. Pero ¿qué se puede esperar de
quien está casado con dos mujeres?

---Y creo que las dos se sientan con él a la mesa, una a la derecha y
otra a la izquierda---dijo don Anatolio.

---Por Dios, hablemos bajo---indicó con timidez don Lino
Paniagua.---Esas cosas no deben decirse.

---Nadie nos oye, y sobre todo\ldots{} Si van a poner a la sombra a
cuantos hablan de estas cosas, pronto se quedará Madrid sin gente.

---Verdad---dijo Ambrosia bajando la voz.---Mi difunto esposo, que santa
gloria haya, y era el hombre de más verdad que ha comido nabos en el
mundo, aseguraba\ldots{} (y crean ustedes que lo sabía de buena tinta)
que cuando el \emph{choricero} quiso que el consejo de Estado habilitase
a la Reina para ser regenta\ldots{} pues, no sé si me explico\ldots{}
era porque tenían el proyecto de despachar para el otro barrio a mi
señor don Carlos; de modo que\ldots{}

---¡Qué abominaciones se dicen hoy!---exclamó el abate.

---Como que es la pura verdad---dijo don Anatolio.

---Yo también lo supe por persona que estaba en el ajo.

---Pero esto no se dice, señores, esto se calla---respondió
Paniagua.---Yo, francamente, no gusto de oír tales cosas. Me da miedo; y
si llega a oídos del señor Príncipe de la Paz, figúrense ustedes qué
disgusto.

---Como no nos ha dado prebendas, ni le pedimos congruas\ldots{}

---En fin, despácheme usted, señora doña Ambrosia, que tengo prisa. Esas
cintas verdes son de etiqueta; pero lo que es las azules, no me atrevo a
presentárselas a la señora condesa de Castro-Limón.

Despacharon al abate, y luego a mí, con más presteza de la que habría
querido, pues de buen grado me hubiera detenido más para oír los
comentarios políticos que tanto me agradaban. Ya iba derecho a casa,
cuando acerté a tropezar con el reverendo padre fray José Salmón, de la
orden de la Merced, el cual era un sujeto excelente que visitaba a doña
Dominguita (la abuela de mi ama), con tanta frecuencia como exigían el
arte de Hipócrates y el piadoso anhelo de bien morir; pues para
administrar lo primero y preparar el ánima a lo segundo era un águila el
buen mercenario Salmón, a quien sólo faltaba una o en su apellido para
llamarse como el portento de la sabiduría. Detúvose en medio de la
calle, e interpelándome con su acostumbrada afabilidad y cortesía, dijo:

---¿Y esa incomparable doña Dominga, cómo está? ¿Qué tal efecto te ha
hecho el cocimiento de cáscaras de frambuesa, o sea, \emph{tetragonia
ficoide}, que llama Dioscórides?

---¡Magnífico efecto!---respondí, aunque estaba en completa ignorancia
del asunto.

---Ya le llevaré esta tarde unas pildoritas\ldots---prosiguió---con las
cuales o yo no soy el padre Salmón de la orden de la Merced, o esa
señora ha de recobrar la agilidad de sus piernas\ldots{} Pero chico: qué
buenas peras llevas ahí---añadió metiendo la mano en el cesto y sacando
la fruta indicada.---Tú tienes buena mano derecha para comprar peras.

Y acto continuo se la guardó, después de olerla, en la manga del luengo
hábito, sin pedir permiso para ello, pues aunque siguió hablando, fue
para añadir lo siguiente:

---Dile que iré esta tarde por allá a contarle las grandes novedades que
ocurren en España.

---Usted que sabe tanto---dije impulsado por mi curiosidad,---¿podrá
explicarme a qué vienen esos ejércitos franceses?

---Si tú tuvieras la mitad del talento que yo tengo---repuso,---te
pondrías al tanto de las diversas razones que me hacen estar alegre
considerando la llegada de esos señores. ¿Por ventura no sabes que
Napoleón fue quien restableció el culto en Francia, después de los
horrores y herejías de la revolución? ¿No sabes también que entre
nosotros no falta algún endiablado personaje en cuya mente bullen
atrevidos proyectos contra la Iglesia? Pues sabiendo esto, ¿a quién no
se alcanza que el objeto de la entrada de esos ejércitos no es ni puede
ser otro que dar merecido castigo al insolente pecador, al polígamo
desvergonzado, al loco enemigo de los derechos eclesiásticos?

---Luego ese señor Godoy ¿no sólo es un bribón, y un acá y un allá, sino
que también es enemigo de la religión y los religiosos?---pregunté
asombrado de ver cómo aumentaba el capítulo de culpas del favorito.

---Sin duda---dijo el fraile.---Y si no, ¿qué nombre tiene el proyecto
de reformar las órdenes mendicantes, quitándoles la vida conventual y
obligando a esos buenos religiosos a servir en los hospitales generales?
También agita en su diabólica mente el proyecto de sacar de las granjas
que nos pertenecen lo necesario para fundar unas a modo de escuelas de
agricultura; que sabe Dios lo que serán las tales escuelitas. ¡Oh! Y si
fuera cierto lo que se dice---añadió alargando la mano para hacer
segunda exploración en mi cesto;---si fuera cierto lo que se dice
respecto a la enajenación de parte de los bienes que ellos llaman de
manos muertas\ldots{} Pero no nos ocupemos de esto, que más bien causa
risa que indignación, y fijemos la vista en el astro de las Galias que
cual divino campeón viene a libertarnos de la tiranía de un necio
valido, poniendo en el trono al augusto príncipe en cuya sabiduría y
prudencia fiamos.

Al concluir esto había trasportado desde mi cesto a las mangas de su
hábito otra pera y hasta media docena de ciruelas, dando después rienda
suelta a los encomios de mi destreza en el comprar. Yo me apresuré a
separarme de un interlocutor que me salía tan caro, y le di los buenos
días, renunciando a las lecciones de su sabiduría.

No había sacado en limpio gran cosa, ni disipado mis dudas, sobre lo que
hoy llamaríamos la situación política, y lo único que vi con alguna
claridad fue la general animadversión de que era objeto el Príncipe de
la Paz, a quien se acusaba de corrompido, dilapidador, inmoral,
traficante de destinos, polígamo, enemigo de la Iglesia, y, por
añadidura de querer sentarse en el trono de nuestros Reyes, lo cual me
parecía el colmo de la atrocidad. También vi de un modo clarísimo que
todas las clases sociales amaban al Príncipe de Asturias, siendo de
notar, que cuantos anhelaban su próxima elevación al trono, fiaban tal
empresa a la amistad de Bonaparte, cuyos ejércitos estaban entrando ya
en España para dirigirse a Portugal.

Volvía a la plazuela para reponer las bajas hechas en el cesto por su
paternidad, y allí encontré\ldots{} ¿no adivinan ustedes a quién? El
infeliz, acompañado de su hija Joaquinita, a quien natura había hecho
\emph{poetisa entre dos platos}, se ocupaba en comprar al fiado no sé
que piltrafas y miserables restos, que eran su ordinario alimento. Él
pedía las cosas, la jorobadilla se las regateaba, y entre los dos
cargaban la ración, cuyo peso no hubiera fatigado a un niño de cinco
años. La miseria había pintado sus más feos rasgos en el semblante de la
hija y del padre, el cual era tan flaco y amarillo, que se dudaba cómo
podía existir y moverse cuerpo tan endeble, no siendo galvanizado por el
misterioso fluido del numen poético. ¿Necesito nombrarle? Era Comella.

---¡Señor don Luciano, usted por aquí!---dije saludándole con mucho
afecto, porque aquel hombre me inspiraba la más viva compasión.

---¡Ah, Gabriel!---contestó,---¿y Pepita y doña Dominga? Tiempo hace que
no las veo. Pero ya saben que aunque no las visito, porque el trabajo me
lo impide, les estoy muy agradecido.

---Hoy espero ir por allá a llevarles a ustedes algún recadito---dije
respondiendo verbalmente a las tristes suplicantes miradas de la hija
del poeta, cuyos ojos me hablaban el lenguaje del hambre.

---Es preciso que vayas por casa---continuó el poeta tomándome el brazo,
e indicando en su gravedad que lo que iba a confiarme era
importantísimo.---Como me has dicho que presenciaste lo de Trafalgar,
quiero consultarte sobre ciertos detalles\ldots{} pues.

---Ya. Escribe usted la historia de aquella batalla.

---No: historia no, un dramita que va a dejar bizcos a los señores.
Verás que pieza. Se titula \emph{El tercer Gran Federico y combate del
21}.

---Buen título---respondí;---pero no entiendo qué es eso del
\emph{tercer Federico}.

---¡Qué tonto eres! El \emph{tercer Gran Federico} es Gravina, y como ya
hubo en Prusia un Gran Federico que era segundo, ¿no comprendes que es
ingenioso, y llamativo y tónico poner a nuestro almirante en la lista de
los Grandes Federicos que ha habido en el mundo?

---Ciertamente. Es una idea que sólo a usted se le hubiera ocurrido.

---Ya Joaquina ha escrito las primeras escenas, que son preciosísimas.
En primer término aparece la cubierta del \emph{Santísima Trinidad}, a
la derecha el navío de Nelson, y a lo lejos Cádiz con sus castillos y
torreones. Debo advertirte que figuro a Nelson enamorado de la hija de
Gravina, el cual se niega a dársela en matrimonio. La escena empieza con
una sublevación de los marineros españoles que piden pan, porque en todo
el barco no hay una miga. El almirante se enfurece y les dice que son
unos cobardes, porque no tienen alma para resistir tres días sin comer,
y les da el ejemplo de la más plausible sobriedad mandándose servir un
pedacito de maroma asada. Nelson se presenta a decir que todo se acabará
al fin si le dan la niña para llevársela a Inglaterra: la muchacha sale
de la cámara bordando un pañuelo, y\ldots{}

No dijo más, porque la violenta risa en que prorrumpí, sin poderme
contener, le desconcertó un poco, aunque yo, para que no se enojara, le
aseguré que me reía por cierto recuerdo despertado en mi memoria.

---La escena del hambre está escrita, y si he de decirte la verdad, no
tiene pero.

---No dudo que esa escena puede ser admirable---dije con
malicia,---sobre todo si ha puesto la mano en ella la señorita Joaquina.

---Ya hemos escrito a todos los teatros de Italia, que se disputarán,
como siempre, el derecho de traducirla---dijo Joaquinita.

---¡Ah! Aquí no se recompensa el verdadero mérito. Bien dicen, que nadie
es profeta en su patria: verdad es que la posteridad hace justicia: pero
entretanto que esa justicia llega, los hombres superiores arrastramos
miserable existencia, y nos morimos como cualquier pelafustán sin que
nadie se acuerde de nosotros. Vamos a ver: ¿de qué me valen ahora a mí
los mausoleos, las inscripciones, las estatuas con que han de honrarme
en tiempos futuros, cuando la envidia calle y a nadie quede duda del
mérito de mis obras? Y si no ahí tienes a Cervantes, que es otro ejemplo
como este mío. ¿No vivió en la miseria? ¿No murió abandonado? ¿Acaso
tocó las ventajas positivas de ser el primer escritor de su siglo? Pues
a mí me pasa dos cuartos de lo mismo: por supuesto que si algo me
consuela es considerar cuánto se avergonzará la España futura al saber
que el autor de \emph{Catalina en Cromstad,} de \emph{Federico II en
Glatz}, de \emph{El negro sensible}, de \emph{La enferma fingida por
amor}, de \emph{Cadma y Sinoris}, de \emph{La escocesa de Lambrun} y de
otras muchas obras, ha vivido algún tiempo almorzando dos cuartos de
sangre frita y otras cosas que no nombro por respeto al arte de la
poesía, pues no lo quiero denigrar, denigrándome a mí mismo\ldots{} Pero
no hablemos de estas cosas, que dan tristeza y obligan a renegar de una
patria que no sabe premiar el mérito, y de unos tiempos en que los
magnates protegen la envidia y persiguen la inspiración.

---Calma, calma, señor don Luciano---dije yo mostrándome interesado por
el triunfo de la inspiración sobre la envidia;---tras esos tiempos
vendrán otros. ¡Quién sabe lo que pasará mañana!

---Eso me han dicho, sí---repuso Comella bajando la voz y con sonrisa de
satisfacción.

---¿Será cierto que Napoleón es del partido del Príncipe de Asturias?
¿Caerá Godoy?

---Eso no tiene duda. ¿Pues qué quiere Napoleón más que el bien de los
españoles?

---Justo; y aunque él y Godoy han sido muy amigotes, ya parece que el
otro ha conocido sus malas mañas, y sabe que todos queremos al heredero,
con lo cual dicho se está que nos hará el gusto. En cuanto a Godoy, yo
estoy en que no existe hombre peor en toda la redondez de la tierra.
Pueden perdonársele los medios de su elevación; puede perdonársele que
sea polígamo, ateo, verdugo, venal, y otras faltas por el estilo; pero
lo que no tiene nombre y prueba mejor que nada la corrupción de las
costumbres, es que proteja a los malos poetas, dando cordelejo a los que
son buenos, y además nacionales, españoles como yo, y no admitimos ese
fárrago de reglas ridículas y extranjeras con que Moratín y otros
poetastros de polaina embaucan a los tontos. ¿No piensas como yo?

---Lo mismito que usted---respondí.---Y ahora verá el señor don Luciano
cómo los franceses, cuando hayan arreglado lo de Portugal, arreglarán a
España y se acabará la protección a los malos poetas.

---Dios lo quiera así\ldots{} Pero es tarde y nos vamos, que antes del
almuerzo hemos de dejar concluida la escena entre Nelson y la hija de
Gravina.

---¿Tanta prisa corre?

---Para fin de mes ha de estar en la Cruz. Tendrá un éxito atroz. Ya
verás, Gabrielillo. Es preciso que vayas a aplaudir, porque me temo
mucho que los de Estala, Melón y Moratinillo han de querer silbarla. Hay
que estar con cuidado, y si ellos tienen la protección del gobierno, no
hay que asustarse por eso, la posteridad juzgará. Conque adiós.

Se marcharon a prisa, y yo me quedé pensando en la serie de maldades que
habría cometido el Príncipe de la Paz, para tener también en contra suya
a los malos poetas. Hasta mucho tiempo después no conocí que entre los
infinitos actos reprensibles de aquel monstruo de la fortuna había
algunos que la posteridad, por el contrario, debía recordar siempre con
agradecimiento.

\hypertarget{x}{%
\chapter{X}\label{x}}

Aún me faltaba oír, antes de volver a casa, otra opinión muy distinta de
las anteriores, y era la para mí respetabilísima de Pacorro Chinitas, el
amolador, personaje que tenía establecida su portátil industria en la
esquina de nuestra calle. Me parece que aún estoy viendo la piedra de
afilar que en sus rápidas evoluciones despedía por la tangente, al
contacto del acero, una corriente de veloces chispas, semejantes a la
cola de un pequeño cometa; y como era mi costumbre no apartar la vista
de la máquina mientras hablaba con el Júpiter de aquellos rayos, el
fenómeno ha quedado vivamente impreso en mi imaginación.

Era Pacorro Chinitas un hombre que aparentaba más de edad de la que
realmente tenía, merced a los disgustos domésticos, de que era autora su
mujer, célebre buñolera del Rastro, a quien llamaban la
\emph{Primorosa}. No puedo menos de dar algunas noticias sobre este
ejemplar matrimonio, porque los dos seres que lo formaban figuran algo
en acontecimientos posteriores, y que he de contar, si para entonces
tengo vida y el lector paciencia, como espero.

Es, pues, el caso que Pacorro Chinitas, varón manso y discreto, no podía
hacer buenas migas con la \emph{Primorosa}, cuya fama, extendida de polo
a polo, es decir, desde la calle de la Pasión hasta el pórtico de San
Bernardino, la acusaba de mujer pendenciera, batalladora y que partía de
un bofetón un par de quijadas, sin que estas y otras hazañas la hicieran
nunca caer en manos de la justicia. Chinitas se vio obligado a pedir una
separación, resignándose a no tener más compañera que la rueda coronada
de chispas, y en esta situación le conocí. Luego que nos hicimos amigos
contome las picardías de su antigua mitad, y así como en otros temas era
discretísimo, en éste era muy pesado, pues no pasaba día sin que me
regalara un nuevo capítulo de la larga historia de sus cuitas
matrimoniales. Como yo encontrara en aquel hombre cierta madurez de
juicio, cierto sentido práctico que en los demás no hallaba, resultó que
me aficioné a su conversación, y cuanto él decía me parecía entonces de
perlas, sin que pudiera explicarme la razón de esta preferencia por los
juicios de un hombre ignorante y rudo. Después he meditado bastante
sobre las cosas de aquel tiempo, y sobre la opinión general, y puedo
deciros sin miedo de equivocarme, que el hombre de más talento que
conocí en aquellos días fue el amolador de la calle del Baño.

Para muestra referiré mi conversación con él.

---¡Hola, Chinitas! ¿Cómo va? ¿Qué es eso que cuentan por ahí? ¿Conque
tenemos a los franceses en España?

---Eso dicen---contestó.---Y la gente está contenta.

---Y parece que van a cogerse a Portugal.

---Pues ello\ldots{} así dicen.

---Eso me parece muy bien. ¿Para qué sirve Portugal?

---Mira Gabrielillo---dijo incorporándose y apartando de la rueda las
tijeras, con lo cual cesaron por un momento las chispas;---tú y yo somos
unos brutos que no entendemos palotada de cosas mayores. Pero ven acá:
yo estoy en que todos esos señores que se alegran porque han entrado los
franceses no saben lo que se pescan, y pronto vas a ver cómo les sale la
criada respondona. ¿No piensas tú lo mismo?

---¿Qué he de pensar? Como Godoy es tan malo de por sí, cátate ahí que
Napoleón viene a quitarlo de enmedio, y a poner en el trono al Príncipe
de Asturias, que dicen es un gerifalte para el gobierno.

Chinitas volvió a aplicar el acero a la piedra, dandole movimiento con
el pie, y después de contestar a mis observaciones con un mohín muy
expresivo, añadió:

---Yo digo y repito que todos estos señores parece que están bobos.
Nosotros, los que no sabemos leer ni escribir, acertamos a veces mejor
que ellos; y lo que ellos no pueden ver, porque les encandila el sol de
un poder que tienen tan cerca, lo vemos nosotros desde abajo; y si no,
di tú: ¿No es preciso estar ciego para comprender que Napoleón no dice
lo que tiene pensado? ¿Ese hombre, no ha revuelto todas las partes del
mundo; no ha quitado de los tronos los reyes que ha querido para poner a
los mocosos de sus hermanos? Dicen que viene a poner al Príncipe de
Asturias y a quitar al \emph{choricero}. De eso me río yo. Sí, porque
Godoy y él no están de compinche para hacer cualquier picardía\ldots{} A
mí con ésas. Lo que menos le importa a Napoleón es que reine Fernandito
o prive don Manuel; lo que él quiere es cogerse a Portugal para darle un
pedazo a Godoy, y otro pedazo a la infanta que han puesto de reina allá
en \emph{Trucha o Truria}\ldots{}

---Pues que lo cojan y lo repartan---dije yo con gran crueldad para
nuestros vecinos,---¿qué nos importa? Con tal que quiten a ese hombre
tan malo\ldots{}

---Si cogen a Portugal, porque es un reino chiquito, mañana cogerán a
España, porque es grande. Yo me enfado cuando veo a esos bobalicones que
andan por ahí, abates, petimetres, frailes, covachuelistas, y hasta
usías muy estirados, que se ríen y se alegran cuando oyen decir que
Napoleón se va a embolsar a Portugal, y con tal de ver por tierra al
guardia, no les importa que el francés eche el ojo a un bocadito de
España, que no le vendrá mal para acabar de llenar el buche.

---Pero como dicen que no hay pecado que el \emph{choricero} no haya
cometido\ldots{}

---Mira, chiquillo---contestó con aplomo, probando con el dedo el filo
de las tijeras;---yo me río de todas las cosas que cuentan por ahí. Es
verdad que ese hombre es un ambicioso que no va más que a enriquecerse;
pero si ha llegado a ser duque y general y príncipe y ministro, ¿de
quién es la culpa sino de quien le ha dado todo eso sin merecerlo? Si
vienen y te dicen a ti: «Gabriel, mañana vas a ser esto y lo otro,
porque me da la gana, y sin que necesites para ello quemarte las cejas
estudiando latín,» ¿qué dirás tú? Dirás, «pues venga.»

---Eso no tiene duda.

---Y aunque ese hombre es una buena pieza y ha hecho muchas maldades, la
mitad de lo que dicen es mentira. También habrás visto que hoy le
escupen muchos que antes le adulaban; es que saben que va a caer, y la
sombra del árbol carcomido no le gusta a la gente. ¡Ah!, me parece que
aquí vamos a ver grandes cosas, sí señor, grandes cosas. Digo y repito,
que de esto va a resultar lo que nadie piensa, y muchos que hoy se
restriegan las manos de contento, llorarán mañana a moco y baba; y si
no, acuérdate de lo que te digo.

Aquellas razones, que me parecían encerrar profunda verdad, me hicieron
pensar; y como persona que ya se preciaba de saber escoger los hombres,
pensé que aquel sabio amolador era digno de ocupar un puesto de
consideración a mi lado, cuando yo fuera generalísimo, primer secretario
de Estado, archipámpano, y tuviera todas las jerarquías que esperaba de
la protección y ayuda de mi divina Amaranta.

---Pues yo lo que deseo---dije,---es que venga de una vez ese príncipe
tan bueno, que todo lo ha de arreglar a pedir de boca. ¿No cree usted,
lo mismo?

---Mira, chiquillo---repuso Chinitas con sibilítico tono,---yo me tengo
tragado que el heredero no vale para maldita la cosa, y esto no se puede
decir sino acá para entre los dos, porque si algunos nos oyeran,
lloverían almendradas. Cuando vivía la señora princesa de Asturias, que
en gloria esté, todos decían que Fernandito era enemigo de los franceses
y de Napoleón, porque éste ayudaba a Godoy, y ahora resulta que los
franceses son la mejor gente del mundo y Napoleón tan bueno como pan
bendito, sólo porque parece arrimarse al partido del Príncipe de
Asturias. Ésa no es gente formal, Gabrielillo; y yo lo que veo es que el
heredero tiene muchas ganas de serlo antes de que muera su padre, aunque
es de creer que el canónigo de Toledo y otros personajes le tienen
sorbidos los sesos, y serían capaces de obligarle a ser mal hijo, con
tal que ellos pudieran después echarse al cuerpo los mejores destinos.
Esa gente de arriba es muy ambiciosa, y hablando mucho del bien del
reino, lo que quiere es mandar; tenlo presente. Yo, aunque no me han
enseñado a leer ni a escribir tengo mi gramática parda; sé conocer a los
hombres, y aunque parece que somos bobos y nos tragamos todo lo que nos
dicen, ello es que a veces columbramos la verdad mejor que otros muy
sabiondos, y vemos clarito lo que ha de venir. Por eso te digo que
veremos cosas gordas, muy gordas; y si no, acuérdate de lo que te digo.

Así habló Chinitas. Cuando me separé de él para entrar en casa,
recuerdo, que iba resumiendo las distintas conferencias de aquella
mañana y lo mucho y vario que sobre un mismo asunto había oído en
anteriores días. Cada cual juzgaba los sucesos según sus pasiones, y
como yo no podía formarme idea exacta de la importancia de aquellos
hechos, en mi juvenil ignorancia y equivocado patriotismo, creía muy
justo que el conquistador del siglo se apoderara de un pequeño reino,
que a mi juicio no servía más que de estorbo. En cuanto a Godoy, no
había duda de que los comerciantes, los nobles, los petimetres, el
pueblo, los frailes, y hasta los malos poetas anhelaban su caída, unos
con razón y otros sin ella; unos por convicción de la ineptitud del
valido; bastantes por envidia, y muchos porque creían a pie juntillas
que habíamos de estar mejor cuando nos gobernara el heredero de la
corona. Fue singular cosa que todos se equivocaran respecto a la marcha
de los futuros sucesos esperando el próximo arreglo de todos los
trastornos; fue singular cosa que el optimismo ciego de la mayoría no
alcanzase a comprender lo que penetró con su ruda desconfianza el buen
juicio del amolador. Cada vez estoy más convencido de que Pacorro
Chinitas fue una de las más grandes notabilidades de su época.

\hypertarget{xi}{%
\chapter{XI}\label{xi}}

Ignoro si fueron las conversaciones de aquel día u otras causas, las que
enfriaron el entusiasmo de que yo estaba poseído por la mañana. «¡Cuánto
he desvariado!---decía para mí.---Y lo más seguro será que Amaranta
habrá visto solamente en mí un chico dispuesto a servirla mejor que
otro.»

Sin embargo, mi curiosidad era tan viva que no podía ocuparme en cosa
alguna, ni estar con calma en ninguna parte. Aquel día ni aun pude
visitar a Inés; y cuando cumplí las obligaciones de la casa me dispuse a
acudir a la cita. Vestime con el mayor esmero, dedicando el conjunto de
las fuerzas de mi inteligencia a conseguir que la persona de un servidor
de ustedes fuese el dechado de todas las gracias, y el resumen de
cuantas perfecciones concedió la Naturaleza a la juventud. El pedazo de
espejo que limpié desde por la mañana aduló mi amor propio, confirmando
ante mí la enfática presunción de que no escaseaban en el semblante del
criado de la González ciertos agradables rasgos, dignos de hacer fijar
la atención. Fue aquélla la primera vez que me sentí presumido: después,
recordándolo, he sentido ganas de abofetearme.

Yo habría deseado tener entonces el vestido más rico, más lujoso, más
elegante, más luciente que pudieran hacer los sastres del planeta que
habitamos; pero tuve que contentarme con el mío humildísimo, sin más
adorno que el del aseo, la pulcritud y esmero de mi peinado. Mi traje
era modesto; pero a pesar de ello, yo conocía que estaba bien, y que mi
persona y aire predisponían en favor mío. Con esto y con pensar durante
un breve rato ciertas frases delicadas y elegantes que me parecían muy
propias para contestar a los ob sequios de la diosa, di por terminados
los preparativos, y salí de la casa, sin dar cuenta a nadie de mi
expedición.

Llegué a la casa de la calle de Cañizares, residencia de la señora
marquesa, de quien era hermano el diplomático, pregunté por Dolores,
apareció ésta, y sin decirme nada me condujo por largos y oscuros
pasadizos, hasta que al fin dio conmigo en un camarín muy lujoso, donde
me ordenó que esperase. Mientras así lo hacía, creí sentir en la pieza
inmediata voces de señoras que hablaban y reían, y también creí escuchar
la desentonada voz del diplomático. Amaranta no me hizo aguardar mucho
tiempo. Cuando sentí el ruido de la puerta, cuando vi entrar a la
hermosa dama, cuando se adelantó hacia mí sonriendo con bondad,
pareciome que un ente sobrenatural se me acercaba, y temblé de emoción.

---Has sido puntual---me dijo.---¿Estás dispuesto a entrar en mi
servicio?

---Señora---contesté sin poder recordar ninguna de las frases que traía
preparadas,---estoy con mucho gusto a las órdenes de usía para cuanto se
digne mandarme.

---O yo me engaño mucho---dijo la dama sentándose junto a mí,---o tú
eres un chico bien nacido, hijo de alguna noble familia, y te hallarás
hoy en posición más baja de lo que te corresponde.

---Mi padre era pescador en Cádiz---respondí sintiendo por primera vez
en mi vida no ser noble.

---¡Qué lástima!---exclamó Amaranta:---sin embargo, no importa. Pepa me
ha dicho que cumples lo que se te encarga con mucha puntualidad, y sobre
todo con gran reserva; que eres formal a toda prueba; me ha dicho
también que tienes imaginación, y que podrías ser en otra esfera un
hombre de provecho.

---Mi ama---dije, disimulando mi orgullo---me hace demasiado favor.

---Bueno---continuó la diosa.---Ya comprendes que entrar en mi servicio
sin más recomendación que el propio mérito es más de lo que pudieras
desear. Pero me parece que tú tienes disposición para más altos empleos,
y\ldots{} creo que no seras desfavorecido por la fortuna. ¿Quién sabe lo
que llegarás a ser?

---¡Oh, sí señora, quién sabe!---dije sin contener el entusiasmo que en
mí producían aquellas palabras.

Amaranta estaba sentada frente a mí, como he dicho: su mano derecha
jugaba con un grueso medallón pendiente del cuello, y cuyos diamantes,
despidiendo mil luces, deslumbraban mis ojos. Tanta era mi gratitud y
admiración hacia aquella mujer, que no sé cómo no caí de rodillas a sus
plantas.

---Por de pronto no te exijo sino una grande fidelidad en mi servicio.
Yo acostumbro recompensar bien a los que bien me sirven, y a ti más que
a nadie, porque me han cautivado tu orfandad, tu abandono y la modestia
y circunspección que hallo en tu persona.

---Señora---exclamé en la efusión de mi gratitud;---¿cómo podré pagar
tantos beneficios?

---Siéndome fiel y haciendo puntualmente lo que te mande.

---Seré fiel hasta la muerte, señora.

---Ya ves que exijo poco. En cambio Gabriel, yo puedo hacer por ti lo
que no has soñado ni podrías soñar. Otros con menos méritos que tú se
han elevado a alturas inconcebibles. ¿No te ha ocurrido que podrías tú
subir lo mismo, encontrando una mano que te impulsara?

---¡Sí, señora! Sí me ha ocurrido, y ese pensamiento me ha vuelto loco
---contesté.---Viendo que usía se dignaba fijar en mí sus ojos, llegué a
creer que Dios había tocado su buen corazón, y que todo lo que hasta
ahora me ha faltado en el mundo, iba a recibirlo de una sola vez.

---Has pensado bien---dijo Amaranta sonriendo.---Tu adhesión a mi
persona y tu obediencia a mis órdenes te harán merecedor de lo que
deseas. Ahora escucha. Mañana voy a El Escorial, y es preciso que vengas
conmigo. Nada digas a tu ama; yo me encargo de arreglarlo todo, de
manera que consienta en el cambio de servidumbre. No digas tampoco a
nadie que me has hablado, ¿entiendes? Pasado mañana irás a mi casa,
desde donde puedes hacer el viaje en los coches que saldrán al mediodía.
Estaremos en El Escorial pocos días, porque regresaremos para ver la
representación que ha de darse en esta casa, y entonces, quizás vuelvas
por unos días al servicio de Pepa.

---¡Otra vez allá!---dije admirado.

---Sí: ya sabrás más adelante todo lo que tienes que hacer. Conque
retírate ya: no faltes mañana.

Prometí ser puntual y me despedí de ella. Diome a besar su mano con tan
dulce complacencia, que me sentí electrizado al poner mis labios en su
blanca y fina piel. Ni sus modales, ni sus miradas, ni ninguno de los
accidentes de su comportamiento para conmigo eran los de una ama para
con su criado. Más bien parecía tratarme como de igual a igual, y en
cambio yo, ciego ya para todo lo que no fuera la protección de Amaranta,
me lancé en la esfera de atracción de aquel astro que inundaba mi alma
de luz y calor.

Salí a la calle\ldots{} ¿a quién comunicar mi alegría? Al punto me
acordé de Inés, y subí la escalerilla que conducía a su sotabanco, pues
no sé si he dicho que la habitación de mis amigos estaba en la misma
casa. Encontré a Inés muy triste, y habiendo preguntado la causa, supe
que doña Juana, cuya naturaleza se desmejoraba con el continuo trabajar,
había caído enferma.

---¡Inés, Inesilla!---exclamé al encontrarme solo en la sala con la
muchacha.---Quiero hablarte. ¿Sabes que me voy?

---¿A dónde?---me preguntó con viveza.

---¡A palacio, a la corte, a correr fortuna! ¡Ah, picarona; ahora no te
reirás de mí; ahora va de veras!

---¿Qué va de veras?

---Que se me ha entrado por las puertas la fortuna, chiquilla. ¿Te
acuerdas de lo que hablamos el otro día? Bien te lo decía yo, y tú no me
hacías caso. ¿Pero no ves, reinita, que eso se cae de su peso?

---¿Qué se cae de su peso?

---Que así como otros han llegado a su mayor altura sin mérito propio, y
sólo porque a alguna gran persona se le antojó protegerles, nada tendría
de extraño que a mí me aconteciera dos cuartos de lo mismo, sí,
señorita.

---Eso es muy claro: avisa cuando llegues arriba. De modo que mañana te
tendremos de general o ministro cuando menos.

---No te burles, ¿estamos? Tanto como mañana, no; pero ¿quién sabe?

Inés empezó a reír, dejándome bastante confuso.

---Pero ven acá, tonta---dije con una seriedad, cuyo recuerdo me hace
morir de risa;---tú no estás oyendo hablar todos los días de un hombre
que no era nada, y hoy lo es todo; de un hombre que entró a servir en la
guardia española, y de la noche a la mañana\ldots{}

---¡Hola, hola!---dijo Inés burlándose de mí con más crueldad.---Ésas
tenemos, señor don Gabriel. ¡Qué callado lo tenía usted! ¿Se puede saber
quién es la dama que se ha enamorado de usted?

---Tanto como enamorarse, no, tonta---respondí, cortado;---pero\ldots{}
ya ves. Como uno no es saco de paja\ldots{} qué quieres. Todo el mundo,
aunque no valga nada, encuentra una persona a quien le gusta\ldots{}

Inés continuó riendo; pero yo conocí que después de mis últimas
palabras, la pobre necesitaba muchos esfuerzos para aparentar alegría.
Como su carácter no era apto para el disimulo, luego cesó de reír y se
puso muy seria.

---Bien, excelentísimo señor---dijo haciéndome una grave cortesía;---ya
sabemos a qué atenernos.

---La cosa no es para enfadarse---dije yo sintiéndome repuesto de mi
turbación;---lo que hay es, que si una persona me quiere proteger, no he
de hacerle ascos. ¡Y si tú la conocieras, Inesilla; si tú vieras qué
mujer, qué señora!\ldots{} Todo lo que te diga es poco; así es que no te
digo nada.

---¿Y esa señora se ha enamorado de ti?

---Dale con el enamoramiento; no es eso, mujer. Es que entro a servirla;
aunque quién sabe lo que podrá pasar\ldots{} Si vieras cómo me
trata\ldots{} Como de igual a igual, y se interesa mucho por mí\ldots{}
y es muy rica\ldots{} y vive en un palacio muy grande cerca de
aquí\ldots{} y tiene muchos criados\ldots{} y lleva en el cuello un
medallón con un diamante como un huevo\ldots{} y cuando le mira a uno,
se queda uno atortolado\ldots{} y es muy guapa\ldots{} y en palacio
puede tanto como el Rey\ldots{} y se llama\ldots{}

Recordé de pronto que Amaranta me había prohibido revelar su entrevista
con ella, y callé.

---Bueno---dijo Inés.---Ya veo que dentro de poco le tendremos a usía
hecho un archipámpano, con muchos galones y cintajos, dando que hablar a
la gente, y teniendo el gusto de oírse llamar ladrón, enredador,
tramposo y cuanto malo hay.

---Mira tú lo que es no entender las cosas---dije algo incomodado.---¿De
dónde sacas tú que todos los hombres célebres y poderosos, sean ladrones
y pícaros? No señor, también pueden ser buenos; y lo que es yo\ldots{}
supón, chiquilla, que por arte del demonio llegara yo a ser\ldots{} no
te rías, que de menos hizo Dios a Cañete; y todos somos hijos de Adán; y
tan de carne y hueso es Napoleón Bonaparte como yo. Pues suponte que
llego a ser\ldots{} no te rías. Si te ríes me callo.

---Si no me río---dijo Inés, conteniendo la hilaridad que de nuevo la
acometía.---Lo que dices está muy en razón, chiquillo. Si no hay más que
ponerse a ello. ¿Qué cuesta ser generalísimo, ministro, príncipe o
duque? Nada. Ni ¿a qué viene el romperse los ojos estudiando por
aprender todas las cosas que se deben saber para gobernar? Si los
aguadores y los mozos de cuerda, y los horteras, y los monaguillos, son
unos tontos de camisón, cuando no se van todos a palacio, sabiendo que
tienen seguro el sueldo de consejeros con sólo guiñarle el ojo a una
dama. Y si todas las damas no son tiernas de corazón, con tocarle el
codo a alguna de las cocineras de palacio, está hecho todo.

---No es eso: veo que tú no entiendes---dije no sabiendo cómo hacerme
comprender de Inés.---Eso que dices de aprender y saber gobernar, y lo
demás, no viene al caso. Verdad es que antes se necesitaba ser hombre de
ciencia para medrar; pero hoy, chiquilla, ya ves lo que pasa. No es sólo
Godoy, son cientos de miles los que ocupan altos puestos sin valer
maldita de Dios la cosa. Con un poco de despejo basta. Si sabré yo lo
que me digo.

---Ven acá, Gabriel---me dijo Inés dejando su costura.---Las cosas del
mundo pasan siempre como deben pasar. Esto lo sé yo sin que nadie me lo
haya dicho. Los hombres que mandan a los demás, están en aquel puesto
por su nacimiento, pues\ldots{} porque así está arreglado, de modo que
los reyes nacen de los reyes\ldots{} Cuando algún hombre que no ha
nacido en cuna real llega a gobernar el mundo, debe ser porque Dios le
ha dado un talento, una cosa celestial que no tienen los demás. Y si no,
ahí me tienes a Napoleón, que es emperador de todo el mundo, y manda no
sé cuántos millones de soldados; pero es porque él se lo ha ganado, y
porque desde chiquito aprendía cuanto hay que saber, y los maestros se
quedaban lelos, viendo que sabía más que ellos\ldots{} El que sube tanto
sin tener mérito es por casualidad, o por mil picardías, o porque los
reyes lo quieren así; ¿y qué hacen para tenerse arriba? Engañan a la
gente, oprimen al pobre, se enriquecen, venden los destinos y hacen mil
trampas. Pero buen pago les dan, porque todo el mundo les aborrece y lo
que se desean es verles por los suelos. ¡Ah, chiquillo! Yo no sé como no
entiendes esto, esto que es tan claro como el agua\ldots{}

A pesar de ser tan claro como el agua, yo no lo comprendía. Muy lejos de
eso, estaba tan obcecado, tan dominado por la vanidad, que no vi sino
impertinencias y majaderías en las juiciosas razones de la costurerilla.
Aún fue más lejos mi soberbia, porque mi amor propio se resintió; me
sentí pavo real, erguí mi cuello, levanté la cola tornasolada, y con mis
feas patas de pájaro vanidoso pisoteé la discreta paloma, diciéndole
estas palabras:

---Inés, hablemos claro. Veo que tú no comprendes ciertas cosas\ldots{}
Tú eres muy buena, y por eso te quiero y te estimo. No dudes por lo
tanto que de aquí en adelante haré en bien tuyo cuanto me sea posible.
Tú eres muy buena; pero es preciso confesar que tienes pocos alcances.
Al fin eres mujer, y las mujeres\ldots{} como no sea hacer calceta, y de
poner el puchero a la lumbre, de nada entienden una higa. Este negocio
que tratamos no es para tu pobre cabecita. Los hombres son los que lo
entendemos bien, porque tenemos un modo de ver las cosas más por lo
alto, porque en fin, tenemos más talento. No extraño lo que me has dicho
porque\ldots{} ¿tú qué puedes entender?\ldots{} Pero eres una chica muy
buena: te quiero, te quiero mucho, no te enfades. Puedes estar segura de
que jamás me olvidaré de ti.

Lector: cuando leas esto te suplico que te despojes de toda benevolencia
para conmigo. Sé justiciero e implacable, y ya que no me tienes, por
ventaja mía, al alcance de tus honradas manos, descarga en el libro tu
ira, arrójalo lejos de ti, pisotéalo, escúpelo\ldots{} ¡ay!, pero no: él
es inocente, déjalo, no lo maltrates, él no tiene culpa de nada; su
único crimen es haber recibido en sus irresponsables hojas lo que yo he
querido poner en él, lo bueno y lo malo, lo plausible y lo irrisorio, lo
patético y lo tonto que al escribir esta historia he ido sacando,
escarbador infatigable, de los escombros de mi vida. Si algo encuentras
que me desfavorezca, tan mío es como lo que te parezca laudable. Ya
habrás conocido que no quiero ser héroe de novela: si hubiera querido
idealizarme, fácil me habría sido conseguirlo, cuidando de encerrar con
cien llaves todas mis flaquezas y necedades, para que sólo quedasen a la
vista del público los hechos lisonjeros, adicionados con lindísimas
invenciones, que en caso de apuro no me habrían de faltar. Pero repito
que no quiero idealizarme: bien sé que a los ojos de muchos, mi
personalidad estaría cien codos más alta, si yo representase en mí a un
mozuelo desvergonzado, pendenciero y atrevido, que en los diez y seis
años de su edad hubiese tenido tiempo y fortuna para matar en duelo a
dos docenas de semejantes, y quitar la honra a igual número de
doncellas, casadas o viudas, esquivando la persecución de la justicia y
la venganza de celosos padres o maridos. Todo esto sería muy bonito;
pero diré con el latino: \emph{sed nunc non erat hic locus}.

Como prueba de mi modestia, no he vacilado en copiar el diálogo con
Inés, que me favorece tan poco, atreviéndome a esperar que si el lector
no me adorase romántico, podrá apreciarme sincero. Hagamos, pues, las
paces y continuaré la narración en el mismo punto en que la dejé; y es
que habiendo espetado las palabras referidas y aun algunas más, hijas de
mi estólida vanidad, dejé a Inés, creyendo que debía buscar interlocutor
más conforme a la alteza y sublimidad de mis pensamientos. Inés no me
dijo una palabra más, y yo, atraído por los alegres sones de la flauta
tocada por don Celestino, fui a buscarle a su cuarto, y con las manos
juntas atrás, y el aire de persona protectora, le hablé así:

---¿Cómo van esos asuntos, señor mío?

---¡Oh, divinamente!---contestó con su optimismo de siempre.---Al fin se
me hará justicia, y según me ha dicho esta mañana el oficial de la
secretaría, no puede pasar de la semana que viene.

---Me parece que a usted no le vendría mal un arciprestazgo de buena
renta o cosa así\ldots{} Dígolo, porque aunque a usted le sorprenda, tal
vez exista alguna persona que se lo pueda conseguir.

---¿Quién, hijo mío, quién, a no ser mi paisano y amigo el Serenísimo
Príncipe de la Paz?

---En donde menos se piensa salta una liebre\ldots{} Ya veremos, ya
veremos---dije yo haciendo todo lo posible para que la expresión de mi
semblante fuera la más misteriosa y grave.

Quedose aturdido con mis palabras, y volví al lado de Inés, de quien no
quería despedirme dejándola enojada. Con gran sorpresa mía, la muchacha
no conservaba enfado alguno, y me habló con aquella incomparable
ecuanimidad que siempre fue su principal atractivo. Despedime
prometiendo que la recordaría siempre, y ella se mostró tan afable, tan
cariñosa como si nada hubiera pasado. Su espíritu, cuya elevación y
superioridad desconocía yo entonces, confiaba firmemente sin duda en mi
pronta vuelta.

A los dos días mi ama me dijo que había convenido con Amaranta en que yo
pasara a servir a ésta. Arreglé mi pequeño ajuar, y fui a la casa de mi
nueva dueña. Allí me pusieron una librea, y subiendo al coche de la
servidumbre, el cual seguía a otro ocupado por la marquesa y su hermano
el diplomático, emprendí el camino de El Escorial, a donde llegamos por
la noche.

\hypertarget{xii}{%
\chapter{XII}\label{xii}}

Como al llegar a El Escorial nos encontrarnos sorprendidos por la
noticia de gravísimos acontecimientos, no estará demás que mencione lo
que por el camino me contó el mayordomo de la marquesa, pues a sus
palabras dio profético sentido lo que ocurrió después.

---Me parece que en el Real Sitio pasa algo que va a ser sonado---me
dijo.---Esta mañana se decía en Madrid\ldots{} Pero lo que haya lo hemos
de saber pronto, pues dentro de tres horas y media si Dios quiere
daremos fondo en la Lonja.

---¿Y qué se decía en Madrid?

---Allí todos quieren al Príncipe y aborrecen a los Reyes Padres, y como
parece que sus majestades se han propuesto mortificar al muchacho,
apartándole de su lado\ldots{} Eso yo lo he visto, y el Príncipe tiene
una cara que da compasión\ldots{} Se dice que sus padres no le quieren,
lo cual está muy mal hecho: a mí me consta que ni una sola vez le lleva
el rey a las cacerías, ni le sienta a la mesa, ni le muestra aquel
cariño que parece natural en un buen padre.

---¿Será que el Príncipe anda metido en conspiraciones y
enredos?---dije.

---Ello bien pudiera ser. Según oí la semana pasada en el Real Sitio, el
Príncipe se da unas encerronas que ya ya\ldots{} no habla con nadie,
está como quien ve visiones, y se pasa las noches en vela. Con esto la
Corte andaba muy alarmada, parece que acordaron vigilarle hasta
averiguar lo que traía entre manos.

---Pues ahora caigo en que me dijeron que el Príncipe era algo literato,
y se pasaba las noches traduciendo del francés o del latín, que esto no
lo recuerdo bien.

---Sí, en El Escorial se cree eso; pero sabe Dios\ldots{} Hay quien
asegura que lo que el Príncipe trae entre manos es cosa gorda; que las
tropas de Napoleón que han entrado en España, lo que menos piensan es
guerrear con Portugal, y parece que vienen a apoyar a los partidarios
del Príncipe.

---Ésas son patrañas; quizás el pobre Fernandito no piensa más que en
traducir sus libros\ldots{}

---Parece que el que tradujo hace poco no gustó a los papás, porque
hablaba de no sé qué revoluciones, y ahora está con otro: como no sea
alguna endiablada tramoya para pescar el trono\ldots{}

Así continuó poco más o menos nuestra conversación hasta que llegamos al
Real Sitio. El diplomático y su hermana se apearon de su coche, y
nosotros del nuestro. Como los dos viajeros debían aposentarse en
palacio y en las habitaciones de Amaranta, que ya había llegado el día
anterior, desde luego el mayordomo nos encaminó allá haciéndonos
recorrer medio mundo en escaleras, galerías, patios y pasillos. Todo
indicaba que ocurría algo extraordinario en la regia morada, porque se
veía por los pasillos y salas de tránsito más gente de la que
acostumbraba estar en pie a tal hora, que era la de las diez. Preguntó
la marquesa; mas le contestaron de un modo tan vago, que nada pudo sacar
en claro.

Instalados en las habitaciones de mi ama, donde me ocupé en acomodar los
equipajes, según las órdenes que se me daban, al poco rato entró
Amaranta tan inmutada, que fue preciso aguardar un poco para que,
repuesta de su zozobra, pudiese explicar lo que pasaba.

---¡Ay!---exclamó, cediendo a las reiteradas preguntas de sus tíos;---lo
que pasa es terrible. ¡Una conjuración, una revolución! ¿En Madrid no
ocurría nada cuando ustedes salieron?

---Nada; todo estaba tranquilo.

---Pues aquí\ldots{} Es una cosa tremenda, y quién sabe si estaremos
vivos mañana.

---Pero hija, dínoslo claramente.

---Parece que se ha descubierto que querían asesinar a los Reyes; todo
estaba preparado para un movimiento en palacio.

---¡Qué horror!---exclamó el diplomático.---Bien decía yo que bajo la
capita de servidores del Rey se escondían aquí muchos jacobinos.

---No es nada de jacobinos---continuó mi ama.---Lo más extraño es que el
alma de la conjuración es el príncipe de Asturias.

---No puede ser---dijo la marquesa, que era muy afecta a Su Alteza.---El
Príncipe es incapaz de tales infamias. Justo y cabal, lo que yo decía.
Sus enemigos han ideado perderle por la calumnia, ya que no lo han
conseguido por otros medios.

---Pues la revolución preparada, que por lo que dicen, iba a ser peor
que la francesa---prosiguió Amaranta,---se ha fraguado en el cuarto del
Príncipe, a quien se han encontrado unos papelitos que ya\ldots{} Dícese
que están complicados el canónigo don Juan de Escóiquiz, el duque del
Infantado, el conde de Orgaz y Pedro Collado, el aguador de la fuente
del Berro, hoy criado del Príncipe.

---Creo que tú, sobrina---dijo el marqués ofendido de que mi ama contase
cosas que él no sabía,---te dejas arrastrar por tu impresionable
imaginación. Tal vez lo que ocurre no tenga importancia alguna, y pueda
yo esclarecerlo con datos y noticias de índole muy reservada que se me
han trasmitido de cierta parte que debo callar.

---Yo contaré lo que me han dicho. Desde algún tiempo llamaba la
atención que el Príncipe pasase las noches encerrado en su cuarto sin
compañía, aunque los Reyes creían que se ocupaba en traducir un libro
francés. Pero ayer se encontró Su Majestad en su cuarto una carta
cerrada, cuyo sobre no tenía más que estas palabras: Luego, luego,
luego. Abriola el Rey, y leyó un aviso sin firma, en que le decían:
«Cuidado, que se prepara una revolución en palacio. Peligra el trono y
la reina María Luisa va a ser envenenada.»

---¡Jesús, María y José!---exclamó la marquesa, que como mujer nerviosa
estuvo a punto de desmayarse.---Pero, ¿qué demonio del infierno se ha
metido en El Escorial?

---Figúrense ustedes cómo se quedaría el pobre Rey. Al punto sospecharon
del Príncipe y decidieron ocuparle sus papeles. Dudaron mucho tiempo
sobre el modo de hacerlo; pero al fin el Rey se decidió a reconocer él
mismo en persona el cuarto de su hijo. Fue allá con pretexto de
regalarle un tomo de poesías, y según dicen, Fernando se turbó de tal
modo al verle entrar, que descubrió con su mirar medroso y azorado el
sitio en que estaban los papeles. El Rey los cogió todos, y parece que
padre e hijo se dijeron algunas cosas un poco fuertes; después de lo
cual, Carlos salió indignado ordenándole que permaneciese en su cuarto
sin recibir a persona alguna\ldots{} Esto fue ayer; en seguida vino el
ministro Caballero, y entre él y los Reyes examinaron los papeles. No
sabemos lo que pasó en esta conferencia; pero debió de ser cosa fuerte,
porque la reina se retiró a su cuarto llorando. Después se dijo que los
papeles encontrados en poder del Príncipe contenían la clave de
terribles proyectos, y según afirmó Caballero después de hablar con los
Reyes, el Príncipe Fernando debía ser condenado a muerte.

---¡A muerte!---exclamó la marquesa.---Pero ¡esa gente está loca!
¡Condenar a muerte a todo un Príncipe de Asturias!

---No hay que apurarse todavía---dijo el diplomático con su acostumbrada
suficiencia.---Tal vez se nos muestren esos papeles para saber nuestro
dictamen, y haremos luminoso estudio de todos ellos para resolver lo que
convenga.

---Pero ¿no se sabe lo que contenían esos papeles?---preguntó la
marquesa.

---Se cuentan tantas cosas en palacio, que no se sabe la verdad. La
reina no nos ha dicho nada, y ha pasado toda la noche a lágrima viva,
lamentándose de la ingratitud de su hijo. También dice que no permitirá
que se le persiga, porque él no tiene la culpa de lo que ha hecho, sino
esos dos o tres pícaros ambiciosos que le rodean.

---Dejémonos de anticipar juicios sobre estos sucesos---dijo el
marqués.---Ya lo averiguaré yo todo, y sabré si es un complot de los
enemigos del Príncipe o simplemente una verdadera y efectiva
conjuración; mas cuando yo lo sepa, guárdense ustedes de preguntarme,
pues ya conocen mis ideas\ldots{}

---Parece que han decidido formar causa para averiguar quiénes son los
delincuentes---continuó Amaranta,---y esta noche va el Príncipe a
declarar a la Cámara regia.

A este punto llegaban de tan interesante conversación, cuando sentimos
cierto rumor como de gente que se agolpaba en sitio cercano a la
habitación en que estábamos. Como no tenía gran cosa que hacer cerca de
mi ama, y además la curiosidad me llamaba fuera, salí, bajé una escalera
y halleme en una anchurosa pieza tapizada, que correspondía por ambos
lados a otras de igual tamaño y parecidos adornos. Recorrí dos o tres
siguiendo la dirección de las personas que se encaminaban a un lugar
determinado, y no vi nada digno de llamar la atención más que algunos
grupos de palaciegos que cuchicheaban por lo bajo con mucho calor.

Yo me enorgullecía de encontrarme en palacio, creyendo que sólo por el
contacto del suelo que pisaban mis pies, tenía nuevos títulos a la
consideración del género humano; y como cuantos llevamos la generosa
sangre española en nuestras venas, somos propensos a la fatuidad, no
pude menos de creerme un verdadero y genuino personaje, y hubiera
deseado encontrar al paso a alguno de mis antiguos conocimientos de
Madrid o Cádiz para mostrarle en gestos y palabras el convencimiento de
mi respetabilidad. Felizmente no conocí alma de Dios entre tanta gente y
me libré de ponerme en ridículo.

Encontrábame en aquella larga serie de habitaciones tapizadas que,
recorriendo toda la extensión de palacio por la parte interior, sirve de
lazo de unión a las moradas regias, cuyas luces se abren en la fachada
oriental del inmenso edificio. Seguí la dirección de los demás sin
reparar si debía aventurar mis pasos por aquellos sitios, mas como nadie
me dijo nada, continué muy impávido. Las salas estaban débilmente
alumbradas, y en la dulce penumbra las figuras de los tapices, parecían
sombras detenidas en las paredes, o débiles reflejos luminosos enviados
por escondido foco sobre el oscuro fondo de las cámaras. Paseé mi vista
por aquella multitud de figuras mitológicas, con cuya desnudez
provocativa se habían adornado las negras murallas construidas por
Felipe, y ya consagraba mi atención a contemplarlas, cuando pasó la
extraña procesión de que voy a dar cuenta.

El Príncipe de Asturias, a quien se había comenzado a instruir sumaria
por el delito de conspiración, volvía de la Cámara real, donde acababa
de prestar declaración. No olvidaré jamás ninguna de las
particularidades de aquella triste comitiva, cuyo desfile ante mis
asombrados ojos, me impresionó vivísimamente aquella noche, quitándome
el sueño. Iba delante un señor con un gran candelero en la mano, como
alumbrando a todos, y para esto lo llevaba en alto, aunque tan poca luz
servía sólo para hacer brillar los bordados de su casacón de
gentil-hombre. Luego seguían algunos guardias españoles, tras de ellos
un joven en quien al instante reconocí no sé por qué al Príncipe
heredero. Era un mozo robusto y de temperamento sanguíneo, de rostro
poco agradable, pues la espesura de sus negras cejas y la expresión
singular de su boca hendida y de su excelente nariz le hacían bastante
antipático, por lo menos a mis ojos. Iba con la vista fija en el suelo,
y su semblante alterado y hosco indicaba el rencor de su alma. A su lado
iba un anciano como de sesenta años, y al principio no comprendí que
pudiera ser el Rey Carlos IV, pues yo me había figurado a este personaje
como un hombrecito enano y enteco, siendo lo cierto que tal como le vi
aquella noche era un señor de mediana estatura, grueso, de rostro
pequeño y encendido, y sin rasgo alguno en su semblante que mostrase las
diferencias fisonómicas establecidas por la Naturaleza entre un Rey de
pura sangre y un buen almacenista de ultramarinos.

En los personajes que le acompañaban, y eran, según después supe, los
ministros y el gobernador interino del Consejo, me fijé más que en la
real persona, y después daré a conocer a alguno de aquellos esclarecidos
varones. Cerraba, por último, la procesión el zaguanete de la guardia
española, y nada más. Mientras pasó la comitiva, sepulcral silencio
reinó en todo el tránsito, y tan sólo se oyeron las pisadas que se
perdían de cámara en cámara hasta llegar a las que formaban el cuarto de
Su Alteza. Cuando entraron en éste la cháchara comenzó de nuevo entre
los circunstantes, y vi a Amaranta, que habiendo salido a buscarme,
hablaba con un caballero vestido de uniforme.

---Creo que al declarar---dijo el caballero,---Su Alteza ha estado un
poco irreverente con el Rey.

---¿De modo que está preso?---preguntó Amaranta con gran curiosidad.

---Sí, señora. Ahora quedará detenido en su cuarto con centinelas de
vista. Vea usted, ya salen. Deben haberle recogido su espada.

La comitiva volvió a pasar sin el Príncipe, y precedida del gentilhombre
con el candelabro que iba abriendo camino. Cuando el Rey y sus ministros
se alejaron, los palaciegos que habían salido a las galerías, fueron
desapareciendo también en sus respectivas madrigueras, y por mucho
tiempo no se oyó más que el violento cerrar de multitud de puertas. Se
apagaron las pocas luces que alumbraban tan vastos recintos, y las
hermosas figuras de los tapices se desvanecieron en la oscuridad, como
fantasmas a quienes el canto del gallo llama a sus ignotas moradas.

Yo subí con mi ama a nuestro departamento, y me asomé por una de las
ventanas que caían hacia el interior para reconocer, como de costumbre,
el sitio en que estaba. Era oscurísima la noche y no vi más que una masa
negra e informe de la cual se destacaban altos tejados, cúpulas, torres,
chimeneas, paredones, aleros, arbotantes y veletas que desafiaban el
firmamento como los topes de un gran navío. Tal imponente vista causaba
cierto terror al espíritu, despertando meditaciones que se mezclaban a
las sugeridas por lo que acababa de ver; mas no pude ocuparme mucho en
trabajos del pensamiento, porque un sutilísimo ruido de faldas, y un
ligero \emph{ce ce} con que se me llamaba, me hizo volver la cabeza, y
apartarme de la ventana.

La transición fue extremadamente brusca, cuando distrayéndome de la
sombría perspectiva exterior, apareció ante mis ojos la figura de
Amaranta y su celestial sonrisa. Reinaba profundo silencio: el marqués
diplomático y su hermana se habían retirado. Amaranta había cambiado su
traje de camino por una vestidura blanca y suelta que aumentaba su
hermosura, si su hermosura fuera susceptible de aumento. Cuando me
llamó, aún no se había apartado su doncella; pero ésta salió sin
tardanza, y luego nuestra seductora dueña, cerrando por sí misma la
puerta que daba a la galería, me hizo señas para que me acercase.

\hypertarget{xiii}{%
\chapter{XIII}\label{xiii}}

---No olvides lo que me has jurado---dijo sentándose.---Yo confío en tu
fidelidad y en tu discreción. Ya te dije que me parecías un buen
muchacho, y pronto llegará la ocasión de probármelo.

No recuerdo bien las vehementes expresiones con que juré mi fidelidad;
mas debieron ser muy acaloradas y aún creo que las acompañé con
dramáticos gestos, porque Amaranta sonrió mucho y me recomendó que
convenía fuera menos fogoso. Después continuó así:

---¿Y no deseas volver al lado de la González?

---Ni al lado de la González, ni al lado de todos los reyes de la tierra
---contesté, pues mientras viva no pienso apartarme del lado de mi ama
querida, a quien adoro.

Si mal no recuerdo, me puse de rodillas ante el sillón en que Amaranta
reposaba con seductora indolencia; pero ella me hizo levantar,
diciéndome que debía pensar en volver a casa de mi antigua ama, aunque
continuara sirviendo a la nueva con toda reserva. Esto me pareció algo
misterioso e incomprensible; pero no insistí en que lo esclareciera por
no parecer impertinente.

---Haciendo lo que te mando---continuó,---puedes vivir seguro de que te
irá bien en el mundo. ¡Y quién sabe, Gabriel, si llegarás a ser persona
de condición y de fortuna! Otros con menos ingenio que tú se han
convertido de la mañana a la noche en verdaderos personajes.

---Eso no tiene duda, señora. Pero yo he nacido en humilde cuna, yo no
tengo padres, yo no he aprendido más que a leer, y eso muy mal, en
libros que tengan letras como el puño, y apenas escribo más que mi firma
y rúbrica en la cual hago más rasgos que todos los escribanos del
gremio.

---Pues es preciso pensar en tu educación: el hombre debe ilustrarse. Yo
me encargo de eso. Pero será con la condición de que has de servirme
fielmente; no me canso de repetírtelo.

---En cuanto a mi lealtad no hay más que hablar. Pero entéreme usía de
cuáles son mis obligaciones en este nuevo servicio---dije anhelando que
satisficiera mi curiosidad respecto a lo que tenía que hacer para
hacerme acreedor a tantas bondades.

---Ya te lo iré diciendo. Es cosa difícil y delicada: pero confío en tu
buen ingenio.

---Pues ya anhelo prestar a usía esos servicios tan difíciles y
delicados ---contesté con todo el énfasis de mi bullicioso
carácter.---No seré un criado, seré un esclavo pronto a obedecer a usía,
aunque pierda en ello la vida.

---No se necesita perder la vida---dijo sonriendo.---Basta con un poco
de vigilancia; y sobre todo teniendo completa adhesión a mi persona,
sacrificándolo todo a mi deseo y no viendo más que la obligación de
satisfacer mi voluntad, te será fácil cumplir.

---Pues estoy impaciente, deshecho por empezar de una vez.

---Ya te enterarás con más calma. Esta noche tengo que escribir muchas
cartas\ldots{} Y ahora que recuerdo; vas a empezar a cumplir lo que
espero de ti, respondiéndome a varias preguntas cuya contestación
necesito para escribir. Dime, ¿Lesbia solía ir a tu casa sin ser
acompañada por mí?

Me quedé perplejo al oír una pregunta que me parecía tan lejos del
objeto de mi servicio, como el cielo de la tierra. Pero recogí mis
recuerdos y contesté:

---Algunas veces, aunque no muchas.

---¿Y la viste alguna vez en el vestuario del teatro del Príncipe?

---Eso sí que no lo recuerdo bien, y por tanto no puedo jurar que la vi,
ni tampoco que no la vi.

---No tiene nada de particular que la hayas visto, porque Lesbia no se
mira mucho para ir a semejantes sitios---dijo Amaranta con mucho desdén.

Después de una pausa en que me pareció muy preocupada, continuó así:

---Ella no guarda las conveniencias, y fiada en las simpatías que
encuentra en todas partes por su gracia, por su dulzura y por su
belleza\ldots{} aunque la verdad es que su belleza no tiene nada de
particular.

---Nada absolutamente de particular---añadí yo adulando la apasionada
rivalidad de mi ama.

---Pues bien---dijo,---ya me enterarás despacio de esta y otras cosas
que necesito saber. Lo primero que te recomiendo es la más absoluta
reserva, Gabriel. Espero que estarás contento de mí y yo de ti, ¿no es
verdad?

---¿Cómo podré pagar a usía tantos beneficios?---exclamé con la mayor
vehemencia.---Creo que voy a volverme loco señora, y me volveré de
seguro. Yo no puedo menos de desahogar mi corazón, mostrando los
sentimientos que lo llenan desde el instante en que usía se dignó poner
los ojos en mí. Y ahora cuando usía me ha dicho que va a hacer de mí un
hombre de provecho, y a ponerme en disposición de ocupar puesto honroso
en el mundo, estoy pensando que aunque viva mil años adorando a mi
bienhechora, no le pagaré tantos favores. Yo tengo deseos muy fuertes de
ser un hombre como algunos que veo por ahí. ¿No es esto posible? ¿Usía
cree que podré ser, instruyéndome con su ayuda? ¡Ay! Cuando uno ha
nacido pobre, sin parientes ricos; cuando se ha criado en la miseria y
en la triste condición de sirviente, no puede subir a otro puesto mejor
sino por la protección de alguna persona caritativa como usía. Y si yo
llegara a conseguir lo que deseo, no sería el primer caso, ¿no es
verdad, señora? Porque gentes hay aquí muy poderosas y muy grandes que
deben su fortuna y su carrera a alguna ilustrísima mujer que les dio la
mano.

---¡Ah!---dijo Amaranta con bondad.---Veo que tú eres ambicioso,
Gabrielillo. Lo que has dicho últimamente es cierto; hombres conocemos a
quienes ha elevado a desmedida altura la protección de una señora.
¡Quién sabe si encontrarás tú igual proporción! Es muy posible. Para que
no pierdas la esperanza, ahí va un ejemplo. En tiempos muy antiguos y en
tierras muy remotas había un grande imperio que era gobernado en
completa paz por un soberano sin talento; pero tan bondadoso, que sus
vasallos se creían felices con él y le amaban mucho. La sultana era
mujer de naturaleza apasionada y viva imaginación; cualidades contrarias
a las de su marido, merced a cuya diferencia aquel matrimonio no era
completamente feliz. Cuando heredó a su padre, el sultán tenía cincuenta
años y la sultana treinta y cuatro. Acertó entonces a entrar en la
guardia genízara un joven que se hallaba casi en el mismo caso que tú,
pues aunque no era de nacimiento tan humilde, ni tampoco dejaba de tener
alguna instrucción, era bastante pobre y no podía esperar gran carrera
de sus propios recursos. Al punto se corrió en la corte la voz de que el
joven guardia había agradado a la esposa del sultán, y esta sospecha se
confirmó al verle avanzar rápidamente en su carrera, hasta el punto de
que a los veinticinco años de edad ya había alcanzado todos los honores
que pueden ser concedidos a un simple súbdito. El sultán, lejos de poner
reparos a tan rápido encumbramiento, había fijado todo su cariño en el
favorecido joven, y no contento con darle las primeras dignidades le
entregó las riendas del gobierno, le hizo gran visir, príncipe, y le dio
por esposa a una dama de su propia familia. Con esto estaban los pueblos
de aquella apartada y antigua comarca muy descontentos y aborrecían al
joven y a la sultana. En su gobierno, el joven valido hizo algunas cosas
buenas; mas el pueblo las olvidaba, para no ocuparse sino de las malas
que fueron muchas, y tales que trajeron grandes calamidades a aquel
pacífico imperio. El sultán, cada vez más ciego, no comprendía el
malestar de sus pueblos, y la sultana, aunque lo comprendía no pudo en
lo sucesivo remediarlo, porque las intrigas de su corte se lo impedían.
Todos odiaban al favorecido joven, y entre sus enemigos más encarnizados
se distinguían los demás individuos de la regia familia. Pero lo más
extraño fue que el hombre a quien una mano tan débil como generosa había
elevado sin merecimientos, se mostró ingrato con su protectora y lejos
de amarla con constante fe, amó a otras mujeres, y hasta llegó a
maltratar a la desventurada a quien todo lo debía. Las damas de la
sultana referían que algunas veces la vieron derramando acerbo llanto y
con señales en su cuerpo de haber recibido violentos golpes de una mano
sañuda.

---¡Qué infame ingratitud!---exclamé sin poder contener mi
indignación.---¿Y Dios no castigó a ese hombre, ni devolvió a aquellos
inocentes pueblos su tranquilidad, ni abrió los ojos del excelente
sultán?

---Eso no lo sé---contestó Amaranta mordiendo las puntas blancas de la
pluma con que se preparaba a escribir;---porque estoy leyendo la
historia que te cuento en un libro muy viejo, y no he llegado todavía al
desenlace.

---¡Qué hombres tan malos hay en el mundo!

---Tú no serás así---dijo Amaranta sonriendo;---y si algún día te vieras
elevado a tales alturas por las mismas causas, harías todo lo posible
porque se olvidara con la grandeza de tus actos, el origen de tu
encumbramiento.

---Si por artes del demonio eso sucediera---respondí,---lo haré tal y
como usía lo dice, o no soy quién yo, pues a mí me sobran alma y corazón
para gobernar, sin dejar de ser un hombre bueno, decente y generoso.

Estas últimas palabras la hicieron reír, y ofreciéndome que al día
siguiente me recomendaría a un padre jerónimo del monasterio para que me
instruyese, me dijo que iba a escribir cartas muy urgentes y que la
dejase sola. La doncella volvió para conducirme al cuarto donde debía
recogerme, y una vez dentro de él me acosté; mas los pensamientos
evocados en mi cabeza por la pasada conferencia, me confundían de tal
modo, que mi sueño fue agitado y doloroso, cual opresora pesadilla, y
creí tener sobre el pecho todas las cúpulas, torres, tejados, aleros,
arbotantes y hasta las piedras todas del inmenso Escorial.

\hypertarget{xiv}{%
\chapter{XIV}\label{xiv}}

Al día siguiente se reunieron a comer en casa de Amaranta, Lesbia, el
diplomático y su digna hermana. He hablado poco de esta buena señora,
que no figura gran cosa en los acontecimientos referidos, lo cual es
sensible, porque su carácter y excelentes prendas, merecería mención muy
detallada. La marquesa era una dama de avanzada edad, mujer orgullosa,
de modestas costumbres, española rancia por los cuatro costados, de
carácter franco y sin artificios, muy natural, muy caritativa, enemiga
de trapisondas y aventuras, muy cariñosa para todo el mundo; en fin, era
la honra de su clase. Su lado flaco, consistía en creer que su hermano
tenía mucho talento. Aunque era modesta en su trato privado, gustaba de
dar grandes fiestas, prefiriendo las representaciones dramáticas, a que
tenía mucha afición. Su teatro era el primero de la corte, y para la
representación de Otello había gastado considerables sumas. Protegía y
trataba a los cómicos; pero siempre a mucha distancia.

También estaba convidado a comer aquel día con mi ama, el señor don Juan
de Mañara; pero cuando fui a llevarle la invitación, contestó
excusándose, por tocarle entrar de guardia a la misma hora. Y a
propósito del pisaverde, no debo pasar en silencio la circunstancia de
que le vi por la mañana en compañía de Lesbia, ambos en traje que
parecía indicar regresaban de uno de esos crepusculares y campestres
paseos, siempre anhelados por los amantes. En la tarde de aquel mismo
día le vi paseando muy cabizbajo por el patio grande, y la mañana
siguiente me detuvo en el mismo paraje suplicándome que llevase una
carta a la señora duquesa. Negueme a esto, y allí quedó. Indudablemente
algo le pasaba al señor de Mañara.

Amaranta pareció muy contrariada de que no se sentase a la mesa el joven
mencionado. Cuando volví con la respuesta estaba de visita en el cuarto
de Amaranta un caballero de los que la noche anterior vi en la procesión
descrita. Conferenciaron más de hora y media: cuando él se retiró le
examiné bien, y por cierto que pocas veces he visto facha más
desagradable. No le daría un puesto en la serie de mis recuerdos, si
aquél no fuera uno de los personajes más célebres de su tiempo, razón
por la cual me resuelvo no sólo a mencionarle, sino a describirle, para
edificación de los tiempos presentes. Era el marqués Caballero, ministro
de Gracia y Justicia.

No vi a semejante hombre más que una vez, y jamás lo he olvidado. Era de
edad como de cincuenta años, pequeño y rechoncho de cuerpo, turbia y
traidora la mirada de uno de sus ojos, pues el otro estaba cerrado a
toda luz; con el semblante amoratado y granulento como de persona a
quien envilece y trastorna el vino; de andar y gestos sumamente
ordinarios: en tanto grado repugnante y soez toda su persona, que era
preciso suponerle dotado de extraordinarios talentos para comprender
cómo se podía ser ministro con tan innoble estampa. Pero no, señores
míos. El marqués Caballero era tan despreciable en lo moral como en lo
físico, pudiendo decirse que jamás cuerpo alguno encarnó de un modo tan
fiel los ruines sentimientos y bajas ideas de un alma. Hombre nulo,
ignorante, sin más habilidad que la intriga, era el tipo del leguleyo
chismoso y tramoyista que funda su ciencia en conocer no los principios,
sino los escondrijos, las tortuosidades y las fórmulas escurridizas del
derecho, para enredar a su antojo las cosas más sencillas.

Nadie podía explicarse su encumbramiento tanto más enigmático, cuanto
que el omnipotente Godoy no pasaba por amigo suyo, mas debió aquél
consistir en que, habiéndose introducido en palacio y héchose valer,
merced a viles intrigas de escalera abajo, usó como instrumento de su
ambición cerca del Rey, la Iglesia; y adulando la religiosidad del pobre
Carlos, pintándole imaginarios peligros y haciendo depender la seguridad
del trono de la adopción de una política restrictiva en negocios
eclesiásticos, logró hacerse necesario en la corte. El mismo Godoy no
pudo apartarle del gobierno ni poner coto a las medidas dictadas por el
bestial fanatismo del ministro de Gracia y Justicia, quien después de
haber perseguido a muchos ilustres hombres de su época, y encarcelado a
Jovellanos, remató su gloriosa carrera contribuyendo a derribar al mismo
Príncipe de la Paz, en marzo de 1808.

Damos estas ligeras noticias respecto a un hombre que gozaba entonces de
justa y general antipatía, para que se vea que la elevación de tontos y
ruines y ordinarios, no es, como algunos creen, desdicha peculiar de los
modernos tiempos.

Después de la conferencia indicada, principió la comida, que yo serví.

---Ya sé---dijo Amaranta al sentarse y sin disimular su intención de
mortificar a Lesbia;---ya sé lo que contenían esos papeles cogidos a Su
Alteza. Caballero me lo ha dicho, encargándome la reserva; pero puesto
que pronto se ha de saber\ldots{}

---Sí, dínoslo. No lo confiaremos más que a nuestros amigos---indicó la
marquesa.

---Pues yo opino que no se diga---objetó el diplomático, que siempre se
incomodaba cuando alguien revelaba secretos que él no conocía.

---Entre los papeles---dijo Amaranta,---hay una exposición al Rey que se
supone hecha por don Juan Escóiquiz, aunque la letra es de Fernando.
Parece que en ella se pintan las malas costumbres del Príncipe de la
Paz, con las frases más indecentes. Allí han salido a relucir sus dos
mujeres y también lo que dicen de los destinos, pensiones y prebendas
que concede a cambio de\ldots{}

---¡Y tan cierto como es!---dijo la marquesa.---Yo sé de un señor a
quien el Príncipe de la Paz ofreció\ldots{}

La buena señora cayó en la cuenta de que estaba yo delante, y se
contuvo. Pero a mí siempre me han bastado pocas palabras para entender
las cosas, y supe pescar al vuelo lo que querían decir.

---En esa exposición---continuó la duquesa,---ponen a la pobre Tudó de
vuelta y media, y aconsejan al Rey que la encierre en un castillo. Por
último, se pretende que el de la Paz sea destituido, embargados todos
sus bienes, y que desde el mismo momento no se separe el Príncipe
heredero del lado de su padre.

---Todo eso está muy puesto en razón---dijo la marquesa, asombrada de
cómo concordaban las ideas de los conjurados con sus propias
ideas;---aunque me guardaré muy bien de decirlo fuera de aquí.

---Pues aquí no temo decirlo---continuó Amaranta.---Caballero no guarda
muy bien el secreto, sé que lo ha dicho ya a varias personas. Otro de
los papeles es graciosísimo, y parece un sainete; pues todo él está en
diálogo y se creería que lo habían escrito para representarlo en el
teatro. Cada uno de los personajes que hablan tiene allí nombre
supuesto: así es que el Príncipe se llama \emph{Don Agustín}; la reina,
\emph{Doña Felipa}; el Rey, \emph{Don Diego}; Godoy, \emph{Don Nuño}, y
la princesa con quien dicen han tratado de casar al heredero es una tal
\emph{Doña Petra}.

---¿Y qué objeto tiene esa comedia?

---Es un proyecto de conversación con la reina, y suponiendo las
observaciones que ésta ha de hacer, se le responde a todo según un plan
combinado para convencerla de las picardías del Príncipe de la Paz.
También aquí abundan las frases soeces, y por último, el Don Agustín
parece que se niega redondamente a casarse con Doña Petra, la cuñada del
ministro y hermana del cardenal y de la de Chinchón.

---También eso está bien pensado---dijo la marquesa;---y si ese
sainetillo se representara yo lo aplaudiría. Pues ¿por qué han de querer
casar al pobre muchacho con la cuñada del otro? ¿No es mejor que le
busquen mujer en cualquiera de las familias reinantes, que a buen seguro
todas ellas se darían con un canto en los pechos por entroncar con
nuestros reyes, casando a cualquiera de sus mozuelas con semejante
príncipe?

---¿Cómo se atreven ustedes a juzgar cosas tan graves?---dijo con
displicencia el diplomático.---Y en cuanto a los documentos citados,
extraño que una persona tan discreta como mi sobrina les dé publicidad
imprudente.

---Vamos, usted dudaba antes que existieran, y ahora, creyendo que no
debe revelarse, los da como ciertos.

---Sí que los doy---repuso el diplomático,---y ya que otra persona ha
descubierto hechos que yo me obstinaba en callar\ldots{}

El diplomático, no pudiendo negar aquellos secretos, resolvió
apropiárselos, fingiendo tener ya noticias de los papeles del proceso.

---¿De modo que ya tú lo sabías todo?---le preguntó su hermana.---Bien
decía yo que tú no podías menos de estar al tanto de estas cosas. La
verdad es que no se te escapa nada, y bien puedes afirmar que eres de
los que ven los mosquitos en el horizonte.

---Desgraciadamente así es---contestó el diplomático con la mayor
hinchazón.---Todo llega a mis oídos, a pesar de mis repetidos propósitos
de no intervenir en nada y huir de los negocios. ¡Cómo ha de ser! Es
preciso tener paciencia.

---Hermano, tú debes saber algo más, y te lo callas---dijo la
marquesa.---Vamos a ver. ¿Napoleón tiene alguna parte en este negocio?

---¿Ya comienzan las preguntillas?---repuso el viejo con retozona
sonrisa.---Déjense de preguntas, porque les juro que no me han de sacar
una sílaba. Ya conocen la rigidez de mi carácter en estas materias.

A todas éstas Lesbia no decía una palabra.

---Pues voy a acabar mi cuento---añadió mi ama.---Aún me falta decir
cuál es el otro papel que se encontró al Príncipe.

---Más valdría que lo callaras, querida sobrina---dijo el diplomático.

---No; que lo diga, que lo diga.

---Pues se ha encontrado la cifra y clave de la correspondencia que el
heredero sostiene con su maestro don Juan Escóiquiz, y además\ldots{}
esto es lo más grave.

---Sí: lo más grave---indicó el diplomático,---y por eso debe callarse.

---Por lo mismo debe decirse.

---Pues se encontró una carta en forma de nota, sin sobrescrito, firma,
ni nombre, en que manifiesta estar dispuesto a elevar al rey la
exposición por medio de un religioso. Lo más notable de este papelito es
que el Príncipe asegura que está decidido a tomar por modelo al Santo
mártir Hermenegildo; que se dispone a pelear\ldots{} óiganlo ustedes
bien\ldots{} a pelear por la justicia. Esto es hablar clarito de una
revolución. Pide después a los conjurados que le sostengan con firmeza;
que preparen las proclamas, y que\ldots{}

---¡Ah, las mujeres!, ¡las mujeres! ¿No aprenderán nunca a tener
discreción? ---interrumpió el marqués.---Me admiro de ver con cuánta
frivolidad te ocupas de asuntos tan peligrosos.

---En este papel---prosiguió la condesa sin atender a las fastidiosas
amonestaciones del diplomático,---se indica a los reyes y a Godoy con
nombres godos. \emph{Leovigildo} es Carlos IV, la reina es
\emph{Goswinda}, y el de la Paz, \emph{Sisberto}. Pues bien: el
Príncipe, que se atribuye el papel de \emph{San Hermenegildo}, dice a
los conjurados que la tempestad debe caer sobre \emph{Sisberto} y
\emph{Goswinda}, y que traten de embobar a \emph{Leovigildo} con vítores
y palmadas.

---¿Y eso es todo?---preguntó la marquesa.---Pues no hay cosa más
inocente.

---Está bien claro---indicó Amaranta con ira,---que se trata de
destronar a Carlos IV.

---No lo veo yo así.

---Pues yo sí---repuso la condesa.---La tempestad debe caer sobre
\emph{Sisberto} y \emph{Goswinda}. De modo que el heredero y sus amigos,
no sólo tratan de mandar a paseo al guardia, sino que también quieren
hacer alguna picardía con la reina, cuando menos llevarla a la
guillotina como a la pobre María Antonieta. Todos saben cuánto ama el
Rey a su esposa. Cualquier ofensa que a ésta se le haga, la considera
como hecha a su propia persona.

---Pues lo que digo es que si algo les pasa, bien merecido se lo
tienen---fue la contestación de la marquesa.

---Y yo sostengo---añadió mi ama alterándose más,---que el Príncipe
podía haber intentado cuantas conjuraciones quisiera para echar del
ministerio a Godoy; pero escribir exposiciones al Rey, poniendo en duda
el honor de su madre, y hablando de arrojar tempestades sobre Sisberto y
Goswinda, lo cual equivale a atentar contra la vida de la Reina, me
parece conducta muy indigna de un Príncipe español y cristiano\ldots{}
Al fin es su madre: cualesquiera que hayan sido las faltas de ésta (y yo
estoy segura de que no son tantas ni tan grandes como las de quien las
publica), no es propio de un hijo el reconocerlas o mencionarlas, ni
menos fundarse en ellas para perseguir a un enemigo.

---Hija, no estás poco melindrosa---dijo con acrimonia la tía de
Amaranta.---Yo creo que el Príncipe hace muy retebién, y si a alguien le
pesa, más valiera no haber dado motivos con lo que todos sabemos a lo
que está pasando. Y si no, hermano, tú que lo sabes todo, dinos tu
opinión.

---¡Mi opinión! ¿Creéis que es fácil dar opinión sobre asunto tan
espinoso? Y lo que yo pueda pensar, conforme a mi experiencia y luces,
¿puedo acaso decirlo en conferencia de mujeres, que al punto van
diciendo por cámaras y antecámaras a todo el que las quiera oír\ldots?

---No hay quien te saque una palabra. Si yo supiera la mitad de lo que
tú sabes, hermano, gustaría de instruir ignorantes.

---Para formar exacto juicio, vengan datos---dijo el marqués.---¿Alguna
de ustedes sabe la opinión de la Reina sobre estas cosas?

---Cuando se leyó en consejo el último de los papeles que he
citado---respondió la condesa,---Caballero dijo que el Príncipe merecía
la pena de muerte por siete capítulos. La Reina, indignada al oírle,
respondió: \emph{¿Pero no reparas que es mi hijo? Yo destruiré las
pruebas que le condenan; le han engañado, le han perdido}; y arrebatando
el papel lo escondió en su seno, y se arrojó llorando en un sillón.
¡Vean ustedes qué generosidad! Francamente, aunque nunca me ha sido
simpática la causa del Príncipe, desde que sé sus proyectos contra los
Reyes, me parece un joven digno de lástima, si no de otro sentimiento
peor.

---¡Qué tontería!---exclamó la marquesa.---Ahora vienen los lloriqueos y
los dengues después de haber sido causa de tantos males. ¿Pues qué,
ocurrirían estas cosas, si no se hubieran cometido ciertas
faltas?\ldots{}

Lesbia, que hasta entonces había permanecido en silencio, con cierta
confusión y amilanamiento, no quiso callar más, y apoyó las últimas
frases de la marquesa. Amaranta entonces se volvió a ella, y con acento
tan amargo como desdeñoso, le dijo:

---¡Cuánto hablar de faltas ajenas! Esa persona no esperaba ser
injuriada públicamente, como lo ha sido, por quien tantos favores
recibió de ella, por quien se ha sentado a su mesa y se ha honrado con
su amistad.

---¡Ah!, el sermoncito no está mal---dijo Lesbia con esa forzada
jovialidad, que a veces es la más terrible expresión de la ira.---Ya lo
esperaba: desde que me negué a ciertas condescendencias; desde que
cansada de un papel admitido con ligereza e impropio de mí, lo cedí a
otras, que lo desempeñan con perfección, se me censura suponiéndome
divulgadora de lo que todo el mundo sabe. Ciertas personas no pueden
hacerse pasar por víctimas de la calumnia aunque lloren y giman, porque
sus vicios, en fuerza de ser tantos y tan grandes, han llegado a
vulgarizarse.

---Es verdad---repuso Amaranta con perversa intención.---No falta quien
sea prueba viva de ello. Pero hija, el vicio más feo es el de la
ingratitud.

---Sí, pero ese es el vicio en que menos fácilmente pueden sentenciar
los hombres.

---¡Oh! no: también sentencian, y pronto lo veremos. Precisamente la
causa del Príncipe es obra pura y simplemente consumada por la
ingratitud. Ya verás cómo ésta se castiga.

---Supongo---dijo Lesbia con malicia,---que no querrás poner en la
cárcel a todos los que estamos aquí por haber cometido el crimen de
desear el triunfo del Príncipe.

---Yo no pongo a nadie en la cárcel; pero quizás no esté muy segura otra
persona muy amada de alguien que me escucha.

---¡Ah!---dijo imprudentemente el diplomático,---me han dicho que
también Mañara está complicado en la causa.

---Creo que sí---añadió Amaranta cruelmente;---pero fía mucho en el
arrimo de elevadas personas. Y como resulten complicadas las que se
sospecha, es de esperar que no les valga ninguna clase de apoyo.

---Eso es---dijo la duquesa.---¡Duro en ellos! Falta todavía conocer el
giro que tomará este negocio; falta saber si algún suceso inesperado
cambiará de improviso los términos convirtiendo a los acusadores en
acusados.

---¡Ya\ldots{} confían en Bonaparte!---afirmó Amaranta con despecho.

---¡Alto, allá!---exclamó el diplomático;---entran ustedes señoras mías,
en un terreno peligroso.

---Se hará justicia---dijo mi ama,---aunque no como se desea, pues no
será posible descubrirlo. Por ejemplo: hay gran empeño en averiguar
quién se encargaba de transmitir a los conjurados la correspondencia del
Príncipe y hasta ahora no se sabe nada. Hay sospechas de que sea alguna
de las muchas damas intrigantes y coquetuelas que hay en palacio\ldots{}
hasta se han fijado en alguna; pero aún no hay suficientes pruebas.

Lesbia no dijo una palabra; pero la pícara se sonreía como quien está
libre de todo temor. Después hasta se atrevió a mortificar a su enemiga
de esta manera:

---Quizás por lo mismo que es intrigante y coquetuela, tenga medios para
burlar a sus perseguidores. Tal vez las circunstancias le hayan
proporcionado medios de desafiar y provocar a sus enemigos\ldots{} Tengo
deseos de saber quién es esa buena pieza. ¿Nos lo podrías decir?

---Ahora no---repuso mi ama;---pero mañana, tal vez sí.

Lesbia rió a carcajadas. Amaranta mudó de conversación, la marquesa
volvió a lamentar la suerte del Príncipe, y el diplomático aseguró que
por nada del mundo descorrería el velo que ocultaba los designios del
capitán del siglo, con lo cual dio fin la comida, y todos, menos mi ama,
se retiraron a dormir la siesta.

\hypertarget{xv}{%
\chapter{XV}\label{xv}}

Al siguiente día, 30 de octubre, ocurrieron grandes y conmovedoras
novedades, si algo podía ya ocurrir capaz de aumentar la turbación de
los ánimos. Desde por la mañana me había despedido mi ama, diciéndome
que fuera a dar un paseo por la octava maravilla del mundo, y al mismo
tiempo me mandó visitase en su celda al padre jerónimo que había de
instruirme en las letras sagradas y profanas. Ambas cosas me contentaron
mucho y más que nada, el ocio de que disfrutaba para recorrer a mi
antojo el edificio y sus alrededores. El primer espectáculo que se
ofreció a mi curiosidad fue la salida del Rey a caza, lo cual no dejó de
causarme extrañeza, pues me parecía que atribulado y pesaroso Su
Majestad por lo que estaba pasando, no tendría humor para aquel alegre
ejercicio. Pero después supe que nuestro buen monarca le tenía tan viva
afición, que ni en los días más terribles de su existencia dejó de
satisfacer aquella su pasión dominante, mejor dicho, su única pasión.

Yo le vi salir por la puerta del Norte, acompañado de dos o tres
personas, entrar en su coche y partir hacia la Sierra, con tanta
tranquilidad como si en palacio dejase la paz más perfecta. Sin duda
debía de ser en extremo apacible su carácter, y tener la conciencia más
pura y limpia que los frescos manantiales de aquellas montañas. Sin
embargo, aquel buen anciano, a pesar de su alta posición y de la paz que
yo suponía en su interior, más me inspiraba lástima que envidia. Aquélla
se aumentó cuando vi que la gente del pueblo, reunida en torno al
edificio, no mostraba a su Rey ningún afecto, y hasta me pareció oír en
algunos grupos murmullos y frases malsonantes, que hasta entonces creo
no se habían aplicado a ningún soberano de esta honrada nación.

Recorriendo después las galerías bajas del palacio y las antecámaras
altas, vi a otros individuos de la regia familia, y me maravilló
observar en todos la misma forma de narices colgantes, que caracterizaba
la casta de los Borbones. El primero que tuve ocasión de admirar fue el
cardenal de la Escala, don Luis de Borbón, célebre después por haber
recibido el juramento de los diputados en la isla de León, y por otros
hechos menos honrosos que irán saliendo a medida que avancen estas
historias. No era el señor cardenal hombre grave, cubierto de canas,
prenda natural de la edad y del estudio, ni representaba su rostro
aquella austeridad que parece ha de ser inherente a los que desempeñan
cargos tan difíciles: antes bien era un jovenzuelo que no había llegado
a los treinta años, edad en la cual Lorenzana, Albornoz, Mendoza,
Silíceo y otras lumbreras de la Iglesia española no habían aún salido
del seminario.

Verdad es que existía la costumbre de consagrar al cardenalato a los
príncipes menores que no podían alcanzar ningún reino grande ni chico, y
el señor don Luis de Borbón, primo del rey Carlos IV, fue en esto uno de
los mortales más afortunados, porque con la leche en los labios empezó a
disfrutar las rentas de la mitra de Sevilla, y no cumplidos aún los 23,
y mal digeridas las Sentencias de Pedro Lombardo, tomó posesión de la
silla de Toledo, cuyas fabulosas rentas habría envidiado cualquier
príncipe de Alemania o de Italia.

Pero cada cosa a su tiempo y los nabos en Adviento. Lo que hemos dicho
era costumbre propia de la edad, y no es justo censurar al infante
porque tomase lo que le daban. Su eminencia, tal y como le vi descender
del coche en el vestíbulo de palacio, me pareció un mozo coloradillo,
rubicundo, de mirada inexpresiva, de nariz abultada y colgante, parecida
a las demás de la familia, por ser fruto del mismo árbol, y con tan
insignificante aspecto, que nadie se fijara en él si no fuera vestido
con el traje cardenalicio. Don Luis de Borbón subió con gran priesa a
las habitaciones regias, y ya no le vi más.

Pero mi buena estrella, que sin duda me tenía reservado el honor de
conocer de una vez a toda la familia real, hizo que viera aquel mismo
día al infante don Carlos, segundo hijo de nuestro Rey. Este joven aún
no aparentaba veinte años, y me pareció de más agradable presencia que
su hermano el príncipe heredero. Yo le observé atentamente, porque en
aquella época me parecía que los individuos de sangre real habían de
tener en sus semblantes algo que indicase la superioridad; pero nada de
esto había en el del infante don Carlos, que sólo me llamó la atención
por sus ojos vivarachos y su carita de Pascua. Este personaje varió
mucho con la edad en fisonomía y carácter.

También vi aquella misma tarde en el jardín al infante don Francisco de
Paula, niño de pocos años que jugaba de aquí para allí, acompañado de mi
Amaranta y de otras damas; y por cierto que el Infante, saltando y
brincando con su traje de mameluco completamente encarnado, me hacía
reír, faltando con esto a la gravedad que era indispensable cuando se
ponía el pie en parajes hollados por la regia familia.

Antes de bajar al jardín habían llamado mi atención unos recios golpes
de martillo que sentí en las habitaciones inferiores: después sucedieron
a los golpes unos delicados sones de zampoña, con tal arte tañida, que
parecían haberse trasladado al Real sitio todos los pastores de la
Arcadia. Habiendo preguntado, me contestaron que aquellos distintos
ruidos salían del taller del infante don Pascual, quien acostumbraba
matar los ocios de la vida regia alternando los entretenimientos del
oficio de carpintero o de encuadernador con el cultivo del arte de la
zampoña. Yo me admiré de que un príncipe trabajase, y me dijeron que el
don Antonio Pascual, hermano menor de Carlos IV, era el más laborioso de
los infantes de España, después del difunto don Gabriel, celebrado como
gran humanista y muy devoto de las artes. Cuando el ilustre carpintero y
zampoñista dejó el taller para dar su paseo ordinario por la huerta del
Prior en com pañía de los buenos padres jerónimos que iban a buscarle
todas las tardes, pude contemplarle a mis anchas, y en verdad digo que
jamás vi fisonomía tan bonachona. Tenía costumbre de saludar con tanta
solemnidad como cortesanía a cuantas personas le salían al paso, y yo
tuve la alta honra de merecerle una bondadosa mirada y un movimiento de
cabeza que me llenaron de orgullo.

Todos saben que don Antonio Pascual, que después se hizo célebre por su
famosa despedida del valle de Josafat, parecía la bondad en persona.
Confieso que entonces aquel príncipe casi anciano, cuya fisonomía se
habría confundido con la de cualquier sacristán de parroquia, era, entre
todos los individuos de la regia familia, el que me parecía de mejor
carácter. Más tarde conocí cuánto me había equivocado al juzgarle como
el más benévolo de los hombres. María Luisa, que le tachó de cruel, en
una de sus cartas profetizó lo que había de pasar a la vuelta de
Valencey, cuando el infante congregaba en su cuarto lo más florido del
partido realista furibundo.

Este pobre hombre, lo mismo que su sobrino el infante don Carlos, eran
partidarios del Príncipe Fernando, y aborrecían cordialmente al de la
Paz; mas excusadas son estas advertencias, porque entonces ningún
español amaba a Godoy, empezando por los individuos de la familia. Pero
basta de digresiones, y sigamos contando. Quedé, si mal no recuerdo, en
el anuncio de ciertas novedades que dieron inesperado giro a los
sucesos; mas no dije cuáles fueran. Parece que a eso de la una el
ilustre prisionero, luego que se enteró de que su padre había salido a
caza, mandó a la Reina un recado suplicándole que fuese a su cuarto,
donde le revelaría cosas muy importantes. Negose la madre; pero envió al
marqués Caballero, quien recogió de labios del Príncipe las
declaraciones de que voy a hablar.

No crean ustedes que tan estupendas nuevas eran del dominio de todos los
habitantes de El Escorial. Yo las supe porque Amaranta las contó al
diplomático y a su hermana, y como por mi poca edad y aspecto de mozuelo
distraído y casquivano, creían que yo no había de prestar atención a sus
palabras, no se cuidaban de guardar reserva delante de mí.

Conforme dijo Amaranta, todas las personas reales andaban azoradas y
aturdidas porque, según las últimas declaraciones del Príncipe, se sabía
ya con certeza que los conjurados tenían de su parte a Napoleón en
persona, cuyas tropas se acercaban cautelosamente a Madrid con objeto de
apoyar el movimiento. También había denunciado Fernando a sus cómplices,
llamándoles pérfidos y malvados; y según las indicaciones que hizo, los
rumores tiempo ha propalados sobre proyecto de atentar a la vida de la
Reina, no carecían de fundamento. En cuanto al Rey, los amigos del
Príncipe no debían de tener muy buenas intenciones respecto a él, porque
éste había nombrado generalísimo de las tropas de mar y tierra al duque
del Infantado en un decreto que empezaba así: «\emph{Habiendo Dios
tenido a bien llamar para sí el alma del Rey, nuestro padre}, etc.»

No se fijaron bien en mi imaginación estos pormenores; pero habiendo
leído más tarde los incidentes de aquel proceso célebre, puedo auxiliar
mi memoria con tanta eficacia que resulte la narración de los hechos tan
viva como hija del recuerdo. Lo que sí me acuerdo es que Amaranta,
alarmada con lo de Bonaparte, tenía gran placer en hacer consideraciones
sobre la bajeza del Príncipe al denunciar vilmente a sus amigos. La
marquesa se resistía a creerlo, y los comentarios, que no copio por no
ser molesto, duraron mucho tiempo.

No había aún oscurecido cuando volvió el Rey de caza, y hora y media
después un gran ruido en la parte baja del alcázar nos anunció la
llegada de otro importante personaje. Corrí al patio grande y ya no pude
verle, porque habiendo descendido rápidamente del coche, subió por la
escalera con prisa de llegar pronto arriba. Únicamente se distinguía un
bulto arrebujado en anchísima capa como persona enferma que quiere
preservarse del aire; mas no me fue posible ver sus facciones.

---Es él---dijeron algunos criados que había junto a mí.

---¿Quién?---pregunté con viva curiosidad.

Entonces un pinche de la cocina, con quien había yo trabado cierta
amistad por ser el funcionario encargado de darme de comer, acercó su
boca a mi oído, y me dijo muy quedamente:

---El \emph{choricero}.

Más adelante tuve ocasión de hablar con este personaje; pero su pintura
pertenece a otro libro.

\hypertarget{xvi}{%
\chapter{XVI}\label{xvi}}

Seguí hablando con el pinche, por no perder tan buena coyuntura de
trabar relaciones con la gente de escalera abajo, y pregunté a mi
abastecedor cuál era la opinión más extendida en las reales cocinas
sobre los sucesos del día. Afortunadamente se aproximaba la hora de
cenar; y llevándome mi amigo al aposento destinado al efecto, me hizo
ver que el cuerpo de cocineros seguía a todo el país en la senda trazada
por los directores del partido fernandista.

Nada más patriótico, nada más entusiasta que la actitud de aquel puñado
de valientes en cuyas cacerolas estaba por decirlo así el paladar de los
reyes de España, y era árbitro hasta cierto punto de su bienestar, si no
de su existencia. Aunque muchos de los hombres que allí vi eran antiguos
y pacíficos servidores, que no participaban de la rebelde inquietud de
la gente moza, la mayor parte habían sido deslumbrados por la perruna y
grotesca elocuencia de Pedro Collado, el aguador de la fuente del Berro,
ya empleado en la servidumbre de Fernando. Este hombre, que con las
gracias de su burdo y ramplón ingenio se había conquistado preferente
lugar en el corazón del heredero, desempeñaba al principio las funciones
de espía en todas las regiones bajas de palacio; vigilaba la
servidumbre, la cual a poco empezó por temerle y concluyó por someterse
dócilmente a sus mandatos. De este modo llegó a ser Pedro Collado,
respecto a los cocineros, pinches y lacayos un verdadero cacique, al
modo de los que hoy son alma y azote de las pequeñas localidades en
nuestra Península.

Cuando Pedro Collado bajaba contento, el regocijo se difundía como don
celeste entre toda la servidumbre: cuando Pedro Collado bajaba taciturno
y sombrío, melancólico silencio sustituía a la anterior algazara. Cuando
alguno perdía la gracia del aguador, ya podía encomendarse a Dios, y los
que tenían la suerte de merecer su benevolencia o de servir de objeto a
sus groseras bromas, ya podían considerarse con un pie puesto en la
escala de la fortuna.

Aquella noche fue para mí muy interesante porque presencié la prisión de
Pedro Collado, contra quien habían resultado cargos muy graves en las
primeras actuaciones de la causa. El favorito del Príncipe comunicaba a
los más autorizados entre sus amigos las impresiones del día, cuando un
alguacil, seguido de algunos soldados de la guardia española, entró a
prenderle. No hizo resistencia el aguador, antes bien con la frente
erguida y provocativo ademán, siguió a sus guardianes que le condujeron
a la cárcel del Sitio, porque a causa de su baja condición no podía
alternar con el duque de San Carlos, ni con el del Infantado, presos en
las bohardillas de la parte del edificio llamado del Noviciado.

La prisión del aguador produjo en la cocina cierto terror y sepulcral
silencio. Interrumpiéronlo después las voces de mando, que cual la de
los generales en la guerra, sirven para dirigir la estrategia de las
cocinas reales, no menos complicada que la de los campos de batalla. Una
voz decía: «Cena del señor infante don Antonio Pascual.» Y al punto la
más rica menestra que ha incitado el humano apetito pasó a manos de los
criados que servían en el cuarto del infante. Después se oyó la
siguiente orden: «La sopa hervida y el huevo estrellado de la señora
infanta doña María Josefa.» Luego «El chocolate del señor infante don
Francisco de Paula,» y nuevos movimientos seguían a estas palabras. Hubo
un instante de sosiego, hasta que el cocinero mayor exclamó con voz
solemne: «¿Está la polla asada de su eminencia el señor cardenal?» Al
instante funcionaron las cacerolas, y la polla asada con otros
sustanciosos acompañamientos fue transmitida al cuarto del arzobispo.
Por último, un señor muy obeso y vestido de uniforme con galones, que
era designado con el estrambótico nombre de \emph{guardamangier}, se
paró en la puerta y diri giendo su mirada de águila hacia los cocineros,
exclamó: «La cena de Su Majestad el Rey.» Era cosa de ver la multitud de
platos que se destinaron a aliviar la debilidad estomacal diariamente
producida en la naturaleza de Carlos IV por el ejercicio de la caza.
Como yo no podía apartar mis ojos de aquella rica colección de manjares,
cuyo aromático vapor convidaba a comer, mi amigo el pinche me dijo:

---Descuida, Gabrielillo, que ya probaremos algo de aquellos platos. Al
Rey le gusta ver muchos platos en su mesa; pero de cada uno no come más
que un poquito. Algunos vuelven como han ido. Voy a preparar el agua
helada.

---¿Qué es eso de agua helada?---pregunté.---¿Y quién se alimenta con
manjar de tan poca sustancia?

---El Rey---me contestó,---una vez que llena bien el buche, pide un vaso
de agua helada como la misma nieve; coge un panecillo, le quita la
corteza, empapa bien la miga en el agua, y se la come después. Jamás
toma más postre que ése.

Un buen rato después de haberse pedido la cena del Rey, pidieron la de
la Reina, y esta diferencia de tiempo llamó tanto mi atención, que
pregunté a mi amigo la razón de que no comieran juntos los Reyes y sus
hijos.

---Calla, tonto---me dijo,---eso no puede ser. En las casas de todo el
mundo, comen padres e hijos en una misma mesa. Pero aquí no: ¿no ves que
eso sería faltar a la etiqueta? Los infantes comen cada uno en su
cuarto, y Su Majestad el Rey solo en el suyo, servido por los guardias.
La Reina es la única persona que podría comer con el Rey, pero ya sabes
que acostumbra comer sola, por lo que callo.

---¿Por qué?, dímelo a mí. Es que tendrá alguna persona que la acompañe
\emph{de ocultis}.

---Quiá: no come delante de alma viviente ni que la maten.

---¿Ni tampoco delante de sus damas?

---Sólo la camarera que la sirve la ve comer. Te diré por qué---añadió
en voz baja.---¿Ves aquellos dientes tan bonitos que enseña la Reina
cuando se ríe? Pues son postizos, y como tiene que quitárselos para
comer, no quiere que la vean.

---Eso sí que está bueno.

En efecto, lo que me dijo el pinche era cierto, y en aquellos tiempos el
arte odontálgico no había adelantado lo suficiente para permitir las
funciones de la masticación con las herramientas postizas.

---Ya ves tú---continuó el pinche,---si tienen razón los que critican a
la Reina porque engaña al pueblo, haciendo creer lo que no es. ¿Y cómo
ha de hacerse querer de sus vasallos una soberana que gasta dientes
ajenos?

Como yo no creía que las funciones de los reyes fueran semejantes a las
de un perro de presa, no pensé lo mismo que mi amigo, aunque me callé
sobre el particular.

Luego pidieron la cena de Su Alteza el Príncipe de la Paz, y la de los
Consejeros de Estado, lo cual me decidió a subir, creyendo llegada la
hora de servir también la de mi ama. Se acercaba para mí el dulce
momento de verla, de hablarla, de escuchar sus mandatos, de pasar junto
a ella rozando mi vestido con el suyo, de embelesarme con su sonrisa y
con su mirada. Ausente de ella, mi imaginación no se apartaba de tan
hermoso objeto, como mariposa que rodea sin cesar la luz que la fascina.
Pero muy contra mi voluntad, aquella noche Amaranta no se dignó ponerme
al corriente de lo que deseaba saber respecto a mis servicios. Estaba
escrito que fuera a la noche siguiente.

Aunque aún no me había acontecido en palacio nada digno de notarse, yo
estaba un sí es no es descorazonado. ¿Por qué? No podía decirlo.
Encerrado en mi cuarto, y tendido sobre el angosto lecho, rebelde mi
naturaleza al sueño, me puse a pensar en mi situación, en el carácter de
Amaranta que empezaba a parecerme muy raro, y en la clase de fortuna que
a su lado me aguardaba. Acordeme de Inés, a quien por aquellos días
tenía muy olvidada, y cuando su memoria, refrescando mi mente, me
predispuso a un dulce sueño, sentía (no sé si fue engañoso efecto del
sueño) unos golpecitos en mi pecho, producidos por vivas y dolorosas
palpitaciones, como si una mano amiga, perteneciente a persona que
deseaba entrar a toda costa, estuviese tocando a las puertas de mi
corazón.

\hypertarget{xvii}{%
\chapter{XVII}\label{xvii}}

A la siguiente noche, Amaranta me mandó entrar en su cuarto. Estaba con
la misma vestidura blanca de las noches anteriores. Hízome sentar a su
lado en una banqueta más baja que su asiento, de modo que apenas faltaba
un pequeño espacio para que sus rodillas fueran cojín de mi frente. Me
puso la mano en el hombro, y dijo:

---Ahora sabré, Gabriel, si puedo contar contigo para lo que deseo.
Veremos si tus facultades están a la altura de lo que he pensado de ti.

---¿Y usía ha podido dudarlo?---repuse conmovido.---No puedo olvidar lo
que me dijo usía la otra noche, y fue que otros, con menos méritos que
yo, han llegado a subir hasta los últimos escalones de la fortuna.

---¡Ah, pobrecillo!---dijo riendo.---Veo que sueñas con subir demasiado,
y esto es peligroso, porque ya sabes lo de Ícaro. Yo contesté que nada
sabía de ningún señor Ícaro; contome ella la fábula, y luego añadió:

---La historia que te conté la otra noche, no debe servirte de ejemplo,
Gabriel. Después de lo que sabes, he leído un poco más y puedo seguirla.

---Quedó usía en aquello de que el joven de la guardia, a quien la
sultana había hecho gran visir, daba muy mal pago a su protectora, lo
cual me parece una grandísima picardía.

---Pues bien: después he leído que la sultana estaba muy arrepentida de
su liviandad, y que el joven genízaro, hecho príncipe y generalísimo,
era cada vez más aborrecido en todo el imperio. El sultán continuaba tan
ciego como antes, y no comprendía la causa del malestar de sus vasallos.
Pero ella, como mujer de agudo ingenio, conocía la tempestad que
amenazaba descargar sobre la real familia. Sus damas la encontraban
algunas veces llorando. Desahogando su conciencia con alguna, le hizo
ver su arrepentimiento por las faltas cometidas. Mas ya parecía
imposible remediarlas; el descontento de los súbditos era inmenso, y se
formó un grande y poderoso bando, a cuya cabeza se hallaba el hijo mismo
de los sultanes, con objeto de destronarles, proyectando quitarles la
vida, si la vida era un estorbo para sus fines.

---Y el gran visir, ¿qué hacía?

---El gran visir, aunque no era hombre de pocos alcances, no sabía
tampoco qué partido tomar. Todos volvían los ojos al gran Tamerlán,
insigne guerrero y conquistador, que había enviado sus tropas a aquel
imperio como paso para un pequeño reino que deseaba conquistar. En él
creían ver un salvador el padre y el hijo y la sultana y el gran visir;
mas como no es posible que el gran Tamerlán les favorezca a todos a un
tiempo, es seguro que alguno ha de equivocarse.

---Y por último, ¿a quién favoreció ese señor guerrero?

---Eso está en el final de la historia que no he leído
todavía---contestó Amaranta;---pero creo que no tardaré en conocer el
desenlace, y entonces podré contártelo.

---Pues digo y repito, que si el gran visir hubiera gobernado bien a los
pueblos, como los gobernaría quien yo me sé, nada de eso habría pasado.
Haciendo justicia como Dios manda, esto es, castigando a los malos y
premiando a los buenos, es imposible que el imperio hubiese venido a
tales desdichas.

---Pero eso ahora no nos importa gran cosa---dijo Amaranta,---y vamos a
nuestro asunto.

---Sí señora---respondí con calor;---¿qué importan todos los imperios
del mundo?

Al decir esto, creyendo que mis palabras eran frigidísima expresión de
lo que yo sentía, crucé las manos en la actitud más patética que me fue
posible, y dando rienda suelta a la ardorosa exaltación que inflamaba mi
cabeza, la expresé en palabras como mejor pude, exclamando así:

---¡Ah, señora condesa! Yo no sólo os respeto como el más humilde de
vuestros criados, sino que os adoro, os idolatro, y no os enojéis
conmigo si tengo el atrevimiento de decíroslo. Arrojadme de vuestro
lado, si esto os desagrada, aunque con esto conseguiríais hacer de mí un
muchacho desgraciado, pero de ningún modo que dejase de amaros.

Amaranta se rió de mis aspavientos y habló así:

---Bueno, me gusta tu adhesión. Veo que podré contar contigo. En cuanto
a tus cualidades intelectuales también las creo atendibles. Pepa me ha
encomiado mucho tu facultad de observación. Parece que tienes una
extraordinaria aptitud para retener en la memoria los objetos, las
fisonomías, los diálogos y cuanto impresiona tus sentidos, pudiendo
referirlo después puntualísimamente. Esto, unido a tu discreción, hace
de ti un mozo de provecho. Si a tantas prendas se añade el respeto y
amor a mi persona, de tal modo que lo sacrifiques todo a mí y a nadie
revelas lo que hagas en mi servicio\ldots{}

---¡Yo revelar, señora! Ni a mi sombra, ni a mis padres, si los tuviera;
ni a Dios\ldots{}

---Además---añadió clavando en mí sus ojos de un modo que me
mareaba,---tú eres un chico que sabe disimular.

---Perfectísimamente.

---Y observas, te enteras de cuanto hay alrededor tuyo\ldots{} todo sin
excitar sospechas.

---Estoy seguro de poseer todas esas cualidades.

---Pues lo primero que has de hacer cuando volvamos a Madrid, es ponerte
al servicio de tu antigua ama.

---¿Cómo? ¿De mi antigua ama?

---Tonto, eso no quiere decir que dejes de servirme a mí. Al contrario,
irás todas las noches a casa, donde nos veremos. Aunque no en
apariencia, en realidad estarás siempre a mi servicio, y te recompensaré
liberalmente.

---De modo que si sirvo a la cómica es\ldots{}

---Es para evitar sospechas.

---¡Oh! ¡Magnífico!, sí, sí, ya comprendo. Así nadie podrá decir\ldots{}

---Justo. Y en casa de tu ama observarás con muchísima atención lo que
allí pasa, quién entra, quién sale, quién va por las noches, en fin,
todo\ldots{}

---¿Y con qué objeto?---pregunté algo desconcertado, no comprendiendo
por qué me quería convertir en inquisidor.

---El objeto no te importa---contestó mi dueña.---Además (esto es lo
principal), en el teatro has de vigilar cuidadosamente a Isidoro
Máiquez, y siempre que éste te dé alguna carta amorosa para tu ama, me
la traerás a mí primero, y después de enterarme de ella, te la
devolveré.

Estas palabras me dejaron perplejo, y creyendo no haber comprendido bien
su misterioso sentido, roguela que me las explicara.

---Oye bien otra cosa---prosiguió.---Lesbia continúa en relaciones con
Isidoro aunque ama a otro, y yo sé que cuando ella vuelva a Madrid, se
darán cita en casa de la González. Tú observarás todo lo que allí pase,
y si consigues con tu ingenio y travesura, que sí lo conseguirás,
hacerte mensajero de sus amores, y siéndolo, me tienes al tanto de todo,
me harás el mayor servicio que hoy puedo recibir, y no tendrás que
arrepentirte.

---Pero\ldots{} pero\ldots{} no sé cómo podré yo\ldots---dije lleno de
confusiones.

---Es muy fácil, tontuelo. Tú vas al teatro todas las tardes. Procura
que la duquesa te crea un chico servicial y discreto, ofrécete si es
preciso a servirla, haz ver a Isidoro que no tienes precio para llevar
un recado secreto, y los dos te tomarán por emisario de sus amores. En
tal caso, cuando cojas una esquela amorosa del uno o del otro, me la
traes y punto concluido.

---Señora---exclamé sin poder volver de mi asombro;---lo que usía exige
de mí, es demasiado difícil.

---¡Oh! ¡Qué salida! Pues me gusta la disposición del chico. ¿Y aquello
de te amo y te adoro?\ldots{} ¿Pero te has vuelto tonto? Lo que ahora te
mando no es lo único que exijo de ti. Ya sabrás lo demás. Si en esto que
es tan sencillo, no me obedeces, ¿cómo quieres que haga de ti un hombre
respetable y poderoso?

Aún pensaba yo que el papel que Amaranta quería hacerme representar a su
lado no era tan bajo ni tan vil como de sus palabras se deducía, y aún
le pedí nuevas explicaciones que me dio de buen grado, dejándome, como
dice el vulgo completamente aplastado. La proposición de Amaranta me
arrojó desde la cumbre de mi soberbia a la profunda sima de mi
envilecimiento.

No era posible, sin embargo, protestar contra éste, y tenía necesidad de
afectar servil sumisión a la voluntad de mi ama. Yo mismo me había
dejado envolver en aquellas redes; era preciso salir de ellas
escapándome astutamente por una malla rota y sin intentar romperla con
violencia.

---¿Pero cree usía---dije tratando de poner orden en mis ideas,---que en
esa ocupación no perderé la dignidad que, según dicen, debe tener todo
aquel que aspira a ocupar en el mundo una posición honrosa?

---Tú no sabes lo que te dices---me contestó moviendo con donaire su
hermosa cabeza.---Al contrario: lo que te propongo será la mejor escuela
para que vayas aprendiendo el arte de medrar. El espionaje aguzará tu
entendimiento, y bien pronto te encontrarás en disposición de medir tus
armas con los más diestros cortesanos. ¿Tú has pensado que podrías ser
hombre de pro sin ejercitarte en la intriguilla, en el disimulo y en el
arte de conocer los corazones?

---¡Señora---repuse,---qué escuela tan espantosa!

---Es indudable que te pintas solo para observarlo, y que sabes dar
cuenta de cuanto ves de un modo asombroso. Esto, y algo que he notado en
ti, me ha hecho creer que eras un muchacho de facultades. ¿No dices que
tienes ambición?

---Sí, señora.

---Pues para medrar en los palacios no hay otro camino que el que te
propongo. Supongamos que desempeñas satisfactoriamente la comisión
indicada: en este caso volverás a mi lado y serás mi paje. Casi siempre
vivo en palacio; ya ves si tienes ocasión de lucirte. Un paje puede
entrar en muchas partes; un paje está obligado a ser galán de las
doncellas de las camaristas y damas de palacio, lo cual le pone en
disposición de saber secretos de todas clases. Un paje que sepa
observar, y que al mismo tiempo tenga mucha reserva y prudencia, junto
con una exterioridad agradable, es una potencia de primer orden en
palacio.

Tales razones me tenían confundido de tal modo, que no sabía qué
contestar.

---¡Cuántos hombres insignes ves tú por ahí que empezaron su carrera de
simples pajes!, paje fue el marqués Caballero, hoy ministro de Gracia y
Justicia, y pajes fueron otros muchos. Yo me encargaré de sacarte una
ejecutoria de nobleza, con la cual, y mi valimiento, podrás entrar
después en la guardia de la real persona. Ésta sería una nueva faz de tu
carrera. Un paje puede escurrirse tras una cortina para oír lo que se
dice en una sala, un paje puede traer y llevar recados de gran
importancia, un paje puede recibir de una doncella secretos de Estado;
pero un guardia puede aún mucho más, porque su posición es más interior.
Si tiene las cualidades que adornaron al paje, su poder es
extraordinario; puede bienquistarse con damas de la corte, que siempre
son charlatanas, puede hacerse un sinnúmero de amigos en estas regiones,
diciendo aquí lo que oyó más allá, adornando las noticias a su modo y
pintando los hechos como le convenga. Tiene el guardia una ventaja que
no poseen los reyes mismos, y que éstos no conocen más que el palacio en
que viven, razón por la cual casi nunca gobiernan bien, mientras aquél
conoce el palacio y la calle, la gente de fuera y la de dentro, y esta
ciencia general le permite hacerse valer en una y otra parte, y pone en
sus manos un número infinito de resortes. El hombre que los sabe manejar
aquí es más poderoso que todos los poderosos de la tierra, y
silenciosamente, sin que lo adviertan esos mismos que por ahí se dan
tanto tono llamándose ministros y consejeros, puede llevar su influjo
hasta los últimos rincones del reino.

---¡Señora!---exclamé,---¡cuán distinto es todo esto de como yo me lo
había figurado!

---A ti---añadió,---te parecerá que esto no es bueno. Pero así lo hemos
encontrado, y puesto que no está en nuestra mano reformarlo, siga como
hasta aquí.

---¡Ah!, confieso mi necedad---exclamé.---Confieso que, alucinado por mi
disparatada imaginación, tuve locos y ridículos pensamientos, aunque
ahora caigo en que deben ser propios de mi poca edad e ignorancia. Es
verdad que yo creía que tonto y vano y humilde como soy, podría imitar a
otros muchos en su inmerecido encumbramiento. Tanto he oído hablar de la
buena fortuna de algunos necios, que dije: «Pues precisamente todos los
necios deben hacer fortuna.» Pero para conseguir esto, yo me
representaba medios nobles y decentes, y decía: «¿Quién me quita a mí de
llegar a ser lo que otros son? De ellos me diferenciaré en que si algún
día tengo poder, he de emplearlo en hacer bien, premiando a los buenos y
castigando a los malos, haciendo todas las cosas como Dios manda, y como
me dice el corazón que deben hacerse.» Nunca pensé ser hombre de fortuna
de otra manera, y si pensé en la necesidad de hacer algo malo, creí
sería de eso que no deshonra, tal y como desafiarse, amar a una dama en
secreto sin decírselo a nadie, reventar siete caballos por ir de aquí a
Aranjuez para traer una flor, matar a los enemigos del Rey, y otras
cosas por el mismo estilo.

---¡Ah!, esos tiempos pasaron---dijo Amaranta riendo de mi
simplicidad.---Veo que tienes sentimientos elevados; pero ya no se trata
de eso. Tus escrúpulos se irán disipando, cuando a las dos semanas de
estar en mi servicio conozcas las ventajas de vivir aquí. Además, esto
te proporcionará en adelante la satisfacción de hacer el bien a muchos
que lo soliciten.

---¿Cómo?

---¡Oh!, muy fácilmente. Mi doncella ha conseguido en esta semana dos
canonjías, un beneficio simple y una plaza de la contaduría de espolios
y vacantes.

---Pues qué---pregunté con el mayor asombro,---¿las criadas nombran los
canónigos y los empleados?

---No, tontuelo; los nombra el ministro, pero ¿cómo puede desatender el
ministro una recomendación mía, ni cómo he de desatender yo a una
muchacha que sabe peinarme tan bien?

---Un amigo mío, muy respetable, está solicitando desde hace catorce
años un miserable destino, y aún no lo ha podido conseguir.

---Dime su nombre y te probaré que, aun sin quererlo, ya comienzas a ser
un hombre de influencia.

Díjele el nombre del padre Celestino del Malvar, con la plaza que
pretendía, y ella apuntó ambas cosas en un papel.

---Mira---dijo después señalándome sus cartas;---son tantos los negocios
que traigo entre manos, que no sé cómo podré despacharlos. La gente de
fuera ve a los ministros muy atareados, y dándose aire de personas que
hacen alguna cosa. Cualquiera creería que esos personajes cargados de
galones y de vanidad sirven para algo más que para cobrar sus enormes
sueldos; pero no hay nada de esto. No son más que ciegos instrumentos y
maniquíes que se mueven a impulsos de una fuerza que el público no ve.

---Pero el príncipe de la Paz, ¿no es más poderoso que los mismos reyes?

---Sí; mas no tanto como parece. Danle fuerza las raíces que tiene acá
dentro, y como éstas son profundas, como se agarran a una fértil tierra,
como no cesamos de regarlas, de aquí que este árbol frondoso extienda
sus ramas fuera de aquí con gran lozanía. Godoy no debe nada de lo que
tiene a su propio mérito; débelo a quien se lo ha querido dar, y ya
comprendes que sería fácil quitárselo de improviso. No te dejes nunca
deslumbrar por la grandeza de esos figurones a quienes el vulgo admira y
envidia; su poderío está sostenido por hebras de seda, que las tijeras
de una mujer pueden cortar. Cuando hombres como Jovellanos han querido
entrar aquí, sus pies se han enredado en los mil hilos que tenemos
colgados de una parte a otra, y han venido al suelo.

---Señora---dije dominado por amarga pesadumbre,---yo dudo mucho que
tenga ingenio para desempeñar lo que usía me encarga.

---Yo sé que lo tendrás. Ejercítate primero en la embajada que te he
dado cerca de la González; proporcióname lo que necesito, y luego podrás
hacer nuevas proezas. Tú harás de modo que se aficione de ti alguna
persona de palacio: fingirás luego que estás cansado de mi servicio, yo
haré el papel de que te despido, y tú entrarás al servicio de esa otra
persona, con la que alguna vez hablarás mal de mí para que no sospeche
la trama; entretanto, diligente observador de cuanto pase en el cuarto
de tu nueva y aparente ama, lo contarás todo a la antigua y a la
verdadera que seré siempre yo, tu bienhechora y tu Providencia.

Ya me fue imposible oír con calma una tan descarada y cínica exposición
de las intrigas en que era la condesa consumada maestra, y yo catecúmeno
aún sin bautismo. Una elocuente voz interior protestaba contra el vil
oficio que se me proponía, y la vergüenza, agolpando la sangre en mi
rostro, me daba una confusión, un embarazo, que entorpecía mi lengua
para la negativa. Levanteme, y con voz trémula, di a la condesa mis
excusas, diciendo otra vez que no me creía capaz de desempeñar tan
difíciles cometidos. Ella volvió a reír, y me dijo:

---Esta noche, aunque es hora muy avanzada, quizás celebren una
conferencia en este mi cuarto dos personajes, ha tiempo reñidos, y a
quienes yo trato de reconciliar. Hablarán solos, y en tal caso, espero
que tú, escondido tras el tapiz que conduce a mi alcoba, lo oirás todo,
para contármelo después.

---Señora---dije,---me ha entrado de repente un vivísimo dolor de
cabeza; y si usía me permitiera retirarme, se lo agradecería en el alma.

---No---repuso mirando un reloj,---porque tengo que salir ahora mismo, y
es preciso que estés en vela, y aguardes aquí. Volveré pronto.

Esto diciendo llamó a la doncella, pidió su cabriolé, especie de manto
que entonces se usaba; la doncella trajo dos, y envolviéndose cada una
en el suyo, salieron con presteza, dejándome solo.

\hypertarget{xviii}{%
\chapter{XVIII}\label{xviii}}

La situación de mi espíritu era indefinible. Un frío glacial invadió mi
pecho, como si una hoja de finísimo acero lo atravesara. La brusca y
rápida mudanza verificada en mis sensaciones respecto de Amaranta era
tal, que todo mi ser se estremeció, sintiendo vacilar sus ignorados
polos, como un planeta cuya ley de movimiento se trastorna de improviso.
Amaranta era no una mujer traviesa e intrigante, sino la intriga misma,
era el demonio de los palacios, ese temible espíritu por quien la
sencilla y honrada historia parece a veces maestra de enredos y doctora
de chismes; ese temible espíritu que ha confundido a las generaciones,
enemistado a los pueblos, envileciendo lo mismo los gobiernos despóticos
que los libres; era la personificación de aquella máquina interior, para
el vulgo desconocida, que se extendía desde la puerta de palacio, hasta
la cámara del Rey, y de cuyos resortes, por tantas manos tocados,
pendían honras, haciendas, vidas, la sangre generosa de los ejércitos y
la dignidad de las naciones; era la granjería, la realidad, el cohecho,
la injusticia, la simonía, la arbitrariedad, el libertinaje del mando,
todo esto era Amaranta; y sin embargo ¡cuán hermosa!, hermosa como el
pecado, como las bellezas sobrehumanas con que Satán tentaba la castidad
de los padres del yermo, hermosa como todas las tentaciones que
trastornan el juicio al débil varón, y como los ideales que compone en
su iluminado teatro la embaucadora fantasía cuando intenta engañarnos
alevosamente, cual a chiquitines que creen ciertas y reales las figuras
de magia.

Una luz brillante me había deslumbrado; quise acercarme a ella y me
quemé. La sensación que yo experimentaba era, si se me permite
expresarlo así, la de una quemadura en el alma.

Cuando se fue disipando el aturdimiento en que me dejó mi ama, sentí una
viva indignación. Su hermosura misma, que ya me parecía terrible, me
compelía a apartarme de ella. «Ni un día más estaré aquí; me ahoga esta
atmósfera y me da espanto esta gente,» exclamé dando paseos por la
habitación, y declamando con calor, como si alguien me oyera.

En el mismo momento sentí tras la puerta ruido de faldas, y el cuchicheo
de algunas mujeres. Creí que mi ama estaría de vuelta. La puerta se
abrió y entró una mujer, una sola: no era Amaranta.

Aquella dama, pues lo era, y de las más esclarecidas, a juzgar por su
porte distinguidísimo, se acercó a mí, y preguntó con extrañeza:

---¿Y Amaranta?

---No está---respondí bruscamente.

---¿No vendrá pronto?---dijo con zozobra, como si el no encontrar a mi
ama fuese para ella una gran contrariedad.

---Eso es lo que no puedo decir a usted. Aunque sí\ldots{} ahora caigo
en que dijo volvería pronto---contesté de muy mal talante.

La dama se sentó sin decir más. Yo me senté también y apoyé la cabeza
entre las manos. No extrañe el lector mi descortesía, porque el estado
de mi ánimo era tal, que había tomado repentino aborrecimiento a toda la
gente de palacio, y ya no me consideraba criado de Amaranta.

La dama, después de esperar un rato, me interrogó imperiosamente:

---¿Sabes dónde está Amaranta?

---He dicho que no---respondí con la mayor displicencia.---¿Soy yo de
los que averiguan lo que no les importa?

---Ve a buscarla---dijo la dama no tan asombrada de mi conducta como
debiera estarlo.

---Yo no tengo que ir a buscar a nadie. No tengo que hacer más que irme
a mi casa.

Yo estaba indignado, furioso, ebrio de ira. Así se explican mis bruscas
contestaciones.

---¿No eres criado de Amaranta?

---Sí y no\ldots{} pues\ldots{}

---Ella no acostumbra a salir a estas horas. Averigua dónde está y dile
al instante que venga---dijo la dama con mucha inquietud.

---Ya he dicho que no quiero, que no iré, porque no soy criado de la
condesa ---respondí.---Me voy a mi casa, a mi casita, a Madrid. ¿Quiere
usted hablar a mi ama?, pues búsquela por palacio. ¿Han creído que soy
algún monigote?

La dama dio tregua a su zozobra para pensar en mi descortesía. Pareció
muy asombrada de oír tal lenguaje, y se levantó para tirar de la
campanilla. En aquel momento me fijé por primera vez atentamente en
ella, y pude observar que era poco más o menos, de esta manera:

Edad que pudiera fijarse en el primer período de la vejez, aunque tan
bien disimulada por los artificios del tocador, que se confundía con la
juventud, con aquella juventud que se desvanece en las últimas etapas de
los cuarenta y ocho años. Estatura mediana y cuerpo esbelto y airoso,
realzado por esa suavidad y ligereza de andar que, si alguna vez se
observan en las chozas, son por lo regular cualidades propias de los
palacios. Su rostro bastante arrebolado no era muy interesante, pues
aunque tenía los ojos hermosos y negros, con extraordinaria viveza y
animación, la boca la afeaba bastante, por ser de éstas que con la edad
se hienden, acercando la nariz a la barba. Los finísimos, blancos y
correctos dientes no conseguían embellecer una boca que fue airosa si no
bella, veinte años antes.

Las manos y brazos, por lo que de éstos descubría, advertí que eran a su
edad las mejores joyas de su persona y las únicas prendas que del
naufragio de una regular hermosura habían salido incólumes. Nada notable
observé en su traje, que no era rico, aunque sí elegante y propio del
lugar y la hora.

Abalanzose como he dicho a tirar de la campanilla, cuando de improviso,
y antes de que aquélla sonase, se abrió de nuevo la puerta y entró mi
ama. Recibiola la visitante con mucha alegría, y no se acordaron más de
mí, sino para mandarme salir. Retireme, pasando a la pieza inmediata,
por donde debía dirigirme a mi cuarto, cuando el contacto del tapiz,
deslizándose sobre mi espalda al atravesar la puerta, despertó en mí la
olvidada idea de las escuchas y el espionaje que Amaranta me había
encargado. Detúveme, y el tapiz me cubrió perfectamente: desde allí se
oía todo con completa claridad.

Hice intención de alejarme para no incurrir en las mismas faltas que tan
feas me parecían; pero la curiosidad pudo más que todo y no me moví. Tan
cierto es que la malignidad de nuestra naturaleza puede a veces más que
todo. Al mismo tiempo el rencorcillo, el despecho, el descorazonamiento
que yo sentía, me impulsaban a ejercer sobre mi ama la misma pérfida
vigilancia que ella me encomendaba sobre los demás.

---¿No me mandas aplicar el oído?---dije para mí, recreándome en mi
venganza.---Pues ya lo aplico.

La dama desconocida había proferido muchas exclamaciones de desconsuelo,
y hasta me pareció que lloraba. Después, alzando la voz, dijo con
ansiedad:

---Pero es preciso que en la causa no aparezca Lesbia.

---Será muy difícil eliminarla, porque está averiguado que ella era
quien trasmitía la correspondencia---contestó mi ama.

---Pues no hay otro remedio---continuó la dama.---Es preciso que Lesbia
no figure para nada, ni preste declaraciones. No me atrevo a decírselo a
Caballero; pero tú con habilidad puedes hacerlo.

---Lesbia---dijo Amaranta,---es nuestro más terrible enemigo. La causa
del Príncipe ha sido en su vil carácter un pretexto más bien que una
causa para hostilizarnos. ¡Qué de infamias cuenta, qué de absurdos
propala! Su lengua de víbora no perdona a quien ha sido su bienhechora y
también se ensaña conmigo, de quien ha contado horrores.

---Contará lo de marras---repuso la dama de la boca hendida.---Tú
cometiste la gran falta de confiarle aquel secreto de hace quince años,
que nadie sabía.

---Es verdad---dijo mi ama meditabunda.

---Pero no hay que asustarse, hija---añadió la otra.---La enormidad y el
número de las faltas supuestas que nos atribuyen nos sirve de consuelo y
de expiación por las que realmente hayamos cometido, las cuales son tan
pocas, comparadas con lo que se dice, que casi no debe pensarse en
ellas. Es preciso que Lesbia no aparezca para nada en la causa.
Adviérteselo a Caballero; mañana podrían prenderla, y si declara, puede
vengarse mostrando pruebas terribles contra mí. Esto me tiene
desesperada: conozco su descaro, su atrevimiento, y la creo capaz de las
mayores infamias.

---Ella es dueña, sin duda, de secretos peligrosos, y quizás conserve
cartas o algún objeto.

---Sí---respondió con agitación la desconocida.---Pero tú lo sabes todo:
¿a qué me lo preguntas?

---Entonces con harto dolor de mi corazón, le diré a Caballero que la
excluya de la causa. La pícara se jactaba ayer aquí mismo de que no
pondrían la mano sobre ella.

---Ya se nos presentará otra ocasión\ldots{} Dejarla por ahora. ¡Ah!,
bien castigada está mi impremeditación. ¿Cómo fui capaz de fiarme de
ella? ¿Cómo no descubrí bajo la apariencia de su amena jovialidad y
ligereza, la perfidia y doblez de su corazón? Fui tan necia que su
gracia me cautivó; la complacencia con que me servía en todo acabó de
seducirme, y me entregué a ella en cuerpo y alma. Recuerdo cuando las
tres salíamos juntas de palacio en aquella breve temporada que pasamos
en Madrid hace cinco años. Pues después he sabido que una de aquellas
noches, avisó a cierta persona el punto a donde íbamos, para que me
viera, y me vio\ldots{} Nosotros no advertimos nada; no conocimos que
Lesbia nos vendía, y hasta mucho después no descubrí su falsedad por una
singular coincidencia.

---Ese estúpido y presuntuoso Mañara---dijo mi ama,---le ha trastornado
el juicio.

---¡Ah!, ¿no sabes que en el cuerpo de guardia se ha jactado ese
miserable de que ha sido amado por mí, añadiendo que me despreció? ¿Has
visto? ¡Si yo jamás he pensado en semejante hombre, ni creo haber
siquiera reparado en él! ¡Ay, Amaranta! Tú eres joven aún; tú estás en
el apogeo de la hermosura; sírvate de lección. Cada falta que se comete,
se paga después con la vergüenza de las cien mil que no hemos cometido y
que nos imputan. Y ni aun en la conciencia tenemos fuerzas para
protestar contra tantas calumnias, porque una sola verdad entre mil
calumnias, nos confunde, mayormente si nos vemos acusadas por nuestros
propios hijos.

Al decir esto me pareció que lloraba. Después de breve pausa Amaranta
continuó así la conversación:

---Ese necio de Mañara, que no sabe hablar más que de toros, de caballos
y de su nobleza, ha tenido el honor de cautivar a Lesbia; tal para
cual\ldots{} Él es quien la ha inducido a andar en tratos con los del
Príncipe, y entre los dos se han encargado de la trasmisión de la
correspondencia.

---¿Pero no me dijiste---preguntó vivamente la desconocida,---que Lesbia
estaba en relaciones con Isidoro?

---Sí---contestó mi ama;---pero este amor, que ha durado poco tiempo, ha
sido un interregno durante el cual Mañara no bajó del trono. Lesbia amó
a Isidoro por vanidad, por coquetería, y continúa en relaciones con él.
Isidoro está locamente enamorado, y ella se complace en avivar su amor,
divirtiéndose con los martirios del pobre cómico.

---¿Y no has pensado que se podría sacar partido de esos dobles amores?

---¡Ya lo creo! Lesbia e Isidoro se ven en casa de la González y en el
teatro.

---Puedes hacer que Mañara los descubra y\ldots{}

---No, mi plan es mejor aún. ¿Qué importa Mañara? Yo quiero apoderarme
de alguna carta o prenda, que Lesbia entregue a cualquiera de sus dos
amantes, para presentarla a su marido, a ese señor que a pesar de su
misantropía, si llegara a saber con certeza las gracias de su mujer,
vendría a poner orden en la casa.

---Indudablemente---dijo la desconocida animándose por grados.---¿Y qué
piensas hacer?

---Según lo que den de sí las circunstancias. Pronto volveremos a
Madrid, porque en casa de la marquesa se prepara una representación de
Otello, en que Lesbia hará el papel de Edelmira, Isidoro el suyo y los
demás corren a cargo de jóvenes aficionados.

---¿Y cuándo es la representación?

---Se ha aplazado porque falta un papel, que ninguno quiere desempeñar,
por ser muy desairado; mas creo que pronto se encontrará actor a
propósito, y la función no puede retardarse. El duque ha prometido dejar
sus estados para asistir a ella. La reunión de todas estas personas ha
de facilitar mucho una combinación ingeniosa, que nos permita castigar a
Lesbia como se merece.

---¡Oh!, sí; hazlo por Dios. Su ingratitud es tal, que no merece perdón.
¿Sabes que es ella quien me ha acusado de haber querido asesinar a
Jovellanos?

---Sí, lo sabía.

---¡Ves qué infamia!---añadió la desconocida, indicando en el tono de su
voz la ira que la dominaba.---Verdad es que aborrezco a ese pedante, que
en su fatuidad se permite dar lecciones a quien no las necesita ni se
las ha pedido; pero me parece que su encierro en el castillo de Bellver
es suficiente castigo, y jamás han pasado por mi mente proyectos
criminales, cuya sola idea me horroriza.

---Lesbia se ha dado tan buena maña para propalar lo del envenenamiento,
que todo el mundo lo cree---dijo Amaranta.---¡Ah, señora, es preciso
castigar duramente a esa mujer!

---Sí, pero no incluyéndola en la causa: eso redundaría en perjuicio
mío. Manuel me lo ha advertido esta tarde con mucho empeño, y es preciso
hacer lo que él dice. Por su parte, Manuel le causa todo el daño que
puede. Desde que supo las infamias que contaba de mí, dejó cesantes a
todos los que habían recibido destino por recomendación suya. Esta
prueba de afecto me ha enternecido.

---No sería malo que Mañara sintiera encima la mano de hierro del
generalísimo.

---¡Oh, sí! Manuel me ha prometido buscar algún medio para que se le
forme causa y sea expulsado del cuerpo, como se hizo con aquellos dos
que nos conocieron cuando fuimos disfrazadas a la verbena de Santiago.
¡Oh! Manuel no se descuida: después que nos reconciliamos por mediación
tuya, su complacencia y finura conmigo no tienen límites. No, no existe
otro que como él comprenda mi carácter, y posea el arte de las buenas
formas aun para negar lo que se le pide. Ahora precisamente estoy en
lucha con él para que me conceda una mitra\ldots{}

---¿Para mi recomendado el capellán de las monjas de Pinto?

---No: es para un tío de Gregorilla, la hermana de leche del
chiquitín\footnote{D. Francisco de Paula.}. Ya ves: se le ha puesto en
la cabeza que su tío ha de ser obispo, y verdaderamente no hay motivo
alguno para que no lo sea.

---¿Y el Príncipe se opone?

---Sí; dice que el tío de Gregorilla ha sido contrabandista hasta que se
ordenó hace dos años, y que es un ignorante. Tiene razón, y el candidato
no es por su sabiduría ninguna lumbrera de la cristiandad; pero hija,
cuando vemos a otros\ldots{} y si no ahí tienes a mi primo, el
cardenalito de la Escala\footnote{El cardenal infante don Luis de
  Borbón, arzobispo de Toledo.}, que no sabe más latín que nosotras, y
si le examinaran, creo que ni aun para monaguillo le darían el
\emph{exequatur}.

---Pero ese nombramiento lo ha de hacer Caballero---dijo Amaranta.---¿Se
opone también?

---Caballero no---contestó riendo la desconocida;---ése ya sabes que no
hace sino lo que queremos, y capaz sería de convertir en regentes de las
Audiencias a los puntilleros de la plaza de toros, si se lo mandáramos.
Es un buen sujeto, que cumple con su deber con la docilidad del
verdadero ministro. El pobrecito se interesa mucho por el bien de la
nación.

---Pues él puede dar la mitra por sí y ante sí al tío de Gregorilla.

---No; Manuel se opone, ¡y de qué manera! Pero yo he discurrido un medio
de obligarle a ceder. ¿Sabes cuál? Pues me he valido del tratado secreto
celebrado con Francia, que se ratificará en Fontainebleau dentro de unos
días. Por él se da a Manuel la soberanía de los Algarbes; pero nosotros
no estamos aún decididos a consentir en el repartimiento de Portugal, y
le he dicho: «Si no haces obispo al tío de Gregorilla, no ratificaremos
el tratado, y no serás rey de los Algarbes.» Él se ríe mucho con estas
cosas mías; pero al fin\ldots{} ya verás cómo consigo lo que deseo.

---Y mucho más cuando estos nombramientos contribuyen a fortificar
nuestro partido. ¿Pero él no conoce que el del Príncipe es cada vez más
fuerte?

---¡Ah! Manuel está muy disgustado---dijo la desconocida con
tristeza;---y lo que es peor, muy acobardado. Afirma que esto no puede
concluir en bien y tiene presentimientos horribles. Estos sucesos le han
puesto muy triste, y dice: «Yo he cometido muchas faltas, y el día de la
expiación se acerca.» ¡Pero qué bueno es! ¿Creerás que disculpa a mi
hijo, diciendo que le han engañado y envilecido los amigos ambiciosos
que le rodean? ¡Ah!, mi corazón de madre se desgarra con esto; pero no
puedo atenuar la falta del Príncipe. Mi hijo es un infame.

---¿Y él espera conjurar fácilmente tantos peligros?---preguntó mi ama.

---No lo sé---repuso la desconocida tristemente.---Manuel, como te he
dicho, está muy descorazonado. Aunque cree castigar pronto y
ejemplarmente a los conjurados, como hay algo que está por encima de
todo esto, y que\ldots{}

---Bonaparte sin duda.

---No: Bonaparte creo que estará de nuestro lado, a pesar de que el
Príncipe lo presenta como amigo suyo. Manuel me ha tranquilizado en este
punto. Si Bonaparte se enojase con nosotros, le daríamos veinte o
treinta mil hombres, para que los sacase de España, como sacó los de la
Romana. Eso es muy fácil y a nadie perjudica. Lo que nos entristece es
otra cosa, es lo que pasa en España. Según me ha dicho Manuel, todos
aman al Príncipe y le creen un dechado de perfecciones, mientras que a
nosotros, al pobre Carlos y a mí, nos aborrecen. Parece mentira: ¿qué
hemos hecho para que así nos odien? Francamente, te digo que esto me
tiene afectada, y estoy resuelta a no ir a Madrid en mucho tiempo. Te
juro que aborrezco a Madrid.

---Yo no participo de ese temor---dijo Amaranta,---y espero que
castigados los conspiradores, la mala yerba no volverá a retoñar.

---Manuel trabajará sin descanso: así me lo ha dicho. Pero es preciso
que se evite todo lo que pueda escandalizar, y sobre todo lo que resulte
desfavorable. Por eso esta noche en cuanto llegó Manuel, vino a
suplicarme que por conducto tuyo, hiciese arrancar de la causa todo lo
relativo a Lesbia, que es poseedora de documentos terribles, y se
vengaría cruelmente en sus declaraciones. Ya sabes que tiene mucha
imaginación, y sabe inventar enredos con gran arte. Desde que Manuel me
habló hasta que te he visto, no he sosegado un momento. Pero ni él ni
yo, podemos hablar de esto con Caballero: háblale tú y arréglalo con tu
buen juicio y habilidad. ¡Ah!, se me olvidaba. Caballero desea el Toisón
de Oro: ofréceselo sin cuidado; que aunque no es hombre para cargar tal
insignia, no habrá reparo en dársela, si se hace acreedor a ella con su
lealtad. ¿Harás lo que te digo?

---Sí, señora. No habrá nada que temer.

---Entonces me retiro tranquila. Confío en ti ahora como siempre---dijo
la desconocida levantándose.

---Lesbia no será llamada a declarar; pero no nos faltará ocasión de
tratarla como merece.

---Pues adiós, querida Amaranta---añadió la dama besando a mi
ama.---Gracias a ti, esta noche dormiré tranquila, y entre tantas penas,
no es poco consuelo contar con una fiel amiga que hace todo lo posible
por disminuirlas.

---Adiós.

---Es muy tarde\ldots{} ¡Dios mío, qué tarde!

Diciendo esto se encaminaron juntas a la puerta, y abierta ésta
aparecieron otras dos damas, con las cuales se retiró la desconocida,
después de besar por segunda vez a mi ama. Cuando ésta se quedó sola se
dirigió a la habitación en que yo estaba. Mi primera intención fue
retirarme del escondite y huir; pero reflexionándolo brevemente, creí
que debía esperarla. Cuando ella entró y me vio, su sorpresa fue
extraordinaria.

---¡Cómo, Gabriel, tú aquí!---exclamó.

---Sí, señora---respondí serenamente.---He empezado a desempeñar las
funciones que usía me ha encargado.

---¡Cómo!---dijo con ira;---¿has tenido el atrevimiento de\ldots?, ¿has
oído?

---Señora---respondí,---usía tenía razón: poseo un oído finísimo. ¿No me
mandaba usía que observara y atendiera?\ldots{}

---Sí---dijo más colérica.---Pero no a esto\ldots{} ¿entiendes bien? Veo
que eres demasiado listo, y el exceso de celo puede costarte caro.

---Señora---repuse con mucha ingenuidad,---quería empezar a instruirme
cuanto antes.

---Bien---repuso procurando tranquilizarse.---Retírate. Pero te advierto
que si sé recompensar a los que me sirven bien, tengo medios para
castigar a los desleales y traidores. No te digo más. Si eres
imprudente, te acordarás de mí toda tu vida. Vete.

\hypertarget{xix}{%
\chapter{XIX}\label{xix}}

Al día siguiente se levantó un servidor de ustedes de malísimo humor, y
su primera idea fue salir de El Escorial lo más pronto que le fuera
posible. Para pensar en los medios de ejecutar tan buen propósito fuese
a pasear a los claustros del monasterio, y allí discurriendo sobre su
situación, se acaloró la cabeza del pobre muchacho revolviendo en ella
mil pensamientos que cree poder comunicar al discreto lector.

Los que hayan leído en el primer libro de mi vida el capítulo en que di
cuenta de mi inútil presencia en el combate de Trafalgar, recordarán que
en tan alta ocasión, y cuando la grandeza y majestad de lo que pasaba
ante mis ojos parecían sutilizar las facultades de mi alma, pude
concebir de un modo clarísimo la idea de la patria. Pues bien: en la
ocasión que ahora refiero, y cuando la desastrosa catástrofe de tan
ridículas ilusiones había conmovido hasta lo más profundo mi naturaleza
toda, el espíritu del pobre Gabriel hizo después de tanto abatimiento
una nueva adquisición, una nueva conquista de inmenso valor, la idea del
honor.

¡Qué luz! Recordé lo que me había dicho Amaranta, y comparando sus
conceptos con los míos, sus ideas con lo que yo pensaba, mezcla de
ingenuo engreimiento y de honrada fatuidad, no pude menos de
enorgullecerme de mí mismo. Y al pensar esto no pude menos de decir: «Yo
soy hombre de honor, yo soy hombre que siento en mí una repugnancia
invencible de toda acción fea y villana que me deshonre a mis propios
ojos; y además la idea de que pueda ser objeto del menosprecio de los
demás me enardece la sangre y me pone furioso. Cierto que quiero llegar
a ser persona de provecho; pero de modo que mis acciones me enaltezcan
ante los demás y al mismo tiempo ante mí, porque de nada vale que mil
tontos me aplaudan, si yo mismo me desprecio. Grande y consolador debe
de ser, si vivo mucho tiempo, estar siempre contento de lo que haga, y
poder decir por las noches mientras me tapo bien con mis sabanitas para
matar el frío: \emph{No he hecho nada que ofenda a Dios ni a los
hombres}. \emph{Estoy satisfecho de ti, Gabriel.»}

Debo advertir que en mis monólogos siempre hablaba conmigo como si yo
fuera otro.

Lo particular es que mientras pensaba estas cosas, la figura de mi Inés
no se apartaba un momento de mi imaginación y su recuerdo daba vueltas
en torno a mi espíritu, como esas mariposas o pajaritas que se nos
aparecen a veces en días tristes trayendo según el vulgo cree, alguna
buena noticia.

Tal era la situación de mi espíritu, cuando acertó a pasar cerca de mí
el caballero don Juan de Mañara, vestido de uniforme. Detúvose y me
llamó con empeño, demostrando que mi presencia era para él nada menos
que un buen hallazgo. No era aquélla la primera vez que solicitaba de mí
un pequeño favor.

---Gabriel---me dijo en tono bastante confidencial sacando de su
bolsillo una moneda de oro,---esto es para ti, si me haces el favor que
voy a pedirte.

---Señor---contesté,---con tal que sea cosa que no perjudique a mi
honor\ldots{}

---Pero, pedazo de zarramplín, ¿acaso tú tienes honor?

---Pues sí que lo tengo, señor oficial---contesté muy enfadado;---y
deseo encontrar ocasión de darle a usted mil pruebas de ello.

---Ahora te la proporciono, porque nada más honroso que servir a un
caballero y a una señora.

---Dígame usted lo que tengo que hacer---dije deseando ardientemente que
la posesión del doblón que brillaba ante mis ojos fuera compatible con
la dignidad de un hombre como yo.

---Nada más que lo siguiente---respondió el hermoso galán sacando una
carta del bolsillo:---llevar este billete a la señorita Lesbia.

---No tengo inconveniente---dije, reflexionando que en mi calidad de
criado no podía deshonrarme llevando una carta amorosa.---Déme usted la
esquelita.

---Pero ten en cuenta---añadió entregándomela,---que si no desempeñas
bien la comisión, o este papel va a otras manos, tendrás memoria de mí
mientras vivas, si es que te queda vida después que todos tus huesos
pasen por mis manos.

Al decir esto el guardia demostraba, apretándome fuertemente el brazo,
firme intención de hacer lo que decía. Yo le prometí cumplir su encargo
como me lo mandaba, y tratando de esto llegamos al gran patio de
palacio, donde me sorprendió ver bastante gente reunida descollando
entre todos algunas aves de mal agüero, tales como ministriles y demás
gente de la curia. Yo advertí que al verles mi acompañante se inmutó
mucho, quedándose pálido, y hasta me parece que le oí pronunciar algún
juramento contra los pajarracos negros que tan de improviso se habían
presentado a nuestra vista. Pero yo no necesitaba reflexionar mucho para
comprender que aquella siniestra turba nada tenía que ver conmigo, así
es que dejando al militar en la puerta del cuerpo de guardia, una vez
trasladadas carta y moneda a mi bolsillo, subí en cuatro zancajos la
escalera chica, corriendo derecho a la cámara de la señora Lesbia.

No tardé en hacerme presentar a su señoría. Estaba de pie en medio de la
sala, y con entonación dramática leía en un cuadernillo aquellos versos
célebres:

\small
\newlength\mleni
\settowidth\mleni{—Y todo te confunde, desdichada.}
\begin{center}
\parbox{\mleni}{                             ... todo me mata,  \\
                todo va reuniéndose en mi daño!                 \\
               —Y todo te confunde, desdichada.}               \\
\end{center}
\normalsize

Estaba estudiando su papel. Cuando me vio entrar cesó su lectura, y tuve
el gusto de entregarle en persona el billete, pensando para mí: ¿Quién
dirá que con esa cara tan linda eres una de las mejores piezas que han
hecho enredos en el mundo?

Mientras leía, observé el ligero rubor y la sonrisa que hermoseaban su
agraciado rostro. Después que hubo concluido, me dijo un poco alarmada:

---¿Pero tú no sirves a Amaranta?

---No señora---respondí.---Desde anoche he dejado su servicio, y ahora
mismo me voy para Madrid.

---¡Ah!, entonces bien---dijo tranquilizándose.

Yo en tanto no cesaba de pensar en el placer que habría experimentado
Amaranta si yo hubiera cometido la infamia de llevarle aquella carta.
¡Qué pronto se me había presentado la ocasión de portarme como un
servidor honrado, aunque humilde! Lesbia, encontrando ocasión de zaherir
a su amiga, me dijo:

---Amaranta es muy rigurosa y cruel con sus criados.

---¡Oh, no señora!---exclamé yo, gozoso de encontrar otra coyuntura de
portarme caballerosamente, rechazando la ofensa hecha a quien me daba el
pan.---La señora condesa me trata muy bien; pero yo no quiero servir más
en palacio.

---¿De modo que has dejado a Amaranta?

---Completamente. Me marcharé a Madrid antes de mediodía.

---¿Y no querrías entrar en mi servidumbre?

---Estoy decidido a aprender un oficio.

---De modo que hoy estás libre, no dependes de nadie, ni siquiera
volverás a ver a tu antigua ama.

---Ya me he despedido de su señoría y no pienso volver allá.

No era verdad lo primero, pero sí lo segundo.

Después, como yo hiciera una profunda reverencia para despedirme, me
contuvo diciendo:

---Aguarda: tengo que contestar a la carta que has traído, y puesto que
estás hoy sin ocupación, y no tienes quien te detenga, llevarás la
respuesta.

Esto me infundió la grata esperanza de que mi capital se engrosara con
otro doblón, y aguardé mirando las pinturas del techo y los dibujos de
los tapices. Cuando Lesbia hubo concluido su epístola, la selló
cuidadosamente y la puso en mis manos, ordenándome que la llevase sin
perder un instante. Así lo hice; pero ¿cuál no sería mi sorpresa cuando
al llegar al cuerpo de guardia me encontré con la inesperada novedad de
que sacaban preso a mi señor el guardia, llevándole bonitamente entre
dos soldados de los suyos! Yo temblé como un azogado, creyendo que
también iban a echarme mano, pues sabía que no bastaba ser
insignificante para librarse de los ministriles, quienes deseando
mostrar su celo en la causa de El Escorial, comprendían en los
voluminosos autos el mayor número posible de personas.

Cometí la indiscreción de entrar en el cuerpo de guardia para curiosear,
lo cual hizo que un hombre allí presente, temerosa estantigua con nariz
de gancho, espejuelos verdes y larguísimos dientes del mismo color,
dirigiese hacia mi rostro aquellas partes del suyo, observándome con
tenaz atención y diciendo con la voz más desagradable y bronca que en mi
vida oí:

---Éste es el muchacho a quien el preso entregó una carta poco antes de
caer en poder de la justicia.

Un sudor frío corrió por mi cuerpo al oír tales palabras, y volví la
espalda con disimulo para marcharme a toda prisa; pero ¡ay!, no había
andado dos pasos, cuando sentí que se clavaban en mi hombro unas como
garras de gavilán, pues no otro nombre merecían las afiladas y durísimas
uñas del hombre de los espejuelos verdes en cuyo poder había caído. La
impresión que experimenté fue tan terrorífica, que nunca pienso
olvidarla, pues al encarar con su finísima estampa, los vidrios redondos
de sus gafas que recomendaban la pupila cuajada, penetrante y
estupefacta del gato, me turbaron hasta lo sumo, y al mismo tiempo sus
dientes verdes, afilados sin duda por la voracidad, parecían ansiosos de
roerme.

---No vaya usted tan de prisa, caballerito---dijo,---que tal vez haga
aquí más falta que en otra parte.

---¿En qué puedo servir a usía?---pregunté melifluamente, comprendiendo
que nada me valdría mostrarme altanero con semejante lobo.

---Eso lo veremos---contestó con un gruñido que me obligó a encomendarme
a Dios.

Mientras aquel cernícalo, con la formidable zarpa clavada en mi cuello,
me llevaba a una pieza inmediata, yo evoqué mis facultades intelectuales
para ver si con el esfuerzo combinado de todas ellas encontraba medio de
salir de tan apurado trance. En un instante de reflexión, hice el
siguiente rapidísimo cálculo: «Gabriel: este instante es supremo. Nada
conseguirás defendiéndote con la fuerza. Si intentas escaparte, estás
perdido. De modo que si por medio de algún rasgo de astucia no te libras
de las uñas de este pícaro, que te enterrará vivo bajo una losa de papel
sellado, ya puedes hacer acto de contrición. Al mismo tiempo llevas
sobre ti la honra de una dama que sabe Dios lo que habrá escrito en esta
endiablada carta. Conque ánimo, muchacho, serenidad y a ver por dónde se
sale.»

Afortunadamente, Dios iluminó mi entendimiento en el instante en que el
curial se sentó en un desnudo banquillo, poniéndome delante para que
respondiera a sus preguntas. Recordé haber visto al feroz leguleyo en el
cuarto de Amaranta, a quien gustaba de ofrecer servilmente sus respetos,
y esto con la idea de que mi antigua ama era desafecta a las personas a
quienes se formaba la causa, me dio la norma del plan que debía seguir
para librarme de aquel vestiglo.

---Conque tú andas llevando y trayendo cartitas, picaronazo---dijo en la
plenitud de su curial sevicia, gozándose de antemano con la
contemplación imaginaria de las resmas de papel sellado en que había de
emparedarme.---Ahora veremos para quiénes son esas cartas, y si te
ocupas en comunicar a los conjurados con los presos, para que burlen la
acción de la justicia.

---Señor licenciado---contesté yo recobrando un poco la
serenidad,---usted no me conoce, y sin duda me confunde con esos
picarones que se ocupan en traer y llevar papelitos a los que están
presos en el Noviciado.

---¿Cómo?---exclamó con júbilo.---¿Estás seguro de que eso pasa?

---Sí, señor---respondí envalentonándome cada vez más.---Vaya usía ahora
mismo con disimulo al patio de los convalecientes, y verá que desde el
piso tercero del monasterio echan cartas a la bohardilla valiéndose de
unas larguísimas cañas.

---¿Qué me dices?

---Lo que usía oye: y si quiere verlo con sus propios ojos vaya ahora
mismo; que esta es la hora que escogen los malvados para su intento, por
ser la de la siesta. Ya me podría usía recompensar por la noticia, pues
le doy este aviso, para que pueda prestar un gran servicio a nuestro
querido Rey.

---Pero tú recibiste una carta del joven alférez, y si no me la das ante
todo, ya te ajustaré las cuentas.

---¿Pero el señor licenciado no sabe---contesté---que soy paje de la
excelentísima señora condesa Amaranta, a quien sirvo hace algún tiempo?
¡Y que no me tiene poco cariño mi ama en gracia de Dios! Mil veces ha
dicho que ya puede tentarse la ropa el que me tocase tan siquiera al
pelo de la misma.

El leguleyo parecía recordar, y como era cierto que me había visto
repetidas veces en compañía de mi ama, advertí que su endemoniado rostro
se apaciguaba poco a poco.

---Bien sabe el señor licenciado---continué---que la señora condesa me
protege, y habiendo conocido que yo sirvo para algo más que para ese
bajo oficio, se propone instruirme y hacer de mí un hombre de provecho.
Ya he empezado a estudiar con el padre Antolínez, y después entraré en
la Casa de Pajes, porque ahora hemos descubierto que yo, aunque pobre,
soy noble y desciendo en línea recta de unos al modo de duques o
marqueses de las islas Chafarinas.

El leguleyo parecía muy preocupado con estas razones que yo pronuncié
con mucho desparpajo.

---Y ahora---proseguí---iba al cuarto de mi ama, que me está esperando,
y en cuanto sepa que el señor licenciado me ha detenido se pondrá
furiosa: porque ha de saber el señor licenciado que mi ama me manda
recorrer estos patios y galerías para oír lo que dicen los partidarios
de los presos, y ella lo va apuntando en un libro que tiene, no menos
grande que ese banco. Ella va a descubrir muchas cosas malas de esa
gente y está muy contenta con mi ayuda, pues dice que sin mí no sabría
la mitad de lo que sabe. Por ejemplo, lo de las cañas apuesto a que
nadie lo sabe más que yo, y agradézcame el señor licenciado que se lo
haya dicho antes que a ninguno.

---Cierto es---dijo el ministril,---que la señora condesa te protege,
pues ahora caigo en la cuenta de que algunas veces se lo he oído decir;
pero no me explico que tu ama se cartee con el alférez.

---También a mí me llamó la atención---repuse,---porque mi ama decía que
ese señor era de los que primero debían ser puestos a la sombra; pero
vea el señor licenciado. La carta que recibí era para mi ama; y le decía
que viéndose próximo a caer en poder de la justicia, solicitaba la
protección de la señora condesa para librarse de aquélla.

---¡Ah, señor Mañara, tunante, trapisondista!---exclamó el representante
de la justicia humana.---Quería escaparse de nuestras uñas, poniéndose
al amparo de una persona que está demostrando el mayor celo en favor de
la causa del Rey.

---Pero no le valieron sus malas mañas, señor licenciadito de mi
alma---añadí entusiasmándome,---porque mi ama rompió la carta con
desdén, y me mandó contestarle de palabra que nada podía hacer por él.

---¿Y a eso venías?

---Precisamente. Ya sabía yo que no lograba nada el señor alférez. Y me
alegro, me alegro. Porque yo digo: esos picarones, ¿no querían quitarle
al Rey su corona, y a la Reina la vida? Pues que las paguen todas
juntas, que bien merecido tienen el cadalso; y como se descuiden, el
señor Príncipe de la Paz no se andará por las ramas.

---Bien---dijo algo más benévolo para conmigo, pero sin que se
extinguiera su recelo.---Iremos juntos a ver a tu ama, y ella confirmará
lo que has dicho.

---Ahora se fue al cuarto del Príncipe de la Paz, a quien piensa
recomendarme para que entre en la Casa de Pajes. Y como el señor
licenciado se descuide, no podrá ver a los que echan la caña por los
balcones del piso tercero del monasterio. Vaya usía a enterarse de esto,
y luego puede pasar al cuarto de mi ama, donde le espero. Ella estará
prevenida y recibirá a usía con mucho agasajo, porque le aprecia y
estima mucho.

---¿Sí? ¿Le has oído hablar de mí alguna vez?---preguntó vivamente.

---¿Alguna vez? Diga el señor licenciado mil veces. La otra noche estuvo
hablando de usía más de dos horas con el Príncipe de la Paz, y con el
marqués Caballero.

---¿De veras?---preguntó plegando su arrugada boca con una sonrisa
indefinible y dejando ver en todo su vasto desarrollo el mapa de su
verde dentadura.---¿Y qué decía?

---Que al señor licenciado se deben todas las averiguaciones que se han
hecho en la causa, y otras cosas que no digo por no ofender la modestia
de usía.

---Dilas picarón, y no seas corto de genio.

---Pues hizo grandes elogios de usía, ponderando su talento, su mucho
saber y su disposición para sacar leyes aunque fuera de un canto rodado.
Después añadió que si no le hacían al señor licenciado consejero de
Indias o de la sala de alcaldes de Casa y Corte, no tendrían perdón de
Dios.

---¿Eso dijo? Veo que eres un chico formal y discreto. Di a la señora
condesa que dentro de un momento pasaré a visitarla, para consultar con
ella gravísimas cuestiones. Ella sabrá cuánto la aprecio y estimo. Con
respecto a ti, al principio pensé que la carta entregada por el alférez
era para la duquesa Lesbia.

---¡Quiá! No voy yo al cuarto de esa señora, porque mi ama y ella están
reñidas.

---Y como hoy---continuó,---se procederá también a prender a esa señora,
que resulta complicada en el proceso lo mismo que su esposo el señor
duque\ldots{}

---¡También prenden a la señora Lesbia!---exclamé asombrado.

---También; ya habrán subido mis compañeros a notificárselo. Conque,
joven, sube al cuarto de tu ama, adviértele mi próxima visita.

No esperé más para separarme de hombre tan fiero, y bendiciendo
fervorosamente a Dios, salí del cuerpo de guardia, muy satisfecho de la
estratagema empleada. Mi primera intención fue correr al cuarto de
Lesbia, no sólo para devolverle la carta, sino para prevenirla acerca
del gran riesgo que su libertad corría; mas cuando subí, noté que la
justicia había invadido su vivienda. Era preciso huir de palacio, donde
corría gran peligro de caer en poder del atroz licenciado, en cuanto
éste, conferenciando con mi ama, descubriese mis estupendas mentiras.
Pies, ¿para qué os quiero?, dije, y al punto subí precipitadamente a mi
camaranchón, cogí y empaqueté de cualquier modo mi ropa, y sin
despedirme de nadie salí del palacio y del monasterio, resuelto a no
detenerme hasta Madrid.

A pesar de mi zozobra, no quise partir sin provisiones, y habiéndome
surtido en la plaza del pueblo de lo más necesario, eché a andar,
volviendo a cada rato la vista, porque me parecía que el licenciado
caminaba detrás de mí. Hasta que no desapareció de mi vista la cúpula y
las torres del terrible monasterio no recobré la tranquilidad, y después
de dos horas de precipitada marcha, me aparté del camino y restauré mis
fuerzas con pan, queso y uvas, seguro ya de que por el momento las
durísimas uñas del representante de la justicia no se clavarían en mis
hombros.

En aquel rato de descanso y esparcimiento, me reí a mis anchas,
recordando las mentiras que había empleado para salvarme; pero no me
remordía la conciencia por haberlas desembuchado con tanta largueza,
puesto que aquellos embustes, con los cuales no perjudicaba a la honra
de nadie, eran la única arma que me defendía contra una persecución tan
bárbara como injusta. Los trances difíciles aguzan al ingenio, y en
cuanto a mí, puedo decir que antes de encontrarme en el que he referido,
jamás hubiera sido capaz de inventar tales desatinos. Bien dicen, que
las circunstancias hacen al hombre tonto o discreto, aguzando el más
rústico entendimiento, u oscureciendo el que se precia de más claro.

Más allá de Torrelodones encontré unos arrieros, que por poco dinero me
dejaron montar en sus caballerías, y de este modo llegué a Madrid
cómodamente, ya muy avanzada la noche.

\hypertarget{xx}{%
\chapter{XX}\label{xx}}

Como era tarde, creí que no debía ir a casa de Inés hasta la mañana
siguiente, y entré en la de la González, que aún estaba levantada y como
sin intención de recogerse todavía. Quedose muy asombrada al verme
entrar, y faltole tiempo para preguntarme lo que me había pasado, y si
había ocurrido alguna novedad a la señorita Amaranta. También quiso
saber lo de la famosa conjuración, asunto que, según dijo, ocupaba la
atención de Madrid entero, y satisfecha su curiosidad en este y otros
puntos, me aseguró haber recibido una carta de Lesbia, en que le
anunciaba su viaje a la corte dentro de algunos días para acabar de
perfeccionarse en el papel de Edelmira.

Aunque el cansancio me rendía, y más deseaba acostarme que hablar, le
conté lo de la carta y también el triste caso de la prisión de la
duquesa. Pepita, muy alterada con estas noticias, me rogó que le
entregase la carta, a lo cual me negué, jurando que la guardaría hasta
que pudiese dársela en propia mano a la misma persona de quien la
recibí. Ella pareció conformarse con mi negativa, y no hablamos más del
asunto. Después le dije que resuelto a aprender un oficio había
abandonado a Amaranta para regresar a la corte y me fui a acostar,
deseando que llegase pronto la mañana por ver a Inés. Excuso decir que
dormí como un talego; levanteme al día siguiente muy a prisa, y mi
primera impresión fue una gran pesadumbre. Les contaré a ustedes: al
vestirme, busqué entre mis ropas la carta de Lesbia, y la carta no
parecía. No quedó en mis bolsillos ni en mi breve equipaje escondrijo
que no fuese revuelto; pero no encontré nada. Muy afanado estaba,
temiendo que la carta hubiese caído en manos indiscretas, cuando le
conté a mi ama lo que me pasaba, preguntándole si había encontrado por
el suelo la malhadada epístola. Entonces la pícara, lanzando una
carcajada de alegría, me contestó con la mayor desvergüenza:

---No la he encontrado, Gabrielillo, sino que anoche, luego que te
dormiste, entré en tu cuarto de puntillas y saqué la carta del bolsillo
de tu chaqueta. Aquí la tengo, la he leído, y no la soltaré por nada.

Aquello me indignó sobremanera. Pedile la carta, diciéndole que mi honor
me exigía devolverla a su dueña sin que nadie la leyera; mas ella me
repuso que yo no tenía honor que conservar, y que en cuanto a la carta
no la devolvería, aunque le diesen tantos azotes como letras estaban
escritas en ella. Acto continuo me la leyó, y decía así si mal no
recuerdo:

«Amado Juan: Te perdono la ofensa y los desaires que me has hecho; pero
si quieres que crea en tu arrepentimiento, pruébamelo viniendo a cenar
conmigo esta noche en mi cuarto, donde acabaré de disipar tus infundados
celos, haciéndote comprender que no he amado nunca, ni puedo amar a
Isidoro, ese salvaje, presumido comiquillo, a quien sólo he hablado
alguna vez con objeto de divertirme con su necia pasión. No faltes si no
quieres enfadar a tu \emph{Lesbia}. P . D. No temas que te prendan.
Primero prenderán al Rey.»

Leída la carta, la González se la guardó en el pecho, diciendo entre
risas y chistes, que ni por diez mil duros la devolvería. Todas mis
súplicas fueron inútiles, y al fin, cansado de desgañitarme, salí de la
casa, muy apesadumbrado con aquel incidente; mas esperando desvanecer mi
mal humor con la vista de la infeliz Inés. Dirigime allá muy conmovido,
y al entrar por la calle, mirando a los balcones de su casa, decía:
«¡Cuán lejos estará de que yo acabo de doblar la esquina y estoy en la
calle! Estará sentada detrás de la cortinilla, y aunque no tendría más
que asomarse un poco para verme, no me verá hasta que no entre en la
casa.»

Llegué, por fin, y desde que me abrió la puerta comprendí que algo grave
allí pasaba, porque Inés no corrió a mi encuentro, a pesar de las
fuertes voces que di al poner el pie dentro de la casa. Quien primero me
recibió fue el padre Celestino, con rostro tan extremadamente
compungido, que atribuirse no podía su escualidez a la sola causa del
hambre.

---Hijo mío, en mal hora vienes---me dijo.---Aquí tenemos una gran
desgracia. Mi hermana, la pobre Juana se nos muere sin remedio.

---¿Pero Inés?

---Buena: pero figúrate cómo estará la pobrecita con el ajetreo de estos
días. No se separa del lado de su madre, y si esto siguiera mucho tiempo
creo que también se llevaría Dios al pobre angelito de mi sobrina.

---Bien le decíamos a la señora doña Juana que no trabajase tanto.

---Y ¿qué quieres, hijo mío?---respondió.---Ella mantenía la casa;
porque ya ves, todavía no me han dado el curato, ni la capellanía, ni la
coadjutoría, ni la ración, ni la beca, ni la congrua que me han
prometido, aunque tengo la seguridad de que a más tardar la semana que
entra se cumplirán mis deseos. Además mi poema latino no hay librero que
lo quiera imprimir aunque le dieran dinero encima, y aquí tienes la
situación. No sé qué va a ser de nosotros si mi hermana se muere.

Al decir esto, las quijadas del pobre viejo se descoyuntaron en un
bostezo descomunal que me probó la magnitud de su hambre. Semejante
espectáculo me oprimía el corazón; pero afortunadamente yo tenía algún
dinero de mis ahorros y además el doblón de Mañara, lo cual me permitía
hacer una hombrada. Echándome la mano al bolsillo, dije:

---Señor cura, en celebración de la congrua que ha de recibir su
paternidad la semana que entra, le convido a chuletas.

---No tengo gana---respondió haciendo alarde de aquella gentil
delicadeza que le caracterizaba,---y además no quiero que gastes tus
ahorros; pero si quieres tú comerlas, que las traigan y aquí te las
aderezaremos.

Al instante mandé a una vecina por la carne, y mientras venía, no
pudiendo contener mi impaciencia, me interné en busca de Inés. Hallela
en la habitación principal, no lejos de la cama de su madre, que dormía
profundamente.

---Inesilla, Inesilla de mi corazón---dije corriendo a ella y dándole
media docena de abrazos.

Por única respuesta Inés me señaló a la enferma, indicándome que no
hiciera ruido.

---Tu madre se pondrá buena---le contesté en voz baja.---¡Ay, Inesita,
cuánto deseaba verte! Vengo a confesarte que soy un bruto, y que tú
tienes más talento que el mismo Salomón.

Inés me miró sonriendo con serena tranquilidad, como si de antemano
hubiera sabido que yo vendría a hacer tales confesiones. Mi discreta y
pobre amiga estaba muy pálida por los insomnios y el trabajo; pero
¡cuánto más hermosa me pareció que la terrible Amaranta! Todo había
cambiado, y el equilibrio de mis facultades estaba restablecido.

---Mira, Inesilla---dije besándole las manos,---acertaste en todas tus
profecías. Estoy arrepentido de mi gran necedad, y he tenido la suerte
de encontrar pronto el desengaño. Bien dicen que los jóvenes nos dejamos
alucinar por sueños y fantasmas. Pero, ¡ay!, no todos tienen un buen
ángel como tú que les enseñe lo que han de hacer.

---¿De modo que ya no le tendremos a usía de capitán general ni de
virrey?---me dijo burlándose de mis locuras.

---No, niñita; no estoy ya por los palacios ni por los uniformes. Si
vieras tú qué feas son ciertas cosas cuando se las ve de cerca. El que
quiere medrar en los palacios, tiene que cometer mil bajezas, contrarias
al honor, porque yo tengo también mi honor, sí señora\ldots{} Nada,
nada: dejémonos de virreinatos y de bambollas. He sido un alma de
cántaro; pero bien dice el señor cura, tu tío, que la experiencia es una
llama que no alumbra sino quemando. Yo me he quemado vivo; pero ¡ay!,
hija, ¡si vieras cuánto he aprendido! Ya te contaré.

---¿Y ya no vuelves allá?

---No, señora; aquí me quedo, porque tengo un proyecto\ldots{}

---¿Otro proyecto?

---Sí, pero este te ha de gustar, picarona. Voy a aprender un oficio. A
ver cuál te parece mejor. ¿Platero, ebanista, comerciante? Lo que tú
quieras. Todo menos el de criado.

---Eso no está mal discurrido.

---Pero detrás de este proyecto, está otro mejor---dije gozando de un
modo indecible con aquel diálogo.---Sí, hijita, tengo el proyecto de
casarme con usted.

La enferma hizo un movimiento, y entonces Inés, atendiendo a su madre,
no pudo dar contestación a mis vehementes palabras.

---Yo tengo diez y seis años---continué,---tú quince; de modo que no hay
más que hablar. Aprenderé un oficio, en el cual pienso ganar pronto
muchísimo dinero, que tú irás guardando para nuestra boda. Verás, verás
qué bien vamos a estar. ¿Quieres, sí o no?

---Gabriel---repuso en voz muy baja,---ahora somos muy pobres. Si me
quedo huérfana, lo seremos mucho más. A mi tío no le darán nunca lo que
está esperando hace catorce años. ¿Qué va a ser de nosotros? Tú no
ganarás nada hasta que no pase algún tiempo: no pienses, pues, en
locuras.

---Pero, tonta, dentro de cuatro años habré yo ganado más de lo que
peso. Entonces, para entonces\ldots{} Mientras tanto, ya nos
arreglaremos. Para algo te ha dado Dios ese talento de doctora de la
Iglesia que tienes. Ahora conozco que sin ti no valgo nada, ni sirvo
para nada.

---Eso después que te reías de mí, cuando te decía: «Gabriel, vas por
mal camino.»

---Tenías razón, cordera. ¡Si vieras qué raro es el hombre por dentro, y
cómo se equivoca, y cómo ignora hasta lo mismo que le pasa! Cuando salí
de aquí creí que no te quería, y como aquella señora me tenía
deslumbrado, apenas me acordaba de ti. Pero no: te quería y te quiero
más que a mi vida, sólo que a veces parece que se le ponen a uno
telarañas en los ojos que tenemos por dentro, y no vemos lo mismo que
nos pasa en\ldots{} pues\ldots{} por dentro. Y al mismo tiempo, querida,
tu carita se me venia a la memoria, cuando, decidido a no ceder a los
caprichos de aquella dama endemoniada, pensaba que el hombre debe
buscarse una fortuna por medios honrosos.

La enferma llamó a su hija, y nuestro dulce coloquio quedó interrumpido.
Pero tras el placer que había experimentado conferenciando con Inés,
Dios me deparó el no menos grato de ver comer las chuletas al padre
Celestino, quien a pesar de la gran necesidad que padecía, no las cató
sin hacer mil remilgos, para poner a salvo su dignidad y pundonor.

---He almorzado hace un rato, Gabriel---dijo;---pero si te
empeñas\ldots{}

Mientras comía recayó la conversación sobre los asuntos de El Escorial,
y él que no ocultaba su afición a Godoy, se expresó así:

---Harán bien en extirpar de raíz la conjuración. Pues no es nada la que
tenían armada contra nuestros queridos Reyes y ese dignísimo Príncipe de
la Paz, mi paisano y amigo protector de los menesterosos.

---Pues la opinión general aquí, como en el real Sitio---le
contesté,---es favorable al Príncipe Fernando, y todos acusan a Godoy de
haber fraguado esto para desacreditarle.

---¡Pícaros, embusteros, rufianes!---exclamó furioso el clérigo.---¿Qué
saben ellos de eso? Si conocieran, como yo conozco, las intrigas del
partido fernandista\ldots{} Descuiden, que ya le contaré todo al señor
Príncipe de la Paz cuando vaya a darle las gracias por mi curato, lo
cual, según me ha dicho el oficial de la secretaría, no puede pasar de
la semana que entra. ¡Ah! Si tú conocieras al canónigo don Juan de
Escóiquiz como le conozco yo\ldots{} Aquí le tienen por un corderito
pascual, y es el bribón mayor que ha vestido sotana en el mundo. ¿Quién
sino él se ha opuesto a que me den el curato? Y todo porque en las
oposiciones que hicimos en Zaragoza hace treinta y dos años, sobre el
tema \emph{Utrum helemosinam}\ldots{} no recuerdo lo demás\ldots{} le
dejé bastante corrido. Desde entonces me ha tomado grande ojeriza.
Cuando estemos más despacio, Gabrielillo, te contaré las mil infames
tretas que ha empleado el arcediano de Alcaraz para conquistar la
voluntad de su discípulo. ¡Ah!, yo sé cosas muy gordas. Él es el alma de
este negocio; él ha urdido tan indigna trama; él ha estado en tratos con
el embajador de Francia, monsieur Beauharnais, para entregar a Napoleón
la mitad de España, con tal que ponga en el trono al príncipe heredero,
sí señor.

---Pues oiga usted a todo el mundo---respondí,---y verá cómo al señor
Escóiquiz le ponen por esas nubes, mientras dicen mil picardías del
primer ministro.

---Envidia, chico, envidia. Es que todos le piden colocaciones, destinos
y prebendas y como no los puede dar sino a las personas decentes como
yo, de aquí que la mayoría se queja, murmura y ya ves. ¿Y podrán negar
que se le deben multitud de cosas buenas, como la protección a la
enseñanza, la creación del seminario de caballeros pajes, el fomento de
la botánica, las escuelas de agricultura, los jardines de aclimatación,
la prohibición de enterrar en los templos, y otras muchas reformas
útiles, que aunque criticadas por los ignorantes, ello es que son
laudables y así ha de reconocerlo la posteridad? Cuando estemos despacio
te contaré otras cosas que te harán variar de opinión, y si no, al
tiempo. Yo bien sé que me arrastrarán los madrileños si salgo por ahí
diciendo estas cosas; pero amigo\ldots{} \emph{super omnia veritas}.

---Pues hablando de otra cosa---le dije,---aquí donde usted me ve, puede
que le haya conseguido un servidor el destinillo que pretendía.

---¿Tú? ¿Qué puedes tú? Godoy quiere servirme, sí, él lo hará sin
necesidad de recomendaciones. Y a fe, hijo mío, que si no me colocan
pronto, y se muere Juana, lo vamos a pasar mal; pero muy mal.

---Pero doña Juana tiene parientes ricos.

---Sí, Manso Requejo y su hermana Restituta, comerciantes de telas en la
calle de la Sal. Ya sabes que son avaros de aquellos de hártate comilón
con pasa y media. Jamás han hecho nada por sus parientes. La pobre Inés
no tiene que agradecerles ni un pañuelo.

---¡Qué miserables!

---Además, cuando yo me establecí en Madrid, hace catorce años, conocí a
ese Requejo. Juana estaba ya viuda, Inés era tamañita así, y tan
lindilla y tan amable como ahora. Pues bien: el primo de Juana, a quien
yo insté en cierta ocasión para que favoreciera a esa familia, me dijo:
«No puedo hacer nada por ellas, porque Juana ha renegado de sus
parientes; en cuanto a Inesilla estoy casi seguro de que no es de mi
sangre. Me han dicho que es una inclusera, a quien Juana ha recogido
haciéndola pasar por hija suya.» Pretexto, nada más que pretexto, para
disculpar su avaricia. No me fue posible convencer a aquel bárbaro, y
desde entonces no le he vuelto a ver.

---¿De modo que no hay que contar con esa gente?

---Como si no existieran.

Estas palabras me llevaron a reflexionar sobre la suerte de aquella
infeliz familia. Hubiera deseado tener los tesoros de Creso para
ponérselos a Inés en el cestillo de la costura. Como nunca, sentí
entonces imperiosa y viva la primera necesidad del hombre honrado, que
está resuelto a no vender su conciencia. No tenía dinero\ldots{} ¿Cómo
adquirirlo?

Fui otra vez al lado de Inés, a quien no podía menos de mostrar a cada
instante mi afecto vehemente; y después que conferenciamos otro poco,
salí de casa, pensando en el ardid que emplearía para que el padre
Celestino recibiese, sin menoscabo de su dignidad, el doblón que me dio
Mañara, y diciendo entre mí a cada paso: «¡Maldito dinero! ¿Dónde
estás?»

\hypertarget{xxi}{%
\chapter{XXI}\label{xxi}}

Al entrar en casa de la González, ésta acudió presurosa a mi encuentro,
y me causó sorpresa el verla muy alegre, con esa alegría inquieta y
febril de los niños, que ríen, cantan, golpean y destrozan cuanto
encuentran al paso. Mi ama me habló lo que después diré, y a cada frase
se interrumpía para cantar alguna tonada o estribillo de los infinitos
que enriquecían su repertorio de sainetes.

---¿Qué pasa para tanta alegría, señora?

---He tenido carta de la señora marquesa---me contestó,---la cual viene
mañana a preparar la función. Yo estoy encargada de dirigir la escena.

\small
\newlength\mlenj
\settowidth\mlenj{y el demonio del gato}
\begin{center}
\parbox{\mlenj}{    Sal quiere el huevo,                        \\
                y el demonio del gato                           \\
                vertió el salero.}                              \\
\end{center}
\normalsize

---Buen provecho---dije.---¿Y qué cuenta de la señora Lesbia?

---Que la pusieron en libertad a la media hora conociendo que nada
resultaba contra ella. También dejaron libre a don Juan. Pronto les
tendremos aquí, y la función no se retrasará. ¡Qué placer! Yo dirijo la
escena.

\small
\newlength\mlenk
\settowidth\mlenk{y el demonio del gato}
\begin{center}
\parbox{\mlenk}{    Madre, y qué gusto                          \\
                es ver a dos gitanos                            \\
                trocar de burros.}                              \\
\end{center}
\normalsize

---Pues sea enhorabuena.

---Pero hay un inconveniente, Gabriel---prosiguió.---Ya sabes que
ninguno de esos señores quiere hacer el papel de Pésaro por ser muy
desairado. Perico Rincón, mi compañero, dijo que lo haría, si le daban
mil reales; pero cátate que ha caído con una pulmonía, y si la función
es para el 6, no sé cómo nos compondremos. ¿Quieres tú hacer el papel de
Pésaro?

---¡Yo!, yo representar---exclamé con espanto.---No quiero ser cómico.

---Pero representas de aficionado, tontuelo; y el honor de salir a las
tablas en un teatro como el de la marquesa es tal, que muchos currutacos
se desvivirían por obtenerlo. ¡Y yo dirijo la escena!

\small
\newlength\mlenl
\settowidth\mlenl{porque amo a un escribiente}
\begin{center}
\parbox{\mlenl}{   En mi casa me dicen                          \\
                que soy usía, que soy usía,                     \\
                porque amo a un escribiente                     \\
                de lotería.}                                    \\
\end{center}
\normalsize

Conque chico, vas a aprender ese papel; que aunque es superior a tu
edad, con unas barbas postizas, arregladas por mí, y teniendo tú cuidado
de ahuecar la voz, quedarás que ni pintado. Además, no olvides que la
señora marquesa ha ofrecido dos mil reales a todas las partes de por
medio que trabajan en esta representación. Juanica, que hace de
Hermanacia, no cobra más de mil.

\small
\newlength\mlenm
\settowidth\mlenm{porque amo a un escribiente}
\begin{center}
\parbox{\mlenm}{   La noche de San Pedro                        \\
                te puse un ramo,                                \\
                y amaneció florido                              \\
                como mil mayos.}                                \\
\end{center}
\normalsize

¿Conque aceptas, chiquillo, sí o no?

No pude menos de discurrir que sería muy tonto si renunciaba a poseer
aquellos dineros, que me venían como anillo al dedo para ofrecer a Inés
un auxilio en su tribulación. Sin embargo, me repugnaba el oficio de
cómico, y más aún la idea de verme nuevamente entre personas a quienes
había cobrado cierta repugnancia. Con todo, después de pesar los
inconvenientes y las ventajas, me decidí al fin, y hasta (debo
confesarlo) el pícaro demonio de la vanidad intentó de nuevo asaltar mi
alma, poniendo ante los ojos de mi imaginación la honra, el lustre, el
tono que me daría alternando con tanta gente aristocrática en aquellas
magníficas salas cuyas alfombras no era dado pisar a todos los mortales.
Pero lo que principalmente me indujo a aceptar fue el premio ofrecido,
que era para mí una cantidad fabulosa, un sueño de oro.

---La Providencia divina me envía esos dos mil reales que son diez duros
y otros diez, y otros diez, y otros diez, etc\ldots{} ¡quiá!, si no se
pueden contar. Buen tonto seré si no los cojo.

Dejé a mi ama que al retirarme yo cantaba

\small
\newlength\mlenn
\settowidth\mlenn{   Alons, madamusella}
\begin{center}
\parbox{\mlenn}{   Alons, madamusella                           \\
                asamble reunion,                                \\
                à tour de la butella                            \\
                feran le rigodon}                               \\
\end{center}
\normalsize

\justifying{ \noindent y volví a casa de Inés a quien participé la riqueza que me aguardaba,
prometiendo regalársela. Pasé allí largas horas entristecido por el
espectáculo que ofrecía la pobre enferma doña Juana, cada vez más
empeorada. Al salir a la calle, y cuando pasaba junto al gran portal,
vi que de un enorme carro sacaban telones pintados y otros aparatos
de teatro, los cuales trastos venían, según me dijo el portero, de casa
de don Francisco Goya.}

---Dentro de tres o cuatro días---añadió,---es la función. Ya es seguro
que vendrá la señora duquesa a hacer el papel de Edelmira.

Oído esto me retiré pensando en que tal vez alcanzaría un triunfo
escénico si tenía serenidad suficiente para no asustarme ante público
tan distinguido.

Los ensayos de mi papel empezaron con gran actividad, y el mismo Isidoro
me dio varias lecciones, haciéndome declamar trozo a trozo los
principales y más difíciles pasajes. Entonces pude comprender mejor que
nunca el violento y arrebatado carácter del célebre actor, pues cuando
yo no aprendía un verso tan pronto y tan bien como él deseaba, se
enfurecía llamándome torpe, necio, estúpido, sin omitir otros
calificativos algo más duros y malsonantes. Ensayando, tuve muy presente
la máxima que corría muy válida entre los cómicos del Príncipe, y era
que, representando con Máiquez, convenía trabajar bien, aunque no
demasiado bien, pues en este caso el gran maestro se enojaba tanto como
en el caso contrario.

A los dos o tres días de trabajo ya sabía regularmente mi parte, siendo
mi principal empeño declamar bien el parlamento de salida, cuando el dux
de Venecia me dice:

\small
\newlength\mleno
\settowidth\mleno{Insigne amigo del valiente Otello.}
\begin{center}
\parbox{\mleno}{Insigne amigo del valiente Otello.}             \\
\end{center}
\normalsize

Hubo un ensayo general, al que asistieron todos, menos Lesbia, y me
parece que no lo hice mal. Por mí la representación no debía retrasarse,
y el día 5 ya recitaba del principio al fin mi papel sin que se me
escapara un verso. Según me dijo mi ama, la señora duquesa había venido
de El Escorial el 4 por la noche.

---De modo que nada falta ya.

---Nada---me contestó con la bulliciosa jovialidad que la afectaba por
aquellos días.---¡Y yo dirijo la escena!

\small
\newlength\mlenp
\settowidth\mlenp{   suenen las castañetas,}
\begin{center}
\parbox{\mlenp}{   Donde yo campo                               \\
                nenguno campa.                                  \\
                   A bailar el bolero                           \\
                y asar castañas,                                \\
                apuesto a todo el orbe                          \\
                con la más guapa.                               \\
                   Dale que dale,                               \\
                suenen las castañetas,                          \\
                rabie quien rabie.}                             \\
\end{center}
\normalsize

Llegó por fin el día señalado, y desde por la mañana muy temprano, me
puse en ejercicio, corriendo de aquí para allí en busca de mil cosas que
mi antigua ama necesitaba. Los afeites de la calle del Desengaño, los
trajes pintados en la de la Reina, las telas y cintas cotonías,
muselinetas, pañuelos salpicados de doña Ambrosia de los Linos, todo se
puso en movimiento para dar cumplida satisfacción a los caprichos de
Pepita. Debo advertir que aunque ésta no trabajaba más que como
directora de escena en la tragedia \emph{Otello}, cantaba en el
intermedio una graciosa tonadilla; y como fin de fiesta el sainete
titulado \emph{La venganza del Zurdillo}, del buen Cruz, corría también
por cuenta suya. Mientras desempeñaba yo por Madrid tantas y tan
diferentes comisiones, iba recitando de memoria los versos de la parte
de Pésaro; y cuando se me trascordaba algún pasaje, sacaba el papel del
bolsillo, y metido en un portal, leía en voz alta, llamando la atención
de los transeúntes.

Durante mi largo paseo por la villa, noté grande agitación. La gente se
detenía formando grupos, donde se hablaba con calor; y en alguno de
éstos no faltaba quien leyese un papel, que al punto conocí era la
\emph{Gaceta de Madrid}. En la tienda de doña Ambrosia encontré ¡oh rara
e inexplicable casualidad!, a don Lino Paniagua y a don Anatolio, el
papelista de en frente, cuyos personajes no ocultaban su inquietud por
los acontecimientos del día.

---Ya me esperaba yo tan inaudita perfidia---dijo este último.---¡Cómo
se ve en este decreto la mano alevosa del infame \emph{choricero}!

---Pero léanos usted de una vez el decreto---dijo doña
Ambrosia,---aunque sin oírlo ya sé que el señor Godoy nos habrá hecho
una nueva trastada.

---No es más---continuó el papelista,---sino que han ido a la prisión
del Príncipe, y poniéndole una pistola al pecho, le han obligado a
escribir estas herejías, sí, señores, porque es imposible que un joven
tan caballeroso, tan honrado y de tan buen entendimiento como es el hijo
de nuestros Reyes, se rebaje y se humille hasta el extremo de pedir
perdón como un chico de la escuela, y de acusar tan villanamente a los
que le han ayudado.

---Pero lea usted, señor don Anatolio.

Entonces don Anatolio limpió el gaznate, y con tono de pedagogo leyó el
famoso decreto de 5 de noviembre, que empieza así: \emph{La voz de la
naturaleza desarma el brazo de la venganza, y cuando la inadvertencia
reclama la piedad, no puede negarse a ello un padre amoroso}\ldots{} Lo
notable de este decreto, en que se anunciaba a la nación el
arrepentimiento del Príncipe conspirador, eran las dos cartas que él
había dirigido a la Reina y al Rey, y que casi puedo transcribir aquí
sin echar mano a la historia, donde están para in \emph{aeternum}
consignadas, porque las recuerdo muy bien; tan originales y gráficos
eran el lenguaje y tono en que estaban escritas. Decía así la primera:

«Papá mío: He delinquido, he faltado a Vuestra Majestad como Rey y como
padre; pero me arrepiento y ofrezco a Vuestra Majestad la obediencia más
humilde. Nada debía hacer sin noticia de Vuestra Majestad, pero fui
sorprendido. He delatado a los culpables, y pido a Vuestra Majestad me
perdone por haberle mentido la otra noche, permitiendo besar sus reales
pies a su reconocido hijo, \emph{Fernando}.»

La segunda era como sigue:

«Mamá mía: Estoy arrepentido del grandísimo delito que he cometido
contra mis padres y Reyes, y así con la mayor humildad le pido a Vuestra
Majestad se digne interceder con papá, que permita ir a besar sus reales
pies a su reconocido hijo, \emph{Fernando}.»

En estas cartas aparecía el pobre Príncipe como el más despreciable de
los seres, pues demostrando no tener ni asomo de dignidad en la
desgracia, confesaba que había mentido, y después de delatar a los
culpables, pedía perdón a sus papás, como un niño de seis años que ha
roto una escudilla. Pero entonces los honrados y crédulos burgueses de
Madrid no comprendían que ocurriera nada malo sin que fuera causado por
el atrevido Príncipe de la Paz, y hasta las malas cosechas, los
pedriscos, los naufragios, la fiebre amarilla y cuantas calamidades
podía enviar el cielo sobre la Península, se atribuían al favorito. Así
es que nadie veía en las citadas cartas una manifestación espontánea del
Príncipe, sino antes bien una denigrante confesión arrancada por sus
carceleros, para ponerle en ridículo a los ojos del país entero. Si ésta
fue la intención de la corte, produjo efecto muy contrario al que se
proponían, pues conocido el decreto, el público se puso de parte del
prisionero, y abrumó al valido con su ardiente maledicencia,
suponiéndole autor, no sólo del decreto, sino de las cartas.

---¿Necesita esto comentarios?---dijo don Anatolio, dejando la
\emph{Gaceta} sobre el mostrador.

---Pues yo---dijo doña Ambrosia---quisiera estar oyendo por el agujero
de una llave lo que dice Napoleón de todas estas cosas.

---Eso---indicó con malicioso gesto don Anatolio---no necesitamos oírlo,
pues bien claro es que ya tiene decidido quitar del trono a los Reyes
padres, para ponernos en él a nuestro Príncipe querido. Sí\ldots{} que
no sabrá hacerlo en menos que canta un gallo el buen señor.

---¡Qué escándalo!---exclamó con timidez don Lino Paniagua.---Y eso se
dice en voz alta, donde pudieran oírlo personas allegadas al gobierno.

---¡Bah, bah!---respondió el papelista.---Amigo don Lino, esto se va por
la posta. Dentro de un mes no queda aquí ni rastro del \emph{choricero},
ni Reyes padres, ni escándalos, ni picardías, ni otras cosas que callo
por respeto a la nación.

---Ojalá tenga usted boca de ángel, señor don Anatolio---añadió la
tendera,---y quiera Dios tocarle pronto en el corazón al señor de
Bonaparte, para que venga a arreglar las cosas de España.

El abate don Lino no quiso oír más y se marchó; despacháronme a mí, y
allí quedaron ambos comerciantes arreglando los asuntos de España.

No quise entrar en casa sin hablar un poco con Pacorro Chinitas que
estaba en su sitio de costumbre, afilando cuchillos y tijeras.

---¡Hola, Chinitas!---le dije.---¡Cuánto tiempo que no nos vemos! Anda
la gente muy alarmada por ahí.

---Sí; la \emph{Gaceta} trae hoy no sé qué papel. En la tienda del
buñolero le oí leer y decían todos que era preciso colgar al
\emph{choricero} por los pies.

---¿De modo que creen ha sido escrito por él?

---¿Y a mí qué más me da?---respondió incorporándose.---Lo que digo es
que todos son buenas piezas, y si no vengan acá. Dicen que el ministro
sacó de su cabeza esas cartas y obligó al Príncipe a firmarlas. ¿Pues
para qué las firmó? ¿Es acaso algún niño que todavía está en planas de
primera? ¿No tiene veintitrés años? Pues con veintitrés años a la
espalda se puede saber lo que se firma y lo que no se firma.

Las razones de Chinitas me parecían de un buen sentido incontestable.

---Aunque no sabes leer ni escribir---le dije,---me parece, Chinitas,
que tú tienes más talento que un papa.

---Pues los tenderos, los frailes, los currutacos, los usías, los
abates, y los covachuelistas y toda esa gente que anda por ahí, están
muy entusiasmados creyendo que Napoleón va a venir a poner al Príncipe
en el trono. Dios nos la depare buena.

---Y tú, ¿qué crees, insigne amolador?

---Creo que somos unos archipámpanos si nos fiamos de Napoleón. Este
hombre que ha conquistado la Europa como quien no dice nada, ¿no tendrá
ganitas de echarle la zarpa a la mejor tierra del mundo, que es España,
cuando vea que los Reyes y los príncipes que la gobiernan andan a la
greña como mozas del partido? Él dirá, y con razón: «Pues a esa gente me
la como yo con tres regimientos.» Ya ha metido en España más de veinte
mil hombres. Ya verás, ya verás, Gabrielillo, lo que te digo. Aquí vamos
a ver cosas gordas y es preciso que estemos preparados, porque de
nuestros reyes nada se debe esperar y todo lo hemos de hacer nosotros.

Mucho meollo encerraban, como conocí más tarde, estas palabras, las
últimas que en aquella ocasión oí a Pacorro Chinitas. Él solo había
previsto los acontecimientos con ojo seguro, y en cambio el héroe del
siglo, que conocía a España por sus Reyes, por sus ministros y por sus
usías, quería saberlo todo y no sabía nada. Su equivocación acerca del
país que iba a conquistar se explica fácilmente: supo sin duda lo que
decían doña Ambrosia, don Anatolio, el hortera, el padre Salmón y otros
personajes; pero, ¡ay!, no oyó hablar al amolador.

\hypertarget{xxii}{%
\chapter{XXII}\label{xxii}}

Llegó la noche y la función de la marquesa era preparada con mucha
actividad. Cuando dejé las ropas de mi ama en el cuarto que se le había
destinado para vestirse, por la escalera pequeña subí al sotabanco, y
encontré a Inés muy apesadumbrada, porque los dolores de la enferma se
habían recrudecido y mostraba la buena mujer mucha inquietud. Yo estuve
allí para consolar a mi amiga y a su buen tío todo el tiempo de que pude
disponer; pero al fin me fue forzoso abandonarlos, y bajé a casa de la
marquesa muy afligido.

Describiré aquella hermosa mansión para que ustedes puedan formarse idea
de su esplendor en tan célebre noche. Don Francisco Goya había sido
encargado del ornato de la casa, y casi es excusado elogiar lo que
corría por cuenta de tan sabio maestro. Desde el recibimiento hasta la
sala había adornado las paredes con guirnaldas de flores y festones de
ramaje, hechas aquéllas con papel y éstos con hojas de encina, ambas
obras tan perfectas que nada más bello podía apetecer la vista. Las
lámparas y candelillas habían sido puestas con mucho arte, también en
forma de guirnaldas y festones de diversos colores, su vivo resplandor
daba fantástico aspecto a la casa toda.

El primer salón, de cuyas paredes las modas nuevas no habían desterrado
aún aquellos hermosos tapices, que pasaban de generación a generación,
entre los tesoros vinculados, no perdía con tan espléndidas luminarias
su grave aspecto; antes bien, las luces, dando reflejos extraños a las
armaduras de cuerpo entero que ocupaban los ángulos, visera calada y
lanza en mano, como centinelas de acero, parecían imprimir el movimiento
y el calor de la vida a los imaginarios cuerpos que se suponían dentro
de ellas. Alegres cua dros de toros disipaban la tristeza producida en
el ánimo por otros, en cuyos oscuros lienzos habían sido retratados dos
siglos antes por Pantoja de la Cruz o por Sánchez Coello, hasta una
docena de personajes ceñudos y sombríos, conquistadores de medio mundo.

Con estas joyas del arte nacional contrastaban notoriamente los muebles
recién introducidos por el gusto neoclásico de la Revolución francesa, y
no puedo detenerme a describiros las formas griegas, los grupos
mitológicos, las figuras de Hora o de Nereida o de Hermes que sobre los
relojes, al pie de los candelabros y en las asas de los vasos de flores,
lucían sus académicas actitudes. Todos aquellos dioses menores, que,
embadurnados en oro, renovaban dentro de los palacios los esplendores
del viejo Olimpo, no se avenían muy bien con la desenvoltura de los
toreros y las majas que el pincel y el telar habían representado con
profusión en tapices y cuadros; pero la mayor parte de las personas no
paraban mientes en esta inarmonía.

El salón donde estaba el teatro era el más alegre. Goya había pintado
habilísimamente el telón y el marco que componían el frontispicio. El
Apolo que tocaba no sé si lira o guitarra en el centro del lienzo, era
un majo muy garboso, y a su lado nueve manolas lindísimas demostraban en
sus atributos y posiciones que el gran artista se había acordado de las
musas. Aquel grupo era encantador, pero al mismo tiempo la más aguda y
chistosa sátira que echó al mundo con sus mágicos colores don Francisco
Goya; porque hasta el buen Pegaso estaba representado por un poderoso
alazán cordobés que, cubierto de arreos comunes, brincaba en segundo
término. En el marco menudeaban los amorcillos, copiados con mucho
donaire de los pilluelos del Rastro. No era aquélla la primera vez que
el autor de los \emph{Caprichos} se burlaba del Parnaso.

Pero dejemos los salones y penetremos entre bastidores, donde el
movimiento y la confusión eran tales, que no nos podíamos revolver. Se
habían dispuesto varios cuartos para que los actores se vistieran: a
Máiquez se señaló uno, otro a mi ama, y en el tercero nos vestíamos, sin
distinción de sexos, todos los demás representantes venidos del teatro.
Lesbia tenía por tocador el mismo de la señora marquesa, y los dos
galanes aficionados se vestían en las habitaciones del amo de la casa.
Creo que yo fui el primero que se arregló, trocándome de festivo
Gabrielillo en el sombrío Pésaro, que es el Yago de la inmortal
tragedia. El traje que me pusieron creo que no pertenecía a época alguna
de la historia, y era como todos los que usaron los malos cómicos en las
pasadas edades. Hubiera servido para hacer de paje; pero con las barbas
que me aplicaron a las quijadas, me transformé de tal modo, que los
sastres allí presentes me dieron por el más tétrico y espantable traidor
que había salido de sus manos.

Mientras se vestían los demás, di un paseo por el escenario,
entreteniéndome en mirar al través de los agujeros del telón la vistosa
concurrencia que ya invadía la sala. A quien primero vi fue al joven
Mañara, sentado en primera fila junto al telón. Luego advertí que
hombres y mujeres dirigieron la vista a la puerta principal, apartándose
para dar paso a alguna persona que en aquel momento entraba, y cuya
presencia produjo en el alegre concurso general silencio, seguido
después de un murmullo de admiración. Una mujer arrogante y hermosísima
entró en la sala y avanzaba hacia el centro recibiendo los saludos de
amigos y amigas. Vestía de blanco, con uno de aquellos trajes ligeros y
ceñidos, que llamaban \emph{volubilís}, llevando sobre el pecho una
banda de rosas que la moda designaba con el nombre de \emph{croissures à
la victime}. Su peinado, de estilo griego, era el que en la tecnología
del arte capilar se llamaba entonces \emph{toilette Iphigenie}. A su
hermosura, a la belleza de su vestido, daba mayor realce la artística
profusión de diamantes que encendían mil luces microscópicas en su
cabeza y en su seno. ¿Necesitaré decir que era Amaranta?

Viéndola no tardaron en encenderse dentro de mí, en los oscuros centros
de la imaginación aquellos fuegos vaporosos y tenues, que se me
representan como si una llama alcohólica bailase caracoleando dentro de
mi cerebro. Mientras la contemplaba, no traje a la memoria el
envilecimiento en que habría caído siguiendo en su servicio. Su
hermosura era tan hechicera, tan abrumadora, su actitud tan
orgullosamente noble, el imperio de sus miradas tan irresistible y
despótico, que valía la pena de doblar por un momento la terrible hoja
que yo había leído en el libro de su carácter misterioso. Con tal fijeza
la miraba, que parecía clavado tras el telón: mis ojos trataban de
buscar el rayo de los suyos, seguían los movimientos de su cabeza, y
observándole las facciones y el casi imperceptible modular de sus
labios, querían adivinar cuáles eran sus palabras, cuáles sus
pensamientos en aquel instante. Dentro de poco se alzaría el telón; en
mí se fijarían las miradas de toda aquella brillante muchedumbre y
especialmente de Amaranta; atenderían a mis estudiadas palabras, y el
desarrollo de la acción en que yo tomaba parte, despertaría sin duda la
sensibilidad, el interés, el entusiasmo de tan escogido auditorio. Estos
razonamientos fueron el aguijón que acabó de despabilar la adormecida
vanidad dentro de mí, y lleno de los más necios humos, pensé que hacerse
aplaudir de tantas señoras y caballeros era una gloria cuyos rayos
debían proyectar clarísima luz sobre la vida entera.

La orquesta, comenzando de improviso la sonata que había de preceder a
la representación, hizo llegar al último grado la excitación de mi
cerebro. La sangre circulaba velozmente por mis venas, dándome una
actividad devoradora; y me ocurrió que tener una casa como aquélla,
convidar a tantos y tan nobles amigos, recibir, obsequiar a tal conjunto
de bellas damas, debía ser la mayor satisfacción concedida al mortal
sobre la tierra. Pero la tragedia iba a empezar; el apuntador estaba en
la concha, Isidoro había salido de su cuarto, y la misma Lesbia, menos
asustada de lo que yo suponía, se preparaba a salir a la escena. Esto me
distrajo, y ya no sentí sino miedo. Pasaron algunos minutos y se alzó el
telón.

La tragedia \emph{Otello o el Moro de Venecia}, era una detestable
traducción que don Teodoro Lacalle había hecho del \emph{Otello} de
Ducis, arreglo muy desgraciado del drama de Shakespeare. A pesar de la
inmensa escala descendente que aquella gran obra había recorrido desde
la eminente cumbre del poeta inglés, hasta la bajísima sima del
traductor español, conservaba siempre los elementos dramáticos de su
origen y la impresión que ejercía sobre el público era asombrosa.
Supongo que todos ustedes conocerán la tragedia primitiva, y así me
costará poco darles a conocer las variantes. Los personajes estaban
reducidos a siete. Otello era el mismo. Los caracteres de Casio y
Roderigo habían sido fundidos en una figura de segundo término, llamada
Loredano, que se presentaba como hijo del Dux. El senador Brabantio era
Odalberto y tenía más intervención en la fábula. Desdémona no había
cambiado más que de nombre, pues se llamaba Edelmira; Emilia se trocaba
en Hermancia, y Yago, el traidor y falso amigo del moro, tenía por
nombre Pésaro. La acción estaba muy simplificada, y los recursos
escénicos del pañuelo habían desaparecido, sustituyéndolos con una
diadema y una carta, que debían pasar de las manos de Edelmira a las de
Loredano para que adquiridas luego por Pésaro y presentadas a Otello,
confirmaran la calumnia de aquél. Pero aparte de estas modificaciones y
del estilo y de la expresión y energía de los afectos que desde la obra
inglesa a la española ponían tanta distancia como del cielo a la tierra,
el drama en su estructura íntima era el mismo, y sus escenas se
repartían igualmente en cinco actos. Para abreviar intermedios, Máiquez
dispuso que en aquella representación se reuniesen los actos segundo y
tercero y el cuarto con el quinto, de modo que la obra quedó en tres
jornadas.

En la segunda escena, después que el Dux recitó algunos versos, me
correspondía salir a mí, haciendo en un parlamento no muy largo la
relación de los triunfos militares de Otello. Con voz muy temblorosa
dije los primeros versos:

\small
\newlength\mlenq
\settowidth\mlenq{¡Que no hayan sido vuestros mismos ojos}
\begin{center}
\parbox{\mlenq}{¡Que no hayan sido vuestros mismos ojos         \\
                fieles testigos de su ardor bizarro!}           \\
\end{center}
\normalsize

Pero me fui reponiendo poco a poco, y la verdad es que no lo hice tan
mal, aunque no corresponda a mi pluma el describirlo. Después entraban
en escena Otello y más tarde Edelmira. Nada puedo deciros de la
perfección con que Isidoro dijo ante el senado, el modo y manera con que
encendió la llama amorosa en el corazón de Edelmira; y en cuanto a ésta,
debo desde luego señalarla como consumada actriz, porque en la misma
escena ante el senado, declamó con una sensibilidad que habría envidiado
Rita Luna.

En el primer entreacto debían recitar versos Moratín, Arriaza y Vargas
Ponce. El escenario se había llenado de personajes que deseaban
felicitar a la triunfante Edelmira. Allí vi al diplomático, que no había
desistido al parecer de hacer la corte a mi ama, pues corrió presuroso
tras ella, diciéndole:

---Puede usted estar segura, adorada Pepita, que nuestra pasión quedará
en secreto, pues ya se conoce mi reserva en estas delicadísimas
materias.

Junto con él había subido al escenario don Leandro Moratín, el cual era
entonces un hombre como de cuarenta y cinco años, pálido y serio, de
mediana estatura, dulce y apagada voz, con cierta expresión biliosa en
su semblante como hombre a quien entristece la hipocondría e inquieta el
recelo. En sus conversaciones era siempre mucho menos festivo que en sus
escritos; pero tenía semejanza con éstos por la serenidad inalterable en
las sátiras más crueles, por el comedimiento, el aticismo, cierta
urbanidad solapada e irónica, y la estudiada llaneza de sus conceptos.
Nadie le puede quitar la gloria de haber restaurado la comedia española,
y \emph{El sí de las niñas}, en cuyo estreno tuve, como he dicho, parte
tan principal, me ha parecido siempre una de las obras más acabadas del
ingenio. Como hombre, tiene en su abono la fidelidad que guardó al
Príncipe de la Paz, cuando era moda hacer leña de este gran árbol caído.
Verdad es que el poeta vivió y medró bastante a la sombra de aquél
cuando estaba en pie, y podía cubrir a muchos con sus frondosas ramas.
Si mi opinión pudiera servir de algo, no vacilaría en poner a don
Leandro entre los primeros prosistas castellanos; pero su poesía me ha
parecido siempre, exceptuando algunas composiciones ligeras, un
artificioso tejido, o mejor, un clavazón de durísimos versos, a quienes
no pueden dar flexibilidad y brillo todos los martillos de la retórica.
Moratín además, en materia de principios literarios, tenía toda la
ciencia de su época, que no era mucha; pero aun así, más le hubiera
valido emplearla en componer mayor número de obras, que no en señalar
con tanta insistencia las faltas de los demás. Murió en 1828, y en sus
cartas y papeles no hay indicio de que conociera a Byron, a Goethe ni a
Schiller, de modo que bajó al sepulcro creyendo que Goldoni era el
primer poeta de su tiempo.

Pido mil perdones por esta digresión, y sigo contando. En el escenario
leía Moratín el romance \emph{Cosas pretenden de mí}, que hizo reír a
los concurrentes, porque en él pintaba con mucha gracia la perplejidad
en que le ponían su médico, sus amigos y sus detractores. El romance era
a cada momento interrumpido por afectuosas palmadas, especialmente al
llegar al pasaje en que está la conversación de los pedantes; ¿pero
quién negará que en aquella composición Moratín no hace otra cosa que
una apoteosis de su persona?

Dejemos al grande ingenio asfixiándose en el humo de los plácemes más
lisonjeros, y sigamos la intriga del drama que iba a representarse entre
bastidores, no menos patético que el comenzado sobre las tablas y ante
el público.

\hypertarget{xxiii}{%
\chapter{XXIII}\label{xxiii}}

Al concluir el primer acto, y cuando aún no habían comenzado los poetas
a recitar sus versos, sorprendí a Isidoro en conversación muy viva con
Lesbia. Aunque hablaban en voz baja, me pareció oír en boca del actor
recriminaciones y preguntas del tono más enérgico, y creí advertir en el
rostro de la dama cierta confusión o aturdimiento. Cuando se separaron,
mi desgracia quiso que Lesbia encarase conmigo, interpelándome de este
modo:

---¡Ah, Gabriel! Buena ocasión de hablarte a solas. Ya podrás figurarte
para qué. He estado llena de inquietud desde que supe que había sido
presa la persona\ldots{}

---¡Ah!, usía se refiere a la carta---dije atusándome los bigotes
postizos, para disimular mi turbación.

---Supongo que no iría a manos extrañas. Supongo que la guardarías, y
que la habrás traído esta noche para devolvérmela.

---No señora, no la he traído; pero la buscaré\ldots{} es decir\ldots{}

---¡Cómo!---exclamó con mucha inquietud,---¿la has perdido?

---No señora\ldots{} quiero decir. La tengo allí\ldots{} sólo que
yo\ldots---fue la única respuesta que se me vino a las mientes.

---Confío en tu discreción y en tu honradez---dijo con mucha
seriedad,---y espero la carta.

Sin añadir una palabra más se retiró, dejándome muy entristecido por el
grave compromiso en que me encontraba. Hice propósito de pedir
nuevamente a mi ama que me devolviese la carta, y con esta idea, la
llamé aparte como si fuese a confiarle un secreto, y le supliqué del
modo más enfático que me diese aquel malhadado objeto, cuya devolución
era para mí un caso de honra. Ella se mostró sorprendida, y luego se
echó a reír, diciendo:

---Ya no me acordaba de tu carta. No sé dónde está.

Comenzó el segundo acto, que no me ocupaba más que durante una escena, y
concluida ésta, me retiré al interior del teatro resuelto a poner en
práctica un atrevido pensamiento. Consistía éste en hacer una requisa en
el cuarto de mi ama, mientras ésta se hallase fuera. Cuando la González
me quitó la carta, recién venido de El Escorial, advertí que la guardó
en el bolsillo de su traje. Aquel traje era el mismo que había traído a
casa de la marquesa; mas habiéndose mudado para la representación de la
tonadilla, se lo quitó, y estaba colgado con otras muchas prendas, tales
como mantón, chal, enaguas, etc., en una percha puesta al efecto sobre
la pared del fondo. Era preciso registrar aquellas ropas. Mi ama, que
dirigía la escena, y era la que indicaba las salidas, disponiéndolo
todo, no vendría. Yo había quedado libre por todo el acto segundo. Tenía
tiempo y coyuntura a propósito para lograr mi objeto, y semejante acción
no me parecía muy vituperable, porque mi fin era recobrar por sorpresa
lo que por sorpresa se me había quitado.

Hícelo así, y con tanta cautela como rapidez registré los bolsillos del
traje, de los cuales saqué mil baratijas, aunque no lo que tan
afanosamente buscaba. Ya había perdido la esperanza de conseguir mi
objeto, y casi estaba dispuesto a creer que la carta no volvía a mis
manos por hallarse demasiado guardada o quizás rota y perdida, cuando
sentí acelerados pasos que se acercaban al cuarto. Temiendo que ella me
sorprendiera en tan fea ocupación y no siéndome posible escapar, me
oculté bajo la percha y tras los vestidos, cuyas faldas me ofrecían el
más seguro escondite. Casi en el mismo instante entraron Lesbia e
Isidoro. Aquélla cerró la puerta y ambos se sentaron.

Desde mi escondrijo les veía perfectamente. Máiquez en su traje de
Otello parecía una figura antigua, que animada por misterioso agente, se
había desprendido del cuadro en que la grabara con los más calientes
colores el pincel veneciano. La tinta oscura con que tenía pintado el
rostro fingiendo la tez africana, aumentaba la expresión de sus grandes
ojos, la intensidad de su mirada, la blancura de sus dientes y la
elocuencia de sus facciones. Un airoso turbante blanco y rojo, sobre
cuya tela se cruzaban filas de engastados diamantes, le cubría la
cabeza; collares de ámbar y de gruesas perlas daban vueltas a su negro
cuello, y desde los hombros hasta el tobillo le cubría un luengo traje
talar de tisú de oro, ceñido a la cintura y abierto por los costados
para dejar ver las calzas de púrpura estrechamente ajustadas. Alfanje y
daga, ambos con riquísima empuñadura, cuajada de pedrerías pendían del
tahalí, y en los brazos desnudos, que imitaban el matiz artificial de la
cara con una finísima calza de punto color de mulato, y terminada en
guante para disfrazar también la mano, lucían dos gruesas esclavas de
bronce en figura de sierpe enroscada. Dábale la luz de frente, haciendo
resplandecer las facetas de las mil piedras falsas, y el tornasol de
tisú verdadero con que se cubría, y añadidas a estos efectos la
animación de su fisonomía, la nobleza de sus movimientos, presentaba el
más hermoso aspecto de figura humana que es posible imaginar.

Lesbia vestía de tisú de plata, con tanta elegancia como sencillez, y
sus cabellos de oro, peinados a la antigua, obedeciendo más bien a la
moda coetánea que a la propiedad escénica, se entrelazaban con cintas y
rosarios de menudas perlas, no ciertamente falsas como las de Isidoro,
sino del más puro y fino oriente. El moro, apretando con sus negras
manos las de Lesbia blanquísimas y finas, le dijo:

---Aquí nos podemos hablar un instante.

---Sí, Pepa nos ha dicho que podríamos vernos en su cuarto---repuso
ella;---pero esta cita no ha de ser larga, porque la marquesa me espera.
Ya sabes que está ahí mi marido.

---¿A qué esa prisa? ¿Por qué no me escribiste desde El Escorial?

---No pude escribir---repuso ella con impaciencia;---pero cuando
hablemos despacio, te explicaré\ldots{}

---Ahora, ahora mismo has de contestar a lo que te pregunto.

---No seas tonto. Me prometiste no ser impertinente, curioso, ni
pesado---dijo con coquetería.

---Eso es lo mismo que prometer no amar, y yo te amo, Lesbia, te amo
demasiado por mi desgracia.

---¿Estás celoso, Otello?---preguntó la dama, y luego, tomando el tono
trágico, dijo entre burlas y veras:

\small
\newlength\mlenr
\settowidth\mlenr{mi corazón reserva su cariño!}
\begin{center}
\parbox{\mlenr}{¡Otello mio! ¡Sí, para ti solo
                mi corazón reserva su cariño!}                  \\
\end{center}
\normalsize

---Déjate de bromas. Estoy celoso, sí, no puedo ocultártelo---exclamó el
moro con viva ansiedad.

---¿De quién?

---¿Y me lo preguntas? Piensas que no he visto a ese necio de Mañara
puesto en primera fila, y mirándote como un idiota.

---¿Y no te fundas más que en eso? ¿No tienes otros motivos de sospecha?

---Pues si tuviera otros, desgraciada, ¿estarías con tanta calma delante
de mí?

---Poquito a poco, señor Otello. ¿Sabes que te tengo miedo?

---En El Escorial ese joven se ha jactado públicamente de que le
amas---afirmó Isidoro, fijando tan terriblemente sus ojos en el rostro
de Lesbia, que parecía querer penetrar hasta el fondo del alma.

---Si te pones así, me marcho más pronto---dijo Lesbia algo
desconcertada.

---He recibido varios anónimos. En uno se me decía que ese joven te
escribió una carta el día de su prisión, y que tú le contestaste con
otra. Además yo sé que ese hombre te obsequia mucho, yo sé que te
visitaba en Madrid. ¿Querrás darme explicación sobre esto?

---¡Ah!, tengo una grande y terrible enemiga, a quien supongo autora de
los anónimos que has recibido.

---¿Quién es?

---Ya te he hablado de esto en otra ocasión. Es Amaranta; y también te
he dicho que tras de la enemistad de la condesa, se esconde el odio de
otra persona más alta. Todas las damas que en otro tiempo le servimos
con fidelidad estamos cansadas de presenciar las liviandades que han
manchado el trono, y no queremos asociarnos a los escándalos que
envilecen esta pobre nación. No te he contado el motivo de nuestra
querella; pero ahora mismo la vas a saber, y no te enfades si oyes el
nombre de ese mismo Mañara, a quien tanto temes. Parece que Mañara
rechazó, cual otro José, los halagos de la elevada persona, cuya pasión
se trocó con esto en odio vivísimo y deseo de venganza. Al mismo tiempo
ese joven dio en hacerme la corte, y la mujer ofendida descargó sobre mí
su rencor, cuando yo ni siquiera había advertido que Mañara me amaba.
Jamás me fijé en semejante hombre. Se emprendió contra mí una guerra
terrible y solapada: quitaron sus destinos a cuantos habían sido
colocados por mi mediación, y todo su afán se dirigía a buscar los
medios de deshonrarme. Viéndome perseguida sin motivo, me hice
partidaria del Príncipe de Asturias, ofrecí mi auxilio a los
conspiradores, y tengo la satisfacción de haber servido eficazmente tan
noble causa. A ti puedo revelártelo sin miedo: yo he sido depositaria
durante algún tiempo, de la correspondencia establecida entre el
canónigo Escóiquiz y el embajador de Francia: en mi casa se reunieron
éstos varias veces con otros personajes: yo sola tenía noticia de las
primeras conferencias celebradas en el Retiro; yo poseía el secreto de
todos los planes descubiertos por una simpleza del Príncipe; yo conocía
el proyecto de casarle a éste con una princesa imperial; sabía que el
duque del Infantado no esperaba más que la orden firmada por Fernando
para lanzar a la calle tropa y pueblo\ldots{} en fin, lo sabía todo.

---Todo cuanto me dices parece inverosímil---dijo Isidoro.---Si es
cierto, ¿cómo no te han perseguido abiertamente, cómo te pusieron en
libertad a la media hora de estar presa?

---Ya sabía yo que no sería molestada. Poseo un escudo terrible que me
defiende contra las asechanzas de la camarilla. Creo haberte contado que
cuando intervine en la primera reconciliación de Godoy, cuando intenté,
por superior encargo, de atraerle de nuevo a palacio, fui depositaria de
secretos, cuya publicación haría estremecer de espanto a ciertas
personas. Poseo papeles que rebajan y envilecen del modo más repugnante
a quien los escribió, y conozco el secreto de la inversión de fondos de
obras pías que se emplearon en lo que no tiene nada de piadoso. Esto
pasó en una época en que hacíamos excursiones clandestinas fuera de
palacio, cuando Amaranta hizo que Goya la retratase desnuda. Hacía un
año que estaba viuda: fue cuando por una coincidencia providencial
descubrí el gran secreto de su juventud, que me reveló una mujer
desconocida que vive orillas del Manzanares, junto a la casa del pintor.
Ya te lo he dicho y pienso hacer de manera que nadie lo ignore. De un
desgraciado y oculto amor que padeció Amaranta antes de su matrimonio
con el conde, nació una criatura que no sé si vive todavía.

---Nunca me hablaste de eso.

---Los padres de Amaranta supieron disimular su deshonra: el joven
amante, que pertenecía a una noble familia de Castilla y había venido a
Madrid buscando fortuna, huyó a Francia y fue muerto en las guerras de
la República.

---Me has referido una curiosa novela---dijo Isidoro;---¡pero con cuánto
arte has desviado la conversación del asunto principal! Al fin confiesas
que Mañara te ha hecho la corte.

---Sí, pero jamás he pensado en corresponderle; ni le trato, ni le veo,
ni le hablo. Tus celos harán que por primera vez me fije en semejante
hombre.

---No, no me convences, no: yo tengo indicios, tengo noticias de que tú
amas a ese hombre. ¡Oh!, si mis sospechas se confirmaran\ldots{} ¿Crees
que no he advertido el embobamiento con que atiende a tu declamación?

---Procuraré entonces hacerlo mal para no conmover al público.

---No, no intentes disculparte ni disimular. ¿Por qué aseguras que no te
fijas en él, si yo mismo, durante la escena del senado, te he
sorprendido mirándole, y aún me parece que le hiciste alguna seña?

---¿Yo?, ¡estás loco! ¡Ah!, no sabes. Mi marido, que dejó sus cacerías
para asistir a esta representación, está ahí esta noche, y la pérfida
Amaranta, sentada a su lado, le habla con mucho interés. Si me ves que
miro al público es porque me inspiran mucha inquietud los coloquios del
duque con Amaranta. Temo que ésta le haya dirigido también algún
anónimo. Su frialdad y ademán sombrío me indican que sospecha.

---¿Lo ves?\ldots{} Y con motivo fundado.

---Sí; porque sospecha de ti.

---No\ldots{} no---exclamó Isidoro.---No trastornes la cuestión. Tú amas
a Mañara; con todos tus artificios no puedes arrancar esta sospecha de
mi ardiente cerebro. ¡Y ese necio está ahí, gozándose en los aplausos
que te prodigan, que adulan su amor propio porque se siente amado de la
gloriosa artista! ¡No, no quiero que representes más! ¡Cuando contemplo
desde arriba el entusiasmo de tus admiradores, cuando les veo con los
ojos fijos en ti, participando de la pasión que indican tus palabras,
siento impulsos de saltar del escenario para cerrarles a golpes los ojos
con que te miran!

---Me haces estremecer---dijo Lesbia.---No eres Isidoro, eres Otello en
persona. Sosiégate por Dios. Harto sabes lo mucho que te amo. ¿A qué me
mortificas con celos ilusorios?

---Disípalos tú.

---¿Cómo, si ninguna razón te convence? Tu violento carácter ha de
traerme algún compromiso. Modérate, por Dios, y no seas loco.

---Lo haré si me amas. Tú no sabes quién soy. Isidoro, no consientas
rivales ni en la escena, ni fuera de ella. De Isidoro no se ha burlado
hasta ahora ninguna mujer, ni menos ningún hombre. Entiéndelo bien.

---Sí, señor mío, estoy en ello---contestó Lesbia en tono jovial y
levantándose para retirarse.---Pero aunque esta conversación me agrada
mucho, tengo que irme. ¿Sabes que te tengo miedo?

---Quizás con razón. ¿Pero te vas tan pronto?---dijo el moro intentando
detenerla aún.

---Sí, me voy---repuso Lesbia.---Ya ha concluido la tonadilla, y pronto
empezará el tercer acto.

Y ligera como una corza se marchó. En aquel instante se oyeron los
aplausos con que era saludada mi ama al acabar la tonadilla, y poco
después entró en su cuarto radiante de júbilo, con el rostro encendido
por la emoción, y tan sofocada que al punto dio con su cuerpo en un
sofá.

\hypertarget{xxiv}{%
\chapter{XXIV}\label{xxiv}}

---¡Oh, Isidoro! ¿Por qué no has querido oírme?---exclamó con
entrecortadas palabras.---Aseguran que lo he hecho muy bien. ¡Cuánto me
han aplaudido!

---¿Quieres dejarte de simplezas?---dijo Isidoro de muy mal talante.

---Y a propósito: dicen que Lesbia hace la Edelmira mejor que yo. ¡Lo
que puede la hermosura! Con su buen palmito trae sin seso a todos los
hombres que hay en la sala. Sobre todo, ahí está uno que no le quita la
vista de encima, y parece\ldots{}

---¡Quieres callar!---exclamó bruscamente el moro.

Después, como hombre que toma repentina resolución, se disipó el
fruncimiento temeroso de sus negras cejas, y sentándose junto a la
González, le habló en estos términos:

---Pepa, espero de ti un favor.

---Mándame lo que quieras.

---Siempre te has mostrado muy agradecida por todo lo que he hecho en
beneficio tuyo. Varias veces has dicho: «¿Qué he de hacer, Isidoro, para
corresponder a lo que te debo?» Pues bien, chiquilla, ahora puedes
prestarme un gran servicio, con lo cual quedará pagado largamente el
hombre que te sacó de la miseria, el que te enseñó el arte escénico,
dándote posición, gloria y fortuna.

---Mi agradecimiento durará mientras viva, Isidoro---respondió la cómica
con serenidad.---¿Qué necesitas ahora de mí?

---Si la contrariedad que experimento afectara sólo a mi corazón, la
resolvería fácilmente, porque sé padecer. Pero tal vez afecte a mi amor
propio, tal vez ponga en trance muy terrible mi dignidad, y me resigno a
sufrir los desengaños más crueles; pero de ningún modo consiento en
hacer ante mis amigos y el mundo un papel desairado y ridículo.

---Ya sé lo que quieres decir. Lesbia me ha dicho que estás celoso; ¡si
vieras cómo se ríe de ti, llamándote el \emph{pobre Otello!}

---No debemos fiarnos de la afición que alguna vez nos muestran esas
personas tan superiores a nosotros por su clase. Un abismo nos separa de
ellas, y si alguna vez las deslumbramos con nuestro talento y nuestro
arte, la ilusión les dura poco tiempo, y concluyen despreciándonos,
avergonzadas de habernos amado. Todos los que hemos brillado en la
escena conocemos tan triste verdad. ¿No la conoces tú también?

---Sí---dijo mi ama;---y yo creí que tú estuvieras en esa parte más
aleccionado que todos los demás.

---Esas personas---prosiguió Isidoro,---nos contemplan desde sus
aposentos; su imaginación se trastorna viéndonos remedar los grandes
caracteres, las nobles y elevadas pasiones, el amor, el heroísmo, la
abnegación, y se enamoran de lo que ven, de un ser ideal en quien se
asocia y confunde con nuestra persona, la del héroe que representamos.
Con la imaginación excitada, nos buscan entre bastidores y fuera del
teatro; pero en cuanto nos tratan un poco y advierten que somos lo
mismo, si no peores que los demás, y que todas las sublimidades del arte
escénico desaparecen con el vestido y las piedras falsas que arrojamos
al concluir el drama, se disipa de un soplo su entusiasmo y no ven en
nosotros más que a una turba de tramposos y embusteros farsantes que
apenas valen el partido con que se les paga. Hasta ahora, Pepilla, no me
habían afectado gran cosa los bruscos desenlaces de las aventuras con
que algunas ilustres personas han honrado nuestra profesión; pero ésta
en que ahora me hallo me afecta profundamente, porque\ldots{} te lo diré
con toda franqueza.

---¿Amas verdaderamente a Lesbia?

---Sí, por mi desgracia; esta pasión no es de aquéllas pasajeras y
superficiales, que pasan satisfaciendo el afán de un día. Esa mujer ha
tenido el arte de ahondar en mi corazón de tal modo, que hoy empiezo a
reconocer en mí el embrutecimiento que acompaña a los amores exaltados.
Sin duda su coquetería, su frivolidad, los mil artificios de su voluble
carácter han realizado en mí este trastorno, y para acabarme de
confundir, los celos, la desconfianza y el temor de ser ridículamente
suplantado por otro, agitan mi alma de tal modo, que no respondo de lo
que podrá pasar.

---¡Hola, hola!, señor Otello, ¿esas tenemos?---dijo mi ama
festivamente.---¿A quién va usted a matar?

---No te rías, loca---continuó el moro.---¿Has visto en el salón a ese
miserable Mañara?

---Sí; ocupa un sillón de primera fila, y no quita los ojos de la señora
Edelmira. Verdaderamente, chico, y sin que esto sea confirmar tus
sospechas, a todos los que están en el teatro ha llamado la atención el
exagerado entusiasmo de ese joven, y más de cuatro han sorprendido las
señas que hace a Lesbia durante la comedia. Y además\ldots, yo no lo he
visto, pero me han dicho que\ldots{}

---¿Qué te han dicho?

---Que la duquesa le mira mucho también, y que parece representar sólo
para él, pues todas las frases notables del drama las dice volviéndose
hacia el tal joven, como si quisiera arrojarse en sus brazos.

---¡Oh! Es cierto. ¡Ves!---exclamó Isidoro bramando de furor.---¡Y se
reirán todos de mí!, y ese vil currutaco\ldots{} ¡Ah! Pepa\ldots{}
quiero descubrir fijamente lo que hay en esto\ldots{} quiero acabar de
una vez estas terribles dudas\ldots{} Quiero desenmascarar a esa infame,
y si me engaña, si ha sido capaz de preferir al amor de un hombre como
yo, los necios galanteos de ese vil y despreciable mozuelo\ldots{} ¡ah!
Pepa, Pepa, mi venganza será terrible. Tú me ayudarás en ella; ¿no es
verdad que me ayudarás? Tú me lo debes todo, yo te saqué de la miseria;
tú no puedes negar a Isidoro la ayuda de tu ingenio para este fin, y
proporcionándome placer tan inefable, quedarás descargada de la inmensa
deuda de gratitud que tienes conmigo.

Al decir esto, Isidoro se había levantado y daba vueltas en la pequeña
habitación como un león enjaulado, pronunciando con trémulo labio
palabras rencorosas. Lo raro fue que mi ama, ya porque tal fuera el
estado de su espíritu, ya porque creyera oportuno fingir en aquellos
momentos, lejos de amedrentarse al ver la ira de su amigo y maestro,
contestó con risas a sus ardientes palabras.

---Te ríes---dijo Máiquez deteniéndose ante ella.---Haces bien: ha
llegado el momento de que hasta los mete-sillas del teatro se rían de
Isidoro. Tú no comprendes esto, chiquilla---añadió sentándose de
nuevo.---Tú no tienes vehemencia ni fogosidad en tus sentimientos. En
esto te admiro, y quisiera imitarte, porque yo sé muy bien que en las
inclinaciones que hasta ahora se te han conocido, has jugado con el
amor, tomándolo como un pasatiempo divertido, que entretiene a uno mismo
y hace rabiar a los demás; pero hasta ahora, y Dios te libre de ello, no
conoces el amor que ocasiona las mortificaciones propias, mientras los
demás se ríen a costa nuestra.

---¡Qué orgulloso eres!---contestó seriamente la González.---Hasta en
esto quieres saber más que todos.

---Pues si amas de veras, guárdate de enamorarte de esos usías
presumidos y orgullosos, que vendrán a ti para satisfacer su vanidad.
Ellos no te amarán con noble y desinteresado amor.

---No creo que jamás pueda amar sino al que siendo igual a mí, no se
avergüence de tenerme por compañera.

---¡Oh, qué buen sentido, Pepilla! ¿Dónde has aprendido eso? Pero te
aconsejo también que no ames a ningún hombre de teatro, si no quieres
tener rabiosos celos de todo el público femenino. ¿Sabes tú lo que es
eso?

---Harto lo sé.

---De modo que tu amor aún está dentro del teatro. Eso sí que es una
desgracia. Tu suerte consistirá en que el galán será de esos que, por
falta de genio, no excitan nunca la arrebatada admiración de las bellas
de la platea. Serás feliz, Pepilla; si quieres casarte, cuenta con mi
protección.

---Estoy muy lejos de aspirar a eso.

---¿Ese bruto será capaz de no amarte? ¿Acaso vale más que tú?

---Muchísimo más---dijo la González aparentando con grandes esfuerzos la
serenidad que no tenía.

---Apuesto a que es algún tenor de la compañía de Manolo García. Déjalo
por mi cuenta. Si es cierto lo que supongo, si ese loco no te
corresponde y prefiere a tu sencillo cariño el falso amor de alguna
damisela de éstas que arrastran su púrpura por entre los bastidores del
teatro, ya sabrás lo que son celos, ¿eh?

---Demasiado lo sé y demasiado padezco, Isidoro---dijo mi ama en tono de
cariñosa confianza;---pero yo tengo una ventaja sobre ti, que no
poseyendo aún la certeza de tu desgracia, ignoras qué partido tomar; yo
conozco ya, sin género de duda que no soy amada, y las circunstancias se
han ordenado de tal modo, que me presentan ocasión de tomar venganza.

---¡Oh! Pepa; estás desconocida. No te creí capaz\ldots---indicó Isidoro
con energía.---Tú tomarás venganza. Descuida, te ayudaré, si tú me
ayudas a mí en la averiguación y en el castigo de las infamias de
Lesbia. Pero dime, chiquilla, dime quién es ese hombre. Sé franca
conmigo; yo soy tu mejor amigo.

---Te lo diré más tarde, Isidoro. Por ahora me he propuesto guardar
secreto.

---Tú vales mucho, Pepilla---añadió el cómico con acento reflexivo.---No
esperaba encontrar en ti un eco tan fiel de lo que en mí está pasando.
¡Y ese miserable te desprecia por otra, ignorando las bondades de tu
fiel corazón! Dime quién es. ¿Será el mismo Manuel García? Por supuesto,
chiquilla, ya sabrás cuánto padece la dignidad, el amor propio, al ver
que otra persona posee el afecto que nos pertenece. Te mortificará
horriblemente la idea de la triste figura que harás ante el mundo, el
pensamiento de los comentarios que hará sobre tu ridícula posición el
envidioso vulgo, y al considerar que tú, la persona acostumbrada a
rendir a tus pies los corazones, se ve menospreciada por uno solo,
rabiará tu orgullo herido, y llorarás en silencio, viéndote más baja de
lo que creías.

---En esto---contestó mi ama con patética voz,---no nos parecemos. Tú
estás frenético de celos; pero antes que al desaire de que ha sido
objeto tu corazón, atiendes a lo que sufre tu dignidad, la dignidad del
gran Isidoro, que siempre desprecia sin ser nunca despreciado; te
enfureces al considerar que se ríen de ti los envidiosos, y esas
terribles voces de venganza no las pronuncia tu amor, sino tu orgullo.
Yo no soy así: amo el secreto; y si triunfara, gustaría de tener oculta
mi felicidad: nada me importaría que el hombre a quien amo, aparentara
galantear a todas las mujeres de la tierra, con tal que en realidad a
ninguna amase más que a mí.

---Eres singular, Pepilla, y me estás descubriendo tesoros de bondad que
no sospechaba existiesen en tu corazón.

---Yo---continuó mi ama más conmovida,---no vivo más que para él, y los
demás me importan poco. Contigo debo ser franca y decírtelo todo, menos
su nombre, que nadie debe saber. Yo no sé cómo ni cuándo empezó mi
funesto amor, y me parece que nací con esta viva inclinación, más
dominadora cuanto más intento sofocarla. Por él sacrificaría gustosa mi
vida. Tú quizás no comprendas esto; ni menos que yo sacrifique mi
reputación de artista, el aprecio y la admiración de la multitud. ¿Qué
importa todo eso? Se ama a la persona por la persona y no por la vanidad
de poseerla.

---El que te ha inspirado tan noble cariño, sin corresponder a él---dijo
Isidoro con brío,---es un miserable que merece arrastrar su existencia
despreciado de todo el mundo. ¿No puedo saber tampoco quién es la mujer
preferida?

---Tampoco debes saberlo---replicó mi ama, y después, no pudiendo
contener el llanto, exclamó así:---Yo no soy cruel; yo no deseaba una
venganza que puede ser muy terrible; pero se me ha venido a las manos y
he de llevarla adelante.

---Haces bien---dijo Isidoro recreándose con pensamientos de
exterminio.---Véngate: yo también me vengaré. Nos ayudaremos el uno al
otro. ¿Puedo servirte de algo?

---De mucho---dijo mi ama secando sus lágrimas.---Espero que tu ayuda
será de la mayor eficacia.

---¿Y yo puedo contar contigo?

---¿Y me lo preguntas?

---Oye bien: Lesbia confía en tu amistad. ¿No ha celebrado en tu casa
entrevista alguna con ese joven?

---Hasta ahora no.

---Pues la celebrará. Si ella no te lo propone, propónselo tú con buenos
modos.

---¿Cuál es tu objeto?

---Sorprenderla en algún sitio con ese Mañara. Ella busca siempre las
casas de las amigas que no son de su clase, para evitar de este modo la
vigilancia de su familia y de su esposo.

---Entiendo.

---Confío en que no te dejarás sobornar por ella, y en que ante todas
las consideraciones, será para ti la primera el servicio que me prestes,
a mí, tu protector, tu amigo. Espero que te será muy fácil lo que
propongo. Si van a tu casa, les entretienes allí, y me avisas. Yo haré
de manera que ese joven se acuerde de mí para toda la vida.

---Ya tiemblas de gozo, al pensar en tu venganza---dijo mi ama.---Lo
mismo me pasa a mí; pero con más motivo, porque la mía está más cercana.

---¿Puedo confiar en ti? ¿Me pondrás al corriente de todo cuanto veas?

---Puedes estar tranquilo, Isidoro. Tú no me conoces bien: en esta
ocasión sabrás lo que soy.

---Y tú ¿qué crees?---preguntó el moro con interés.---¿Crees que tengo
razón? ¿Lesbia amará a ese hombre?

---Sí; creo que te engaña del modo más miserable; creo que todos los que
asisten a la representación se ríen de ti esta noche y el afortunado
amante no cabe en sí de satisfacción y orgullo.

---¡Rayos y centellas!---dijo Máiquez con más furia.---Le escupiré la
cara desde el escenario. ¡Oh! Pepilla: yo admiro y envidio tu
tranquilidad. No desees nunca parecerte a mí; ojalá no sepas nunca lo
que son estas culebras de fuego que se enroscan dentro de mi pecho y
desparraman por mis arterias su veneno. ¡Oh, qué gran talento tuvo ese
poeta inglés que inventó el Otello! ¡Qué bien pintó la rabia del celoso,
la horrible fruición con que se recrea, pensando que ha de poner el
cuerpo inanimado y sangriento de su rival ante los ojos que le
cautivaron! ¡Qué razón tuvo al suponer el corazón de la mujer antro de
maldades y perfidias; qué bien se comprende la espantosa determinación
del moro, y el terrible placer de su alma al considerarse sepultando el
cuchillo en los miembros palpitantes de quien le ofendió, y arrastrar
después su infame cadáver!

---¿Qué cadáver, Isidoro? ¿El de él o el de ella?---preguntó mi ama con
frialdad.

---El de los dos---contestó Otello cerrando los puños.---¿Conque dices
que se ríen de mí? ¡Y lo saben todos, y me observan, y estoy sirviendo
de espectáculo a ese miserable zascandil! De modo que Isidoro es el
hazme reír de las gentes, y tendrá que ocultarse y huir para evitar las
burlas de los envidiosos, y ya ninguna mujer se dignará mirarle a la
cara. Pero tú, si sabías esto que pasa, ¿por qué no me lo dijiste? ¡Eres
tonta sin duda! ¡Oh!, no tengo amigos verdaderos\ldots{} nadie se
interesa por mi honor ni por mi decoro. ¡Estoy solo!\ldots{} pero solo
¡vive Dios!, sabré volver al lugar que me corresponde.

Diciendo esto, se levantó con resuelto ademán. En aquel momento sonaron
algunos golpes en la puerta: era la señal que llamaba a todos los
actores para empezar el tercer acto. Máiquez iba a salir; pero al dar
los primeros pasos un objeto cayó de su cintura al suelo. Era la daga
con puño de metal y hoja de madera plateada: Pepa, durante la
conversación había estado jugando con la larga cadena que la sostenía y
ésta se rompió.

---Se ha saltado un eslabón---dijo mi ama recogiendo el arma:---yo te la
compondré en seguida atándola fuertemente.

Isidoro salió, y mi ama, acercándose a una mesa arrimada a la pared de
en frente, se entretuvo durante un rato y con mucha prisa en una
operación que no pude ver; pero presumí fuera la compostura de la cadena
rota. Al fin salió, y quedándome solo, pude dejar mi sofocante escondite
para correr a la escena.

\hypertarget{xxv}{%
\chapter{XXV}\label{xxv}}

Dio principio el último acto, donde ocurren las principales escenas del
drama. En él Pésaro despierta poco a poco los celos en el alma del
crédulo moro hasta que, engañándole con cruel y mañosa calumnia,
precipita el trágico desenlace. La importancia de mi papel me obligaba,
pues, a fijar en él toda mi atención, apartándola de las impresiones
recientemente recibidas. Durante mi primera escena con Otello, advertí
que Máiquez, inquieto y receloso, dirigía sus miradas al joven Mañara,
sentado muy cerca del escenario: a causa de la ansiedad de su alma, el
gran histrión desatendía impensadamente la representación. A veces
algunas de mis frases se quedaban sin réplica; también suprimía él
bastantes versos, y hasta llegó a trabarse su expedita lengua en uno de
los pasajes donde acostumbraba hacerse aplaudir más. El auditorio estaba
descontento, pues aunque conocía las genialidades de Isidoro, no creía
natural que se permitiera tales descuidos en una representación de
confianza y amistad verificada ante lo más selecto de sus admiradores.
El silencio reinaba en la sala, y sólo un sordo murmullo de sorpresa o
enfado acogía los versos, mal sentidos y fríamente dichos por el
príncipe de nuestros actores.

Mas se esperaba verle repuesto en la segunda escena entre Otello y
Pésaro. Éste, urdiendo muy bien la trama que ideó contra Edelmira su
diabólica astucia, adquiere al fin las pruebas materiales que Otello
exige para creer en la infidelidad de la veneciana. Aquellas pruebas son
una diadema entregada por Edelmira a Loredano, y cierta carta que su
padre le obligó a firmar, amenazándola con matarse si no lo hacía. Ni la
entrega de la diadema ni la carta firmada por fuerza eran pruebas que
ante la fría razón comprometerían el honor de la esposa de Otello: pero
éste, en su ciego arrebato y salvaje impetuosidad, no necesitaba más
para caer en la trampa.

Antes de comenzar esta escena, y hallándome entre bastidores, oí a los
concurrentes quejarse de la torpeza de Isidoro, y alguno achacó este
defecto no al gran actor, sino a mí, por haberle irritado con mi
detestable declamación. Esto me ofendió, y creyéndome autor del
deslucimiento de la pieza, resolví hacer todos los esfuerzos de que era
capaz para arrancar algún aplauso.

Mi ama, como he dicho, dirigía la escena; indicaba las entradas y
salidas; cuidando de entregar a cada actor los objetos de que debía
hacer uso durante la representación. Diome la diadema y la carta y salí
en busca de Otello que estaba solo en las tablas concluyendo su
monólogo. Entonces empecé aquella grandiosa escena, que es patética,
sublime y arrebatadora aun después de haber sido tamizada por el romo
ingenio de don Teodoro La Calle.

\small
\newlength\mlens
\settowidth\mlens{—¿Sabes tú padecer?}
\begin{center}
\parbox{\mlens}{\textit{—¿Sabes tú padecer?}}                   \\
\end{center}
\normalsize

\justifying{\noindent le dije; y al punto Isidoro, mirándome sombríamente, repuso:}

\small
\newlength\mlent
\settowidth\mlent{—Y sin agitación—dije yo,—¿el triste aviso}
\begin{center}
\parbox{\mlent}{                       \textit{—Me han enseñado.}               \\
                \textit{—Y sin agitación}—dije yo,—\textit{¿el triste aviso}  \\
                \textit{de un infortunio grande escuchar puedes?}               \\
                \textit{—Hombre soy.—}}                                         \\
\end{center}
\normalsize

\justifying{\noindent respondió con calma.}

Continuó el diálogo, y parecía que Isidoro recobraba todo su genio, pues
los versos, inspirados por el recelo y la ansiedad, le salían del fondo
del alma. Cuando dijo:

\small
\newlength\mlenu
\settowidth\mlenu{¡Infeliz!, ¡la prueba necesito!}
\begin{center}
\parbox{\mlenu}{¡Infeliz!, ¡la prueba necesito!                   \\
                ¡Conque dámela luego!}                            \\
\end{center}
\normalsize

\justifying{\noindent me apretó tan fuertemente la muñeca y sus rabiosos ojos me miraron con tanta
furia, que perdí la serenidad, y por un instante los versos que seguían
a aquella demanda huyeron de mi memoria. Pero no tardé en reponerme: le di la
diadema, y poco después la carta.}

Mas en el momento en que vi en sus manos el fatal papel, un súbito
estremecimiento sacudió todo mi ser, y me quedé mudo de espanto. En el
color y en los dobleces del papel, en la forma de la letra, que
distinguí claramente cuando él fijó en ella la vista, reconocí la carta
que Lesbia me había dado en El Escorial para Mañara, y que después mi
ama sustrajo de mis ropas al llegar a Madrid.

Otello debía leer en voz alta la carta, que según el drama decía: Padre
mío: Conozco la sin razón con que os he ultrajado. Vos sólo tenéis
derecho de disponer de vuestra hija, \emph{Edelmira}. Pero el pliego que
la pícara Pepa había hecho llegar a sus manos, decía: Amado Juan: Te
perdono la ofensa y los desaires que me has hecho; pero si quieres que
crea en tu arrepentimiento, pruébamelo viniendo a cenar conmigo esta
noche en mi cuarto, donde acabaré de disipar tus infundados celos,
haciéndote comprender que no he amado nunca, ni puedo amar a Isidoro,
ese salvaje, presumido comiquillo, a quien sólo he hablado alguna vez
con objeto de divertirme con su necia pasión. No faltes si no quieres
enfadar a tu \emph{Lesbia}.

P. D. No temas que te prendan. Primero prenderán al Rey.

Ocurrió una cosa singular. Isidoro leyó el papel en silencio; sus labios
secos y lívidos temblaron, y como si aún creyera que era ilusión lo que
veía, lo leyó y releyó de nuevo mientras el público, ignorando la causa
de aquel silencio, mostró su asombro en un sordo murmullo. Isidoro al
fin alzó la vista, se pasó las manos por la frente; parecía despertar de
un sueño; balbuceó algunas voces terribles, cerró los ojos, como
tratando de serenarse y reanudar su papel; dio algunos pasos hacia el
público y retrocedió luego. Los rumores aumentaron; el apuntador le
llamó repitiendo con fuerza los versos, hasta que al fin Isidoro se
estremeció todo, su semblante se encendió vivamente, cerró los puños,
agitó los brazos, golpeó el suelo, y declamó los terribles versos
siguientes:

\small
\newlength\mlenv
\settowidth\mlenv{pues yo quiero empaparlos, sumergirlos,}
\begin{center}
\parbox{\mlenv}{   Mira: ves el papel, ves la diadema;            \\
                pues yo quiero empaparlos, sumergirlos,           \\
                en la sangre infeliz y detestable,                \\
                en esa sangre impura que abomino.                 \\
                ¿Concibes mi placer, cuando yo vea                \\
                sobre el cadáver, pálido, marchito,               \\
                de ese rival traidor, de ese tirano,              \\
                el cuerpo de su amante reunido?}                  \\
\end{center}
\normalsize

Jamás estos versos se habían declamado en la escena española con tan
fogosa elocuencia, con tan aterradora expresión. El artificio del drama
había desaparecido, y el hombre mismo, el bárbaro y apasionado Otello
espantaba al auditorio con las voces de su inflamada ira. Un aplauso
atronador y unánime estremeció la sala, porque nunca los concurrentes
habían visto perfección semejante.

Después las facciones del moro se alteraron; su rostro palideció:
oprimiose el pecho con ambas manos, y su voz, trocando el áspero tono en
otro desgarrador y patético, dijo:

\small
\newlength\mlenw
\settowidth\mlenw{pues yo quiero empaparlos, sumergirlos,}
\begin{center}
\parbox{\mlenw}{   Las recias tempestades                         \\
                el viento anuncia con terrible ruido;             \\
                el rayo con relámpagos avisa                      \\
                su golpe destructor, y los rugidos                \\
                del león su presencia nos advierten;              \\
                mas la mujer con ánimo tranquilo                  \\
                y aparentes halagos nos destroza                  \\
                el corazón cual pérfido asesino.}                 \\
\end{center}
\normalsize

Nueva explosión de entusiastas aplausos. Las mujeres lloraban, algunos
hombres no podían conservar su entereza y lloraban también. La
concurrencia estaba estremecida, atónita, electrizada, y cada cual,
suspensa y postergada su propia naturaleza, vivía momentáneamente con la
naturaleza y las pasiones de Otello.

La representación seguía: fuese Otello, cambió la escena, apareció la
cámara de Edelmira. Entretanto, todos me preguntaban la causa de la
turbación y desasosiego de Isidoro; mas yo no sabía qué responder.

Entre bastidores le buscamos con inquietud, pero no le podíamos ver por
ninguna parte, ni nadie se daba razón de dónde pudiera encontrarse.
Edelmira dijo los versos de su monólogo con extraordinaria sensibilidad:
no cesaba de mirar a Mañara, y la vanidosa coquetería de sus ojos,
parecía decir: «¡qué bien represento!» mientras el afortunado amante,
embebecido en contemplarla, parecía contestarle: «¡qué guapa estás!»

Y así era. Lesbia estaba encantadora, con los cabellos sueltos sobre la
espalda, y el ligero vestido blanco que le ceñía el cuerpo indolente.
Entró luego Hermancia, la fiel amiga, y Edelmira le contó sus tristes
presentimientos. ¡Qué tono tan melancólico y dulce tenía su voz al
expresar el temor de la muerte funesta! ¡Cuán grande interés despertaba
su pena! Aunque yo había visto muchas veces la misma tragedia, dentro de
la escena, y había perdido toda ilusión, en aquella noche sentía un
terror inexplicable, y me conmovía la suerte de la infeliz e inocente
Edelmira.

La esposa de Otello, ansiando desahogar la sofocante angustia de su
pecho, toma el arpa y entona la canción de Laura al pie del sauce, cuyos
lastimeros quejidos son la voz de la misma muerte. Edelmira, a quien
Manuel García había enseñado la hermosa estrofa, cantó con dulce y
poética expresión. Su voz parecía que nos penetraba hasta los huesos, y
nos hacía estremecer con horripilante escalofrío, como el contacto de
una hoja de acero.

Cesó la canción y sonó la tempestad en el interior del teatro. El
público estaba tan impresionado que ni siquiera aplaudía. Acostose
Edelmira y todo quedó en profundo silencio. Otello debía aparecer, y en
el breve momento en que estuvo la escena muda, profundísimo silencio
reinaba en la sala. Yo creí sentir el palpitar de los corazones; pero
sólo escuchaba las oscilaciones del mío. La más ardorosa inquietud se
había apoderado de mí, y miré en torno buscando una persona de confianza
a quien comunicar mis recelos; pero no vi sino el pálido semblante de mi
ama que se esforzaba en reír diciendo:

---¡Qué bien ha hecho Lesbia su papel! Me confieso derrotada, pues
representa mil veces mejor que yo. Pero ahora verán ustedes a Isidoro.
Esta noche está más inspirado que nunca.

Observé a Máiquez que ya decía los primeros versos de la escena junto al
lecho de la veneciana. Su rostro aparentaba una serenidad meditabunda.
Cuando alzó las cortinas del lecho y dijo con voz calmosa:

\small
\newlength\mlenx
\settowidth\mlenx{su hermosura estas lúgubres antorchas!}
\begin{center}
\parbox{\mlenx}{No... tú no morirás... ¡cuánto realzan            \\
                su hermosura estas lúgubres antorchas!}           \\
\end{center}
\normalsize

\justifying{\noindent un rumor confuso surgió del apiñado auditorio; lloraban casi todas
las mujeres, y los hombres se esforzaban en sostener el decoro de
la insensibilidad. Otello acerca su rostro al de Edelmira y dice con extasiado amor:}

\small
\newlength\mleny
\settowidth\mleny{¿Qué poderoso hechizo es el que arrastra}
\begin{center}
\parbox{\mleny}{¡Con qué pureza respirar la siento!               \\
                ¿Qué poderoso hechizo es el que arrastra          \\
                mi persona a la suya con tal fuerza?}             \\
\end{center}
\normalsize

Edelmira despierta con sobresalto. Otello disimula al principio; mas
luego no oculta el objeto que le trae, y Edelmira, aterrada y confusa,
jura que es inocente. Nada convence al terrible moro, que mudando de
improviso la expresión de su fisonomía, exclama con ferocidad y
descompuestos ademanes:

\small
\newlength\mlenz
\settowidth\mlenz{¿Qué poderoso hechizo es el que arrastra}
\begin{center}
\parbox{\mlenz}{Mírame, ¿me conoces... me conoces?}               \\
\end{center}
\normalsize

El auditorio se estremeció de terror. Algunas señoras se desmayaron, y
oyéronse voces acongojadas que decían: «Piedad, piedad para
Edelmira\ldots{} es inocente\ldots{} ese infame Pésaro tiene la
culpa\ldots{} que traigan a Pésaro.»

Isidoro sacó el papel y lo mostró con fiero ademán a Lesbia, quien lanzó
un grito terrible sin decir los versos que correspondían en aquel
momento. Otello se acercó más a Edelmira, y Edelmira hizo un movimiento
para saltar del lecho. Se le habían olvidado los versos; pero al fin,
dominando un poco su turbación, recordó algo, y el diálogo siguió así:

\small
\newlength\mylna
\settowidth\mylna{OTELLO. *********************************** Este acero os lo señala.}
\begin{center}
\parbox{\mylna}{\textsc{Edelmira.} ¿Y qué quieres decirme?                                      \\
                \textsc{Otello.}                                                     Preparaos. \\
                \textsc{Edelmira.} ¿Pero a qué?                                                 \\
                \textsc{Otello.}                                   Este acero os lo señala.}    \\
\end{center}
\normalsize

Diciendo esto, Isidoro desenvainó la daga; en lugar de la hoja de madera
plateada, vimos brillar en su mano una reluciente hoja de acero. La
conmoción fue general entre bastidores. Lanzose Edelmira del lecho con
precipitación y azoramiento, y recorrió la escena gritando como una
loca: «¡Favor, favor\ldots{} que me mata! ¡Al asesino!»

No puedo pintaros lo que fue aquel momento en la escena y fuera de ella.
Los espectadores de primera fila trataron de subir al escenario en el
momento en que Lesbia, perseguida por Isidoro, fue asida por el vigoroso
brazo de éste. En el mismo instante, no pudiendo contenerme, me abalancé
hacia la dama como impulsado por un resorte, y abraceme estrechamente a
ella. El puñal de Isidoro se levantó sobre mí. La presencia inesperada
de una víctima extraña hizo sin duda que el moro volviera en sí de su
furiosa obcecación; conmoviose todo, pareció que un velo se descorría
ante sus ojos, arrojó el puñal, quiso recobrar su aplomo; pronunció
algún verso tremendo clavando sus manos en mí, como si yo fuera
Edelmira; ésta, desprendiéndose de mis brazos, cayó al suelo desmayada,
y al punto nos vimos rodeados de multitud de personas. Todo esto pasó en
unos cuantos segundos.

\hypertarget{xxvi}{%
\chapter{XXVI}\label{xxvi}}

El escenario se llenó de gente. La condesa, alzada al instante del
suelo, fue objeto de solícitos cuidados. Al poco rato desvaneciose su
desmayo, abrió los ojos y dijo algunas palabras. No tenía la más ligera
lesión, y todo había concluido sin más consecuencias que las del susto.
Su palidez y la alteración de su semblante eran extraordinarias; pero
aún había entre los circunstantes una persona más alterada y más pálida:
era mi ama.

Isidoro parecía embrutecido y avergonzado. Transcurrió media hora, y
cuando fue indudable que no había ocurrido ninguna desgracia que se
temía, entablose una discusión muy viva sobre aquel acontecimiento, que
la mayoría de los presentes consideraba bajo el punto de vista
artístico; y era opinión de muchos que exaltado hasta un extremo de
delirio el genio artístico de Máiquez, se identificó con su papel de un
modo perfecto.

---Pues lejos de ser el camino de la perfección artística---dijo
Moratín,---lleva derecho a la corrupción del gusto, y extinguirá en las
ficciones el decoro y la gracia, para confundirlas con la repugnante
realidad.

---Ni eso es representar, ni eso es nada---dijo Arriaza, que como es
sabido, detestaba a Isidoro.---Desde que ese caballero introdujo aquí la
escuela francesa, ha corrompido el arte de la declamación.

---Nunca he visto a Máiquez tan apasionado y fogoso---indicó un
caballero que se unió al grupo.---Me parece que en la escena ha pasado
algo extraño a la comedia.

Otro joven acercó sus labios al oído del primero, y por un rato le habló
en voz muy baja. Después a los cuchicheos siguieron las risas. Pasó
Mañara no lejos de allí, y todos fijaron la vista en él.

---Bien se explica la ferocidad de Isidoro---dijo uno.

---Hasta aquí---añadió Moratín,---siempre se le ha visto contenerse
dentro del límite de las conveniencias escénicas.

---Me acuerdo de cuando Isidoro era un pedazo de hielo---dijo
Arriaza.---En el teatro no le llamaban sino el marmolillo.

---Es verdad---repuso Moratín.---Pero cuando volvió de París vino muy
corregido, y no puede negarse que es un actor de gran mérito. En lo
patético no tiene igual; en lo trágico suele carecer de fuego: pero esta
noche lo ha tenido con exceso.

---Le he tratado bastante---dijo un tercero.---Es hombre de pasiones
enérgicas. Como actor consumado, comprende bien que el arte es una
ficción, y representando no deja nunca de ser comedido y decoroso. Esta
noche, sin embargo, le hemos visto tal cual es.

Otro personaje se acercó al grupo.

---¿Qué le ha parecido a usted, señor duque, el desenlace de la
tragedia?---le preguntó Arriaza.

---¡Magnífico! Esto se llama representar---contestó el marido de
Lesbia.---Parecía aquello la misma realidad. Pero no consentiré que mi
esposa salga otra vez a la escena. Representa demasiado bien y
entusiasma y trastorna a los actores que la acompañan.

Un abanico tocó el hombro del señor duque; volviose éste, y Amaranta
entró en el corrillo. Todos la saludaron, disputándose a porfía el honor
de dirigirle la palabra. Ella habló así:

---Bien dije a usted, señor duque, que no había nada que temer. Un
exceso de inspiración dramática y nada más.

---El exceso es malo en todo: yo creí que la duquesa iba a perecer a
manos de Isidoro por un exceso de inspiración.

---Además---dijo Amaranta,---quizás alguna causa que no
conocemos\ldots{}

Al decir esto pareció que los pies de la hermosa dama habían tocado
algún objeto arrojado en el escenario. Apartose ella vivamente,
apartáronse todos, y las faldas de Amaranta, al deslizarse sobre el
piso, dejaron ver un papel arrugado. Como si aquel papel fuera un tesoro
de inestimable precio, Amaranta bajose a cogerlo, y después de mirarlo
rápidamente, lo guardó en su bolsillo. Era la carta fatal, como diría un
novelista.

---¿Alguna causa que no conocemos?\ldots---preguntó el duque continuando
la conversación interrumpida.

---Sí---contestó la dama;---y me parece que puedo sacarle a usted de
dudas\ldots{} Pero tengo que ir al cuarto de la González. Allí le
aguardo a usted y hablaremos.

Quedaron solos los hombres otra vez. La marquesa atravesó la escena
preguntando por Isidoro.

---¿Será posible---decía,---que no pueda representarse \emph{La venganza
del Zurdillo}? ¡Pepa!\ldots{} ¿Pero dónde está Pepa?

Esta pregunta se dirigió a mí, y al instante marché en busca de mi ama.
No estaba en su cuarto, y sí en el de Máiquez, quien una vez pasada la
excitación del terrible momento, se esforzaba en aparecer tranquilo y
hasta risueño, aunque era fácil conocer que la rabia no se había
extinguido en su pecho.

---¡Qué broma tan pesada, Isidoro!---dijo la marquesa asomándose a la
puerta.---Aún no me he recobrado del susto.

---Es verdad, señora---dijo el actor;---pero la señora duquesa tiene la
culpa, por la perfección con que ha hecho su papel. Su incomparable
talento tuvo el don, no sólo de transportarla a ella, sino de
transportarme a mí mismo a la esfera de la realidad. Jamás me ha pasado
cosa igual desde que piso las tablas. Un actor inglés, representando en
cierta ocasión a Otello, mató a la cómica que hacía de Desdémona. Esto
me parecía inverosímil; pero ahora comprendo que puede ser verdad.

---¿No se suspenderá \emph{La venganza del Zurdillo}?

---Por ningún caso. Hace falta reír un poco, señora marquesa.

Retirose ésta y después que salieron algunos amigos de Máiquez, que le
acompañaban, el actor quedó solo con mi ama y conmigo.

---Ven acá---me dijo el actor, apretándome vigorosamente el
brazo.---¿Quién te dio aquella carta?

Señalé a mi ama.

---Fui yo---dijo ésta.---Quería que conocieras el corazón de Lesbia.

---¿Por qué no me la diste en otra parte? Me has puesto al borde del
abismo; he estado a punto de cometer un crimen. Mi furor fue tan grande
cuando leí aquel papel, que lo olvidé todo, y aunque en el instante que
estuve fuera de la escena procuré serenarme, mi cólera se encendió más,
y\ldots{} ya sabes lo que pasó. Cuando la vi en la escena final quise
contenerme; pero sus miradas, su acento, me irritaban cada vez más, y
sentí en mí una crueldad, una ferocidad que nunca había conocido.
Recordaba sus tiernas promesas, sus apasionados arrebatos de amor, su
falsa sencillez, y por un momento creí que hasta era un deber castigar a
aquel monstruo de falsedad e hipocresía. Cuando saqué el puñal y advertí
que era una hoja de acero, experimenté un placer indecible. ¡Ay, Pepa!
¡Qué momento! No sé cómo no la maté; no sé cómo en aquel instante no me
perdí y me deshonré para siempre. Si Gabriel no se hubiera abrazado a
ella cubriéndola con su cuerpo, creo que a estas horas\ldots{} no lo
quiero pensar.

---A estas horas---dijo mi ama,---estarías llorando sobre el cadáver de
tu amante, herida por tu propia mano.

---No, Pepa, no; ya no la amo. La lectura de la carta ha ahuyentado de
mí todo sentimiento amoroso: ya no tengo para ella más que un desprecio,
una repugnancia de que no puedes formar idea. Me espanto de haber amado
a semejante mujer. Pero di: ¿fuiste tú quien trocó el puñal de teatro
por la hoja de acero?

---Sí; yo fui.

---¿Luego tú---exclamó con asombro, lo preparaste todo? ¿Qué interés,
qué intención\ldots?

---¡La aborrezco con toda mi alma!

---¡Y quisiste hacerme instrumento de un crimen! Hace poco hablabas de
tu venganza. ¿Por qué aborreces a Lesbia?

---La aborrezco porque\ldots{} la aborrezco.

---¿Y no te remuerde la conciencia de un sentimiento que te lleva hasta
el crimen?

---¡La conciencia!\ldots{} ¡Un crimen!---dijo mi ama con cierta
enajenación, y después ocultando el rostro entre las manos empezó a
llorar amargamente, exclamando.---¡Oh! ¡Dios mío, qué desgraciada soy!

---Pepa, ¿qué tienes? ¿qué es eso?---dijo Isidoro sentándose junto a
ella, y apartándole las manos del rostro.---Pero tú\ldots{} Conque
tú\ldots{} De modo que tú\ldots{}

Dieron golpes en la puerta, y una voz dijo: «El sainete: que va a
empezar el sainete.»

El aviso no distrajo a los dos actores. Pepa seguía llorando e Isidoro
lleno de asombro.

\hypertarget{xxvii}{%
\chapter{XXVII}\label{xxvii}}

Creí prudente retirarme, no sólo porque allí no hacía falta ninguna,
sino porque en mi mente bullía, inquietándome mucho, un proyecto que al
fin decidí poner en ejecución sin pérdida de tiempo. Dirigime lleno de
resolución al cuarto de mi ama. Amaranta estaba allí y estaba sola.

---¡Oh Gabriel!---me dijo,---¿tienes valor para presentarte delante de
mí? ¿Sabes que tienes un modo singular de despedirte? Veo que eres un
farsantuelo de quien nadie debe fiarse. Di: ¿es esa la lealtad con que
tú acostumbras pagar a tus favorecedores?

---Señora---repuse desafiando el rayo de sus ojos, como el marino
desafía la tempestad,---el oficio a que usía me pensaba dedicar en
palacio no era de mi gusto. Si no me despedí de mi ama, fue porque el
temor de que me prendieran me obligó a salir del Real Sitio.

---No puedo negar---dijo riendo,---que te burlaste con mucha gracia del
licenciado Lobo. Bien decía yo que eras un chico de mucha disposición.
Pero el talento más fecundo permanece oculto hasta que encuentra ocasión
de mostrarse. Aquel rasgo de ingenio habría sido completo, habría sido
sublime, si me hubieras entregado la carta.

---No me la habían dado para usía.

---Lo cierto es que no fue a poder de su dueña. Pepa te la quitó, y ha
hecho de ella el uso que sabes. Tampoco ella quiso entregármela; pero al
fin la casualidad la ha traído a mis manos. ¿La ves?

---Creo que usía me la entregará, porque esa carta es mía, me pertenece,
tengo que devolverla a su dueño---dije con resolución.

---¡Devolvértela! ¿Tú estás loco?---exclamó Amaranta riendo como quien
oye un gran despropósito.

---Sí señora, porque el recobrarla es para mí una cuestión de honor.

---¡Honor!---dijo la dama riendo más fuerte. ¿Acaso tienes tú honor?
¿Sabes tú lo que es eso, chiquillo?

---¿Pues no he de saber?---respondí.---Cuando usía me propuso el oficio
de espía, sentí que se me subía un calorcillo a la cara; y me pareció
que me estaba viendo a mí mismo en aquel empleo y en los de engañar,
fingir y mentir\ldots{} y viéndome me daba espanto\ldots{} y un sudor se
me iba y otro se me venía, porque el Gabriel que mi madre echó al mundo,
se entretiene a veces oyendo lo que él mismo se dice por dentro acerca
de la manera de ser caballero decente y honrado. Cuando la señora
duquesa me pidió su carta, y yo no podía dársela, sentí el mismo
embarazo\ldots{} y también me ocurrió que no devolviendo el papel y
permitiendo que otras personas sigan haciendo mal uso de él, el señor
Gabrielillo no vale dos cuartos. Si esto no es el honor, que venga Dios
y lo vea.

Amaranta pareció muy sorprendida de estas razones, y me dijo con bondad:

---Tales ideas no son propias de ti. Tiempo tienes, cuando seas mayor,
de tener todo el honor que quieras. Cada vez te encuentro más propio
para desempeñar a mi lado los empleos de que te hablé. Me parece que has
empezado bien el curso en la universidad del mundo; y o mucho me engaño,
o te bastarán pocas lecciones más, para ser maestro.

---Creo que usía no se equivoca---respondí,---y en cuanto a las
lecciones que usía me ha dado, me parece que han sido de provecho.

---¿Y no renuncias a tus proyectos de ser\ldots{} como
decías?\ldots---me preguntó irónicamente.

---No señora, sigo en mis trece---contesté sin turbarme,---y a lo mejor
va a tener usía el gusto de verme príncipe o tal vez de rey en cualquier
reino que las damas de la corte sacarán para mí. Si no hay más que
ponerse a ello, como dice Inesilla.

---Pero di, chiquillo: ¿de veras creíste tú que ya te estaban labrando
la espada de general o la corona de duque?

---Como ésta es noche. Y usía, que se me figuraba una divinidad bajada
del cielo para favorecerme, acabó de trastornarme el juicio, enseñándome
lo que debía hacer para echarme a cuestas el manto regio o cuando menos
para ponerme los galones de capitán general.

---Parece que te burlas; ¿qué quieres decir?

---Digo que desde que usía me dijo que el camino de la fortuna estaba en
escuchar tras de los tapices, y llevar y traer chismes de cámara en
cámara, se han arreglado las cosas de tal modo, que sin querer estoy
descubriendo secretos, y aunque quiero taparme las orejas, las picaronas
se empeñan en oír\ldots{}

---¡Ah! Tú quieres revelarme algo que has oído---dijo Amaranta con
complacencia.---Siéntate y habla.

---Lo haré de buena gana, si usía me devuelve la carta de la señora
duquesa.

---Eso no lo pienses.

---Pues entonces callaré como un marmolejo. En cambio contaré una
historia parecida a la que usía me refirió, aunque no es tan bonita. No
la he leído en ningún libro viejo, sino que la oí\ldots{} Estas
condenadas orejas mías\ldots{}

---Pues empieza---dijo la condesa con alguna perplejidad.

---Hace quince años había en Madrid una damita muy guapa, muy guapa, que
se llamaba\ldots{} no me acuerdo su nombre. Esto no pasaba en ningún
reino apartado ni antiguo, sino en Madrid, y no se trata de sultanes ni
de grandes ni pequeños visires, sino de una damita muy linda, la cual
damita se enamoró de un joven de buena familia que vino a la corte a
buscar fortuna. Parece que los padres se oponían; pero la damita amaba
ciegamente al joven; y como todo lo vence el amor, entre éste y el
demonio proporcionaron a los dos jóvenes entrevistas secretas
que\ldots{}

Amaranta se puso pálida, y su mismo asombro la tenía muda.

---Pues es el caso que la damita dio a luz una criatura---continué.

---No estoy aquí para oír necedades---dijo Amaranta dominando su ira.

---Pronto concluyo. Dio a luz una criaturita: huyó el joven a Francia
temiendo ser perseguido, y los padres de la damita se dieron tan buena
maña para echar tierra a aquel negocio, que nada se supo en la corte. La
damita se casó después con el conde de no sé cuántos\ldots{} y nada más.

---Veo que eres rematadamente necio. No quiero oír más tus
simplezas---dijo la dama, cuyo semblante se cubría de vivísimo carmín.

---Aún falta un poquito. Más tarde lo descubrieron algunas personas; y
hablaron de esto en sitio donde yo lo oí; pero como soy tan curioso, y
ahora ando amaestrándome en los chismes y enredos para ver si llego a
general o a príncipe, no me contento con aquellas noticias y voy a que
me dé más una mujer que vive orillas del Manzanares, junto a la casa de
don Francisco Goya.

---¡Oh!---exclamó Amaranta furiosa.---Sal de aquí, desvergonzado
mozalbete. ¿Qué me importan tus ridículas historias?

---Y como estas noticias no tienen valor hasta que no se traen de aquí
para ahí, pienso comunicárselas a la señora marquesa para que me ayude
en mis pesquisas. ¿No cree usía señora condesa, que ésta es una
excelente idea?

---Veo que sabes manejar la calumnia y las bajas y miserables intrigas.
Supongo quién habrá sido tu maestro. Vete Gabriel, me repugnas.

---Me iré y callaré; pero es preciso que usía me vuelva la carta.

---Miserable rapaz: ¡quieres burlarte de mí, quieres medir conmigo tus
indignas armas!---exclamó levantándose de su asiento.

Su actitud decidida me turbó un poco; pero hice esfuerzos por reponerme,
y continué así:

---Para hacer fortuna no hay medio mejor que el espionaje y la
intriguilla: el que posee secretos graves lo tiene todo, y ahora salimos
con que voy a conseguir dos mitras, ocho canonjías, veinte bastones de
coronel, cien capellanías y mil plazas de contaduría para todos mis
amigos.

---Déjame, no quiero verte. ¿Has oído?

---Pero antes me dará usía la carta. Si no he de llevar un recadito a la
señora marquesa, o al señor diplomático, que como hombre reservado no lo
dirá a alma viviente.

---¡Ah!, imbécil, cuánto te desprecio---dijo revolviendo en su bolsillo
con febril inquietud.---Toma, toma la carta, vete con ella, y jamás
vuelvas a ponerte delante de mí.

Diciendo esto, arrojó en el suelo la carta que recogió un servidor de
ustedes.

Después, sentándose de nuevo, volvió hacia mí su rostro siempre bello, y
me dijo:

---¡Quién te ha enseñado esas travesuras? Eres un necio.

---De los necios se hacen los discretos---contesté.---Dando con un buen
maestro\ldots{} Si usía no me hubiera despabilado tanto\ldots{} Oyendo y
viendo se aprende mucho, señora; y yo, desde que entré al servicio de
usía hasta hoy, no he desperdiciado el tiempo. Bien haya quien me ha
abierto los ojitos que ven y las orejitas que oyen. Para ser discreto es
preciso haber sido tonto.

Cuando pronuncié esta extraña sentencia, Amaranta echó sobre mí una
mirada de orgulloso desdén, y señalome la puerta. ¡Ay!, estaba hermosa,
hermosa como nunca. Su noble ademán, sus mejillas teñidas de leve
púrpura, el incendio de sus ojos, la agitación de su seno encantaban la
vista, y no era posible aborrecerla. Indudablemente, señores, el mal es
a veces lindísimo.

Ya me marchaba, cuando entró el señor duque acompañado del diplomático.

---Aquí estoy, Amaranta---dijo el primero.---Me habló usted de causas
que no conocemos\ldots{}

---No le hagas caso, sobrina---exclamó el marqués.---¿Pues no ha dado en
la flor de estar celoso? Y dice que en el caso de Otello él haría lo
mismo.

---Sí---dijo el duque.---Si yo sospechara de mi mujer la mataría.

---No me refería a nada que no fuese algún motivo artístico---indicó
secamente Amaranta.

---No consiento que mi mujer salga más a las tablas en compañía de ese
bárbaro Otello. La pobrecita debe de haber padecido mucho. Pero veo que
en mi ausencia han ocurrido grandes novedades. Parece que también han
querido ponerla presa. ¡Pobre cordera mía! ¿Cómo es posible que haya
dado motivos para eso?\ldots{} Si es la bondad, si es la dulzura en
persona.

---Son tantos los que han sido incluidos en la causa\ldots---dijo
Amaranta.---Pero por mediación mía se la puso al instante en libertad.

---¡Oh!, gracias, querida condesa. Verdad es que Lesbia es amiga de
usted desde la infancia, y entre amigas\ldots{} ¿Y no se la molestará
más?

---No---dijo el diplomático.---Felizmente puede arrancarse de la causa
todo lo que conviene, ¿no es verdad, sobrina?

---Sí; precisamente se ha hecho eso con todo lo que se refiere al
Príncipe, porque como ha confesado y hecho acto de contrición de todas
sus faltas\ldots{} Los jueces tienen buena mano, y suprimirán todo lo
que se quiera, dejando la causa tal como convenga presentarla al
público.

---Eso está muy bien dispuesto---afirmó el diplomático,---y prueba que
hay tacto en el Gobierno. ¿Y Napoleón?

---Napoleón ha exigido que no se le nombre para nada, y por esto ha sido
preciso eliminar también cuanto a él se refiere. Aunque consta que el
Príncipe le escribió y tuvo tratos con su embajador, los jueces se
comerán todas las declaraciones y documentos en que esto se vea, para
que Bonaparte quede contento.

---Bien, bien, eso me tranquiliza---afirmó el diplomático con mucho
énfasis,---y así lo pondré en conocimiento del Príncipe Borghese, del
príncipe Piombino, de Su Alteza el gran duque de Aremberg. Por supuesto,
os encargo que no digáis a nadie mis propósitos; ¿lo oyes Amaranta? ¿Lo
oye usted, señor duque? ¡Ah!, al duque no se le puede confiar un
secreto. Todo lo dice.

---¿Qué?---preguntó Amaranta.

---Por más que me empeño en que la más absoluta reserva sirva de
impenetrable velo a lo que ocurre entre la González y yo\ldots{}

---El señor marqués no abandona sus antiguas mañas---dijo el duque.

---No hijo; es que sin saber cómo ni cuándo\ldots{} Nada he puesto de mi
parte. Hace tiempo que Pepita ha manifestado que hallaba en mí cierto
encanto\ldots{} Pero la pícara no se cuida de disimular; ahora mismo,
durante el sainete, me echaba unas miradas\ldots{} ¡Y qué bien ha
representado! Nunca la he visto tan alegre, tan graciosa, tan juguetona,
tan vivaracha. La verdad es que me está comprometiendo. ¿Lo creerás,
sobrina? Yo me empeño en ocultarlo, porque\ldots{} ya sabes\ldots{} ése
es mi carácter, y ella\ldots{} pero si todo el mundo lo sabe. Al
concluir el sainete, no he podido menos de acercarme a ella, y le he
dicho: «Disimule usted Pepa, no olvide usted que la reserva es hermana
gemela de la\ldots{} digo, del amor.» Sin duda por obedecer esta
advertencia, se ha marchado con Isidoro, fingiéndose muy contenta en su
compañía. Ambos iban muy amartelados, y cualquiera menos listo que yo,
los habría tenido por amantes.

~

---Tal vez---dijo Amaranta.

Salí del cuarto. Cuando después de buscar ávidamente a Lesbia por el
escenario, di con ella al fin y le entregué la carta, me dijo con mucha
ansiedad mientras la guardaba:

---¡Ah, Gabrielillo! Esta noche me has salvado la vida dos veces.

\hypertarget{xxviii}{%
\chapter{XXVIII}\label{xxviii}}

No quise estar más allí; salí decidido a huir para siempre del
vergonzoso arrimo de cómicos y danzantes, de damas intrigantuelas y de
hombres corrompidos y fatuos. Al salir, un vivo deseo de correr a casa
de Inés llenaba mi alma toda. Volé al cuarto piso tomando la pequeña
escalera, y por el camino, en mi precipitada marcha, iba arrojando los
postizos y adornos que me habían servido para la representación. Aquí
dejé las barbas y bigotes, allí las plumas de mi sombrero, más allá la
escarcela, y por último eché a rodar el tahalí y el collar. Me parecían
prendas de ignominia que no debían ir sobre mí al presentarme en la casa
del reposo.

Subí y entré: el padre Celestino me abrió la puerta, y al punto advertí
que sus ojos habían llorado.

---La pobre doña Juana ha muerto hace dos horas---dijo contestando a mis
preguntas.

Esta noticia dio a todo mi ser el frío y la inmovilidad de una estatua.
Sepulcral silencio reinaba en la casa. En el fondo del pasillo vi la
puerta de la sala, cuyo recinto iluminaba una claridad rojiza. Acerquéme
con pasos lentos y conteniendo con la mano el latir de mi corazón que
parecía querer salírseme del pecho. Desde el umbral vi el cuerpo de la
santa mujer vestido de negro, y sobre el mismo lecho en que había sido
abandonado por el alma: sus manos cruzadas en actitud de orar, sus
cerrados ojos y la apacible y tranquila expresión de su semblante blanco
como el mármol, más que el aspecto de la triste muerte, dábanle la
fisonomía propia de un recogimiento meditabundo y de aquel místico sueño
que es en las gentes de exaltada piedad como un viaje al cielo para
volver.

Junto a ella, y sentada en el suelo, con la cabeza entre las manos y
apoyada en el lecho, estaba Inés. Su llanto tranquilo era el natural
desahogo de un dolor resignado, propio de quien acostumbraba a
relacionar las penas y las alegrías con la voluntad de arriba. No hizo
movimiento alguno para mirarme, ni yo seguramente lo merecía. Una sola
vela de cera, cuya llama puntiaguda y movible señalaba al cielo con leve
oscilación, iluminaba la silenciosa sala; y las imágenes de vírgenes y
santos que había en la pared, como afectadas del fúnebre cuadro,
parecían tener en sus rostros inusitada gravedad.

A pesar de mi aflicción, yo experimentaba ante aquel espectáculo una
especie de alivio moral que me es imposible expresar con palabras.
Aquella tranquilidad que acompañaba a una gran pena, aquella paz de
espíritu que cubría el dolor, como las alas del misterioso ángel
protegen el alma, al salir turbada y temerosa del cuerpo pecador; aquel
silencio de la mujer muerta, que me hacía oír en lo profundo de mi mente
un lejano y celeste coro de triunfante música; el sereno llorar de la
huérfana, cuyo dolor modesto no acusaba a la suerte, ni a la casualidad,
ni a otro alguno de los irrisorios dioses que ha creado el holgazán
entendimiento humano; aquel aspecto de resignación; el reposo
imperturbable que ni aun la muerte había alterado en aquella mansión de
la conciencia pura, de los deberes, de la religión, del sencillo amor,
fueron para mi espíritu como un aura serena, como un templado y
regenerador ambiente que equilibra y uniforma la atmósfera por
tempestades revuelta o agitada por opuestas corrientes. Jamás he podido
comparar con más propiedad mi alma con la imagen de un terso lago, de
igual y no alterada superficie, ni jamás he distinguido con tanta
claridad el lejano fondo. Cual si mi pecho hubiese estado por largo
tiempo privado de fácil respiración, mis pulmones se dilataron y mi
aliento sacaba del corazón un gran peso\ldots{}

El cura me sacó de tales abstracciones llamándome fuera.

---La pobre Juana---me dijo enjugando una lágrima---no tuvo tiempo de
ver satisfecho el deseo de toda mi vida.

---¿Pues qué? Usted\ldots{}

---Sí, hijo mío; poco antes de su muerte recibí este papel en que se me
nombra ecónomo de la iglesia parroquial de Aranjuez. Al fin se me ha
hecho justicia. No me ha cogido de nuevo, y bien te decía yo que había
de ser esta semana. ¿Ves, Gabrielillo? Dios ha acudido oportunamente a
nosotros en esta desgracia. Ya Inés no quedará desamparada, ni tendrá
que pedir auxilio a los parientes de Juana.

---¡Pobre Inés!---exclamé.---A ella consagraré mi vida entera. Viviré
por ella y sólo por ella.

---¡Ah!---dijo el clérigo.---Ocurre una cosa singularísima, querido
Gabriel. ¿Sabes que la pobre Juana me ha hecho antes de morir una
revelación que\ldots{} a ti puedo confiarlo porque casi eres de la
familia.

---¿Qué?

---Después que confesó, llamome aparte y me dijo que Inés no es hija
suya\ldots{} ¡Si vieras qué historia tan singular! Estoy confundido,
absorto. Pues, sí, Inés no es hija suya, sino de una gran señora
que\ldots{}

---¿Qué dice usted?---exclamé con el mayor asombro.

---Lo que oyes: la verdadera madre\ldots{} ya comprenderás que en esto
hubo una de esas secretas aventuras, que deshonran a una noble familia.
La verdadera madre abandonó a esa pobre niña, y\ldots{} ya te contaré
despacio.

---Pero el nombre, el nombre de esa señora es lo que quiero saber.

---Juana iba a revelármelo: su relación la había fatigado mucho, y la
palabra tembló en sus labios ya paralizados por la muerte.

Tal noticia produjo en mí espantosa confusión: volví a la sala y
contemplé a la muerta, casi esperando que sus labios pudieran articular
el deseado nombre.

---¿Es posible, Dios mío---dije dirigiendo mi mente al cielo,---que no
hagas bajar un rayo de vida a este yerto cadáver para que su fría lengua
se mueva y pronuncie una sola palabra?

En mi ansiedad, hasta tuve por un momento la esperanza de que el cadáver
reanimado por mis ruegos, volviese a la vida para revelarme el misterio
del nacimiento de Inés.

---¡Qué loco soy!---dije después.---No faltarán medios de averiguarlo.

\hypertarget{xxix}{%
\chapter{XXIX}\label{xxix}}

Desde entonces Inés fue para mí el resumen de la vida. Si antes no la
hubiera amado, su desgracia me habría inclinado con invencible fuerza
hacia ella.

Cuando se acerca el fin de la jornada causa gozo el considerar de qué
extraña manera nos prepara la Providencia, allá en los comienzos de
nuestra vida, el camino que hemos de recorrer y hasta los tropiezos o
facilidades, penas y alegrías que en él hemos de encontrar. El tránsito
de la niñez a la juventud parece el esbozo de un drama, cuyo plan apenas
se entrevé en el balbuciente lenguaje de los primeros afectos y en la
indecisión turbulenta de las primeras acciones varoniles.

Cosas hay en mi vida que parecerán de novela, aunque no creo que esto
sea peculiar en mí, pues todo hombre es autor y actor de algo que si se
contara y escribiera habría de parecer escrito y contado para
entretenimiento de los que buscan recreo en las vidas ajenas, hastiados
de la propia por demasiado conocida. No hay existencia que no tenga
mucho de lo que hemos convenido en llamar \emph{novela} (no sé por qué),
ni libro de este género, por insustancial que sea, que no ofrezca en sus
páginas algún acento de vida real y palpitante.

Empleé los dos mil reales en el entierro de la difunta y en el viaje que
el Padre Celestino y la huérfana hicieron a Aranjuez, donde se
instalaron. Yo regresé a Madrid. Inés, reclamada después por los
parientes de doña Juana, sufrió martirios y desgracias, cuyo recuerdo
hace aún estremecer de angustia mi corazón. Creímos al fin asegurada
nuestra felicidad; pero vinieron aciagos y terribles días, aquellos días
que se anunciaban de un modo terrorífico en nuestras imaginaciones, como
el presentimiento de una catástrofe, Yo, con ser casi un niño, no me
libraba de la aprensión general, y por mi mente pasaban, a modo de
relámpagos, ideas tan tristes como vagas acerca de desastres futuros. En
la atmósfera, en el ambiente moral del pueblo había no sé qué sombras
avanzadas de aquellos desastres no conocidos todavía. Sin explicarme el
motivo de mis temores, yo creía ver por todas partes la imagen lúgubre
de la guerra con formas que no podía determinar, y aquella imagen pasaba
ante mí veloz, horripilante, ordenándome que la siguiera\ldots{} ¡Oh,
cuán pronto corrimos tras ella todos los españoles! Vino la revolución
de Aranjuez; vino el Dos de Mayo, día de sangre y luto; los franceses
inmolaron muchas víctimas; Inés cayó en poder de los invasores\ldots{}
Pero ahora me faltan fuerzas para relatar tan horrorosos
acontecimientos. Estoy fatigado y necesito tomar aliento para seguir
contando.

\flushright{Madrid, abril-mayo 1873.}

~

\bigskip
\bigskip
\begin{center}
\textsc{fin de la corte de carlos iv}
\end{center}

\end{document}
