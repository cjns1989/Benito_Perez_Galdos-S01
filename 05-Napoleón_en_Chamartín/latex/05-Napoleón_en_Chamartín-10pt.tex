\PassOptionsToPackage{unicode=true}{hyperref} % options for packages loaded elsewhere
\PassOptionsToPackage{hyphens}{url}
%
\documentclass[oneside,10pt,spanish,]{extbook} % cjns1989 - 27112019 - added the oneside option: so that the text jumps left & right when reading on a tablet/ereader
\usepackage{lmodern}
\usepackage{amssymb,amsmath}
\usepackage{ifxetex,ifluatex}
\usepackage{fixltx2e} % provides \textsubscript
\ifnum 0\ifxetex 1\fi\ifluatex 1\fi=0 % if pdftex
  \usepackage[T1]{fontenc}
  \usepackage[utf8]{inputenc}
  \usepackage{textcomp} % provides euro and other symbols
\else % if luatex or xelatex
  \usepackage{unicode-math}
  \defaultfontfeatures{Ligatures=TeX,Scale=MatchLowercase}
%   \setmainfont[]{EBGaramond-Regular}
    \setmainfont[Numbers={OldStyle,Proportional}]{EBGaramond-Regular}      % cjns1989 - 20191129 - old style numbers 
\fi
% use upquote if available, for straight quotes in verbatim environments
\IfFileExists{upquote.sty}{\usepackage{upquote}}{}
% use microtype if available
\IfFileExists{microtype.sty}{%
\usepackage[]{microtype}
\UseMicrotypeSet[protrusion]{basicmath} % disable protrusion for tt fonts
}{}
\usepackage{hyperref}
\hypersetup{
            pdftitle={NAPOLEÓN EN CHAMARTÍN},
            pdfauthor={Benito Pérez Galdós},
            pdfborder={0 0 0},
            breaklinks=true}
\urlstyle{same}  % don't use monospace font for urls
\usepackage[papersize={4.80 in, 6.40  in},left=.5 in,right=.5 in]{geometry}
\setlength{\emergencystretch}{3em}  % prevent overfull lines
\providecommand{\tightlist}{%
  \setlength{\itemsep}{0pt}\setlength{\parskip}{0pt}}
\setcounter{secnumdepth}{0}

% set default figure placement to htbp
\makeatletter
\def\fps@figure{htbp}
\makeatother

\usepackage{ragged2e}
\usepackage{epigraph}
\renewcommand{\textflush}{flushepinormal}

\usepackage{indentfirst}

\usepackage{fancyhdr}
\pagestyle{fancy}
\fancyhf{}
\fancyhead[R]{\thepage}
\renewcommand{\headrulewidth}{0pt}
\usepackage{quoting}
\usepackage{ragged2e}

\newlength\mylen
\settowidth\mylen{...................}

\usepackage{stackengine}
\usepackage{graphicx}
\def\asterism{\par\vspace{1em}{\centering\scalebox{.9}{%
  \stackon[-0.6pt]{\bfseries*~*}{\bfseries*}}\par}\vspace{.8em}\par}

 \usepackage{titlesec}
 \titleformat{\chapter}[display]
  {\normalfont\bfseries\filcenter}{}{0pt}{\Large}
 \titleformat{\section}[display]
  {\normalfont\bfseries\filcenter}{}{0pt}{\Large}
 \titleformat{\subsection}[display]
  {\normalfont\bfseries\filcenter}{}{0pt}{\Large}

\setcounter{secnumdepth}{1}
\ifnum 0\ifxetex 1\fi\ifluatex 1\fi=0 % if pdftex
  \usepackage[shorthands=off,main=spanish]{babel}
\else
  % load polyglossia as late as possible as it *could* call bidi if RTL lang (e.g. Hebrew or Arabic)
%   \usepackage{polyglossia}
%   \setmainlanguage[]{spanish}
%   \usepackage[french]{babel} % cjns1989 - 1.43 version of polyglossia on this system does not allow disabling the autospacing feature
\fi

\title{NAPOLEÓN EN CHAMARTÍN}
\author{Benito Pérez Galdós}
\date{}

\begin{document}
\maketitle

\hypertarget{i}{%
\chapter{I}\label{i}}

El Sr.~D. Diego Hipólito Félix de Cantalicio Afán de Ribera, Alfoz,
etc., etc., conde de Rumblar y de Peña-Horadada, hacía en Madrid la
siguiente vida:

Levantábase tarde, y después de dar cuerda a sus relojes, se ponía a
disposición del peluquero, quien en poco más de hora y media le
arreglaba la cabeza por fuera, que por dentro sólo Dios pudiera hacerlo.
Luego daba al reloj de su cuerpo la \emph{cuerda del necesario
alimento}, como decía Comella, la cual cuerda pasaba aún más allá de la
media docena de bollos de Jesús reblandecidos en dos onzas de chocolate.
Incontinenti tenía lugar la operación de vestirse y calzarse, no
consumada a dos tirones, sino con toda aquella pausa, aplomo,
espaciosidad y mesura que la índole de los tiempos exigía. Una vez en la
calle, dirigía sus pasos a cierta casa de la Cuesta de la Vega, donde es
fama que habitaba la discreta mayorazga, con cuyo linaje la casa de
Rumblar concertara genealógico y utilitario ayuntamiento. Esta visita no
era de mucho tiempo, y al poco rato salía D. Diego para encaminarse
ligero como un corzo a la calle de la Magdalena, donde vivía un señor de
Mañara, de quien era devotísimo y fiel amigo. Era creencia general que
comían juntos, y luego leían la \emph{Gaceta}, el \emph{Semanario
patriótico}, el \emph{Memorial literario} y cuantos papeles impresos
venían de Valencia, Sevilla o Bayona, tarea que les entretenía hasta el
anochecer; y por fin a la hora y punto en que las calles de Madrid se
tapujaban con aquel manto de simpática oscuridad que el positivismo
alumbrador de estos tiempos ha rasgado en mil pedazos, nuestros dos
galanes salían juntos en luengas capas embozados, y a veces con traje
muy distinto del que usaban durante el día. Aquí tenía principio, según
opinión de los sesudos autores que se han ocupado de D. Diego de
Rumblar, la verdadera existencia de aquel insigne rapazuelo, así como
también es cierto que todos los cronistas, si bien desacordes en algunos
pormenores de sus escandalosas aventuras, están conformes en afirmar que
siempre le acompañaba el supradicho Mañara, y que casi nunca dejaban de
visitar a una altísima dama, la cual lo era sin duda por vivir en un
tercer piso de la calle de la Pasión, y tenía por nombre la \emph{Zaina}
o la \emph{Zunga}, pues en este punto existe una lamentable discordancia
entre autores, cronistas, historiógrafos y demás graves personas que de
las hazañas de tan famosa hembra han tratado.

Ante el inconveniente de aplicar a Ignacia Rejoncillos los dos apodos
con que la apellidaban sus amigos, yo me decido a llamarla siempre la
\emph{Zaina}, y en verdad que ignoro por qué la aplicaron tal nombre,
pues aunque a los caballos castaños se les llama \emph{zainos}, no sé si
esto cuadra a los cabellos del mismo color: ello es, sin embargo, que la
palabreja significa también \emph{traidor}, \emph{falso} y \emph{poco
seguro en el trato}, y falta saber si la hija del tío Rejoncillos, alias
Mano de mortero, merecía aquellos dictados, y por lo tanto, el ser
tenida por la flor y espejo de la \emph{zainería}.

Pero no quiero desviarme de mi principal objeto, que ahora es decir a
cuáles sitios iba D. Diego y a cuáles no: y firme en tal propósito,
afirmo y juro en realidad de verdad, y sin que ninguna persona honrada
me pueda desmentir, que D. Diego y el Sr.~de Mañara iban de noche a una
reunión de masonería incipiente del género tonto, que se celebraba en la
calle de las Tres Cruces, y a otra del género cómico fúnebre, que tenía
su sala, si no me falta la memoria, en la calle de Atocha, número 11
antiguo, frente a San Sebastián; en cuyas reuniones, amén de las muchas
pantomimas comunes a esta orden famosa, leíanse versos y se pronunciaban
discursos, de cuyas piezas literarias espero dar alguna muestra a mis
pacienzudos leyentes.

Sobre todo en la calle de Atocha, donde estaba la logia
\emph{Rosa-Cruz}, el rito era tal, que algunas veces púseme a punto de
reventar conteniendo las bascas y convulsiones de mi risa, pues aquello,
señores, si no era una jaula de graciosos locos, se le parecía como una
berenjena a otra. En una oscurísima habitación, que alumbraban
macilentas luces, y toda colgada de negro, se reunían los tales masones;
y porque allí fuera todo misterioso, tenían a la cabecera un Santo
Cristo acompañado del compás, escuadra y llana, y a la derecha mano, un
esqueleto muy bien puesto en un sillón, con la cabeza apoyada en la
mano, en ademán meditabundo, y por debajo un letrerito que decía:
\emph{Aprende a morir bien}.

Debo indicar que en aquel año la masonería española era pura y
simplemente una inocencia de nuestros abuelos, imitación sosa y sin
gracia de lo que aquellos benditos habían oído tocante al \emph{Grande
Oriente Inglés} y al \emph{Rito Escocés}. Yo tengo para mí que antes de
1809, época en que los franceses establecieron formalmente la masonería,
en España ser masón y no ser nada eran una misma cosa. Y no me digan que
Carlos III, el conde de Aranda, el de Campomanes y otros célebres
personajes eran masones, pues como nunca les he tenido por tontos,
presumo que esta afirmación es hija del celo excesivo de aquellos
buscadores de prosélitos que no hallándolos en torno a sí, llevan su
banderín de recluta por los campos de la historia, para echar mano del
mismo padre Adán, si le cogen descuidado.

Después de 1809 ya es otra cosa. De aquellas dos logias infantiles, que
yo conocí en la calle de las Tres Cruces y en la de Atocha, y donde se
regocijaban con candorosas ceremonias unos cuantos desocupados, salieron
la famosa logia de la \emph{Estrella}, la de \emph{Santa Justa patrona
de Córcega,} la sociedad de caballeros y damas \emph{Philocoreitas}, la
de los \emph{Filadelfios} de Salamanca, la Gran logia nacional que
estuvo en el edificio ocupado antes por la Inquisición, la logia de
Santiago el Mayor en Sevilla, y las de Jaén, Orense, Cádiz y otras
ciudades. Entrometiéndome en la Gran logia nacional, oí hablar de cosas
más serias y graves que los discursitos \emph{filosóficos en verso} que
le echaban al esqueleto de la \emph{Rosa-Cruz}; oí hablar mucho de
política, de igualdad, y entonces fue cuando anduvo de boca en boca, y
llegó a ser muy de moda la palabra \emph{democratismo}, que luego
desapareció para presentarse de nuevo al cabo de medio siglo, aunque
reformada en su forma y tal vez en su significación. De la larva de
aquellas logias, no es aventurado afirmar que salió al poco tiempo la
crisálida de los clubs, los cuales a su vez, andando el voluble siglo,
dieron de sí la mariposa de los comités.

Pero otra vez, sin quererlo, me aparto de mi objeto, y no ha de ser así,
sino que vuelvo atrás para deciros que el señor conde de Rumblar, luego
que esparcía su ánimo en aquello del esqueleto, y hablaba por los codos
durante una hora, iba en busca de entretenimientos más agradables, y
aquí es donde viene como anillo en el dedo la ocasión de nombrar a la
Zaina, porque a eso de las once era cuando penetraba en sus
\emph{salones} el joven de que me ocupo, no acompañado sólo por el
citado Mañara, sino también por D. Luis de Santorcaz, que siempre se le
unía en la \emph{Rosa-Cruz} para seguir juntos hasta la madrugada.

Es preciso tener presente que no era la Zaina la única gran dama de
aquellos aristocráticos barrios que abría de par en par las puertas de
la casa y de su alma a nuestros tres amigos, y a fe mía que si hubiera
yo de enumerar todas las ilustres casas de los cuarteles de San Lorenzo
y San Millán que por aquellos días obsequiaban a un pequeño número de
\emph{habitués} (¿por qué no decirlo en francés?) llenaría de seguro
todo este libro y medio más. Pero, sin renunciar a ser cronista de los
saraos de aquella matritense \emph{high life} (¿por qué no decirlo en
inglés?) seré muy breve por ahora, señores míos. Estenme atentos, y no
me interrumpan con exclamaciones de admiración, que me harían perder mal
de mi grado el hilo del relato.

Los salones de la \emph{Zancuda}, en la calle de Ministriles, se abrían
muy temprano, y allí había cierta grave etiqueta, con poco de fandango y
menos de seguidillas, razón por la cual escaseaba la concurrencia. Era
la\emph{Zancuda} mujer de grandes atractivos, a pesar de su feísimo
nombre, pero no gustaba de alborotos, porque su marido o lo que fuera,
el Sr.~Regodeo, era al modo de diplomático, hombre estirado, serio,
ceñudo, y que en esto de burlar con sutilísima perspicacia las socaliñas
de las aduanas, almojarifazgos o arbitrios de puertas, no se cambiaría
por los más famosos de Sevilla y Ronda en el tal oficio. D. Diego y sus
dos amigos frecuentaban poco esta casa, donde comúnmente se estaba como
en misa.

En los salones de la \emph{Pelumbres} (calle de la Torrecilla del Leal,
tienda de hierro viejo) era todo animación, todo alegría, no sólo por
ser la dueña de la casa una de las mujeres más malignamente graciosas,
más divertidas y de mejor mano para tocar las castañuelas que han
existido a principios del siglo, sino porque allí concurrían personajes
célebres en varias artes y oficios, tales como el distinguido curtidor
\emph{Tres pesetas}; el \emph{Sr.~Medio diente}, uno de nuestros más
esclarecidos trajineros, natural de las Tenerías de Toledo, y
\emph{Majoma}, curtidor de carne, el cual, cuando contaba sus viajes por
las distintas cortes del mundo, tales como Melilla, Ceuta y el Peñón,
les dejaba a todos con la boca abierta. Y como no faltaban tampoco ni la
Narcisa, ni Menegilda, ni Alifonsa, todas tres estrellas esplendorosas
del firmamento manolesco, la una vendedora de castañas, la otra de
callos y caracoles y la postrera de sal; como no se escatimaba el vino,
ni las boleras, ni se ponía fin a los dichos, ni a la sabrosísima
libertad en lengua y manos, D. Diego tenía sumo gusto en frecuentar
aquella casa. Verdad es (y la historia no debe permanecer silenciosa en
este punto) que las tertulias solían concluir con un refresco de palos,
que, a oscuras y cual lluvia del cielo, caían de improviso sobre la
escogida reunión; pero aquellos más bien regocijaban que afligían a D.
Diego, el cual, ocupándose antes en darlos que en recibirlos, no se
apuraba por unos cuantos cardenales más o menos, ni renunciaría a las
fiestas de la \emph{Pelumbres}, aunque llevara en sus espaldas todo el
cónclave romano.

Pues ¿y qué diré de aquellas elegantísimas y suntuosas fiestas de
\emph{Rosa la Naranjera}, tan célebres en toda la redondez de Madrid,
que hay historiadores muy concienzudos que aseguran haber visto a más de
un príncipe traspasar los umbrales de su bodegón, calle de las
Maldonadas? Y si esta última atrevida afirmación no fuera cierta, eslo
en lo tocante a duques, marqueses, condes y vizcondes, de lo cual
certifico, por haberlos visto. No digo lo mismo de príncipes y reyes,
pues de estos no recuerdo más que los de copas, bastos, oros y espadas,
los cuales no faltaban ni una noche, y con toda familiaridad y franqueza
se dejaban llevar de mano en mano. Eso sí; diga lo que quiera la ruin
envidia y la mala fe de los que allí se quedaron limpios como patenas,
el banquero Juan Candil era una persona honrada, y de recomendables
antecedentes en aquel oficio, y hartas veces decía la Naranjera que en
su casa no se consentían trampas, razón por la cual creemos que aquel
era juego de ley, y que cuanto se decía acerca de las diestras manos de
Candil y de las marcas de sus mugrientos naipes era, o cavilaciones de
los parroquianos o efecto de esa viciada atmósfera que rodea a las
grandes instituciones cuando se las plantea entre gente díscola y
pendenciera. ¡Y cómo gozaba D. Diego en aquella casa! ¡Y cuánto le
querían y mimaban, y cómo se hacían lenguas todos en alabanza de su
liberalidad, de su desprendimiento, de su nobleza, de aquel donaire con
que entregaba sin muestras de aflicción la cantidad perdida! A tanto
efecto correspondía Rumbrar con una asistencia tan puntual, que si fuera
al aula le habría hecho en poco tiempo un segundo Aristóteles.

Mas en aquella casa y en las que antes he mencionado no se consagraba
todo el tiempo a los reyes, sotas y demás real familia, pues siguiendo
la general corriente de los tiempos, se hablaba mucho de política. Iba a
ellas con frecuencia, y durante sus días de vagar, el tío Mano de
Mortero, que siempre llevaba noticias frescas. También concurría
Pujitos, joven instruidísimo y de gran erudición, pues no dejaba de
saber leer (aunque con pausa y cierto dejo o sonsonete), razón por la
cual aquel esclarecido concurso estaba al tanto de las \emph{Gacetas} y
papeles nacionales y extranjeros, porque es de advertir que si el tío
Mano de Mortero conocía afondo la geografía ibérica (merced a sus
frecuentes viajes \emph{científicos} para desesperación del Estado y
quebrantamiento del fisco); si por esta circunstancia conocía la
posición de los ejércitos beligerantes, Pujitos iba mucho más allá;
Pujitos se elevaba en alas del genio, y su pensamiento cerníase en las
vertiginosas altitudes del arte militar y diplomático, como el águila
sobre las eminentes cumbres.

Estas conversaciones no duraban toda la noche, y entre juego y juego
solía haber bolero y manchegas así como también algo de aquello que los
eruditos llaman palos y el vulgo también; pero sabido es que los palos
son para ciertas gentes gustosísimo postre, después de los manjares
fuertes del amor y del vino. ¡Ay! puedo asegurar que D. Diego era muy
feliz con aquella vida.

Pero el dorado alcázar, el Medina-al-Fajara, el Bagdad, la Sibaris y la
Capua de sus impresionables sentidos estaban en casa de la Zaina,
aquella beldad incomparable, aquella que al aparecer por las mañanas en
la esquina de la calle de San Dámaso, dentro de su cajón de verduras,
daría envidia a la misma diosa Pomona, en su pedestal de frutas y
hortalizas. ¿Y qué diremos de aquella gracia peculiar con que lavaba una
lechuga, arrancándole las hojas de fuera con sus divinas manos,
empedradas de anillos? ¿Qué del donaire con que hacía los manojitos de
rábanos, que entre sus dedos racimos de orientales corales parecían?
¿Qué de aquella por nadie imitada habilidad para poner en orden los
pimientos y tomates, cuya encendida grana se eclipsaba ante el rosicler
de su cara? ¿Qué de aquel lindísimo gesto con que metía los cuartos en
la faltriquera, olvidándose casi siempre de dar la vuelta? ¿Qué de
aquella postura (digna de llamar la atención de Fidias), cuando
descolgaba una sarta de ajos, que al enroscarse en sus brazos no se
tomarían por otra cosa que por rosarios de descomunales perlas? ¿Qué de
la destreza y soltura con que arrojaba las hojas de col sobre los usías
que iban a requebrarla? ¿Qué de su ciencia en el vender, y su labia en
el regateo, y su diplomacia en el engañar, que a esto y a nada más
propendían todas y cada una de las sales y monerías de su lengua y
ademanes? Válgame Dios que tuvo buen gustó D. Diego al prendarse de
aquella princesa o semidiosa, pues tal era su mérito y de tal modo y con
tanta presteza la rodeaba de poéticos atributos la imaginación, que el
puesto era un trono y las lechugas ramos de olorosas yerbas, y los
rábanos jacintos de Holanda, y los repollos abiertas magnolias, y los
ajos cerradas azucenas, y las cebollas conjunto perfumado de todas las
flores; así como también podía suponerse que el agujereado mandil de la
Zaina era un rico sayal de finísima puntilla de Flandes, y el cuchillo
de partir varita de oro para dar gusto y ocupación a las móviles manos,
y los ochavos desparramadas joyas que los príncipes y reyes, de remotas
tierras venidos, echaban a sus pies para rendir el fuerte castillo de su
honestidad.

¿Y qué me diréis si os aseguro que D. Diego, a pesar de sus atractivos y
de su dinero, no había podido rendir a la Zaina? ¡Oh inflexible ley de
los hados que en aquella ocasión dispusieron que la Zaina fuese esclava
en cuerpo y alma de otro galán, al cual de antiguo mis lectores conocen,
y no es otro que el propio don Juan de Mañara, por segunda vez
presentado en el escenario de estas historias! Pues sí; el Sr.~de
Mañara, como la muerte, lo mismo ponía el pie en \emph{pauperum
tabernas} que en \emph{regumque turres}; y aunque era persona de alta
posición por aquellos días, y estaba a punto de ser nombrado regidor de
Madrid, sus preferencias en materia de costumbres y de amor, íbanse del
lado de lo que Horacio llamó \emph{tabernas}, y en castellano podemos
nombrar ahora con la misma palabra.

\hypertarget{ii}{%
\chapter{II}\label{ii}}

Por las noches, este caballero, lo mismo que D. Diego, se vestían de
majos, y\ldots{} aquí viene ahora la coyuntura de describir la casa de
la Zaina y su gente, con las fiestas y bailes y el refresco aparatoso
que les ponía fin; pero como aún me resta por manifestar un poquito de
lo referente a D. Diego y a su vida, principal objeto que en este
comienzo del libro me propuse, dejo aquello para después y sigo diciendo
que el hijo de doña María, bien solo, bien acompañado de Santorcaz, iba
de tertulia alguna vez a las librerías principales, que era donde más se
hablaba de política.

No sé si recordaré todas las tiendas de libros que había entonces en
Madrid; pero sí puedo asegurar que casi igualaba su número al de las que
ahora existen, y las más concurridas eran las de Hurtado, Villarreal,
Gómez Escribano, Bengoechea, Quiroga y Burguillos (antes Fuentenebro) en
la calle de las Carretas; la de la viuda de Ramos, en la carrera de San
Jerónimo; la de Collado, en la calle de la Montera; la de Justo Sánchez,
en la de las Veneras; la de Castillo, frente a San Felipe el Real, y el
puesto de Casanova, en la plazuela de Santo Domingo. En estas tiendas se
reunían muchos jóvenes escritores o que pretendían serlo, poetas hueros
o con seso, aunque estos eran los menos; personas más aficionadas a la
conversación que a los libros, gente desocupada, noticieros, y
muchísimos patriotas. D. Diego era patriota.

Como yo me metía bonitamente en todas partes, también me daba una vuelta
por las librerías, bien acompañando a D. Diego, bien solo, echándomelas
de gran patriota, y en la de las Veneras, me acuerdo que dije una noche
muy estupendas cosas que me valieron calurosos aplausos. ¡Ay! allí
conocí al sombrerero Avrial y a Quintana, el mochuelo y el mirlo, el
cisne y el ganso de aquellos tiempos literarios, tan turbados, tan
confusos, tan varios y antitéticos en grandeza y pequeñez como los
políticos. Parece, en verdad, mentira que Moratín y Rabadán, que Comella
y Meléndez hayan vivido en un mismo siglo. Pero España es así.

Tampoco dejaba D. Diego de concurrir al teatro alguna que otra vez,
porque era muy de patriotas el ir a la representación de las famosas
comedias de circunstancias \emph{La alianza de España e Inglaterra},
\emph{con tonadilla}, y \emph{Los patriotas de Aragón y bombeo de
Zaragoza}, que en aquellos días se representaban con frenético éxito. Y
para que nada faltase en el círculo de relaciones de aquel joven
ilustre, también asomaba las narices por el cuarto de Pepilla González,
actriz famosa, si bien un día puso punto final a sus visitas porque le
hicieron no sé qué ingeniosa burla.

En casa de la Zaina, en casa de la Pelumbres, en la de la Naranjera, en
la logia de \emph{Rosa-Cruz}, en la librería de la calle de las Veneras,
y en el teatro solíamos encontrarnos D. Diego y yo, pues como he dicho
yo tenía especial empeño en seguirle a todas partes, venciendo para
entrar en algunas la repugnancia de mi conciencia. El joven se
franqueaba espontáneamente conmigo y yo mientras más me decía más
procuraba sacarle, para que ningún escondrijo ni pliegue de su vida me
fuese secreto. Sólo cuando iba en compañía de Santorcaz, me guardaba muy
bien de preguntarle ciertas cosas.

¡Pobre D. Diego y a cuántas pruebas se vieron sujetas su impetuosa
juventud e inexperiencia! ¡Y qué de simplezas hizo, y qué terribles
caídas tuvieron los atrevidos saltos de su entusiasmo, y qué porrazos se
dio con las peñas del fondo al arrojarse desaforadamente en el mar de la
vida, creyéndole sin arrecifes, ni sumideros, ni bajíos! ¡Y cuánto se
encanalló; y de qué extraña manera el mayorazgo poderoso, viose en
ocasiones pobre y miserable, con la circunstancia de que no podía menos
de sostener el pie de su lujo y representación! Como era tan manirroto,
gastaba en una semana la renta de un año, y aquí de los acreedores,
usureros, prestamistas, judíos y demás chupadores de sangre que se
bebían la de mi condesito. Este llegó a verse muy afligido, pues nadie
le fiaba ya el valor de una peseta, y recuerdo que cierta noche cuando
salíamos del teatro del Príncipe, D. Diego me hizo una pintura horrenda
de la plenitud de sus apuros y vaciedad de sus bolsillos; dijo después
que se iba a suicidar, y luego me llamó insigne varón, ilustre amigo y
el más caballeroso y caritativo de los hombres, siendo de notar que
todos estos rodeos, elipsis, metonimias e hipérboles terminaron con
pedirme dos reales. Dile cuatro que tenía y se despidió, suplicándome
que dijese algo en su favor a cierto prestamista llamado Cuervatón,
vecino mío, pues tenía pensado darle un tiento al siguiente día, aunque
las cantidades adeudadas subían al sétimo cielo. Yo le prometí
interceder en su favor, y deseándole las buenas noches entré en mi casa.

\hypertarget{iii}{%
\chapter{III}\label{iii}}

La cual era aquella misma honrada mansión donde fuí recogido, curado y
asistido en mi penosa enfermedad del mes de Mayo, y vea el lector cómo
de manos a boca nos encontramos de nuevo en la dulce compañía del Gran
Capitán y de su esposa, y en alegre familiaridad con el Sr.~de
Cuervatón, con D. Roque, con el lañador y respetable familia, con la
bordadora en fino y otras personas que si no gozan en la historia de
celebridad apropiada a sus méritos y eminentes calidades, tendranla en
esta relación, mal que le pese a la ruin envidia, siempre empeñada en
rebajar los altos caracteres.

Desde mi vuelta de Andalucía, yo moraba en casa de D. Santiago
Fernández. Santorcaz no vivía ya allí, ni tampoco Juan de Dios, ni sus
antiguos patronos sabían dónde estaba, pues habiendo salido cierto día
de Agosto muy de mañana, hasta la fecha de lo que voy contando, que era
por Noviembre, no había vuelto, lo cual hacía decir a doña Gregoria:

---No puede \emph{por menos} sino que a ese bienaventurado Sr.~de Arróiz
le ha sucedido alguna desgracia, como no se haya ido al cielo en cuerpo
y alma; que para eso estaba.

La casa (y aunque me parece que esto lo saben Vds. no estará de más
repetirlo) era de esas que pueden llamarse mapa universal del género
humano por ser un edificio compuesto de corredores, donde tenían su
puerta numerada multitud de habitaciones pequeñas, para familias pobres.
A esto llamaban casas de Tócame Roque, no sé por qué. No lo indagaremos
por ahora, y sepan que en aquellos días el que hubiera entrado en casa
del Gran Capitán, habría visto a este en el centro de un animado
corrillo, donde estábamos hasta ocho personas, todos buenos españoles, e
inflamados de patriótico afán por saber cómo iban las cosas de la
guerra; habría visto con cuánta diligencia y precipitación acudían unos
y otros en cuanto Fernández volvía de la oficina; habría visto cómo
amorosamente preparaba doña Gregoria el sahumado brasero, para que no se
enfriara la concurrencia; cómo Fernández, golpeando la caja de rapé,
tomaba un polvo, sonábase mirando a todos por encima del pañuelo, y
luego se apresuraba a satisfacer la sed de su curiosidad en estos
términos:

---La cosa va mejor de lo que se creía, y lo de Lerín no fue tan
desgraciado como se nos quiso pintar. Señores, hay que poner en
cuarentena lo que dicen los papeles impresos, porque los diaristas no se
cuidan más que de sorprender al público con noticiones, y como ninguno
de ellos sabe palotada de lo que llamamos el arte de la guerra\ldots{}

---Pues a mí me han dicho que lo de Lerín fue un desastre muy
grande---afirmó D. Roque.---¡Bah! Si tenemos unos generales\ldots{} De
lo que está pasando tienen ellos la culpa, y bien sabía yo que
vendríamos a parar a esto. Pues qué, si esos señores, en vez de estarse
en Madrid todo el mes de Setiembre mordiéndose unos a otros; si en vez
de estar aquí diciéndose «yo soy mejor que tú» y disputándose el mando
de los cuerpos como perros que riñen por un hueso; si en vez de esto,
digo, se hubieran marchado al Norte a perseguir al enemigo, ¿estarían
los franceses tan envalentonados?

---Tiene razón que le sobra por los tejados el Sr.~D. Roque---dijo la
mujer del lañador.---Y yo, que no sé de guerra, le decía a mi marido
todas las noches cuando nos acostábamos: «Mira, Norberto, los generales,
en lugar de estar aquí y en Aranjuez hablando mal unos de otros y
revolviéndolo todo con sus envidias y reconcomios, debieran andar por
toda esa tierra de Burgos y Rioja persiguiendo al francés. Que si Llamas
manda tal tropa; que si ya no la manda Llamas sino Pignatelli; que si
Castaños se opone a que venga Cruz; que si Blake quiere ser más que
Cuesta y Cuesta más que todos; que si Palafox manda este cuerpo; que si
La Peña no quiere mandar el otro\ldots{} en fin, cuando después de la
batalla de Bailén creímos vernos libres de franceses, y emperadores, y
reyes de copas, ahora salimos con que por estarse los generales mano
sobre mano en Madrid al olorcillo de la corte y de los obsequios y de
las fiestas, han dejado que los otros se arreglen bien y tengan
dispuesto todo para darnos un susto.»

---Ha hablado Vd. como un padre de la Iglesia, señora doña María
Antonia---dijo con oficiosa exaltación doña Melchora, la bordadora en
fino.---A mis niñas les dije yo eso mismo el mes pasado. ¿No es verdad,
Tulita, no es verdad, Rosarito? Sí, señores, esa es la pura verdad; yo
voy viendo que desde que empezó la guerra, desde que hubo aquello de
venir los franceses y caer Godoy, nadie ha sabido acertar más que
nosotras, y cuando anunciábamos lo que iba a pasar, los hombres graves
se reían diciendo: «¿Qué entienden las mujeres de guerras, ni de
historias?» Pues vean ahora si entendemos.

---Tiene razón doña Melchora---dijo el señor de Cuervatón.---También se
reían de mí cuando anuncié lo que iba pasar. Pero, señores; cuando los
de arriba pierden la chaveta como ha pasado aquí, a los tontos y a las
mujeres corresponde el imperio del buen sentido.

---No obstante---dijo el Gran Capitán, impaciente por poner el peso de
su autorizado dictamen en aquella contienda,---aún no se puede hablar
mal de esos valientes generales. Yo no les he explicado a Vds. todavía
el plan de campaña. Es preciso que Vds. se penetren bien de esto. Las
tropas que mandan Blake, Llamas, Castaños y Palafox, colocadas y
extendidas desde el Ebro hasta Burgos, forman un gran semicírculo.
Vienen los franceses: el semicírculo se cierra convirtiéndose en
círculo, y aquí me tienen Vds. a mi emperador cogido en una ratonera.

---Pero en resumidas cuentas, ¿viene o no viene?---preguntó doña
Melchora.

---Yo creo que no---dijo el Gran Capitán, echándosela de malicioso.---Y
tengo para mí que todo eso que dicen los papeles acerca de lo que
Napoleón leyó en el Senado, es pura invención. Como que hay quien dice
que Napoleón está muy enfermo de un tumor que le ha salido en el sobaco
izquierdo, y que ya le han sacramentado.

---¿Y Vd. es de los que dan crédito a los mil desatinos que cuentan los
patriotas?---exclamó D. Roque levantándose de su asiento.---Aquí creen
que se sale del paso contando mentiras y matando de calenturas o
alfombrilla a todos nuestros enemigos.

---Y qué, ¿soy hombre para tragar todas las bolas que cuentan
diariamente los papeles?---dijo el Gran Capitán sin disimular el
desprecio que le merecía la prensa.---Vamos a ver, ¿qué saca Vd. en
limpio, Sr.~D. Roque, de todas esas hojas que lee día y noche, y que le
van a volver loco como al bueno de D. Quijote los libros de caballería?

---Quédese cada uno en su sitio, y no se meta en los trigos
ajenos---repuso D. Roque procurando contener su irascibilidad,---que así
como yo no me meto jamás en las honduras del arte de la guerra que no
entiendo, así debe Vd. respetar las ciencias que no están a su alcance.
¡Qué sería de la sociedad sin papeles públicos! Aquí tengo yo el
\emph{Semanario Patriótico}---añadió sacando un voluminoso legajo de uno
de los luengos bolsillos de su levitón,---que es el mejor papel que
hasta ahora se ha escrito, y contiene cosas muy lindas, y en todo lo que
dice no parece sino que habla por boca de Aristóteles y Platón. Desde
que en el primer número vi aquello de La \emph{opinión pública es mucho
más fuerte que la autoridad malquista y los ejércitos armados}, les digo
a Vds. francamente que el tal papelito me enamoró. Yo me quito el
garbanzo de la boca para ahorrar los 20 reales que me cuesta cada
trimestre; y ¿cómo no hacerlo, si este manjar del espíritu es tan
necesario a la vida como el alimento del cuerpo? Así es que los
miércoles por la noche no duermo y todo es dar vueltas en la cama,
pensando en lo que traerá el \emph{Semanario} al siguiente día. Los
jueves son para mí días de delicia, y leyendo mi \emph{Semanario}
olvídaseme el comer y el beber, a más de todas mis penas y tristezas que
son muchas. ¡Y cómo trata todas las cuestiones! ¡Y con qué gracia le da
a cada uno lo que es suyo! ¡Y qué sal tiene para decirle a la Francia
todas sus picardías! ¿Pues y el paralelo que hace entre Bonaparte y
Maximiliano Robespierre? No pierde ripio para decir a todos las
verdades, y a los españoles les suele sacar los trapitos a la colada,
como quien dice. En fin, señores, me entusiasma tanto, que el otro día,
no pudiendo satisfacer mi deseo de conocer al autor de tan divino
escrito, y averiguado que lo es un tal Manolito Quintana, me fui derecho
allá, y abrazándole le dije: «Venga acá el extremo de toda discreción,
el resumen de la elocuencia y del buen decir, el dechado de la lengua
castellana, el azote de los tiranos, el heraldo del patriotismo y el
cisne de los derechos del hombre.» A lo cual me contestó que él cumplía
con su deber y que agradecía tales alabanzas.

---¿Toda esa arenga le echó Vd. al buen autor del \emph{Semanario
Patriótico}?---dijo el Gran Capitán.---Pues en verdad digo que si la
Junta oyera mis consejos, al punto mandaría suprimir ese y todos los
demás papeles. ¿Para qué se quieren papeles?

---Hombre irracional, ¿y cómo se difunden las luces y se propaga la
buena doctrina, y se instruye a toda la gente del reino, chicos y
grandes? Pues malitas verdades trae el \emph{Semanario
Patriótico}\ldots{} Como todos dieran en leerlo con tanto fervor como
yo, pronto se remediarían los males de la Nación. Y no hay que darle
vueltas, señores, lo que este dice es el Evangelio. ¿Quién podrá
desmentir aquello de \emph{el tirano es un hombre que abusa de las
fuerzas de la sociedad para someterla a sus pasiones propias, y así la
tiranía no es otra cosa que la injusticia apoyada en la violencia}? ¿Qué
tal? ¿Pues y dónde me dejan Vds. aquello de los derechos
\emph{esenciales, sagrados e imprescriptibles} que corresponden al
hombre, y que le usurpa el pícaro del poder absoluto?\ldots{} Nada,
nada, Sr.~D. Santiago, amigo Cuervatón, señoras y señoritas: tengan Vds.
presentes estas palabras: «La violencia, la opresión, la credulidad,
llegan frecuentemente a adormecer a los pueblos, a fascinar su
entendimiento, a quebrantar en ellos los resortes de la naturaleza; pero
cuando por favorables circunstancias abren los ojos y oyen la voz de la
razón; cuando la necesidad les fuerza a salir de su letargo, entonces
ven que los pretendidos derechos de sus tiranos, no son sino efectos de
la injusticia, de la fuerza o de la seducción; entonces es cuando las
Naciones, acordándose de su dignidad, ven que ellas no se han sometido a
la autoridad sino para su bien, y que jamás han podido dar a nadie el
derecho irrevocable de hacerlas felices.»

\hypertarget{iv}{%
\chapter{IV}\label{iv}}

Dotado de maravillosa memoria, D. Roque recitaba trozos enteros de lo
que había leído en sus papelitos, sin mudar una sílaba. No he conocido
varón más sencillo e inofensivo que aquel fogoso lector del Semanario,
comerciante que había venido muy a menos y a la sazón, sin negocios, sin
familia y con poquísimo dinero, vivía en aquella casa, manteniéndose con
su casi invisible renta. Así como el Gran Capitán oyó lo de la opresión
y la injusticia, con los razonamientos puestos a continuación, que no
entendiera menos, si estuvieran escritos en caldeo, se encaró con su
amigo, y burlonamente le dijo:

---¿Se ha acabado la jerga? Qué lástima que no viniera por aquí el padre
Salmón, para que le contestase, y entre los dos se armara una marimorena
de \emph{distingo acá}\ldots{} \emph{distingo allá}\ldots{}
\emph{necuacua}\ldots{} \emph{útiquis}\ldots{} \emph{reñega
mayora}\ldots{} y otras palabrillas que se usan en las disputas de los
\emph{tiólogos}.

---¡Teólogos a mí! ¡A mí teólogos y con cascabeles!\ldots{} ¡Y de la
madera del padre Salmón!---exclamó D. Roque guardando el Semanario en el
almacén de sus profundas faltriqueras.

---Y ha de venir esta tarde Su Paternidad---dijo agridulcemente la menor
de las hijas de doña Melchora,---pues prometió darme una receta para
este mal de la barriga que ha diez días tengo.

---Sí que vendrá---añadió la mayor,---pues quedé en pegarle dos botones
en el cuello, y él dijo que traería la cinta azul.

---Pronto tendremos aquí a ese reverendo Salmón---añadió doña
Gregoria,---y ya tengo echada la llave a la despensa, porque para
saqueos bastante tenemos con los de los franceses.

No había concluido estas palabras la discreta esposa de Fernández,
cuando se oyó en el patio de la casa gran ruido de voces, entre las
cuales descollaba una cencerril, abajetada y bronca, que no era otra
sino la de aquel lucero de la Merced, el padre Anastasio José de la
Madre de Dios, vulgarmente conocido por padre Salmón; que este era su
apellido, y no Salomón como algunos le llamaban sin intención de burla.

---Ahí está, ahí está ese bendito---dijeron en coro las hembras de la
reunión.---Gabriel: corre y tráele acá, porque si le cogen por su cuenta
las del polvorista\ldots{} ¡ay, qué pesadas son! Ya están llamándole las
escofieteras. Pues no, no ha de venir sino acá.

Salí para impedir que la persona del reverendo fuera secuestrada por
cualquiera de las familias que salían a su reclamo por las diversas
puertas que se abrían en aquellos largos corredores, y lo primero que vi
fue al fraile rodeado de enjambre de chiquillos, los cuales haciendo mil
cabriolas y juegos en su derredor, le mostraban según su arte propio, la
satisfacción de la casa toda por verle en ella.

---Tomad, piojosos, tomad esas almendras fallidas que para vosotros
serán bocado de ángel---les decía el padre.---¿Ya salió tu padre de la
cárcel, Jacintillo? Y por fin ¿llevasteis a vuestra abuela a los
Desamparados? Dime, hijo de la Canela, ¿está el oficialillo en el cuarto
de tu madre? ¿Con que se os murió la gallina?

Y al mismo tiempo el antepecho del vasto corredor parecía la barandilla
de un teatro, pues no había un palmo vacío, sino que allí estaba la
vecindad toda, aguardando a que Su Paternidad subiese.

---Venga acá, padre, que este trapalón de mi marido me quiere pegar por
celos. Pero di, cabeza jilvanada, ¿no soy la mujer más honrada del
mundo?

---Venga acá, padre, y verá qué chocolate le tengo. ¿Pues no me está
diciendo la capitana que Su Paternidad le comió ayer todas las magras?

---Venga acá, padre, y suba pronto que ya le apunta el diente a la niña.
Míralo allí, cordera, resol, reina del mundo. Mírale, llámale con tu
manecita\ldots{} así, así.

---Venga acá, padre, que ya parió la Zoraida cinco criaturas como cinco
estrellas.

---Suba pronto, padrito, que mi abuela pregunta si se le deben dar más
friegas.

Y así continuaban llamándole de distintas partes, cada uno según para
aquello que le necesitaba y todos con tan cariñosas palabras, que Salmón
no sabía a qué sitio volverse, ni a cuáles solicitaciones contestar más
pronto; y saludando a un lado y otro como un matador de toros que en
medio de la plaza hace cortesías a la redonda, mostró a sus amigos que
su corazón no era insensible a tantas bondades. En esto llegué yo y
besándole la correa, le dije:

---Doña Melchora y sus niñas, que están en casa del Gran Capitán, me
mandan para suplicar a Su Reverencia que tenga la magnanimidad de subir,
que allí le aguardan también don Roque, el Sr.~de Cuervatón y doña María
Antonia.

Pero antes que concluyera, el padre Salmón, con gran sorpresa mía, clavó
en mí sus ojos llenos de admiración, y echándome los brazos al cuello,
exclamó a gritos:

---Ven acá, portento de la sabiduría, milagro de precocidad, fruta
temprana de las humanas letras. ¿Con que ha más de un año que te conozco
y hasta hoy mismo he ignorado que eres un gran latino, autor del más
famoso poema que han escrito modernas plumas? ¿Con que así te callabas
tus méritos, picarón?\ldots{} A ver, muéstrame pronto ese poema\ldots{}
¡Quién me había de decir, cuando te conocí paje de la González, que bajo
la montera de tal gaterilla estaba el cacumen de un \emph{Erasmus
Rotterodamensis}, de un \emph{Picus Mirandolanus}!

Turbado y confuso le contesté que sin duda Su Paternidad se equivocaba
confundiendo mi ignorancia con la sabiduría de algún desconocido de mi
mismo nombre, oyendo lo cual, dijo mientras subíamos la escalera:

---No; que lo acabo de saber por el licenciado D. Severo Lobo, el cual
te conoció desde el proceso de El Escorial y luego estuvo a punto de
empapelarte, cuando el príncipe de la Paz te quiso dar una placita en la
interpretación de lenguas. ¿Y tú qué culpa tenías de que el otro te
quisiera colocar? Por lo que me han dicho, tu modestia iguala a tus
méritos; ¡oh joven! yo he visto la minuta en que Godoy te recomendaba;
pero qué guardado te lo tenías, raposilla\ldots{} ¿Y tú en qué te
ocupas? ¿Por qué no pides un hábito; por qué no eres fraile? Yo me
encargo de catequizarte. ¿Sabes que he hablado de ti a los padres de la
Merced y todos quieren conocerte? A ver si te pasas por allí, rapaz; y
ve después de la hora del refectorio. ¿Te gustan las pasas? Además tengo
que conferenciar contigo, Horacio Flacco en ciernes y Virgilio en
pañales; y como al salir de esta casa se me olvide hablarte (pues ya
sabes que soy muy débil de memoria), ¿me lo recuerdas, eh?

A tal punto llegaba, cuando entramos en la sala del Gran Capitán.
Levantáronse todos, y después de besarle uno tras otro la correa,
diéronle el asiento del centro junto al brasero.

---Aquí está la seda azul---dijo el mercenario dando lo indicado a
Tulita.

---Mañana mismo tendrá Su Paternidad arreglado el cuello---contestó la
muchacha.---Veamos ahora lo que me manda para este malestar de la
barriga, que es tal que yo no lo puedo resistir, y todas las mañanas me
dan unas arcadas, unos mareos y bascas tan fuertes, que no me para
dentro nada.

---Bendito sea el nombre de Dios---exclamó el padre tomando un polvo de
la caja del Gran Capitán.---A fe, doña Melchora, que si esta matutina
estrella de su hija de Vd. fuera casada, ya sabríamos el pie de que
cojea su estómago; pero no siéndolo, y tratándose ahora de una familia
con quien la misma honradez no podría ponerse en parangón, ordeno y
mando que con siete palitos del árbol de Santo Domingo, cocidos en
baño-maría, por espacio de tres credos rezados con pausa, y por supuesto
con devoción, esta niña se quedará como nueva. ¡Qué nueces frescas las
de ayer, señora doña Melchora! ¡Qué nueces frescas! Pero dígame, ¿qué
santo del cielo le hizo tan rico presente? Yo no sabía que en montes
alcarreños, asturianos ni encartados existiesen unas tan hermosas obras
de Dios.

---Obsequio fue de un primo mío que es guarda de las dehesas del señor
duque de Altamira, en tierra de Cameros, y como, sino de buen salario,
el pobrecito disfruta de ojos listos y manos libres, siempre nos manda
lo mejor de aquellos castañares y nocedales.

---Así le hicieran canónigo---añadió Salmón.---¿Y qué noticias, Sr.~D.
Santiago Fernández?

---No me digan nada, ni me calienten más la cabeza---exclamó el Gran
Capitán encubriendo bajo la ficción de un estudiado cansancio el placer
que le causaba el ver sacado a plaza un tema tan de su gusto.---Mire Su
Paternidad que estoy ya que no doy por mi cuerpo un real. ¡Qué ir y
venir! ¡Qué jaleo! ¡Todo el día poniendo nombres en la lista, y haciendo
recuento de cartuchos, y examinando armas, y disponiendo, y mandando!
Aquellos señores son muy remolones, y todo lo tengo que hacer yo.

---¿Y resistiremos, si como dicen, se nos viene encima ese monstruo, ese
troglodita, ese antropófago, señores, que no se sacia nunca de devorar
carne humana?

---¡Pues no hemos de resistir!---exclamó el Gran Capitán.---¿Hemos de
ser menos que los zaragozanos? Además de que yo creo que no viene.

---¡Y sabe Dios---dijo doña María Antonia,---si será cierto lo que dicen
de que allá en Rusia o Prusia le echaron unos polvitos en el cocido para
que reventara!

---Como que hay quien asegura que está sacramentado y que hizo
testamento, devolviendo todas las naciones que ha robado y abjurando de
sus herejías.

---¡Oh gente ignorante y crédula!---exclamó de improviso D. Roque,
desenvainando su cartapacio de papeles públicos.---¡Y cómo se conoce la
rusticidad de los que atienden más a los dichos y simplezas del vulgo
que a la palabra impresa de los hombres doctos! Vean, vean lo que dice
ese papel, y no hagan caso de tonterías: «Napoleón se presentó al Senado
el 25 del pasado, y dijo que \emph{bien pronto pondría sus banderas en
las torres de Madrid y en las fortalezas de Lisboa»}. También cuenta la
Gaceta que ciento sesenta mil hombres del ejército grande están sobre la
frontera de España, y que el Emperador dijo que \emph{antes de fin de
año no quedará aquí una sola aldea en insurrección}.

---Con que ni una sola aldea\ldots---dijo el fraile.---Pero sabe Dios la
intención que llevará el que ha escrito esos papeles. Lo que es por mí,
mandaría suprimir todos los que se imprimen en España, pues para
envolver especias, mejor es el papel no impreso y limpio como sale de
las fábricas.

---¿Pues eso qué duda tiene?---dijeron a una las dos niñas de doña
Melchora.

---Y yo---exclamó como un basilisco don Roque,---mandaría suprimir todos
los frailes o les quitaría el hábito, dando a cada uno un fusil para que
fueran a limpiar a España de franceses.

---Sin fusil lo hacemos, hermano---dijo Salmón riendo.---Lejos de
suprimir frailes, yo los aumentaría en grado máximo, y así la mayor
parte de los españoles vivirían gordos y contentos, y no veríamos tanto
vagabundo mendigo por esas calles.

---Chúpate esa y vuelve por otra---dijo a D. Roque la menor de las hijas
de la bordadora en fino, suponiendo al viejo completamente apabullado
bajo el peso de aquellas incontestables razones.

---¿Con que más todavía? Pues sepa mi señor Salmonete---dijo D. Roque,
llevando al último extremo su familiaridad con el fraile,---que ahora se
va a reunir la nación en Cortes. ¿No lo quieren creer? ¡Ah! Pues no doy
dos maravedises por lo que de Gobierno absoluto hubiere después de la
guerra. ¡Abajo los tiranos! ---añadió poniéndose en pie y alzando los
brazos con endemoniada exaltación.---Y si hay un frailazo chocolatero
que me desmienta, alce la voz, y venga delante de mí, que yo le reto a
singular polémica, aunque traiga más textos que escribió Pedro Lombardo,
y más latines y aforismos y comprobatorias y distingos que han eructado
en diez siglos las cátedras salmantinas y complutenses.

---¿Y cómo había yo de ponerme a disputar con semejante pedazo de
acebuche con nudos, más duro que roca? ¿Y de qué valdrían mis argumentos
contra la asnal cerrazón de su mollera?---exclamó el padre Salmón
levantándose también de su asiento; mas no enfadado ni nervioso, sino
riendo a todo reír, pues su humor de mantequillas era tal que no se le
vio colérico mas que una sola vez.

---Pues empecemos---dijo D. Roque poniéndose verde.

---Empecemos---añadió Salmón restregándose las manos y haciendo después
grotescos gestos, como de quien imita los movimientos de un grave
predicador.

---No quisiéramos más para reírnos de don Roque---dijo la mayor o la
menor (que esto no lo tengo bien presente) de las hijas de doña
Melchora.

---Pero para restaurar nuestras fuerzas, señores y señoras mías---dijo
Salmón,---venga ese chocolate, que aquí mi amigo D. Roque dice que no se
puede pasar sin él.

---Quien no se puede pasar sin él---contestó el aludido,---es su
magnificencia reverendísima, que en llegando a estas horas, como no
ponga un puntal al estómago, se cae rendido.

---Pues Vd. lo dice, amigo papelista eminentísimo---contestó Salmón
dando otra vez rienda suelta a la risa,---así sea, y venga ese
chocolate; y pues es más agradable el goce de una amena tertulia que el
disputar, dejémonos de discusiones, y pelillos a la mar, y cada uno
piense lo que quiera, y ruede la bola, y viva Fernando VII.

---Es lo más conveniente, toda vez que este D. Roque está
chiflado---dijo Fernández,---y un día hemos de verle por esas calles con
una \emph{Gaceta} en cada dedo.

---¡Pero qué graves y circunspectas están mis niñas!---añadió Salmón
dando unas palmaditas en el hombro, no recuerdo bien si de la mayor o de
la menor de las hijas de doña Melchora.---Y esos piquitos de oro, ¿por
que no echan una canción por todo lo alto, para que se nos alegren los
espíritus?

---Bueno, bueno.

Y una de ellas rompió al instante a cantar de esta manera:

\small
\newlength\mlena
\settowidth\mlena{\qquad \qquad Bonaparte en los infiernos}
\begin{center}
\parbox{\mlena}{\quad Con un albañilito                   \\
                Madre, me caso,                           \\
                Porque son de mi gusto                    \\
                Los hombres blancos.}                     \\
\end{center}
\normalsize

---Eso tiene poca gracia---dijo Salmón.---A ver otra.

---Pues allá va la que está de moda:

\small
\newlength\mlenb
\settowidth\mlenb{\qquad \qquad Bonaparte en los infiernos}
\begin{center}
\parbox{\mlenb}{\quad Bonaparte en los infiernos          \\
                Tiene su silla poltrona,                  \\
                Y a su lado está Godoy                    \\
                Poniéndole la corona.                     \\
                 \quad \quad \quad  Sus compañeros        \\
                 \quad \quad Van de dos en dos;           \\
                 \quad \quad Murat, Solano,               \\
                 \quad \quad Junot y Dupont.}             \\
\end{center}
\normalsize

---¡Bravo, magnífico! Doña Melchora, tiene Vd. dos niñas que envidiaría
cualquier princesa. Y qué tal, ¿se gana mucho?

---En estos tiempos, padrito---dijo la madre,---suele caer algún bordado
de uniforme; pero ¿dónde se ven aquellos ternos de plata y oro, aquellas
estolas, aquella ropa de altar que tanta ganancia nos daban antes de
estas malditas guerras? Ya sabe su grandeza que las mejores capas
pluviales, las mejores casullas que se han lucido en procesiones, así
como las mejores chaquetas toreras que han brillado en plazas y
redondeles, pasaron por estas manos. ¡Ay, quién me lo había de decir!
¡La que bordó los calzones que llevaba Pepe-Hillo cuando le cogió aquel
enrabiscado toro; la que bordó la capa que llevaba en sus santos hombros
el Eminentísimo Cardenal de Lorenzana el día que tomó posesión, está hoy
consagrada a miserables letras de cuello de uniforme, y a las dos o tres
insignias de consejero, o ropón de Niño Jesús, que caen de peras a
higos! ¡Buenos están los tiempos!

---Cuando esto se acabe\ldots---dijo el fraile.

---¿Cómo, cuando esto se acabe?---gritó de improviso D. Roque
interrumpiendo con muy feo gesto a su amigo.---Antes, muy antes de que
esto se concluya se reunirá el país en Cortes. ¡Y estos alcornoques no
lo quieren creer!

---Que te despeñas, Roque amigo.

---¿También eso lo dicen los papeles?---preguntó con mucha sorna el Gran
Capitán.

---También lo dicen, sí señor. Pues no lo han de decir. Y cómo se me ha
de olvidar, si lo sé de memoria y anoche, luego que me acosté, estuve
recitando en voz alta aquello de\ldots{} «Después de tantos años de
abatimiento y opresión en que los leales y generosos españoles han
sufrido mayores ultrajes y vilipendios que los salvajes africanos,
amanecerá el glorioso día en que se reúnan los pueblos por medio de sus
representantes para tratar del bien común. Este es el objeto con que se
instituyeron las sociedades civiles; no el engrandecimiento de un solo
hombre con perjuicio de todos los demás. Reunidas aquellas, es como
puede conocerse afondo el estado de una nación, sus recursos, sus
necesidades y los medios que deben adoptarse para su bienestar y
prosperidad; y donde faltan estas solemnes Asambleas, los monarcas, mal
aconsejados, caminarán ciegamente al despotismo, tal vez contra sus
buenos deseos.»

---¡Lindísimo sermón!---exclamó el Gran Capitán.---Ayer le contaba a mi
compañero en la portería de Cuenta y Razón las extravagancias de mi
vecino D. Roque, y me dijo que esto se llamaba \emph{el democratismo}.
¿Es así, padre?

---Llámese como se quiera---repuso el venerable Salmón,---lo que digo es
que este chocolate, que ahora nos trae la señora doña Gregoria, y cuyo
olor se adelanta hasta nosotros anunciándonos la nobleza de lo que viene
en el cangilón, me parece tal, que sólo podría servírsele semejante al
Sumo Pontífice.

---Y a la abadesa de Las Huelgas de Burgos---dijo doña Gregoria;---que
ella y el Papa son las dos más altas personas de la cristiandad, y por
eso se dice que si el Papa se casara, la única mujer digna de ser su
esposa es la tal abadesa de Las Huelgas.

---Así es---añadió Salmón, olvidándose de todo lo que no fuera el
cangilón;---y por lo que hace a eso del \emph{democratismo}, yo le
aconsejo a D. Roque que se deje de tonterías y no piense en novedades,
pues por ahora que ahora y en muchísimos años para adelante, estamos y
estaremos libres de ellas.

---Los españoles guerrean porque no quieren que los manden los
franceses---dijo la mayor de las hijas de doña Melchora,---y también
para defender los usos y \emph{pláticas} del reino contra las novelerías
que quiere poner aquí Napoleón. Así me lo dice todos los días Paco el
plumista, que es sargento de voluntarios.

---Pues a mí me dijo Simplicio Panduro, ese saladísimo paje de D. Gaspar
Melchor de Jovellanos---añadió la otra,---que los españoles guerrean por
echar a los franceses y por mejorar la mala condición de los reinos,
quitando las muchas cosas malas que hay, al modo de lo que dice D. Roque
por las noches cuando predica a solas y a oscuras en su cuarto.

Estas dos opiniones dieron pie a una acalorada disputa que no copio
porque nada sacarían de ella en limpio mis lectores, toda vez que es
público y notorio que en lo que va de siglo, la historia, la grave y
cachazuda historia no ha podido dilucidar la cuestión planteada por
aquellas dos niñas, y aun hoy andan a la greña eminentes escritores por
averiguar si decía verdad la mayor o la menor de las hijas de doña
Melchora.

Salmón, consumido su chocolate, dijo:

---Con que, amiguitos, ¿me dan Vds. su venia para retirarme?

---¿Tan pronto, padre? ¡Que siempre nos ha de tener Vuestra Reverencia
con hambre de su compañía!

---Bastante os acompaño, hijitas mías.

---Pues siempre nos sabe a poco.

---Ya sabéis que tenemos en casa desde esta tarde \emph{octava misión y
solemnes cultos para desagraviar a Jesús Nazareno y a María Santísima,
de los sacrílegos insultos que han sufrido en nuestros templos, de los
impíos ejércitos franceses, e implorar de la divina misericordia que
robustezca y ampare a nuestros soldados y conserve y dirija en todos los
negocios a los que nos gobiernan. Después habrá procesión a la Virgen de
la Paloma, patrona de todo el majerío.} ¿Pero no lo sabíais, pajaritas
volanderas? Por supuesto que no faltaréis el día que me toque predicar.

---Antes faltará la tierra y prados en ella, como dijo el otro.

Ya estaba en pie para retirarse el padre mercenario, cuando el Sr.~de
Cuervatón, que poco antes había sido llamado de su casa, donde le
esperaba una visita, volvió dando voces; y lleno de cólera, que en los
ojos con fulminantes rayos le centelleaba, habló así:

---¡No sé cómo no le ahogo!\ldots{} ¡Vaya con el lindo currutaco, harto
de ajos!\ldots{} ¡Cuando creí que vendría a pagarme, viene a pedirme más
dinero!\ldots{} ¡Y ahora sale con que su señora mamá es muy rica!
Miserable, pringoso, vestido con harapos de príncipe, ¿por qué esa
señora no reventó antes que os pariera?

---¿Qué hay, Sr.~de Cuervatón? ¿qué le pasa?

---Que después que me estoy arruinando por favorecer con mi pequeña
hacienda a los necesitados, he aquí que un señor condesito de Rumblar o
de Barrabás con pintas, me debe más de nueve mil reales, y después de no
pagarme ni un céntimo de interés (que no son más de peseta por duro al
mes), viene a pedirme más dinero. Canalla, catacaldos: ¿qué me importa
que sea noble y que le vayan a caer dos mayorazgos?

---¿D. Diego de Rumblar?---dijo Salmón: y luego volviéndose a mí
añadió:---no olvides, Gabriel, que tenemos que hablar.

---Pues o me paga---prosiguió Cuervatón,---o el mejor día le desnudo en
medio del Prado delante de las damas.

En esto salimos al corredor, y ¡oh espectáculo lamentable! se ofreció a
nuestra vista el de D. Diego azuzado en medio del patio por todos los
chicos de la vecindad como novillo en plaza. Muchas mujeres habladoras
habían salido por los cien agujeros de aquella colmena, y unas con
cáscaras de castañas, otras con palabras picantes le mortificaban en lo
moral y en lo físico. Especialmente la mujer de Cuervatón, que era una
hidra con más rabos y espinas y escamas en su alma, que las mitológicas
en su cuerpo, poniéndose de pechos en el barandal, después de escupirle,
le decía:

---Tío pingajo de oro, ¿tenemos nuestro dinero para mantener
haraganes?\ldots{} ¿Ahorramos nosotros para daros esa agua de bergamota
de que apestáis? Coma Vd. clavos, y si es noble y espera mayorazgos,
póngase a roer sus \emph{jicutorias}, o coja una espuerta y vaya a
vender arena, como hacen mis dos hijos, que aunque no les falta para
comer y vestir como niños de príncipe, andan al trabajo de la arena
desde que saben llevar la mano a la boca. Cuidado con el señorito D.
Pelagatos; y dice que es conde\ldots{} Conde es él como mi abuelo. Ea,
muchachos, rociadle un poco con la esencia de ese fango de azahar
argentino que hay en el patio\ldots{} Coged también esas cáscaras de
nuez, y la ceniza de aquel braserillo.

Los muchachos que esto oyeron, y que se habían adelantado a poner en
ejecución \emph{auctoritate propria} lo del rociar, descargaron sobre el
infeliz D. Diego, a punto que este salía, tal lluvia de inmundas
sustancias, le persiguieron tan encarnizadamente por el portal y luego
por toda la calle del Barquillo, que daba compasión ver al infeliz
magnate corrido, avergonzado y lloroso.

El padre Salmón, que era hombre caritativo, reprendió a los muchachos su
grosería, y a la señora de Cuervatón su crueldad. Cuando se dispuso
abajar, todos se lo disputaban, no queriendo dejarle de la mano: este le
enseñaba los cinco perritos recién paridos por Zoraidilla, aquel le
hacía tocar con el dedo el diente de la niña, uno le pedía receta para
el dolor de muelas, otro le cantaba una seguidilla nueva, y todos le
daban tales muestras de cariño y admiración, que bien se le podía
considerar como el hombre más popular de su tiempo.

Cuando bajaba, allí eran de oír las exclamaciones, las palmadas, los
vítores, y de ver los besos de correa, y el pedir y dar bendiciones.

---¿Cuándo me receta para estos desmayillos?

---Ya sé de cabo a rabo la oración a San Antonio. ¿Cuándo se la echo a
Su Paternidad?

---Razón tenía el padrito en decir que el aguardiente de Chinchón da
mejor gusto a los puches que el de Ocaña, y que no hay plato de lentejas
sin dos ajitos machacados. Así lo hemos hecho.

---Padre, ¿las ranas son carne, o son pescado? porque mi abuela las
comió el viernes y está llena de escrúpulos.

---¿Qué nombre le pondremos a lo que ha de venir si sale macho?
Pondrémosle Anastasio como Su Reverendísima, en señal de agradecimiento
por habernos ayudado a criar al mayorcito.

---Ya están compradas las dos velas para la Virgen de la Buena Dicha, y
aquí Ramona las está adornando con flores y lentejuelas.

---Viva cientos de miles de años su magnitud sapientísima y
empingorotadísima para alivio de estos pobres a quienes socorre.

Y así continuaban hasta que el padre salía a la calle. No; no ha
existido hombre más popular que el padre Salmón. Casi, casi estoy por
asegurar que su popularidad excedió dos dedos y aun tres a la de
Fernando VII. ¡Desventurado Salmón! Oh tú, varón felicísimo, harto de
lisonjas, de regalos y de bienestar; oh tú, teólogo de tumba y hachero,
predicador burdo y de cuatro suelas, fraile mercenario que si no
redimiste ningún cautivo, tampoco hiciste daño a nadie; oh tú, hombre
dichoso sobre todos los dichosos de la tierra, pues no cavilaste jamás
ni te apasionaste, ni aborreciste, ni padeciste mal alguno en muchos
años, ni viste turbada tu apacible existencia: ¡quién te había de decir
entonces que aquel mismo pueblo tan solícito en victorearte, en
regalarte en aplaudirte, en venerarte y adorarte como a persona divina,
te había de coser a puñaladas veinte y seis años después en la
enfermería de tu santa casa, y cuando ya viejo, enfermo, inválido y sin
alientos no pensabas más que en Dios! ¡Quién te había de decir que aquel
mismo pueblo de quien fuiste ídolo, te había de echar al cuello un
cordel de cáñamo para arrastrarte por los profanados claustros,
sirviendo tu antes regalado cuerpo de horrible trofeo a indecentes
mujerzuelas! ¡Ay! ¡lo que es el mundo y que cosas tan atroces ofrece la
historia! Y así es bien que digas: si buen chocolate sorbí, buenos palos
me dieron; si buenos abrazos, y agasajos, y besos de correa recibí, con
buen pie de puñaladas se lo cobraron.

\hypertarget{v}{%
\chapter{V}\label{v}}

Pero como nada de esto viene ahora al caso, voy a dar cuenta del asombro
que me causó la conversación que inmediatamente después de su salida
tuve con aquel popularísimo fraile; y lo ocurrido fue que apoyándose en
mi brazo para descargar sobre él parte del peso de su bien aprovechada
humanidad, me dijo:

---Gabriel, o mejor, Sr.~D. Gabriel, pues a todo un Pico de la Mirandola
se le debe tratar con miramiento: has de saber que necesito que me
informes detenidamente de la vida de ese D. Diego de Rumblar, en cuya
compañía te he visto varias veces. Tú dirás que qué me importa a mí si
el tal niño canta o llora; pero a esto te respondo que no soy yo quien
tiene interés en saber sus malas mañas, sino una elevadísima familia,
cuya casa frecuenta mi inutilidad las más de las tardes. Como D. Diego
está para casar con la niña, las señoras, que ya barruntan la mala vida
que lleva el rapaz en Madrid, están muy disgustadas. Ayer cuando afirmé
que le había visto en esta casa, me dijo la señora condesa: «Por Dios,
padre Salmón, haga Vd. el favor de averiguar con qué hombres se junta, a
qué sitios va, en qué gasta su dinero, porque si es cierto lo que
sospechamos, antes se hundirá el cielo que entre él en nuestra familia.»

---Pues el señor conde---le respondí,---es un poco calavera. Cosas de la
juventud\ldots{} yo creo que se enmendará.

---Se enmendará. Luego es malo. Bien, Gabriel. Has dicho lo que
necesitaba saber. ¿A dónde va por las noches? ¿Con quién se junta?

---Todo lo sé perfectamente---respondí,---y no da un paso sin que yo me
entere de ello.

---¿De modo que podré satisfacer a la señora condesa? ¡Oh! Bendito seas,
que me proporcionas la ocasión de corresponder a las grandes finezas de
la dama más hermosa de España, al menos según mi indocto parecer en
asunto de mujeres. Mañana tengo que ir a su casa, porque has de saber
que la señora condesa es la que ha formado la \emph{Congregación de
lavado y cosido.}

---¿Y qué es eso?

---Una junta de señoras de la nobleza para lavar y coser la ropa de los
soldados en estas críticas circunstancias. Y no creas que es cosa de
engañifa, sino que ellas mismas con sus divinas manos lavan y cosen.
También pertenece la señora condesa a la junta de las \emph{Buenas
patricias}, en que hay damas de todas categorías, desde la duquesa a la
escofietera. Pero esto no hace al caso, sino que mañana tengo que ir a
esa casa, y les diré todo lo que tú me confíes. Aunque ahora me ocurre
que más fácil y expedito será cogerte por la mano y plantarte en
presencia de tan alta señora para que por ti mismo y con tus buenas
explicaderas, le des cuenta y razón de lo que desea saber.

---Padre, no sé si estará bien que yo vaya a esa casa---dije tratando de
disimular la alegría que el anuncio de la visita me causara.

---Yendo conmigo, no tengas cuidado. Además, has de saber que la señora
condesa es una persona ilustradísima, y que entiende de poesía y letras
humanas, de modo que al saber tus conocimientos en la lengua latina, es
seguro que te recibirá bien, y aun espero que te proporcione una buena
colocación.

---Eso será lo de menos, con tal que yo consiga prestar a tan buena
señora el servicio que desea. Y dígame, padre, ¿conoce Su Reverencia,
por ventura, a la que va a ser mujer de D. Diego?

---¡Que si la conozco! Como que soy su amigo, y su confidente, y desde
que entro en la casa viene a mí saltando y brincando, y todo el día
está: «padre Salmón por aquí, padre Salmón por acullá.»

---¿Y es Vuestra Paternidad su confesor?

---Eso no, que lo es mi compañero y amigo el padre Castillo, el cual va
también todas las tardes a la casa.

---Y ella estará tan enamorada de D. Diego, que beberá los vientos por
él.

---Me figuro que no le puede ver ni en pintura. Es opinión general en la
casa que la niña tiene puesto el pensamiento y el corazón en otra
persona; pero aunque se vuelven locos, no ha sido posible dar con ella.
El señor marqués y su hermana no piensan más que en averiguar quién
podrá ser ese desconocido zascandil que ha trastornado el seso a la más
discreta y bella muchacha que ha peinado azabaches y llorado perlas en
el mundo; y todo se vuelve averiguaciones y acechos, y observa por aquí
y husmea por allí. La condesa no se afana tanto y suele decir: «Eso se
le pasará;» pero yo conozco que no las tiene todas consigo. He aquí la
causa de que hayan querido apresurar el casamiento; pero aquí viene lo
de que Rumblarito es un perdido y un mala cabeza, y todo proyecto se
desbarata, y allá va el estira y afloja de las consultas: «Padre, ¿qué
haremos? ¿Padre, ¿qué no haremos? Padre, ¿qué no haremos?» A cuyo
apremiante cuestionar les contesto: «Calma, señoras mías, calma, que a
mucha prisa gran vagar. Que mi estrella querida doña Inés es el
\emph{super omnia} de la virtud, de la buena crianza, del recato, de la
modestia, no queda duda alguna, y capaz soy de decirlo en el púlpito si
me pinchan tanto así. Al mismo tiempo tampoco puede dudarse que algo le
hace cosquillas en su pensamiento, que algo como triste recuerdo o vago
deseo la trae a mal traer, porque ¿cómo se explica aquel no hablar en
dos días, aquel suspirar tan tierno, con la añadidura de mirar al suelo
en ademán cogitabundo, sin que razones ni halagos, ni aun mis chistes
escogidos, ni mis cuentos entresacados del \emph{Tesoro de los dichos
agudos} la hagan pestañear?» Y oyendo estas prudentes razones, la
marquesa se entristece, y me vuelve a consultar, y aquí viene lo de:
«Averígüelo el padre Salmón, que como tiene arte para el confesionario y
es el mayor sacador de pecados que hemos conocido, sabrá explorarla.»
Entonces el marqués añade: «Si por artes del demonio esa muchacha
durante el tiempo en que vivió lejos de nosotros tuvo el mal gusto de
enamorarse de algún cabrahígo de esas calles, ¿cómo es posible que en su
nueva posición no le haya olvidado?» Y yo lleno de celo por el reposo de
tan ilustre familia, llamo a la niña, me la llevo a un rinconcito de la
casa o a uno de los cenadores del jardín, y le tomo una mano, y se la
acaricio y le cuento dos cuentos, y le digo tres gracias, y le doy una
flor, y echando a correr con estas mis pesadas piernazas, le digo: «a
que no me cogéis,» y ella vuela y me coge del hábito a los tres pasos, y
con estos juegos preparo su ánimo para la confesión de amigo, no de
sacerdote, que de ella espero. Sentados otra vez, le digo: «Niñita mía,
flor de esta casa, retoñito temprano, fresa de abril, ¿queréis decirme
cuál es la causa de esa melancolía? Vamos a ver, acá para entre los dos,
pues esto no ha de salir de mí. Antes de que vuestro papa os recogiera,
¿amasteis a alguien?» Y al oír esto, los ojos se le llenan de lágrimas,
echa a correr, la sigo y al poco trecho la veo parada, mirando al suelo
y mordiendo la punta del pañuelo. Vuelvo a mis preguntas y nada saco en
limpio, lo cual me desespera. Entonces la marquesa y su hermano me
preguntan si creo conveniente que se rompa el trato hecho con la familia
de don Diego, a lo cual les contesto: «Calma, señores: indagaremos
primero si es cierto lo que del mozalbete se cuenta. Yo me encargo de
hacer diligencias, pues varias veces le he visto entrar en cierta casa
que frecuento, y conozco un joven que le acompaña a menudo. «Nada, hijo
mío, lo dicho dicho. Mañana vas allá y les cuentas todo lo que sabes et
\emph{quibusdam alliis}, con lo cual mi encargo queda hecho y el Rumblar
desenmascarado.

Gran sorpresa me causó la relación del venerable mercenario, y cuando me
separé de él prometiéndole ir en su compañía al siguiente día, quedeme
pensando en las extrañas cosas que había oído, y muy dudoso acerca de si
había obrado cuerdamente al comprometerme en tan arriesgada visita. Pero
debo explicar las causas de mis dudas, así como el estado de mi ánimo
por aquellos días, pues algo hay que mis lectores no deben ignorar,
aunque les sean indiferentes las desdichas de este su humilde servidor.
El palacio de mi señora la condesa (y debo advertir que a la sazón
vivían todos reunidos en el de la Cuesta de la Vega), era un asilo
infranqueable para mí. Desde mi vuelta de Andalucía ni por el
pensamiento me pasó el poner allí los pies, teniendo como tenía la
seguridad de una expulsión ignominiosa cual la de Córdoba. Entrar
valiéndome de la astucia habría sido, si posible, infructuoso, pues la
superchería o ficción de que me valiera, no podrían durar sino hasta que
la señora Amaranta me viese el rostro. Frecuentemente iba a pasear de
noche por los callejones que rodean el palacio, y allá en lo alto del
muro la claridad de una ventana atraía mis miradas. Falto de la imagen
de su persona, aquel cuadro de débil luz se me representaba como ella
misma. Una noche tanto miré y con tanto arrobo contemplaba aquella
ventana, que me entraron tentaciones de dar a conocer mi presencia al
habitante del palacio que con semejante luz se alumbraba, habitante que
según mi capricho era Inés y no otro alguno. Resolvime a ello, y tomando
una chinita la arrojé contra los cristales: al poco rato se dibujó en
ellos una sombra: pero esta y la luz desaparecieron pronto. Repetí el
disparo a la noche siguiente, y catad la sombra otra vez. Pero cuando
esperaba ver abierta la ventana, y oír una voz querida ceceando dulces y
temblorosas sílabas en el silencio de la noche, apareciose en el fondo
del callejón y como saliendo de las cocheras del palacio, un grupo de
hombres en actitud hostil contra mi persona. Me puse en cobro a toda
prisa, y no volví más.

Pasó Agosto, pasaron también Setiembre y Octubre, y aquellos noventa
días depositándose unos tras otros como noventa capas de tierra en el
hoyo de mi existencia, iban sepultando ilusiones, alegrías, sueños,
porvenir. De improviso la diferencia de jerarquía social había puesto
entre Inés y yo murallas inexpugnables, y para romper su jaula no
bastaban mis fuerzas, pues no era la nueva como aquella de los Requejos
hecha de frágiles cañas y alambres, sino de fuertísimos barrotes, más
que el diamante duros.

Entonces comprendí más claramente que antes que yo no era nada, ni valía
en el mundo más que un grano de anís, y esta consideración, irritándome
en sumo grado, me infundía el mayor desprecio hacia mí mismo. ¿Por qué
he nacido como he nacido? me preguntaba; y según es fácil comprender, no
podía acertar con la contestación.

Y después decía: El espesor y fortaleza de estas paredes es tal, que si
toda mi vida la empleara en hacerme más sabio que Séneca, más valiente
que el Cid y más rico que los Fúcares, aun así no podría romperlas. Sin
embargo, tal rumbo pueden llevar las cosas, que venga un día en que a
los Fúcares no se les pida su ejecutoria para emparentar con la nobleza.
Pero vamos a ver, ¿cómo me las compondré para llegar a ser rico? ¡Oh,
miserable de mí! ¡Rico quien nada tiene! Es evidente que no se pueden
ganar dos sin tener uno\ldots{} Pues estudiaré hasta que pierda el seso,
por ver si me hago sabio\ldots{} o entraré formalmente en el ejército,
por ver si de soldado raso llego a general en estos revueltos
tiempos\ldots{}

Y considerando esto, me golpeaba el cráneo, castigándole por su
estupidez y su tardanza en dar a luz felices pensamientos. Entretanto la
idea de la imposibilidad de mi dicha, de lo inútil de mis esfuerzos, y
de la inconmensurable pequeñez a que estaba reducido iba labrando en mi
alma con tanta tenacidad, que bien pronto aquel laborioso gusanito me
minó de parte a parte, me socavó, llenó de agujeros los fundamentos de
mi entusiasmo y fe poderosa, y\ldots{} ¡misericordia! todo yo caí al
suelo.

Las dificultades insuperables, la imposibilidad evidente de destruir con
el solo auxilio de mis dedos aquella montaña que Dios había puesto en mi
camino, me rendían de tal suerte, que me crucé de brazos, hallándome
incapaz para todo. Y desde abajo, desde la inmensa profundidad donde me
encontraba, decía, mirando el pedacito de cielo que difícilmente
percibía encima de mí:---¡Oh, cielo! ¡Cuán lejos te veo, y qué bajo
estoy después que creí tocarte con mi mano! Pero pues Dios ha dispuesto
mi caída, renuncio por ahora a estar cerca de ti, y me arrastraré por
estos oscuros fondajes, buscando un pedazo de pan que comer, sin más
objeto ni aspiración que dar a la bestia de mi despreciable persona el
forraje que diariamente necesita.

Así dije, si bien no recuerdo si empleé las mismas palabras.

¿Qué es el hombre sin ideal? Nada, absolutamente nada: cosa viva
entregada a las eventualidades de los seres extraños, y de que todo
depende menos de sí misma; existencia que, como el vegetal, no puede
escoger en la extensión de lo creado el lugar que más le gusta, y ha de
vivir donde la casualidad quiso que brotara, sin iniciativa, sin
movimiento, sin deseo ni temor de ir a alguna parte; ser ignorante de
todos los caminos que llevan a mejor paraje, y para quien son iguales
todos los días, y lo mismo el ayer que el mañana. El hombre sin ideal es
como el mendigo cojo que puesto en medio del camino implora un día y
otro la limosna del pasajero. Todos pasan, unos alegres, otros tristes,
estos despacio, aquellos velozmente, y él sin aspirar a seguirlos,
ocúpase tan sólo del cuarto que le niegan o del desprecio que le dan.
Todos van y vienen, cuál para arriba, cuál para abajo, y él se queda
siempre, pues ni tiene piernas para andar, ni tampoco deseos de ir más
lejos. Es, pues, la vida un camino por donde mucha y diversa gente
transita, y sobre cuyos arrecifes y descansos se encuentran también
muchos que no andan: estos, según mi entender, son los que no tienen
ideal alguno en la tierra, así como aquéllos son los que lo tienen, y
van tras él aprisa o con calma, aunque los más antes de llegar suelen
hacer alto en la posada de la muerte, donde por lo pronto se acaban los
viajes de este camino.

Pues bien; en aquellos tres meses yo lo había perdido todo y me
encontraba tullido y con muletas en mitad del camino. La meditación, la
razón, la evidencia que tenía delante, mil poderosos estímulos me
llevaron al siguiente resultado: renunciar completamente a Inés, si no
en mi corazón, en lo real de la vida. Era lo justo, lo lógico, lo
natural.

Y con esto queda dicho todo lo necesario para que se comprenda la
impresión vivísima que experimenté cuando el padre Salmón quiso tan
impensadamente y por tan raros caminos llevarme en presencia de la
condesa.

---Iré y sea lo que Dios quiera---dije para mí, ocupándome en arreglar
el vestido que en tan solemne ocasión debía llevar sobre mi
cuerpo.---¡Oh, infeliz de mí! Era el mes de Noviembre y no tenía más
traje decente que uno de verano, sutilísimo, a quien cuidaba más que si
fuera las telas de mi corazón, y me lo puse, con peligro de perecer
helado. Aquello a más de incómodo era ridículo; así es que al acostarme
pedí fervorosamente a Dios y a los santos que aclararan el día siguiente
haciéndolo como los de Mayo, templado y hermoso; pero los de arriba no
me oyeron o sin duda juzgaron más atendibles las razones de los
labradores que pedían agua y más agua.

Tomando algunas cosas que creía indispensables para la visita, salí a la
calle tiritando, encogido, hecho un ovillo y resguardando de los
canalones la limpieza de mi ropa, pero aun así no pude salvar sino una
pequeña parte de mi persona. Al fin aprovechando los claros y alguno que
otro descanso de las llovedoras nubes, después de hacer varias paradas y
estaciones en los portales, llegué al convento y juntándome con Salmón,
él muy festivo y yo más serio y pálido que si me llevaran a ajusticiar,
no dirigimos al palacio de Amaranta.

Cuando entramos, salionos al encuentro en el piso bajo el diplomático,
quien no aparentó reconocerme, y después de hablar aparte con el fraile
cosas que no entendí, nos mandó subir, diciendo que arriba estaba
Amaranta con el padre Castillo, revolviendo unos libros que le habían
traído. Subimos, y sin tardanza nos introdujo un paje. Al punto en que
Amaranta se fijó en mí, púsose pálida y ceñuda, demostrando la cólera
que por verme allí experimentaba. Pero como hábil cortesana, la disimuló
al instante y recibió a Salmón con bondad, ordenándome a mí que me
sentase junto a la gran copa de azófar que en mitad de la sala había, de
lo cual colijo que ella debió de comprender el gran frío que a causa del
rigor de la estación y de la diafanidad de mis veraniegas ropas me
mortificaba.

\hypertarget{vi}{%
\chapter{VI}\label{vi}}

---Este muchacho---dijo Salmón,---enterará a usía de aquello que deseaba
averiguar, pues todo lo sabe de la cruz a la fecha; y al mismo tiempo
tengo el honor de decir a usía que aquí tenemos un portento de
precocidad, un gran latino, señora, autor de cierto inédito poema, por
quien S. A. el Príncipe de la Paz le destinaba a la secretaría de la
interpretación de lenguas.

El padre Castillo volviose a mí y dijo con afabilidad:

---En efecto, ayer nos habló de Vd. el licenciado Lobo. ¿Y en qué aulas
ha estudiado usted? ¿Querrá leernos algo de ese famoso poema?

Yo le contesté que lo de mi ciencia latina era una equivocación, y que
el licenciado Lobo me daba aquella fama usurpándola a otro.

---¡Oh, no!\ldots{} que también, si mal no recuerdo, nos dijo que en Vd.
la modestia es tanta como el talento, y que siempre que se le habla de
estas cosas lo niega. Bien está la modestia en los jóvenes; mas no en
tanto grado que oscurezca el mérito verdadero.

Amaranta no dijo nada. El padre Castillo pasaba revista a varios libros,
en montón reunidos sobre la mesa, y los iba examinando uno por uno para
dar su parecer, que era, como a continuación verá el lector, muy
discreto. Hombre erudito, culto, ilustrado, de modales finos, de figura
agradable y pequeña, de ideas templadas y tolerantes que le hacían un
poco raro y hasta exótico en su patria y tiempo, Fr.~Francisco Juan
Nepomuceno de la Concepción, en los estrados conocido por el padre
Castillo, se diferenciaba de su cofrade, el padre Salmón, en muchísimas
cosas que al punto se comprenden.

---Estos son los libros y papeles que han salido en los tres últimos
meses---dijo Amaranta.---Buena remesa me han mandado hoy Doblado y
Pérez, mis dos libreros; pero no me pesa; pues entre tantas obras malas
y de circunstancias como aparecen en estos revueltos días alguna habrá
buena; y hasta las impertinentes y ridículas tienen su mérito para
ilustrar la historia de los actuales en los venideros tiempos.

---Así es---indicó el padre Castillo.---No hay obra por mala que sea,
que no contenga algo bueno, y hace bien vuestra grandeza, en comprarlas
todas.

---He leído un poco de este voluminoso papel---dijo Amaranta tomando un
folleto que parecía recién salido de la imprenta,---y me ha causado
mucha risa. El título es de los de legua y media. Dice así:
\emph{Manifiesto de los íntimos afectos de dolor, amor y ternura del
augusto combatido corazón de nuestro invicto monarca Fernando VII,
exhalados por triste desahogo en el seno de su estimado maestro y
confesor D. Juan Escóiquiz, quien por estrecho encargo de S. M. lo
comunica a la nación en un discurso}.

---Pues aquí veo otro---dijo Castillo hojeándolo,---que si no es del
mismo autor, lo parece. Se titula \emph{La inocencia perseguida o las
desgracias de Fernando VII: poesía}. Verdad que está en verso, y ahora
es moda tratar en metro las más serias cuestiones, aun aquellas más
extrañas al arte de la poesía, como por ejemplo este papel que ahora me
viene a las manos y se llama \emph{Explicación del capítulo IX del
Apocalipsi, aplicado según su sentido literal al extraordinario
acontecimiento de la pérfida irrupción de España: oda por un capellán}.

---Y ha de saber Vuestra Reverencia que también nuestro prisionero
monarca da en la flor de hablar en verso---dijo Amaranta con
sorna,---pues aquí tengo la \emph{Epístola férvida que nuestro amado
soberano el Sr.~D. Fernando VII dirige a sus queridos vasallos desde su
prisión: pieza patética, tierna y de locución majestuosa}.

---Pues ¿y qué me dice la señora condesa de este otro librito que ahora
me cae en las manos, y lleva por nombre \emph{La Corte de las tres
nobles artes, ideada para el inocente Fernando VII: anacreónticas}? Y la
primera de estas anacreónticas se encabeza así: \emph{Reglas que
contribuyen a que un pueblo sea sano y hermoso}. Por mi hábito de la
Merced que no entiendo esto del pueblo \emph{sano y hermoso}, que se ha
de conseguir por la corte de las tres nobles artes, y ha de exponerse en
anacreónticas. Con permiso de vuecencia me lo llevaré al convento para
leerlo esta noche.

---Lleve también Su Paternidad este papel suelto que dice:
\emph{Lágrimas de un sacerdote en dos octavas acrósticas}.

---Esto de los acrósticos y pentacrósticos, es juego del ingenio,
indigno de verdaderos poetas---dijo Castillo,---y más aún de un
sacerdote, cuyo entendimiento parecería mejor consagrado a graves
empleos. Pero démelo acá usía, que me lo llevaré, juntamente con este
sermón que se titula \emph{Bonaparciana, u oración que a semejanza de
las de Cicerón, escribió contra Bonaparte un capellán celoso de su
patria}. Y en verdad que no anduvo modesto el tal capellancito
comparándose con Cicerón; pero en fin, eso me anuncia qué tal será la
dichosa Bonaparciana.

---Por Dios, señora condesa---dijo a esta sazón el padre José Anastasio
de la Madre de Dios.---Ruego a vuecencia que me deje llevar al convento
para leerlo esta noche, este otro graciosísimo libro que se titula:
\emph{Las Pampiroladas, letrillas en que un compadre manifiesta a su
comadre que en las circunstancias actuales no debe temer a la fantasma
que aterraba a todo el mundo}. ¡Qué obra más salada! Si no queda cosa
que no se les ocurre\ldots{}

---También puede llevarse, pues viene muy bien al ingenio y buen humor
de Su Paternidad---agregó Castillo,---este otro que aquí veo, y es
\emph{Deprecación de Lucifer a su Criador contra el tirano Napoleón y
sus secuaces, asusta el ver entrar tantos malvados franceses en el
infierno}. ¡Hola, hola! también está en octavas. Serán mejores que las
de Juan Rufo, Ercilla y Ojeda.

---¡Oh! Este sí que es bueno. ¡Válgame nuestra santa Patrona!---exclamó
Salmón.---Oíganme: \emph{Seguidillas para cantar las muy leales y
arrogantes mozas del Barquillo, Maravillas y Avapiés, el día de la
proclamación de nuestro muy amado Rey.} ¿Me las llevo, señora condesa?

---Sí, padre; ya que está por seguidillas, aquí veo otras que le
parecerán muy buenas. \emph{Seguidillas que cantó el famoso Diego López
de la Membrilla, jefe de la Mancha, después que consiguió las gloriosas
victorias contra los franceses.}

---El pueblo español---declaró Castillo,---es de todos los que llenan la
tierra el más inclinado a hacer chacota y burla de los asuntos serios.
Ni el peligro le arredra, ni los padecimientos le quitan su buen humor;
así vemos que rodeados de guerras, muertes, miseria y exterminio, se
entretiene en componer cantares, creyendo no ofender menos a sus
enemigos con las punzantes sátiras que con las cortadoras espadas. ¿Y
qué me dicen usías de este \emph{Asalto terrible que dieron los ratones
a la galleta de los franceses, poema en dos cantos? ¿Qué de este Elogio
del Sr.~D. Napoleón, por un artífice de telescopios? ¿Qué de esta Gaceta
del infierno, o sea Noticia de los nuevos amores de la Pepa Tudó con
Napoleón, y celos de Josefina?}

---Esas son groserías de vulgares e indecentes escritores---afirmó con
enfado Amaranta,---pues todo el mundo sabe que ni la Tudó ha tenido
amores con Bonaparte, ni este ha hecho nada que menoscabe su fama de
hombre de buenas costumbres.

---Cierto es---dijo Castillo,---pero si usía me lo permite, le haré una
observación, y es que el pueblo no entiende de esas metafísicas, y al
verse engañado y oprimido por un tirano y bárbaro intruso, no debemos
extrañar que le ridiculice y aun le injurie. El pueblo es ignorante, y
en vano se le exige una decencia y compostura que no puede tener, razón
por la cual yo me inclino a perdonarle estas chocarrerías si conserva la
dignidad de su alma, donde el grande sentimiento de la patria como que
disimula y oscurece los rencorcillos pequeños y vituperables.

---No me defienda Vd. tales chocarrerías, padre---repuso
Amaranta.---¿Tiene perdón de Dios este otro impreso que ahora leo? Oiga
Vd. el título: \emph{Lo que pueden cuatro borrachos, o sea despique al
vil dictado con que se han querido oscurecer los honrados procedimientos
de un pueblo fiel a su religión, rey y patria}.

---La obra---dijo riendo el fraile,---tiene traza de no ser un segundo
\emph{D. Quijote} ni mucho menos; pero en su mismo título hallará
vuecencia la explicación del llamar \emph{borrachos} a los Bonapartes,
dictado que tanto repugna a mi señora condesa. Cierto que los Bonapartes
no son borrachos, y harto sabemos que el pobre rey José ni por pienso lo
bebía; pero el pueblo no lo entiende así, del mismo modo que jamás dejó
de llamarle \emph{tuerto}, aunque harto bien pudo reparar la hermosura
de sus dos ojos. El pueblo le llamó borracho y tuerto sin motivo, es
cierto; pero ¿tienen razón los franceses en llamar \emph{insurgentes,
bandidos y ladrones de caminos} a los héroes que en los campos de
batalla defienden generosamente la independencia patria?

---Convengo en ello---contestó Amaranta;---pero la cosa más justa si se
hace con malas formas, parece como que se deslustra y encanalla. Vea Vd.
Para hacer una pintura de las calamidades ocasionadas por la guerra, no
era preciso que el autor de este papel lo titulara \emph{Inventario de
los robos hechos por los franceses en los países donde han invadido sus
ejércitos}.

---Señora, convengo que al autor se le ha ido un tanto la mano en la
forma---dijo Castillo;---pero por lo poco que de este libro he leído, me
parece que dice verdades como el puño.

---¡Y tan como el puño!---exclamó Salmón alzando los ojos de un libelo
cuyas páginas recorría a la ligera.---Pues lo que es este que al azar ha
caído en mis manos, tiene unas explicaderas\ldots{}

---¿Cuál?

---Es de lo más gracioso y bien parlado que imaginarse puede. Su anónimo
autor lo titula \emph{Carta primera de un vecino de Madrid a un su
amigo, en que le cuenta lo ocurrido después de la prisión del execrable
Godoy, hasta la vergonzosa fuga del tío Copas}. La agudeza de los
dichos, la oportunidad de los chistes, apodos y chanzonetas es tal, que
harían reír a la misma seriedad.

---¡Bonito modo de escribir la historia! Y ese palurdo vecino de Madrid,
que sin duda será algún sacristán rapavelas o bodegonero del Rastro,
¿qué entiende de execrables Godoyes ni otras zarandajas?

---¿Pues no ha de entender, señora?---dijo el padre Castillo.---A veces
en personas rudas y zafias se ve mejor sentido y criterio de las cosas
que en las ilustradas y quizás por su misma ilustración desvanecidas. Lo
que les falta es el decoro en la forma. Oiga mi señora condesa una
observación que quiero hacerle. Entre esta multitud de papeles, que los
libreros de Madrid le envían para que coleccione todo lo publicado, hay
tal balumba de despropósitos y estolideces, que sería más necio y simple
que sus autores el que dejara de reconocerlo así. Pero en medio de tanta
faramalla, encuentro algunos productos del ingenio que suspenden,
cautivan y enamoran, por ser fruto espontáneo de la mente popular, como
lo son las heroicas acciones que desde el principio de la guerra estamos
presenciando. Vea vuecencia: aquí hay una \emph{Convocatoria que a todos
los pastores de España dirige un mayoral de la sierra de Soria para la
formación de compañías de honderos}. Este es un hombre ignorante, cuya
actividad e interés por la patria no puede menos de elogiarse. También
merece encomios lo que ha escrito esta doña María Piquer y Pravia, con
el título de \emph{¿Qué es héroe? Exhortación a los jóvenes españoles},
pues todo lo que tienda a encender los alientos de la juventud en las
actuales circunstancias, es digno de aplauso. No le negaré tampoco los
míos a estos \emph{Cargos que hace el tribunal de la razón de España al
Emperador de los franceses}, porque los tales cargos están hechos con
mesura; ni tampoco a este \emph{Engaño de Napoleón descubierto y
castigado, obra en que se manifiesta con claridad la infidelidad del
Emperador en sus convenios con España}, porque todo cuanto se diga
acerca de la manera desleal y traidora con que nos declararon la guerra,
me sabe siempre a poco. No seré tan benévolo con esta \emph{Carta del
licenciado Siempre y Quando al Doctor Mayo de} 1808, porque me repugnan
las formas chocarreras en formales asuntos, ni daré dos higos por esta
\emph{Alegoría poética que descubre las iniquidades del más perjudicial
y maligno hipócrita del mundo, Bonaparte}, porque ya dije que este afán
de tratar en malos versos lo que está pidiendo a gritos clara y valiente
prosa, me indigna y pone fuera de mí.

\hypertarget{vii}{%
\chapter{VII}\label{vii}}

---Gracias a Dios---dijo entonces Amaranta,---que encuentro entre esta
garrulería una obra de reconocida utilidad durante los tiempos de
guerra. Vea Su Reverencia: \emph{Arte universal de la guerra del
príncipe Raimundo de Montecuculi}.

---En efecto, señora: yo daría un par de abrazos y otros tantos
apretones de manos a Quiroga y Burguillos, que son impresores y editores
de esta gran obra. Y aquí veo otra a cuyo autor le pondría yo en los
cuernos de la luna, pues no conozco hoy por hoy tarea más meritoria que
escribir un \emph{Prontuario en que se hallan reunidas las obligaciones
del soldado, cabo y sargento para la pronta metódica instrucción de las
compañías.} Vea mi señora condesa, cómo también sacamos pepitas de oro
puro del escorial de este montón que tenemos delante. Aquí veo la
\emph{Higiene militar o arte de conservar la salud del soldado en
guarniciones, marchas, campamentos, hospitales, etc}. Queden a un lado,
para que no se confundan con lo demás; y en su compañía vaya \emph{El
buen soldado de Dios y del Rey, libro donde se asocian las máximas
militares con las cristianas}. Esto me parece muy del caso, pues será
mejor soldado aquel que lleve en su corazón la fe, única fuente de toda
heroica acción y de la humildad y obediencia, que mantienen la
disciplina, remedo mundano del divino orden puesto por Dios a la
autoridad religiosa.

---Pues hagamos aquí un apartado de los buenos libros---dijo la condesa
graciosamente, reuniendo los que el fraile le indicaba.

---Pero tate, señora mía---dijo este,---que me parece que en ese
departamento de las cosas buenas se ha colado \emph{El laurel de
Andalucía y sepulcro de Dupont}, que, aunque muy patriótica, es de las
más necias y enfadosas comedias que se han impreso en estos tiempos.
Vaya fuera, y lléveselo Salmón si quiere leerlo, y en su lugar póngase
esta \emph{Colección de proclamas, bandos, diversos estados del ejército
y relaciones de batallas}, que por ser un conjunto de documentos
fehacientes, será en día no lejano de grande interés para la historia,
que en tales tesoros se alimenta y bebe la verdad, sin la cual no puede
vivir. ¿Pero qué libro es ése que con tanta atención vuecencia lee?

---Leo---repuso la condesa,---las \emph{Poesías patrióticas de D. Manuel
Josef Quintana}, que ahora salen por segunda vez a luz. Este tomo
contiene la \emph{Expedición de la Vacuna, las odas a Juan de Padilla, a
España libre, al panteón del Escorial y a la Invención de la imprenta}.

---¡Oh!---exclamó el padre Castillo.---Bien lo decía yo: no pepitas de
oro, sino perlas orientales habían de aparecer entre esta balumba.
Póngame vuecencia a ese poeta sobre las niñas de mis ojos, pues no me
canso nunca de leerlo, y es tan grande el encanto que en mí producen su
fogosa entonación, su grave estilo, su arrebatado estro, su numerosa
cadencia, la gallardía de las imágenes, la verdad de los pensamientos,
la elegancia de los símiles, la escogida casta de todas las voces y
frases, que me olvido del apasionamiento y saña con que ataca institutos
y personas que yo a causa de mi estado no puedo menos de reverenciar.
Pero tal es el privilegio del arte cuando da en buenas manos; y es que
enamora con la forma aun a aquellos ánimos a quienes no puede conquistar
con las ideas.

---Quítenmelo de delante---dijo Salmón,---y no pongan a ese autor ni a
cien leguas del de esta composición que ahora tengo en la mano:
\emph{Godoy, sátira por D. José Mor de Fuentes}.

---Pues si Su Paternidad es tan entusiasta de Mor de Fuentes, nosotros
se lo regalamos, para que lo disfrute por los siglos de los siglos. ¿No
es verdad, señora condesa? ¿A ver qué otro volumen es este, que parece
recién publicado? \emph{Poesías líricas o rimas juveniles por don Juan
Bautista Arriaza}. Este no debe ser despreciado, pero tampoco agasajado.
El aprecio que conquista con su gracia y primorosa frivolidad, lo pierde
por maldiciente, sin que tenga como Juvenal el mérito de reprender los
vicios y malas costumbres. Sus mejores obras son las que podríamos
llamar \emph{Vejámenes}, dirigidas contra cómicos y poetas; y estas
\emph{Rimas juveniles} son finas, pulcras, bonitas, pasajeras; pero
carecen de aquella sal de la inspiración, sin cuyo ingrediente no hay
manjar poético que se pueda traspalear. ¿Qué hacemos, señora condesa?
¿Se lo damos a Salmón o se queda en el departamento escogido?

---Quédese aquí---dijo Amaranta,---aunque no sea sino porque me ha
dedicado casi todos sus versos llamándome Clori, Belisa, Dorila, Mirta,
Dafne, Febea y Floridiana. Y para que el reverendo Salmón no se enfade,
le daremos el \emph{Napoleón rabiando, casi-comedia}; el \emph{Bonaparte
sin máscara}, y la \emph{Descomunal batalla de los invencibles gabachos
contra los ratones del Retiro}, que aquí están pidiendo que Vuestra
Reverencia les de su dictamen.

---Pues vengan---dijo Salmón,---y no creo que vuestra grandeza me niegue
este saladísimo papel, cuyo solo título hace desternillar de risa, y es:
\emph{El juego de Fernando VII con Napoleón y Murat al tresillo, libro
en el que baxo las voces propias del tresillo se da una idea de lo
acaecido con nuestro augusto soberano, del orgullo de Napoleón, y
concluye con las exclamaciones más tiernas de nuestro oprimido Monarca}.

---Esto de decir en términos de tresillo lo que se puede expresar en
castellano seco, me enamora---indicó Castillo.

---Precisamente en lo intrincado está el mérito de la
invención---observó el otro fraile.---La prosa llana se cae de las
manos, y así no comprendo cómo Vuestra Paternidad está ahora tan
embebecido en la lectura de ese folleto, \emph{Gobierno pronto y
reformas necesarias}.

---Más que por lo que dice, me interesa por lo que todos los papeles de
esta clase indican de alteraciones y disputas para lo por venir.

---Los españoles---dijo la condesa,---no se cuidan ahora de lo porvenir.

---Permítame usía que la diga que está muy equivocada---repuso
Castillo.---Observando atentamente todos los impresos que salen a luz (y
los papeles impresos son quien más que otra cosa alguna da a conocer lo
que piensa y anhela un pueblo cualquiera); observando, digo, esto que
aquí tenemos, se ve que los españoles, bajo la aparente conformidad que
nos da la guerra, estamos muy divididos, y eso se conocerá cuando con
las paces venga el deseo de establecer las nuevas leyes que nos han de
regir. Aquí tengo unas \emph{Reflexiones de un español, y modo de
organizar un gobierno que concluya la grande obra de la eterna libertad
y prosperidad de la nación}. No parece mal escrito, y apunta con timidez
la idea que creo desarrolla atrevidamente este cuaderno que se intitula
\emph{Política popular acomodada a las circunstancias del día: propone
la Constitución que la España necesita para cortar de raíz el
despotismo}. Por el mismo estilo y con igual tendencia está hecho este
otro que dice \emph{Reflexiones de un viejo activo a un amigo suyo sobre
el modo de establecer una Constitución}.

---Y por lo que veo---dijo Amaranta leyendo la portada de otro
libro,---este trata del mismo asunto: \emph{Manifiesto del español,
ciudadano y soldado, donde se da conocimiento de nuestros anteriores
padeceres y esperanzas en nosotros mismos, respecto al mundo
individual}.

---Por San Buenaventura y los cuatro doctores, que no sé lo que ha
querido decir ese buen hombre con lo del \emph{mundo individual}: pero
lo apartaremos para leerlo después.

---¿Y cree Vuestra Paternidad que hay divergencia de pareceres entre los
diversos autores que tratan de política y de Constitución?---preguntó
Amaranta.

---¡Oh!---exclamó Castillo,---por aquí aparece la punta de un impreso,
en quien desde luego conozco la opinión contraria. Sí, señora condesa:
no hay más que leer este título, \emph{Higiene del cuerpo político de
España, o medicina preservativa de los males con que la quiere contagiar
la Francia}, para comprender que éste es amigo del despotismo. Pues, ¿y
dónde me deja usía estas \emph{Conclusiones político-morales que ofrece
a público certamen contra los herejes de estos tiempos un fraile
gilito?} No me gusta que los regulares se ocupen de estos asuntos, y
desearía que concretándose a su ministerio de paz, aguardaran tranquilos
lo que los tiempos futuros traigan de calamitoso para nuestro instituto.
Pero no es posible contener esta gritería que por todos lados sale en
defensa de opuestos intereses, y venga lo que viniere, que si Dios no lo
remedia, será gordo y sonado. Entretanto, póngame usía a un ladito estos
libros que tratan de la Constitución y el despotismo, pues pienso
examinarlos espaciosamente. ¿Pero qué veo? ¿Ha puesto vuecencia en el
montón escogido esos cuatro librillos de novelas simples? Parece mentira
que en esta época empleen nuestros libreros su tiempo y dinero en
traducir del francés tales majaderías\ldots{} ¿A ver? \emph{La marquesa
de Brainville, la Etelvina, los Sibaritas, el Hipólito}. Vaya toda esta
romancil caterva a deleitar al padre Salmón, y si tarda en devolverla,
mejor, que así podrá vuestra grandeza entretenerse en mejores lecturas.

---En esto de novelas andamos tan descaminados---dijo Amaranta,---que
después de haber producido España la matriz de todas las novelas del
mundo y el más entretenido libro que ha escrito humana pluma, ahora no
acierta a componer una que sea mayor del tamaño de un cañamón, y traduce
esas lloronas historias francesas, donde todo se vuelve amores entre dos
que se quieren mucho durante todo el libro, para luego salir con la
patochada de que son hermanos.

---Pues para mí---dijo Salmón,---no hay más regocijada lectura que esa;
y vengan todos para acá.

---Abulta bastante, señora condesa---indicó Castillo,---el apartado de
los que defienden la Constitución. Hágame vuestra merced otro con los
apóstoles del despotismo que hasta ahora parecen los menos. Pero no; por
aquí sale un libelo titulado \emph{Gritos de un español en su rincón},
que al instante puedo colocar entre los del despotismo.

---Y aquí hay otro---dijo Amaranta,---que si no me equivoco, también es
del mismo estambre. Titúlase \emph{Carta de un filósofo lugareño que
sabe en qué vendrán a parar estas misas}.

---¡Magnífico! Desde que oí eso del \emph{filósofo lugareño} lo diputé
por enemigo de los constitucionales. Vaya al segundo montón; y los
leeremos a unos y a otros para saber, como dice el encabezamiento, en
qué \emph{vendrán a parar estas misas}. Esta lucha, señora mía, o yo me
engaño mucho, o ahora es un juego de chicos comparada con lo que ha de
venir. Cuando se acabe la guerra, aparecerá tan formidable y espantosa,
que no me parece podrá apaciguarla ni aun el suave transcurso de todos
los años de este siglo en cuyo principio vivimos. Yo, que observo lo que
pasa, veo que esa controversia está en las entrañas de la sociedad
española, y que no se aplacará fácilmente, porque los males hondos
quieren hondísimos remedios, y no sé yo si tendremos quien sepa aplicar
estos con aquel tacto y prudencia que exige un enfermo por diferentes
partes atacado de complicadas dolencias. Los españoles son hasta ahora
valientes y honrados; pero muy fogosos en sus pasiones, y si se desatan
en rencorosos sentimientos unos contra otros, no sé cómo se van a
entender. Mas quédese esto al cuidado de otra generación, que la mía se
va por la posta al otro mundo, con más prisa de lo que yo deseo. Y
entretanto, guárdeme usía esos dos montones de libros, que todos quiero
leerlos. Aquí el departamento de la Constitución, a este otro lado el
del despotismo\ldots{} pero ¡pecador de mí! A vuecencia se le ha ido la
mano, dejando que se colara en estas regiones un papelillo, que desde su
principio fue destinado al paladar de mi reverendo amigo. Afuera ese
desvergonzado intruso.

---¡Ah!---exclamó Amaranta riendo.---Es un \emph{Retrato poético del que
vende santi barati y el sartenero victoreando al primer pepino que
plantó un corso en tierra de España, y no ha prendido}.

---¡Venga acá!---exclamó con gran alegría Salmón.---¡Y cómo se escapaba
esa joya! Al convento me lo llevo junto con este otro, que aunque no
trata de la guerra ni de política, parece libro de recreación científica
y de honestísimo divertimiento. Es la \emph{Pirotécnica entretenida,
curiosa y agradable, que contiene el método para que cada uno pueda
formarse en su casa los cohetes, carretillas y bombas, etc., con tres
láminas demostrativas de todas las operaciones del sublime arte de
polvorista}.

---Y ahora, señora condesa de mi alma---dijo el padre Castillo
levantándose,---ya que he molestado bastante a usía, y hecho el
escrutinio que vuestra grandeza deseaba, me retiro, pues esta tarde
celebra solemne rosario la hermandad del Socorro de Nuestra Señora del
Traspaso, y me toca predicar.

---Yo pertenezco a la del Rescate---indicó Amaranta,---y creo que es la
semana que entra cuando hacemos nuestra función de desagravios. Y
Vuestra Paternidad, padre Salmón, ¿no predica en estas fiestas?

---Señora, la real congregación y esclavitud de Nuestra Señora de la
Soledad, me ha encargado dos pláticas para la semana que entra. Veremos
qué tal salgo de ellas.

El padre Castillo, que sin duda tenía prisa, se fue, y allí quedamos
Salmón y yo. Desde que hubo salido su compañero, tomó aquel la palabra,
y dijo:

---Pues, como tuve el honor de indicar a usía, este muchacho sabe todo
lo concerniente a don Diego, a sus artimañas, trapicheos y correrías, y
él satisfará a vuecencia mejor que cuanto yo, \emph{relata referendo},
pudiera decirle. Pero ¿será cierto, señora mía, lo que al entrar me ha
dicho el señor marqués D. Felipe?

---¿Qué?

---Que usía ha tenido anoche la felicísima suerte de hacer confesar a
esa linda niña todo lo que de ella queríamos saber.

---Así es---dijo Amaranta.---Todo me lo ha confesado.

---La paz de Dios sea en esta ilustre casa. ¿Dónde está ese blanco
lirio, que la quiero felicitar por el buen acuerdo que ha tenido?

---Esta tarde no se la puede ver, padre. Ya que su merced ha tenido la
buena ocurrencia de traerme este joven, a quien supone al tanto de lo
que quiero saber, tenga la bondad de dejarme a solas con él, para que la
presencia de persona tan grave y respetabilísima como Vuestra
Reverencia, no le impida decirme todo lo que sabe, aunque sea lo más
secreto.

---Con mil amores obedeceré a usía---dijo el padre Salmón;---y con esto
se retiró dejándome solo con aquella estrella de la hermosura, con
aquella deslumbradora cortesana, a quien nunca me había acercado sin
sacar de su trato el fruto de una gran pesadumbre.

\hypertarget{viii}{%
\chapter{VIII}\label{viii}}

---No ha sido una simpleza de este buen religioso lo que te ha traído
aquí---me dijo severamente;---esto ha sido obra de tu astucia y
malignidad.

---Señora---le respondí,---por mi madre juro a usía que no pensaba
volver a esta casa, cuando el padre Salmón se empeñó en traerme, con el
objeto que él mismo ha manifestado.

---¿Y qué sabes tú de D. Diego?

---Yo no sé más sino aquello que no ignora nadie que le trata.

---D. Diego es jugador, franc-masón, libertino; ¿no es cierto?

---Usía lo ha dicho; y si lo confirmo, no es porque me guste ni esté en
mi condición el delatar a nadie, sino porque eso de D. Diego todo el
mundo lo sabe.

---Bien; ¿y tú querrías llevarme a mí o a otra persona de esta casa a
cualquiera de los abominables sitios que el conde frecuenta por las
noches, para sorprenderle allí, de modo que no pueda negarnos su falta?

---Eso, señora, no lo haré, aunque usía, a quien tanto respeto, me lo
mande.

---¿Por qué?

---Porque es una fea y villana acción. Don Diego es mi amigo, y la
traición y doblez con los amigos me repugna.

---Bueno---dijo Amaranta con menos severidad.---Pero me parece que tú
eres tan necio como él, y que le llevas a la perdición, incitándole y
adulando sus vicios.

---Al contrario, señora, a menudo le afeo su conducta, diciéndole que
tal proceder es indigno de caballeros, y que al paso que deshonra su
casa, deshonra también a aquella con quien va a emparentarse.

---Eso está muy bien dicho---exclamó con pesadumbre.---Lo que hace
Rumblar no tiene perdón de Dios. ¿Y quién le acompaña en su libertinaje?

---El señor de Mañara y D. Luis de Santorcaz.

---¡También ese!---dijo con sobresalto y súbita transformación en su
bello rostro.---¿Qué hombre es ése? ¿Le conoces tú? ¿Dónde vive? ¿En qué
se ocupa?

---Si he de decir verdad, aún ignoro qué clase de hombre es. Tampoco sé
dónde vive; pero he oído que es espía de los franceses, y que estos le
dan un sueldo para que les escriba todo lo que pasa. Esto me han dicho;
pero no lo aseguro.

Entonces Amaranta acercó su silla a la mía, mirome como quien se dispone
a entablar relaciones de confianza, y me habló así con voz dulce:

---Gabriel, está de Dios que me prestes de vez en cuando servicios de
esos que no se encomiendan sino a la despierta observancia y a la
discreta malicia. ¿Querrás averiguar si D. Diego anda también en
conspiraciones y malos pasos con ese que has llamado espía de los
franceses?

---No sé si podré hacerlo, señora. Tendría que hacerme dueño de su
confianza para abusar de ella. Por otro conducto podrá averiguarlo su
señoría.

---Estás orgulloso; pero ven acá, chicuelo: ¿quién eres tú? ¿A quién
sirves ahora?

---No sirvo a nadie, ni quiero servir. Por ahora soy soldado, si soldado
es ser alguna cosa. Vivo de la paga que da el Ayuntamiento de Madrid a
las tropas que ha levantado. Pero no tengo afición a las armas, y si las
tomo hoy es por puro patriotismo y sólo mientras dure la guerra. Después
Dios dispondrá de mí, aunque, como no tengo riquezas, ni padres, ni
parientes, ni papeles de nobleza, ni protección alguna, espero que no
saldré de esta humilde esfera en que he nacido y vivo.

---¿Quieres que te proteja yo? ¿Necesitas algo?---me preguntó con
bondad.---Te buscaré un buen acomodo, te socorreré, si por acaso no
estás muy desahogado.

---Aunque el recibir limosnas no deshonra a nadie, antes me asparían que
tomarlas de vuecencia.

---¿Por qué? ¿Pero qué pretendes tú? Yo sé que tú picas muy alto, y no
te andas por las ramas. Vamos, Gabriel, si me abres tu corazón, si me
confías francamente todo lo que sientes, te prometo ser benévola
contigo. ¿Crees que no estoy al tanto de tus atrevimientos? Y sí no,
dime: ¿a qué paseas de noche por ese callejón cercano? ¿A qué arrojas
piedrecitas a las ventanas?

---¿Usía me vio?---pregunté muy confuso.

---Sí, y aunque me causó ira, reconozco que nadie es dueño de borrar de
un golpe lo pasado, mucho más cuando uno no es autor de la situación en
que ahora o después se encuentra, sino que es Dios quien a ella le
conduce. Tú tienes aspiraciones ridículas y absurdas, y ahora yo,
renunciando a medios violentos, hablándote con templanza y sensatez, voy
a quitártelas de la cabeza.

---Hable vuecencia; pero debo advertirle que no tengo ya pretensiones
ridículas, pues todo aquello que vuecencia recordará de mi afán de ser
generalísimo pasó, y\ldots{}

---No me refiero a eso, y bien sabes a qué aludo, tunantuelo. No puedo
ocultarte el disgusto que tuve cuando en Córdoba me dijiste con mucha
ingenuidad: «Señora, Inés y yo éramos novios.» Tal despropósito,
tratándose de mi prima, me indignó al principio; pero después me hizo
reír. ¡Ay! cuánto he reído con esto. Por supuesto, no creas que ella se
acuerda de ti. ¡Eres tan inferior a ella! Bien sabe Inés que si en otro
tiempo y lugar la aparente igualdad de vuestra condición permitía que os
estimarais, hoy el solo pensar en tal cosa es un crimen. ¡Pues si vieras
cómo se ríe de ti, y cuenta tus simplezas!\ldots{} Eso sí, dice que te
está agradecida porque dice que la salvaste de no sé qué peligro; pero
nada más. Mi primita ha sacado tal dignidad y estimación de su linaje,
que no digo yo con condes, con emperadores se casaría, y aún se juzgara
rebajada.

---¡Bendito sea Dios, y cómo se mudan las personas!---dije yo,
comprendiendo no ser cierto lo que oía.

---Pero si esto te digo---continuó Amaranta,---también añado que me
intereso por ti y quiero recompensar los servicios que prestaste a Inés
cuando estaba en la miseria; de modo que te daré lo necesario para que
hagas fortuna con tu trabajo; mas con la condición de que has de
marcharte de Madrid y de España mañana mismo, para no volver nunca.

Oí con mucha calma estas razones que la condesa dijo, queriendo
aparentar una tranquilidad de espíritu que no tenía, y le contesté:

---¡Ay, señora, y qué mal me ha comprendido usía! Hábleme ahora
vuecencia sin ninguna clase de artificio, pues yo con el corazón en la
mano le digo que conozco muy bien quién soy y todo lo que puedo esperar.
En mi corta vida he aprendido a conocer un poco las cosas del mundo, y
sé que aspirar a lo que por mi humildad, mi ignorancia y mi pobreza está
tan lejos de mí como el cielo de la tierra, sería una estupidez. No
ocultaré a usía nada de lo que me ha pasado. Cuando Inés, quiero decir,
la señorita Inés, estaba en casa del cura de Aranjuez, nosotros nos
tuteábamos, hablando de nuestro porvenir como si nunca hubiéramos de
separarnos. Después en casa de D. Mauro Requejo, parecía como que
nuestras desgracias nos hacían querernos más. Teníamos mil bromas, y yo
le decía: «Inesilla, cuando seas condesa, ¿me querrás como ahora?» Y
ella me contestaba que sí, y yo me lo creía\ldots{} Después todo ha
cambiado. Cuando fui a la guerra, yo no pensaba sino en ser un hombre de
provecho para hacerla mi mujer; mas al mirar de cerca la esfera a donde
ella había subido, al verme a mí mismo sin poder subir un solo peldaño
en la escala de la sociedad, me entró una tristeza tal, que pensé
morirme. Pero al fin se ha ido abriendo paso mi razón por entre este
laberinto de atrevidas locuras, y he dicho para mí: «Gabriel, eres un
loco en pensar que el mundo se va a volver del revés para darte gusto.
Dios lo ha hecho así, y cuando su obra ha salido con tantas
desigualdades, él se sabrá por qué. Renuncia a tus vanos sueños; que
esto, y ser generalísimo de un tirón, como antes pensabas, es todo uno.»
Al fin, señora condesa, he llegado a costa de grandes tristezas a
adquirir una resignación profunda, con cuyo auxilio ya estoy curado de
mis atrevimientos. He renunciado a lo imposible. Si así no lo hubiera
hecho, sería real y efectivo lo que cuentan las malas novelas de que se
reía hace poco el padre Castillo, y en las cuales se ve a una
archiduquesa que se casa con un paje, y a un porquerizo enamorado de una
emperatriz. No, señora: vengamos a la realidad triste; pero que es lo
único que no engaña. Ya no tengo las aspiraciones que usía me supone, y
no es necesario que vuececia compre con dinero mi resignación ni mi
alejamiento de esta casa, de Madrid y de España.

Amaranta mirábame de hito en hito durante aquel mi largo discurso, y
después habló así:

---Gabriel, o eres un hipócrita, o en verdad que me vas pareciendo un
joven no sólo discreto, sino de honradas ideas. Ya veo que comprendes el
sentido natural y templado de las cosas, y que sabes enfrenar la
impetuosidad y petulancia propias de la juventud.

---Señora, lo que he dicho a usía es la pura verdad; así me conceda Dios
una buena muerte en mi última hora.

---Pues ya que me hablas con tanta franqueza, no quiero ser menos
contigo. ¿Serás tú hombre a quien se pueda confiar un pensamiento
delicado, un pensamiento de esos que la vulgaridad no comprende, ni
estima en su justo valor?

---Creo que podrá vuecencia confiarme lo que quiera.

---¿Lo comprenderás tú? Vamos a ver. Dices que has renunciado a que te
ame mi prima, reconociendo la inmensa inferioridad de tu posición.

---Sí, señora, así es.

---Muy bien; pero es el caso\ldots{} no sé cómo decírtelo. Al indicarte
que te daría riquezas, quise expresar que esperaba de ti un grande, un
extraordinario favor.

---Si está en mí el prestarlo, no necesito que se me de nada. ¿Quiere
usía que me marche? Pediré mi licencia. Pues qué, ¿acaso la señorita
Inés se acuerda alguna vez de este miserable?

---Respóndeme lo que te inspire tu buena razón, Gabriel---me dijo la
condesa con grave acento.---Figúrate tú que a la señorita Inés se le
pusiese en la cabeza el no querer a nadie más que a ti\ldots{} no es
así\ldots{} pero va como ejemplo: figúratelo.

---Ya está figurado.

---Pues bien: ¿no te parece natural que yo y mis tíos nos opongamos a
ello por todos los medios posibles?

---Sí señora, me parece muy natural---repliqué con asombro;---pero si
ella se empeña\ldots{}

---Ella no se empeña\ldots{} no es eso\ldots{} Es que\ldots{} vamos, te
lo diré francamente. Aunque no aseguro yo que Inés te ame, ni mucho
menos, porque esto sería un gran despropósito, ocurre que\ldots{} es
natural que sienta algún afecto hacia los que fueron compañeros de sus
desgracias\ldots{} Todo es un capricho, una obcecación pueril, que se le
pasará seguramente. ¿No crees que se le pasará?

---Sí señora, le pasará.

---Pero para que esto acabe de una vez, necesito tu ayuda. Puesto que te
veo tan razonable, puesto que reconoces que sería en ti una estupidez
aspirar a casarte con ella\ldots{} ¡Casarte con ella! ¡qué risa! ¡un
pelagatos como tú!\ldots{} parece esto cosa de comedia. ¿Pero no te ríes
tú también?

---Sí señora, ya me estoy riendo---respondí haciéndolo de muy mala gana.

---Pues decía---continuó, cesando en su afectada hilaridad,---que, en
vista de tu buen sentido, espero de ti lo que vas a oír. Repito que te
daré lo necesario para que en otro país lejos de España puedas hacer una
fortuna; te daré la fortuna hecha si quieres\ldots{}

---¿Y qué he de hacer para eso?

---Nada\ldots{} vienes aquí estos días so color de entrar a servirme,
tratas a Inés, y luego durante algún tiempo fingirás hacer las cosas más
feas, cometer las acciones más abominables y los delitos que más rebajan
al hombre, de modo que ella con el espectáculo de tu envilecimiento
vuelva en sí del trastorno que por ti tiene y todo acabe. Es sumamente
fácil para ti: entras aquí en mi servicio, y a los pocos días me robas
una sortija u otra prenda cualquiera; luego fingimos nosotros haber
descubierto tu crimen y afeamos en público tu conducta; después si
hablas con ella, me calumniarás, diciendo de mí mil herejías, y también
hablarás mal de ella delante de alguna criada que venga a
contárnoslo\ldots{} y por este estilo harás una serie de maldades de
esas que más envilecen a la criatura.

---¡Señora!---exclamé sin poder sofocar por más tiempo la ira.---Si usía
me da toda esta casa llena de dinero, no haré lo que me pide. ¡Cometer
delante de ella una infame acción! Me dejaré matar mil veces antes que
tal haga. Cuando éramos amigos, más temía a sus censuras que a mi
conciencia, y si algo bueno hice, hícelo por que ella lo viera y me
aplaudiera; que más estimaba su aprobación que todos los bienes del
mundo. Huiré para ir a donde no me vuelva a ver; pero pensar que he de
envilecerme delante de ella, eso jamás. Adiós, señora, me voy de
aquí---añadí levantándome.---Por segunda vez me quiere usía envolver en
intrigas y fingimientos cortesanos en que es tan gran maestra.

---Aguarda---dijo deteniéndome.

---¿No está más en el orden natural lo que yo quiero
hacer---añadí,---que es marcharme y no aparecer más por Madrid?

---Eres un majadero---dijo con despecho.---¿Qué te cuesta hacer lo que
te propongo? ¿Pierdes tú algo en ello? Ven acá, truhán de las calles:
¿acaso tienes algún nombre que deslustrar o alguna posición que perder?
¡Cuántos mejores que tú no se apresurarían a prestar este servicio por
el aliciente de la recompensa que yo te ofrezco! ¿Pues acaso podías tú
ni soñar con la fortunilla que te pienso ofrecer, farsantuelo? ¡Miren el
caballerón finchado, siempre a vueltas con su honor y su conciencia, y
su deber acá y su reputación allá!

---Si usía me da licencia, me retiraré---dije, resuelto a poner fin a la
conferencia.

---No, aquí has de estar todavía. Por lo que veo, crees que mi primita
se acuerda alguna vez de tus simplezas y majaderías---declaró con
enfado.---Anda noramala, chicuelo andrajoso: ¿piensas que creo en tus
hipócritas declamaciones? ¿Piensas que tomo en serio los generosos
pensamientos que con tanto arte me has manifestado, echándotela de
caballero? ¡Oh! ¡Esto me pone fuera de mí! Yo le diré a esa antojadiza
quién eres tú y cuáles son tus mañas. O hará lo que yo le mando---añadió
con creciente enojo,---y pensará como yo quiero que piense, o esa niña
no es de mi sangre, no, no puede serlo. ¡Cuánta contrariedad, Dios
mío!\ldots{} No quiero verte más, Gabriel, vete de aquí\ldots{} pero no,
ven acá: tú no tienes la culpa de esto. Dime, ¿quién eres tú? ¿Dónde has
nacido? ¿Tienes alguna noticia de tus padres?\ldots{} A veces suele
acontecer que el que se creía humilde\ldots{}

---No espere usía---repuse sonriendo,---que de la noche a la mañana me
caiga en herencia un gran ducado. Eso pasa algunas veces, como ha
sucedido con Inés; pero de tales pasos de novela entran pocos en libra.
Humilde nací, y humildísimo seré toda mi vida.

---Lo digo por que si tú fueras una persona decente, te sentarían bien
esos aspavientos que has hecho---me contestó.---No lo decía por otra
cosa, desdichadote; no te vayas a envanecer sin motivo. Vete, estoy muy
disgustada.

Y luego olvidándose de mí para no pensar más que en sus propias
contrariedades, exclamó así:

---¿Por qué, Dios mío, cuando trajiste a esa niña a nuestra casa, nos
trajiste también esta gran pesadumbre?

---¿Quiere usía mucho a su hija?---le pregunté.

---A mi prima, querrás decir.

---Eso es, me equivoqué.

---¡Que si la quiero! Desde que entró aquí no vivo más que para ella. Es
un santo delirio lo que siento, y si Inés me faltara, me moriría sin
remedio. Mi desesperación consiste en que al traerla aquí no podemos o
no sabemos darle la felicidad que ella merece. ¿Pero es acaso culpa
nuestra?

---¿Y persiste vuecencia en casarla con don Diego?

---¡Oh, no! D. Diego es un libertino; ya no me queda duda. Yo me opondré
a que se case con él.

---Hace bien usía, y a la señorita Inés no le faltarán jóvenes de
familia distinguida entre quienes elegir esposo. Por de pronto, señora,
yo me atrevo a aconsejar a usía que rompa definitivamente con D. Diego.
Las malas compañías de este joven son un peligro para la tranquilidad de
esta casa.

---¿Qué quieres decir? Ahora me viene a la memoria ese hombre que hace
poco nombraste y que me causa miedo.

---¿Santorcaz? Sí, señora; y ya que le nombro, voy a tener el valor de
poner a vuecencia al corriente de ciertas asechanzas, para que esté
prevenida. Yo asistí a la batalla de Bailén, y allí por casualidad
singular, vinieron a mis manos unas cartas\ldots{}

Amaranta se inmutó.

---Señora, si he sabido casualmente alguna cosa que no debía saber, yo
juro a usía que el secreto no ha salido de mis labios ni saldrá mientras
viva.

La condesa pareció poseída de nerviosa exaltación.

---¡Estás loco!---exclamó.---¡Qué majaderías me cuentas! Ni qué tengo yo
que ver con esas cartas ni con ese hombre\ldots{}

---En fin, señora, aunque de a usía un mal rato, quiero entregarle las
dichas cartas.

---A ver, a ver---dijo pasando de la exaltación a un desvanecimiento y
palidez intensa que la puso como difunta.

---Vea Vd. esta primera---dije entregándole la que ella había dirigido a
Santorcaz.

---Esto parece un sueño---exclamó reconociéndola.---Pero ¿cómo ha
llegado a tus manos este papel? ¡Miserable chiquillo de las calles!
¿quién te mete a leer estas cosas\ldots?

Entonces le conté el suceso que me puso en posesión de aquellas
esquelas, lo cual oyó muy atentamente, y después oprimiéndose las sienes
con ambas manos, exhaló lamentos dolorosos.

---Pues ahora vea usía esta otra que parece contestación a la
precedente, y que no llego a ponerse en el correo, pero que al fin viene
a su poder, aunque tarde, por mi conducto.

Amaranta leyó ávidamente la carta, y a cada rato la indignación se
traslucía en su hermoso semblante. Cuando la hubo leído, rompiola
coléricamente en menudos pedazos, y dijo así:

---¡Ese miserable me amenaza! ¡Dice que si su hija no está hoy en su
poder lo estará mañana!

---Vuecencia recordará lo que ocurrió cuando la familia toda vino de
Andalucía. Yo vine en la escolta que acompañó a sus mercedes desde
Bailén hasta Santa Cruz de Mudela, y contribuí a poner en fuga a la
canalla que detuvo los coches.

---Eran ladrones.

---Sí; pero su intento no era despojar a los viajeros. Usía recordará
que nos fue muy fácil darles una severa lección; pero lo que sin duda
ignora es que allí estaba el Sr.~de Santorcaz, escondido entre las
cercanas malezas, pues él y no otro mandaba aquella brillante tropa de
forajidos. Yo que había leído la carta y además tenía sospechas por
ciertas palabras que en Bailén oí a ese D. Luis, solicité un puesto en
la escolta que al señor marqués concedió el general, y en ella formaron
también algunos de mis buenos compañeros. Pero todavía falta a vuecencia
el leer la más curiosa de las tres cartas que en aquella ocasión
memorable vinieron a mis manos. Aquí está, y ella le hará ver la infame
deslealtad de un criado de su propia casa.

Tomó la condesa la carta en que Román daba a Santorcaz noticia
circunstanciada de lo ocurrido con motivo de la legitimación de Inés, y
mientras la leía, tan pronto hacía brotar lágrimas de sus ojos la rabia
como los inflamaba con vivo resplandor.

---Ya sospechaba yo la infidelidad de ese vil que todo nos lo
debe---exclamó;---pero mi tía le tiene cariño y por eso sigue en la
casa\ldots{} ¡Qué infamia! Pero necio mozalbete, ¿para qué has leído
estas cosas? Vete, quítate de mi presencia\ldots{} no, no, ven acá: tú
no eres culpable.

---Señora---respondí,---ningún nacido sabrá de mí lo que usía no quiere
que se sepa. Yo esperaba una ocasión de entregar a vuecencia esas
cartas, y mientras han estado en mi poder, nadie, absolutamente nadie
más que yo las ha leído.

---¡Oh! ya sé lo que debo hacer para defenderme, y defender a mi hija de
tan miserables asechanzas.

---Santorcaz es íntimo amigo de D. Diego, le acompaña a todas partes, le
aconseja y le dirige. Yo he sorprendido sus conversaciones íntimas, y
por ellas veo que el pérfido amigo y consejero de Rumbrar no ha
desistido de sus proyectos.

---Yo estoy trastornada, yo estoy confusa---dijo Amaranta levantándose
de su asiento.---No, no, Gabriel, no te vayas, tú eres un buen muchacho:
yo quiero recompensarte de algún modo dándote lo necesario para que
vivas con el decoro que mereces\ldots{} Pero no pienses en Inés ¿sabes?
Es una demencia que pienses en ella. ¡Pobre hija mía! La hemos sacado de
la miseria, la hemos dado nombre, fortuna, posición, y no podemos
hacerla feliz. ¡Esto me vuelve loca! Cuando la veo indiferente a todas
las distracciones que le proporcionamos; cuando veo la imposibilidad de
hacerme amar por ella, como yo quiero que me ame; cuando la observo
pensativa y muda, y considero que echa de menos la apacible estrechez y
contento que disfrutaba viviendo con el cura de Aranjuez, me siento
morir de pena y paso llorando largas horas. ¡Pobre hija mía! ¡Ni
siquiera le puedo dar este nombre, pues hasta con los de casa he de
guardar secreto! ¡Ella y yo somos igualmente desgraciadas!\ldots{} ¿Por
qué no haces lo que te propuse, Gabriel? ¿A que vienes con humos
caballerescos? ¿Eres acaso más que un infeliz? Pero no: tienes razón, no
te degrades a sus ojos; tú tienes sentimientos nobles, tú eres un
caballero, aunque no lo parezcas; tú mereces mejor suerte; Dios no es
justo contigo\ldots{} ¡Ay! voy viendo que tú también eres muy
desgraciado.

Esto decía la condesa con muestras no sólo de gran dolor sino también de
cierta confusión mental hija de las diversas sensaciones a que se había
visto sometida; y sentándose luego, permaneció en silencio gran rato.
Así estaba cuando creí sentir lejano ruido de voces en el interior de la
casa, rumor que apenas se percibía y que para mí hubiera pasado
inadvertido, a no haber corrido Amaranta súbitamente hacia una de las
puertas, prestando atención a lo que tan débilmente se oía.

---Es mi tía---dijo después de una larga pausa;---es mi tía que no cesa
de reñirla. Porque no quiere someterse a las majaderías de un ridículo
maestro de baile, ni hacer dengues ante los petimetres que nos visitan,
la tratan de este modo. ¡Y yo no puedo impedirlo, Dios mío!---añadió
juntando las manos con mucha aflicción.---¡Pero si no soy nada aquí, ni
tengo autoridad alguna sobre ella! He de presenciar sus martirios,
fingiendo aprobarlos, y estoy condenada a aplaudir las violencias, las
intolerancias, las imposiciones, las mezquindades que la hacen tan
infeliz.

Amaranta hizo ademán de salir; contúvose junto a la puerta, retrocedió
luego indicando en su marcha y ademanes una grandísima agitación.
Después me miró con asombro, como si se hubiese olvidado de mi presencia
y de improviso me viera.

---Gabriel---me dijo.---Vete, vete al punto de aquí, y no vuelvas más.
¡Ay! ¿Por qué no querrá Dios que, en vez de ser quien eres, seas otra
persona?

La conmoción me impedía hablar, y sin decir sino medias palabras,
despedime de ella, besándole respetuosamente las manos. Entonces
Amaranta me tomó una de las mías, y mirándome con calma, derramando
lágrimas de sus bellos ojos, me dijo esto, que no olvidaría aunque mil
años viviese:

---Gabriel, eres un caballero; pero Dios no ha dispuesto darte el nombre
y la condición que mereces. Si quieres darme una prueba de la nobleza de
tus sentimientos y de la rectitud de tu juicio, prométeme que has de
desaparecer para siempre de Madrid, y no presentarte jamás donde ella te
vea. Se le dirá que has muerto.

---Señora---respondí,---ignoro si me permitirán salir de Madrid, pero si
algo impide esta mi resolución, yo prometo a usía, por Dios que nos oye,
salir de Madrid; y entretanto que aquí esté, juro que no me presentaré a
ella, ni haré por verla, ni consentiré en cosa alguna por la cual venga
a conocer que estoy en el mundo. Este es mi deber.

---Tendré presente lo que me has jurado---dijo ella.---No te
arrepentirás de tu conducta. Adiós.

\hypertarget{ix}{%
\chapter{IX}\label{ix}}

Estrechome entre las suyas mis manos la condesa con muestras de vivo
agradecimiento, y salí de aquella estancia y del palacio con tan
profunda emoción, que no era dueño de mí mismo. Cuando llegué a mi casa,
después de vagar por Madrid toda la tarde, arrojeme sobre mi lecho,
donde en vela pasé la noche entera, revolviendo en mi mente las palabras
del diálogo con Amaranta, llorando a veces, a veces profiriendo gritos
de rabia, y tan excitado, que mis buenos patronos creyéronme atacado de
violenta fiebre.

A la mañana siguiente, después que rendido a la fatiga dormí con sueño
irregular y espantoso durante algunas horas, doña Gregoria llegose a mí
y me despertó diciendo:

---¿Qué es esto? Durmiendo a las diez de la mañana. Arriba, arriba,
mocito. ¡Y se ha acostado vestido! Vamos, que son las diez\ldots{} Pero,
chiquillo, ¿qué haces, en qué piensas? Por ahí ha pasado la quinta
compañía de voluntarios, tan majos y tan bien puestos con sus uniformes
nuevos que darían envidia a un piquete de guardias walonas. ¡Ay qué
monísimos iban! A los franceses les dará miedo sólo de verlos. Nada les
falta, si no es fusiles, pues como en el Parque no los había, no se los
han podido dar; pero llevan todos unos palitroques grandes que les caen
a las mil maravillas, y de lejos parece que llevan escopetas. Vamos,
levántese el señor Gabrielito: ¿no eres tú de la quinta compañía?
Levántate, que ya dicen que está Napoleón Bonaparte a las puertas de
Madrid, montado en una mula castaña y con la lanza en el ristre para
venir a atacarnos.

---Mujer, ¿qué disparates estás diciendo?---observó el Gran
Capitán.---Napoleón no está en Madrid, sino que parece entró ya en
España y anda sobre Vitoria. Por cierto que dicen ha habido una
batallita\ldots{} Pero, chico, ¿no vas a coger tu fusil?

---Hoy mismo me voy de Madrid, Sr.~D. Santiago.

---¿Que te vas de Madrid, después de alistado? Pues me gusta el valor de
este mancebo.

---Es que voy a ver si me permiten pasar al ejército del Centro que está
en Calahorra, y creo que me lo permitirán.

---¡Oh! no lo esperes, porque aquí, según me dijeron en la oficina, lo
que quieren es gente y más gente, pues como algunos dan en decir que hay
malas noticias\ldots{} Yo creo que todo es cosa de los papeles públicos,
y a mí no me digan; los papeles públicos están pagados por los
franceses.

---¿Con que malas noticias?

---Paparruchas\ldots{} En primer lugar, ahora salen con que lo de
Zornoza que creíamos fue una gran victoria, es una medianilla derrota, y
que el general Blake ha tenido que escapar refugiándose en las montañas.
No se pueden oír estas cosas con calma, y yo mandaría que se le
arrancara la lengua al que las repite.

---¡Mentiras, todo mentiras!---exclamó doña Gregoria.---¡Si no sé cómo
la Junta no manda ahorcar en la plazuela de la Cebada a todos los que se
divierten con tales disparates!

---Has hablado muy bien---dijo el Gran Capitán.---Ahora han dado en
decir que si en Espinosa de los Monteros ha habido o no ha habido una
batalla.

---¿En que también hemos perdido?---preguntó doña Gregoria.

---¡Así lo dicen; pero quia! Bonito soy yo para tragarme tales bolas.
Ahora encontré al volver de la esquina al Sr.~de Santorcaz, el cual me
lo dijo, fingiéndose muy apesadumbrado\ldots{} ¡Pícaro marrullero! Como
si no supiéramos que es espía de los franceses\ldots{}

---¿Con que en Espinosa de los Monteros? ¿Y hemos tenido muchas
pérdidas? ---pregunté yo.

---¿También tú?---dijo Fernández sin poder disimular el pésimo humor que
tenía.---Te voy descubriendo que tienes muy malas mañas, Gabriel.

---No hagas caso de este chiquillo mal criado---dijo doña Gregoria.

---Es preciso que aprendas a tener respeto a las personas
mayores---afirmó el Gran Capitán, mirándome con centelleantes
ojos.---¿Qué es eso de pérdidas? ¿He dicho acaso que nos han derrotado?
No mil veces, y juro que no hay tal derrota. ¿Hombres como yo pueden dar
crédito a las palabras de gente desconsiderada y vagabunda?

Calleme por no irritar más a mi ingenuo amigo, y mientras me daban de
almorzar, entró una visita que en mí produjo el mayor asombro. Vi que
avanzaba haciéndome pomposos saludos, y mostrándome en feroz sonrisa su
carnívora dentadura, un hombre de espejuelos verdes, en quien al punto
conocí al licenciado Lobo. Lo que más llamaba mi atención eran los
extremos de cortesía y benevolencia que en él advertí, y el de su osado
respeto hacia mi persona que en todos sus gestos y palabras mostrara
aquel implacable empapelador, y antes enemigo mío.

---¿Qué bueno por aquí, Sr.~de Lobo?---díjele, ofreciéndole junto a mí
una silla en que se repantigó.

---Quería tener el gusto de ver al Sr.~D. Gabriel.

---\emph{¿Señor Don} tenemos? \emph{Malum signum}.

---Y de poner en su conocimiento algo que le importa
mucho---añadió.---¿Pero cómo no ha ido a verme el Sr.~D. Gabriel?

---Ya le he encontrado a Vd. muchas veces en la calle, y como no ha
tenido a bien saludarme\ldots{}

---Es que no habré visto a Vd.---me contestó melosamente.---Ya sabe el
Sr.~D. Gabriel que soy más que medianamente ciego\ldots{} Pues bien:
como decía\ldots{} El Gobierno ha tenido a bien remunerar los buenos
servicios de Vd.

---¡Mis buenos servicios!---exclamé asombrado.---¿Y qué buenos ni malos
servicios he prestado yo al Gobierno?

El Gran Capitán y su esposa con medio palmo de boca abierta, prestaban
gran atención.

---Modestito es el joven---prosiguió Lobo con aquel artificioso sonreír,
que le hacía más feo, si es que cabía aumento en las dimensiones
infinitas de su fealdad.---Yo he oído que Vd. se lució mucho en la
batalla de Bailén, y no sé si también en la de Trafalgar, donde parece
que mandó un par de fragatitas o no sé si un navío.

Prorrumpí en risas, y los dos ancianos, mis amigos, mirándose uno a otro
con espontánea admiración por mis inéditas hazañas.

---Sí\ldots{} algo de esto ha llegado a oídos del justiciero Gobierno
que nos rige, y las comisiones ejecutivas de la Junta se disputan cuál
de ellas echará el pie adelante en esto del recompensar a usía.

---Hola, hola, ¿también soy usía? Pues esto sí que me llena de asombro.

---Pero sea lo que quiera, amigo mío---continuó el leguleyo,---ello es
que se ha decidido darle a usía un empleo en América, al inmediato
servicio del señor Virrey del Perú.

---¿Trae Vd. mi nombramiento?---dije comprendiendo al fin de dónde venía
todo aquello.

---No; hoy sólo vengo a notificarle a usía este gran suceso, y a
advertirle que cualquier cantidad que necesite para preparar su viaje,
me la pida con franqueza, pues tengo orden de la\ldots{} digo, del
Gobierno, para entregar a usted lo que tenga a bien pedirme, previo
recibito que me extenderá vuecencia.

---¿También soy vuecencia?---dije recreándome en la estupefacción de mis
dos amigos.

---El nombramiento---prosiguió,---lo tendrá usía dentro de dos o tres
días; pero le advierto que es voluntad de la Junta Suprema que el Sr.~D.
Gabriel se haga a la vela al punto para las Américas, donde pienso que
es de gran necesidad su presencia.

---Bueno---repuse;---pero entretanto yo le ruego al Sr.~de Lobo diga a
la Junta que no me hace falta dinero, y que muchas gracias.

---Eso no está bien---dijo doña Gregoria muy incomodada.---Pero tonto,
si te lo dan, recíbelo y guárdalo sin averiguar de dónde viene. Estas
cosas no pasan todos los días. Apuesto a que la Junta ha sabido lo de
tus latines y te manda allí para que enseñes esa lengua a los salvajes,
con lo cual se convertirán todos. ¿No es verdad, Sr.~de Zorro, que así
ha de ser?

---No me llamo Zorro, sino Lobo---repuso este,---y hará muy bien el
Sr.~D. Gabriel en tomar lo que le haga falta, pues a su disposición lo
tiene.

---Pues bien---dije yo,---vaya usted de mi parte a la señora Junta que
le dio tan buen recado para mí, y dígale que para servir a la patria y
al Rey, yo no pensaba pasar a América, sino al ejército del Centro y de
Aragón, en cuyo Reino pienso quedarme y no volver a Madrid mientras
viva. Para este viaje no se necesitan gastos.

---¿Y qué va a hacer el Sr.~D. Gabriel en el ejército de Aragón? Aquello
está mal ---dijo Lobo.---Por el de la izquierda no andan mejor las
cosas, y después de la batalla que hemos perdido en Espinosa de los
Monteros, nuestras tropas quedan reducidas a nada, y Napoleón vendrá a
Madrid.

---¡Eso será lo que tase un sastre!---exclamó el Gran Capitán echando
chispas.---¿Quién hace caso de los papeles?

---Desgraciadamente---continuó Lobo,---esa sensible derrota no puede
ponerse en duda.

---Pues yo la pongo---afirmó Fernández rompiendo un plato que al alcance
de la mano tenía sobre la mesa.---Sí señor, yo la pongo en duda, y es
más, yo la niego.

---El señor---dijo doña Gregoria,---seguramente no sabe quien eres tú, y
el cómo y cuándo de lo bien enterado que estás de todo.

---Yo sé la noticia por buen conducto, y aseguro que es
indudable---indicó Lobo.---El secretario del ramo de guerra me lo ha
dicho.

---Buen caso hago yo del secretario del ramo de guerra,---dijo Fernández
amoscándose en grado supino.

---Vamos, no porfíes, Santiago\ldots---añadió doña Gregoria.---Estás más
encarnado que pimiento de Calahorra, y no está bien que te dé el reuma
en la cara por una batalla de más o de menos.

---Pues que no me falten al respeto. Eso de que le insulten a uno en su
propia casa---dijo Fernández dando un puñetazo en la mesa.---Porque,
digan lo que quieran, donde menos se piensa salta un espía de los
franceses, ¡Madrid está lleno de traidores!

Asustado Lobo del enérgico ademán de don Santiago, no quiso insistir en
lo de la derrota, y proclamó muy alto que la batalla de Espinosa de los
Monteros había sido ganada y reganada y vuelta a ganar por los
españoles, oyendo lo cual se apaciguó nuestro veterano de las
portuguesas campañas y habló así:

---Me parece que tiene uno autoridad para decir quién gana y quién
pierde en esto de las batallas\ldots{} y todos no entienden de achaque
de guerra\ldots{} y una acción parece derrota de diablos hasta que viene
una persona inteligente y la explica, y resulta victoria de
ángeles\ldots{} y no digo más, porque sé dónde me aprieta el zapato, y
en Espinosa de los Monteros lo que hubo fue que todos los franceses
echaron a correr, y el hi de mala mujer que me desmienta, sabrá quién es
Santiago Fernández.

Dijo y levantose, cantando entre dientes un toquecillo de corneta; y
dirigiéndose luego a donde desde lueñes edades tenía su lanza, la cogió,
y con un paño la empezó a limpiar del cuento a la punta, dándole
repetidas friegas, pases y frotaciones, sin atender a nosotros ni cesar
en su militar cantinela. En tanto Lobo, que en todo pensaba menos en
llevarle la contraria, continuó hablándome así:

---Ahora, Sr.~D. Gabriel, me resta tocar otro punto, y es que me diga
Vd. algo de su parentela y abolengo, porque es preciso sacarle una
ejecutoria. Con diligencia, el Becerro en la mano, y un calígrafo que se
encargue del árbol, todo está concluido en un par de días.

---Mi madre entiendo que lavaba la ropa de los marineros de guerra---le
contesté,---y hágamela su merced duquesa del Lavatorio, o para que suene
mejor de \emph{Torre-Jabonosa} o de \emph{Val de Espuma} que es un
lindísimo título.

---No es broma, señor mío. Al contrario, el destino que Vd. lleva al
Perú, no se le puede dar sin una información de nobleza. Es cosa fácil.
Y de su papá de Vd., ¿qué noticias se pueden encontrar en la tradición o
en la historia?

---¡Oh! Mi papá, Sr.~de Lobo, si no mienten los pergaminos que se
guardan en el archivo de mi casa, y están todos roídos de ratones (lo
cual es muestra de su mucha ranciedad), fue cocinero a bordo de la
goleta Diana, por lo cual le cae bien un título que suene a cosa de
comida\ldots{} pero ahora recuerdo que un mi abuelo sirvió de
alquitranero en la Carraca, y puede Vd. llamarle el archiduque de las
\emph{Hirvientes Breas}, o cosa así.

---Vd. se burla, y la cosa no es para burlas. ¿Su apellido?

---Los tengo de todos los colores. Mi madre era Sánchez.

---¡Oh! Los Sánchez vienen de Sancho Abarca.

---Y mi padre López.

---Pues ya tenemos cogidos por los cabellos a D. Diego López de Haro y a
D. Juan López de Palacio, ese famosísimo jurisconsulto del siglo XV,
autor de las obras \emph{De donatione inter virum et uxorem, Allegatio
in materia hæresis, Tractatum de primogenitura.}..

---Pues de ese caballero vengo yo como el higo de la higuera. También me
llamo Núñez.

---Por las alturas genealógicas de Vd., debe de andar el juez de
Castilla Nuño Rasura. ¿Y no hubo algún Calvo en su familia?

---¿Pues no ha de haber? Mi tío Juan no tenía un pelo en la cabeza.
También me llamo \emph{Corcho}, sí señor, yo soy nada menos que un
\emph{Corcho} por los cuatro costados.

---Feísimo nombre del cual no podemos sacar partido. Si al menos fuera
Corchado\ldots{} pues hay en tierra de Soria un linaje de Corchados que
viene de la familia romana de los \emph{Quercullus}. En lugar del
\emph{Corcho} le podemos poner al Sr.~Gabrielito un \emph{Encina} o
\emph{Del Encinar}, que le vendrá al pelo.

---A mi madre la llamaban la señora María de Araceli.

---¡Oh, bonitísimo! Esto de Araceli es bocado de príncipes, y más de
cuatro se despepitarían por llevar este nombre. Suena así como
Medinaceli, \emph{Cælico Metinensis}, que dijo el latino. No necesito
más.

A todas estas doña Gregoria no sabía lo que pasaba oyendo el diálogo de
linajes; y absorta y suspensa aguardaba en silencio en qué vendría a
parar todo aquel belén de mis apellidos.

---Que es de buena sangre el niño, no lo puede negar---dijo al
fin,---porque bien se conoce en la nobleza de su condición, que hartos
hay por ahí llenos de harapos, y a lo mejor salen con la novedad de que
son hijos de un duque; y aquí estoy yo que tampoco doy mi brazo a
torcer, pues los Conejos de Navalagamella no son ningún saco de paja.

---¿Qué Conejos son esos, señora mía?

---El mejor linaje de toda la tierra. Yo soy Coneja por los cuatro
costados. El señor licenciado sabrá de qué fuentes antiguas vendrá este
arroyo genealógico de la Conejería.

---Como estos gazapos---contestó el licenciado,---no vengan de aquellos
tiempos remotísimos en que a España la llaman cunicullaria, es decir,
\emph{tierra de los conejos}, no sé de dónde pueden venir.

---Así debe de ser. ¿Y el Sr.~D. Gabriel de dónde viene?

---Eso lo dirá el Becerro. Ahora veo que este señor de Araceli no es
cualquier cosa, y aquí en dos palotadas hemos encontrado robustas
columnas donde apoyar la grandiosa fábrica de su alcurnia. Pero hablando
de otra cosa, señor de Araceli, ¿quién me abonará los gastos de la saca
de ejecutoria, Vd. o la persona que me ha dado el encargo de hacer estas
diligencias y de ofrecer el dinero?\ldots{} Porque los gastos son
muchos. Además, esta comisión tan bien desempeñada, ¿no merece alguna
recompensa? Yo creo que la dará la señora cond\ldots{} quiero decir la
Junta Central, que es quien me la ha enviado.

---Más vale que el señor licenciado no se tome el trabajo de revolver
papeles ni pintar árboles, pues yo no se lo he de pagar, y ese dinero
que me ofrece tampoco lo he de tomar.

---Eso sí que no lo consiento---manifestó doña Gregoria.---No ha de ser
así. Santiago: oye lo que dice este porro.

---Usted lo meditará mejor---dijo el leguleyo levantándose.---En cuanto
a mí, espero ganar algo en estos jaleos, porque, amigo mío, ¿cómo se da
de comer a diez hijos, mujer y dos suegras? Dentro de unos días volveré
a traer a usted el nombramiento, y un poco más tarde la ejecutoria. Y en
cuanto al dinero, con ponerme dos letritas\ldots{}

---Bueno---respondí, considerando que me convenía disimular por de
pronto mis intenciones.---Yo haré lo que me parezca, y nos veremos
Sr.~D. Severo.

---Adiós, mi querido e inolvidable amigo---dijo deshaciéndose en
cumplidos.---Que esto sirva para estrechar más los lazos de la dulce
amistad que desde ha tiempo nos profesamos.

---Sí, desde el Escorial.

---Justamente. Desde entonces le eché el ojo al Sr.~de Araceli, y
comprendiendo sus excelentes prendas, lo diputé por grande amigo mío.
Venga un abrazo.

Se lo di, y fuese tan satisfecho. Entretanto habían acudido a casa del
Gran Capitán los vecinos, traídos todos por el olor de mi estupendo
destino y del encumbramiento novelesco, que ninguno quiso creer, si doña
Gregoria no lo jurara en nombre de todos los Conejos de
navalagamellescos.

---¿Que no lo creen ustedes?---decía el Gran Capitán a las niñas de doña
Melchora.---Como que me lo han hecho virrey del Perú.

---¡Virrey del Perú!!!

---Sí\ldots{} y no quedó cosa que no sacó aquí ese Sr.~de Lobo, Zorro o
Leopardo---añadió doña Gregoria.---Y ahora parece que está tan clara
como la luz del sol la nobleza de este niño. ¡Si vieran Vds. la sarta de
duques, condes y marqueses, que han aparecido entre sus abuelos! ¡Jesús,
y quién lo había de decir!\ldots{} Y le dan todo el dinero que quiera
pedir por esa boca\ldots{} Como que pretenden que se vaya pronto para
las Américas a arreglar a aquella gente que anda toda revuelta\ldots{}
¿No te lo decía yo, picaronazo? Alguna cosa gorda te tenía reservada
Dios por ese tu buen natural\ldots{} y que eres tú tonto en gracia de
Dios\ldots{} Nada, nada, toda esa parentela que te ha salido hirviendo
como garbanzos en puchero te está muy bien merecida.

---Pues convídenos al señor perulero a piñones---dijo doña Melchora.

---¿De modo que ya no coges el fusil?---me dijo D. Roque.

---Y ahora hace falta---añadió Cuervatón.---Pronto tendremos aquí a ese
infame \emph{córcego}.

---Sí, porque lo de Espinosa de los Monteros ha sido un menudo
descalabro.

---¡Cómo descalabro!---exclamó furiosamente una voz que no necesito
decir a quien pertenecía.

---Sí señor, un descalabro. Ya lo sabe todo el mundo. La retirada fue
además desgraciadísima, y ha perecido mucha gente.

D. Santiago Fernández, que ya estaba de muy mal humor, se puso en punto
de caramelo, y después de dudar un rato si contestaría a tales
insolencias con un abrumador desprecio o con enérgicas negativas,
decidiose por lo último, diciendo:

---En esta casa no se consiente gente perdida, porque juro y rejuro que
los que hablan así de la batalla de Espinosa de los Monteros son espías
de los franceses, y no digo más. Basta de disputas: cada uno meta su
alma en su almario\ldots{} y silencio, que aquí mando yo, y cuidadito
con lo que se habla, que a mí no se me falta el respeto.

\emph{Conticuere omnes}.

\hypertarget{x}{%
\chapter{X}\label{x}}

Quiere el buen orden de esta narración, que ahora deje a un lado la gran
figura del Gran Capitán, con cuyas eminentes dimensiones se llena toda
la historia de aquellos tiempos; que también pase en silencio por ahora
no sólo las hazañas que piensa hacer, sino sus admirables sentencias y
el dictamen profundo que sobre los asuntos de la guerra daba, y pase a
ocuparme de D. Diego de Rumblar. Es el caso que una noche encontrele
camino de la calle de la Pasión; y al instante me cosí a su capa,
resuelto a seguirle hasta la mañana, si preciso era.

---¡Oh Gabriel! ¡Qué caro te vendes! Chico, toma tus dos reales. No me
gustan deudas.

---¿Ya ha salido Vd. de apuros? No será por lo que le haya dado el
Sr.~de Cuervatón.

---¡Miserable usurero! No pienso pedirle más porque ahora tengo todo lo
que me hace falta. ¿A que no saltes quien me lo da? Pues me lo da
Santorcaz.

---Eso es raro, porque yo suponía al señor D. Luis más en el caso de
recibir que de dar.

---Pues ahí verás tú. Ahora tiene mucho dinero, sin que sepa yo de dónde
le viene. Parece un potentado el tal Santorcaz. ¡Cuánto me quiere y con
cuánto talento me indica todo lo que debo hacer! Habías de verle cómo me
ofrece dinero y más dinero, por supuesto dándole un recibito en toda
regla. Ayer me prestó mil y quinientos reales que necesitaba para
comprarle un collar de corales a la Zaina.

---¿Y es posible que gaste Vd. su dinero en tales obsequios, cuando
tiene una tan linda novia con quien se ha de casar?\ldots{}

---Qué quieres, chico: una cosa es el noviazgo, y otra es tener uno una
mujer\ldots{} pues. La Zaina me vuelve loco.

---¿Pero no se casa Vd.?

---¿Pues no me he de casar? Por de contado. Me parece que alguien de la
familia se opone; pero no me apuro mientras tenga de mi parte a la
marquesa. El casamiento es indispensable, porque es cosa de
conveniencia. Mi madre me dice en todas sus cartas que si no me caso
pronto, me abrirá en canal. La boda sobre todo; pero lo cortés no quita
a lo valiente. ¿Has conocido mujer más salada, más seductora que la
Zaina?

---Pues yo he oído, y esto lo digo para que Vd. se ande con tiento, que
el Sr.~de Mañara es el cortejo de la Zaina.

---Así se dice\ldots{} pero a mí con esas\ldots{} Puede que en un tiempo
mi amigo D. Juan tuviera ese capricho; pero ya no hay tal cosa.

---Y que D. Juan salía al amanecer de casa de la Zaina, cierto es,
porque yo lo he visto.

---Nada de eso hace al caso---repuso D. Diego con petulancia.---Lo que
es hoy, Ignacia se está muriendo por el que está dentro de esta capa. Ya
verás esta noche cómo no me quita los ojos de encima. Además, yo sé que
Mañara bebe los vientos por otra mujer.

---¿Por otra?

---Mejor dicho, por dos. Mañara ha vuelto a enredarse con la señora
aquella que fue causa de un escándalo el año pasado, según oí contar, y
además anda en tratos con la María Sánchez, hermana de la Pelumbres. Y
que con la Zaina no tiene nada, lo prueba que anoche se pusieron de
vuelta y media en casa de esta. ¡Bonito pañuelo de encajes, y bonita
mantilla blanca lució en los novillos de anteayer la Pelumbres! Todo es
regalo de Mañara, y anoche estuvieron juntos en la cazuela del Príncipe,
y fueron después a cenar en casa de la González. De modo que nadie me
disputa a mi Zainita de mi alma.

En esto llegamos a casa de la semidiosa de las coles, lechugas y
tomates, y vímosla trasegando de un pequeño tonel a media docena de
botellas una buena porción de aguardiente, al cual, como católica
cristiana, administraba el primer sacramento con el Jordán de un botijo
de agua que allí cerca tenía. Lejos de ella, y a otro extremo de la
salita, se calentaban junto a un braserillo el tío Mano de Mortero,
padre de la Zaina, Pujitos y el simpático cortador de carne, a quien
llamaban Majoma, los tres muy enredados en una calurosa conversación
sobre los negocios públicos. Sin hacer caso de aquel grupo, que a su vez
no lo hacía de los visitantes, D. Diego y yo nos fuimos derechamente a
la Zaina, y aquí me corresponde hacer de ella la más exacta pintura que
esté a mis cortos alcances.

Era Ignacia Rejoncillos la más hermosa escultura de carne humana que he
visto; y digo esto no porque yo la viese jamás en aquel traje que suelen
usar la Venus de Médicis, la de Milo ni otras marmóreas damas por el
mismo estilo, sino porque claramente se le traslucían, a favor de los
vestidos de entonces, la corrección, elegancia y proporcional forma de
las distintas partes de su cuerpo; que el traje, lejos de afear estas
femeninas esculturas, antes bien las hermosea, y más admirables son
supuestas que vistas.

Guapísima de rostro, tenía un blanco nacarado, sin que jamás se hubiese
puesto otro afeite que el del agua clara, y unos ojos chispos, pardos,
adormecidillos, tan pronto lánguidos como enardecidos, de esos medio
santurrones y medio borrachos, que suelen encontrarse viajando por
tierra de España, detrás del cajón de una plazuela, al través de las
rejas de un convento, y para decirlo todo de una vez, lo mismo en
cualquier paraje público que privado. Aunque algo chatilla, sus dientes
de marfil, su linda boca, que era puerta de las insolencias, su garganta
y cuello alabastrino bastaban a oscurecer aquel defecto. Las manos no
eran finas, como es de suponer; pero sí los pies, dignos de reales
escarpines, y tenía además otro encanto particularísimo, cual era el de
una voz suave, pastosa y blanda, cuyo son no es definible, y a quien
daba mayor gracia lo incorrecto de la pronunciación y los solecismos que
embutía en el discurso.

---Querida Zaina---le dijo amorosamente don Diego,---anoche soñé
contigo.

---Y yo con las monas del Retiro---contestó ella.

---Soñé que me querías mucho, y cuando desperté estuve llorando media
hora al ver que todo era sueño.

---¿Y cuánto me quiere su merced? Lo que hace yo, estoy toda muerta y
tengo el corazón hecho un ginovesado de tanto quererle.

---¡Si dijeras verdad, ingrata Proserpina, orgullosa Juno, artificiosa
Circe! Tu corazón es de duro diamante o risco, y en vano mi amor quiere
traspasarle con los acerados dardos de su carcaj.

---¿Qué motes son esos que me ha puesto, señor conde?---exclamó la Zaina
riendo a carcajada tendida.---¡Puerco-espina yo! ¿Y qué es eso de los
carcajales y de los diamantes duros?

---Esto lo he oído en una poesía que leyeron esta noche en la Rosa-Cruz,
y a ti te viene de molde. Dime: ¿por qué no me contestaste a la
tiernísima carta que te escribí el otro día?

---¿Yo contestar, hombre de Dios? Así cuervos se lo coman. ¿Cómo he de
contestar si no sé escrebir? Allí leyeron el papé los amigos, y tuvieron
dos horas de fiesta y risa con aquello del llagado corazón de su merced,
y que yo era una paloma torcaz y una ruiseñora, y que me tiene un amor
edial y pantásmico.

---¡Ideal y fantástico! decía la carta, lo cual significa que te quiero
con amor puro y platónico, sin mezcla de ningún liviano apetito.

---¡Ande y que le den garrote! No me hable usía en lengua gringa que no
entiendo.

---¿Y qué te han parecido los corales?

---¿Los colares? Mazníficos, como ahora se dice. Sólo que ya podía usía
haberlos acompañado de la friolera de un par de zarcillos y de una
peineta de carey de las que hoy se usan. Y no se olvide mi condito del
alma que me ha prometido un coche pa dir el lunes a los novillos, ni de
aquellas doce varas de cotonía para hacerme lo que llaman ahora un
\emph{savillé}. Si no, manque se güelva irmitaño y alacoreta, como dice
en su cartapacio, no le he de querer.

---Todo eso tendrás y aún mucho más---dijo D. Diego tomándole un brazo.

---En el ínterin, manos quietas, Sr.~D. Diego, que quien es platono y
pantásmico, como usía dice, no ha de gustar de pelliscar carne fofa como
la mía. Pero venga acá y contésteme. ¿Se afirma en lo que anoche me
contó del señor de Mañara?

---Punto por punto, Zainilla de mis entrañas.

---No es que me importe nada de lo que hace ese calaverilla---añadió la
verdulera,---sino que una amiga mía quiere saberlo.

---Pues dile a tu amiga que el Sr.~de Mañara no la quiere ya, porque
está enamorado de una cierta duquesa y de la Pelumbres, entrambas a dos.

---¡Duquesitas a mí!---exclamó Ignacia haciendo un gesto aterrador con
su derecha mano.---Si es la señora que usía nombró anoche\ldots{} ya, ya
la conozco bien. Hace dos años solía ir en ca la Primorosa con otra
amiguita suya, condesa o no sé qué, alta y morena, y con la Pepilla
González, comicastra del treato del Príncipe. ¡Pues no armaban mal jaleo
entre las tres!\ldots{} ¿Y también está con la Pelumbres?

---No: con su hermana Mariquilla; me equivoqué. Eso todo el barrio lo
sabe. ¡Pues no está poco satisfecha Mariquilla! Pero deja eso que nada
te importa, Zaina. ¿Me quieres mucho?

---¡Pues no le he de querer, niño---respondió la Zaina sin mirar a D.
Diego,---si tengo el corazón que no parece sino que en él me enclavan
alfileres!\ldots{} ¿Vendrá D. Juan esta noche?

---¿A ti qué te va ni te viene, capullito de rosa?

Diciendo esto, D. Diego volvió a extender los alevosos dedos para
pellizcarla el brazo; pero en esto alzó la voz el tío Mano de Mortero,
diciendo:

---¿Ya estamos de secreticos? A bien que el Sr.~D. Diego es un caballero
muy apersonado y principal, y viene acá con buenos fines. Nacia, no seas
ortiguilla ni te pongas tan picona con mi señor conde; que si su
grandeza te quiere dar un pellizco es por ver lo que vas engordando, y
no con intención de ser pesado. Sí, que yo iba a consentir otra cosa en
esta casa de la mesma honradez. Pero, ¿dónde están, señor conde, las
espuelas de plata que me prometió?

---Mañana, si Dios quiere, las acabará el platero,---dijo D. Diego
acercándose al grupo.

---¿No sabe usía las noticias que corren?

---Que se ha perdido una batalla en Espinosa de los Monteros.

---Y parece que también anda mal el ejército de Castaños, y que ya
Napoleón va sobre Burgos.

---Todo eso es misa rezada---dijo Pujitos,---porque ya tenemos en
Portugal obra de veinte mil inglesones, que manda uno a quien llaman el
tío \emph{Mor}.

---Buen tiempo viene ahora para el comercio, tío Mano---dijo
Majoma.---Con esto de la guerra, los franceses por el lado de acá y los
ingleses por el lado de allá, la fardería corre que es un primor.

---Dices bien, niñito. La raya de Portugal está hoy que es un bocado de
ángeles, y los comerciantes de Madrid me traen ahora en palmitas. Además
de que no falta género inglés muy barato puesto en Portugal, por la
frontera y por las sierras de Gata y Peña de Francia no se ve un pícaro
guarda, porque todos se han juntado a los ejércitos, de modo que viva mi
señora la guerra mil años, y abajo Napoleón.

---Como venga a Madrid el infame \emph{córcego}---dijo Pujitos,---se va
a quedar asombrado al ver los batallones que hemos formado acá en un
ráscate ahí. ¿Han dido Vds. al enjercicio de hoy? ¡Válgame mi Dios y qué
tropa! Aquello metía miedo, y si en vez de palos llegamos a tener
fusiles, nosotros mesmos nos hubiéramos asustado de nosotros mesmos,
echando a correr por todo el campo de Guardias palante.

---Pues yo no me he querido enganchar---dijo Majoma,---porque una peseta
es poco, y si el tío Mano de Mortero me lleva a la raya, mejor estoy
allí que en Flandes, y dejémonos de coger las armas, que por haberlas
tomado una vez contra un alguacil, me han tenido diez años mirando a la
Puntilla\footnote{Cabo en la entrada de Melilla.} y a los
Farallones\footnote{Peñasco en la entrada de Melilla.}, con una cuenta
de rosario en los pies, que si no es por la jura de mi D. Fernando VII,
allá me comen los cínifes otros diez.

---Eso no debe apesadumbrarte, Majomilla---dijo Mano de Mortero;---que
es de personas cabales el pasear la vista por los Farallones, y testigo
soy yo, que aunque no fui allá por el aquel de ninguna sangría mal dada,
como tú, echáronme dos años por mor de un paseo a caballo en compañía de
cuarenta quintales de hilo de patente, con su \emph{Londón} y todo, que
metí allá por los Alcañices. Pero hijo, acá estamos todos y Dios y la
Virgen nos acompañen para no tener que llevar en los tobillos aquellas
telarañas de a dos arrobas, que es el peor corte de polainas que he
calzado en mi vida.

Llamaron en esto a la puerta, y vimos entrar al Sr.~de Mañara y a
Santorcaz, el primero vestido elegantísimamente de majo, con capa de
grana y sombrero apuntado.

---Gracias a Dios que parece su eminencia por acá---dijo el padre de la
Zaina acercándole una silla a Mañara.

---Ya sabrán Vds. que le tenemos de regidor de Madrid---gritó Santorcaz.

---¡Regidor el Sr.~de Mañara!

---¡Que viva mil años!---exclamaron todos.

---Así es. La sala de alcaldes me ha nombrado---respondió D. Juan,---y
es probable que acepte.

---¿Y no se suspenderán los novillos del lunes?---preguntó con mucho
interés Majoma.

---Como yo mande, habrá novillos, aunque tengamos a las puertas de la
plaza a todos los emperadores del mundo.

---¡Viva el regidor!

---Y dígame usía, angelito de mi alma---preguntó el tío Mano de Mortero
con visible enternecimiento,---esos probrecitos que hace dos meses están
en la cárcel de Villa porque jugaron a la pelota con seis pellejos de
vino por sobre las tapias de Gilimón; esos probrecitos corderos, que son
más buenos que el buen pan y más caballeros que el Cid, ¿no merecerán de
su generosidad que les quite del mal recaudo en que se hallan? ¡Ay, mis
queridos niños! ¡y cómo se me aguan los ojos y se me arruga el corazón
al verlos entre rejas! ¿Cómo no, excelentísimo señor, si les he criado a
mis pechos y enstruido con mis liciones y enderezado con mis palos? No
parece sino que su carne es mi carne, y mal haya el que los vio tan
listos de piernas como de ojos por Peña de Francia y ahora les ve con
los brazos cruzados, entre alguaciles, carceleros y toda esa canalla que
debería estar frita en aceite para que todo el mundo anduviera en regla.

---Sosiéguese el buen Mortero---dijo Mañara,---que si de algo vale mi
influjo, abrazará pronto a sus amigos.

---¡Que suba al quinto cielo el Sr.~D. Juan, y juro que le he de traer
la mejor muda de camisas en pieza que ha tapado carne de corregidor
desde que el mundo es mundo! Ea, a bailar, a cantar. Nacia, trae aquello
blanco del barrilito que apandamos en este viaje.

---¿No han venido Menegilda, ni Alifonsa, ni Narcisa?---preguntó
Mañara.---Esto está más triste que un entierro. Tú, Zainilla, echa unas
boleras para hacer boca.

---¡Yo, yo, boleras!---repuso la Zaina con tono desapacible y
malhumorado.---No me pide el cuerpo boleras.

---Echalas por amor de Dios.

Digo que no me da la gana. ¿Soy figurilla de tutilimundi?

---Nacia---dijo gravemente el padre de la consabida,---no se contesta de
esa manera, y pues el señor regidor de mi alma lo manda, cantarás,
aunque te pudras.

---Un par de seguidillas al menos.

La Zaina cambió de parecer, y rasgueando una guitarra, cantó:

\small
\newlength\mlenc
\settowidth\mlenc{\qquad \qquad Todas las duquesitas}
\begin{center}
\parbox{\mlenc}{\quad Todas las duquesitas                \\
                 De los madriles,                         \\
                 No sirven pa calzarme                    \\
                 Los escarpines.                          \\
                 \quad Dale que dale                      \\
                 \quad Y póngame esa liga                 \\
                 \quad Que se me cae.}                    \\
\end{center}
\normalsize

---¡Otra, otra! Tiene en el cuerpo esta Maldita Zaina toda la gracia del
mundo.

La Zaina continuó:

\small
\newlength\mlend
\settowidth\mlend{\qquad \qquad Todas las duquesitas}
\begin{center}
\parbox{\mlend}{\quad Señora principesa                   \\
                 De panza en trote,                       \\
                 Las sobras que yo dejo                   \\
                 Usted las coge.                          \\
                 \quad Viva quien vive,                   \\
                 \quad Le regalo ese peine                \\
                 \quad Que no me sirve.}                  \\
\end{center}
\normalsize

Aquí fue el batir palmas y el patear suelos y el romper sillas, con
tanto estruendo y algazara que no parecía sino que la casa se venía al
suelo. La Zaina arrojó después lejos de sí la guitarra con tal fuerza,
que aquel sensible instrumento, al dar violentamente contra una silla,
lanzó un quejido lastimero y se le saltaron dos cuerdas. Acto continuo
sentose junto a D. Diego. Poco después entraron metiendo mucho ruido la
Menegilda, la Alifonsa y la Narcisa, que con ser sólo tres, no parecía
sino que entraban por las puertas todos los demonios del infierno.

---Tarde venís, ninflas---dijo Mano.

---Sí, hemos estado picando lomo para las salchichas. Como esta tarde no
lo pudimos hacer por ir al rosario\ldots---contestó una de ellas.

---Pos yo, por no perder el rosario, cerré mi almacén de hierro---dijo
otra,---y desde prima noche he tenido que andar desapartando los clavos
de herradura de los clavos de puerta.

---¡Ay qué bueno ha estado el rosario! ¿Lo has visto, Majomilla?

---¡Qué había de ver, si me entretuve en el puente de Toledo, esperando
un cinco de copas que no quería salir, y gancheado a dos payos de
Valmojado que malditos de ellos si sudaban dos cuartos! Pero lo rezaré
mañana, que para el bien nunca es tarde.

---Ende que lo supimos---dijo la Narcisa,---nos plantamos allá. Yo le
mandé al pariente que pusiera el puchero y cuidara de los chicos, y pies
para qué vos quiero. Este rosario lo ha sacado la congregación de María
Santísima del Carmen de la pirroquia de San Ginés, en rogativa de las
presentes calamidades. Salió a las dos. ¡Qué lucimiento, qué devoción!
Allí iban todos, desde el señor más estirado hasta el último comiquín, y
todos con su vela. ¿No ha estado Vd., Mano de Mortero?

---¿Qué había de ir, mujer---respondió,---si estoy aquí con el corazón
traspasado por la pena de no haber metido mi cucharada en ese rosario?
Pero pues mi alma lo necesita, mañana tengo de asistir a la función que
da la cofradía de María Santísima de los Dolores, a quien tengo ley por
los malos pasos de que me ha sacado en bien, intercediendo con su divino
hijo. Creo que predica mi grande amigote el padre Salmón.

---Esa función---añadió Pujitos,---es en el convento de padres
dominicos, y se celebra para implorar el divino auxilio por la felicidad
de las armas de esta monarquía, salud de nuestro S. P. Pío VII y
libertad de nuestro amado Monarca.

---Justo y cabal---prosiguió Mano de Mortero;---y pues hay procesión,
pienso asistir con vela, que todos, el que más y el que menos, estamos
llenos de pecados, y aun yo que no hago mal a nadie, allá me voy con los
demás; porque el justo peca tres veces, cuanti más los que no lo son.
Por lo que a mí hace, no tengo comeniente en que Su Divina Majestad
saque en bien los ejércitos, que españoles somos y lo debemos desear; ni
tampoco en que le dé mucha salud y años mil a ese señor D. Pío VII; pero
en lo de poner en libertad a Fernando, que es como si dijéramos acabarse
la guerra, por allá me lo tenga un par de añitos más.

---Mal patriota es el Sr.~Mano---dijo enfáticamente Pujitos,---pues ni
coge el fusil, ni ruega por la libertad de nuestro amado Monarca.

---Diez fusiles, que no uno cogeré si es preciso, pues hartos agujeros,
raspones y abolladuras hay en los cuerpos de los guardas, que podrán dar
fe de cómo manejo el gatillo. También quiero y reverencio a mi querido
Rey, pues no puedo olvidar que me apretó la mano el día que entró
viniendo de Aranjuez, ni que le alabó a mi Zainilla el garbo para tocar
el pandero, pero los probres somos probres, y yo pondría a mi Fernando
en siete tronos\ldots{} Hijo, dame pan y llámame tonto, y como dijo el
otro, el abad de lo que canta yanta.

---Hoy no vi al señor de Pujitos en la formación---dijo Santorcaz
acercándose al grupo.

---Cómo había de ir, compañero---respondió el maestro de obra prima, que
al oírse interpelado sobre aquel asunto recibió más gusto que si le
regalaran tres tronos europeos.---Cómo había de ir si todo el día he
estado en el parque apartando fusiles, contando piedras de chispa y
repasando cartuchos, tan atareado, jeñores, que tengo en los lomos una
puntada que no me deja respirar.

---¿Y se defenderá Madrid?

---Pues ya. No hay muchos fusiles que digamos; pero se han reunido un
sin fin de sables viejos, muchas lanzas, cascos antiguos del tiempo del
rey que rabió por gachas, cacerolas que pueden servir de escudos, mazas
que para partir cabezas de franceses serán una bendición de Dios,
guanteletes, pinchos, asadores, llaves viejas, y otras mil armas
mortíficas.

---De nada servirá nuestro valor---dijo Santorcaz,---si antes no
acabamos con todos los traidores que hay en Madrid.

---Lo mismo digo---afirmó Mortero.

---Por todas partes no se ven sino espías de los franceses, y ahora es
ocasión de que este señor regidor que aquí tenemos se luzca.

---Así es la verdad---dije yo.---Sé de muchos que se fingen muy
patriotas, y están vendidos a los franceses. Los que hacen más
aspavientos y dan más gritos, y más gallardean de patriotas, son los
peores. ¿No es verdad, Santorcaz?

---Pues acabar con ellos.

---Para eso nos bastamos y nos sobramos---añadió Majoma.---Y vengan
malos patriotas y gabachones para dar cuenta de ellos.

---Personajes conozco yo---dijo Mañara,---que han de morir arrastrados,
si Dios no lo remedia; y si llego a ser regidor, ya nos veremos las
caras, señores afrancesados.

---Esa es la gente más mala---afirmó Santorcaz con mucho
desparpajo,---más desvergonzada y más traidora que hay; y si no ponemos
mano en ellos, no saldremos bien de es ta guerra. Porque yo sé que hay
quien está tramando abrir las puertas de Madrid si nos ponen asedio.

---Pues despacharlos, y se acabó la junción---dijo Pujitos.---En mi
compañía están tan rabiosos, que sólo con decir «ese es gabacho,» se le
van encima y le quieren despedazar.

---Los peores---repetí yo, teniendo el gusto de que el tío Mano apoyara
enérgicamente mi opinión,---son los que chillan y enredan, y están a
todas horas hablando de traidores; y si no aquí está Santorcaz que
conoce a la gente y lo puede decir.

---Así es, en efecto---repuso el franc-masón algo contrariado,---pero
que hay traidores no tiene duda.

\hypertarget{xi}{%
\chapter{XI}\label{xi}}

D. Diego, la Zaina y las otras tres damas, no menos que esta famosas,
habían entablado animada conversación, formando otro corrillo.

---No se olvide el señor condito---dijo Menegilda,---que nos prometió
traer una noche a su novia.

---Si yo no tengo novia.

---Sí que la tiene. ¿No es verdad, Gabriel, que tiene novia?

---Y más bonita que el sol---respondí acercándome.

---Vamos, la tengo---dijo Rumblar,---pero no la quiero, Zainilla. No te
vayas a poner celosa.

---Ya estoy frita con los tales celos, niño mío---contestó la
maja.---¿Pero por qué no la trae aquí una noche?

---Antes traerá una estrella del cielo---afirmó Mañara acercándose al
grupo femenino.

---D. Diego me ha prometido traerla y la traerá---dijo Santorcaz atraído
también por aquel coloquio.

---Sí---indicó Mañara,---la familia de ese señorito iba a permitir que
una tan delicada doncella viniera a estas casas.

---¡A estas casas!---exclamó la Zaina.---¿Estamos en algún presillo? Más
honrada es mi casa, Sr.~D. Juan, que muchas de señoras amadamadas, por
donde usía anda en malos pasos.

---Calla, tonta---dijo Mañara de mal humor.

---Y buenas princesas ha traído Vd. a esta casa, y a la de la Pelumbres
y de la Primorosa---añadió Ignacia.---Toas semos unas, y no lo igo por
esa duquesa con quien fue hace dos noches en ca la Pelumbres. Alifonsa,
¿sabes quién es? ¿Te acuerdas de aquella duquesilla amojamada, que
parece un almacén de huesos? Si D. Juan la trae por aquí, pondremos una
fábrica de botones.

---¿Qué hablas ahí, zafiota, animal sin pluma?---exclamó Mañara con vivo
arrebato de ira.---Habla mejor si no quieres que con tu lengua haga una
pantufla para azotarte la cara.

---¡A mí con esas el asno regidor!---vociferó la Zaina.---Después que le
he despreciao, después que he tenido que escupirle en la cara para que
no anduviera tras de mí chupándose la tierra que yo pisaba, ¿ahora viene
con esa? Con las barbas de un usía friego yo los cacharros de la cocina,
y tripas de caballero le echo a mi gato.

---¡Condenada manola!---dijo Mañara cada vez más encolerizado.---La
culpa tiene quien te ha dado esas alas y quien con personas bajas se
entretiene. ¿Para qué tomas en tu ruin boca el nombre de señoras
respetables de quien no mereces besar la suela del zapato? ¡Cuidado con
los celitos de la niña!

---¿Celos yo?---exclamó la maja más encendida que la grana.---¡Por Dios,
que me quiera Vd., so pringoso: tomelo por estera y se creyó cortejo!

Y diciendo esto, lanzó un salivazo en medio del corrillo.

---¡Miserable mujerzuela! ¡La culpa tiene quien se arrima a ti, por
hacerte gente siquiera un día!

---Eh, eh, poco a poquito---dijo a este punto el tío Mano de Mortero,
que de espectador indiferente de aquella escena se trocaba en actor de
ella.---Eso de mujerzuela es de gente mal hablada, y aquí no se habla
mal de nadie, y lo que es mi hija tiene su siempre y cuando como
cualquier otra. Que el Sr.~D. Juan no nos toque a la honor, porque a mí
no me falta un saco de onzas de oro ensayadas para apedrear a
cualquiera. Y tú, princesa mía, ¿a qué le haces tantos cocos ahora al
Sr.~de Mañara, cuando ha pocos días te chiflabas por él, y si alguna
noche faltaba su señoría a hacerte compañía o a ayudarte a rezar el
rosario, ponías en el cielo unos suspiros como catedrales? Anda, que
todos son buenos, y váyase lo uno por lo otro.

---¿Suspiritos tenemos?---preguntó Mañara con presunción.

---Y si hubo suspiros---dijo Mortero,---mi hija es una persona de
etiqueta, y los puede echar como cualquiera otra, aunque sea por el Rey;
que si está en el cajón de verduras, es porque quiere; que su padre ya
le ha prometido varias veces ponerla al frente de una casa de bebidas
finas.

---¡Yo suspirar por ese animal!---dijo la Zaina.---Por lástima le he
mirao una vez cuando iba al cajón a echarme flores.

---Eso quisieras tú; pero no se estila echar margaritas a puercos.

La Zaina hizo un movimiento. El demonio fue sin duda quien llevó a sus
irritadas manos una botella de las que en la mesa contigua había, y
disparola con tanta fuerza contra Mañara, que a no apartarse este
vivamente, viéramos allí partida en dos la cabeza más dura que ha
gastado regidor en el mundo. Levantose este furioso para castigar el
descomedimiento de la Zaina; pero con tanta presteza acudió D. Diego en
defensa de la verdulera, que sobre él cayeron los primeros golpes. Lleno
de rabia al verse aporreado, arremetió contra Mañara, a punto que el tío
Mano de Mortero empezaba a probar la exactitud de su apodo, repartiendo
algunos puñetazos sobre tirios y troyanos. Las majas Narcisa, Menegilda
y Alifonsa, declaráronse también en guerra, por dar gusto a las
inquietas manos, y bien pronto de todos los allí presentes no quedó uno
que no llevase su óbolo a tal colecta de golpes y gritos. Era aquello
una bendición de Dios, y juro que jamás habría yo metido mis manos en
tal fregado, si no me incitara a ello una caricia que sentí en mitad de
la espalda, hecha por mano desconocida. Y lo peor fue que Majoma, hombre
ingenioso, inclinado siempre a sacar partido de tales alteraciones del
orden privado, descargó varios palos sobre el candil que la escena
iluminaba, y al punto nos vimos todos de un color. Aquí fue el arreciar
de los puñetazos, y el esfuerzo de los gritos y el rodar unos sobre
otros, y si bien el peso de un cuerpo nos oprimía a veces, también el
nuestro caía en humanas blanduras, de cuyos choques provenían los
pellizcos, arañazos y demás proyectiles menudos. Por aquí se oían voces
lastimeras, por allá gritos de venganza, y sobre toda especie de
rumores, descollaba la voz estentórea del tío Mano de Mortero, diciendo:

---En mi casa no ha de haber escándalos, y el que diga que aquí se
siente el vuelo de una mosca, miente. Vamos, amiguitos; no meter tanto
ruido ni pegar tan recio. Esto es una broma: conque paz y pan, y
divirtámonos.

Y a todas estas la vecindad se alborotaba, y en la calle deteníase la
gente curiosa, no porque le hiciera novedad aquel ruido, sino por gozar
de él, y se temió la intervención de la justicia, lo cual hería al
Sr.~Mano en lo más delicado de su dignidad, y por fin hubo uno que pudo
dar con la puerta y abrirla y echarse fuera, con lo cual, habiendo
entrado un poco de luz, pudimos vernos. Todo indicaba que íbamos a tener
una visita alguacilesca, lo que me impulsó a coger por un brazo a D.
Diego y echarlo conmigo afuera, y bajar a saltos la escalera hasta dar
con nuestros cuerpos en la calle, por la que nos escurrimos, sin miedo a
la corchetería.

Cuando nos vimos lejos, acortamos el paso, contemplándonos uno a otro.
D. Diego había padecido más averías que yo en la refriega, y ostentaba
en la cara un verdugón hecho por buena mano.

---¡Maldito de mí!---exclamó tentándose los bolsillos de sus
calzones.---¿Sabes que me han quitado mis dos relojes? ¡Pues también el
dinero, todo el dinero que llevaba!

---Era de suponer, Sr.~D. Diego---le respondí registrándome
también,---pues no salimos de ninguna misa cantada. Y por lo que veo, a
mí también me han desplumado.

---¿Te quitaron el reloj?

---No señor, el reloj no me lo han quitado ni me lo quitarán todos los
cacos del mundo, porque no lo tengo; pero sí perdí un dinerillo\ldots{}
bien poco, por cierto.

---¡Dios mío! Sin relojes, sin dinero\ldots---clamó doloridamente D.
Diego.---¿Con qué compraré ahora las diez y siete varas de cotonía que
quiere la Zaina? ¿Con qué alquilaré el coche para que vaya el lunes a
los novillos? Si Santorcaz no me presta, me moriré.

---Diez y siete varas de fresno, que no de cotonía, es lo que merece esa
gentuza ---le contesté;---pues es necesario estar loco o enamorado para
poner los pies en tales casas.

\hypertarget{xii}{%
\chapter{XII}\label{xii}}

Como antes indiqué, no pude obtener licencia para salir de Madrid,
porque la villa, viéndose pronto en gran aprieto, cayó en la cuenta de
que necesitaba de toda su gente para defenderse. ¿Por qué no me marché?
¿Quién me lo impidió? ¿Quién torció el camino de mi resolución? ¿Quién
había de ser, sino aquel que por entonces era el trastornador de todos
los proyectos, el brazo izquierdo del destino, el que a los grandes y a
los pequeños extendía el influjo de su invasora voluntad? Sí: el
baratero de Europa, el destronador de los Borbones y fabricante de
reinos nuevos, el que tenía sofocada a Inglaterra, y suspensa a la
Rusia, y abatida a la Prusia, y amedrentada al Austria, y oprimida a la
hermosa Italia, osó también poner la mano en mi suerte, impidiéndome
pasar a otro ejército.

Es, pues, el caso, que el D. Quijote imperial y real, como algunos de
nuestros paisanos le llamaban, no sin fundamento, había entrado en
España a principios de Noviembre, con ánimos de instalar de nuevo en
Madrid la botellesca corte. A él se le importaba poco que los españoles
llamasen tuerto a su hermano; y fijo en el número y fuerza de nuestros
soldados, no atendía a lo demás. Una vez puesto el pie en tierra de
España, no le agradó mucho que el mariscal Lefebvre ganase la batalla de
Zornosa, porque sabido es que no era de su gusto que se adquiriese
gloria sin su presencia y consentimiento. Mandó, sin embargo, al
mariscal Víctor que persiguiese a nuestro degraciado Blake, cuyas tropas
se habían reforzado con las del marqués de la Romana, escapadas de
Dinamarca, y aquí tienen Vds. la batalla de Espinosa de los Monteros,
dada en los días 10 y 11, y perdida por nosotros, por más que el Gran
Capitán, con más celo que buen sentido, se empeñe en negarlo. ¡Ay!
Valientes oficiales perecieron en ella, y grandes apuros y privaciones
pasaron todos, sin un pedazo de pan que llevar a la boca, ni una venda
que poner en sus heridas.

Así sucumbió el ejército de la izquierda, cuyos restos salvándose por
las fragosidades de Liébana, recalaron por tierra de Campos, para ser
mandados por el marqués de la Romana. No fue más dichoso el ejército de
Extremadura en Gamonal cerca de Burgos, pues Bessieres y Lasalle lo
destrozaron también el mismo fatal día 10 de Noviembre, y el 12 entraba
en la capital de Castilla el azote del mundo, publicando allí su traidor
decreto de amnistía. Aún nos quedaba un ejército, el del Centro, que
ocupaba la ribera del Ebro por Tudela: mandábalo Castaños; pero nadie
confiaba que allí fuéramos más afortunados, porque una vez abierta la
puerta a las calamidades, estas habían de venir unas tras otras a toda
prisa, como suele suceder siempre en el pícaro mundo. También nos
preparaba el cielo en el Ebro otra gran desgracia; pero a mediados de
Noviembre, cuando corrieron por Madrid las tristes nuevas de Espinosa y
de Gamonal, aún no se había dado la batalla de Tudela.

El pánico en Madrid era inmenso, y se creía segura la pronta
presentación del corso en las inmediaciones de la capital. ¿Qué podía
oponérsele? No quedaba más ejército que el del Centro, situado allá
arriba a orillas del Ebro. ¿Quién detendría al invasor en su marcha
terrible? La Junta se desesperaba y los madrileños creían acudir a
remediar la gravedad de las circunstancias, entusiasmándose. ¡Ay!
Después de mandar algunas tropas a los pasos de Somosierra y
Navacerrada, ¿qué ejército de línea quedaba para defender a Madrid? Da
pena el decirlo. Quinientos soldados.

Los paisanos armados eran ciertamente muchos; pero había muy pocos
fusiles, y de estos la mitad eran inútiles por falta de cartuchos; y,
¿con qué se hacían los cartuchos si no había pólvora? A esto habíamos
llegado cuatro meses después de la victoria de Bailén. Todo al revés.
Ayer barriendo a los franceses, y hoy dejándonos barrer; ayer poderosos
y temibles, hoy impotentes y desbandados. Contrastes y antítesis y
viceversas, propias de la tierra, como el paño pardo, los garbanzos, el
buen vino y el buen humor. ¡Oh España, cómo se te reconoce en cualquier
parte de tu historia adonde se fije la vista! Y no hay disimulo que te
encubra, ni máscara que te oculte, ni afeite que te desfigure, porque a
donde quiera que aparezcas, allí se te conoce desde cien leguas con tu
media cara de fiesta, y la otra media de miseria, con la una mano
empuñando laureles, y con la otra rascándote tu lepra.

---Hola, Gabriel, ¿tú por aquí?---me dijo Pujitos en la puerta del Sol
el día 20 de Noviembre.---Ya sabes que tenemos de regidor a nuestro
amigo D. Juan de Mañara. Él es el encargado de la cartuchería. ¿Tienes
fusil?

---Y bueno. ¿Pero todavía no se dice nada de fortificar a Madrid, ni se
trata de abrir fosos y levantar parapetos y abrigos, ya que a esta villa
y corte la hicieron sin murallas ni otra defensa alguna?

---Todo se va a hacer. Pero lo que más falta hace es la cartuchería y
armas.

---¿Dónde hacen cartuchos?

---En varias partes. Allá junto al colegio de Niñas de la Paz hay más de
sesenta personas trabajando en ello noche y día.

---Pero de nada nos sirven los cartuchos sin armas, Sr.~de Pujitos---le
dije.---Yo conozco muchísimos hombres valientes que no tienen sino
chuzos, pedreñales y espadas llenas de orín.

---Eso será nonada, y si no nos hacen traición\ldots{}

---¡Traición!

---Sí; aquí hay muchos traidores.

---Ahora como la gente anda tan exaltada, es común llamar traidores a
los más mejores patriotas.

---Gabriel---dijo deteniéndose en medio de la calle y asomando por el
embozo de su capa un dedo con el cual ciceronianamente acentuaba sus
palabras,---cuando yo lo digo, sabido me lo tengo. ¿Te acuerdas de lo
que se habló hace noches en casa del tío Mano? ¿Te acuerdas cómo se puso
furioso el Sr.~de Santorcaz contra los traidores? Pues hemos descubierto
que ese Sr.~de Santorcaz o D. Demonio, es espía del córcego. Velay por
qué estaba tan enfoguetado.

---No es la primera vez que lo oigo.

---Él les escribe cartas de lo que aquí pasa, y con el dinero que le dan
paga gente alborotadora, que arme querellas entre la tropa. Como este
hay muchos, y se dice que señores muy alcurniados están vendidos a los
franceses. Pero, Gabriel, que se nos amostacen las narices, y veremos a
dónde van a parar. Hay otros que aunque no son traidores, son
melindrosos, y no quieren lo que llaman Constitución, la cual se va a
poner ahora pa acabar con el espotismo. ¿Sabes tú lo que es el
espotismo? Pues el espotismo es una cosa muy mala, muy mala. A bien que
desde que acabamos con Godoy y los lairones que con él vivían, se
acabaron todas las picardías, y ahora luego que demos fin a esto del
córcego, los reinos de España se van a gobernar de otra manera, y
estaremos tan bien, que no nos cambiaremos por los ángeles del cielo.

Y diciendo esto, dio media vuelta y marchose lejos de mí a toda prisa.
No tardé yo en acudir pronto a la formación de mi compañía.

Ante las evidentes muestras de alarma que a todas horas se observaban en
Madrid, mal podía el optimismo del Gran Capitán sostenerse en las
ideales regiones donde le hemos visto cernerse, como el águila de la
patria a quien ni el peligro ni el miedo pueden obligar a abatir su
majestuoso vuelo. Ya no era posible negar la derrota de Espinosa, ni
tampoco la de Gamonal, y sólo los locos podrían suponer a Napoleón
dispuesto a detenerse en su victorioso camino. Muchos días resistiose el
fuerte espíritu de mi amigo a la evidencia de tantos descalabros; por
muchos días sostuvo que nuestras armas victoriosas echarían a los
franceses con su malhadado emperador del otro lado del Bidasoa; por
muchos días continuó atribuyendo a los papeles públicos la pérfida
invención de aquellos absurdos acontecimientos que no cabían en su
homérica cabeza; pero al fin la muchedumbre de las noticias malas, la
agitación pública, el pánico de todos, la general zozobra, y el tumulto
y laberinto de los preparativos de defensa rindieron golpe tras golpe el
formidable castillo de su terquedad, dando en tierra con tantas
ilusiones. El héroe no aparentó desmayar con esto, antes bien se reía
tomando la cosa como una fiesta. Lleno de confianza en la capital,
siempre negaba que Napoleón se atreviese a ponerse delante de los
madrileños, y esta fue una tenacidad que le duró contra viento y marea
hasta el 25 de Noviembre, en cuya noche al retirarse a su casa,
preguntole doña Gregoria, como siempre, las noticias de la tarde.

---Nada, mujer---repuso frotándose las manos, y promulgando con
desdeñosas sonrisas la categórica confianza que llenaba su
espíritu.---Nada, mujer: emperadorcito tenemos.

\hypertarget{xiii}{%
\chapter{XIII}\label{xiii}}

Y el emperadorcito salió de Burgos el 22; detúvose en Aranda el 24; el
29 estaba en Boceguillas, y por fin el 30 llegó a Somosierra.

En Madrid la alarma crecía en tales términos, que ya en 23 de Noviembre
se pensaba en una defensa formal, guarneciendo el circuito de la corte
para hacer de ella con el valor de sus habitantes una segunda Zaragoza.
Era capitán general de Castilla la Nueva el marqués de Castelar, y
gobernador de la plaza don Fernando de la Vera y Pantoja; pero a este no
se le conceptuaba muy entendido en materias facultativas, y como se
tratara de obras de defensa, fue nombrado para el caso el célebre don
Tomás de Morla, sucesor de Solano en Cádiz cinco meses antes; hombre
feísimo de rostro, de carácter aparentemente enérgico aunque en realidad
muy débil. Gozaba en el conocimiento de la artillería de gran
reputación, que aún conserva, pues sus estudios sirven hoy para la
enseñanza de la juventud que a la guerra científica se consagra.

Morla dirigió las obras de defensa, que consistían en grandes fosos
abiertos fuera de las puertas de Fuencarral, Santa Bárbara, Los Pozos,
Atocha y Recoletos; en aspillerar toda la muralla de la parte Norte; en
desempedrar las calles de Alcalá, Carrera de San Jerónimo y calle de
Atocha para levantar barricadas; y por último, en fortificar el Retiro
con trincheras y una mediana artillería, la única que teníamos, pues
todo se reducía a unas cuantas piezas de a 6 y poquísimas de a 8. Esto
se hizo precipitadamente a última hora; mas con tanto entusiasmo y
determinación, que la diligencia parecía suplir con creces a la
previsión.

En las obras trabajaba todo el mundo sin reparos de clase. Las señoras,
no contentas con afiliarse en la congregación del \emph{Lavado y
cosido}, dirigieron a las autoridades una exposición en que se ofrecían
a ayudar \emph{ya llevando espuertas de tierra}, ya ocupándose en lo que
se les mandase. No es esto invento mío, y la exposición existe impresa
donde el incrédulo podrá verla si aún duda de la grandeza de ánimo de
las señoras de aquel tiempo. Y al decir \emph{señoras}, se comprende que
no me refiero a aquellas de quienes en otro lugar de este relato tengo
hecha mención, pues las del Rastro y Maravillas tenían especial gusto en
pasearse por todo Madrid arrastrando un cañón entre seguidillas y
chanzonetas: me refiero a las más altas hembras, a quienes vi empleadas
en menesteres indignos de sus delicadas manos.

De los hombres no hay que hablar, porque todos trabajábamos a porfía día
y noche sacando tierra de los fosos para construir los espaldones de la
artillería. En poco tiempo quedó la calle de Alcalá tan limpia de
guijarros como tierra de sembradura, y desde las Baronesas al Carmen
Calzado levantamos un parapeto formidable.

El personal de la defensa era el siguiente:

1.º Quinientos soldados de línea que apenas bastaban para el servicio de
las bocas de fuego. 2.º Las tropas colecticias formadas por el
alistamiento voluntario de 7 de Agosto, y a las cuales pertenecía un
servidor de Vds. (no pasábamos de tres mil hombres). 3.º Los conscriptos
pertenecientes a Madrid en el llamamiento de doscientos cincuenta mil
hombres que hizo la Junta, y cuyo sorteo se verificó en 23 de Noviembre.
4.º La milicia urbana llamada \emph{honrada} que se formó por enganche
voluntario el 24 del mismo mes.

Voy a deciros algo de esta conscripción y de estos señores
\emph{honrados}. Hízose aquella llamando a las armas a todos los
ciudadanos desde 16 a 40 años, y declarando derogadas todas las
excepciones que establecían las Reales Ordenanzas de 27 de Octubre de
1800 para el reemplazo del ejército. Se declararon útiles los viudos con
hijos, los hijosdalgo de Madrid, los nobles que no tuvieran más
excepción que su nobleza, los tonsurados sin beneficio que estuviesen
asignados a servicio eclesiástico, para cuya determinación se cubrió con
un velo el concilio de Trento; los que disfrutaban capellanía sin estar
ordenados \emph{in sacris} (muchos de estos eran los llamados
\emph{abates}); los novicios de órdenes religiosas; los doctores y
licenciados, que no fueran catedráticos con propiedad; los retirados del
servicio y los quintos que hubieran servido su tiempo; los hijos únicos
de labradores; en una palabra, no se exceptuaba a rey ni a Roque.

Los \emph{honrados} eran una milicia sedentaria creada con objeto de
guarnecer las ciudades, para \emph{precaver los desórdenes, reprimir los
facinerosos, bandidos, desertores y díscolos, que perturbando la pública
tranquilidad intenten saciar su ambición o su codicia}.

De modo que en Madrid tuvimos en 23 de Noviembre sorteo para el
reemplazo del ejército, y algunos días después alistamiento de
\emph{milicianos honrados}. Aquella y esta operación se verificaban de
diez a tres en los claustros de la Trinidad Calzada, de los Mostenses,
de San Francisco, y en los de otros conventos situados en el punto más
céntrico de cada cuartel, ante un alcalde de Casa y Corte o un señor
regidor de Madrid, un oficial militar, un alcalde de barrio y un
escribano. Bastaron, pues, pocos días para que las filas de la
guarnición de Madrid se llenaran con muchos miles de hombres. A la poca
tropa de línea y al regular número de voluntarios ya disciplinados,
uniose la muchedumbre de quintos y la caterva de urbanos, gente toda muy
entusiasta; pero casi en general carecían de fusiles y estaban tan
ignorantes de lo que habían de hacer como la madre que les echó al
mundo.

Sucedió también que los voluntarios antiguos, aquellos que desde Agosto
habían paseado presuntuosamente sus fachas uniformadas por Madrid,
miraron con mal ojo a los \emph{honrados}, los cuales, llamándose así,
parecían querer resumir en su instituto toda la honradez española, y
hablaban pestes de los antiguos. Los \emph{honrados} que no tenían
armas, decían que estas debían quitarse a los antiguos que las tenían:
juraban estos entregarlas antes a Napoleón que a los \emph{honrados}, y
en tanto los quintos recién sorteados, aquellos infelices viudos,
nobles, sacristanes, novicios, beneficiados sin beneficio y demás gente
antes exceptuada, miraban al cielo, esperando que se les pusiese en la
mano alguna cosa con que matar. En resumen: mucha, muchísima gente de
última hora; pocas y malas armas; ningún concierto, falta de quien
supiese mandar aunque fuese un hato de pavos; mucho mover de lenguas y
de piernas; un continuo ir y venir, con la añadidura inseparable de
gritos, amenazas y recelos mutuos, y la contera de los gallardetes,
escarapelas, banderolas, signos, letreros y emblemas, que tanto emboban
al pueblo de Madrid.

El aspecto de uno de aquellos claustros en que se verificaba el
alistamiento, era digno de ser eternizado por los más diestros pinceles.
Dichoso yo si con la pluma pudiera dar efímera existencia a uno de ellos
¿A cuál? Todos eran igualmente pintorescos, y si alguno contenía mayor
número de curiosidades, era el claustro de la Trinidad Calzada, en la
calle de Atocha.

En mitad de la ancha crujía estaba la mesa donde el regidor iba
recibiendo los nombres, que asentaba un escribiente en barbudas
cuartillas de papel. En su derredor resonaba tal chillería y alboroto,
que no sé cómo el señor de Mañara (que era el regidor allí presente)
podía aguantarlo; pero inútil era el imponer silencio, porque la
multitud de mujeres aglomeradas a la puerta, no callarían aunque el
Espíritu Santo se lo mandara. Un pobre alguacil había sido destinado a
sostener la debida compostura, y nunca tal hubiera intentado el infeliz
instrumento de la justicia, porque le cogieron y le magullaron, y roto y
molido dio vueltas por el arroyo.

---¿Pero qué buscan Vds. aquí?---exclamó Pujitos abriendo los brazos en
actitud amenazadora.---Fuera mujeres, que no sirven sino de estorbo.
Condenaas, ¿por qué no van a sacar tierra en los Pozos?

---Ya hemos sacado tierra, ¡y lástima que no fuera de tu sepultura!

---¿Pues qué queréis, demonios?

---¿Qué hamos de querer? ¡Fusiles, piojo! ¿Te los han dado a ti y a tu
batallón pa quitar telarañas? Vengan acá pronto, que nosotras también
nos alistamos.

---Afuera, afuera de aquí, canalla.

---Paz, paz---dijo desde el interior del claustro una gruesa y campanuda
voz que al punto reconocí por la del venerable Salmón.---Haya paz, y no
me levante ninguna el gallo.

Al punto el apretado grupo de mujeres se dividió en dos, dando paso a la
procerosa figura del mercenario, que avanzó con majestuoso paso y
risueño continente.

---Aquí está el padrito. ¡Que viva el padre Salmón! Ven, Pujitos del
demonio, a echarnos afuera.

---Arrastrao---dijo una cogiendo a Pujitos por el cuello y mostrándole
el puño.---¿Tus muelas han salido a misa esta mañana? ¿Quieres que
salgan a vísperas esta tarde? Pues boquea y verás.

---Déjenlo, dejen en paz a ese pobre hombre---dijo socarronamente
Salmón,---y perdónenle su gran descortesía con tan dignas señoras; que
yo prometo que se enmendará. Ya os he dicho varias veces que si no sois
buenas, no contéis para nada con vuestro queridito padre Salmón. Vamos a
ver, señoras mías; duquesas y princesas; ¿para qué os agolpáis aquí?

---También nosotras queremos alistarnos.

---Alistaros, ¡oh valientes amazonas! Pero niñas, ¿no veis que en
vuestras manos mejor sienta el hilo de oro y las sartas de perlas, que
el temido alfanje damasquino? Vaya, idos a rezar, que la mujer honrada
la pierna quebrada y en casa.

---Todos esos son unos calzonazos. Nosotras hemos cargado ya muchas
espuertas de tierra. Ahora llevamos dos cañones a Los Pozos, y queremos
que nos los dejen disparar.

---Bueno, bueno, todo se hará. Cada una a su casa, y cuidado con lo que
les tengo prevenido. Tú, Nicolasa, eres una tramposa, que en cada libra
de carne pones dos onzas menos de peso. Tú, Bastiana, te condenarás por
la usura de prestar a dos pesetas por duro a la gente del Rastro; y tú,
Alifonsa, aguardentera de todos los diablos, ten entendido que tantas
docenas de estos verás a la hora de tu muerte como cortejos has
mantenido en vida, y no digo más por no escandalizar delante del
público.

Con estas y otras filípicas iba Salmón despejando la puerta, en tales
términos, que pronto quedó practicable; mas no por eso tornose adentro
el popular fraile, sino que siguió adelante, diciendo a cada uno su
palabrita y dando a besar la correa a viejos, mujeres, hombres y
muchachos. Cuando me vio echome los brazos al cuello, saludándome con
mucho afecto.

---¿Vienes a alistarte?---me dijo.

En esto abalanzose hacia nosotros un hombre que besó las manos a Salmón
con fervoroso cariño, y luego le habló así:

---¡Ay mi padrito de mi alma! ¡Gracias a Dios que este probe tiene el
refrigerio de encontrarle y verle y hablarle, que es para él de más
gusto que si le dieran todos los reinos del mundo limpios de fronteras!
¿Recibió Su Paternidad las siete libras de rapé y el barrilito?

---Si, hijo mío, y gracias se os dan, pues sois el caballero más
cumplidor de juramentos y palabras que conozco.

---Sí: que soy hombre para desairar a un Paternidad tan reverendo. Mande
mi frailito por esa boca, que yo le traeré la Inglaterra toda, aunque
gaste en pólvora y balas todo mi dinero.

---¿Y la Zainilla?

---¡Está malucha! La otra noche tuvimos junción en casa, y todo concluyó
con un sainetillo de lo que llaman palos, que aquello parecía una
gloria. La pobrecita niña de mis entrañas está desde esa noche que no
come ni bebe, y manda al cielo unos suspiros que parten el corazón de
bronce de su padre.

---Eres un zopenco, tío Mano---dijo Salmón.---Cuando estuve en tu casa
el día de Difuntos\ldots{} ¿recuerdas que me diste aquellos puches; que
con el aditamento de un cierto aguardiente de Chinchón, estaban propios
para que metiera en ellos las barbas el mismo emperador del Sacro Romano
Imperio?

---Me acuerdo, sí.

---Pues aquella noche te dije: «Morterillo, ándate con cuidado, que tu
Zaina y el Sr.~de Mañara están de mucho paliqueo, y míralos en aquel
rincón con la cabeza inclinada el uno sobre el otro como dos higos
maduros.» ¡Y cómo se le caía la baba a tu hija!

---Verdad es, señor; y ya sé que de ahí viene todo.

---Entonces te dije: «Morterillo, mucho ojo, que el Mañara quiere
enmarañar a tu hija, y vas a perder este bocadito de ángeles que tú
destinabas a un Veinticuatro.» ¿Acerté?

---¿Pues ello?\ldots{} Yo no quería reñir con Mañara---dijo Mortero
rascándose una oreja.---Verdad que él iba allá todas las noches\ldots{}
pero mi pobrecita niña es más inocente que una paloma.

---Apuesto a que el demonio ha metido el rabo en tu casa, Morterillo.
Dices que tu hija ni come ni bebe, y da unos suspiros\ldots{}
¿suspiritos?

---Sí; y en tres días no le he podido sacar palabra de la boca, y a
veces heme puesto a acecharla tras la puerta de su cuarto, y cata a mi
niñita diciendo unas palabrotas\ldots{} pues\ldots{} así como los
cómicos en los treatos\ldots{} Y a ratos la veía enjugándose las
lágrimas, y a ratos echando centellas por los ojos\ldots{} «Dime qué
tienes, serafín de tu padre,» le he preguntado algunas veces; pero no me
contesta más que un poste. Anoche nos pusimos a rezar el rosario (porque
yo no falto jamás amén a esta devota costumbre ni en casa, ni en campo
raso), y ella empezó con mucha devoción, diciendo los santamarías con un
dejo y un canticio meloso que llegaba al alma; pero de repente, padrito,
empieza a dar manotadas como una loca, rompe en mil pedazos el rosario,
levántase, y con las manos en la cabeza, dando paseos por el cuarto,
dice así: «Virgen de la Paloma, no puedo, no puedo.» Luego púsose el
mantón y corrió a la calle, adonde la seguí\ldots{} ¿Creerá Su
Reverencia que fue hasta la casa donde vive ese condenado regidor,
parose en la puerta, y arrimando la cabeza contra una reja, dio a llorar
como un chiquillo? Tuve que traerla en brazos a mi casa, y al día
siguiente no pudo ir al cajón porque cayó mala.

---Ya lo veo clarito. Es que Mañara le tiene sorbidos los sesos, y no es
la primera, Mortero, no es la primera; pero yo iré por allí, echarele un
sermón a la niña, y veremos si te la curo\ldots{} Pero calle\ldots{} ¿No
es aquella que asoma por allí? Sí, es ella misma. Zaina, Zainilla, ven
acá.

---Sí, es mi flor temprana, es el lucero de su padre. Llégate aquí,
arrastradilla ---dijo el tío Mano llamando a su hija.---¿De dónde
vienes?

---De llevar tierra---contestó la Zaina, en cuyo semblante fresco y
animado no se veían señales de aquel hondo pesar que acababa de referir
el respetable progenitor.---Ya hemos puesto tres cañones en la puerta de
Atocha, y están clavadas las estacas y armado tal ramaje de palitroques,
que parece un nacimiento.

---¿Y para qué andas tú en esas faenas, solito de justicia? Padre,
échele Su Reverencia un buen sermón, o dos, si es menester, para que se
quede en casa.

---Tú no tienes buena cara, Zaina---le dijo Salmón.---Tú estás triste,
te lo conozco.

---¡Qué buen barruntador tenemos! ¿Y por qué estoy triste?

---Dime, ¿has visto por ahí al Sr.~D. Juan de Mañara?

La Zaina se puso pálida y cesó de reír.

---Ya está cogida---exclamó Salmón batiendo palmas.---Esa cara no
miente. Mira, Ignacia, en la huerta de mi convento hay un pajarito que
todas las mañanas viene a mi celda a contarme las picardías de las
muchachas que conozco. ¿Sabes lo que me dijo de ti? Pues me dijo\ldots{}

---Está más encarnada que un tomate---añadió Mano;---déjela Su
Paternidad por ahora.

---¿Qué dejar? ¡Bueno soy yo!\ldots{} Conque, niña, ¿ha habido
gatuperio? Mucho cuidado con los galanes que van a casa, mucho ojo, que
si me enfado\ldots{} Fuera pecados mortales, fuera cosas malas, que
entonces no hay lo de padrito por acá, padrito por allá, sino que saco
unas disciplinas y a zurriagazos enderezo yo a mis niñas. Conque ven
acá, loquilla, ¿ese señor de Mañara te ha trastornado el juicio?

---¿A mí?---chilló la Zaina con súbita expresión de despecho que la puso
más arrogante y más hermosa de lo que realmente era.---¿A mí ese pelón?
Sé que se lustrea diciéndolo por ahí; pero que se aspere un poquito, que
astavía tengo mucho orgullo y no me echo a perros.

---Vamos, no lo niegues.

---¿Yo? Voyme al zumo, que no a las cáscaras, y sobre que no me gustan
los usías estirados, ni los madamos que huelen a bergamota, cuanti más
los malinos traidores, gabachones\ldots{}

---¡El Sr.~de Mañara traidor!---exclamó con asombro el
mercenario---¿Cómo hablas así de un caballero tan principal y tan buen
patricio, de ese bendito regidor, que ahora está allí dentro alistando
soldados?

---Traidor, más traidor que Judas---afirmó la Zaina.---¿Y Su Reverencia
se hace de nuevas? Pues todo el mundo lo dice, y no queda en Madrid
quien no lo sabe.

---De otros lo he oído yo, pero no de Mañara---indicó Mortero.

---Está vendido a los franceses, y todo ese papel que hace, es por
disimular sus maldades---dijo la Zaina.---Pero se la tienen sentenciada
a ese pícaro, arrastrao, endino, criado del tío Copas. ¡Viva Fernando
VII!

---Yo creí que estabas embobada---dijo Salmón,---y ahora veo que estás
loca.

---¡Ay mi niñita!---dijo el tío Mano;---no hables tales cosas, que
pueden llegar a las orejas del Sr.~de Mañara, y ya sabes que ando en
empeños con él para que ponga en libertad a aquellos dos angelitos
seráficos que están en la cárcel de Villa, Agustinillo y el Manco, los
cuales por diez pellejos de mal vino de Esquivias, están pasando el
purgatorio en vida, aunque pienso que en la otra Dios les ha de
descontar estas penas.

---¡Me han de oír los sordos!---exclamó la Zaina,---que aquí no queremos
traidores. ¡Acabar con ellos, y Napoleón es muerto!

---Cuidado, muchacha---dijo Salmón,---que palabra y piedra suelta no
tienen vuelta, y palabra en boca es lo mismo que piedra en honda.

---Sea lo que Dios quiera. A mí quien me la hace me la paga.

---¿Ves cómo todo es el rencorcillo que te ha quedado?

Iba a contestar Ignacia, cuando apareció D. Diego, y luego que aquella
le vio, hízole entrar en el corro, diciéndole:

---Aquí estoy, aquí está su princesa, señor conde; no me busque con esos
ojazos de pájaro bobo.

---¿También el señor conde te corteja, harpihuela?---preguntó el fraile
haciendo una reverencia a D. Diego.

---¡Y que le quiero más que a las niñas de mis ojos!---dijo la
maja.---Los zarcillos son chicos, y otra vez tenga más miramiento; que a
las señoras no se las obsequia con colgajitos de a cuatro duros; y un
novio tuve yo, que en barras de plata y oro me llevó a casa los tesoros
del Rey.

D. Diego turbado por la presencia del mercenario, no acertaba a decir
palabra. En cambio el padrito se encaró con él, y campanudamente
endilgole la siguiente homilía:

---Ya sé que anda el señor conde en malos pasos, y mis señoras la
condesa y marquesa lo saben también. ¿Conque es cortejo de la Zaina?
\emph{¡Optime, superlative!,} Sr.~D. Diego. Y no lo digo porque esta sea
ningún guiñapo, sino porque cada oveja con su pareja. ¡Qué dirá la
señora doña María Castro de Oro, condesa de Rumblar, a quien no conozco
sino para servirla; qué dirá cuando sepa los traeres de su hijo! Y
pensar que a un jovenzuelo casquivano se le ha de dar por esposa aquella
flor sin tacha, aquel lucero matutino, que cual oro en paño guardan
donde usía sabe, es pensar en las nubes de antaño. Pues no faltaba
más\ldots{} ¡Un Afán de Ribera, metido en tales tapujos! ¿No le da a Vd.
vergüenza? Y no lo digo porque recuente la casa de este Sr.~D. Mano de
Mortero, que es persona honradísima, sino porque mi niño va también a
casa de la Zancuda, donde se juega de lo lindo, y jóvenes muy acomodados
conozco que han dejado allí los hígados.

---Verdad es---dijo Mortero.---Lo que es en mi casa, nadie se deja nada,
como no sea el malhumor, porque a conversaciones honestas, y a lenguas
castas, y a manos quietas nadie nos gana; que a veces la casa parece un
monasterio de tanto afinamiento y quinta sustancia de la conmenencia.

---Pero el Sr.~D. Diego no sólo frecuenta esas deshonestísimas
regiones---añadió Salmón,---sino que también va a las logias de los
masones, \emph{infernalis espelunca}, donde se pasa la noche entre
herejías y diabluras. ¡Veo que es aprovechado el rapazuelo! ¡Y quería la
señora marquesa que yo le trajese al buen caminito con sermones y
consejos! No está la Magdalena para tafetanes, Sr. D. Diego, y yo
primero arrojo el hábito que llevo, que decir a usía por ahí te pudras,
y lléveselo el diablo con sus bobadas y truhanerías.

Más que una mona corrido, quedose D. Diego con esta filípica, y de buena
gana habría contestado a Salmón, vomitando todas las abominaciones que
acerca de los frailes había aprendido ya, si no le detuviera la
vergüenza y las muchas miradas de enojo que de distintas partes le
observaban. Así es que sólo protestando a medias palabras contra el
\emph{frailazo pancista}, se escurrió bonitamente entre el gentío,
llevando consigo a la Zaina y a Mortero, que no quiso dejarle escapar
sin previa entrega de las ofrecidas espuelas de plata.

Quedámonos allí Salmón y yo, y como mi amigo oyera lo de \emph{frailazo
pancista}, palabras que ya en aquellos días empezaban a menudear en
bocas populares, se enfureció y quiso seguir tras el jovenzuelo para
reprenderle su osadía; mas el agolpamiento de la gente, junto con las
muestras de simpatías que recibió, se lo impidieron.

---Temple Su Paternidad la ira---le dije,---y vayase en buen hora D.
Diego.

---Tienes razón---repuso,---que \emph{aquila non capit muscas}. Su
castigo tendrá en ver que se queda sin novia.

---Pues él está tan firme en casarse---dije,---que lo da por hecho, y
añade que llevará adelante lo del matrimonio, contra viento y marea.

---¡Oh, qué ilusión! ¡Pues están contentas de él mis señoras la condesa
y marquesa! Y por lo que hace a la novia\ldots{} Acompáñame a la Merced
y te contaré. ¿Hablaste largo con la señora condesa? ¿Le dijiste todo lo
que sabes de este botarate?

---Un poquito, sí señor. ¿De modo que no se casará?

---Lo dudo, porque si las personas mayores de la casa no lo pueden ver,
lo que es la joven\ldots{} Anda esta trastornadilla después que se le
han descubierto todos los escondrijos de su almita. Por fin lo dijo
todo. Ya te conté que ni yo con mi gran autoridad y mis chistes y
juegos, ni la marquesa con su mal genio, ni el marqués apedreándola a
regalos y obsequios, pudimos hacerle confesar la causa de sus
melancolías; pero al fin, apretada por su prima la señora condesa que la
ama mucho, un día entre lágrimas y suspiros le confesó todo.

---Y no resultaría nada\ldots{}

---Nada más sino que todo aquel mal gesto y aquellas tristezas le venían
de amar a un muchachuelo, a un perdidillo, a un cascaciruelas de esas
calles, a quien conoció y tuvo por novio en toda regla, allá cuando
vivía lejos de sus padres. ¡Cosa de niños! Lejos de parecerme mala, me
parece un buen signo de virtud la firmeza de sus sentimientos lo mismo
en la adversa que en la próspera fortuna. Con todo, la marquesa y su
hermano rabian, como es natural, viendo que no pueden desencantar a la
niña, pues lo que tiene, más parece encanto que otra cosa. Y todo se les
vuelve decir: «Padre Salmón, ¿qué haremos? Padre Salmón, ¿qué no
haremos?» Yo me voy al cuarto de la madamita, y después de decirle
cuatro gracias, y de imitar el graznido de los cuervos, y el relincho de
un caballo, y el rum rum de las viejas rezando en la iglesia, con lo
cual ella se ríe mucho, le digo: «Pero hijita de mi corazón, ¿por qué no
desecha vueseñoría todo pensamiento que no sea el de su actual grandeza?
¿Qué cosa puede apetecer ahora? ¿Le falta algo? ¿No tiene todas las
comodidades, todos los miramientos, todos los mimos que una doncella
puede apetecer?» A lo que me contesta que ella no desea nada, y después
se calla. Entonces le tomo las manos, se las acaricio y le digo: «El
pajarito de mi convento me ha contado que amasteis a un jovenzuelo. ¿Por
qué no arrojáis esta idea de la cabeza? ¿No comprende usía que en una
tan principal casa no pueden entrar por las puertas del matrimonio
personas de baja condición? Seguramente que ese zascandil que fue
vuestro novio no se acuerda para nada de mi querida niña.» Y ella al
punto se sonríe, muda de conversación y empieza a hablar de otro asunto
con tan buen tino y tanto talento, que a mí y al padre Castillo nos deja
atónitos.

---Pues veo que cuando dos tan buenos predicadores no la pueden quitar
con sus sermones el desencanto, encantada estará toda la vida.

---No, hijo; que se han intentado varios medios para quitarle eso de la
cabeza. La condesa díjole que el zascandil ese había muerto según sus
averiguaciones, y la marquesa y su hermano, tomando otro camino, han
concertado hacerla creer que el tal desconocido jovenzuelo es un pícaro
ladroncillo de las calles, un tramposo, estafador, a quien persigue la
justicia por sus robos, chuladas y granujerías.

---¡Vive Dios!---exclamé sin poderme contener,---que eso es mentira, y
le romperé el alma al que me diga que es cierto.

---¡Cómo, muchacho!---dijo muy absorto el fraile.---¿Pero a ti qué te va
ni qué te viene en esa cuestión para tomarla tan a pechos?

---Y a todas esas, ella, ¿qué decía?

---Nada. Hasta hoy la verdad es que el ingenioso artificio no ha hecho
gran efecto, y mientras la doncella sin par aparenta no darse por
entendida, la señora marquesa se incomoda más cada día, y a todas horas
exclama: «Esto no puede seguir así.» Riñe con su sobrina, esta suele
llorar, aunque en ella todo revela más paciencia que dolor, y aquí de la
condesa, que se pone como un basilisco en cuanto mortifican a su prima.
Tía y sobrina se dicen cuatro cosas: yo las apaciguo, y hasta el otro
día, que sucede lo mismo.

En esto llegamos a la puerta de la Merced, y Salmón deteniéndose, me
dijo:

---¿Quieres subir? Te daré chocolate crudo y una copita.

---Gracias, padre; estoy rabiando, y no tengo ganas de chocolate ni de
copitas.

Y sin más palabras, despedime de aquella lumbrera de la Iglesia para
irme a mi casa.

\hypertarget{xiv}{%
\chapter{XIV}\label{xiv}}

Llegó con el 28 de Noviembre la noticia de la batalla de Tudela, y una
vez que se consideró deshecho nuestro ejército de Aragón y del Centro,
ya todos vimos el sombrero de Napoleón asomando por la Mala de Francia.
Las fortificaciones avanzaban, y en los días 27, 28 y 29 recuerdo que
menudearon bastante las que podremos llamar fortificaciones y armamentos
espirituales, que eran las rogativas, rosarios, funciones de
desagravios, novenas y otras devociones para alcanzar de la Divina
Providencia, no que apartase los peligros, sino que enardeciera nuestros
ánimos para salir victoriosos. Hubo rosario en San Ginés, jubileo en los
Dominicos de la Pasión, solemnes cultos en el Carmen Calzado, y, por
último, en la iglesia de Nuestra Señora de Gracia, sita en la plazuela
de la Cebada, se inauguró un novenario que fue la más popular de las
devociones de aquellos días, por predicar allí popularísimos oradores.
La gente piadosa al par que patriota no tenía tiempo para acudir a
tantas partes, y vacilaba entre la iglesia y la trinchera. Los hombres
aunque lo deseáramos no teníamos tiempo para frecuentar las iglesias, y
especialmente los armados no dábamos paz a los pies ni a las manos con
el frecuente ejercicio y ensayo de nuestra fuerza. Los soldados, los
voluntarios, los conscriptos, los \emph{honrados} que tenían armas, nos
confundimos por algunos días en comunes trabajos y preparativos, dando
al olvido discordias importunas. Y no estaba el tiempo para andarse con
juegos, porque ya Napoleón se nos venía encima. Mientras existió la
pueril confianza de que las tropas enviadas a Somosierra estorbarían el
paso del tirano, menos mal: íbamos viviendo, alimentando nuestro
espíritu con risueñas ilusiones, y soñando con ver hecho pedazos el
poder de Bonaparte en la era del Mico.

Pero el día 1.º de Diciembre comenzaron a circular desde muy temprano
rumores gravísimos acerca de la derrota del general San Juan en
Somosierra. Echose todo el mundo a la calle en averiguación de lo
ocurrido, y corriendo de boca en boca las nuevas, exageradas por la
ignorancia o la mala fe, bien pronto llegó a decirse que los franceses
estaban en Alcobendas, y hasta alguno aseguró haberlos visto paseándose
en el Campo de Guardias. Desde el famoso 2 de Mayo no había visto a
Madrid tan agitado: corrían hombres y mujeres por las calles, y entonces
era el lamentar la ciega confianza, el echar de menos la actividad y
previsión propias de un pueblo realmente decidido a defenderse. El Gran
Capitán y yo habíamos salido desde muy temprano, él para tomar
disposiciones importantes en el cuerpo de \emph{honrados} a que
pertenecía, y yo por acudir a mi puesto, o curiosear en caso de que aún
no se tratara de cosa formal.

---Lejos de acoquinarme yo, como estos gallinas---decía el Gran
Capitán,---me animo y me gallardeo y me esponjo al saber que los tenemos
tan cerca. Y a mí no me hablen de que el general San Juan ha sido
derrotado. Para los que conocemos las artimañas y recovecos del arte de
la guerra, esa dispersión de las tropas de San Juan que parece derrota,
no es otra cosa más que un hábil movimiento para engañar a Napoleón,
dejándole pasar el puerto. Y si no, figúrate si será bonito ver a lo
mejor que cuando tranquilamente avanzan los franceses creyéndose
seguros, aparecen como llovidas por el flanco derecho las tropas
españolas y me los cogen ahí sin disparar un tiro entre Alcobendas y San
Agustín.

---Podrá suceder---dije yo sin manifestarle mi incredulidad;---pero
figúrese el Sr. Fernández que no pasa nada de esto, sino que viene
Napoleón sano y entero y nos pone cerco. ¿Cómo saldremos de este apuro?

---Admirablemente---repuso.---Podrá suceder que si trae muchas,
muchísimas tropas, vamos al decir, un par de milloncitos de hombres,
dure el sitio dos o tres años, después de cuyo tiempo tendrá que
retirarse\ldots{} porque pensar que Madrid se ha de rendir, es pensar en
lo excusado. Y si no, pasea tus ojos por esas fortificaciones que en
diferentes partes se han hecho en lo que el diablo se restriega un ojo;
esparcía tu vista por esos hondos fosos, por esos gruesos parapetos, por
esos inexpugnables montones de tierra, y por esas terroríficas baterías
de cañones de a 6, y si la admiración te da tregua a las reflexiones,
comprenderás que es imposible tomar a Madrid, aunque Napoleón trajera
mejor gente que aquella que fue a Portugal con el Sr.~Marqués de Sarriá.

---Dios le oiga a Vd. Por mi parte haré lo que pueda. ¿Y Vd. manda, o es
mandado?

---Yo mando; que a ello me han obligado antiguos amigos, cuya ciega
confianza en mis conocimientos raya en fanatismo. Yo no quería mandar
porque no me gustan papeles; pero he tenido que ceder, y entre todos
hemos formado una compañía que ha recibido orden de operar en Los Pozos,
sitio el más arriesgado y peligroso y temerario de este gran asedio que
nos espera. Casi todos tenemos fusiles, y los que no, manejarán la
lanza.

---¡Lanza para defender murallas!---exclamé sin poder disimular la risa.

---Sí, hijo; ¿qué entiendes tú de eso? Figúrate que a esos tontos se les
ponga en la cabeza dar un asalto, ¿qué mejor cosa para
impedirlo?\ldots{} Por cierto que voy a reunir mi gente para ir a ocupar
la posición, no sea que el señor \emph{córcego} quiera darnos una
sorpresa con su acostumbrada mala fe.

---Ahora dejémonos llevar a la Puerta del Sol con todo ese gentío que
allá va ---dije yo;---y parece que ocurre alguna cosa grave, según
gritan.

---Efectivamente; pero esa gritería es de mujeres. Sin duda esas
valerosas matronas piden que se les den armas.

---Bajemos por la calle de la Montera\ldots{} Por allí sube, si no me
engaño, el Sr. de Santorcaz. Llamémosle: él sabrá lo que ocurre\ldots{}
¡Eh, Sr.~D. Luis!

---¿Qué hay en la Puerta del Sol, que tanto chilla la gente?---preguntó
Fernández cuando el otro se nos acercó.

---Es que el pueblo pide armas y no se las quieren dar---repuso
Santorcaz.---Es una picardía y todos esos mandrias de la Junta deben ser
arrastrados.

---¡La Junta! ¡Los señores de la Junta Central!

---No hablo de la Central---prosiguió Santorcaz;---que esa, si es cierto
lo que dicen, ha acordado hoy retirarse de Aranjuez, buscando refugio en
el Mediodía. Hablo de la juntilla que se ha formado aquí para la defensa
de Madrid, y que está en permanencia en la casa de Correos. ¡Aquí hay
muchos traidores---añadió en voz alta,---y algunos han cogido dinero
para entregar la plaza a los franceses! Canallas de traidores. Ahora
salimos con que se han acabado las armas y los cartuchos. ¡Mentira! Yo
sé dónde hay armas y cartuchos. ¡Nos están engañando, nos van a vender!

Diciendo esto, se apartó de nosotros; después de lo cual seguimos hacia
abajo, y al llegar a la Puerta del Sol vimos que estaba de bote en bote
llena de gente. Aquel hueco abierto en el apelmazado caserío de Madrid
es el corazón de la antigua villa, y a él afluye con precipitada
congestión la sangre toda en sus ratos de cólera, de alegría o de miedo.
La Puerta del Sol latía con furia. Hombres y mujeres hablaban a la vez y
a sus voces se unían actitudes y gestos amenazadores. La masa más
inquieta, más hirviente, más loca y alborotadora estaba al pie de la
casa de Correos.

---Busquemos algún conocido que nos informe de lo que aquí ha
pasado---dije metiéndome con el Gran Capitán por lo menos apretado del
gentío.

---Astavía no ha pasado nada---dijo un caballero que envuelto en una
capa se nos apareció, y en quien al punto reconocí al Sr.~de
Majoma.---Astora nada; pero\ldots{} ya verán.

---¿Qué pide esa gente?

---¿Qué ha de pedir? Armas y cartuchos.

---Ya están repartidos todos los que hay.

---¡A mí con esas!---exclamó el apreciable sujeto.---Ya estamos de
traidores hasta el gañote. ¡Pillos lairones! Si no les espachamos nos
van a entregar a los franceses. ¡Perros gabachos! Les conozco bien y se
la tengo sentenciada, sí señor; y el que diga que no son traidores, que
se vea conmigo, porque yo soy más español que Santiago y más patriota
que Fernando VII.

---Pero desde hace tiempo se sabe que la plaza tenía muy pocas armas, y
en cuanto a los cartuchos, todos los que había y los fabricados en esta
semana, se han repartido ya. El Sr.~de Mañara ha estado ocho días
ocupado en dirigir la fábrica de cartuchos y ayer tarde repartió muchos
miles en el Ave-María y en la Comadre.

---¡No me lo nombres!---exclamó Majoma afectando una indignación que más
tenía de cómica que de trágica.---Ahí tienes al traidor más que Judas,
al gabachón más que Copas\ldots{} Gabriel, ¿eres tú traidor también?
¿Estás vendido a los franceses, como ese regidorcillo hambrón? Dime que
sí y verás\ldots{} mia tú\ldots{} aquí mismo te pongo en pipitoria con
esto que traigo debajo de la capa.

---¿La navajita? Guarda tu coraje para mejor ocasión, Majomilla---le
respondí.---Me parece que estás borracho.

---¿Borracho yo? Si no lo he probao, chico\ldots{} Esta mañana me
convidó el Sr.~de Santorcaz a beber unas copas, y\ldots{} por esta que
no bebí más que dos azumbres\ldots{} ¿qué hacer sin la calorcilla en el
estómago?\ldots{} Pero di, ¿eres tú traidor? Di que no, porque te
rajo\ldots{} pues yo (y se daba fuertes golpes en el pecho) tengo un
corazón como un bronce, y soy más valiente que el Ciz y nadie me tosa,
si no quiere ver quién es Majoma.

Y sin oír más, nos apartamos del insigne varón.

---Esto no me gusta---dijo Fernández,---y me parece que si la alta
empresa que entre manos traemos no sale tan bien como debiera,
consistirá en esta inmunda canalla motinesca, díscola y bullanguera, que
en circunstancias tan críticas se vuelve contra sus jefes. Gabriel, de
buena gana te digo que si nuestro D. Tomás de Morla nos mandase cerrar
contra esta gentuza, la meteríamos en un puño prontamente. Y has de
saber que estos perdularios chillones, más sirven de estorbo que de
ayuda en la defensa, y verás cómo son ellos los primeros que se rinden.

Miramos al balcón de la casa de Correos y vimos que en él aparecía un
hombre alto, moreno, hosco, vestido de uniforme; le vimos accionar
hablando a la multitud; pero no pudimos oír sus palabras, porque la
femenil chillería de abajo habría impedido oír tiros de cañón, que no
digo humanas voces. Después aquel militar, el cual no era otro que D.
Tomás de Morla, encogíase de hombros y cruzaba los brazos. Este lenguaje
le entendimos mejor, y evidentemente quería decir: «No hay nada de lo
que me pedís: se acabaron las armas y los cartuchos.»

Pero la multitud se enfurecía con la negativa y le silbaba, pidiendo con
su omnipotente antojo y volubilidad que saliese Castelar, personaje más
conocido que Morla. Salió el marqués de Castelar, habló sin poder
apaciguar a sus admiradores, y repitiose el encogimiento de hombros y el
gesto desconsolador. Aquí de los silbidos, de los gritos, de las
amenazas: poco después el pueblo empezó a arremolinarse y a culebrear
como dragón de mil colas que se dispone a emprender movimiento; y vimos
que muchos se desparramaban por la calle Mayor y que otros subían hacia
Santa Cruz.

---Vamos allá a ver en qué para esto---dijo D. Santiago apoyándose en mi
brazo y siguiendo el general torrente.---Estos majaderos primero dejarán
de existir que de hacer alguna atrocidad. ¿Por qué piden armas, si con
las que hay repartidas basta y sobra? ¿A qué piden cartuchos, si no hay
cartucho que mate más franceses que el entusiasmo español, ni mejor
pólvora que nuestra indignación?

---Todo eso es verdad, Sr.~D. Santiago---repuse;---pero no habría sido
malo que la Junta Central o el Consejo, en vez de ocuparse en discutir
sus rivalidades, hubiera depositado en Madrid unos cuantos barriles de
indignación, de esa que se hace con salitre, carbón y azufre, que la
otra sin esta de poco sirve. Pero aquí no ha habido previsión, ni
iniciativa, ni actividad, ni eminentes cabezas que dirijan, sino que la
defensa ha quedado a merced de la voluntad, de la invención y del buen
sentido del pueblo, Sr.~D. Santiago; y no llamo pueblo a esa miserable
turba gritona que de nada sirve, sino a todos nosotros, altos y bajos,
grandes y chicos\ldots{} ¿Pero quién es aquel que corre? Es el insigne
patriota a quien llaman Pujitos. ¡Eh\ldots{} Sr.~de Pujitos, lléguese
acá y díganos lo que ocurre!

---Ahora va la gente hacia la calle de la Magdalena---contestó,---donde
vive el regidor Mañara. Esta mañana estuvimos allí: salió al balcón y
nos dijo que los miles de cartuchos que ha fabricado los entregó ya, y
que no hay más pólvora. ¿Van Vds. hacia el Avapiés? Por allá hay gran
alboroto, y dicen que Mañara es un traidor, y que acá y allá.

---¿Y Vd. qué piensa de Mañara?

---Mañara es hombre cabal, porque lo igo yo---afirmó Pujitos en tono
misterioso.---Los traidores son otros y andan por ahí revolviendo la
gente y armando estas tramoyas. Gabriel, acuérdate de lo dicho. Los que
más chillan son los piores: pero yo ando con mucho ojo, porque así me lo
ha mandado el jefe, y como les eche la mano encima, verán quién es
Pujitos.

Siguió a toda prisa hacia la Puerta del Sol, y nosotros atravesando la
plaza Mayor, entramos en la calle de Toledo, arteria de toda la
circulación manolesca, centro de las chulerías, metrópoli de las
gracias, bazar de las bullangas, cátedra de picardías y teatro de todas
las barrabasadas madrileñas.

Pasando luego a la calle de Embajadores oímos de nuevo que hacia el
Avapiés había gran marejada, por lo cual atravesando por los Abades
hacia el Mesón de Paredes, nos fuimos a presenciar el tumulto, que no
era flojo, según el rumor de voces que desde lejos se oía. En efecto,
habíase armado un zipizape que déjelo usted estar.

De manos a boca tropezamos con el tío Mano de Mortero, que se llegó a
nosotros diciendo:

---¡Cómo nos engañan, Gabriel! ¡Quién lo había de decir en un caballero
tan bueno como el Sr.~de Mañara!

---¿Pero es traidor el Sr.~de Mañara? Vamos, tío Mano. ¿Vd. también? Vd.
que es una persona de tantísimo talento\ldots{}

---Es verdad, niño de mi alma; ¿pero qué quieres tú? Lo dicen por ahí. A
mí no me consta; pero al son que me tocan, bailo. Pues dicen que hay
traidores, ¡abajo los traidores!

---¿Y qué dicen de Mañara?

---Que tiene arreglado con los franceses el entregarle la puerta de
Toledo.

---¿Y cómo lo saben?

---¡Qué sé yo! Pero cuando el río suena agua lleva. Yo no he de ser
menos que los demás, y pues hay traidores,¡abajo los traidores!

---¿Y la Zaina?

---¿Pues no la oyes? Si es la que más grita en medio de la plaza ¡Santa
Virgen! ¡Y no está poco furiosa esa leoncilla! Ahora se ha vuelto la
patriota más patriota de todo Madrid. ¡Ay mi Dios, qué nacionala tengo a
mi niña!

De rato en rato aumentaba el gentío en la plazuela del Avapiés, y los
hombres de mala facha unidos a las mujeres más desenvueltas de los
cercanos barrios, menudeaban sus gritos y vociferaciones de tal modo,
que ninguna persona honrada podría ante tal espectáculo permanecer
tranquila.

---Acerquémonos---me dijo Fernández.---Yo con todo mi corazón te aseguro
que si Su Majestad y en su real nombre la sala de Alcaldes de Casa y
Corte, me mandase despejar este sitio, lo haría de mil amores con dos
lanzazos o sablazos, que para el caso lo mismo daría.

---Guárdese Vd. de decir en alta voz tales cosas, y acerquémonos a aquel
grupito de damas.

La Primorosa salió del grupo.

---Eh\ldots{} Primorosa, ¿qué traes por aquí?---le pregunté.

---¡Cachiporros!---exclamó la harpía alzando los brazos, cerrando los
puños, y dirigiéndose a algunos hombres que la rodeaban.---¿Pa qué
estáis aquí? ¿No vos quieren dar cartuchos? Pues iz ca el regidor y
sacárselos de las asaúras. ¡Él los tiene escondíos! Él los tiene
enterraos en paquetes pa dárselos a los franceses.

Entonces la Zaina abriéndose paso presentose en el centro del corrillo
formado en torno a la Primorosa. Estaba la hermosa verdulera amoratada y
ronca, con los ojos encendidos, las ropas hechas pedazos, y con tan
fiera expresión retratada en su semblante y en toda su persona, que
causaba espanto. En el momento de presentarse, traía un cartucho entre
los dedos, y lo mordía y derramaba en la palma de la mano lo que debía
ser pólvora y resultaba ser arena.

\hypertarget{xv}{%
\chapter{XV}\label{xv}}

---¡De arena! Los cartuchos están llenos de arena---exclamó la muchacha,
mostrando a todos aquel objeto.

Y al mismo tiempo los hombres allí presentes sacaban de sus sacos otros
cartuchos, los mordían, y en efecto, en todos o en casi todos aparecía
arena.

---¡Ese traidor nos ha dado cartuchos de arena!

La terrible voz cundió por la plaza. Allí cerca había un retén de
guardia de voluntarios. Sacaron el depósito de cartuchos, mordíanlos, y
por cada dos o tres con pólvora había uno con arena. Esto lo vimos el
Gran Capitán y yo, y ambos nos quedamos mudos de indignación.

---Pues indudablemente ha habido traición---dije yo.

---¡Poner arena en los cartuchos! ¡Qué alevosía! Esto es entregar la
patria villanamente al extranjero.

---El que tal ha hecho---exclamé no ocultando mi rabia,---es un
miserable que debe ser castigado.

Gabriel, no lo creí---vociferó mi amigo, derramando lágrimas de
coraje;---no creí que hubiera españoles capaces de semejante vileza. No,
el que tal ha hecho no es español.

Y los dos casi sin darnos cuenta de ello, hicimos coro con la rabiosa
multitud, gritando: ¡Mueran los traidores!

---¡Ese Mañara, ese ladrón!---gritaron a nuestro lado.

---¡Él ha sido! ¡Mueran los traidores y viva Fernando VII!

¡De arena! ¡Los cartuchos de arena! Esta funesta frase corrió por todo
Madrid más rápidamente que si la llevara la electricidad. En muchas
partes, que no en todas, pudo confirmarse la verdad de la afirmación;
pero la ira era general, y el que había puesto arena en los cartuchos
fue condenado a muerte por la indignación popular. Mi amigo y yo
observamos que la multitud corría en todas direcciones; pero los más
iban hacia la Merced. Desapareció de nuestra vista la Pelumbres, el tío
Mano, y desapareció también la Zaina. Corrimos por la calle de Jesús y
María, y al llegar a la de la Magdalena, la vimos completamente llena de
gente: todo el vecindario estaba en los balcones, y un clamor inmenso
llenaba la vasta longitud de la calle. Hacia el centro de ella existía
entonces, y existe aún, una casa suntuosa, pero de bastarda y ridícula
arquitectura, por haber puesto en ella su mano D. Pedro de Ribera, autor
de la fachada del Hospicio. A aquella casa histórica, residencia antes y
también hoy de una respetabilísima familia, por mil títulos merecedora
de la estimación pública, se dirigían las amenazas de la muchedumbre,
borracha de ira. Todos querían entrar; pero las puertas estaban
cerradas. Este obstáculo no tardó en desaparecer, y terribles hachazos
hicieron temblar las labradas maderas de la puerta señorial, protegida
por el ancho escudo que en esculpidos emblemas representaba hazañas y
virtudes de otros tiempos. Mas ¿quién reparaba en esto? El pueblo, que
ya había pisoteado en Aranjuez la real corona, no vacilaba en pasar por
sobre la de un noble. Hicieron, pues, pedazos la puerta, y el pueblo
entró desbordándose e invadiendo el palacio, como un río que rompe los
diques que durante siglos le han contenido y se extiende por el llano
con ímpetu destructor. Entraron todos, los que iban con algún objeto y
los que no iban más que a gritar. No debía, pues, hacerse esperar mucho
la satisfacción de la popular furia, y bien pronto nos quedamos helados
de terror, oyendo decir: «Le han matado, ya le han matado.»

¡Pobre y desgraciado Mañara! Ayer ídolo, ayer amigo, ayer compañero de
la vil plebe, cuyo traje y costumbre, y hablar y modos imitaba, hoy
inmolado por ella con barbarie inaudita, con esa cruel presteza que ella
emplea ¡la infame furia! en todas sus cosas.

Pero lo espantoso, lo abominable, y más que abominable vergonzoso para
la especie humana, fue lo que ocurrió después. La plebe tiene un sistema
especial para celebrar las exequias de sus víctimas, y consiste en
echarles una cuerda al cuello y arrastrarlas después por las calles,
paseando su obra criminal, sin duda para presentarse a los piadosos ojos
en la plenitud de su execrable fealdad. Esto pasó con el cadáver del
infeliz regidor, a quien conocimos amante de Lesbia, amante de la Zaina,
amante de todas, pues no hubo otro que como él prodigara su hermosa
persona en altas y bajas aventuras; esto pasó con el cadáver del infeliz
a quien llamo D. Juan de Mañara, no porque este fuera su nombre, sino
porque me cuadra designarle así, para no andar trayendo y llevando los
títulos de respetables casas, por los altibajos de esta puntual
historia. Pero apartemos los ojos, no miremos, no, ese despojo
sangriento que por la calle de la Magdalena, y después por la del
Avapiés abajo, arrastran en inmunda estera unos cuantos monstruos,
hombres y mujeres tan sólo en la apariencia: cerremos los oídos a sus
infames gritos, y sobre todo no miremos ese destrozado cuerpo, aún
caliente, a quien las puñaladas, los golpes, el frecuente tropezar van
quitando la figura humana, haciendo un jirón lastimoso de lo que fue, de
lo que era pocos minutos antes hombre gallardo y gentil, y lo que es más
digno de consideración, hombre dichoso y amable. Y mientras pasa esa
salvaje bacanal, ese río de sangre y de infamia y de crimen, meditemos
sobre las mudanzas mundanas, y especialmente sobre las cosas populares,
las más dignas de meditación y estudio.

¿Era Mañara autor de la traición indudable descubierta en los cartuchos
de arena? Histórica, no hija de nuestra invención, es la persona de
Mañara; histórica es también su vida licenciosa, sus hábitos manolescos,
sus aventuras y trato con la gente de los barrios bajos; histórica es
también la Zaina, y tan históricos como la jura en Santa Gadea y el
compromiso de Caspe, son sus amores con el regidor, su abandono, sus
celos, su despecho, su ira, su sed de venganza y el descubrimiento,
fatalmente hecho por ella, de los cartuchos de arena. Para saber todo
esto basta leer media página de la historia mejor y más conocida que
sobre aquellos tiempos se ha escrito. Pero ni en este eminente libro, ni
en otro alguno, ni en boca de ningún viejo oiréis razones para contestar
categóricamente a la pregunta que antes hice. ¿Fue Mañara traidor?
¿Intervino él en la obra criminal de los cartuchos de arena?

Os diré francamente que yo tampoco lo sé; pero debo advertiros que nunca
tuve a aquel desgraciado por capaz de acción tan fea. Mañara pecaba de
libertino, de ligero, de vano y más que nada de enamorado. Jamás se
distinguió en otras maldades que en las del amor, por cierto bien
perdonables. Le conocí alevoso y traidor en cuestiones de faldas; pero
no supe nunca que en asuntos graves faltara a las leyes del honor. Con
estos antecedentes casi puede asegurarse que no fue Mañara autor de la
superchería de los cartuchos. ¿Pues quién lo fue entonces? Esto sí que
ni la historia, ni la tradición, ni los viejos, ni yo podemos decíroslo.
¿No habéis observado que todos los movimientos populares llevan en su
seno un germen de traición, cuyo misterioso origen jamás se descubre? En
todo aquello que hace la plebe por sí y de su propio brutal instinto
llevada, se ve tras la apariencia de la pasión un tejido de alevosías,
de menguados intereses o de criminales engaños; pero ningún sutil dedo
puede tocar los hilos de esta tela escondida en cuyas mallas quedan
enredados y cogidos mil bárbaros incautos.

¿Quién hizo correr la voz de la traición de Mañara? ¿Fue todo obra
deliberada de la Zaina? La historia dice que sí; pero yo creo haber oído
tachar de sospechoso al pobre regidor en parajes muy distantes de la
calle de la Pasión. Sin duda el frecuente roce con la plebe había
desconceptuado mucho a D. Juan en la opinión de sus iguales. Carecía en
absoluto de respetabilidad, y el que la pierde entre los de arriba
queriendo sustituirla con bajas amistades, que son siempre inconstantes,
está expuesto a perderlo todo en un momento, y a que cualquier chispa
fugaz incendie de improviso la fábrica de una reputación que no se funda
en nada sólido.

Mañara había adulado a la plebe imitándola. Con este animal no se juega.
Es como el toro que tanto divierte, y de quien tantos se burlan; pero
que cuando acierta a coger a uno, lo hace a las mil maravillas. Vimos
caer a Godoy, favorito de los reyes, y ahora hemos visto caer a Mañara,
favorito del pueblo. Todas las privanzas que no tienen por fundamento el
mérito o la virtud suelen acabar lo mismo. Pero nada hay más repugnante
que la justicia popular, la cual tiene sobre sí el anatema de no acertar
nunca, pues toda ella se funda en lo que llamaba Cervantes el \emph{vano
discurso del vulgo, siempre engañado}.

---Pero vámonos de aquí---dije a mi amigo.---¿No oye Vd. lo que dicen
esos que pasan? Dicen que los franceses han aparecido por Fuencarral.

---Vamos, vamos a cumplir con nuestro deber---repuso el Gran Capitán,
siguiéndome por la calle de las Urosas.---Pero me temo que lo que debía
ser gloriosísima jornada, va a ser cualquier cosa, gracias a esa vil
gentualla. La traición mina la plaza. Eso de los cartuchos de arena me
ha puesto triste y el miserable canalla que tal hizo merece mil muertes.

Madrid, después de inmolado Mañara, continuaba inquieto, como
presagiando grandes males, mientras los frailes agonizantes arrancaban
de manos del pueblo el cadáver informe. La noticia de que los franceses
estaban a las puertas de la villa, lo hizo, sin embargo, olvidar todo, y
corría la gente azorada y medrosa, creyendo ver asomar al volver de una
esquina la figura característica del azote de Europa.

\hypertarget{xvi}{%
\chapter{XVI}\label{xvi}}

El cuerpo de voluntarios a que yo pertenecía fue destinado a defender la
puerta de los Pozos (la misma que después se llamó de Bilbao al extremo
de la calle de Fuencarral), y el inmediato jardín de Bringas. Consistía
su fortificación en un foso no muy profundo en un gran espaldón de
tierra y piedras, a toda prisa levantado, y en seis cañones de a 6. La
tapia que no tenía facha de inexpugnable, como recordarán los que han
alcanzado alguno de sus heroicos trozos, había sido aspillerada en toda
su extensión. Iguales poco más o menos, eran las fortificaciones de las
vecinas puertas de Santa Bárbara y Fuencarral. El sitio donde se habían
levantado obras más considerables era la puerta de Recoletos, monumento
que ha durado hasta ayer y que no necesito designar topográficamente,
con su costanilla de la Veterinaria ni su convento de Agustinos, porque
los mozuelos barbilampiños los han conocido. Pero volvamos a los Pozos,
puerta destinada a ser teatro de nuestro heroísmo, y empecemos diciendo
que en la noche del 1.º de Diciembre nos situamos allá, tan convencidos
de que íbamos a ser atacados que estuvimos largas horas sobre las armas,
dispuestos a vender caras nuestras vidas. La fuerza se componía de estos
elementos: unos sesenta soldados, que aunque no todos artilleros, hacían
de tales por necesidad imprescindible; cuatro compañías de voluntarios
antiguos, con los cuales mezclábase un número irregular de conscriptos,
y como ochenta hombres de la milicia \emph{honrada}, a quien mandaba o
quería mandar el Gran Capitán, no sé si con el título de sargento,
coronel o general, pues cualquiera de estos grados le cuadraría. Los
soldados estaban fríos y con poco ánimo; los voluntarios inflamados en
patriotismo y llenos de ilusiones; pero tan inexpertos, que no daban pie
con bola, como vulgarmente se dice, a pesar de estar entre ellos el gran
Pujitos; y finalmente los \emph{honrados} no cabían en sí de entusiasmo,
no obstante ser todos ellos personas de paz, y tener algunos buena carga
de años a la espalda, especialmente los de la compañía, o mejor, los del
grupito en que alzaba el gallo D. Santiago, cuya hueste se componía de
respetables porteros y criados de la oficina de Cuenta y Razón.

En cuanto a jefes, debo decir que allí no existían en todo el rigor de
la palabra, pues si bien entre la tropa había oficiales valientes y
entendidos, no sabían o no querían hacerse obedecer de los paisanos,
resultando de esta desconformidad que allí cada cual hacía lo que le
daba la gana y según su propia inspiración; y aunque mi amigo tenía
pretensiones de imponer su autoridad, esto no pasó nunca de un conato de
dictadura que más se inclinaba a lo cómico que a lo trágico.

En cambio reinaba gran fraternidad, y cuando avanzada la noche tuvimos
la certeza de que no había tales franceses por los alrededores, nos
reunimos en el jardín de Bringas, y encendida una gran hoguera,
celebramos agradable tertulia, donde se habló de temas patrióticos con
la verbosidad, facundia y exageración propia de españolas lenguas. Cuál
encomiaba la defensa de Zaragoza; cuál ponía la defensa de Valencia
contra Moncey por cima de todos los hechos de armas antiguos y modernos;
quién decía que nada podía igualarse a lo del Bruch; quién encomió hasta
las nubes la vuelta de las tropas de la Romana; y por último, no faltó
uno que, sin quitar su mérito a estas gloriosas acciones, pusiera sobre
los cuernos de la luna cierta campaña famosa de Portugal en 1762.

Disipado todo temor, muchas mujeres fueron a visitarnos, y entre ellas
no faltó doña Gregoria, ni doña Melchora con las niñas, ni tampoco la
señora de Cuervatón, pues ha de saberse que su marido formaba en las
filas de los \emph{honrados}. Para que no se crea que todos éramos gente
de poco más o menos, añadiré que algunas altísimas damas fueron a
visitar a sus hijos, hermanos o maridos, que allí se andaban mano a mano
con nosotros, o como voluntarios o como sorteados.

Cenamos, bebimos, cantamos, hablamos, y por último, a todos nos vino el
deseo de llevar adelante alguna hazaña aquella misma noche. El primero
que emitió la idea fue D. Santiago y al punto se la aceptó con alborozo,
determinando hacer una exploración camino arriba hasta Fuencarral, por
ver si realmente estaban los franceses tan cerca como se creía. A toda
prisa se preparó la salida, y a eso de las dos de la madrugada nos
pusimos en marcha unos doscientos hombres, en buen orden, y mandados por
un coronel de ejército.

---¡Qué bueno fuera---me decía Fernández,---que ahora tropezáramos con
una avanzada enemiga y la derrotáramos en un abrir y cerrar de ojos,
volviendo a Madrid con unos cuantos miles de prisioneros!

---Todo podría ser, amigo mío---le respondí,---que para la voluntad de
Dios no hay nada imposible.

---Más gracioso aún sería---prosiguió,---que el bergante del Emperador
se anduviera paseando por ahí, mirando desde lejos la gran ciudad que
aspira a ganar, y le sorprendiéramos de sopetón, echándole mano para
llevarle a Madrid sobre un asno foncarralero.

---También es posible---repuse,---y pongamos que ese señor se haya
aburrido de estar en su campamento, y tomando una escopeta, a pesar de
la oscuridad de la noche, se venga con un par de generales y un par de
perros por esos trigos a levantar y correr perdices; que todos los
monarcas suelen ser cazadores.

---Eso no me parece verosímil---dijo;---pero bien podría suceder que ese
hombre, conociendo que no puede vencernos por la fuerza, intente dar al
traste por la astucia con nuestro poderío, y se disfrace con el traje de
un payo huevero de Alcobendas, para acercarse a nuestras formidables
fortificaciones y estudiarlas cómodamente.

Con estos y otros coloquios rebasamos más allá de la venta, situada en
lo que hoy se llama Cuatro Caminos, sin hallar alma viviente ni sentir
rumor alguno; pero cuando estábamos cerca del camino que a mano derecha
conduce a Chamartín, percibimos un ruido lejano que a todos nos dejó
suspensos, pues no parecía sino que temblaba la tierra al galopar de
millares de caballos.

---¡Es una avanzada de caballería!---gritó nuestro
coronel.---Retirémonos.

---¿Qué es eso de retirarse?---gritó con enojo el Gran Capitán.---¿Somos
españoles o qué somos?

---No tenemos más que cuatro caballos---le dijo el jefe.---Si nos dan
una carga, ¿qué va a ser de nosotros?

---¡Qué cargas ni cargas! ¡Buenos son ellos para meterse en cargamentos!
Ea, muchachos, el que quiera seguirme que me siga; yo voy adelante.

Los \emph{muchachos}, cuyo patriotismo invocaba Fernández, eran seis o
siete vejestorios como él, compañeros en la portería y servicio interior
de las oficinas de Cuenta y Razón. Pero aquellos valientísimos
militares, más duchos en el manejo de la escoba que en el de otra arma
alguna, profesaban aquel principio tan sabio como famoso, de que una
retirada a tiempo es una gran victoria, y todos a una manifestaron al
Gran Capitán que no le seguirían en tan temeraria empresa, pues hazañas
sin cuento podrían realizar tras las fortificaciones.

El escuadrón francés avanzaba, a juzgar por el acrecentamiento del
ruido, pero no veíamos cosa alguna. Se dio orden de retirada, y para
hacerla más a salvo, nos desviamos del camino, escurriéndonos por una
hondonada que caía hacia la dehesa de Amaniel. D. Santiago renunció a
regaña dientes a los peligros de una lucha con los dragones que a toda
prisa avanzaban, y me decía:

---Pensar que de esta manera hemos de vencer, es una necedad. En la
guerra ha de fiarse todo a lo imprevisto, a la sorpresa y a los golpes
de mano. ¿Qué nos costaba esperar esos caballos, sorprenderlos, matar a
los jinetes y entrar en Madrid caballeros los que salieron peones?

En esto vimos un bulto, un hombre, que saliendo precipitadamente de
detrás de unos tejares, corrió hacia la carretera, al parecer huyendo de
nosotros.

---¡Eh! ¡Un hombre! ¡Un espía!\ldots{} ¡Quién vive!---gritamos,
corriendo algunos en su persecución.

Detúvose el hombre ante nosotros con muestras de tener mucho miedo, y
entonces advertimos que su traje era el de un paleto, con ancho sombrero
y una manta por capa. Cuando nos llegábamos a él, pareció vacilante e
indeciso; pero al fin oyéndonos hablar, abalanzose hacia nosotros,
diciendo:

---¡Ah! Sois españoles. Gracias a Dios: ya me he salvado.

Acabando de decir esto, cayó de rodillas. Pero en el mismo instante
llegose a él con aire resuelto el Gran Capitán, y poniéndole en el pecho
la boca de un fusil, exclamó con voz exaltada y furiosa:

---Dese a prisión Vuestra Majestad Imperial y Real. Bien lo decía yo;
pero a mí no me la da Vd\ldots. digo, Vuestra Majestad; que soy perro
viejo, y harto se ve que disfrazado con traje de paleto, se acerca
Vuestra Majestad Imperial a nuestra gran plaza para estudiar las
fortificaciones.

---Hombre de Dios---dijo el payo,---Vd. es loco, o me toma por el
emperador Napoleón.

---¡Por quién le he de tomar, hermano! A mí no se me engaña con
palabritas. Es Vuestra Majestad mi prisionero, y no le he de soltar
aunque me dé siete condados. ¡Viva España y viva Fernando VII!

Todos los circunstantes nos reímos, lo cual desconcertó a D. Santiago, y
al punto el prisionero dijo levantándose:

---Yo, señores, soy oficial del ejército de D. Benito San Juan, y he
asistido al desastre más funesto de esta campaña. Perdí en la acción de
Somosierra a mi padre y a dos hermanos, y vengo huyendo de las
guerrillas francesas que persiguen a los dispersos. Tuve que disfrazarme
en Roblegordo para evitar que me cogieran, y a pie he llegado hasta
aquí. Pero si quieren que les diga más, denme algo que me sustente, pues
con dos días de no probar bocado, estoy cayéndome muerto por instantes.

Un compañero nuestro le dio a beber un trago de aguardiente, con lo cual
tomó fuerzas y pudo seguirnos, reanimado también moralmente por verse en
nuestra compañía. El Gran Capitán, corrido y confuso, marchaba
silenciosamente a su lado, pero no las tenía todas consigo, y todo se
volvía mirarle y remirarle, sospechando que si no el mismo Emperador,
podía ser algún generalazo o cualquier archipámpano de la corte
imperial.

---Con ser tantas mis personales desdichas---dijo el desconocido,---pues
en el campo de batalla quedaron mis dos hermanos y mi buen padre (que
somos de un antiguo solar de tierra de Sepúlveda), todavía abruma mi
ánimo más que nada la catástrofe nacional de que he sido testigo.
Nosotros acudimos a tomar las armas en defensa de la patria. Felices mil
veces los que murieron por tan santo objeto, y mal hayan los que
quedamos para contar tan gran desventura. ¿Se sabe ya en Madrid la
derrota de San Juan? ¿Cómo se cuenta? ¿Qué se dice? Se nos tachará de
medrosos o cobardes. ¡Oh, señores! Yo no creo que sea posible llevar más
adelante el heroísmo. Nuestros soldados se han conducido con bravura
portentosa, y si no vencieron, fue porque la superioridad de los
enemigos y su mucho número lo han hecho imposible.

---Eso será lo que tase un sastre---dijo el Gran Capitán.---¿Por dónde
anda ahora San Juan? Porque yo entiendo que fingió retirarse para atacar
después en mejor posición.

---¡Qué ha de fingir, hombre, qué ha de fingir!---repuso el
oficial.---San Juan, si es que vive, andará fugitivo como yo, y sin un
solo soldado.

---Eso no puede ser, caballero. ¿Cómo se entiende? Si eso fuera cierto,
señor mío, significaría ni más ni menos una especie de derrota.

---Pues ya lo creo. Pero les contaré punto por punto. San Juan tomó
buenas posiciones en el paso de Somosierra y puso una vanguardia en
Sepúlveda. Atacaron esta los franceses anteayer de madrugada; mas no
pudieron romper su línea y tuvieron que retirarse.

---¿Los franceses? bien---dijo el Gran Capitán.---Pues si se retiraron,
¿cómo se entiende nuestra derrota?

---Paciencia, señor mío, paciencia. Sepa usted que sin aparente motivo,
aunque es fácil comprender que ha habido algo de traición, la vanguardia
de Sepúlveda, a pesar de quedar victoriosa, se retiró a Segovia.
Avanzaron los franceses, y nos atacaron en nuestras posiciones de
Somosierra. Nosotros no teníamos fuerzas bastantes para defender el
paso, y mucho menos después de la defección, o no sé cómo llamarlo, de
la vanguardia. Sin embargo, nos resistimos toda la mañana de ayer,
aglomerando nuestra gente en el camino, y sin disponer de fuerzas
ligeras que flanquearan las alturas. Los franceses que traen muchos
soldados y cuerpos de todas clases, dispusieron guerrillas de cazadores
que en un instante tomaron las alturas, y con un cuerpo de caballería
polaca nos cargaron en la carretera de un modo espantoso. No puede
formarse idea de aquel ataque sino viéndolo. Escuadrones enteros se
estrellaban contra nuestra batería y centenares de jinetes caían
despeñados a los abismos que costean el camino; pero sus recursos son
inmensos; tras un escuadrón inútilmente sacrificado, lanzaban otro y
otro, sin que se les importara ver morir oficiales a centenares y
generales por docenas. Con este ataque incesante combinaban el fuego de
las tropas ligeras, desparramadas por los altos, y al fin sucumbimos al
número, que no al valor. Los franceses se abrieron paso a costa de
inmensas pérdidas, y luego persiguieron a los restos de nuestras tropas
con tanto encarnizamiento, que dudo que hayan podido sobrevivir muchos.
La mayor parte, pereciendo en aquellas fragosidades, han cumplido con su
deber, que era defenderlas mientras tuvieran cuerpo vivo en que recibir
una bala. No fue posible más, porque más habría sido hacer milagros, y
estos sólo Dios los hace.

Calló el oficial, y todos los que le oíamos estábamos tan apesadumbrados
y tristes con su relato, que nada le contestamos. Tampoco él habló más,
y así silenciosos y taciturnos llegamos a Madrid y a nuestra puerta de
Los Pozos, donde el desgraciado tránsfuga halló una hoguera en que
calentarse, y un bocado con que reanimar sus fuerzas. Todos le
prodigaban solícitos cuidados, menos D. Santiago Fernández, el cual no
podía desechar cierta comezón y desasosiego.

---Gabriel---me dijo, llevándome aparte.---No insisto por no parecer
pesado; pero digan lo que quieran los demás, ese hombre que hemos
encontrado no me gusta, y quiera Dios no tengamos que sentir; porque yo
sé, y tú sabraslo también, que en las guerras es muy común eso de
disfrazarse para visitar el campo enemigo y examinar a mansalva las
fortificaciones, así como también es cosa corriente sobornar a algún
infeliz para que fingiéndose amigo penetre en la plaza y haga circular
noticias falsas que desalienten a los sitiados.

Amaneció el 2 de Diciembre, y a favor de las primeras luces del día se
distinguieron fuertes columnas de caballería francesa en los cerros del
Norte. Ya estaban allí, y no eran pocos ciertamente.

\hypertarget{xvii}{%
\chapter{XVII}\label{xvii}}

Aquella mañana fue muy alegre para nosotros, porque sin motivo alguno
que lo justificara, nos sentíamos tan animados, que no nos cambiáramos
por los sitiadores. El peligro había acallado por el momento todas las
discordias, y nuestro patriotismo nos achicaba las circunstancias
desfavorables, aumentando considerablemente las ventajosas. Todo se
volvía a gritar, dando vivas y mueras, pues nada cuesta triunfar de este
modo con las fáciles armas de la lengua.

Nos desayunamos muy contentos con lo que las mujeres del barrio, altas y
bajas, bonitas y feas nos traían en repletas cestas. También fue con la
suya doña Gregoria, mas del contenido de ella no probó bocado D.
Santiago, porque, según decía, en los momentos supremos no debe
embrutecerse el cuerpo con viciosos regalos.

Lejos de asentir a la más mínima concupiscencia del paladar, increpó D.
Santiago a los glotones, y luego, pasando revista a sus compañeros, que
todos desiguales en estatura, armamento y vestido, no tenían más
uniformidad que la de su vejez, ni otro aspecto respetable que el de sus
canas, les arengó así:

---Muchachos, acordaos de que todos sois unos buenos chicos, y de que
todos os habéis cubierto de gloria en los reales ejércitos. Ha llegado
la ocasión suprema, y desde el momento en que se presenta a las puertas
de Madrid ese monstruo infame, ya no pertenecéis a vuestros hogares, ya
no pertenecéis a la oficina de Cuenta y Razón, ya no pertenecéis sino a
la patria. Compañeros: todos sois hombres experimentados; no como estos
mocosos rapazuelos que no saben coger un fusil. ¡Ya se ve! ¡Cuándo las
han visto ellos más gordas! Y basta de sermones, que ahora obras y no
palabras, y más vale una buena puntería que cien discursos; conque,
compañeros, ¡viva Fernando VII! y sepan que los estima su amigo y seguro
servidor Santiago Fernández.

Esta alocución del veterano hizo reír a muchos de sus amigos, y casi,
casi\ldots{} si no fuera por temor a denigrar la memoria de varón tan
insigne, diría que la recibieron con chistes, jácaras y todas las
zandunguerías que son propias de los españoles aun en apretadas
ocasiones de la vida; pero Fernández, sin hacer caso de bromas, seguía
tomando enérgicas disposiciones. Quiso también meter su cucharada en la
artillería, echándoselas de gran balístico; pero le mandaron que fuera a
rezar el rosario, cuyo insulto le exasperó de tal manera, que, a no
reparar en consideraciones patrióticas de gran peso, habríale abierto en
dos tajadas la cabeza al descomedido y grosero que tal dijo.

En confianza revelaré a mis lectores que el deslenguado y procaz que de
tal modo prohibió a nuestro Gran Capitán que se acercase a los cañones,
fue el insigne Pujitos, flor y espejo de los entremetidos, personaje de
todas las ocasiones y de todos los sitios, a quien la suerte nos deparó
también por compañero en aquella gran jornada.

A eso de las doce nos visitó el capitán general con D. Tomás de Morla, y
aunque los victoreamos hasta quedar roncos, no me pareció que estaban
ellos muy satisfechos. Aún permanecían allí cuando distinguimos un gran
tropel de franceses por la Mala de Francia abajo y flanqueando el
camino. Era la avanzada del Cuerpo de Bessieres que venía a intimarnos
la rendición. Cuando el parlamentario llegó a los Pozos, poco faltó para
que los más belicosos y trapisondistas le despidieran a puntapiés; pero
al fin fue recibido decorosamente, y se le contestó que no nos daba gana
de rendirnos.

---Como no sea por medio de artimañas, embaucamientos o pérfidas tretas,
semejantes a aquella del caballo de Troya, no nos rendiremos---me dijo
Fernández.---Mira qué cabizbajo se va el oficial a dar la infausta nueva
a su Emperador. Me parece que veo a este pateando y arrancándose los
pelos de rabia al saber nuestra respuesta.

Durante aquella tarde no volvieron parlamentarios, ni se presentó fuerza
alguna francesa; pero a lo lejos distinguíamos el movimiento de las
columnas, tomando posiciones y estableciendo trincheras para la
artillería, lo cual indicaba que los franceses diferían la función para
el día 3. Durante la noche el mariscal Ney hizo otra intimación, pero
fue hacia la parte de Recoletos o puerta de Alcalá.

---¿Ves cómo no se atreven a volver acá, ni quieren más cuentas con
nosotros? ---dijo el Gran Capitán, cuando lo supo;---pero allá les
habrán contestado lindezas. Ya se ve, comprendiendo que por las armas no
pueden nada, ponen en juego melosidades, agasajos y socaliñas. Pero
durmamos, Gabriel, con toda tranquilidad, pues me parece que mañana 3
tampoco habrá nada, y sabe Dios si al ver el aparato de estas intomables
fortalezas, habrán decidido retirarse del lado allá de la sierra.

No necesito decir que de todo en todo se engañaba mi optimista amigo,
pues cuando dormíamos a pierna suelta en la huerta de Bringas al calor
de una hermosísima hoguera, nos despertaron unos tremendos cañonazos que
retumbaban en todo Madrid con pavoroso ruido.

---¡A las armas!---dijo Fernández.---Levántense todos, y si cae una
granada, arrojarse de barriga. Yo soy opinión es que hagamos una salida
para ver de ponerle las peras a cuarto a esos de los cañoncitos. Mirad,
chicos, hacia Chamberí hay una batería.

Al punto nuestros artilleros, que eran mitad de línea y mitad paisanos,
se dispusieron a la defensa, y como dos de las piezas hicieran fuego, no
quisimos ser menos los infantes, y allá fue una descarga sin saber
contra quién.

Densa niebla envolvía la tierra, y no se percibían los lejos, lo cual
hizo que figurándonos nosotros tener enfrente un formidable ejército,
disparásemos cañones y fusiles en ruidosísima salva sin resultado
alguno, pues los franceses no soñaban con atacar los Pozos, y las
detonaciones oídas eran las de la artillería que empezaba a embestir la
puerta de Recoletos.

---Cese el fuego---dijo nuestro jefe.---No nos atacan ni hay enemigos en
la Mala de Francia.

---¿Pues cómo ha de haber?---dijo el Gran Capitán dando fuerte patada en
el suelo,---¿cómo ha de haber si han huido todos?

---No hay tal trinchera ni cosa que lo valga en Chamberí. Los franceses
están hacia la Fuente Castellana.

---A mí no me vengan con músicas---exclamó el Gran Capitán preparando su
arma.---Favorecidos de la niebla, esos miserables quieren engañarnos.
Haré fuego mientras me quede un cartucho.

Seguía disparando como si quisiera acribillar la espesa cortina de
niebla, por cuyo insensato acaloramiento pronto se quedó sin municiones.
Y como continuaran oyéndose tiros de cañón hacia nuestra derecha,
Fernández exclamaba, volviéndose a sus amigos:

---Van en retirada, valientes compañeros. Gracias a vuestro arrojo
temerario, todo se acabará felizmente.

Por largo tiempo estuvimos quietos y mudos esperando con la mayor
ansiedad a que de una vez se nos atacara; pero pasaban horas, y como no
fuera D. Santiago, nadie veía enemigos enfrente, ni lejos ni cerca.
Entre ocho y nueve el fuego de cañón y de fusilería arreció tanto por
Recoletos que no dudamos era este sitio teatro de una vigorosa lucha; y
al mismo tiempo como comenzase a disiparse la niebla, vimos que cesaba
poco a poco aquel desdeñoso abandono en que el Emperador nos tenía,
porque corrían de Oriente a Poniente algunas columnas con apariencia de
tener en respeto a las cuatro puertas septentrionales.

---Gracias a Dios---dijo Fernández,---que se atreven a atacarnos. Por
detrás del parador del Norte me parece que avanza un cuerpo de
artillería de batalla.

No tardaron en romper el fuego contra las trincheras de los Pozos, y
nuestros seis cañones, que ya rabiaban por tomar formalmente la palabra,
contestaron con precisión; mas para que todo fuera desastroso, mientras
la bala rasa de sus piezas nos deterioraba los espaldones, nuestros
proyectiles, lanzados por la carretera adelante o hacia la derecha,
apenas llegaban hasta ellos: tan inferior era la artillería española en
aquel trance. Entonces comenzó una lucha, que antes que lucha debería
llamarse simulacro, harto deslucida para nosotros, pues más nos hubiera
valido ser destrozados por el enemigo que soportar tan cruel situación;
y fue que los franceses nos cañoneaban desde muy lejos con sus piezas de
superior calibre, y mientras recibíamos cada poco rato la visita de una
bala rasa o de una granada, a nosotros no nos era posible hacerles daño
alguno.

---Pero esos cobardes, canallas, ¿por qué no se acercan?---decía
Fernández bufando de cólera.---Eso no es de caballeros, no señor;
cañonearnos sin piedad destruyendo los parapetos con tanto trabajo
levantados, y ponerse en donde no alcanzan las balas de aquí; eso no es
de gente hidalga, y bien dicen que Napoleón ha hecho siempre la guerra
de mala fe.

---¡Malditos sean!---dijo el oficial que nos mandaba.---Esta era ocasión
para hacer una salida, si tuviéramos un puñado de gente de la buena que
yo conozco.

---¿Pues y nosotros, pues y mis amigos, todos estos bravos muchachos de
la compañía de \emph{honrados}?---dijo el Gran Capitán dando un fuerte
golpe en el suelo con la culata.---¿Pues qué desean ellos, si no es
salir para que esa canalla se marche de ahí o se ponga al alcance de
nuestros fuegos?

---Lo que es eso, buenos tontos serán si lo hacen pudiendo foguearnos a
pecho descubierto.

---Saldremos, sí, saldremos---insistió mi amigo.---Muchachos, os conozco
en la cara el ardor sublime y el generoso patriotismo que os inflama.
Rabiando estáis por cebaros en esa gentuza. ¿Salimos, señor coronel?

El coronel se rió con lástima y pena al ver la bravura del anciano. Uno
de los \emph{honrados}, a quienes Fernández llamaba \emph{muchachos},
aseguró que no podía dar un paso porque el reúma se lo impedía; otro
dijo que el ruido de los cañonazos le habían vuelto completamente sordo,
y un tercero se tendió en el suelo de largo a largo, lamentándose de
haber cogido una pulmonía por razón del mucho frío y desabrigo en que
toda la noche estuvieran. Entre los demás \emph{honrados}, había alguna
gente fuerte y valerosa; pero casi todos los del grupito que rodeaba a
D. Santiago, se componía de unos Matusalenes tan mandados recoger, que
daba compasión verles. Cuando algunas mujeres de Maravillas y del
Barquillo vinieron tumultuosamente a los Pozos y pidieron con gritos y
chillidos que les dieran las armas de los ancianos, yo creo que se hizo
mal en no acceder a su petición, y aunque todos ellos rechazaron
indignados tan deshonrosa propuesta, sospecho que alguno pedía
interiormente a la Virgen Santísima que lograran su objeto aquellas
valientes semidiosas de San Antón y de la Chispería.

La defensa de aquella posición continuó por espacio de más de una hora,
sin más accidentes que los que he referido. Hacíamos fuego de cañón
ineficazmente, y lo sufríamos de los franceses sin poder causarles daño.
Indudablemente su intención era entretenernos, mientras se verificaba el
ataque formal por Recoletos; y seguros de su triunfo, no querían
sacrificar hombres inútilmente, lanzándolos contra posiciones que al fin
se habían de rendir. Cerca de las diez, el que nos mandaba recibió aviso
de enviar a Recoletos la gente de infantería que no necesitase, y así lo
hizo, tocándome a mí marchar entre los cien hombres destinados a aquella
operación.

Por el camino, mientras atravesamos las calles de San Opropio y de las
Flores hasta llegar a la plazuela de las Salesas, encontramos mucha
gente que corría alarmadísima, dando a entender con sus gritos y
agitación que la cosa iba mal. Extendiéndonos luego por la calle de los
Reyes Alta\footnote{Hoy de las Salesas.}, bajamos por la del Almirante a
la ronda de Recoletos, donde reinaba gran confusión. Fuerte cañoneo se
oía por detrás de la Veterinaria, edificio que Vds. habrán conocido en
el solar de la comenzada Biblioteca, y también por detrás de los Hornos
de Villanueva y del Pósito, hacia la puerta de Alcalá. El convento de
Recoletos estaba ocupado por tropa española; pero en el momento en que
nosotros llegamos casi toda la fuerza salía por ser más necesaria fuera
que dentro. En el principio del ataque, la batería puesta detrás de la
Veterinaria rechazó con tanta energía el empuje de los franceses,
mandados en persona por el mismo Emperador, que este tuvo que retroceder
a toda prisa.

Suprimid con la imaginación el barrio de Salamanca y todos los jardines
y palacios del costado oriental de la Castellana: figuraos aquella casi
desnuda planicie poblada por numerosas tropas francesas de todas armas,
con dos frentes que operaban uno contra el Retiro y la Plaza de Toros,
otra contra la Veterinaria y Recoletos, y tendréis completa idea de la
situación. En el centro de aquellas tropas y en lo que hoy es parte de
la calle de Serrano, poco más o menos entre el jardín llamado del
Pajarito y las casas de Maroto, estaba Napoleón sereno y tranquilo,
montado en aquel caballejo blanco que había pateado el suelo de las
principales naciones del continente; allí estaba disponiendo los
movimientos de sus soldados, y sin quitarse del ojo derecho el catalejo
con que alternativamente miraba ya a este punto ya al otro. Como es
fácil comprender, yo no le vi en aquella ocasión; pero me lo figuraba y
me lo figuro por lo que me contara quien lo vio muy de cerca; y por
cierto que aquel testigo ocular observó detenidamente algunos pormenores
muy curiosos de su persona, que no nombra la historia, cuales eran
ciertos monosílabos o gruñiditos que emitía mientras miraba por el
anteojo, un movimiento maquinal de apretarse el vientre con la mano
izquierda, repentinos fruncimientos de cejas y algunas veces una sonrisa
dirigida a su mayor general Berthier. Con su anteojo, su tosecilla, sus
mugidos, sus golpes en la barriga, sus polvos de tabaco y sus delgadas y
finas sonrisas, el \emph{ogro de Córcega} nos estaba partiendo de medio
a medio.

\hypertarget{xviii}{%
\chapter{XVIII}\label{xviii}}

Y digo esto porque la batería de la Veterinaria, después de una defensa
heroica, caía en poder de los franceses, precisamente en el momento en
que llegamos, refuerzo tardío, los de la puerta de los Pozos. Ya no
había nada que hacer allí. ¿Podía prolongarse aún la resistencia en el
Retiro? Así lo creímos en el primer momento; pero no tardamos en perder
esta ilusión, porque atacado aquel sitio por treinta cañones, no tardó
en entregar sus débiles tapias, que lo eran de jardín y no de fortaleza.
Así es que mientras un regimiento de voluntarios y otro de ejército
recibían a tiros con admirable arrojo en Recoletos a la primer columna
francesa que se destacó a apoderarse de la puerta, los defensores del
Retiro, faltos de recursos, de armas y de jefes, retrocedían al Prado,
fiando la defensa a las barricadas de la calle de Alcalá. El momento
aquél lo fue de gran pánico y de consternación; pero la verdad es que
entre mucha gente apocada, la hubo también resuelta y decidida.

Perdido al fin Recoletos, corrimos todos por la calle del Barquillo
hacia la de Alcalá, y cuando llegamos, ya los franceses eran dueños del
Pósito, del palacio de San Juan, y procuraban apoderarse de San Fermín y
de la casa de Alcañices. Fue muy mala idea la de construir la gran
barricada más arriba del Carmen Calzado, dejando al descubierto la calle
del Turco y todos los edificios del extremo de aquella gran vía; así es
que los imperiales, apoderáronse fácilmente de estos y abriéndose paso
después por el interior a la citada calle del Turco, dominaron de tal
modo la posición, que al cabo de un cuarto de hora de estéril tiroteo,
vimos que era preciso buscar la nuestra un poco más arriba, entre
Vallecas y el callejón de Sevilla. Se hacía fuego tenazmente desde los
balcones de ambos lados de la calle, y no había casa alguna que no fuese
improvisada fortaleza, pues la tenacidad de nuestros paisanos era tanta,
que no les acobardaba ver la creciente ventaja del enemigo, su inmensa
fuerza y arrogancia. La población, antes indecisa, cobraba ánimos al
verse invadida, y un furor parecido al del 2 de Mayo inflamaba el pecho
de sus habitantes. Escenas parciales de encarnizada y cruel lucha se
repetían a cada rato en las casas invadidas; batíanse con ferocidad a
arma blanca los que no la tenían de fuego, y el Emperador pudo ver muy
de cerca aquella enajenación popular, y aquel divino estro de la guerra,
que varias veces mostró no comprender en paisanos y menos en mujeres.

En medio de esta refriega se hizo la tercera intimación, y cuando
creímos que nuestros jefes contestarían a ella mandando redoblar el
fuego, observamos que este cesaba en la gran barricada, y que a todo
escape corría a caballo el marqués de Castelar hacia la casa de Correos,
donde estaba la Junta permanente.

---¿Qué hay, Sr.~D. Diego?---pregunté a este viéndole venir hacia mí,
con su escarapela de \emph{honrado---}. No sabía que también estaba
usted entre nosotros.

---He estado en el Retiro desde el amanecer---me contestó.---Pero ¿qué
se había de hacer, con tan mala y tan poca artillería?

---¿Pero por qué ha cesado el fuego?

---El marqués de Castelar ha pedido una tregua para consultar a la
Junta. Creo que habrá capitulación. ¿Has visto a Santorcaz?

---¿Yo?\ldots{} Ni ganas.

---Pues te andaba buscando ayer tarde con mucho empeño.

---¿También se ha batido D. Luis?

---Vaya: en el Retiro estaba hace poco gritando como un furioso y
jurando matar a los que nos han hecho traición. Pero luego nos ha
aconsejado que nos retiremos a nuestras casas, porque es imposible
pelear contra los franceses.

Subía la calle arriba mucha gente del bronce, gran número de
\emph{honrados}, voluntarios y algunas mujeres, y según las
imprecaciones que oí en boca de todos, se comprendía que los defensores
de Madrid no habían recibido bien la suspensión de armas.

---¡Como que les han untao!---decía un majo de trabuco y charpa.

---¡Que nos han vendío!---exclamaba una mujer, en quien me pareció
reconocer a la viuda de Chinitas.

---Si cojo a Castelar por delante me lo como.

---Ya me percataba yo que el Tomasillo Morla estaba vendido al Tuerto.
¿Cuánto va a que él puso los cartuchos de arena?

---¡Más vale morir que rendirse! Canallas, cobardes: si tenéis miedo,
quitaos de en medio, y dejadnos a nosotros.

---Compañeros, antes que la corte de las Españas y la mapa del mundo,
que es Madrid, caiga en poder de los gabachones, tuertos, botelludos,
dejémonos matar tras esas piedras.

---¡Que hayamos vivido para ver esto!

---Ni la Junta, ni el Consejo, ni los generales, ni el corregidor, ni
ninguno de esos Caifases tienen tanto así de vergüenza.

De este modo, en diversos estilos, expresaba el pueblo de Madrid su
rabia, no tanto por verse casi vencido, como por echar de menos el
amparo de las autoridades, y encontrarse solo entre un enemigo
formidable y un poder débil, incapaz de imitar las desesperadas
sublimidades de Zaragoza y Valencia. Así es que desde la suspensión de
la lucha cundió el desaliento tan rápidamente, y la idea de una
capitulación indispensable se apoderó tan pronto de todos los ánimos,
que las armas se caían de las manos. Cercados por poderoso enemigo, ¿qué
podía hacerse sin entusiasmo, y qué entusiasmo cabía allí donde los
jefes no contaban para nada con lo extraordinario, con lo divino, con
aquella táctica ideal y no aprendida, que o detiene las catástrofes o
las hace gloriosas, no dejando al vencedor sino lo material de la
victoria, la posición topográfica, aquello que podrá ser lo principal en
los hechos de un día, pero que es lo secundario y lo último en la
historia?

El pueblo español, que con presteza se inflama, con igual presteza se
apaga, y si en una hora es fuego asolador que sube al cielo, en otra es
ceniza que el viento arrastra y desparrama por la tierra. Ya desde antes
del sitio se preveía un mal resultado por la falta de precaución, la
escasez de recursos y la excesiva confianza en las propias fuerzas, hija
de recuerdos gloriosos a todas horas evocados, y que suelen ser
altamente perjudiciales, porque todo lo que aumenta la petulancia, lo
hace quitándoselo al verdadero valor. Lo que habían preparado las
discordias, la impremeditación y la soberbia, rematolo la excesiva
prudencia de autoridades timoratas, que, además de no ver dos palmos más
allá de sí mismas, no comprendieron que la capital no debía rendirse con
menos aparato que la última aldea de Castilla. La presencia de Napoleón
traía a aquellos pobres señores muy azorados, y tanto se preocuparon de
sus togas, de sus posiciones, de sus fajas y de sus sueldos, que con
todas estas telarañas ante los ojos era imposible que pudieran ver otra
cosa.

\hypertarget{xix}{%
\chapter{XIX}\label{xix}}

Diose orden de que los cuerpos ocuparan sus primitivas posiciones, y
partí otra vez a los Pozos, contemplando por el camino el espectáculo de
Madrid abatido y desilusionado. En algunas partes, escenas de
escandalosa protesta contra las autoridades y amenazas y gritos: en
otras, vergonzoso silencio y raras manifestaciones de la general
angustia.

Cuando llegué a la puerta de los Pozos, los soldados y voluntarios
estaban en actitud un tanto sediciosa. El Gran Capitán, que continuaba
en el jardín de Bringas, no quería creer la noticia de la próxima y ya
inevitable capitulación.

---Gabriel---me dijo,---eso que cuentan no puede ser cierto, y sin duda
es alguna estratagema de D. Tomás de Morla. ¡Cómo se miente! ¡Creerás
que unas desvergonzadas mujeres llegaron aquí diciendo que el Prado y
media calle de Alcalá estaban en poder de la Francia! Me dio tal enfado
que si no estuviera mi mujer entre las que tal insolencia decían, las
habría atravesado de parte a parte.

No quise darle un disgusto, y callé.

---Aquí hemos tenido un combate terrible---continuó.---Se atrevieron a
acercarse, y esa compañía de voluntarios salió y les hizo tan terrible
fuego que no han vuelto a asomar las narices. En tan grande acción, no
tuvimos más que cinco muertos y once heridos.

Vi en efecto, que Pujitos se ocupaba en acomodar estos últimos en las
casas inmediatas con auxilio del generoso vecindario, y que en torno a
los cinco primeros una multitud de mujeres entonaban estrepitoso
miserere de imprecaciones y lamentos. En las cuatro puertas
septentrionales no había ocurrido otra lucha importante que aquella que
Fernández me refería.

El cual prosiguió así:

---Pensar que aquí nos rendiremos, es pensar en lo imposible. Ríndase
todo Madrid; mas no se rendirán Los Pozos. ¿No es verdad, muchachos?

Los \emph{muchachos}, sentados en el suelo del citado jardín, y a la
redonda, despachaban unas sopas, acompañados de mujeres y chiquillos; y
con tanta gana comían, y tal era su pachorra y tranquilidad, que no me
parecieron dispuestos a secundar los gigantescos planes del portero de
la oficina de Cuenta y Razón. Antes bien, el uno con su reumatismo, el
otro con sus toses, y aquel con sus escalofríos, tenían cara de
satisfechos por el fin de una aventura que empezó con visos de ser broma
pesada.

---Pues si está de Dios que nos rindamos, nos rendiremos---dijo un
bravo, que lo menos tenía a cuestas sesenta años y pico.

---Hemos hecho todo lo que exigía el honor. No es posible más---dijo
otro.---Cuando los jefes han acordado la rendición, ya sabrán que es
imposible resistir.

---Yo---añadió un tercero,---he cumplido con mi deber. Lo menos he
disparado tres tiros.

---Y yo, aunque no he disparado ninguno, le cargaba la escopeta a aquel
soldadillo del bigote rubio.

---Esto no se puede oír---exclamó bramando de ira D. Santiago.---Pero
¿qué se puede esperar de unos hombres que se ponen a comer sopas, cuando
tenemos a cien varas de nosotros al vencedor de Europa? ¡Fuera de aquí,
almas de mazapán, cuerpos momios y sangre de arrope! ¿De qué os valen
esas canas que estáis deshonrando? ¿De qué vuestros años, hasta ahora no
envilecidos? ¿De qué el haber asistido a aquellas gloriosas
campañas?\ldots{} Nada, lo dicho dicho. Se rendirá Madrid; pero no se
rendirán los Pozos.

---Mira, marido mío---dijo a esta sazón doña Gregoria que en unión de
las otras vecinas, había venido con un canastillo y algo de bebida para
D. Santiago,---ya has cumplido con tu deber; ya te has portado como un
valiente, y tan verdad es esto, que por todo Madrid andan contando tus
hazañas que has hecho, y hasta el capitán general dicen que echó un
discurso poniéndote por modelo de los buenos patriotas. Basta ya, y
puesto que todo se acabó, y no hay más guerra por ahora, no seas
testarudo. ¿Qué vas a hacer tú solo?

El Gran Capitán no contestaba, y paseo arriba, paseo abajo, con el arma
al brazo, atendía tan sólo a sus agitados pensamientos.

---Dejémonos de tonterías, marido mío---añadió doña Gregoria,---y vamos
a despachar este cocidito y esta botella de vino. ¿Acaso puede Napoleón
decir que te ha vencido? Eso no, porque buen cuidado tuvo de no asomar
por aquí; que si tú lo llegas a coger\ldots{}

---Quítate de mi vista, vete de aquí---gritó de improviso el
veterano,---y no me seduzcas con tu cocidito y tu bebida, que no soy
hombre que se entrega a la molicie en días de peligro. Afuera los cantos
de sirena, y las seducciones del amor y los ricos manjares. No como: he
dicho que no como, y basta. He dicho que no volveré a mi casa vencido, y
no volveré. Se rendirá Madrid; pero yo no me rindo.

---¡Hay hombre más cabezudo!

Entonces el Gran Capitán llamó a su mujer y llevándola aparte conmigo a
un rincón de la huerta de Bringas, que era donde estábamos, le habló así
muy gravemente:

---Señora doña Gregoria Conejo, ¿cuánto hace que nos casamos?

---Cuarenta y cinco años, tres meses y nueve días, si no cuento
mal,---respondió absorta la anciana, sin comprender en que pararía
aquello.

---En estos cuarenta y cinco años, tres meses y nueve días, ¿le he dado
algún disgusto a la señora doña Gregoria Conejo?

---No, marido mío---respondió algo conmovida.

---Pues bien: si le he dado alguno, le ruego que me lo perdone, y está
dicho todo.

---Tú estás loco, Santiaguillo. ¿A qué dices esas necedades?

---¿Tiene Vd. alguna queja de su marido?

---Yo no, y como él no la tenga de mí\ldots{}

---Pues por mi parte---dijo el Gran Capitán con alguna emoción,---yo le
digo a doña Gregoria Conejo que la quiero hoy lo mismo que el día que
nos casamos, y que todavía me parece tan guapa, tan mona y tan salada
como cuando éramos novios, y que no tengo ninguna queja de ella, más que
la de no haberme dado hijos, lo cual en verdad ha sido voluntad de Dios.

---Sí, niñito mío---respondió la vieja;---pero ¿a dónde va tanto hablar?

---Esto va a que te retires y me dejes, porque si no, reñimos por
primera vez. Pero te has de ir perdonándome todo agravio que te haya
hecho en el discurso de nuestra común vida. En mi testamento te dejo
todo lo que poseo, que no es mucho, y además de las ocho misas que dejo
mandadas, harás que me digan otras ocho. Y quiero que me entierren con
mi lanza y con los dos reales que me dio D. Luis Daoiz, cuando le llevé
las botas a la calle de la Ternera, y basta ya de palabras.

---¡Ay, Santa Virgen de Maravillas, que mi marido está loco y se quiere
matar! ---exclamó doña Gregoria echándole los brazos al
cuello.---Santiaguillo, no digas tales simplezas\ldots{} ¿Me quieres
dejar viuda? ¿Qué es eso de testamentos y misas?

---He dicho que si Madrid se rinde, no se rendirán los Pozos; y si los
Pozos se rinden, no se rendirá el jardín de Bringas---afirmó secamente
el anciano, deshaciéndose de los brazos de su esposa.---¡Atrás,
seductora; atrás, sirena; atrás, flaqueza de mi valor!

---¡Bárbaro, animal!---dijo llorando la buena mujer.---¡Este pago me
das, así tratas a la que te ha querido tanto! Si fue ayer cuando nos
casamos, y me parece que te estoy viendo venir con tu gorra de cuartel,
tan garboso y tan chusco, a la reja de la casa donde yo servía\ldots{} A
ver, chiquillo, si te acuerdas de aquellas coplitas que me
cantabas\ldots{}

---Yo no estoy para coplitas, señora. Retírese Vd.

---¡Y estar una queriendo a un hombre cincuenta años, estar una
enamorada toda la vida y mirándose en los ojos de su marido, para
recibir este pago!\ldots{} Santiago, mira que me enfado. Vámonos a casa,
y maldito sea el Emperador, causante de mis desgracias, y a quien vea yo
comido de perros.

Ni los ruegos, ni las amenazas, ni los artificios de su mujer
quebrantaron la entereza de mi ilustre amigo, el cual resistiéndose a
tomar alimento, por no caer en la molicie, rechazando toda idea de
descanso, volvió a pasearse de largo a largo en la extensión de la
huerta, arma al brazo.

Y sucedió que una infinidad de chiquillos del barrio, a quienes antes se
había prohibido introducirse allí, vencieron por fin con la gran fuerza
de su curiosidad y travesura los rigores de la guardia; se colaron
repentinamente y en tropel, recorrieron la fortificación metiendo las
narices por todas partes, y tocando con sus manos los cañones y cureñas,
gozosos de ver tan de cerca todo aquel tremendo aparato. Como el asedio
se daba por concluido, nadie se cuidaba de estorbar su impertinentísima
inspección y entrometimiento. Luego que en todo pusieron las manos, las
narices y los ojos, empezaron a echárselas de soldados, dando gritos de
guerra y marchando a compás, todo según en las personas mayores habían
visto, y con estos militares aspavientos entráronse por la huerta de
Bringas adelante, batiendo cajas, disparando tiros, soplando cornetas y
relinchando al modo de caballos, todo hecho con la boca, en mil
discordes sones que atronaban el espacio. Y en cuanto divisaron a D.
Santiago Fernández, a quien los más conocían, fueron derechos a él y le
rodearon, gritando entre saltos, brincos, cabriolas y corcovos: «¡Viva
el Gran Capitán, viva el Grandísimo Capitán!»

Visto y oído lo cual por nuestro insigne veterano, parose, y quitándose
el sombrero hizo varios saludos y cortesías diciendo:

---Gracias, mil gracias, señores míos. Ya he dicho que si Madrid se
rinde, yo no me rindo.

Las aclamaciones y los chillidos siempre acompañados de zapatetas,
cabriolas y vueltas de carnero, tocaron los límites del delirio.

---Todos vosotros sois grandes patriotas, ¿no es verdad?---prosiguió mi
amigo;---y no como estos cobardes, corrompidos por los placeres. Ya veo
que la juventud vale más que la edad madura, y a mi lado os quisiera
ver, valientes españoles, defendiendo a nuestro amado Monarca.

La algazara y jaleo de los muchachos al oír esto fue tal, que no cabe en
descripción ni en pintura, pues no parecía sino que cuantos angelitos
engendraron los matrimonios de un siglo estaban allí haciendo de las
suyas. Allí vierais el correr, el atropellarse, el darse de coscorrones,
el cantar y gritar, el batir palmas, el tirar coces, el correr y dar
vueltas arremolinándose en torno de mi amigo, cuyas piernas por largo
tiempo estuvieron sin movimiento en medio de aquel zumbador enjambre.

---Tantas muestras de afecto, señores---dijo al fin,---me conmueven, y
no las puedo considerar sino como una prueba de lo bien acogida que ha
sido en Madrid mi conducta. Pero digan ustedes por ahí, que el
cumplimiento del deber no merece alabanzas, pues estas sólo son para lo
extraordinario y heroico. Mi deber es defender este sitio, y le
defenderé. Conque basta ya de aclamaciones y aplausos.

Pero que si quieres. Buena familia era aquella para hacer caso de tales
amonestaciones. Fue preciso que uno de los jefes diera orden de echarlos
fuera, y aun así costó trabajo librar a D. Santiago de la ruidosa
ovación. Además quiso nuestro coronel que todas las personas extrañas
desalojaran el recinto fortificado, y al fin, no sin esfuerzo, hicimos
salir a las mujeres, inclusa doña Gregoria, que se fue llorosa y
entristecida, encargándome que no perdiese de vista a su buen marido.

No sé si he dicho que por los Pozos había pasado poco antes a caballo D.
Tomás de Morla camino de Chamartín, donde el corso tenía su cuartel
general. Largo rato duró la conferencia con el Emperador, porque el
regreso de Morla fue muy tarde, y por cierto que al volver, su rostro
demudado y tenebroso demostraba que en la entrevista había habido sapos
y culebras. Aquel gigante con corazón de niño fue tratado por Napoleón
como un muchacho de escuela. Después se supo que el vencedor le puso
cual no digan dueñas, sacándole a relucir el haber permitido que no se
cumpliera la capitulación de Bailén, y amenazándole con fusilarle a él y
a sus tropas, si la población no se rendía antes de las seis de la
mañana del día siguiente.

La tarde pasó sin ningún acontecimiento militar digno de contarse. Los
franceses ocupaban sus posiciones sin hacer fuego, y nosotros, seguros
de que todo se daría por concluido, estábamos también quietos y en
expectativa. La agitación en el interior de la villa persistía; y según
oí, numeroso gentío, nada tranquilo por cierto, llenaba la Puerta del
Sol, con la atención fija en la casa de Correos, residencia de la Junta.

Rendido de cansancio, el gran Pujitos tendiose en el suelo junto a mí, y
me dijo:

---Ya esperaba yo esto que ha pasado. ¿No te dije que los traidores iban
a vendernos a los franceses?

---Más que a la traición---respondí con mucha tristeza,---debemos
atribuir este mal resultado a la falta de recursos para la defensa.

---¿Qué?---exclamó el héroe con mucho enojo.---¡Qué falta de recursos ni
qué niño muerto! Con los voluntarios basta y sobra. Pero, hijo, contra
traidores nada podemos, y así los vea yo podridos, y mala sarna se los
coma. Hace poco estuvo aquí el malcarado y peor chapado Santorcaz, y no
lo despabilé por aquello de que uno no quiere meter bulla en estas
ocasiones, pero\ldots{}

Y dio un resoplido que anunciaba exterminadores proyectos contra los
enemigos de la patria.

---¿Y a qué vino acá ese charlatán embaucador?

---A buscarte, muchacho. ¡Sabes que debes andarte con cuidado! Cuando le
dijimos que no estabas, dio la gran patá en el suelo y apretó los
dientes. Venían con él Majoma, Tres Pesetas y otros perdidos que ahora
le hacen la comitiva, junto con un tal Román, que fue criado de una casa
rica. Este, cuando oyó que no estabas y vio que Santorcaz daba aquella
gran patá, le dijo: «Pues esta noche no se nos escapará.» ¿Qué tal? Mala
gente es esa, Gabriel, y ya te dije que están vendidos en cuerpo y alma
a los franceses. De modo que ahora hay que huir de ellos como de la
sarna, porque los meterán en lo que llaman pulicía, que es al modo de
alguaciles, para prender al que se les antoje.

---No me prenderán a mí---dije,---por lo menos mientras sea soldado.
Después de la rendición, yo buscaré medios de que no me cojan, aunque la
verdad, amigo Pujitos, no sé por qué me quieren mal esos señores, ni por
qué hablan de si me escaparé o no me escaparé.

---Te digo que son malos más que Judas, y que ahora harán ellos migas
con los franceses, como que todos son unos, lobos y zorros\ldots{} pues,
y a todo el que tengan entre ojos le molerán a palos, si no es que me le
arman un trementorio de otrosís, y me lo empapelan y me lo ponen a la
sombra.

---En todo eso que ha dicho el amigo Pujitos---respondí,---hay mucho de
verdad. Quiera Dios no nos den que sentir esos bergantes; y si en Madrid
no podemos vivir, afuera todo el mundo y combatamos allí donde sepan
morir antes que rendirse a los franceses.

Levantose el héroe, y poniéndose la mano en el pecho, hizo exclamaciones
de ardiente patriotismo, después de lo cual nos separamos.

Al avanzar la noche, la tropa de línea que estaba en los Pozos, recibió
orden perentoria de internarse y fue que cuando la Junta acordó
formalmente la capitulación; no queriendo el marqués de Castelar
presenciar este hecho, ni tampoco que se rindiera la tropa, discurrió el
escapar con ella por la puerta de Segovia, lo que verificó con toda
felicidad a media noche. Solo los paisanos, ¿qué esperanza quedaba? Para
que la rendición de Madrid fuera honrosa, la diplomacia, no las armas,
debía hacer un esfuerzo.

Yo conté al Gran Capitán lo que pasaba, con la esperanza de que
desalentado se retirase a su casa, como habían hecho otros pobres
veteranos, convencidos de su inutilidad. Él juró y perjuró que era
imposible una capitulación acordada por la Junta, pero contra lo que yo
esperaba, de repente dijo:

---Tengo que ir a mi casa, Gabriel; ¿quieres acompañarme?

---Al instante---le contesté.

Y pedimos permiso al jefe, que nos lo concedió de buen grado. Era ya muy
entrada la noche.

\hypertarget{xx}{%
\chapter{XX}\label{xx}}

Pronto llegamos a nuestra morada de la calle del Barquillo. Abrió mi
amigo la puerta de su casa, con llave que consigo llevaba, subimos,
abrió la entrada de su domicilio de la misma manera, y encontrámonos
dentro de la salita donde tantas veces me ha visto el discreto lector en
compañía de mis amables vecinos. En la pared del fondo, donde desde
inmemoriales tiempos tenía asiento la lanza consabida, había una especie
de altarejo, sobre cuya tabla, dos velas de cera puestas en candeleros
de azófar, alumbraban una imagen de la Virgen de los Dolores, un San
Antonio y otros muchos santos de estampa, que de los cuatro testeros
habían sido descolgados para congregarlos allí. Algunas cintas y lazos a
falta de flores, servían de adorno al improvisado tabernáculo, con
varios jarros y cacharros antaño lujosos y bonitos, pero ya
perniquebrados, mancos y heridos. Delante de todo esto, estaba el sillón
de cuero, y sentada en él doña Gregoria, profundamente dormida. La pobre
mujer que de tal modo se había rendido al cansancio tenía la cabeza
inclinada sobre el pecho, aún humedecida la cara por recientes lágrimas,
y sus cruzadas manos indicaban que el sueño la había sorprendido en lo
mejor de su fervorosa oración.

Quedose suspenso el espeso al verla, y después me dijo:

---Gabriel, no hagamos ruido, porque no se despierte; que más vale que
descanse la pobrecita.

Después llegándose a una cómoda vieja que en un rincón había, añadió en
voz muy baja:

---Aquí en la tercera gaveta está mi testamento: y en esta otra todo el
dinero que tengo ahorrado, con el cual mi mujer puede mantenerse en lo
que le quedare de vida, que no será mucho. Voy a escribir mis últimas
disposiciones. No chistes ni me respondas nada.

Y acto continuo sentose junto a la mesilla y con una pluma de ganso mal
cortada trazó sobre un papel dos docenas de torcidas líneas.

---Aquí dispongo---añadió alzando la vista del papel,---que las misas me
las digan en San Marcos, donde está enterrado D. Pedro Velarde, ese
valiente entre todos los valientes. En cuanto a mis huesos, no dispongo
nada, porque no sé dónde caerán.

---Todavía está Vd. con esas manías---dije.---Hablaré en voz alta para
que despierte doña Gregoria y le ponga a Vd. las peras a cuarto.

---No harás tal, porque te estrangularé; que no quiero que ella abandone
su blando sueño para pasar amarguras. Aquí en esta primera gaveta dejo
mi última disposición.

Y luego levantándose y acercándose de puntillas a su mujer, la contempló
un buen espacio, pálido y conmovido: después de un rato, llevome a la
alcoba inmediata, y sentándose en la cama en sitio desde el cual, al
través de la mampara medio abierta, se veía el rostro de doña Gregoria
iluminado por las luces del altar, hablome así:

---Si algo enflaquece mi ánimo, es la vista de mi inocente esposa, a
quien voy a dejar viuda. Te confieso que al considerar esto, se me
nublan los ojos, se me oprime el corazón y estoy a punto de dar al
traste con toda mi fiereza. ¿No la ves desde aquí? Parece que fue ayer
cuando nos casamos; parece que no han pasado cuarenta y cinco años, y se
me representa con la misma celestial figura que tenía allá por los
tiempos de Maricastaña, cuando yo iba a la reja, llevándole media libra
de peras en el pañuelo o un par de mantecadas de Astorga. En todo este
tiempo no me ha dado nada que sentir, y hemos vivido juntos como dos
palomos, queriéndonos lo mismo que el primer día. ¿No la ves desde aquí?
¿No ves su hermosa cara, tan serena y tranquila a pesar de su tristeza?
Yo la estoy viendo con sus cabellos de oro, con su boquita encarnada
como un casco de granada, con sus dulces ojos azules, que al mirarte
parece que se abre el cielo delante de los tuyos, estoy viendo el nácar
de su tez y su airoso y gentil cuerpecito, lo mismo que su garganta
alabastrina. ¡Oh, Dios mío! ¡Tan hermosa, tan buena y tan desgraciada!

Bien por efecto de la imaginación, ofuscada por aquellas palabras, bien
porque la situación diese a doña Gregoria ideales encantos, lo cierto
fue que a pesar de sus blancos cabellos, de su tez arrugada y de su en
tantas partes notoria vejez, la estaba viendo tan hermosa como el Gran
Capitán decía. ¡Milagroso efecto del pensamiento!

---Mira, Gabriel; desde que nos vimos hace cincuenta años, nos quisimos:
vernos y querernos fue todo uno, lo mismísimo que cuentan de los amantes
de Teruel. Un lustro duró nuestro noviazgo, porque yo no tenía posibles;
pero desde el primer día concertamos la boda. Durante aquel tiempo, ni
riñas, ni bromicas, ni celillos. Nunca hemos tenido celos el uno del
otro, porque desde el primer día la confianza fue nuestro norte. Todos
me tenían envidia. ¡Ay! Cuando nos casamos fuimos tan felices, que no
hubiéramos cambiado nuestra casa por siete imperios. Y desde entonces,
hijo, esta felicidad no se ha alterado. ¡Ay! se me parte el corazón al
pensar que desde mañana se acostará sola en esta cama, que por cuarenta
y cinco años nos ha visto juntitos.

Al decir esto, el Gran Capitán se llevó el pañuelo a los ojos para secar
sus lágrimas.

---Vamos, amigo---le dije;---de veras no sé si reírme o enfadarme,
oyendo lo que usted dice. ¿Está loco por ventura?

---Si tú no comprendes esto---me contestó,---es porque eres un simplón y
un majadero egoísta. ¿Tú sabes lo que significa cumplir uno con su
deber? ¿Tú sabes lo que significa el honor? y si sabes todo esto,
¿ignoras lo que es la honra de la patria, que vale más que la propia
honra? Escúchame bien: si me causa angustia y pesar la consideración de
la viudez de Gregorilla, mayor, mucha mayor pena me causa el considerar
que la capital de España se entrega a los franceses. Esto es terrible,
esto es espantoso, y no vacilaría en dar mil vidas y en sufrir todos los
tormentos por impedirlo. ¡España vencida por Francia! ¡España vencida
por Napoleón! Esto es para volverse loco; ¡y Madrid, Madrid, la cabeza
de todas las Españas en poder de ese perdido! De modo que una Nación
como esta, que ha tenido debajo de la suela del zapato a todas las otras
naciones, y especialmente a Francia; de modo que esta Nación que antes
no permitía que en la Europa se dijera una palabra más alta que otra,
¿ha de rendirse a cuatro troneras hambrones? ¿Cómo puede ser eso? Eche
Vd. a los moros, descubra y conquiste Vd. toda la América, invente usted
las más sabias leyes, extienda Vd. su imperio por todo lo descubierto de
la tierra, levante Vd. los primeros templos y monasterios del mundo,
someta Vd. pueblos, conquiste ciudades, reparta coronas, humille países,
venza naciones, para luego caer a los pies de un miserable
Emperadorcillo salido de la nada, tramposo y embustero. Madrid no es
Madrid si se rinde. Y no me vengan acá con que es imposible defenderse.
Si no es posible defenderse, deber de los madrileños es dejarse morir
todos en estas fuertes tapias, y quemar la ciudad entera, como hicieron
los numantinos. ¡Ay! todos mis compañeros se han portado cobardemente.
España está deshonrada, Madrid está deshonrado. No hay aquí quien sepa
morir, y todos prefieren la mísera vida al honor.

---Pero cuando no se puede triunfar---le dije,---es una temeridad seguir
peleando, y más vale guardar la vida para emplearla con éxito en mejor
ocasión.

---¡Simplezas y tonterías! El honor mandaba a los madrileños morir antes
que rendirse, y el honor nos manda a los de la puerta de los Pozos, que
muramos todos allí antes que entregarla.

Pues no creo que estén dispuestos a ello.

---Pues yo lo estoy, porque mi conciencia, que es la voz de Dios, me lo
manda. Se rendirá la puerta; pero el jardín de Bringas está bajo mi
mando, y el que quiera entrar en él pasará sobre mi cadáver.

---¡Temeridad loca, y hasta ridícula!

---Así será para los que no tienen idea de la honra de la patria, y para
los que no ven nada más allá de esta ruin existencia, ni nada más allá
del pan que comen todos los días.

---Entregarse de ese modo a la muerte es un suicidio, y el suicidio es
un gran pecado.

---No es suicidio, no. La ley ineludible de la patria me ha puesto en un
lugar que debo defender aun a costa de la vida. ¿Que vienen fuerzas
superiores? ¡pues vengan! La patria me manda esperar tranquilo, y la ley
me veda el apartar los pies de aquel sitio. ¿No morían los mártires por
la religión? Pues la patria es una segunda religión, y antes que faltar
a su ley, el hombre debe morir. ¿Y qué es la muerte? Los necios se
asustan de la muerte, porque la muerte les quita el comer y el gozar.
¡Mentecatos! ¿Por ventura, no son mejor comida y mejor goce los de la
bienaventuranza eterna? Ve ahí a mi esposa. Cierto que me aflige
dejarla; pero sé que la perderé de vista tan sólo por algún tiempo, y
que sus virtudes la llevarán luego a donde la tenga delante de mis ojos
durante todas las eternidades, sin cuya compañía creo que el mismo cielo
me sería fastidioso. ¡Morir! ¡Ahí es gran cosa morir, y apañado tienes
el ojo! ¿Pues acaso el morir es mal que puede compararse siquiera al
dolor de un rasguño recibido en la tierra? Y si el morir no es nada para
el miserable cuerpo, ¡cuán grande y fausto suceso no es para nuestra
alma, mayormente si por la nobleza de nuestro fin nos empingorotamos
sobre todas las cosas nacidas! ¡Morir por la patria, morir en el puesto
que a uno le marca su deber, morir no por conquistar un pedazo de
tierra, ni por un cacho de pan, ni por una baja ambición, sino por una
cosa que no se ve, ni se toca cual es una idea y un sentimiento puro!
¿No es equipararnos a los santos del cielo y acercarnos a Dios todo lo
que acercarse puede una criatura?

Dicho esto, calló. No le contesté nada, porque tanta grandeza me tenía
anonadado.

Al cabo de un buen espacio volvimos de la alcoba a la sala; acercose él
con pasos muy quedos a doña Gregoria, y le dio muchos besos, tan en flor
por no despertarla, que apenas tocaban sus labios el arrugado cutis de
la anciana.

Luego enjugose las lágrimas, y dirigiendo una mirada en redondo a todos
los objetos de la sala, me dijo con voz grave y entera:

---Gabriel, vamos.

\hypertarget{xxi}{%
\chapter{XXI}\label{xxi}}

No valían razones contra él, y cuanto yo pudiera decirle habría sido
predicar en desierto, razón por la cual determiné cesar en mi
obstinación, reservándome el emplear después cualquier estratagema para
impedir una desgracia. Como durante la visita a la casa había
transcurrido mucho tiempo, cuando salimos principiaba ya a clarear la
aurora, y advirtiendo por las calles más gente de la que en tales horas
suele encontrarse, nos fuimos a curiosear un poco, antes de volver a los
Pozos. Serían las seis cuando entrábamos en la calle de Fuencarral, y
como era esta la hora señalada para la rendición, subían y bajaban por
la citada vía numerosos grupos de hombres, armados unos, sin armas
otros, pero todos puestos en mucha agitación. Había quien en alta voz
declamaba contra lo capitulado, poniendo a Morla, a la Junta y a
Castelar como ropa de pascua; otros se desahogaban insultando a
Napoleón; muchos rompían las armas arrojándolas al arroyo; no faltaba
quien disparase al aire los fusiles, aumentando así la general
inquietud; y por último, hacia el Arco de Santa María, vimos algunos
frailes dominicos y de la Merced que arengando a la muchedumbre
procuraban calmarla.

---Vamos, corramos a nuestro puesto---dijo Fernández,---no sea que nos
tengan preparada una sorpresa.

---Aún no es la hora designada---dije procurando entretenerle de modo
que llegáramos tarde.

---¿Cómo que no?---clamó con exaltación, avivando el paso.---Corramos,
no sea que lleguemos tarde y entreguen los Pozos. Mal hemos hecho en
abandonar nuestro puesto por una necia sensiblería. ¡Quién sabe lo que
hará esa gente si no estoy yo por allí! Corramos, pues ya he dicho que
se rendirá Madrid, que se rendirán los Pozos; que se rendirá el jardín
de Bringas; pero que el Gran Capitán no se rinde.

Empezamos a correr, cuando detúvome de improviso un hombre que en
opuesta dirección venía. Era Pujitos.

---Gabriel---me dijo muy sofocado;---vuelve atrás, no vayas a los Pozos;
echa a correr y escapa como puedas.

---¿Por qué? ¿Qué pasa?---preguntó mi amigo con la mayor zozobra.---¿Ha
venido Napoleón en persona?

---¡Qué Napoleón ni qué Juan Lanas!---añadió Pujitos empujándome para
que retrocediera.---Corre presto, que si llegas allá te echan mano.
Ahora mismo han estado esos perros por ti.

---¿Quién?

---¿Quién ha de ser sino D. Luis Santorcaz, ese que llaman Román, y los
tres o cuatro pillos que andan con ellos?

---¿Y a mí para qué me buscan?

---Para prenderte.

---¿Y quién es él para prenderme?---exclamé lleno de ira.---¿Pero no
dijeron por qué me quieren prender? ¿Qué he hecho yo?

---Sí dijeron, y es un aquel de traiciones que has hecho, y no sé qué
diabluras. Conque a correr. Mira que vienen. Aire a los pies y buenos
días.

---¡Eh!\ldots{} Basta de simplezas---dijo el Gran Capitán,---y no me
detengo más, que hago falta en otra parte.

Y marchose resueltamente hacia arriba sin decir nada más. Luego que me
quedé solo con Pujitos, proseguimos nuestro altercado, él queriendo
obligarme a que retrocediera, y yo obstinándome en seguir, pues me
parecía una fábula aquello de mi prisión y la mudanza de Santorcaz y
Román en alguaciles, y sobre todo en perseguidores míos por traiciones
que yo no había soñado en cometer. Pero al fin logró convencerme
recordando pasados sucesos que podían explicar, ya que no justificar,
aquel hecho como una venganza; creí prudente seguir el consejo de mi
compañero de armas, hombre que no por ser tonto dejaba de ser honrado, y
me escurrí a buen andar en dirección al Espíritu Santo.

Cerca de la calle Ancha tuve un feliz encuentro en la aparición de mi
reverendo amigo el fraile mercenario, que seguido de mucha gente venía
en dirección opuesta.

---¿A dónde vas, Gabriel?---me dijo deteniéndome.

---Voy huyendo, padre---le respondí;---huyendo de infames enemigos que
me persiguen sin motivo alguno.

---¿Quién, quién es el atrevido que te acosa?---exclamó briosamente.

---Hombres pérfidos, hombres inicuos que han sido espías de los
franceses, y ahora aparecen como oficiales de la justicia.

---¿Pero de qué justicia? ¿Quién nos manda? Sepámoslo de una vez. ¿Nos
manda aún nuestra Sala de Alcaldes, o nos manda un bigotudo general
francés, en nombre de Napoladrón? ¿Ha capitulado ya la plaza?

---No lo sé, padre; pero es lo cierto que esos hombres me buscan para
prenderme, y con autoridad o sin ella, llevan sus reales despachos en
toda regla, que maldito sea el que se los dio para que satisfagan
infames venganzas personales.

---Vamos a ver qué es eso\ldots{}

---No, padre, yo no pienso ver nada más que la calle por donde corro,
porque conozco la clase de gente en cuyas manos voy a caer.

---Por la Santísima Virgen del Carmen, que nadie te ha de tocar el pelo
de la ropa, al menos yendo conmigo. Ea, señores---añadió Salmón
volviéndose a los que le seguían,---me voy a mi casa. Se despide de Vds.
el padre Salmón, de la orden de la Merced; ya no soy nada, hijos míos;
ya no tenéis padrito Salmón; ya no tenéis quien os predique, ni quien os
aconseje, ni quien os diga cosas alegres. Se acabó todo: España es de
los franceses; adiós frailes y monjas, que a todos nos van a quitar de
en medio, hijos míos, y no hagáis pucheros, que de nada valen ahora
estos pucheros, pues no se defiende la religión con lagrimitas\ldots{}
No lloréis, que tarde \emph{piache}, como dijo el otro, y sucumbamos.
Adiós, hijos míos, que ahora os quieren hacer a todos herejes, y los
religiosos estamos de más. Yo os echo la bendición, y cuidado, cuidadito
con los pecados. Y tú, joven desgraciado, arrímate a mí, que aún nos
queda un poquillo de influjo, y nadie te hará nada yendo en mi compañía.
Ven conmigo a la Merced, y allí procuraremos ponerte en salvo.

Cuando marchamos juntos hacia la calle Ancha, oímos en derredor nuestro
estentóreas y acaloradas voces de hombres y mujeres que gritaban: «¡Viva
el padre Salmón! ¡Muera Napoleón! ¡Muera el rey de Copas!»

---En mi convento estarás seguro---me dijo luego el mercenario,---hasta
que puedas salir de Madrid. ¿Piensas salir?

---En cuanto pueda, padre; no puedo ni debo estar más aquí.

---Haces bien: algunos compañeros míos piensan marcharse también a
levantar por ahí el espíritu de los pueblos. Yo no saldré de Madrid,
porque mi naturaleza es tan delicada y flatulenta, que no resiste los
trabajos, hambres y estrecheces de una misión. A la casa de Madrid me
atengo: ni quito ni pongo rey, y aunque dicen que el hermano de Copas
nos quiere quitar, todo es filfa, hijito mío. Yo sé que andan por Madrid
emisarios del Emperador que nos hacen la mamola a cencerros tapados para
que le rindamos pleito-homenaje y transijamos con él, requisito
indispensable para tratarnos a maravilla, por lo cual opino que tan bien
se sirve con Pedro como con Juan, y adelante con los faroles, porque si
tienes hogazas no pidas tortas, y si te dan la vaquilla acude con la
soguilla, que como dijo el otro, mano que da mendrugo, buena es aunque
sea de turco.

Tan sumergido estaba yo en mis pensamientos que no contesté a mi amigo,
si bien mi silencio no fue parte a que dejara de seguir hablando por
todo el trayecto, durante el cual no nos ocurrió desgracia alguna, ni
tuvimos ningún mal encuentro.

---Ya estamos en casa---me dijo cuando entramos.---Sube y probarás de
unas magritas de la olla de ayer que el refitolero me ha guardado para
hoy, poniéndolas con arroz; y te advierto que en todo lo que sea de
arroz soy una especialidad, y a mí se me debe la introducción de las
almejas y de la canela en la valenciana paella.

Entramos en su celda, donde me dejó, volviendo al poco rato con un
cazuelillo debajo del manteo, y con esto y una botella que sacara de la
alacena juntamente con una cesta llena de pedazos de pan, higos,
aceitunas, nueces, embutidos, queso, dátiles y otras viandas, aderezó un
almuerzo que me vino de perillas.

---Esta misma celda en que estás, y que es la mía---dijo mientras
comíamos,---fue ocupada hace más de doscientos años, allá en los de
1620, por aquel insigne mercenario fray Gabriel Téllez, a quien
generalmente se conoce por el maestro Tirso de Molina. Es fama que en
este sitio, y quizás en esta misma mesa, escribió su célebre
\emph{Crónica de la Orden}, porque comedias se cree que no hizo ninguna
después de meterse a fraile.

---¿No le ha dado a Vuestra Paternidad por hacer comedias?---le
pregunté.

---Hombre, algunas he hecho, y ahí están pudriéndose en aquella alacena.
Mas no he intentado que se representen, porque el prior nos lo prohíbe,
aunque son todas devotas. Una hice que no me parece mala, y se titula
\emph{El Santo Niño de la Guardia}. No deja de tener su sal otra que
compuse con el rótulo de \emph{La tutora de la Iglesia y doctora de la
Ley}, toda en sonetos arreo, entreverados con lo que se llaman séptimas
reales; y me daba tanto el naipe por estas obrillas que enjaretaba dos
en una semana, y si no me lo prohibieran, le hubiera echado la
zancadilla a Bustamante que escribió trescientas veintinueve comedias de
santos.

---¿Y en qué se ocupa ahora Vuestra Paternidad?

---¿En qué me he de ocupar, muchacho, sino en hacer jaulas de grillos?
¿No sabes que soy el primer jaulista de Madrid? Pues a fe que me dan
poco trabajo las tales obras. Mira cuántas hay allí. Aquella que tiene
tres pisos, con dos hermosísimas torres y su reloj figurado en el
centro, es para las monjas de Constantinopla; y aquella otra redonda que
está por concluir, para las Carmelitas Descalzas que ha un mes me tienen
loco con la dichosa obra.

En efecto, todo un rincón de la celda estaba lleno de jaulas hechas y
por hacer, con todos los materiales y herramientas propias de aquel
oficio. De libros no vi sino los folletos y papeles que días antes
recogió en casa de Amaranta.

---Yo soy un hombre que abomina la holgazanería---continuó Salmón,---y
no me parezco a otros de esta misma casa que no se ocupan en maldita la
cosa; aunque hay algunos, la verdad sea dicha, como el padre Castillo,
que noche y día están metidos en un mar de libros y papeles.

---Y en verdad, padre---le dije,---ya que no hay cautivos que redimir,
todos Vds. deberían pasar el tiempo en algún útil menester.

---Pues hay frailes que como no sea tirar a la barra en la huerta y
jugar al tute en la solana, no hacen nada. Y si no, en la celda de al
lado tienes al padre Rubio que se pasa la vida haciendo acertijos y
enigmas, los cuales envía a las monjas para que ellas le devuelvan la
solución y nuevos problemas, y tienen establecidas ganancias y pérdidas
para el que acierta y para el que yerra, las cuales pérdidas y ganancias
consisten siempre en algo de condumio. ¿Pues y el padre Pacho, que se ha
dedicado a hacer punto de media y labra unos primores?\ldots{} Esto es
andar a mujeriegas, lo cual no me gusta. Yo al menos he hecho en lo
tocante al arte eminentísimo de las jaulas adelantos admirables, y
además me dedico a la medicina, para lo cual, con aquel Dioscórides que
está a la cabeza de mi cama tapando la escudilla, me basta y me sobra.

Por estos caminos siguió nuestra conversación, hasta que me entró gana
de dormir. Mi amigo pidió permiso al prior para que me quedase allí todo
el día y aun toda la noche, refugiado contra una injusta persecución, y
me llevaron a una celda vacía, donde en lecho muy blando me acomodé,
rindiéndome de tal modo el sueño, que hasta el siguiente día no di
acuerdo de mí.

\hypertarget{xxii}{%
\chapter{XXII}\label{xxii}}

Cuando me levanté, y hube despachado el desayuno que con sus propias
caritativas manos me llevó el padre Salmón, salí al claustro alto, donde
mi amigo me dijo:

---Hay grandes novedades. Ayer a eso de las diez, se entregó la plaza a
los franceses, una vez firmada la capitulación por el Emperador en su
cuartel general de Chamartín.

---¿Y ha habido algo en los Pozos?---pregunté acordándome pesaroso del
Gran Capitán.

---Creo que es el único punto donde hubo alguna resistencia, pues de
todos los demás se apoderó sin dificultad el general Belliard,
gobernador de la plaza.

Salió al encuentro de Salmón un fraile pequeño y viejo, que se apoyaba
en un palo; hombre al parecer enfermo y de mal genio, que dijo:

---¿Sabe su merced, Sr.~Salomón jaulista, las bases de la entrega?

---Hermano Palomeque, no las sé; pero creo que ha llegado fray Agustín
del Niño Jesús, el cual dicen tiene una copia que le suministró un
individuo de la Junta.

---¿Qué vuelta por el claustro, padre Palomeque?---dijo un frailito
joven, barbilindo, ancho de cuello, pulcro de rostro, arrebolado de
nariz, nimio de cerquillo y con cierto aire galán, el cual de improviso
se unió a nuestro grupo.

---Lo que hay---contestó Palomeque con rabia, dando un fuerte bastonazo
en el suelo,---es que anoche me han robado una gallina, de las seis que
tenía en el corral, y ¡ay del pícaro zorrón si le descubro, que por
nuestro santo hábito, si fuera cierta la sospecha que tengo de un fraile
madamo y almibaradillo, yo le juro que me la ha de pagar!

\emph{---¡Oh curas hominum!} \emph{¡Oh quantum est in rebus inane!}
\emph{¡Oh cupidinitas gallinacea!} ¿Y todo ese enfado es por una polla
seca y encanijada, con cuyo caldo se podía administrar el bautismo?

---Basta de bromas; y si era encanijada, no la tenía yo para ningún
zángano ---exclamó Palomeque.---Pero a otra, y díganme de una vez en qué
términos se ha hecho esa maldita capitulación. Por ahí asoma fray
Agustín del Niño Jesús.

Llegó en efecto con paso grave el tal Niño Jesús, que era un fraile
altísimo de estatura, moreno, de pelo en pecho, de aspecto temeroso,
ojos fieros y una voz, por raro contraste, tan infantil y atiplada, que
parecía salir de otra garganta que la suya. Seguíanle otros dos frailes.

---Vamos a ver, señor músico, ¿qué dice esa minuta?---le preguntó el
fraile barbilindo.

---Ahora lo veredes dijo Agrages---fue la contestación del padre
Agustín.---Creo que Napoleón ha aceptado todos los artículos, excepto
dos o tres de los menos importantes.

---El primero---dijo Salmón,---habla de la conservación de la religión
católica, sin que se consienta otra.

---Justo---respondió el Niño Jesús sacando un papel;---y el segundo de
la \emph{libertad y seguridad de las vidas y propiedades de los vecinos
de Madrid}. Igualmente establece el respeto a \emph{las vidas, derechos
y propiedades de los eclesiásticos seculares y regulares de ambos sexos,
conservándose el respeto debido a los templos, todo con arreglo a
nuestras leyes.}

---Como no lo han de cumplir---indicó Palomeque,---excusado es que lo
digan. Siga adelante.

---¿Para qué ha de leer más? Lo que sigue poco interés tendrá y apuesto
a que habla de que si las tropas saldrán de Madrid con los honores de la
guerra o no.

---Justo---dijo fray Agustín,---y también hay otro artículo en que se
establece que no se perseguirá a persona alguna por opinión ni escritos
políticos.

---Eso está muy mal pensado y peor resuelto---dijo otro de los presentes
que era el padre Rubio, fabricador y artífice de acertijos,---porque si
no quitan de en medio a los franc-masones y diaristas\ldots{}

Luego el frailito almibarado, que era nada menos que maestro de
teología, llegose a Salmón y le dijo:

---¿Se atreve Vuestra Paternidad a echar dos tantos a la barra esta
tarde después de la siesta?

---¿Pues no me he de atrever?---contestó.---Y tú, Gabriel, ¿juegas a la
barra?

---Este joven---dijo el maestro de teología con bondad,---¿es aquel
portento de las humanidades, aquel consumado latinista de quien Vuestra
Merced me habló?

---El mismo que viste y calza, o por mejor decir, el segundo Pico de la
Mirandola. Puede examinarlo Vuestra Merced y verá lo que son castañas.

Yo repetí que no sabía palabra de latín, y que toda mi fama en dicha
lengua provenía de una equivocación.

\emph{---Modestus es}---dijo el teólogo.---Y puesto que es Vd. tan gran
latino, contésteme a esto: ¿qué quiere decir \emph{Vino a lo que vino}?

---Eso no es latín, sino castellano---dijo Salmón.

---¡Oh!---exclamó el otro batiendo palmas.---Los dos se atascaron.
¿Conque castellano? Pues es tan latín como el \emph{Arma virumque}.
\emph{Vino a lo que vino}, o lo que es lo mismo \emph{vi no aloque
vino}, que traducido literalmente, quiere decir \emph{con fuerza nado y
me alimento con vino}.

---Este fray Jacinto de los Traspasos de María es un pozo de
ciencia---dijo Salmón.---Gabriel, te atascaste.

---Y díganme ustedes---prosiguió el otro,---¿qué quiere decir
\emph{Archiepiscopi toletani onerati sunt mulieribus}?

---Eso más claro es que el agua, mi señor don teólogo---repuso
Salmón.---Es una blasfemia y calumnia; pero valga lo que valiere, quiere
decir, salva la intención, que los arzobispos de Toledo están cargados
de mujeres.

---¡Oh gansos, oh acémilas! Ya les cogí otra vez---dijo fray
Jacinto.---El \emph{archiepiscopi} que parece nominativo plural, es
genitivo singular. De la palabra que suena \emph{mulieribus} hago dos, a
saber; \emph{muli æribus} y resulta: \emph{los mulos del arzobispo de
Toledo están cargados de riquezas}. ¡Ajajá! Pues y lo de tú comes
caracoles, ¿qué significa?

---¡Oh! No estoy para quebraderos de cabeza---replicó Salmón.---Dejemos
eso, y ya que en el latín me ha vencido, esta tarde le venceré a la
barra.

---Esta tarde no---dijo Rubio,---pues fray Jacinto ha prometido venir
conmigo a ver a las Constantinoplas, que están locas por conocerle.

---Y Castillo, ¿dónde está?---preguntó Palomeque.

---En misa.

---¡Oh \emph{patres conscripti!}---dijo otro fraile que vino a toda
prisa por el claustro adelante.---¡Grandes y estupendas novedades! Han
llegado tres consejeros de Castilla, y están en conferencia con el
prior.

---¿Y a qué vienen esos consejeros del diantre?

---Según he olido, les manda Napoleón para que nos emboben, por ver si
consigue que una diputación de regulares de todas las ordenes vaya a
cumplimentarle y hacerle randibú en su cuartel de Chamartín.

---Antes al demonio.

---¿Conque \emph{randibú} al azote de los pueblos, al enemigo de la
religión, al carcelero de nuestro Rey? Muy bien; tras de cornudo
aporreado, y vengan palos, que con besar la mano que nos los da, todo
queda concluido.

---Como se han de levantar contra Napoleón hasta las piedras, y al fin
ha de marcharse con su hermano, excusado es andarse con mieles.

A esta sazón llegó el padre Castillo, que venía de decir su misa, aquel
discreto y agudo fraile que en casa de la señora condesa había hecho el
expurgo de libros.

---Padre Castillo, ¿conque tenemos visita de consejeros de Castilla,
para que nos humillemos ante Napoleón?

---No sé nada de esto.

---Yo estoy determinado a salir de Madrid e irme por esas provincias a
predicar la guerra, juntando gente armada---dijo Rubio.

---Y yo, como me suelte por tierra del Barco de Ávila y eche allá cuatro
sermones, levanto hasta las piedras---afirmó el Niño Jesús.

---Yo no me moveré de aquí---dijo Castillo.---En esta casa me mandan los
estatutos que resida, y aquí residiré mientras no me echen. Fundose
nuestra orden para redimir cautivos, no para predicar guerra ni armar
soldados.

---Muy bien dicho; mas tampoco se fundó para que la patearan Emperadores
y la escupieran Juntas.

---Dios hará de nuestra orden lo que fuese servido---repuso
Castillo.---En tanto, nosotros nos estamos mejor en nuestra casa, que
por montes y valles incitando a los hombres a matarse. Y no es que
dejemos de ser patriotas. Más harán las oraciones de un fraile piadoso
en pro de nuestros ejércitos, que los sermones furibundos y crueles de
esos desgraciados que con los hábitos al cinto se han lanzado a la
guerra. Y dígame el buen Niño Jesús, ¿le parece meritoria y digna de un
cristiano y de un sacerdote la conducta de ese dominico que no quiero
nombrar y que se ha señalado por sus sanguinarias excitaciones a la
matanza de franceses? No, nada que sea contrario a las generales leyes
de la caridad debe sacarnos de nuestra ordinaria vida.

---Con buenas retóricas se viene ahora el padre Castillo---dijo otro de
los presentes.---No, si no hagámonos miel, para que nos papen imperiales
moscas.

---Dígame---preguntó un tercero,---¿ha oído decir el Sr.~D. Librote y
Cata-pergaminos, que Napoleón va a reducir el número de regulares a la
tercera parte? Pues sí, eso está muy bonito. Apláudalo el padre
Castillo. Y nosotros veámoslo y callemos, ¿no? ¡Pues me gusta! De modo
que si un conquistador atrevido pone en peligro nuestro instituto, lo
daremos por bien hecho.

---¿Con que reducirnos a una tercera parte?---dijo Salmón.---¡Bonita
invención! Esas son las tan decantadas novedades de los filósofos y de
todos esos masones a la francesa que hay ahora.

---No disputaré sobre si es conveniente o no reducir el número de
conventos---dijo Castillo.---Cuestión es esta delicada y sobre la que se
podría hablar mucho. Lo que sí afirmo es que la reducción del número de
regulares, y las ideas de poner coto a tantas fundaciones son bastantes
antiguas, y se han ocupado de ello mil eminentes repúblicos. Ya saben
todos que en el siglo pasado se ha clamoreado bastante sobre esto. ¿Y
qué más? A principios del décimo sétimo siglo, cuando aún no se soñaba
en enciclopedias, ni en revoluciones, ni en logias, ni en filosofías,
personajes respetables y entre ellos algunos españoles sapientísimos se
expresaron en igual sentido. Como me dedico a buscar papeles viejos,
¡vean mis caros hermanos la casualidad! en estos días he encontrado dos
que vienen como de molde a terciar en esta contienda.

Y al punto fue a su celda, que muy cerca estaba, y volviendo con dos
libros viejos, los mostró a sus hermanos.

---Aquí están---dijo.---Uno es el \emph{Memorial que al Rey D. Phelipe
III dio en su consejo de Estado fray Luis de Miranda, lector jubilado de
la orden de San Francisco, acerca de la ruyna y destrucción que
amenazaba a la república y monarquía de España, si con presteza no se
acude al remedio}. Las causas y razones que expone son:
{\textsc{Primera}}, \emph{la muchedumbre de hacienda que de secular se
está convirtiendo en eclesiástica.} {\textsc{Segunda}}, \emph{las
innumerables personas, que por sus particulares fines, de seglares se
hacen religiosos, sin aver de ello necesidad, antes con daño de las
mismas religiones.} Esto se escribía en los primeros años del siglo
décimo sétimo, y si el mal era cierto, juzguen vuestras paternidades si
habrá aumentado, no habiendo nadie acudido al remedio. El otro libro se
titula \emph{Discurso del doctor D. Gutiérrez, marqués de Careaga, en
que intenta persuadir que la monarquía de España se va acabando y
destruyendo a causa del estado eclesiástico, fundación de Religiones,
Capellanías, Aniversarios y Mayorazgos.} Esto está impreso en 1620. De
modo, hermanos míos ---añadió con zunga el buen Castillo,---que hace
doscientos años hubo quien ya dio en la flor de decir que éramos muchos.
Ahora, pues, carísimos, cada uno meta la mano en su pecho, consulte a su
conciencia y pregúntese a sí mismo si cree estar de más:
\emph{intelligenti pauca.} ¿Y esas gallinas, padre Palomeque, cuántos
huevos han puesto en la semana? ¿Y cómo van esas jaulas, padre Salmón?
¿Qué me dice Vuestra Paternidad de aquellos enigmillas tan reservados
que le enviaron ayer las Constantinoplas, padre Rubio? ¿Halos acertado
ya? ¿Y qué tal van esos toques de flauta, fray Agustín del Niño Jesús?

Y así fue dirigiendo a todos graciosas pullas, si bien ellos no se
irritaban por esto, gracias al respeto que le tenían. Con esto y con la
retirada de Castillo se desbarató el corro y casi todos fueron a husmear
a la puerta de la celda del prior por ver si descubrían cuál era la
misteriosa comisión de los consejeros de Castilla. Cuando Salmón y yo
íbamos a espaciarnos un poco por la huerta, vimos un fraile anciano que
leyendo devotamente su libro de oraciones se paseaba en el claustro
bajo. Pregunté a mi amigo quién era aquel venerable sujeto, y me dijo:

---Este es el padre Chaves, el más piadoso y recogido de todos los
frailes de este convento, si bien me parece que es algo mentecato. No
hace más que rezar, leer libros santos y asistir a todos los enfermos de
la casa. Hace catorce años que no ha salido una sola vez a la calle. No
recibe regalos, sino aquellos que puede dar a los pobres. Apenas come, y
cuanto le dan aquí lo guarda para repartirlo los sábados a una chusma
que viene a la portería, porque según dice él, ya que no puede redimir
cautivos, quiere redimir a los que padecen la peor esclavitud de todas,
que es la miseria. Antes te dije que era un mentecato; pero la verdad,
hijo, Chaves es un excelente hermano.

---Dios ha puesto de todo en el mundo---pensé yo,---y así como no hay
nada perfecto, tampoco hay cosa alguna que sea rematadamente mala.

\hypertarget{xxiii}{%
\chapter{XXIII}\label{xxiii}}

Al día siguiente Salmón me dio muy malas noticias.

---¿Sabes lo que pasa, Gabriel?---me dijo entrando muy de mañana en la
celda que se me había asignado.---Pues he sabido que el Gobierno
francés, que ahora nos rige, ha nombrado alguacil, o como ahora dicen,
oficial, jefe o no sé qué de policía, a ese mismo Santorcaz que quería
prenderte. Esto tiene indignados a cuantos le conocían, y prueba a las
claras que ya estaba vendido a los franceses desde antes del sitio.
También es indudable que en los días del sitio fue nombrado alguacil por
la Sala de Alcaldes, sin que nadie acierte a darse cuenta de cómo
consiguió tal cosa. Le acompaña hoy como antes su escuadrón de gente de
mal vivir, que como sabes, era la que días pasados acaloraba los ánimos
contra los franceses en los barrios bajos, haciéndose pasar por
ardientes patriotas. Pero di, ¿qué has hecho para que te quieran
prender? Porque me han dicho que él y los suyos te buscan con verdadero
frenesí, registrando todos los rincones de Madrid.

---En verdad que no sé en qué fundan su persecución---respondí;---pues
por más que me devano los sesos, no puedo traer al pensamiento ninguna
acción mía que a cien leguas se parezca a un delito. Pero esos hombres
son muy malos, y no hay que buscar fuera de ellos la causa de sus
maldades.

---Pues me han dicho que en todo el día de ayer, ese Santorcaz no ha
hecho más que prender gente sospechosa, es decir, gente a quien supone
hostil a los franceses.

---Es una venganza personal---dije,---o tal vez deseo de apoderarse de
mí para una baja intriga.

---¡Qué inmunda canalla! ¡Y de esta manera quieren el rey de Copas y su
hermano hacerse amar de los españoles! Pues no es mal chubasco el que se
nos viene encima. Dicen que Napoleón ha rasgado el acta de capitulación,
expidiendo con fecha de ayer varios decretos contrarios a lo estipulado.

---Pues, padre mío---dije,---veo que me es preciso huir de Madrid a toda
prisa.

---¡Huir de Madrid! ¿Crees que es fácil ahora? Estate unos días más en
esta casa, que el prior no tendrá inconveniente en ello, y después
veremos cómo te sacamos de la villa. ¡Oh! Me han asegurado que la salida
es muy difícil hasta para las ratas. Parece que la gente de los pueblos
inmediatos a Madrid está levantada en armas. Temen los franceses que
esto sea cosa urdida con los de aquí para favorecer un movimiento
insurreccional dentro de la corte, y han resuelto incomunicar a Madrid.
La vigilancia que hay en las puertas es peor que de inquisidores; no
dejan salir a alma viviente sin registrarle y darle mil vueltas; y como
el viajero no lleve un papelucho que llaman \emph{carta de seguridad},
expedida por esa bendita superintendencia de policía, a quien vea yo
comida de lobos, lo someten a un consejo de guerra. Conque, hijo, estás
en peligro; no puedes vivir en Madrid, y la salida es muy difícil. ¡Ah!
En este momento se me ocurre una cosa, y es que podemos solicitar el
amparo de la señora condesa, en cuya casa estuviste el otro día, la cual
me han dicho que es amiga de los franceses.

---¡La señora condesa amiga de los franceses!

---Quiero decir partidaria. Su primo, el duque de Arión, que ha pasado
toda su vida en Francia, entró en España con Bonaparte, de quien es muy
devoto, y actualmente está en el cuartel general de Chamartín. Anteayer
estuve en casa de la condesa, y le esperaban de un día a otro. Como haya
venido, no nos sería difícil que aquella bondadosa señora te consiguiese
una carta de seguridad para evadirte. Entretanto, hijo, aquí estás más
seguro, y por sí o por no, vamos tú y yo ahora mismo a ver al prior del
convento, que es hombre de mucho mundo, y de tanta trastienda, que sería
capaz de pegársela al lucero del alba. Él nos dirá si lo que me ha
ocurrido es razonable, o si hay otro medio más expedito para ponerte en
salvo.

Y sin más dimes ni diretes, llevome a la celda del padre prior, que en
aquel momento había vuelto de decir su misa y despabilaba dos onzas de
chocolate. Era el padre Ximénez de Azofra un hombre pequeño, de edad
madura, ojos muy vivos, sonrisa maliciosa, cortesanos modales y
simpática conversación. Recibiome con mucha bondad, y cuando Salmón le
expuso las apreturas en que yo me encontraba, dijo lo que sigue:

---En otras circunstancias, joven incauto, fácil nos habría sido
socorreros poniéndoos al abrigo de esta casa. Pero ahora todo está del
revés. El Gobierno intruso nos mira con muy malos ojos, y bastaría que
le protegiéramos a usted para que se nos acusara de cómplices de la
insurrección, que así llaman ellos a nuestra santa causa\ldots{} En
verdad que cada vez odio más a esa canalla. Ved lo que hacen ahora.
Desde que Madrid se ha rendido, ya les ha faltado tiempo para quebrantar
lo convenido, y si prometieron respetar las vidas, libertades y hacienda
de este vecindario, ayer todo ha sido prender y encarcelar gentes
honradas, a quienes se acusa de auxiliar a los insurgentes de Talavera y
de Cuenca. Todo es sospechar, y acusar, y asustarse hasta de vanas
sombras; y como los restos del ejército de San Juan y las tropas del de
Castaños que se unieron al duque del Infantado andan por estas
inmediaciones levantando los pueblos contra los franceses, estos ven un
espía en cada vecino de Madrid, y han resuelto impedir toda comunicación
entre los habitantes de esta villa y los de Ocaña, Toledo, Talavera e
Illescas; por lo cual no permiten la entrada de los paletos, fruteros y
verduleros, razón de la gran carestía que hoy tienen todos los
artículos.

---Mala situación es esta---dijo Salmón.---¿De modo, señor prior de mi
alma, que en buenos tiempos no recibiremos nada de nuestras granjas de
Leganés, Valmejado, Casarrubielos, Bayona de Tajuña y Santa Cruz del
Romeral? ¡Bonito porvenir! ¿Y entonces \emph{quid manducaverunt vel
manducavere?}

---¡Oh! amigo Salmón---contestó el prior con malicia;---aquí viene bien
aquello de \emph{ventorumque regat pater}, que quiere decir \emph{viento
en panza}, según traducía aquel gilito descalzo de quien tanto nos hemos
reído. Es preciso hacer penitencia.

---Bien, retebién---exclamó Salmón bufando.---¡Viva el emperador de los
franceses, y Rey de Italia y protector de la confederación del Rhin! De
esa manera conseguirá Vuestra Majestad Imperial y Real, que asada en
parrillas vea yo, conquistar las simpatías del clero regular.

---No se cuida él de nuestras simpatías, amigo Salmón.

---Pero en resumidas cuentas, señor padre prior, este muchacho, de cuya
moralidad y buen proceder respondo, necesita salir de Madrid, y no dudo
que Vd. con su influencia le podrá sacar una \emph{carta de seguridad},
con la cual y disfrazado\ldots{}

---¡Qué cosas tiene Salmón!---dijo Ximénez de Azofra.---¿Qué puedo yo
hacer? Conque en priesa me ve, y doncellez me demanda. ¿No le he dicho
que desconfían de los regulares, y especialmente han tomado entre ojos a
los de esta casa?

---No sabía tal cosa. Al contrario: oí decir que Vuestra Paternidad es
de los que van a Chamartín a cumplimentar a mi señor D. Caco imperial,
rey de los pillos, y protector de la congregación del Rin\ldots{} conete
y Cortadillo.

---¿Yo?---exclamó Ximénez con asombro.---No he nacido para besar la mano
que me azota. Español soy, y español seré mientras viva. He predicado en
el púlpito de la Merced contra el Emperador, y no imitaré a los que
siendo primero desaforados patriotas, ahora son patriotas tibios con
vislumbres, amagos y pintas de afrancesados. Cierto es que va a
Chamartín una diputación de todas las clases de la sociedad; cierto que
me han invitado para ir, y vea su merced aquí la carta que sobre este
punto me ha dirigido el corregidor, y que de haber justicia en la
tierra, debería ser quemada por la mano del verdugo. ¿No es una
vergüenza que de este modo se humillen los hombres? Ayer todo era
inquina contra el \emph{ogro de Córcega}, todo insultarle y ponerle por
esos suelos; hoy todas son blanduras. El mismo señor corregidor de
Madrid que en su bando del 25 de Noviembre decía: \emph{La España está
invadida por el tirano que domina en Francia, el cual ha quebrantado
pérfidamente las santas leyes, etc}.; ese mismo señor corregidor don
Pedro de Mora y Lomas, caballero de la orden de Carlos III, del consejo
de Su Majestad, su secretario con ejercicio de decretos, intendente de
los reales ejércitos y de esta provincia, corregidor de esta villa,
subdelegado de Rentas reales, intendente de la real Regalía de Casa de
aposento, superintendente general de Sisas reales y municipales de ella,
y subdelegado de Montes y Pósitos, etc., etc., pues la retahíla de
títulos no tienen fin; ese mismo corregidor, repito, es el que hoy
dirige un llamamiento \emph{ante diem} a todas las autoridades. ¿Para
qué creerán Vds.? Pues nada menos que para hacer presente que la
\emph{villa de Madrid habrá tenido el honor de ofrecerse a los pies de
S. M. I. y R. para manifestarle el reconocimiento a la bondad e
indulgencia con que ha tratado esta corte, felicitarse por tener a S. M.
en su seno, y expresarle que si lograba merecer la dignación y aprecio
de S. M. se contemplaría dichosa}. ¿Qué tal? ¿Es este un lenguaje digno
y patriótico? Además en la convocatoria---añadió recorriendo con la
vista el papel,---se llama a Napoleón \emph{padre amoroso}, y a sus
atropellos \emph{benéficas miras}, y el objeto es reunir un cierto
número de personas respetables que piquen espuelas hacia Chamartín para
pedir a Bonaparte \emph{se digne conceder la gracia de que vean en
Madrid a su augusto hermano nuestro rey Josef}. Vamos, vamos, no puedo
leer más, porque tanta bajeza me saca los colores de la cara. Verdad es
que los que esto han firmado lo han hecho cediendo a amenazas del
comandante general Mr. Belliard que les pone el puñal al pecho; pero no
por eso es disculpable, pues si no traición a la patria, debe
imputárseles una debilidad y flaqueza que raya en crimen.

---¿De modo que usted no va a Chamartín?

---¿Yo? Ni por pienso. He oído que van en representación de los
regulares el padre Amadeo, abad de San Bernardo, y el padre Calixto
Núñez, abad de los Basilios. Ya se ve: ¿qué se puede esperar de esos
infelices tan dejados de la mano de Dios? Caerán en el garlito los
Mínimos, algunos pobres Franciscos, los desdichados Agonizantes, no
pocos Agustinos, todos los Gilitos, los Hospitalarios, los Donados, los
Carmelitas descalzos, y esos infelices Afligidos, que son los mayores
mentecatos de la cristiandad; pero la Merced sostendrá su bandera, la
Merced no adulará Emperadores, la Merced en unión con los Dominicos
desafiará el poder del tirano, contra franceses ladrones y empecatados
españoles.

---Y los víveres por esas nubes, y las puertas de Madrid cerradas al
buen vino, al rico aceite, a los huevos, a las coles, al extremeño
tocino y a los jamones de Candelario. Bueno, bueno, comamos ensalada de
perejil y cañutillos de monjas mojados en agua de limón. ¡Viva la
patria, Sr.~Ximénez, viva el orgullito que nos pondrá como espátulas!

---Pues bien; lo que he dicho a Vd.---continuó el prior,---lo he dicho a
los que vinieron a sonsacarme, y oídas mis palabras, tratáronme con tal
acritud, que espero grandes desdichas para nuestra orden y nuestra casa.
De modo que nada puedo hacer por este joven.

A esto llegaban cuando entró el padre Castillo acompañado de otros dos
frailes. El uno supe después que se llamaba el padre Vargas, y aunque
del mismo hábito y orden, pertenecía al convento de la Trinidad calzada,
también de mercenarios redentores de cautivos, y el otro era dominico,
del convento de Santo Tomás, y tenía por nombre el padre Luceño de
Frías.

---Ya, ya pareció aquello---exclamó Vargas con estrepitosa voz.---Ya no
podemos dudar de la veracidad de esos decretos, porque por ahí los
reparten impresos y aquí tengo un ejemplar. Todos los decretos llevan la
fecha del 4, y son tales que podrían arder en un candil en noche de
aquelarre.

---Veámoslos. ¿Es cierto que nos reducen a la tercera parte?

---Tan cierto, que\ldots---dijo el dominico,---no nos reducen a la
tercera parte, sino que nos parten por el eje, Sr.~Ximénez de Azofra.

---Atención, que leo---dijo Vargas, poniendo ante los ojos, de verdes
antiparras armados, un papel impreso.---Los decretos rezan lo siguiente:
\emph{En nuestro Campo Imperial de Madrid a 4 de Diciembre de 1808.
Napoleón Emperador de los etc\ldots{} Considerando que el Consejo de
Castilla se ha comportado en el ejercicio de sus funciones con tanta
debilidad como superchería\ldots{} que después de haber reconocido y
proclamado nuestros legítimos derechos al trono, ha tenido la bajeza de
declarar que había suscrito a estos diversos actos con restricciones
secretas y pérfidas, hemos decretado y decretamos lo siguiente: Art. 1.º
Los individuos del Consejo de Castilla quedan destituidos como cobardes
e indignos de ser magistrados de una nación brava y generosa}.

---Pues digo---exclamó Ximénez,---que eso está muy lindísimamente hecho.

---Es verdad---afirmó el dominico,---porque esos señores han estado
jugando a dos juegos, y con todo el mundo quieren comer. Adelante.

---Otro---prosiguió Vargas.---\emph{En nuestro Campo Imperial,
etc\ldots{} Napoleón, etc\ldots{}} Este no hace exposición de motivos,
ni considerando alguno, sino que dice simplemente: \emph{Artículo. 1.º
El Tribunal de la Inquisición queda suprimido como atentatorio a la
soberanía y a la autoridad civil. Art. 2.º Los bienes pertenecientes a
la Inquisición se secuestrarán y reunirán a la corona de España.}

---Ya se ve---exclamó el dominico sin disimular su enojo.---Sin eso no
podía pasar. Afuera Inquisición y vengan herejes, y lluevan masones,
¿qué les importa esto a los que no se cuidan de lo espiritual?

---Poco significa esto---dijo Castillo,---porque el Santo Tribunal casi
no existe ya de hecho, abolido por la suavidad de las costumbres.

---Pero se conservan las fórmulas, señor mío---contestó con aspereza el
dominico,---y las fórmulas tienen gran fuerza. Verdad es que no se
quema, ni se descuartiza (lo cual dicho sea de paso es excesiva
blandura, según estamos hoy comidos de herejía); pero hay todavía
degradaciones y simulados tormentos, que tienen muy buen ver para los
malos.

\emph{---Item}---prosiguió Vargas.---\emph{Art. 1.º Un mismo individuo
no puede poseer sino una sola encomienda}.

---Adelante, que eso nos interesa poco.

\emph{---Item. Art. 1.º El derecho feudal queda abolido en España. Art.
2.º Toda carga personal, todos los derechos exclusivos de pesca, de
almadrabas u otros derechos de la misma naturaleza, en ríos grandes y
pequeños; todos los derechos sobre hornos, molinos y posadas, quedan
suprimidos, y se permite a todos, conformándose a las leyes, dar una
extensión libre a su industria.}

---Eso no es nuevo---dijo Castillo,---y es lástima que nuestros
gobernantes con su indolencia hayan permitido a los franceses el
jactarse de promulgar una ley tan buena.

---Eso, eso es, ¡hágale su merced la mamola!---dijo Luceño de Frías con
el mayor desabrimiento, sentándose a horcajadas en una silla para apoyar
los brazos en el respaldo.---Me gustan las ideas del padre Castillo. Si
para eso pasa Vuestra Paternidad la vida entre la polilla de los libros,
buenas nos las de Dios.

Y sacando su tabaquera y alargando la mano hacia el prior, añadió:

---Señor Ximénez, un polvito, que los duelos con rapé son menos.

---No lo gasto---repuso el prior.

---Vamos, amigo Vargas, un polvito.

---No lo gasto, que eso es cosa de viejas. Aquí tengo unos cigarritos de
la Habana, que merecen ser chupados por los ángeles del cielo. Si el
señor prior me da su permiso\ldots{}

---Vengan---gritó Salmón,---esos tabaquíferos incensarios y pebetes de
Oriente, que tan bien matan el fastidio.

---Allá van---dijo Vargas.---Son regalo de la señora marquesa del
Fresno, y fuéronme remitidos poniéndolos en la mano de un Niño Jesús,
que me envió para que le diera una mano de pintura.

---Pues en lo relativo a ese decreto que acaba de leerse---dijo
Castillo,---mi conciencia no me dicta sino alabanzas, y alabanzas le
daré, aunque lo haya escrito el gran Tamerlán. ¿Por ventura no son esas
las mismas ideas que han hecho célebre en toda la redondez de la tierra
a nuestro gran Jovellanos? El mismo conde de Floridablanca, ¿no intentó
algo en ese asunto? Y los sabios consejeros de Carlos III, ¿no se dieron
de cabezadas por quitar esas trabas a la industria? Todos sabemos que a
aquel eminente Rey se le pasaron ganas de promulgar este decreto.

---¡Cosas de los jesuitas!---exclamó el dominico meciéndose en la
silla.---Pero esos pelanduscas andan también al retortero de Napoleón,
por ver si sacan tajada. Adelante con la lectura.

---Pues adelante---continuó Vargas.---\emph{Considerando que uno de los
establecimientos que perjudican a la prosperidad de España son las
aduanas y registros existentes de provincia a provincia, hemos decretado
lo siguiente: Desde 1.º de Enero próximo, las aduanas y registros de
provincia a provincia quedan suprimidos. Las aduanas se colocarán y
establecerán en las fronteras.}

---Tampoco eso tiene pero---observó Castillo,---y la Junta Central, ya
que pensó decretarlo, no debió esperar a que lo hicieran los franceses.

---También esto le parece bocadito de ángeles al Reverendo
Castillo---dijo Luceño.---Medrados estamos. ¿Tratan de eso los libros de
Vuestra Merced?

---Atención---indicó Vargas haciendo un gesto dramático,---que ahora
viene lo gordo. \emph{Considerando que los religiosos de las diversas
órdenes monásticas en España se han multiplicado con exceso; que si un
cierto número es útil para ayudar a los ministros del altar en la
administración de los Sacramentos, la existencia de un número demasiado
considerable es perjudicial a la prosperidad del Estado, decretamos lo
siguiente: Art. 1.º El número de los conventos actualmente existentes en
España se reducirá a una tercera parte. Esta reducción se ejecutará
reuniendo los religiosos de muchos conventos de la misma orden en una
sola casa. Art. 2.º No se admitirá ningún novicio ni permitirá que
profese ninguno, hasta que el número de religiosos se reduzca a una
tercera parte. Art. 3.º Los regulares que quieran renunciar a la vida
común y vivir como eclesiásticos seculares, quedan en libertad de salir
de sus conventos. Art. 4.º Los que renuncien a la vida común, gozarán de
una pensión que se fijará en razón de su edad, y que no podrá ser menor
de tres mil reales ni mayor de cuatro mil. Art. 5.º Del fondo de los
bienes de los conventos que se supriman, se tomará la suma necesaria
para aumentar la congrua de los curas. Art. 6.º Los bienes de los
conventos suprimidos quedarán incorporados al dominio de España, y
aplicados a la garantía de los vales y otros efectos de la Deuda
pública.}

Durante la lectura de este decreto, no se oyó en la celda de Ximénez
otro rumor que el producido por el vuelo de una mosca, que andaba a
vueltas tras los restos del chocolate prioral, como Bonaparte tras los
reinos de España. Después de leído, aún duró bastante el silencio.

\hypertarget{xxiv}{%
\chapter{XXIV}\label{xxiv}}

---¡Toquen castañuelas, repiquen panderos, machaquen almireces, punteen
vihuelas y aporreen zambombas para celebrar el talento del sabio
legislador, harto de bazofia y comido de piojos, que sacó de su cabeza
ese pomposo y coruscante decreto!---exclamó al fin Luceño dando un
porrazo en el respaldo de la silla y levantándose de ella.

---¿Conque a la tercera parte?---dijo Salmón.---¿De modo que de cada
tres no ha de quedar más que uno?

---Eso es, y los demás a la calle, a pedir limosna, porque una pensión
de tres mil reales para personas que han de vivir decentemente, es
aquello de hártate comilón con pasa y media.

---Y afuera novicios.

---¡Y no más profesar!

---Y con los bienes se aumentará la congrua de los curas.

---También eso está bien---dijo el dominico.---Alábelo su merced, padre
Castillo. ¡Que nos quiten lo nuestro para darlo a los curas! ¿Quiénes
son los curas, ni qué hacen esos zanguangos en bien de la cristiandad?
Ya\ldots{} como los curas son tan tibios patriotas\ldots{} ¡Estoy que
bufo!

---Lo mejorcito es que los bienes de los conventos suprimidos pasen al
dominio de España.

---¿Qué tiene que ver España, ni San España, ni Marizápalos, con esos
bienes?

---¿De modo que nuestras granjas de Leganés, de
Valmojado\ldots?---preguntó Salmón.

---¡Ya se ve! De esto se ríen todos esos infelices Mínimo, Gilitos y
Franciscos que nada tienen. A ellos, ¿qué les importa? Por eso van a
hacerle el \emph{como la porta bu}. Bien, retebién. Y lo mismo hacen los
Afligidos, que son la cáfila de majaderos más desaforados que he visto.

---No murmurar, hermano---indicó Castillo.

---Dios me lo perdone---dijo Luceño,---y no lo digo por nada malo, que
hay Afligidos de todas clases. ¿Pero creen vuestras mercedes que se
llevará a cabo esto de las tercera partes?

---Yo creo que va a ser dificilillo.

---Pues yo temo que lo llevarán adelante---afirmó Luceño;---que esta
mañana me ha dicho en confianza un regidor que va a Chamartín, que ya
tienen hecho su plan, y que dentro de pocos días comenzará el restar y
dividir, para dar principio a la demolición de los conventos.

---¡La demolición!

---Sí; que todas estas casas las destinan a oficinas del Estado, y la
primera que va a caer hecha pedazos es este monasterio de la Merced en
que ahora estamos.

---¡Cómo, la Merced! ¡Se atreverán a ello!---exclamó Ximénez de Azofra,
dándose un golpe en el brazo de la silla.---¡Cómo! ¿Se atreverán a
derribar esta casa que lo fue del gran Tirso de Molina? ¿Y la gran
devoción que inspira la Virgen de los Remedios que está en una de
nuestras capillas? ¿Pues y el sepulcro de los nietos de Hernán-Cortés?
No, no puede ser. Derriben en buen hora otras casas de religiosos, pero
no esta por tantos títulos, además de su antigüedad, venerable.

---Y también está amenazada la Trinidad Calzada---apuntó Luceño,---si no
de que la derriben, al menos de que la vacíen.

---Eso no puede ser---declaró Vargas,---que más glorias encierra mi casa
que todos los demás claustros de Madrid reunidos. Díganlo si no el beato
Simón de Rojas y el padre Hortensio de Paravicino, autor del libro
\emph{De locis theologicis}.

---Autor de las \emph{Oraciones evangélicas, de la Historia de Felipe
III y de la España probada}, querrá decir Vuestra Paternidad---indicó
Castillo con malicia;---que el libro De locis theologicis, hasta los
chicos de las calles saben que es de Melchor Cano.

---Tiene razón Castillo: me equivoqué. Pero sea lo que quiera, también
tiene mi convento la honra de haber rescatado, mediante los padres Bella
y Gil, al inmortal Cervantes, autor del \emph{Quijote}, Sr.~Castillo,
pues yo también entiendo algo de autores. En caso de desalojar conventos
para oficinas, ahí está Santo Tomás, donde caben todas.

---¡Cómo es eso! ¡Santo Tomás! ¡Desalojar a Santo Tomás, el más ilustre
de los conventos de Madrid!---exclamó impetuosamente el dominico.---¿Y
qué sería de este pueblo si te quitaran el espectáculo de las
procesiones que de allí salen con motivo de las funciones del Santo
Oficio? A fe que hartas casas hay en Madrid, si quieren hacer plazuelas,
como dicen, aunque más vale que no se toque a ninguna, porque
\emph{setenta y dos} conventos para una población de 160.000 almas, me
parece que no es mucho. Las casas de religiosos apenas ocupan un poco
más de la mitad del perímetro de esta gran villa, lo cual no es nada
desmedido, y de todas las casas que se alzan en ella, sólo \emph{cuatro
quintas partes} pertenecen a conventos, memorias pías, capellanías y
otras fundaciones.

---Y dígame, Luceño---preguntó Ximénez,---¿van dominicos a la reunión
que convoca el corregidor?

---Creo que no. Según he oído, sólo se prestan a ir a Chamartín el
prepósito de San Cayetano, el abad de Montserrat, dos Agonizantes, un
par de Franciscos, un rector de Niñas de la Paz y un Afligido.

---Pues estos sacarán tajada, no lo duden vuestras mercedes. Sobre
nosotros lloverán los decretos y las terceras partes.

---Mi opinión es---dijo Salmón,---que pues cuesta bien poco ir de aquí a
Chamartín, nada se pierde con que vayan un par de padres, y yo me brindo
a ello, que bueno es estar bien con todos, y el orgullo es pecado, y
quien al cielo escupe en la cara le cae.

---No en mis días: de esta casa no irá nadie---aseguró Ximénez de
Azofra,---y en cuanto a este joven, nada podemos hacer. Indigno sería
pedir favores a quien nos trata mal, amenazándonos con terciarnos y
partirnos como si fuéramos aranzadas de tierra. Conque busque usted
quien le proporcione la \emph{carta de seguridad} para salir de Madrid.

---Dificilillo es---afirmó Luceño,---pues entiendo que se miran mucho
para dar las tales cartas, y sin ellas no es posible dar un paso de
puertas afuera.

---Sin embargo---dijo el discreto Castillo,---hay multitud de personas
que por estar en bien con los franceses, pueden socorrer a este joven.
¿No conoce Vd. ninguna persona de alta posición y de influencia?

---Sí, ya me ocurrió acudir a la señora condesa---indicó Salmón,---y
confío en que su generosidad sacará a este joven del mal empeño en que
se ve. El señor marqués se ha afrancesado y dicen que va a entrar en la
alta servidumbre del rey José.

---El Sr.~D. Felipe bebe los vientos porque cualquier Gobierno se
acuerde de él ---dijo Castillo.---Algo debe de haber de cierto en eso,
pues hace tres días, después de haberse presentado a Belliard, fuese al
Pardo, donde se ha instalado con su hija. Ayer creo que debió llegar a
dicho real sitio el rey José. A pesar del influjo que en la botellesca
corte tiene el señor marqués, yo no me fiaría de él para ningún delicado
asunto. De más eficacia me parece en el caso presente el señor duque de
Arión, pariente de esta familia y que goza de gran poder en el cuartel
general.

---¡Admirable idea! Veremos al señor duque.

---No ha llegado aún a Madrid, y como no sea exponiéndose a los peligros
de un viaje a Chamartín, este joven no podría verle.

---Lo mejor---añadió Salmón,---es que veamos hoy mismo a la señora
condesa. ¿Va hoy allá la Paternidad del Sr.~Castillo?

---Dentro de un rato, pues la señora marquesa me ha mandado llamar hoy
con toda premura. Si quiere este joven venir conmigo, le llevaré.

---Oportunísimo---añadió Salmón.---Yo iré también. Pero hijo, si en la
calle acertamos a pasar por junto a esos cafres\ldots{}

---Pues bien---dijo Ximénez;---para que vaya más seguro, yo les presto
mi coche, que con sus dos gallardas mulas debe de estar ya en la huerta.

---Muy bien---declaró Salmón batiendo palmas.---Me parece buena idea la
del coche; pero para mayor seguridad, te vestiremos de novicio. Venga la
carroza prioral y a casa de la condesa.

---Pues entrareme también en ella, y me dejarán de paso en Santo
Tomás---añadió Vargas.

---Pues allá voy también---dijo Luceño,---si me dejan en las Descalzas
Reales.

Y así acabó la conferencia sin más resultas que las de mi improvisado
disfraz de novicio y mi viaje a casa de la condesa, donde me pasó lo que
el lector verá a continuación si tiene paciencia para seguir leyendo.

\hypertarget{xxv}{%
\chapter{XXV}\label{xxv}}

La condesa mostró mucho asombro al verme. Hallábase en la misma
habitación donde algunos días antes me había recibido, y cuando
entramos, apartose del secreter donde escribía, para venir a nuestro
lado. Castillo principió preguntándole por la salud de todos, y luego en
breves palabras le expuso los motivos de mi visita y de mi nuevo
vestido. Cumplida esta misión, y añadiendo que necesitaba ver a la
señora marquesa, pidió a Amaranta venia para pasar adentro, y con esto
nos quedamos Salmón y yo solos con ella.

---Por ahí se murmura que yo soy afrancesada---dijo Amaranta,---pero no
es cierto. Mi tío sí ha abrazado la causa del rey José con tanto
entusiasmo, que cuando le contradecimos en algún punto relativo a estas
cosas, nos quiere comer a todos. Vive en el Pardo con su hija desde hace
tres días en el mismo palacio real, pues el Rey intruso se ha empeñado
en incluirle en su alta servidumbre. Está mi tío loco de contento, y si
viene esta tarde a Madrid, como decía, yo le rogaré que me proporcione
una \emph{carta de seguridad} para este mancebo.

---Ya estás en salvo, Gabriel---exclamó el mercenario.---¿No te dije que
esta excelsa señora te sacaría de tan mal paso?

---Aún mejor puedo conseguirla por mi primo el duque de Arión, el cual
más que afrancesado, es francés puro, y si viene mañana a Madrid, como
espero, no olvidaré este encargo.

---Vaya, no hay que pensar en que te echen mano---dijo Salmón
levantándose.---Ya estás salvado, chiquillo; prostérnate ante Su
Grandeza y dale un millón de gracias por tantas mercedes. Y ahora,
señora condesa, si usía me da su licencia, voy a pasar a ver a mi señora
la marquesa, que el otro día me habló de unos requesones, acerca de cuyo
mérito quería saber mi voto.

Nos quedamos solos Amaranta y yo, lo cual me agradó, pues deseaba hablar
con ella sin testigos.

---Señora---le dije,---¡cuánto agradezco a vuecencia esta nueva bondad!
Ahora me cumple pedir perdón a usía por no haber salido de Madrid, como
hubiera sido mi deseo.

---Estarías alistado.

---Justamente, y ahora que el desarme me permite salir, una persecución
injusta, cuya razón no puedo explicarme, me detiene en Madrid, oculto en
el convento de la Merced.

En seguida contele el incidente de Santorcaz, añadiendo que el antiguo
desleal mayordomo de la casa andaba a la zaga del flamante jefe de
policía.

---Ya lo sé---me dijo Amaranta,---y he tenido miedo de que algún peligro
amenazara nuestra casa. Por eso me alegro mucho de que Inés esté con mi
tío en el palacio del Pardo, donde no puede ocurrirle nada malo. El
primer día sentía yo gran zozobra; pero nosotros tenemos antiguas
amistades y relaciones con las primeras personas del partido francés, y
ya estoy tranquila. Nada temo de esos miserables.

---Me falta---dije yo,---dar las gracias a vuecencia por los otros
favores de que me dio cuenta el licenciado Lobo. No los necesitaba para
llevar adelante mi resolución, y sin destino en el Perú, sin ejecutoria
de nobleza y sin promesas de dinero, sabré hacer de modo que usía no
tenga queja alguna de mí.

---No---me dijo sonriendo,---el destino que solicité de la Junta, espero
que ahora me lo conceda también el Gobierno francés, y de todas estas
diligencias está encargado Lobo, a quien he dado cartas para Cabarrús y
para Urquijo. Irás al Perú, tendrás tu ejecutoria de nobleza, y con esto
y con la ayuda de Dios podrás llegar a ser un hombre de provecho. La
conciencia me impulsa a hacer esto en pro de una persona desvalida que
tiene derecho a mi consideración. En cambio no olvidaré que has hecho
una promesa, y cuanto hago por ti no es más que la recompensa anticipada
que ganas cumpliendo lo pactado.

---Señora condesa, yo cumpliré religiosamente lo prometido---le contesté
con resolución,---y no puedo admitir la recompensa. Mi dignidad no me lo
permite.

---¿Pues acaso tú tienes dignidad?---me dijo riendo.---Pero no, no debo
reírme. ¿Por qué no habías de tenerla como otro cualquiera? La verdad es
que los que estamos en cierta posición, no vemos más que a nosotros
mismos. En cuanto a la determinación de no aceptar nada, yo arreglaré
las cosas de modo que aceptes.

Así hablábamos cuando regresó Salmón a nuestro lado, y al punto cortó el
hilo de nuestro coloquio, diciendo:

---Gran satisfacción, señora condesa, me ha causado la noticia que en
este momento acabo de oír de los autorizados labios de mi poderosa
señora la marquesa. La paz sea en esta casa, señora, bendigamos la mano
de Dios.

---¿Habla Su Paternidad del asunto de mi prima?---dijo Amaranta.---Sí,
ya creo que la tenemos en vías de curación.

---Veo que el ingeniosísimo recurso ideado por el gran entendimiento de
vuestra merced ha surtido su efecto. ¿Y cómo recibió la noticia? ¿Se
turbó, derramó muchas lágrimas\ldots? Porque en realidad, señora,
decirle de buenas a primeras que el joven ese\ldots{}

Y Salmón se detuvo como hombre prudente, temiendo hablar de negocio tan
delicado delante de un extraño.

---Puede Vuestra Paternidad hablar sin reticencias---dijo Amaranta con
un tonillo que me pareció algo intencionado,---porque no estando en
antecedentes la única persona que nos oye, poco importa\ldots{}

---Pues preguntaba, señora, si cuando se le dijo y se le probó la muerte
de ese joven, no mostró su pena de un modo ruidoso, con desmayos,
gritos, lloros y demás desahogos propios de la debilidad femenina.

---Nada de eso, padre---repuso Amaranta con muestras de
satisfacción.---Al principio no lo quería creer; luego cuando se le
probó de un modo irrecusable, con los papelotes que trajo el licenciado
Lobo, pareció dudarlo, y por último cuando yo se lo dije, aparentando
sentirlo y doliéndome mucho de la muerte de ese infeliz, empezó a
creerlo. Lo que más la ha convencido fue el artificio verdaderamente
teatral que puse en práctica para hacérselo creer. Estaban todos
hablándole de este asunto, cuando entré de improviso, fingiendo mucho
enojo porque sin preparación alguna le daban tan tristes noticias;
arranqué de las manos de Lobo aquellos papeluchos que fingían ser
partidas de defunción, copias del libro del hospital o no sé qué, y los
hice pedazos delante de ella. Al mismo tiempo empecé a disponer que se
dieran cordiales y otros remedios del caso, asegurando que tenía ella
mucha razón en sentir la muerte de aquel con quien tuvo tan honesta
amistad. Esto hizo efecto, y después cuando encerrándonos las dos en mi
alcoba, le dije: «Sosiégate, todavía puede ser que se salve. Yo te
prometo que si vive le verás, y quién sabe, primita mía\ldots{} puede
ser, puede ser\ldots» Ella se afligió mucho, y yo añadí: «Es preciso
tener resignación, es preciso aprender a padecer. Yo no quiero
contrariar ya una inclinación tan decidida, porque antes que todo es tu
felicidad. Desgraciadamente Dios quiere resolver la cuestión de otro
modo y llamar a ese joven a su seno. Esta mañana he estado en el
hospital, le he visto, y la verdad\ldots{} había pocas o ningunas
esperanzas.» Y con esto aumentaba su tristeza; pero sin llantos ni
exclamaciones. Luego yo también me puse a llorar y la abracé y le di mil
besos, diciéndole: «Ya ves cómo no está en mi mano hacerte feliz. Te
aseguro que por mi parte no repararía en nada para conseguirlo; pero
Dios lo ha dispuesto de otro modo. Procura calmarte y ten resignación:»
cuando esto le dije, la dejé convencida. ¡Ay! Después su aspecto era el
de la resignación. Hablaba poco y parecía meditar. Se ha desmejorado
mucho en pocos días; pero esto se le pasará indudablemente. Ahora ha ido
al Pardo, pues la variación de localidad es muy buen remedio para estas
enfermedades del espíritu. Su manía caprichosa y ciega nos ha disgustado
mucho; pero me parece que dentro de algún tiempo estará todo concluido.

---¡Oh! ¡qué felicidad!---exclamó Salmón,---hay un gran médico del dolor
que se llama el doctor tiempo. Perdida con la idea de la muerte la
esperanza, ese señor médico hace maravillas en un par de semanas.

Yo oía este diálogo y admiraba la extremada habilidad artística de
aquella encantadora cortesana, tan maestra en engaños y ficciones.

---Ha hecho muy bien usía---continuó Salmón,---en poner en juego esos
ingeniosos ardides que prueban su grandísimo talento. Era una cosa que
daba vergüenza ver a mi niña enamoriscada de un haraposo de las calles,
que sin duda es de lo más arrastrado y despreciable que han echado
madres al mundo.

---¡Oh! no---dijo Amaranta con cierto énfasis jovial.---Nosotros nos
esforzábamos en pintárselo así; pero no tiene nada de despreciable. Yo
tengo noticias ciertas de sus antecedentes y conducta. Además de que ha
demostrado en varias ocasiones una nobleza de sentimientos que no puede
caber sino en personas bien nacidas; su posición es más que regular.
Cierto es que por desgracias de familia, tan comunes en estos tiempos,
viose reducido a la indigencia; pero está probado que procede de una
nobilísima familia de los mejores solares de Andalucía, como lo acredita
la ejecutoria que posee, y además, figúrese Su Paternidad si tendrá
méritos personales, cuando la Junta Central le dio espontáneamente un
gran destino en el Perú, cuyo destino parece le confirmará ahora el
Gobierno francés.

Tuve que hacer un esfuerzo para contener la risa que asomaba a mis
labios.

---Pues eso sí que no lo sabía yo. De modo que la discreta ninfa no
había puesto sus ojos en ningún piruétano. De todos modos, bueno es que
se haya quitado de en medio por una engañosa ficción la importuna
memoria del empleado del Perú. Por supuesto, señora, no hay que pensar
en D. Diego.

---¡Oh! no\ldots{} estamos decididas. D. Diego no será de modo alguno su
esposo, aunque renunciemos a la buena amistad de la de Rumblar. Al fin
he convencido a mi tía, y pronto hasta impediremos a ese joven que entre
en esta casa. Aún viene aquí; pero tanto nos disgusta su presencia, que
de un día a otro le vedaremos la entrada.

---Y ese pariente de vueseñorías---dijo el mercenario,---ese duque de
Arión, a quien se tiene por un joven instruidísimo, ¿no estará destinado
a ser esposo de la joya de esta casa? Perdone usía mi curiosidad.

---No lo sé---respondió Amaranta.---No hay nada proyectado. Mi primo ha
vivido catorce años en París, apenas nos conoce.

Así continuó la conversación por un buen espacio de tiempo, cuando
sentimos ruido de voces, y vimos que con gran estrépito y barahúnda
entraba el diplomático, en traje de camino, y tan alegre, tan festivo,
tan charlatán, que al punto le tuvimos por poseedor de los más altos
secretos de Estado.

---Sobrina---gritó al entrar,---aquí me tienes. Pero soy el juego de la
correhuela: cátate dentro y cátate fuera. Ahora mismo tengo que salir,
pero si no miente mi lista, son ciento dos las personas que he de ver de
aquí a las cuatro de la tarde. ¡Si me vuelvo loco! Si no es mi cabeza
para tantos negocios. Que vaya el señor marqués a explorar el ánimo del
duque de Alba para ver si cede o no cede; que forme el señor marqués una
lista de las personas de la grandeza que están dispuestas a acatar a
José; que vea el señor marqués al corregidor de Madrid; que se dé una
vuelta por los Cinco Gremios a ver si anticipan o no anticipan fondos;
que vaya, que venga, que corra, que escriba, que aconseje, que consulte,
que tantee\ldots{} ¡Jesús, María, José! Esto no es vivir. Yo no quería
meterme en tales faenas. Pero me han obligado, me han cogido, me han
puesto el cordel al cuello. Cuando el rey José dice que no puede hacer
nada sin mí; cuando me presenta a su hermano elogiándome con frases que
no repito por no parecer jactancioso, no es posible evadirse\ldots{}
¡Oh! ¡Qué belén, qué ir y venir! Nada se ha de hacer sin que yo diga
\emph{hágase}. Y Vd., Sr.~Salmón, ¿qué dice de estas cosas?

---Qué he de decir, sino que Dios le conserve a usía mil años al lado de
ese Rey, para ver si evita lo de las terceras partes con que nos han
amenazado.

---Todo se arreglará, hombre, todo se arreglará. A pesar del decreto de
proscripción, hemos salvado la vida a Infantado, Alba, Santa Cruz del
Viso, Medinaceli, Hijar, Fernán-Núñez, Altamira, Castel Franco,
Cevallos, y al obispo de Santander, sentenciados a muerte por el decreto
dado en Burgos el 12 de Noviembre. Se les envía a Francia simplemente.
Otras muchas cosas ha dispuesto el Emperador, modificando sus primitivas
determinaciones; pero no las puedo decir, no, no te diré una palabra,
sobrina, de estos delicados negocios; ya te veo sonreír\ldots{} Ya te
veo a punto de emplear las armas de tu seducción para poner sitio a la
fortaleza de mi secreto; pero no te diré nada, no, ni una sílaba; ni
tampoco a Vd., padre Salmón, que me mira con esos ojazos, que revelan
toda la concupiscencia de la curiosidad.

---No quiero saber nada de eso---dijo Amaranta.---¿Y mi primita?

---Contentísima.

---¿Cómo contentísima?

---No, no, quiero decir, tristísima. En dos días creo que no habrá dicho
seis palabras. Se ocupa en sus labores con una asiduidad que me asombra,
y no hay quien la haga presentarse en el gran salón de Palacio.

---Ha hecho Vd. muy mal en dejarla sola---dijo la condesa con cierto
enfado.

---¿Y qué le ha de pasar? ¿No quedan allí los criados? ¿No está con tu
doncella y con Serafina, que ni un instante se separa de su lado?

---Pero ya le dije a Vd. que Inés no debe quedarse sola con doncellas y
criadas en ninguna parte---añadió Amaranta notoriamente contrariada.

---¿Estamos viviendo en despoblado?---dijo el marqués riendo.---En el
Pardo, en el mismo palacio del Pardo, donde vive un Rey con numerosa
servidumbre y guardia, ¿no puede quedarse sola mi hija, por cuatro o
cinco horas? ¡Si vieras qué habitación tan magnífica me han destinado en
el piso bajo! Dan sus balcones al jardín del Mediodía, y se goza allí de
una deliciosa vista. Ayer y hoy por la mañana, Inés salió a dar un paseo
por el jardín. ¡Buen rato pasó la pobrecita!\ldots{} ¿Pero cuándo vienes
al Pardo? Por Dios y María Santísima, que sea pronto. Allí se pasan las
noches deliciosamente y no puedes figurarte cuán amable, cuán discreto,
cuán bondadoso es el rey José\ldots{} ¡Cuánto nos reímos anoche! Él me
preguntó: «¿Por qué dicen los españoles que soy borracho, cuando no bebo
más que agua?» Yo me quedé un tanto cortado; pero disculpé a mis
paisanos como pude.

---Mañana---dijo Amaranta,---nos iremos mi tía y yo, pues ya a fuerza de
sermones, voy logrando vencer su repugnancia a los franceses. Y ahora
que me acuerdo, tío, tiene usted que procurarme una \emph{carta de
seguridad} para que pueda escaparse de Madrid una persona, injustamente
perseguida.

---¡Oh, no, de ningún modo!---dijo el diplomático.---Yo no oculto
insurgentes, ni favorezco de modo alguno la insurrección. ¿Cartitas de
seguridad? Nada, nada, sobrina, no ampares pícaros, ni protejas a los
que se obstinan en aumentar los males de la patria. Sométanse todos a
ese bendito soberano que no bebe más que agua, y entonces se acabarán
las precauciones. Es preciso sofocar la insurrección que hierve en los
alrededores de Madrid, y hacen muy bien en no dejar salir ni una mosca.

---Bueno---dijo Amaranta.---Mañana ha de llegar mi primo el duque de
Arión, y él me dará cuantas cartas de seguridad se me antoje pedirle.

---¡Que viene mañana!---dijo el marqués.---Yo le esperaba esta noche. Me
han dicho que ya cumplió la misión que le dio el Emperador en Burgos y
ha regresado al cuartel general. Entrará también en la servidumbre del
Rey José. Si llega mañana, inmediatamente os marcharéis todos juntos al
Pardo. ¡Cuánto deseo verle! Era tamañito así cuando su madre se fue a
vivir a París hace catorce años. Era muy travieso; yo, jugando a todas
horas con él, le inculcaba los rudimentos de la historia patria. ¿Me
deparará Dios un excelente yerno?

---Veremos---repuso Amaranta.---No puedo dar mi opinión mientras no le
trate. El duque de Arión se ha educado en París.

---Educación a la francesa---dijo Salmón.---\emph{Vade retro}.
¿Apostamos a que viene mi señor duque hecho un filosofillo de tomo y
lomo?

---¡Oh, no!---exclamó el diplomático.---Desde que supe que se había
afiliado al bando napoleónico, le tuve por muy discreto. Su entrada en
España con el Emperador, las difíciles comisiones que este le ha dado
para entrar en tratos con las ciudades rebeldes, prueban\ldots{} ¿pero
qué veo?\ldots{} Las dos, y yo aquí de conversación olvidando las mil
comisiones\ldots{} adiós, sobrina, adiós, padre Salmón y la compañía. Yo
me vuelvo loco con tanto ir y venir\ldots{} Es terrible que esos señores
no puedan hacer nada sin uno\ldots{} adiós, adiós.

Y sin cesar de hablar salió de la habitación y de la casa
apresuradamente.

\hypertarget{xxvi}{%
\chapter{XXVI}\label{xxvi}}

Referidos estos curiosos diálogos, me cumple ahora contar de qué medio
se valió la condesa para facilitarme la deseada fuga. Mandome, pues, que
volviera al día siguiente, prometiéndome tener todo concertado y en
regla, de modo que pudiese sin pérdida de tiempo emprender la marcha,
desafiando la vigilancia ejercida en las matritenses puertas. Hicimos
Salmón y yo lo que se nos mandaba, y al otro día, cuando nos disponíamos
a volver de nuevo a casa de Amaranta, llamonos el padre prior, y nos
dijo:

---Este joven no puede estar aquí ni un día más, y esta noche misma, si
no encuentra medio de escaparse, es fuerza que busque un asilo más
seguro.

---¿Más seguro que la Merced?

---Sí---añadió Ximénez de Azofra.---Han venido a avisarme que se
sospecha de los conventos; que se nos acusa de ocultar a los
conspiradores y a los espías de los insurgentes, y parece que mañana
mismo registrarán todas estas casas, principiando por la Merced.

---Por fortuna la señora condesa te amparará hoy mismo---dijo
Salmón.---Vamos allá sin perder un instante.

Vestido de novicio y en coche, como el día anterior, fuimos a casa de
Amaranta, y desde que nos vio entrar, díjome con semblante alegre:

---Mi primo el duque de Arión ha llegado anoche, y me ha prometido
conseguir la carta de seguridad antes de tres días.

---Es que yo quisiera partir esta misma noche, señora condesa---dije.

---¿Esta misma noche?

---Tememos que esos hotentotes registren mañana nuestra casa---añadió
Salmón.

---Pues es preciso hacer un esfuerzo y salir de este mal paso---indicó
Amaranta.---La principal contrariedad consiste en que no puede uno
fiarse de nadie. Me han asegurado que la policía francesa ha extendido
sus ramificaciones a muchas casas principales, y que sobornando lacayos
y pajes tiene bajo su vigilancia a las familias que juzga desafectas. No
quisiera poner en el secreto a ningún criado, y\ldots{} ¡Ah! ¿no podría
salir con ese mismo traje de novicio?

---Mal vestido es, señora, para estas circunstancias---dijo
Salmón.---Tengo entendido que el registro que se hace en las puertas es
tan escrupuloso, que hace difícil toda superchería. A unos les hacen
desnudar, no librándose de este vejamen, ni aun las pudorosas doncellas
y las que no lo son. Examinan con farolitos las facciones,
confrontándolas con las notas de la carta, hacen vaciar las
faltriqueras, y esta ceremonia se repite en dos o tres puntos, y ante
los ojos de distintos esbirros.

---Un criado de casa---dijo la condesa,---tiene carta de seguridad. Con
ella y disfrazándose de paleto, ¿no sería fácil burlar la suspicacia de
esa gente?

---Los paletos---dije yo,---son los más perseguidos y a los que primero
detienen, porque se teme que comuniquen a los conspiradores de aquí con
los insurgentes de fuera.

---En este momento---exclamó Amaranta,---se me ocurre una idea
salvadora.

Diciendo esto, llamo a un criado y mandole un recado al duque de Arión,
que vino sin tardanza alguna, pues residía en la propia casa. El cual
duque de Arión, a quien llamo así porque se me antoja, callando su
verdadero título que es de los más conocidos entre los de España, era un
joven de veintidós a veintitrés años, delgado, de regular estatura,
semblante frío y sin expresión, de modales elegantes y comedidos, como
de persona habituada a la alta etiqueta, y sin otra cosa notable en su
persona que la atildada perfección del vestir. Digo mal, pues también
llamaba la atención en él un acento francés tan marcado y un tan
incorrecto uso de nuestro lenguaje, que a veces no era posible oírle con
seriedad. Hijo único de una señora que no nombro, y que fue mujer muy
corrida y muy tomada en lenguas allá por los últimos años del siglo
antecedente, marchó con ella a París a los siete años de edad y en
tiempo del Directorio: allí se educó, permaneciendo tres lustros fuera
de su patria. Era primo no sé si en segundo o tercer grado de los que yo
llamo de Leiva; pero la marquesa que le había criado, casi le
consideraba como hijo. Ya saben Vds. que este joven, a quien no faltaba
cierta discreción y muy buenas luces, era partidario decidido de
Bonaparte, más que por aficiones políticas, por la amistad que le unía
al mariscal Berthier. Cuando verificó el Emperador su expedición a
España, trájole consigo, dándole no sé qué puesto en la casa imperial.
Desde Somosierra fuele encargada una comisión confidencial cerca de los
vecinos acomodados de Burgos; desempeñola bien, según entendí después, y
al venir a Chamartín, después de un día de descanso, pasó a Madrid con
objeto de abrazar a aquellos sus parientes, y con ansia también de
visitar su posesión de Parla donde había nacido. Llegó Arión por la
noche, y al siguiente día tuve el honor de verle y ocurrieron sucesos
muy notables, a consecuencia de un diálogo que no puedo menos de copiar,
reuniendo los más oscuros recuerdos que almacena en sus antros sin fin
mi memoria.

---Primito---dijo Amaranta,---me vas a hacer un favor.

---¡Oh! Mi querida prima---repuso Arión,---\emph{de tout mon cœur}.

---Préstame, o mejor dicho, dame tu carta de seguridad. No dudo que me
harás este obsequio, ya que has mostrado tantos deseos de obsequiarme.

---¡Oh, \emph{ma belle contesse!}---dijo el currutaco llevándose la mano
al corazón.---Yo estoy muy obligado a vuestras bondades, y si pudiera
exprimaros lo que siento\ldots{} Mi deseo fuera que me demandaríais
\emph{quelque chose} de más difícil, extraordinario y peligroso, para
probaros que\ldots{}

---Gracias por la condescendencia, primo, y excusemos galanterías. Yo
soy una vieja. ¿Se usa en Francia que los petimetres galanteen a las
viejas? Por aquí no ha llegado todavía esa moda; pero me parece que tú
traes los primeros figurines de ella.

---¡Oh, oh!

---¿Y no te enfadarás si tomo tu nombre para una obra de caridad? Deseo
facilitar la evasión de Madrid a un joven desgraciado, a quien persiguen
miserables polizontes por satisfacer una ruin venganza.

---¡Oh, oh, \emph{volontiers!} \emph{Ma belle contesse} es dueña de
hacer lo que querrá con mi nombre.

---También me darás uno de tus vestidos, primito ¿no es verdad?---dijo
Amaranta con encantadora gracia y examinándome rápidamente de pies a
cabeza,---uno de esos magníficos trajes que has traído de París, hechos
conforme a las últimas modas, y que servirán de desconsuelo a todos los
petimetres de por acá.

---¡Oh, oh! yo soy \emph{très} contento de daros mi \emph{hábito}.

---Pues bien---dijo Amaranta con satisfacción.---Creo que podré salir
adelante con mi invento. Al anochecer escapará este joven de Madrid con
el menor riesgo posible.

Y tomando de mano de Arión la carta de seguridad, me la dio diciéndome:

---Esta tarde antes de marchar al Pardo con mi tía y mi primo, lo dejaré
arreglado todo. Puede este joven retirarse tranquilo; y si el discreto
Salmón tiene la bondad de pasar por aquí esta tarde, yo le daré las
necesarias instrucciones para que todo marche a pedir de boca.

---Señora---dijo el fraile,---volveré al anochecer o cuando usía quiera;
que tan a pechos he tomado este negocio como el mismo interesado.

---Vuelva su merced antes de las tres, pues hemos de salir para el Pardo
temprano, por sernos preciso visitar de paso en la Moncloa a mi madrina
que allí reside y está enferma, aunque no de gravedad.

Di yo las gracias a la condesa por sus muchas bondades; rogome ella que
si salía en bien, como esperaba, se lo comunicase, indicándole el sitio
de mi residencia para enviarme nuevos testimonios de su protección, y
con esto salimos el mercenario y yo muy satisfechos para tomar el camino
del convento.

Más tarde, cuando el fraile regresó de su segundo viaje a la misma casa,
conocí en conjunto el plan maravilloso de Amaranta, que era digno
ciertamente de su habilidoso y enredador talento.

---No he visto más graciosa invención---dijo mi amigo.---Te pones el
vestido que te mandarán, para que puedas pasar por persona principal, y
como tú y el señor duque tenéis la misma estatura y talle, quedarás que
ni pintado. Con esto y la carta de seguridad que ya tienes, esta noche
no eres Gabriel, ni Pico de la Mirandola, sino el señor duque de Arión
que sale por la puerta de Toledo para ir a su posesión de Parla.
Asimismo estará a tu disposición un coche\ldots{} ¡pero qué coche! La
señora condesa tiene sospechas de que alguno de su servidumbre está
sobornado por esos indignos corchetes y teme confiarles el secreto. Para
quitar de en medio esa dificultad ha solicitado de una amiga que le
facilite un \emph{bombé}\ldots{} ¡Conque en \emph{bombé} nada menos,
chiquillo! Te advierto que al cochero y lacayo se les dice que eres el
propio Arión; y como no conocen a este, es imposible que te vendan,
aunque alguno fuese bastante malo para hacerlo. Tendrán orden de
llevarte a donde tú les digas; pero se te aconseja que no pases más allá
de Navalcarnero si sales por la Puerta de Segovia, o de Leganés si vas
por la de Toledo, en cuyos puntos no creo que haya peligro. Conque señor
duque, beso a usía las manos. Es imposible que sospechen nada al ver tu
empaque y tu carta de seguridad\ldots{} Ya verás cómo lejos de ponerte
reparos esos gaznápiros, se quitarán los sombreros ante ti, y aun se
brindarán a acompañarte hasta tu palacio de Parla. ¡Qué las tenga
vuecencia muy felices!

La idea de Amaranta era de éxito casi seguro, y no tropezando con
Santorcaz, con Román o con otro cualquiera que personalmente me
conociese, era inevitable mi escapatoria, siendo, como era, el nombre de
mi carta de seguridad, el de una principalísima persona, reputada por
muy adicta a la causa francesa. Con esta confianza estuve todo el día, y
antes del anochecer llegó un criado con el traje, el cual me caía, que
ni pintado. Era elegantísimo, y de mucho lujo por la finura del paño, el
primor de los adornos y lo exquisito de todos sus accesorios; mas no era
traje de corte, sino de diario traer, si bien de esos que por sí solos
hacen resaltar sobre el vulgo a cualquiera que se los pone, aunque más
los lleve colgados que puestos. Consistía en casaca, chupa y calzón de
paño verde muy oscuro, con medias del mismo color; cuello blanco, de
infinidad de randas compuesto, y un rendigot pardo con vueltas y solapas
de pieles. Esta prenda tenía algún uso, pero aún conservaba muy buen
ver.

Cuando me encajé sobre mi cuerpo aquellas prendas, todos los frailes
vinieron a verme, y a porfía dijeron que nada podía pedirse en el arte y
buen parecer; que el sastre, autor de tales ropas, por fuerza había
adivinado las medidas de mi cuerpo, y que de tan linda manera vestido,
podía echarme a buscar aventuras por las altas casas de Madrid, seguro
de encontrar en alguna quien me mirase con agrado. A estas alabanzas
contestaba yo con risas y bromas, pero la verdad era (y en conciencia no
quiero ocultar esto aunque me desfavorezca) que yo estaba un poquillo
envanecido con mi traje, y todo se me volvía dar vueltas ante un espejo;
pues también en los conventos había espejos. El más satisfecho de todos
era Salmón, que no cesaba de hacer reverencias ante mí, llamándome señor
\emph{duque}; y por fin lleváronme como en jubileo a la celda del prior,
el cual se rió mucho, alabando con exageración mi buen empaque.

Vestido ya, vinieron a decir al fraile que un joven le buscaba con mucho
empeño. Salimos los dos y en el claustro bajo hallamos a D. Diego,
pálido, azorado, inquieto, el cual llegose impaciente al mercenario, y
le habló así:

---Padre, la Zaina se muere y quiere confesarse.

---¡Pobre Zainilla!---exclamó el mercenario.---¿Y qué es ello?

---Un mal que nadie conoce, ni se ha visto otro parecido, pues unos lo
tienen por locura, otros por consunción, estos por reumatismo, y
aquellos por melancolía. Lo cierto es que se muere sin remedio, y ahora
ha dado en llorar después de dos días en que no ha hecho más que
morderse, arrancarse los cabellos, e insultar a todos, a mí
principalmente, llamándome necio y mentecato.

---¡Era Vd. su cortejo!---dijo con desabrimiento Salmón.---¡Oh, entre
qué gente anda metido el señor conde de Rumblar!

---Padre, dejémonos de discusiones, y vaya pronto a confesar a la Zaina,
que se muere, pues ahora a ratos llora mucho y habla con razón diciendo
que quiere confesar sus pecados a Dios para irse al cielo, y a ratos le
entra un delirio en que dice mil disparates, y manda a todos que laven
las piedras de la calle que están manchadas de sangre, y luego pregunta
que cuándo acaba de pasar la estera que ya lleva tantos años y tantos
siglos de estar pasando por delante de sus ojos: en fin, mil desatinos
que no son para contados.

---Pues voy allá al momento; pero antes pediré licencia al prior, por
ser ya de noche.

---Gabriel---me dijo Rumblar, cuando nos quedamos solos en el
claustro,---¿qué traje es ese? ¿Te has vuelto caballero?

---Amigo D. Diego---le contesté,---de menos nos hizo Dios.

---¿Y qué es de ti? No se te ve por ninguna parte. ¿Qué traes a vueltas
con estos frailuchos?

---Más respeto, Sr.~D. Diego, para esta buena gente---le
dije,---siquiera porque estamos en su casa.

---No les puedo ver. Santorcaz que todo lo sabe, me ha contado mil
cuentos indecentísimos que prueban lo mala que es esta canalla. Es
preciso acabar con ellos. De veras te digo que desde que veo un fraile
me horripilo. Especialmente a este Salmón, a quien llamo el padre
Tragaldabas, no le puedo ver ni en estampa. Verdad es que él tampoco me
adora, y seguramente es quien intrigando en casa de la marquesa ha hecho
fracasar mi proyectado casamiento.

---¿Ya no se casa el señor conde? Eso no le será penoso porque me parece
haber oído decir a Vd. que no amaba mucho a la novia.

---Verdad es que la tal Inés no me hace mucha gracia; pero yo estoy
decidido a que sea mi esposa, porque así conviene a mis intereses.
¿Sabes? Santorcaz me ha dicho que todo hombre debe mirar por sus
intereses, porque sin esto no se puede tener representación alguna en el
mundo. Además él, que todo lo sabe y es más listo que el demonio, me
asegura que yo tengo talento, disposición y estoy llamado a muy grandes
cosas, por lo cual me dice: «Don Diego; a Vd. le es necesaria una buena
posición, que le permita desplegar sus dotes.»

---¿Pero Vd. no tiene por sí una desahogada posición?

---Bicoca: el patrimonio de Rumblar es de esos que hacen en las ciudades
chicas un mediano papel; pero aquí apenas puedo presentarme en quinta
fila. Nuestra casa ha vivido desde hace tiempo con la esperanza de que
se le incorpore ese mayorazgo de Leiva que es uno de los primeros de
España. Si cuando apareció Inés, como legítima heredera, mi señora mamá
se disgustó mucho, luego que se concertó el casarnos para evitar pleitos
y cuestiones, quedose muy satisfecha. Conque figúrate cuál será su rabia
y la mía, ahora que las señoras marquesa y condesa me han dicho
terminantemente que no hay nada de lo convenido. Mi madre a quien lo
escribí me contesta furiosa, llamándome tonto y necio y estúpido, y
amenazándome con venir a darme mil palmetazos si no llevo adelante el
negocio de la boda, como puede hacerlo un caballero resuelto y de
pesquis. A mí, francamente, no se me ocurre nada; pero para dicha mía
tengo ahí a ese bendito Santorcaz que me aconseja como un padre de la
Iglesia, y últimamente ha discurrido el más ingenioso arbitrio para que
las de Leiva no se burlen de mí.

---Yo creo que al señor conde no le será difícil llegar al casamiento, y
con el casamiento a la posesión del mayorazgo, con tal que esa joven
esté dispuesta a darle su mano.

---Eso no, porque no estoy loco por ella, que digamos, y de buena gana
renunciaría a todo, si exclusivamente de mí dependiera. Has de saber,
compañero, que yo, más que todos los mayorazgos del mundo, apetezco una
libertad sin límites para hacer lo que me dé la gana; ir a las logias,
dar gritos en las calles cuando hay alborotos, cortejar a las mozas del
Avapiés, echar un par de pesetas a un caballo de oros, y divertirme en
paz y en gracia de Dios: pero Santorcaz, que es mi mejor amigo y mentor,
como él dice, me tiene sujeto, y me hinca las espuelas en esto del
mayorazgo, afeándome mi descuido en cuestión tan importante. Como además
le debo enormes cantidades que no sé de qué modo pagarle, aquí tienes el
siempre y cuándo de esta mi resolución mayorazguil. Te advierto que lo
que me deslumbra y me vuelve lelo es la esperanza de poseer una renta de
esas que le permiten a uno gastar y gastar y gastar todo lo que se le
antoja. ¿Hay mayor gusto, muchacho, que ir un día por casa de todos los
amigos y convidarlos a una merienda en el Canal, poniendo comida para
más de cuatrocientas bocas, con tanta abundancia como en aquellas
célebres bodas de Camacho? ¿Hay mayor gusto que visitar los interiores
del teatro del Príncipe o de los Caños, y saber que no habrá entre
aquellos lienzos pintados actriz española, cantarina italiana, ni
bailarina francesa que no se le rinda a uno de toda voluntad? ¿Hay mayor
satisfacción que dar una corrida de toros, permitiendo la entrada gratis
a todo el pueblo, pagando con doble sueldo a los lidiadores y lidiando
uno mismo con un traje fino bordado de plata y oro? Pues esto y aún más
espero tener, si sale bien lo que hemos tramado.

Quedéme absorto y mudo, meditando en la inconmensurable degradación a
que en pocos meses había caído aquel joven tan estrecha y
meticulosamente educado bajo la inspección de su rigorosa madre;
instruido tan sólo en cosas aparentemente buenas, en el temor excesivo a
los superiores, en el desprecio de las novedades, en el aborrecimiento
de las cosas mundanas, en el respeto a la tradición, en el encogimiento
del espíritu; educado para ser gran señor, y representante de todas las
virtudes patriarcales. Ved a dónde había ido a parar su imaginación
atada durante la infancia con cien cadenas; ved por qué derrumbaderos
tenebrosos se despeñaba salvajemente su voluntad, criada en el respeto;
ved qué clase de pájaro atrevido salía de aquel huevo empollado al calor
de las mezquinas ideas del siglo pasado. Verdad es que cuando aquella
inocente gallina sacó al mundo su echadura, se encontró que de los rotos
cascarones salían en vez de pollos otras mil alimañas desconocidas, y la
infeliz cacareó con angustia, sin saber quién las había engendrado.

---Pero si ella no le quiere a Vd. tampoco---dije a D. Diego,---lo que
proyecta no será tan fácil.

---Eso me parecía a mí; pero Santorcaz, que sabe más que siete, me ha
llenado la cabeza de catálogos, principiando por decirme que yo era un
papanatas, y burlándose de mí con tanta zunga, que al fin me enfadé y
dije: «Pues yo seré más osado que Judas, y me atreveré a cuanto hay que
atreverse, pues ni las de Leiva, ni Vd. ni nadie se reirán de mí.»

---¿Y qué hace ahora el Sr.~de Santorcaz?

---Le han hecho los franceses jefe de la policía menuda, cargo que
desempeña a las mil maravillas. A todos los desafectos al nuevo Gobierno
me les echa mano lindamente. Verdad es que por ahí le critican mucho,
llamándole traidor; pero él se ríe de todo y dice que no hay mejor Rey
que José, y que los españoles son unos animales. Esto al principio me
enfadaba mucho; pero ya me he acostumbrado a oírselo decir, y yo mismo,
que era antes más español que Fernando VII, ya no doy dos higos por
España, y al son que me tocan bailo\ldots{} Pero verás lo que tenemos
proyectado. Para probarle a él y a todos sus amigos que no merezco esas
burlas, he decidido que si Inés no se quiere casar conmigo
voluntariamente, se casará por fuerza.

---Eso me parece difícil.

---Así lo parece: pero no lo es. Tú no tienes grandes ideas ni un
corazón osado, como yo lo voy a tener ahora, de modo que no podrás
comprender esto. Figúrate que consigo engañar a la muchacha, y sacarla a
hurtadillas de su casa, sin que lo adviertan tías ni primas, y
llevármela bonitamente a donde me diese la gana por unos días\ldots{}

---Pero eso no podrá ser, porque esa honesta joven no saldrá con Vd. de
su casa, y mucho menos, si como dice, no le quiere ni pizca.

---Tú eres sandio, por lo que veo---me contestó con petulancia
truhanesca.---Eso mismo me parecía a mí; pero Santorcaz y sus amigos me
llamaron el Papamoscas de Burgos. Te advierto que es preciso tener el
corazón echado para adelante, como dicen ellos, y atreverse a todo. Con
tal que Inés salga conmigo\ldots{} llévela yo a una casa que tenemos
preparada al efecto, y después su misma familia nos echará la bendición.
El siglo lo tiene dispuesto así.

Tuve que hacer un esfuerzo para refrenar la indignación que tanta bajeza
me producía.

---Poco me importa---añadió,---que Inés no me ame en este momento. Yo
estoy seguro de que se volverá loca por mí en cuanto nos tratemos con
cierta intimidad. Todos dicen que tengo yo cierto atractivo\ldots{}
así\ldots{} pues\ldots{} un gancho para pescar muchachas\ldots{} Desde
que se le pase la tristeza\ldots{} No sé si te he contado que allá en
los tiempos en que mi novia andaba abandonada por el mundo, tuvo por
novio a un perdido, un raterillo, un granuja\ldots{} ¡Qué cosas se ven
en el mundo! Lo más raro de todo es que le ha guardado a su galán
zarrapastroso una fidelidad de novela sentimental, que causa vergüenza a
todos los de la casa. Como que han tenido que hacerla creer que ese
joven ha muerto, para que no deshonrara a la familia pensando en él.

---Pero nada de eso hace al caso, y cada vez veo más difícil que Vd.
pueda sacar de su casa a tan honrada joven.

---Animal, claro es que no saldrá, si le digo a dónde la llevo; pero
como no lo he de decir, sino que tenemos preparado un cierto artificio.

---¿Cuál?

---Ya he sobornado a Serafina, su doncella, a quien he tenido que dar
una buena suma, y es seguro que mañana muy temprano saldrán las dos a
dar un paseo por los jardines de palacio, encontrándose en cierto sitio
solitario, donde es lo más fácil del mundo poner en ejecución mi
pensamiento. Santorcaz asegura que esto saldrá muy bien, y él es quien
lo dispone todo, quien prepara los coches, quien ha buscado la casa,
quien ha dado el dinero para sobornar a la criada. ¡Si vieras qué
interés tan grande se toma!

---Lo creo.

---Mañana temprano queda todo hecho. A esa hora la marquesa está
entregada a sus devociones, la condesa no se habrá levantado aún, y el
marqués estará en el primer sueño.

---Sr.~D. Diego---dije disimulando la ira cuanto me fue posible,---¿y
Vd. no ve en eso una serie de repugnantes bajezas, infamias y
desvergüenzas, indignas, no digo de un caballero, sino del más
desarrapado chalán? El que es capaz de hacer esto, está destinado a
acabar sus días en un presidio.

---Te hablaré francamente. Cuando Santorcaz y sus amigos me manifestaron
su plan, sentí aquí dentro cierta repugnancia y no la oculté. Pero se
rieron mucho de mí, y allí fue el llamarme zanguango, corazón de mirlo,
hombre de alfeñique y otras injurias que me indignaron mucho. Al mismo
tiempo, por otro lado Santorcaz me apremia para que le pague las grandes
sumas que le debo, y que ya exceden a cinco años de renta de mi
patrimonio. Además de esto, mi madre me manda de Bailén unas cartitas en
que me pone como chupa de dómine. Dice que si no llevo adelante por
cualquier medio este casamiento, soy un necio y un badulaque, y que
pierdo y arruino a mi familia con mi dejadez y pazguatería. Hasta D.
Paco me escribe diciéndome que seré para siempre indigno del
\emph{altísono} nombre de Rumblar, si no pesco ese mayorazgo, y ahí
tienes\ldots{} No hay más remedio que hacerlo. Fuera, pues, escrúpulos
de monja, y adelante. Ahora voy a probar que soy un hombre hasta allí,
capaz de todo y dispuesto a las más atrevidas cosas. ¿Qué te parece? ¿No
apruebas mi conducta? ¿No te entusiasmas oyéndome?

---¿De modo que mañana temprano\ldots?---pregunté con mas interés que D.
Diego en aquel asunto.

---Al rayar el día. No sé si te he dicho que ella madruga mucho.
Santorcaz dice que cuanto más pronto mejor. Ninguno de la familia se
enterará del caso, hasta que estemos en Madrid. Ya he escrito una carta
a la marquesa, fingiéndome muy enamorado y diciéndole que la fuerza
irresistible de mi pasión me impele a obrar así, y otras muchas cosas
muy bien puestas; como que la ha escrito Santorcaz\ldots{} Pero, chico,
es tarde y me retiro; quiero ver en qué para esta pobre Zaina y si se
muere o no se muere. La verdad es que me quería bastante; y sabe Dios si
habrá influido en su enfermedad\ldots{} Como ahora me tiene loco la
hermana de la Pepa Ramos\ldots{} ¿La conoces tú? ¡Qué guapa y qué mona
es! Adiós: me voy allá. ¿Quieres venir? ¿Qué haces aquí con esos
frailucos? Pero dime: ¿has heredado por ventura? No te conozco. Mira que
los frailes son muy intrigantes\ldots{} adiós, adiós, que aún tengo algo
que arreglar para mi viaje al Pardo a la madrugada.

Y diciendo esto, se marchó, dejándome solo en el claustro. En éste me
paseaba yo, presa de la más grande agitación, cuando me avisaron la
llegada del coche enviado por Amaranta para mi fuga. Al instante corrí a
la calle y entrando en él, pregunté al lacayo:

---La señora condesa, ¿dónde está?

---Esta tarde ha marchado al Pardo---me contestó respetuosamente,
sombrero en mano.---¿A dónde quiere usía que le llevemos?

---Al Pardo---contesté con resolución.

---Dijo la señora condesa que saldríamos por la puerta de Toledo, camino
de Illescas, ¿es que quiere usía dar un rodeo?

---Al Pardo, majadero, al Pardo derecho y sin rodeos---exclamé con
furia.---¿No he dicho que al Pardo? A toda prisa.

Las mulas partieron a escape, llevándome camino del real sitio.

\hypertarget{xxvii}{%
\chapter{XXVII}\label{xxvii}}

Fue detenido el coche en la puerta de San Vicente, abrieron la
portezuela, presenté mi carta de seguridad, y después de abrumarme con
cumplidos y cortesías, me dejaron pasar. Sufrí nueva detención hacia San
Antonio, y una tercera en la puerta de Hierro de cuyas repetidas
molestias deduje que era arriesgadísimo salir disfrazado y enteramente
imposible sin el documento prescrito. Pero yo pasé el camino felizmente,
y ninguno de los que echaron su mirada importuna dentro de mi coche,
sospechó el papel que un servidor de ustedes estaba representando.

Iba yo en un estado de agitación indefinible, y la marcha de las mulas
me parecía tan desproporcionada a mi febril impaciencia, que sentía
impulsos de bajar y correr a pie, creyendo de este modo llegar más
pronto. Arrastrado por una ciega e invencible determinación, yo la había
formulado en estos términos sencillísimos: «Llegaré, haré por ver a la
condesa, informarela de la alevosa intención de D. Diego, y partiré
después. No es preciso nada más.» Yo no pensaba en dificultades de
ninguna clase, y las contrariedades subalternas eran despreciadas
entonces por mi impetuosa voluntad. Tampoco atendía en manera alguna a
mi proyectada fuga, ni me cuidaba de si iba vestido de esta o de la otra
manera. Caer en poder de la policía, una vez llevado a efecto mi
pensamiento, me importaba poco.

Por fin, en poco más de una hora llegamos a la plaza de Palacio, donde
vi una gran escolta de caballería y muchos coches. El cochero del mío
azotó las mulas y las hizo penetrar por la ancha puerta hasta el
vestíbulo de donde arranca la gran escalera. Todo lo vi iluminado; todo
lleno de guardias españolas y francesas. Una música militar tocaba el
himno imperial en la galería que domina la escalera. Napoleón, que había
ido a comer con su hermano, estaba allí todavía.

Figuraos que uno se muere y despierta en otro planeta, en otro mundo,
encontrándose con forma distinta, en atmósfera diversa, en un medio
diferente, donde crecen Fauna y Flora que no se parecen a la Flora y
Fauna del mundo donde nació. Esta fue mi impresión: yo estaba aturdido y
atontado. Sin embargo, saliendo precipitadamente del coche, pregunté al
primer criado que se me apareció por los aposentos del señor marqués de
X. En el mismo instante, el lacayo me decía:---Venga vuecencia por aquí,
que es en este piso bajo a la izquierda.»

Dos o tres, no sé cuántos se apresuraron a franquearme la entrada, y mi
lacayo, entrando delante de mí, dijo a los criados que salían a su
encuentro:

---Ya está aquí el señor duque; avisad que ha llegado el señor duque de
Arión.

Yo no sé por dónde me llevaron; yo no sé por dónde entré; yo no sé en
qué sitio me encontraba; yo sólo sé que me vi en un recinto muy
alumbrado y caliente, y que el diplomático, estrechándome en sus brazos,
exclamaba:

---¡Picarón, gracias a Dios que te vemos!\ldots{} Pero ¿por qué has
venido tan tarde? Ya se ha acabado la comida\ldots{} ¡Ah, picarón, qué
alto estás!

Yo balbucí algunas excusas; pero comprendiendo al punto que era preciso
disipar aquel engaño, dije:

---¿No está la señora condesa?

---No ha venido. Estoy solo con mi hija. Pero, chico, no tienes acento
francés, y me dijeron que hablabas como un amolador. Ven, ven, al
instante te voy a presentar al rey José, que tanto desea verte. Ahí está
el Emperador. ¡Albricias!\ldots{} Ha convenido en que su hermano vuelva
a ser Rey de España, y ya están zanjadas todas las diferencias. Conque
ven\ldots{} ven\ldots{} Pero primo, ¿cómo es eso?---añadió examinando mi
traje.---¿Cómo no has venido de etiqueta? Pues oiga\ldots{} también te
has venido sin relojes\ldots{} Pues ¿y tus cruces, y tu Legión de Honor,
tu Cristo de Portugal, y tu Carlos III, y tu San Mauricio y San Lázaro,
y tu Águila Negra?

---Déjese Vd. de bromas---repliqué sin poder disimular mi
impaciencia.---Ahora vengo para un asunto urgente y del cual
depende\ldots{}

---¿La suerte de Europa?---dijo interrumpiéndome.---Corro, corro al
instante a ponerlo en conocimiento de Urquijo. ¿Vienes del cuartel
general? ¿Ha llegado allí algún correo de Francia con noticias del
Austria?

---No, no es eso---repuse sin atreverme a disipar el engaño.---¿Pero
dice Vd. que no está aquí mi señora la condesa?

---¿Tu prima? Esta tarde la esperábamos; pero debía pasar por la Moncloa
a ver a su madrina, y como ésta se halla in \emph{articulo mortis},
presumo que Amaranta y mi hermana habrán determinado quedarse allí toda
la noche. ¿Vienes tú de Madrid, o directamente de Chamartín?

---Siento mucho---manifesté con la mayor zozobra---que no esté aquí la
señora condesa.

---Te presentaré a mi hija, ven. Pues es lástima que no hayas venido de
etiqueta. Verdad es que tú tienes familiaridad con el Emperador, y si te
anuncias, puedes pasar a verle con ese traje\ldots{} Pero dime, ¿qué
noticias traes? ¿Ha llegado algún correo al cuartel general? A que me he
salido yo con la mía\ldots{} ¿apostamos a que el Austria?\ldots{} A mí
puedes contármelo. Ya sabes que el Emperador me consulta todo\ldots{}
Pero chico, ¿sabes que tienes una arrogante figura? A mí me habían dicho
que eras\ldots{} así\ldots{} un poco cargado de espaldas y\ldots{} la
nariz chata, y un ojo un poco\ldots{} pero no\ldots{} veo que me habían
engañado. Eres mejor de lo que yo suponía, y lo que es tu cara\ldots{}
casi juraría que no me es desconocida\ldots{} pues\ldots{} que te he
visto en alguna parte.

Estábamos en un lujoso salón, con magníficos muebles alhajado. Sentíase
ruido de voces en las habitaciones inmediatas; pero allí no había nadie
más que nosotros dos. El diplomático, asiendo las solapas de mi
casaquín, me sacudía, me sofocaba, me volvía loco con su charlar
inacabable. En vano era que yo pretendiese quitarle la palabra, hablando
de otras cosas y principalmente indicando algo del móvil de mi viaje.
Aquel insensato me quitaba la palabra de la boca, ávido y hambriento de
hablárselo él todo, y con sus gesticulaciones, su cotorreo sempiterno,
semejante al son de una matraca, me tenía aturdido, colérico, nervioso.

---¡Ay sobrinillo de mi alma!---continuó.---Si me confiaras las noticias
que traes\ldots{} Ya habrá llegado a tu conocimiento que yo soy la misma
reserva\ldots{} Porque no me queda duda de que tú traes algo, sí señor,
algo grave. Si hubieras venido a la comida, habríaslo hecho más temprano
y con otro traje. Y no es más sino que estabas en el cuartel general, y
el mayor general Berthier te envió a toda prisa con una comisión. A ver,
dímelo a mí solo, a mí solo\ldots{} ¿Vas ahora mismo a ver al Emperador?
Si quieres pasaré aviso al gentil-hombre para que te introduzca. Ya han
concluido de comer, y están conferenciando juntos el Emperador, el Rey,
el secretario Hugues Maret, Urquijo y monseñor de Pradt, ex-arzobispo de
Malinas. Anda, anúnciate, subamos\ldots{}

---Señor mío---dije bruscamente sin poder disimular ya mi impaciencia y
desasosiego.---Yo no vengo a hablar con el Emperador ni con el Rey, ni
con el arzobispo, ni tengo nada que ver con ninguno de esos señores. Yo
vengo a\ldots{}

Y callé, sin atreverme a decirle el objeto de mi visita.

---¿Conque no está aquí la señora condesa?---volví a preguntar después
de una pequeña pausa.

---Dale con la condesa. Que no, que no está. La esperábamos esta tarde;
pero según entiendo, se ha detenido en la Moncloa por acompañar a su
madrina, que se muere por momentos. Puede ser que llegue antes de media
noche.

---Pues la esperaré---dije resueltamente sentándome en un sillón.

---Veo que Amaranta te interesa más, y es para ti de mayor importancia
que la suerte del mundo. ¿Pero no querrás decírmelo?\ldots{} Aquí en
confianza\ldots{} a mí solo ---dijo sentándose junto a mí y poniéndome
la mano en el muslo.

---¿Qué, hombre de Dios, qué le he de decir, si no sé nada?

---Pesado estás sobrino. Para mí sería muy satisfactorio saberlo antes
que el mismo Emperador y poderlo decir a todos esos que están ahí
muertos de sed por una noticia.

---¿Dice Vd. que la Condesa vendrá antes de media noche? ¿Cuánto hay de
aquí a la Moncloa?

---¿Pero qué traes tú con la Amarantilla?\ldots{} Todo eso es para
disimular. Pero ven\ldots{} quiero que conozcas a mi hija. Ya tendrás
noticias de ella. ¡Pobrecita! La he recogido y reconocido\ldots{} Es
preciso reparar de algún modo los errores de nuestra juventud. En París
habrás oído hablar mucho de mí. Bastantes ruinas hay allí todavía de mi
ímpetu destructor en materias amorosas. Pero ven\ldots{} conocerás a
Inés\ldots{} es guapísima. No se ha recogido aún, y si está acostada,
haré que se levante.

---No---dije yo,---la veré mañana.

Mi situación, queridos señores míos, era bastante comprometida. La
condesa, a quien necesitaba ver y hablar, no estaba allí. Yo no quería
faltar al solemne compromiso contraído con ella, cuando le prometí no
presentarme jamás a su hija; y en verdad si Amaranta me hubiera
sorprendido allí en compañía de Inés, todas mis explicaciones le habrían
parecido artificios y malas artes y la aventura de mi disfraz un ardid
alevoso para arrebatarle aquel tesoro de su familia, que por la sociedad
y por otras mil consideraciones, me estaba tan implacablemente vedado.
En todo esto pensé, mientras D. Felipe de Pacheco y López de Barrientos
me volvía loco para que le contara las noticias del cuartel general.
Discurriendo rapidísimamente sobre aquella situación vine a deducir que
era preciso valerme del mismo diplomático para mi objeto, no hallándose
en palacio ninguna otra persona de la familia; mas para esto era también
preciso no perder el disfraz, ni correr el velo de aquel gracioso
engaño, pues si esto ocurría, todo acababa con echarme a la calle o
ponerme a disposición de un alguacil. Meditando en breves términos mi
plan, di principio a su ejecución de la siguiente manera:

---Después, mi querido tío, informaré a usted de todo lo que se dice en
el cuartel general. Por ahora quiero hablarle a Vd. de otro importante
asunto.

---¿Importante? Vamos a ver---dijo en voz baja y tan impaciente como un
niño.

---Importantísimo.

---Ya adivino. La Inglaterra, el enemigo común\ldots{}

---No es nada de eso. Lo que digo es que ese condesito del
Rumblar\ldots{} ¡oh! es un joven de malísimas costumbres.

---Ya lo sabemos; pero dejemos ahora a don Diego, ¡qué
majadería!---exclamó con desagrado.

---Es preciso que Vd. esté prevenido, por si\ldots{}

Entraron en aquel momento en la sala dos personajes vestidos de
uniforme, uno de los cuales era español y el otro francés; pero los dos
se expresaban en nuestra lengua. Levantámonos y el diplomático me
presentó gravemente a ellos, diciendo después:

---Por más que le pincho, nada, no suelta una palabra. Viene del cuartel
general, con noticias interesantísimas.

---¿Sube Vd. a ver al Emperador?---me preguntó uno de ellos.

---No señor---respondí, obligado a llevar adelante la farsa.---No
necesito ver por ahora a Su Majestad Imperial.

---En el cuartel general---me dijo el otro,---¿qué se dice de la actitud
del Emperador respecto a su hermano?

---¡Oh!---exclamé yo, dándome importancia,---se dicen muchas cosas.

---¡Muchas cosas!---repitió el marqués haciendo aspavientos.

---Aún no está decidido---añadió el que parecía francés,---que el
Emperador, nuestro señor, ceda el reino de España a su hermano. ¿Qué ha
oído Vd. en Chamartín? ¿Insiste el Emperador en la idea de considerar a
España como país conquistado?

---Sí señores, como país conquistado---dije con mucho aplomo, metiendo
mi cucharada en los arreglos y desarreglos del mundo.

---La verdad es---dijo otro,---que los dos hermanos no están muy
acordes. ¿Va tomando cuerpo la idea de agregar la España al territorio
de Francia?

---Sí señores---afirmé condoliéndome de la suerte de mi país.---España
se unirá a Francia.

---¡Oh! ¡qué calamidad!---clamó D. Felipe.---No podemos en modo alguno
seguir al servicio de la causa francesa. ¿Y se insiste en dividir a
nuestro país en cinco virreinatos?

---¡Pues qué duda tiene, señores!---repuse en tono de hombre
listo.---Pero aún se duda si serán cinco o seis.

---Sin embargo dijo el que parecía francés,---yo creo que esta noche se
reconciliarán.

---Por supuesto que si el Emperador se decide a tratar a España como
país conquistado, le mueven a ello las intrigas de Inglaterra.

---De Inglaterra, justo---repuse yo vivamente.---Me lo ha quitado Vd. de
la boca.

---Y la insensata resistencia del pueblo español.

---Exactamente\ldots{} la insensata resistencia\ldots{}

---A pesar de todo---dijo el español,---yo dudo mucho que el Emperador
pueda llevar adelante tan atrevido pensamiento, y menos ahora cuando
corren rumores de que el Austria\ldots{}

---¿Qué dicen los últimos despachos? Parece que el Austria se arma.

---Sí señores---respondí yo en tono profético, misterioso y
sibilítico.---El Austria se arma y\ldots{} no diré más.

---Pero hombre---apuntó el diplomático.---Si aquí somos todos amigos. Di
de una vez todo lo que sabes.

---Dispénsenme Vds. señores---indiqué cortesmente.---De buena gana lo
haría por complacer a personas tan amables; pero antes que mi deseo está
mi deber, antes que la satisfacción de un capricho amistoso, la
conciencia de mi discreción, cuyo inexpugnable baluarte en vano atacan
galantes sugestiones o arteras amabilidades. Callaré por ahora; pero
tengan ustedes entendido que el Austria\ldots{} el Austria\ldots{}

Los tres cortesanos se miraron, y yo examiné las pinturas del techo.

De improviso entraron dos, a quienes igualmente me presentó mi augusto
tío; pero aquí fui menos afortunado, porque uno de ellos, al saludarme,
me dijo con cierta malicia:

---Es muy particular. Hace tres años vi en París al señor duque de Arión
y no reconozco su fisonomía en la de Vd. O yo estoy trascordado, o Vd.
ha variado considerablemente.

Por mi suerte el diplomático se había apartado un poco, y además yo tuve
buen cuidado de no engolfarme en conversaciones con aquel caballero.
También quiso mi buena estrella que viniese a sacarme de apuros, otro
que llegó de repente y con gran prisa, diciendo:

---Señores, la conferencia va tomando carácter de altercado. Alzan mucho
la voz y desde el corredor de Poniente se oyen los gritos. Vamos allá y
oiremos algo.

Vierais allí cómo aquellos cortesanos coman por los pasillos, cómo se
escurrían por los laberintos de palacio, cómo se precipitaban unos
delante de otros disputándose cuál llegaba primero a pescar una noticia,
una voz perdida, un gesto visto al través de un resquicio, un accidente,
un destello de reales miradas, cualquier mezquindad que les fuera
favorable. Yo seguí tras ellos, y salí también; atravesamos un gran
salón, donde había hasta una veintena de personas de distintos
uniformes; internáronse en nuevos pasillos, pasaron de sala en sala,
llegando por último a un largo y oscurísimo corredor que tenía ventanas
a un angosto patio. Allí había otros cinco o seis, asomados a las
ventanas, y muy atentos a no sé qué, pues yo no veía nada digno de
llamar la atención. Todos se acercaban con pasos quedos, chicheaban muy
por lo bajo, y atendían y miraban; pero ¿qué miraban y a qué atendían?

El patio a que me refiero era muy estrecho. En la pared de enfrente
había una gran ventana cuyas hojas de cristal, cerradas y por dentro
cubiertas con una cortina de gasa, daban paso a la luz interior. Los
gruesos cortinones de invierno estaban recogidos a un lado y otro, de
modo que quedaba un triángulo de luz, con el ángulo más agudo en la
parte superior. En este triángulo se dibujaban varias sombras, pero con
toda precisión una sola, efecto de linterna mágica producido por la
presencia de un hombre entre la luz que iluminaba aquella pieza y el
hueco de la ventana. Movíase la sombra al tenor de los diversos grados
de animación de la palabra, y en esta sombra y en sus irregulares
movimientos fijaban la vista y el oído y la atención y el alma toda los
cortesanos allí reunidos.

---Ahora hablan más bajo---dijo muy quedamente uno de ellos,---pero hace
poco se han oído con claridad algunas palabras.

Y alargaban los cuerpos fuera del corredor, por ver si sus pabellones
auriculares cogían al vuelo alguna sílaba. Yo también atendí; pero la
verdad es que allí se oía tanto como en un desierto. Lo que sí excitó
mucho mi curiosidad, fue la sombra que ocupaba el centro del triángulo.
Era la de un hombre rechoncho y de cabeza redonda, con pelo corto.
Notábase el movimiento pausado de sus brazos al hablar, el de su cabeza
al atender; notábanse claramente las señales de asentimiento, las
negaciones vagas y las fuertes; notábanse la tenacidad, la duda, el
ademán de la pregunta, el de la respuesta, y tanta era la verdad con que
aquella silueta reproducía a la persona misma, que hasta se creía
advertir en ella la sonrisa, el fruncimiento de cejas, el asombro y
cuantos modos de lenguaje posee y usa el rostro humano. Unas veces la
cabeza puesta de frente, proyectaba en la vidriera una forma redonda,
otras volviéndose proyectaba su perfil; luego veíamos que a su altura
subía una mano y distinguíamos perfectamente el dedo índice afianzando y
dando energía a la palabra; después desaparecían las manos, y los
brazos, juntándose a la masa del cuerpo, indicaban que se habían
cruzado; luego transcurría mucho tiempo sin que la figura hiciese ademán
alguno, señal de que oía o de que meditaba, hasta que de nuevo volvía a
ponerse en acción.

---Miren Vds. ahora---dijo uno de los cortesanos,---cómo dice que no,
que no y que no con la cabeza.

En efecto, la sombra movió su cabeza haciendo la señal negativa por
espacio de algunos segundos.

---De seguro está diciendo que no cederá a nadie sus derechos a la
corona de España---indicó uno.

---Lo que indudablemente estará diciendo---habló otro,---es que pasará
por todo, menos porque los ingleses se metan aquí.

---¡Quia!---exclamó un tercero.---Lo que debe de estar diciendo es que
los españoles no podrán resistir mucho tiempo.

Entonces la sombra movió la cabeza en señal afirmativa repetidas veces y
con mucha insistencia, acentuando con la mano aquel movimiento.

---Pues ahora dice que sí, que sí y que sí---indicó uno.

---Sin duda habla de que son indudables sus derechos de conquista.

---Y de que puede disponer del trono de España como se le antoje.

---¡Patarata! Apuesto a que no es nada de eso, sino que asegura vencerá
a los ingleses.

Poco después la sombra se llevó la mano a la nariz.

---Toma tabaco---dijeron los cortesanos.

---Ya van trece veces desde que estamos aquí.

Luego la sombra acercó un bulto a su cara, inclinándola después, y se
oyó desde nuestro observatorio un lejano ronquido.

---¡Se suena!---exclamaron los cortesanos.

---¡Buena señal!---dijo uno.

---¡No, sino muy mala!---añadió otro.

Después la sombra se levantó, y al instante confundiose entre otras
sombras. Un momento después, separadas las demás, volvió a destacarse;
pero ya estaba transfigurada, porque la cabeza redonda había
desaparecido en otra mayor sombra trapezoidal. Una vez puesto el
sombrero, se hubiera distinguido de cuantas sombras suele engendrar la
noche, y de cuantas pueden volver de los Elíseos Campos o de los
cristianos cementerios a pasearse por el mundo.

---Ya sale\ldots---dijeron los cortesanos.

---Corramos al salón.

Y aquello no fue correr, sino volar a la desbandada.

---¿No vienes al salón?---me preguntó el diplomático.

---¿No ve Vd. que no vengo de etiqueta?

---Es verdad; pero tú\ldots{} Te advierto que el Emperador se marcha.
¿Acaso vienes a hablar con el rey José?

---Yo no quiero ver al Emperador esta noche---le respondí.---Aunque él
me trata con bastante intimidad, y solemos jugar un poco al tute\ldots{}

---¡Al tute!\ldots{} hombre\ldots{} eso sí que no lo sabía.

---Sí\ldots{} pues decía que aunque tenemos mucha confianza, y nos
tratamos como dos amigotes, no puedo presentarme así en el salón, cuando
los demás van de etiqueta. Vd. no irá tampoco\ldots{}

---¡Oh, sí! Yo voy al salón\ldots{} porque te advierto que el Emperador
al entrar me miró, y después preguntó quién era yo. De modo que
ahora\ldots{}

---¿Pero no le ha hablado Vd. nunca?

---Te diré, lo que es hablarle\ldots{} así\ldots{} pues\ldots{} así como
estoy hablando ahora contigo, no\ldots{} pero hemos cambiado notas, y no
creas\ldots{} en ocasiones con la pluma en la mano nos hemos puesto como
ropa de pascuas.

---¿Vd. se retirará a su aposento? Hablaremos un poco y luego me
marcharé.

---¡A estas horas! No\ldots{} aquí te has de quedar. No dudes que vendrá
la condesa mañana temprano. Hablaremos todo lo que quieras; pero después
que yo vaya al salón, y haga por ver si S. M. I. me mira otra vez, y me
entera de todo lo que se dice\ldots{} ¿Qué sabes tú si el rey José
querrá llamarme como anoche, para que le dé un poco de conversación?

---Antes hablemos los dos de un asunto que nos interesa\ldots{} es cosa
de pocas palabras.

---Entremos en mi cuarto---dijo cuando llegamos al salón donde me
recibió la vez primera.

---No, aquí mismo---repuse.---Ahora caigo en que tengo que marcharme, en
cuanto hablemos dos palabras.

---¡Qué singular! Hombre, aquí me hielo de frío. Entremos en mi cuarto.

En efecto, pasamos a otra pieza, nos sentamos, pero aún no se habían
arrellanado nuestros cuerpos en el sofá, cuando entró un criado
diciendo:

---Aquí está un gentil-hombre que viene a decir a usía que el señor
conde de Cabarrús quiere verle al momento.

---Al instante, corro al instante. ¡Oh, ministro amabilísimo!---exclamó
el diplomático con súbita e inmensa alegría.---Primo, ahí te quedas.
Vendrá Inés a hacerte compañía.

---No\ldots{} que no se moleste---repuse yo con inquietud.---Esperaré
solo.

---Que venga la señorita Inés---dijo el diplomático al criado.

El criado me miraba atentamente.

---Que venga mi hija---repitió el marqués.---Dile que está aquí el señor
duque de Arión, su pariente; que venga al instante a hacerle compañía,
porque el Emperador\ldots{} digo, el rey José\ldots{} digo, el ministro
Cabarrús, me ha mandado llamar para consultarme un grave asunto.

Y sin esperar más, porque su impaciencia era febril, salió dejándome
solo. Yo estaba tan agitado que no me era posible apreciar la extensión
del tiempo que iba pasando mientras permanecía en la soledad de aquel
cuarto, sin percibir otro ruido que el tic-tac de un reloj de chimenea,
y el chisporroteo de los leños que en ella se quemaban. Yo no cabía en
mi mismo de inquietud, de ansiedad y desasosiego, y juntamente se me
representaban en espantosa lucha, la inefable felicidad de ver a Inés y
el pesar de mi conciencia turbada por quebrantar una leal promesa. A
veces me parecía que los minutos corrían con inconcebible rapidez, y a
veces que se estaban quietos delante de mí, mirándome como geniecillos
desvergonzados. Mi espíritu a ratos impaciente y lleno de amorosas
ansias, me impulsaba a penetrar en las habitaciones interiores, buscando
a la que no parecía; y a ratos me venían deseos de abrir la ventana,
echarme por ella al jardín inmediato, y huir para siempre de aquella
casa. Sentado estaba mal, y mal estaba en pie y mal también paseándome
de un ángulo a otro en la reducida estancia: el pulso y las sienes me
latían con furia, y aquel violento y acompasado golpear determinó bien
pronto en mí una viva calentura que me inflamaba todo. Inés tardaba
mucho. «Si no viene, me muero,» dije para mí, olvidándome al fin de
todas las consideraciones que al principio me habían hecho temer su
llegada. Pasaron no sé si horas o minutos; sólo sé que muchas ideas mías
se iban quedando atrás y que venían otras a sustituirlas, para marcharse
luego. De este modo apreciaba el transcurso del tiempo. El reloj avanzó
mucho sin que Inés pareciese. Aquella soledad empezó a hacérseme
insoportable, y la idea de que ella no vendría, se representó en mi
pensamiento produciéndome un dolor inmenso. Después de mis primeras
dudas, habíase entregado mi espíritu al gozo de suponer que vendría, y
su tardanza me ponía en estado febril.

Arrastrado por una fuerza irresistible, sir reparar en mi situación ni
en circunstancia alguna, casi ignorando lo que hacía, abrí la pequeña
puerta que comunicaba aquella pieza con la inmediata. Al pasar a esta,
halleme en una sala sin luz; pero como entraba alguna claridad por la
puerta recién abierta, pude ver por dónde andaba. Con pasos muy quedos
atravesé aquella sala, y al ver reflejada oscuramente mi imagen en los
espejos, sentía miedo de mí mismo. En el testero del fondo vi otra
puerta que cedió al punto a mi mano, y encontreme en una tercera sala
más pequeña. Profunda oscuridad reinaba en ella, pero al poco tiempo de
estar allí, distinguí en el fondo negro una perpendicular raya de luz.
Al mismo tiempo creí que sonaban voces de mujer por aquel lado, y esto,
con la débil claridad, impeliome más hacia allí. Andaba muy lentamente,
extendiendo las manos para no tropezar con los muebles; andaba como un
ladrón, conteniendo el aliento, apagando el ruido de los pasos, creyendo
que hasta las oscilaciones del aire a mi tránsito iban a delatar mi
presencia a los de la casa. Yo había perdido todo dominio sobre mí
mismo, y en nada reparaba más que en llegar pronto a aquella raya
luminosa, tras la cual sentía más claramente ya la voz de Inés. Al fin
llegué. Por la estrecha rendija no se veía nada; pero se oía. Dos
mujeres hablaban.

Al poco rato una de las voces dijo algo como despidiéndose; sentí el
ruido de una puerta, y todo quedó en completo silencio. Aguardé un poco.
Puse luego la mano en el picaporte, y con mucha, muchísima lentitud lo
fui levantando, levantando, de modo que no hiciera ruido. Cuando me
pareció bastante, empujé y la puerta cedió; empujé más, y la fui
abriendo poco a poco, cuidando de que no rechinara. Durante esta
operación, toda mi sangre se paró dentro de mí. A medida que la puerta
se abría, iba viendo todo lo que había dentro de aquella estancia.
Primero vi un lecho con cortinas blancas, luego una mesa con labores de
mujer, y por último, vi una figura puesta de rodillas delante de un
reclinatorio, con la cabeza inclinada y oculta enter las manos en
actitud de profundo recogimiento. Vuelta hacia mí aquella figura, que
apoyaba la frente en el reclinatorio, no era fácil reconocerla, pues de
su cabeza no se veía sino el cabello; pero yo la reconocí, y era ella
misma; era Inés.

Avanzando resueltamente, pero siempre con pasos muy quedos, entré y me
dirigí hacia ella.

\hypertarget{xxviii}{%
\chapter{XXVIII}\label{xxviii}}

Cuando Inés alzó la cabeza y me vio delante, tras un estremecimiento que
indicaba el mayor espanto, quedose atónita, sin habla, con disposición a
perder el sentido. La emoción me impedía al mismo tiempo el pronunciar
algunas palabras para tranquilizarla. Mi presencia le causaba terror;
iba a gritar sin duda.

---Inés, Inesilla---dije al fin,---no te asustes, soy yo, soy yo mismo.
¿Creías tú que me había muerto? No, mírame bien, estoy vivo. No me
tengas miedo.

Diciendo esto la abrazaba, estrechándola contra mi pecho.

---¿Creías tú no volver a verme más?---proseguí.---Te dijeron que me
había muerto. Infames, ¡cómo te engañan! Aquí estoy; no me preguntes
cómo he venido. No lo sé. Creo que Dios me ha traído por la mano para
que nos veamos.

Inés tardaba mucho en volver de aquel estupor que por algunos minutos
pareció quitarle el conocimiento; mirábame con ojos asombrados, derramó
algunas lágrimas, y su rostro, fluctuando entre el llanto y la sonrisa,
revelaba en cada segundo una sensación distinta. Pasado un rato, fijando
la atención en mi vestido, pareció profundamente asombrada, volvió a
reír y me interrogó con los ojos. Sus manos, sus brazos temblaban entre
los míos de un modo alarmante; y temiendo que la impresión producida en
su organismo por tan fuerte sorpresa fuera demasiado lejos, la tomé en
brazos, púsela con el mayor cariño sobre el sofá cercano y senteme junto
a ella, procurando calmarla y explicándole en términos precisos mi
inesperada aparición.

---¿Pero dónde estabas tú?---me dijo.

---En la habitación de tu padre. Allá me dejó cuando te llamaron, y allí
te estaba esperando. ¿Por qué no fuiste? Mi impaciencia era tanta que no
pude resistir, y como un ratero me metí por esas habitaciones hasta
llegar aquí.

---¿Y cómo entraste en palacio?

---Eso es largo de contar. Me han pasado muchas cosas, Inesilla de mi
corazón. Yo no sé cómo he venido aquí. Había prometido no verte más ni
hablarte; pero yo no sé por qué me encuentro a tu lado y te veo y te
hablo. ¿Conque me creías muerto?

---Sí, ¡muerto!---dijo con tristeza.---Sin embargo, yo confiaba en que
fuera mentira y muchas veces he tenido el pensamiento de que ibas a
venir. Anoche, ayer, ahora mismo he estado pensando en esto, y al
quedarme sola he sentido mucha zozobra creyendo verte en los espejos, o
salir de detrás de esos armarios, o entrar por cualquiera de esas
puertas como un fantasma. ¿Pero cómo has venido aquí? ¿De qué invención
te has valido? Si te descubren\ldots{} Estás vestido como un caballero.

---Sí, Inesilla---respondí besándole las manos.---Pero aunque me veas
vestido de caballero, no creas que lo soy. Soy lo mismo que era antes,
cuando estábamos en casa de D. Mauro, es decir, no soy nada. Tú estás
tan por encima de mí que debes avergonzarte de mirarme.

Al oír esto, todo cambió en su espíritu, y la vi sonreír de un modo
espontáneo y festivo, perdida ya la emoción dolorosa del primer momento.

---Yo no pensaba verte más---continué;---pero la casualidad o la
Providencia han querido que te vea. ¡Qué desgraciados somos o mejor
dicho, qué desgraciado soy! Porque yo tengo que renunciar a ti, tengo
que marcharme para no volver más. ¿No comprendes tú que ha de ser así,
que no puede ser de otra manera? Para mí valiera más no haber nacido.
¿Por qué te conocí? ¿Por qué te volviste gran señora? ¿Por qué Dios que
a ti te sacó de la humildad para traerte a los palacios me dejó a mí en
la miseria y en la oscuridad de mi nombre?

---No me has dicho todavía por qué estás vestido así---indicó con el
mayor asombro.

---Nada de esto es mío, Inesilla---repliqué con profundo dolor.---Estas
ropas son como las que se ponen los cómicos cuando salen a la escena
vestidos de reyes. Después se las quitan y quedan hechos unos mendigos:
lo mismo soy yo. Si ahora se descubre la farsa que me ha traído aquí,
tus criados me echarán del palacio ignominiosamente. No soy nadie, no
soy nada. Yo creí que no te vería más; pero algún poder superior nos ha
puesto esta noche juntos, y yo que he jurado ante la condesa tu prima no
verte ni hablarte más en la vida, estoy ahora a tu lado para decirte que
te quiero y te adoro y me muero por ti. Seré un malvado, un tramposo, un
miserable que se burla de todas las conveniencias de la sociedad; pero
siendo todo esto, y aún más, insisto en decir que no puedo dejar de
quererte aunque me lo prohíban todas las potencias de la tierra, y
aunque entre los dos se pongan con la espada en la mano todos tus
parientes y antecesores desde que el mundo es mundo.

Inés parecía meditar. Después de un rato de silencio, me dijo con
tristeza:

---Mis parientes son muy crueles conmigo.

---No, hijita mía; considera tú su posición, su nombre, lo que deben a
la sociedad, y comprenderás que no pueden hacer otra cosa. ¿Cómo han de
admitirme en su familia? La idea de que me amas les causa horror, y se
creen deshonrados con sólo mirarme. Tu prima la condesa es muy buena. Si
tuviera tiempo para contarte los beneficios que le debo y el afecto que
me muestra, te asombrarías.

---Ha llegado el caso de que yo devuelva mi familia todo lo que me ha
dado, y tome por mí misma lo que no ha querido darme---dijo Inés.

---Tú tendrás prudencia y esperarás.

---Hablaré francamente a mi prima. Ella me ha dicho que quiere verme
feliz a toda costa, y es la que me defiende de las impertinencias de mis
cinco maestros, y la que me salva de la etiqueta, que es lo que más
aborrezco. Yo le diré que has estado aquí\ldots{}

---No, no, por Dios; no le digas que he estado aquí---exclamé.---Yo debo
marcharme ahora mismo, Inés; yo no puedo estar más aquí.

---No te has de ir---me dijo asiendo mis dos brazos para detenerme.---Yo
se lo diré todo a mi prima, le diré que no te has muerto; que yo sé que
no te has muerto; que nos hemos visto, y que has de volver.

---No, no le digas eso: desde este momento ya no merezco la benevolencia
que ha manifestado.

---¡Oh!---exclamó Inés con mucha pena.---Pues entonces, ¿qué recurso nos
queda? ¿Qué podemos hacer? ¿Cuándo vuelves tú?

---Nunca---le respondí sin reparar en lo que decía, pues mi exaltación
no me permitía formular ideas concretas sobre nada.

---¿Cómo nunca?

---Sí, volveré cuando quieras---dije estrechándola contra mi
corazón.---Si tú me mandas que vuelva, si tú despreciando las
resoluciones de tu familia, insistes en quererme lo mismo que cuando
éramos dos pobres criaturas desamparadas, volveré, quebrantaré las
promesas que hice a tu prima, porque ¡ay! sin duda tu prima no sabe
cuánto te quiero, cuánto te adoro, y de qué manera nosotros nos hemos
dado un juramento que está por encima de todos los demás. Dile que no me
he muerto, ni me moriré, mientras tú vivas, porque no quiero ni debo
morirme; dile que aquí estaré, mientras tú no me eches, y que antes que
fueras condesa, y duquesa, y princesa, habías resuelto casarte conmigo
que no soy caballero ni soy nada, aunque teniendo tu cariño no me cambio
por todos los nobles de la tierra.

Inés al oírme se animaba mucho. Encendiéronse sus mejillas y el vivo
resplandor de sus ojos indicó una irrupción de sensaciones agradables y
de ideas de felicidad, que de improviso se apoderaban de su abatido
espíritu. Tomándome la mano me dijo:

---Juro que no me he de casar sino contigo, cualquiera que sea tu
suerte, cualquiera que sea tu posición. Dicen que yo soy rica, y que soy
noble. ¿No es esto bastante? Yo les diré que si no me quieren de este
modo, me quiten todo lo que me han dado. Les diré que tú eres para mí
más caballero que todos los demás; y por último, que ninguna fuerza
humana me obligará a dejarte de querer, porque Dios lo ha ordenado así.
Tengamos confianza en Dios y esperemos. Lo que parece más difícil, se
hace de pronto fácil. Yo sé, sin que nadie me lo haya enseñado, que
cuando las cosas deben pasar, pasan, y que la voluntad de los pequeños
suele a veces triunfar de la de los grandes.

Al decir estas palabras que indicaban junto con un firme amor, un
profundo sentido, Inés me mostraba la superioridad de su alma, bastante
fuerte para poner las leyes inmortales del corazón sobre todas las
conveniencias, preocupaciones y artificiosas leyes de la sociedad.

---¡Inés!---le dije prodigándole las más tiernas muestras de cariño.---A
pesar de estar tan alta, tú eres hoy tan desgraciada como yo; pero para
los dos vendrán días felices y tranquilos.

Yo había olvidado todo temor, las causas de mi presencia en aquel sitio,
lo avanzado de la hora, no me acordaba de su familia, ni de mi fuga, ni
de la policía, ni de nada; no veía más mundo que aquel pequeño, ¡qué
digo pequeño!\ldots{} aquel mundo infinito que mediaba entre nuestros
ojos.

---Tú sabes y sientes mejor que yo---exclamé;---tú me señalas el camino
que debo seguir, y lo seguiré. Te amo tanto que querría morirme aquí
mismo, si supiera que habías de ser para otro. Y vengan contrariedades,
vengan orgullos, vengan rigores de familia, vengan obstáculos, venga
todo, que todo lo desprecio. ¿Qué valen cien mil coronas condales, y las
mayores riquezas del mundo? Todo eso no será suficiente razón para
quitarme lo que es mío; mi Inesilla de mi alma y de mi corazón. Si soy
pobre y miserable, que lo sea: nada importa puesto que miserable y
pobre, quieres tú más uno de mis cabellos que las coronas y tesoros de
todos los duques de la tierra. ¿No es cierto? Y que venga ahora toda la
sociedad y toda Europa, y toda la historia y el mundo todo a decirme que
no podrás ser mía. Que vengan y yo les diré que se vayan a paseo, porque
nosotros no necesitamos de ellos para nada, y nosotros valemos más que
todo eso. ¿No es verdad? Cuando prometí a tu prima renunciar a ti,
prometí lo absurdo y lo imposible, lo que no estaba en mi mano hacer,
porque el amor que nos tenemos es obra de Dios, es como la vida, y sólo
puede quitarlo el mismo que lo da.

Así me expresé yo, y en este tono hablamos un poco más: luego cambiamos
de asunto, y seguimos departiendo en serio y en broma sobre mil cosas
que nos ocurrían, sin acordarnos de nada que no fuera nosotros mismos, y
menos del tiempo que iba transcurriendo a toda prisa. De tema en tema
vino a mi pensamiento el objeto que allí me había llevado y le conté el
incidente de D. Diego con sus torpes y abominables planes. Ella se
sorprendió de esto y me dijo que nunca había supuesto a Rumblar tan
rematadamente malo. Seguimos luego hablando de otros asuntos, y ella se
reía de mi traje, y yo de lo que ella me contaba al referir las
ceremonias palaciegas a que había asistido. Repetidas veces pasó por mi
mente la idea del gran peligro que allí corría; pero era tan feliz que
yo propio arrojaba lejos de mí aquella idea importuna. Al fin entró de
pronto una criada y dijo:

---¿Se le ofrece a la señorita alguna cosa?

Díjole Inés que no, y se fue; pero me observó de soslayo el tiempo que
allí estuvo.

Seguimos hablando y al poco rato apareció otra criada que me miró mucho
también, preguntando:

---¿Ha llamado la señorita?

Y luego que esta se retiró pareciome sentir cuchicheos y ruido de pasos
tras de la puerta. Comuniqué a Inés mi recelo, y al punto convinimos en
que me debía retirar. ¡Qué escándalo! Era mucho más de media noche. Ella
misma me llevó al cuarto donde antes me había dejado el diplomático, y
después de discutir un rato sobre lo más conveniente para salir en bien
de aquel paso, acordamos que esperaría al Sr.~D. Felipe, continuando
cuando volviera, el mismo papel de duque de Arión, y que con cualquier
pretexto saliese después poniéndome en salvo antes de la mañana y hora
en que necesariamente habían de llegar Amaranta o su tía. Despidiose
Inés de mí, dándome muchas esperanzas y prometiéndome que nos veríamos
cuando menos lo pensase, y me quedé solo otra vez donde antes estaba.

Cansado de esperar, quise salir; pero encontré la puerta cerrada por
fuera, y en el mismo instante en que lo advertía, sentí que una mano
desconocida, cerraba también la que me había dado paso hacia la
habitación de Inés. Estaba preso.

Presté atención a ciertos ruidos cercanos y percibí otra vez cuchicheo
de voces diversas, como risas y chacota de criados y gente menuda, cuya
circunstancia acabó de revelarme el peligro en que me encontraba, y la
proximidad de un lance desastroso. A esto había venido a parar el duque
de Arión.

Oí a poco también la voz del diplomático, que algo turbada decía:---Id a
avisar al cuerpo de guardias. ¿Estáis seguros de que no lleva armas?

Luego los rumores se extinguieron para resonar de nuevo hacia el cuarto
de Inés, con voces de hombre y de mujer, confundidas en viva disputa. Y
la voz de Inés se oyó muy cerca aunque me fue imposible entender lo que
decía. Lleno de congoja, mas también colérico ante la idea de que se me
tomase por un ladrón, di golpes en la puerta con pies y manos, pidiendo
que se me abriera, lo cual aumentó las risas del exterior.

---Es muy posible que lleve pistolas---dijo el diplomático.---No abráis,
mientras no venga un pelotón de la guardia.

Pero el criado a quien tan prudentes advertencias se dirigían, no hizo
caso de ellas; abriome la puerta, y abalanzándose hacia mí con otros dos
de su misma estofa, dijo:

---No te escaparás, no. A ver, registradle bien los bolsillos y sacadle
todo lo que lleve.

---Canallas---exclamé, luchando con ellos.---Yo no me llevo nada.
Ladrones y rateros seréis vosotros, que no yo.

---Creo que debéis amarrarle, muchachos---dijo el diplomático, entrando
con gran arrojo.---Desde luego sospeché que este joven no era mi
pariente. Por fuerza ha de tener los bolsillos llenos de alhajas:
registradle bien. ¿Decís que estuvo en el cuarto de mi hija más de tres
horas? Eso no puede ser, caballerito ---añadió encarándose
conmigo.---¿Quién es Vd.? Vive Dios que esto es algo misterio.

---Este es el que en el Escorial sirvió de paje a la señora
condesa---dijo uno de los criados empujándome con tal fuerza que me hizo
caer al suelo.

---Este estaba en Córdoba hace seis meses, y todos los días venía a la
puerta de casa---dijo otro dándome con el pie, una vez que caído me vio.

---Y es, si no me engaño, el que tiraba chinitas a la ventana---afirmó
una criada, hundiendo sus uñas en mi carne.

---Me parece que le he visto en casa vestido de fraile---dijo otra
dándome en la cabeza con las tenazas de la chimenea.

---Ya le conozco, y sé muy bien lo que le trae por aquí---indicó una
tercera tirándome fuertemente del cabello.

---¿Con que nada menos que duque de Arión?---dijo un lacayo dándome una
manotada en la chupa con tanta fuerza que me la rasgó de arriba abajo.

---¡Miren el duque de papelón! ¡Pues no vino poco finchado!---exclamó
otro anudándome la corbata tan violentamente que pensé morir
estrangulado.

---Desnudadle en el acto.

---No: aguardad a que venga la autoridad---ordenó el marqués.---¿Conque
es un paje de Amaranta que fue a Córdoba, y que arrojaba chinitas
vestido de fraile? Bien decía yo que esta cara no me era desconocida. En
el Escorial, en Córdoba\ldots{} ¿te llamas tú Gabriel? ¡Gabriel,
Gabriel!\ldots{} Conque Gabriel.

Y diciendo esto, D. Felipe Pacheco y López de Barrientos dio algunas
vueltas por la estancia, revolviendo sin duda en su mente
contradictorios pensamientos. Juzgue el lector de mi martirio al verme
entre aquellos soeces criados, cuyas almas experimentaban deliciosa
fruición en degradar al que creyeron duque, y en pisotear mi supuesta
nobleza y caballerosidad. Defendime al principio rabiosamente de sus
groseros insultos; mas nada podían contra tantos mis fuerzas por
momentos enflaquecidas, y me entregué a las vengativas manos de aquella
pequeña plebe irritada que no podía tolerar el encumbramiento ficticio
de uno de los suyos. Yo creo que me habrían roto los huesos, que me
habrían arrastrado en tropel por la casa, que me habrían arrancado
pedazo a pedazo los vestidos y con los vestidos la carne; que me habrían
deshecho a pellizcos, pinchazos y rasguños, si la llegada de la condesa
no hubiera puesto fin de repente a la dolorosa escena de mi
crucificación. La vi aparecer cuando ya iluminaban completamente la
habitación las primeras luces del día, y pareciome un ángel salvador. La
sorpresa que tal espectáculo le causó junto con lo que a su llegada le
contaron, habíanla puesto como fuera de sí. La ira y la compasión se
sucedían rápidamente una tras otra en su semblante. Parecía no dar
crédito a sus ojos, me miraba casi exánime y maltratado, y reconocía en
mis ropas las del duque de Arión, que ella me diera para fugarme. Por de
pronto, a pesar de su enojo, me libró de toda aquella canalla, y
haciendo que los criados saliesen afuera, quedose sola conmigo, mientras
su tío iba en busca de quien me llevase a la cárcel.

\hypertarget{xxix}{%
\chapter{XXIX}\label{xxix}}

---Señora---exclamé comprendiendo con rápida penetración sus
pensamientos en aquel instante,---no me condene vuecencia sin oírme; no
me juzgue ingrato, desleal y mentiroso si tan impensadamente me
encuentra aquí.

---¡De qué indigna manera me has engañado!---repuso con voz turbada por
la ira.---Jamás lo creí: yo pensé que tenías en tu baja e innoble alma
una chispa del fuego de honor. No: tu abyecta condición se revela en tus
actos, y no es posible esperar del miserable pilluelo de las calles sino
doblez y maldad. Hipócrita, ¿dónde has aprendido a fingir? ¿Cómo tu
despreciable carácter, formado de todas las perfidias y malos intentos,
ha podido disimularse con la apariencia de la sencillez honrada y de
sentimientos nobles?

---Señora---respondí,---usía me tratará de otro modo cuando sepa qué
motivos me han traído aquí.

---No quiero saber nada. ¿Has visto a mi hija? ¿La has hablado?

---Sí señora.

---¡Oh! No es posible que viéndote haya dejado de comprender qué clase
de persona eres. ¿Dónde está Inés? Que venga aquí, y si al ver este
pillastre desarrapado que se disfraza de gran señor para llegar hasta
ella, si al ver una palpable muestra de tu bajeza y vil condición en
esta lastimosa figura de duque magullado y roto se arrastra por el suelo
pidiendo misericordia, persiste en creerte digno de un recuerdo, Inés no
es lo que yo quiero que sea, no es mi hija, no es de mi sangre.

Y en efecto, yo me arrastraba por el suelo, magullado y roto; y
confundido por el anatema de la condesa, imploraba con inconexas
palabras que me perdonase, indicando a medias frases los hechos que
atenuaban mi falta.

---Señora---exclamé prosternándome hasta tocar con mis labios los pies
de Amaranta,---verdad es que he faltado a mi palabra. Arrójeme usía de
aquí, entrégueme a los alguaciles, permita que me lleven a la cárcel, al
presidio; mándeme matar si gusta, pero no me pida, no, de ningún modo me
pida que deje de amar a Inés, porque es pedirme lo imposible y lo que no
está en mi mano prometer. Usía me hablará de su casa y de todas las
casas. Yo confieso mi pequeñez, yo reconozco que al lado de la grandeza
de vuecencia soy como un grano de arena comparado con el tamaño de todo
el mundo; yo no soy nadie, yo soy un insensato, un malvado, un miserable
y todo lo que usía quiera que sea; pero yo no puedo dejar de amar a
Inés. Cuando sus padres la abandonaban yo la amé; cuando estaba sola en
el mundo yo fuí su amigo; cuando era pobre yo trabajaba para ella. Creí
que su repentino cambio de fortuna la apartaría de mí para siempre;
prometí en falso, prometí lo que no podía ni debía cumplir, lo que
estaba fuera de mi albedrío; prometí renunciar a lo que siempre ha sido
mío, y mi ceguera y mi error han durado hasta esta noche en que la he
visto y la he hablado, señora condesa; hasta esta noche en que he
comprendido que Inés no puede, no puede de modo alguno resistir el peso
abrumador de su nobleza.

Amaranta golpeó mi humillado rostro con sus pies. Sentí las suelas de
sus zapatos hiriendo mi cabeza, y los encajes de sus faldas barrieron mi
frente. La condesa estaba frenética y cruel en su desbordada ira.

---¿Qué has dicho?---exclamó.---¿Que no renuncias?\ldots{} ¿Sabes que un
miserable como tú puede desaparecer del mundo sin que el mundo lo
advierta? ¡Despreciable gusano! ¡No te aplasto por compasión y te
levantas para insultarme!

---Yo no insulto a usía---dije.---Yo respeto y venero a la que tantos
deseos de favorecerme ha manifestado. Vuecencia puede hacerme
desaparecer del mundo si gusta; sin duda lo merezco. Yo prometí a usía
no verla más y no he cumplido mi palabra; soy un truhán y un miserable.
Vine a este palacio sin intención de verla; encontreme solo y una fuerza
irresistible, una fiebre que me devoraba lleváronme a su cuarto, donde
la vi y nos hablamos largo rato. ¡Oh! ¿Me pide usía que deje de amarla?
No puede ser. ¿Me pide usía que no la vea más? Pues haga Su Grandeza de
modo que me den la muerte, porque mientras tenga un solo aliento de vida
y mientras me quede fuerza para arrastrarme, correré tras ella, la
buscaré, penetraré en lo más escondido y subiré a lo más alto, sin ceder
en esta persecución hasta que Inés no me diga que se ha concluido la
guerra a muerte trabada entre ella y su noble familia.

---¡Oh! Quiero concluir de una vez---afirmó sin poder contener su
agitación;---que venga aquí mi hija; la traeré aquí, te verá delante de
mí, y si todavía\ldots{} No, no puede ser. ¡Dios mío! ¿Qué aberración,
qué absurdo es este que presenciamos? Miserable mendigo---añadió
volviéndose a mí,---vete. La culpa tiene quien te ha dado más
importancia de la que mereces. Inés te desprecia: si has creído otra
cosa te equivocas. ¿Por qué no hiciste lo que te mandé? ¿Por qué viniste
aquí? Mereces la muerte, sí, la muerte. No soy cruel; pero ¿acaso la
vida de un indigno ser, que se perdería en el mundo sin que nadie lo
echara de menos, debe estorbar la felicidad de toda una familia, debe
estorbar mi reposo y echar por tierra la grandeza de una casa como la
mía? No, no puede ser\ldots{} Vete de aquí; que te lleven, que te
arrastren como infame ladrón que eres. Si ella lo siente que lo sienta,
si padece que padezca. Así no se puede vivir. Seré inflexible; yo
enseñaré a mi hija cuáles son sus deberes; yo le enseñaré el respeto que
debe tener a su nombre y me obedecerá, cueste lo que cueste.

---Deje usía---le dije,---que la maten los demás; y cuando haya
sucumbido a las violencias, a las vejaciones y a la tiranía de sus
parientes, quédele a la madre el consuelo de no haber puesto las manos
en ella.

---¿Qué dices? ¿Qué has dicho?---preguntó Amaranta mirándome fijamente y
cambiando por completo en un instante de tono, de actitud, de
expresión.---¿Qué has dicho?

---He dicho que usía no debe, que no puede contribuir a matarla.

---¡A matarla!---exclamó con estupor y como vacilando entre admitir o
rechazar aquella idea.

---Sí señora. Bien sabe usía que Inés es muy desgraciada.

Vi entonces cómo se disipaba la ira en el rostro de Amaranta, cómo se
aclaraba su semblante, cómo todo aparato de indignación y de biliosidad
y de tirantez nerviosa desaparecía, sucediendo a aquella tempestad
aplacada una quietud reflexiva en que al instante se sumergió su
espíritu, lanzado desde las cimas de la cólera a los abismos de la
meditación. Me miró largo rato y yo la miré. Estaba profundamente
pensativa. Estaba en poder de uno de esos invasores pensamientos que
vienen de repente y ocupan toda el alma y suspenden todas las
sensaciones, y envuelven y embargan las facultades todas. Al fin, sin
pestañear, sin apartar los ojos de mí, sin hacer movimiento alguno
exhaló un profundo suspiro y después dijo:

---Sí, mi hija es muy desgraciada.

No era sin duda la primera vez que a sí misma se decía aquellas
palabras.

Sentada en el sofá, apoyó la barba en los dedos pulgar e índice, y el
codo en el brazo del asiento, y así estuvo largo espacio de tiempo. Me
parece que la estoy mirando. ¡Cuán hermosa y cuán imponente y
subyugadora! \emph{¡Digna concha de tal perla!} como ha dicho, no por
cierto refiriéndose a esta, sino a otra, un gran poeta contemporáneo.

Alzó luego la vista, y me examinó atentamente; ¡pero de qué modo, con
cuánto interés me miraba! De sus ojos había desaparecido el rayo de la
indignación que antes la hacía tan terrible. Yo no me atrevía a decir
nada. Una dulce sensibilidad embargaba mi espíritu.

Amaranta, esclava de su pensamiento, volvió a repetir:

---¡Oh! sí: mi hija es muy desgraciada, y yo no puedo hacerla feliz.

Dicho esto, me miró con cierta perplejidad. En sus ojos se retrataba una
viva compasión hacia mi persona, quizás algún sentimiento más favorable.
Al principio creí engañarme, pero mi corazón con su misterioso lenguaje
me indicó que habían cambiado de súbito los sentimientos de la condesa
respecto a mí. De mi pecho pugnaban por desbordarse los míos.

Acerqueme a ella y me dijo:

---¿Qué has hablado con Inés? ¿Qué te ha dicho?

No le pude contestar de otro modo que arrojándome de rodillas a sus
pies. Pero ella repitió la pregunta intentando con sus manos alzar mi
frente que se había adherido con fuerza a sus rodillas.

---Señora---le contesté al fin,---me ha dicho la verdad; me ha dicho que
a nadie puede amar más que a mí.

Yo besaba sus manos y la sentí llorar.

Duró poco tiempo aquella situación. Sentimos gran ruido de voces,
abriose la puerta y en el dintel apareció la marquesa, terrorífica,
abrumadora de cólera y de severidad. Con ella venían el diplomático, D.
Diego, el verdadero duque de Arión, algunos criados y soldados de la
guardia. Amaranta no dijo nada ni yo tampoco. La actitud en que nos
encontraron debió sorprenderles más que la noticia de que había un
ladrón en la casa, y estoy seguro de que cada individuo de la familia
interpretaba de un modo distinto aquella escena. En cuanto a esto mis
lectores verán más adelante algo que les interesará.

Como era opinión general que yo era un ladronzuelo, vino gente de la
policía, y cuando Santorcaz penetró en la habitación y ordenó a los
suyos que se apoderaran de mí, huyeron con el rápido paso del terror las
dos nobles damas. La algazara de aquel momento no me impidió percibir
lejanos gritos y alteradas voces de mujer en las cuadras interiores. Un
oficial de la guardia francesa, llamado a última hora no sé por quién,
echó de palacio de un modo algo despreciativo a alguaciles y
alguacilado, tratándonos a todos como a gente de perversa ralea.

\hypertarget{xxx}{%
\chapter{XXX}\label{xxx}}

No tengáis compasión de mí al verme en esta cuerda ignominiosa,
enracimado con otros veinte infelices. No somos ladrones, ni asesinos,
ni falsificadores; somos patriotas, insurgentes de aquella gran epopeya,
y nos llevan a Francia. Felizmente no se cumplió en nosotros aquel
consejo del capitán del siglo que decía a su hermano: «\emph{ahorcad
unos cuantos pillos y esto hará mucho efecto}.» Por lo que pasó después,
se ha venido a conocer que también Álvarez el de Gerona entraba en el
número de los pillos. No nos ahorcaron, pues aún vivo para contarlo, y
cuando digo que no me tengáis compasión es porque después de preso, la
policía no me supuso otra criminalidad que la traición a la causa
francesa, y me juzgó bastante castigado con el destierro.

---Bien sé yo que no eres ladrón---me dijo Santorcaz en Madrid, cuando
me ponían en la cuerda que estrechaba en cordial apretón las cuarenta
manos de los insurgentes;---pero eres un vil soplón y entrometido, a
quien es preciso poner a cien leguas de Madrid. Si te dieras a partido y
quisieras ser mi amigo, yo te conseguiría un puesto en la policía, con
tal que me sirvieses bien en este negocio.

No con palabras, porque no las merecía, sino con una mirada de desprecio
le contesté, y estuve después meditando sobre mi suerte, hasta que la
cuerda se movió y los cuarenta pies de aquella serpiente humana se
pusieron en marcha. Eramos los pillos, que el Gobierno francés,
demasiado generoso, no había querido ahorcar, y se nos mandaba a
Francia. Con nosotros iba el gran poeta Cienfuegos. Isidoro Máiquez y
Sánchez Barbero fueron poco después, aunque no ensartados.

Al dar los primeros pasos miré al que iba a mi derecha, atado su codo al
mío. ¡Oh ventura sin igual! Era D. Roque el lector de periódicos.

---¡Ah, Sr.~D. Roque!---le dije,---¿también habla de esto el
\emph{Semanario Patriótico?}

---¡Queridísimo Gabriel! Dios nos ha puesto juntos en la desgracia como
en la prosperidad. Paciencia y que la Virgen nos deje ver algún día a
nuestra inolvidable villa.

---¿Por qué le destierran a Vd.?

---Hijo, por una calaverada. Cometí la indiscreción de decir en un
paraje público que nuestro desgraciado vecino D. Santiago Fernández era
un héroe no menos grande que los de la antigüedad y podía compararse a
Codro, Leónidas, Horacio Cocles, Mucio Scévola y al mismo Catón por la
entereza de su ánimo. ¿No lo crees tú así?

---¿Murió nuestro amigo?

---Sí, cuando el general Belliard fue a tomar posesión de los Pozos,
todos entregaron las armas. D. Santiago continuaba encerrado en el
jardín de Bringas. ¿Qué pensarás que hizo? Pues por la mañana al volver
de su casa amontonó toda la leña puesta allí para calentarnos. Ya
recordarás que también había una gran cantidad de madera vieja de la
casa que han derribado en la esquina. Pues con aquellos materiales y la
leña hizo un gran parapeto en el rincón del fondo, donde estaba el
gallinero vacío, y púsose dentro de su improvisada fortaleza. Derribaron
los franceses la puerta del jardín, y cuando vieron aquel monte de
madera, de cuyo interior salía una hueca voz diciendo: «\emph{Se rendirá
Madrid, se rendirán los Pozos, pero el Gran Capitán no se rinde},»
tuvieron al que tal decía por loco y diéronse a reír. Pero Fernández
había puesto dentro una buena cantidad de cartuchos y dale que le das,
empieza a hacer fuego por las aberturas y resquicios de su montón de
leña. Los franceses que se vieron heridos (y alguno de ellos murió)
arremetieron contra el gallinero destruyendo los parapetos de madera
vieja. Fernández no cesaba de hacerles fuego desde adentro. Pero cátate
que a lo mejor empieza a salir humo, y luego llamas que crecieron
rápidamente, y la ronca voz del defensor del gallinero
gritaba:\emph{¡Viva España; mueran los franceses y el granuja de
Napoleón!}

Mandó el oficial que se apartase la madera para sacar a aquel
desgraciado, que sin duda excitaba su admiración; pero Fernández gritó
de nuevo:---«Se rendirá Madrid, se rendirán los Pozos; pero el Gran
Capitán no se rinde,» hasta que cesó la voz; y las llamas, extendiéndose
vorazmente, destruyéronlo todo. La inmensa hoguera estuvo humeando todo
el día. Cuando aquello se acabó buscaron el cuerpo, pero estaba hecho
ceniza.

Calló D. Roque, y en el mismo instante el que nos conducía por la Mala
de Francia mandó que hiciéramos alto. Al detenernos vimos que por el
camino y hacia Chamartín venían algunos coches y gran número de jinetes
con deslumbradores uniformes. Era el Emperador que volvía de su visita
al palacio de Madrid y caminaba hacia su cuartel. Iba en coche, y al
pasar, nuestro guía y los soldados que nos custodiaban mandáronnos que
le diéramos vivas. Fue preciso repartir algunos culatazos para que
obedeciéramos, y cuando el grande hombre pasó, algunos le saludaron. Sin
duda por estas y otras ovaciones de la misma clase escribía con fecha 17
de Diciembre: \emph{En las poblaciones por donde paso me manifiestan
mucha simpatía y admiración}.

---Acabe Vd. de contarme la muerte de nuestro amigo---dije a D. Roque
una vez que pasó la procesión.

---Ya no queda nada---repuso,---sino que con toda su grandeza y poder el
hombre que acaba de pasar no llega ni con mucho a la inmensa altura del
Gran Capitán. Algunos han dicho que nuestro amigo estaba loco; pero ese
que ahí va, ¿está en su sano juicio?

\flushright{Enero de 1873.}

~

\bigskip
\bigskip
\begin{center}
\textsc{Fin de Napoleón en Chamartín}
\end{center}

\end{document}
